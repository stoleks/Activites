%%%%
\teteTermStssAlim

%%%% titre
\numeroActivite{3}
\titreActivite{Brunissement d'une pomme}


%%%% objectifs
\begin{objectifs}
  \item Étudier la dégradation d'une pomme et les facteurs physico-chimique responsables.
\end{objectifs}

\begin{contexte}
  Quand on coupe une pomme en tranches, si on laisse les tranches à l'air libre elle peuvent brunir.
  
  \problematique{
    Comment expliquer et prévenir le brunissement d'une pomme ?
  }
\end{contexte}


%%%%
\begin{doc}{Mécanisme de brunissement}{doc:A3_brunissement_pomme}
  \begin{wrapfigure}{r}{0.3\linewidth}
    \centering
    \vspace*{-18pt}
    \hspace{30pt}
    \chemfig{*6(-= -(-OH) =(-OH) -=)}
    
    Benzène-1,2-diol : \\ exemple de polyphénol
  \end{wrapfigure}
  Les pommes changent de couleurs à cause du brunissement \important{enzymatique.}
  Ce processus a besoin de trois éléments :
  \begin{listePoints}
    \item du dioxygène.
    \item Une enzyme spéciale appelée polyphénol oxydase.
    \item Des polyphénols. 
    Ce sont des molécules organiques cyclique contenant au moins un groupe hydroxyle \chemfig{OH} lié à un cycle insaturé \chemfig{C_6H_{6 - n}}, (n = nombre de groupe hydroxyle).
  \end{listePoints}
\end{doc}

\begin{doc}{Emplacement des enzymes et des polyphénols}{doc:A3_cellules_enzymes}
  Le \important{polyphénol oxydase} et les \important{polyphénols} se trouvent à l’intérieur des cellules d’une pomme.
  Le polyphénol oxydase se trouve dans de petits compartiments appelés chloroplastes, qui sont entourés d’une membrane.
  La plupart des polyphénols sont contenus dans d’autres tissus cellulaires.

  Quand les cellules d'une pomme sont endommagées, l'enzyme et les polyphénols entrent en contact.
\end{doc}

\question{
  Quelle est l'enzyme responsable du brunissement d'une pomme ?
}{

}{1}

\question{
  Expliquer pourquoi couper une pomme entraine l'apparition d'un brunissement.
}{}{4}

\question{
  Donner la formule semi-développée du benzène-1,2-diol. Entourer et nommer ses groupes caractéristiques.
}{}{5}

\begin{doc}{Réaction chimique responsables du brunissement}{doc:A3_reaction_brunissement}
  Lorsque les \important{polyphénols} se mélangent au \important{polyphénol oxydase \chemfig{PPO}} et au \important{dioxygène}, ils créent un composé appelé \important{ortho-quinone}.

  \begin{center}
    \chemfig{*6(-=(-R_1) -(-OH) =(-OH) -=)} + \chemfig{O_2} + \chemfig{PPO}
    \reaction
    \chemfig{*6(-=(-R_1) -(=O) -(=O) -=)}
  \end{center}

  Ensuite, les molécules d'ortho-quinone se connectent ensemble pour former de longues molécules.
  Ce processus est appelé \important{polymérisation}.
  Il crée un composé appelé \important{mélanine}, qui donne à la pomme une apparence brune.
\end{doc}

\begin{doc}{Conditions pour la formation de la mélanine}{doc:A3_conditions_melanine}
  \begin{wrapfigure}{r}{0.3\linewidth}
    \centering
    \vspace*{-38pt}
    \image{0.75}{images/organique/eumelanine}

    Eumélanine
  \end{wrapfigure}
  La polymérisation des polyphénols en mélanine ne se produit que sous certaines conditions.
  \begin{listePoints}
    \item La réaction de formation de l'ortho-quinone a un rendement optimal à $\sim\qty{20}{\degreeCelsius}$.
    Si les températures sont très élevées ou très basses, le polyphénol oxydase devient complètement inactif et ne réagit plus.
    \item Le pH de la pomme a un rôle important. La réaction de formation de l'ortho-quinone a un meilleur rendement pour un pH neutre.
  \end{listePoints}
\end{doc}

%%
\question{
  Quelle est la molécule responsable du brunissement de la pomme ?
  Citer un autre organisme vivant ou cette molécule joue un rôle important.
}{}{2}

\question{
  En utilisant les documents~\ref{doc:A3_reaction_brunissement} et~\ref{doc:A3_conditions_melanine}, proposer trois méthodes de conservation qui empêcheraient une pomme coupée de brunir.
}{
}{6}
