%%%%
\teteTermStssAlim

%%%% titre
\vspace*{-30pt}
\numeroActivite{5}
\titreActivite{Procédés de conservations des aliments}


%%%% objectifs
\begin{objectifs}
  \item Connaître quelques procédés de conservations industriel
  \item Distinguer un procédé de conservation chimique et un procédé physique
\end{objectifs}

\begin{contexte}
  De nombreux organismes microscopiques sont présent naturellement dans les aliments et notre environnement.
  Des micro-organismes toxiques peuvent coloniser les aliments et les rendre impropre à la consommation.

  \problematique{
    Quels procédés de conservation peut-on utiliser pour limiter la prolifération de micro-organisme ?
  }
\end{contexte}


%%%% docs
\begin{doc}{Conservation des aliments}{doc:A4_conservation_aliments}
  Les procédés de conservation des aliments cherchent à préserver leurs propriétés gustatives, nutritives et leur comestibilité.

  La conservation d'un aliment implique d'empêcher la croissance de micro-organisme (microbes ou bactéries) et de retarder l'oxydation des corps gras pour que les corps gras ne deviennent pas rances.
  Les méthodes de conservation peuvent modifier l'environnement physique des molécules qui composent l'aliment, ou transformer chimiquement les molécules qui composent l'aliment.
\end{doc}

\numeroQuestion Nommer tous les changements d'états sur le schéma ci-dessous.

\begin{center}
  \image{0.8}{images/thermodynamique/transformation_etats_large}
\end{center}


%%
\begin{doc}{Procédé physique et chimique de la conservation}{doc:A4_procedes_phys_chim}
  Pour améliorer le temps de conservation d'un aliment on peut utiliser

  \begin{importants}
  \separationBlocs{
    \begin{center}
      \important{Un procédé chimique}
    \end{center}
    \vspace*{-12pt}

    Un procédé chimique agit par modification des molécules de l'aliment ou par ajout d'une espèce chimique (conservateur, antioxydant).
  }{
    \vAligne{-55pt}
    \begin{center}
      \important{Un procédé physique}
    \end{center}
    \vspace*{-8pt}

    Un procédé physique modifie l'environnement dans lequel se trouve les molécules de l'aliment (température, état physique).
  }
  \end{importants}
\end{doc}


\begin{doc}{Quelques méthodes de conservation}{doc:A4_méthodes de conservation}  
  \begin{tblr}{
      colspec = {Q[wd=0.22\linewidth, m, c] | X[2,m] | Q[wd=0.2\linewidth, m]}, 
      row{1} = {couleurPrim!20, c}
    }
    \textbf{Procédé} & \textbf{Principe} & \textbf{Physique ou chimique ?} \\ \hline
    \important{La salaison} &
    L'aliment est salé. Le sel diminue la quantité d'eau disponible pour le développement des bactéries.
    & \\ \hline
    \important{Le sucrage} &
    L'aliment est sucré. Le sucre diminue la quantité d'eau disponible pour le développement des bactéries.
    & \\ \hline
    \important{La lyophilisation} &
    L'aliment est congelé, puis l'eau est complètement évaporée sous vide par sublimation.
    & \\ \hline
    \important{La déshydratation} &
    L'eau présente dans l'aliment est évaporée dans une endroit chaud et sec.
    & \\ \hline
    \important{La fermentation} &
    Des bactéries non-toxiques consomment le dioxygène et empêchent l'apparition de bactéries nocives.
    & \\ \hline
    \important{Saumurage} &
    L'aliment est trempé dans un bain d'eau salée, ce qui prévient l'apparition de bactéries.
    & \\  \hline
    \important{La stérilisation} &
    La température élevée tue toutes les micro-organismes.
    & \\ \hline
    \important{La réfrigération} &
    Le froid ralenti les réaction chimiques et la prolifération des bactéries.
    & \\ \hline
    \important{La congélation} &
    L'eau solide ne permet pas aux bactéries de se développer et le froid ralenti les réaction chimiques.
    & \\ \hline
    \important{La surgélation} &
    La surgélation est une congélation dû à un abaissement de température à \qty{-20}{\degreeCelsius} à cœur de l'aliment, en moins de vingt minutes. La surgélation permet de conserver toutes les qualités du produit et d'arrêter le développement des bactéries.
    & \\ \hline
    \important{L'atmosphère contrôlée} &
    L'air ambiant est remplacé par un gaz inerte, ce qui prive les bactéries de dioxygènes.
    & \\ \hline
    \important{Additifs conservateur} &
    Des molécules minérales ou organiques aux propriétés anti-bactérienne sont ajoutées dans l'aliment pour en améliorer la conservation.
    Ces additifs sont désigné par un code type E2XX, (E200, E210, etc.)
    & \\ \hline
    \important{Le fumage} &
    La fumée contient des molécules bactéricides.
    & \\ \hline
    \important{L'huile} &
    L'huile empêche l'oxygène d'arriver à l'aliment en remplaçant l'eau dans l'aliment.
    & \\
  \end{tblr}
\end{doc}

\numeroQuestion Pour chaque méthode de conservation, indiquer si c'est un procédé physique ou chimique.


% \begin{doc}{Quelques aliment}{doc:A4_aliments}
%   \begin{multicols}{4}
%     \image{1}{images/photos/conservation/glace} \\
%     \texteTrou{Congélation}
    
%     \image{1}{images/photos/conservation/cafe_lyophilise} \\
%     \texteTrou{Lyophilisation}
    
%     \image{1}{images/photos/conservation/confiture} \\
%     \texteTrou{Sucrage}
%     %
    
%     \image{1}{images/photos/conservation/conserve} \\
%     \texteTrou{Stérilisation}
    
%     \image{1}{images/photos/conservation/lait_sterilise} \\
%     \texteTrou{Stérilisation}
    
%     \image{1}{images/photos/conservation/mangue_sechee} \\
%     \texteTrou{Séchage}
%     %
    
%     \image{1}{images/photos/conservation/olive_huile} \\
%     \texteTrou{Huile}
    
%     \image{1}{images/photos/conservation/pate_sechee} \\
%     \texteTrou{Séchage}
    
%     \image{1}{images/photos/conservation/sauce_soja} \\
%     \texteTrou{Saumurage}
%     %
    
%     \image{1}{images/photos/conservation/pizza_surgele} \\
%     \texteTrou{Surgélation}
    
%     \image{1}{images/photos/conservation/tofu_fume} \\
%     \texteTrou{Fumage}
    
%     \image{1}{images/photos/conservation/fromage} \\
%     \texteTrou{Fermentation}
%   \end{multicols}
% \end{doc}


% %%%%
% \numeroQuestion Associer à chaque aliment une méthode de conservation.