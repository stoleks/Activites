%%%%
\teteTermStssAlim

%%%% titre
\numeroActivite{5}
\titreActivite{Contrôle qualité d'un dessert à base de lait}


%%%% objectifs
\begin{objectifs}
  \item Comprendre le principe de la \important{D}ose \important{J}ournalière \important{A}dmissible (\important{DJA})
  et de la \important{D}ose \important{J}ournalière \important{T}olérable (\important{DJT}).
\end{objectifs}

\begin{contexte}
  Une inspectrice sanitaire contrôle un restaurant et réalise un prélèvement sur un dessert pour enfant.

  \problematique{
    Est-ce que ce dessert respecte les doses toxicologiques de référence ?
  }
\end{contexte}


%%%% docs
\begin{doc}{Dose Journalière Admissible (DJA) et Dose Journalière Tolérable (DJT)}{doc:A5_DJA_DJT}
  \begin{encart}
    Les \important{Doses Journalières} sont les quantités d'une substance, qu'une personne peut manger tous les jours de sa vie sans risques pour sa santé.
  \end{encart}  

  Les \important{Doses Journalières} sont estimées par des études scientifiques et exprimées en \unit{\mg} de substance par \unit{\kg} de masse corporelle, soit en \unit{\mg\per\kg}.

  \begin{encart}
    On distingue deux types de \important{Doses Journalières}
    \begin{listePoints}  
      \item \important{DJA :} pour les substances autorisées et ajoutées volontairement ;
      \item \important{DJT :} pour les contaminants présent involontairement (métaux lourds, radionucléides, composés organiques, etc.)
    \end{listePoints}
  \end{encart}
\end{doc}

\begin{doc}{Lait de vache cru}{doc:A5_lait_vache_cru}
  Le lait de vache cru est le lait issu de la traite des vaches.
  Le lait est constitué à \qty{87,5}{\percent} d'eau, de glucides, de protéines et de matières grasses.

  Le lait peut-être contaminé par des polluants comme la mélamine, ou infecté par des micro-organismes.
  Ces micro-organismes peuvent venir de l'environnement (terre, paille, mouche, déjection, camion-citerne, etc.) ou être présents sur la vache (infection des mamelles).

  Pour que les micro-organismes prolifèrent, il faut de l'humidité (de l'eau), de l'énergie (sous forme de chaleur), de la nourriture (contenue dans le lait) et en général du dioxygène.
\end{doc}

\begin{doc}{Techniques de conservation des aliments}{doc:A5_technique_conservation}
  Pour tuer les micro-organismes, on peut augmenter la température d'un aliment pendant une certaine durée, avant de refroidir l'aliment rapidement.

  \begin{center}
    \begin{tblr}{
      colspec = {|c |c |c |c |}, hlines,
      column{1} = { couleurPrim!10 },
      row{1,2} = { couleurPrim!20 },
      cell{1}{1} = {r=2}{c},
      cell{1}{2} = {r=2}{c},
      cell{1}{3} = {c=2}{c},
    }
      Technique & Pasteurisation & Stérilisation & \\
      %
      & & Appertisation & Upérisation (UHT) \\
      %
      Température &
      Entre \qty{65}{\degreeCelsius} et \qty{100}{\degreeCelsius} &
      Environ \qty{120}{\degreeCelsius} &
      Environ \qty{140}{\degreeCelsius} \\
      %
      Durée &
      Quelques dizaines de secondes &
      Quelques secondes &
      Quelques secondes \\
    \end{tblr}
  \end{center}
\end{doc}
  
\begin{doc}{Doses toxicologiques de références}{doc:A5_doses_references}
  \centering
  \begin{tblr}{
    colspec = {|c |c |c |}, hlines,
    row{1} = { couleurPrim!20 },
    row{2} = { couleurPrim!10 },
    cell{1}{1} = {c=2}{c},
  }
    \textbf{DJA (\unit{\mg\per\kg})} & & \textbf{DJT (\unit{\mg\per\kg})} \\
    %
    E102 & E122 & Mélamine \\
    %
    \num{7,5} & \num{4} & \num{0,5} \\
  \end{tblr}
\end{doc}

\separationBlocs{
  \begin{doc}{Gâteau à analyser}{doc:A5_gateau}
    \image{1}{images/photos/conservation/gateau_DJA_DJT}
  \end{doc}
}{
  \begin{doc}{Analyse du dessert}{doc:A5_analyse_dessert}
    \centering
    \phantom{b}\vspace*{-12pt}
    
    \begin{tableau}{|c |c |c |}
      E102 (jaune) & E122 (rouge) & Mélamine \\
      %
      \qty{150}{\mg} & \qty{50}{\mg} & \qty{4}{\pico\g}
    \end{tableau}
    
    \phantom{b}
  \end{doc}
}

\smallskip
\question{
  Indiquer les méthodes de conservations utilisé pour conserver le lait dans le gâteau à analyser.
  Identifier si les méthodes de conservations sont des procédés physiques ou chimiques.
}{}{5}

\question{
  Calculer les masses maximales de colorant jaune E102, de colorant rouge E122 et de mélamine qu'un-e enfant de \qty{20}{\kg} peut ingérer chaque jour.
}{}{6}

\question{
  Le dessert peut-il être servi sans danger dans le restaurant ?
  Justifier en répondant à la problématique posée dans le contexte.
}{}{6}