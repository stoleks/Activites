%%%%
\teteTermStssBiom

%%%% titre
\numeroActivite{2}
\vspace*{-36pt}
\titreActivite{Structure des protéines}

%%%% objectifs
\begin{objectifs}
  \item Comprendre la structure tridimensionnelles des protéines.
  \item Comprendre les actions biologiques des protéines.
\end{objectifs}

\begin{contexte}
  Les protéines sont des molécules complexes dont la structure tridimensionnelle unique leur donne des propriétés biologiques particulières.
  
  \problematique{
    Comment la structure tridimensionnelle des protéines influence leur propriétés biologiques ?
  }
\end{contexte}


%%%% docs
\begin{doc}{Structures des protéines}{doc:A2_structure_proteine}
  \QRCode[2]{https://www.youtube.com/watch?v=wvTv8TqWC48}
  \phantom{b}\vspace*{-20pt}
  
  \begin{encart}
    Un \important{polypeptide} est une chaîne d'acides $\alpha$-aminés.
    
    Une \important{protéine} est une \important{macromolécule} composée d'un ou plusieurs \important{polypeptides repliés.}
    C'est la géométrie tridimensionnelle de la protéine qui lui donne ses propriétés particulière.
  \end{encart}
  
  Une des plus petites protéine du corps humain est \important{l'insuline,} avec deux chaînes composées au total de 51 acides $\alpha$-aminés.
  La plus grande protéine du corps humain est la \important{titine,} composée d'une chaîne avec plus de \num{34000} acides $\alpha$-aminés.

  On peut décomposer la structure des protéines en quatre échelles :
  \begin{center}
    \image{0.6}{images/proteines/structure_proteines_insuline}
  \end{center}
\end{doc}

\begin{doc}{Production des protéines}{doc:A2_production_proteine}
  Les protéines sont produites dans les cellules avec l'information génétique contenue dans \important{l'ADN.}
  
  Pour produire une protéine, l'information génétique est transmise par les \important{ARN messagers} aux \important{ribosomes.}
  Les ribosomes assemblent les acides $\alpha$-aminés pour former des \important{peptides} en lisant les différent \important{codons} stockés dans la séquence nucléotidique de l'ARNm.
  Ces peptides sont ensuite \important{repliés} et assemblées dans la cellule pour leur donner leur structure tertiaire (ou quaternaire) et en faire une protéine fonctionnelle, avec les bonnes propriétés biologiques.

  Un mauvais repliement des peptides engendre des protéines inactive ou dysfonctionnelle, ce qui peut être dangereux pour l'organisme.

  La structure tertiaire peut aussi être détruite, en modifiant le pH ou en augmentant la température, ce qui rend inactive la protéine : on parle de \important{dénaturation.}
\end{doc}

\begin{doc}{Rôle des protéines dans l'organisme}{doc:A2_role_proteine}
  Les protéines sont souvent spécialisées pour remplir un rôle biologique et assurent le bon fonctionnement de notre organisme.
  On trouve ainsi des :
  \begin{listePoints}
    \item protéines \important{structurelles,} pour assurer la cohésion de certains tissus (kératine pour les ongles ou les cheveux, collagène pour la peau, titine dans les muscles) ou des cellules en formant leur cytosquelette ;
    \item protéines \important{transporteuses,} pour assurer le transfert de molécule dans et en dehors des cellules (hémoglobine qui transporte le dioxygène) ;
    \item protéines \important{régulatrices,} pour régler l'activité d'autres protéines ou pour contrôler l'expression des gènes ;
    \item protéines \important{de signalisation,} qui captent des signaux extérieurs pour les transmettre dans l'organisme ou dans une cellule, comme les protéines hormonales qui assurent la communication entre différentes parties du corps (insuline produite par les reins pour réguler les glucides dans le sang) ;
    \item protéines \important{réceptrices,} pour détecter les molécules ou les protéines envoyées par les autres cellules et agir en conséquence. On distingue
    \begin{listePoints}
      \item les \important{protéines sensorielles,} qui détectent les signaux environnementaux (lumière, température, etc.) et répondent en émettant d'autres signaux dans la cellule ;
      \item et les \important{récepteurs d'hormones,} qui détectent les hormones et entraînent une action de la cellule (la détection d'insuline entraine l'absorption du glucose) ;
    \end{listePoints}
    \item protéines \important{motrices,} pour permettre à certaines parties du corps de bouger (l'actine et la myosine permettent au muscle de se contracter) ;
    \item protéines \important{défensives,} pour protéger la cellule contre les agent infectieux (anticorps).
    \item protéines \important{de stockage,} pour stocker des acides aminés et permettre la biosynthèse d'autres protéines (l'ovalbumine dans le blanc d'oeuf permet le développement des embryons de poulet) ;
    \item protéines \important{enzymatiques,} pour modifier la vitesse de presque toutes les réactions chimiques qui ont lieu dans la cellule.  
  \end{listePoints}
\end{doc}

% \question{
%   Q
% }{}{2}

% https://fr.wikipedia.org/wiki/Prot%C3%A9ine
% https://pdb101.rcsb.org/motm/185
% https://fr.wikipedia.org/wiki/Hormone
% https://fr.wikipedia.org/wiki/Titine_(prot%C3%A9ine)