%%%%
\teteTermStssBiom

%%%% titre
\numeroActivite{4}
\titreActivite{Les vitamines}


%%%% objectifs
\begin{objectifs}
  \item Connaître la définition de « hydrosoluble » et « liposoluble ».
  \item Relier les quantités des besoins journaliers en vitamine avec leur caractère hydrosoluble.
  \item Savoir analyser la structure des vitamines A, C et D pour déterminer leur caractère lipo- ou hydrosoluble.
\end{objectifs}

\begin{contexte}
  Contrairement aux glucides, lipides et protéines, les vitamines n'ont pas une structure commune qui permet de les distinguer d'autres molécules.
  Les vitamines sont simplement des molécules essentielles au bon fonctionnement du corps humain et qui ne peuvent pas être synthétisée par un organisme humain.
  
  \problematique{
    Comment la solubilité des vitamines influence les besoins journaliers de chaque vitamine ?
  }
\end{contexte}


%%%%
\begin{doc}{Molécule hydrosoluble et liposoluble}{doc:A4_hydro_liposoluble}
  \begin{encart}
    Une molécule est \important{hydrosoluble} si elle est soluble dans l'eau.
    Une molécule est hydrosoluble si elle possède plusieurs \important{liaisons polaires.}
  \end{encart}
  
  Pour une molécule organique, les groupes hydroxyle \chemfig{O-H} sont des liaisons polaires.
  Une molécule organique qui possède plusieurs groupe hydroxyle sera donc hydrosoluble, car elle va former plusieurs liaisons hydrogène avec l'eau et sera très soluble dans l'eau.

  \begin{encart}
    Une molécule est \important{liposoluble} si elle \important{n'est pas hydrosoluble.}
  \end{encart}
\end{doc}

\begin{doc}{Structure des vitamines A, C et D}{doc:A4_vitamines_A_C_D}
  \centering
  \begin{tblr}{
    colspec = {X[1.5,c] X[0.5,c] X[c] X[c] X[4.5,c,m]}, hlines, vlines,
    column{1} = {couleurPrim!20},
    row{1} = {couleurPrim!10},
  }
    Solubilité & Vita- mine & Nom & Besoin journalier & Formule topologique \\
    %
    Hydrosoluble & C & Acide ascorbique & 
    \qty{110}{\milli\g\per\jour} &
    \chemfig[atom sep = 2em]{
      HO-[-1] -[1](-[3]OH) -[-1]  *5(-(-OH)=(-OH)-(=O)-O-)
    } \\
    %
    \SetCell[r=2]{c} Liposoluble & A & Rétinol & 
    \qty{0,750}{\milli\g\per\jour} &
    \chemfig[atom sep = 1.5em]{
      *6(--(-)= (
        -[1] =[-1] -[1](-[3]) =[-1] -[1] =[-1] -[1](-[3]) =[-1] -[1] OH
      )
      -(-[1])(-[5])--)
    } \\
    %
    & D & Cholé- calciférol &
    \qty{0,015}{\milli\g\per\jour} &
    {\small
    \chemfig[atom sep = 1.3em]{
      OH-[-1]
      *6(---(=)-( % premier cycle
        = -[::-60] =[::60] *6(- % second cycle
          *5(--- (-(-[::60]) -[::-60] -[::60] -[::-60] -[::60](-[::60]) -[::-60]) --) % troisieme cycle
        -(-[::0])----) % fin second cycle
      ) % fin de la chaine sur le premier cycle
      --) % fin premier cycle
    }} \\
    %
  \end{tblr}
\end{doc}

\begin{doc}{Stockage des vitamines dans l'organisme}{doc:A4_vitamine_organisme}
  Les vitamines liposolubles se trouvent principalement dans les aliments riches en matière grasse et sont stockées dans les tissus adipeux et le foie.

  Les vitamines hydrosolubles se trouvent dans de nombreux aliments et ne sont pas stockées dans l'organisme.
  Tout excédent en vitamine hydrosoluble est évacuée par les urines ou la sueur.
\end{doc}

%%
\question{
  En comparant les structures des vitamines A, C et D, justifier leur caractère hydrosoluble ou liposoluble.
}{
}{3}

\question{
  Expliquer pourquoi les besoins journalier en vitamine C sont beaucoup plus important que pour les vitamines A et D.
}{}{3}

\begin{doc}{Quelques vitamines communes}{doc:A4_vitamines}
  \centering
  \image{0.8}{images/organique/vitamins}
\end{doc}