%%%%
\teteTermStssDosa

%%%% titre
\vspace*{-30pt}
\numeroActivite{2}
\titreActivite{Enjeux sanitaire et milieux naturels}


%%%% objectifs
\begin{objectifs}
  \item Comprendre comment tracer une substance en milieu biologique ou naturel.
  \item Connaître les effets d’un polluant chimique sur la santé.
  \item Comprendre l'acidification d’une eau par dissolution du dioxyde de carbone.
\end{objectifs}


%%%% docs
\begin{doc}{Bioaccumulation et traçabilité}{doc:A2_bioaccu_tracabilite}
  Certains végétaux, animaux ou champignons peuvent absorber des substances chimiques dans leur organisme : c'est la \important{bioaccumulation.}

  \begin{importants}
    La \important{traçabilité} est la capacité à retracer tous le chemin suivi par une substance, du producteur au consommateur.
  \end{importants}

  Certains organismes comme les moules, les champignons ou les lichens absorbent les métaux présents dans le sol qui les entourent. 
  Ils sont utilisés comme \important{bio-indicateur} par les scientifiques et les agences publiques pour analyser la composition des milieux naturels.
  \medskip

  Par exemple, en Guyane on utilise des moules d'eau douce comme bio-indicateurs, pour mesurer la concentration en mercure dans les rivières. 
  Le mercure est rejeté illégalement dans les rivières par les orpailleurs (chasseurs d'or), car il va former des amalgames avec l'or des rivières, ce qui facilite sa récupération : il suffit de chauffer l'amalgame pour obtenir de l'or pur.

  Pour lutter contre l'or récupéré de cette façon, le \important{programme Traçabilité Analytique de l'Or (TAO)} a été mis en place, pour rendre difficile la vente d'or de provenance inconnue.
  L'orpaillage illégal est actuellement une des plus important source de déforestation et de pollution chimique au mercure en Guyane.
  Cela menace les populations locales, ainsi que la biodiversité des forêts et des rivières.
\end{doc}

%%
\begin{doc}{Particules fines}{doc:A2_particules_fines}
  Les particules fines provoquent des troubles respiratoires, cardiovasculaires et sont cancérigènes.
  Leur toxicité dépend de leur composition, mais aussi de leur taille. \important{Les plus petites particules peuvent passer dans le sang et sont plus dangereuses.}

  \begin{importants}
    On classe les particules par tailles :
    \begin{listePoints}
      \item les \important{PM$_{2,5}$} sont les particules avec un diamètre inférieur à \qty{2,5}{\micro\m} ;
      \item les \important{PM$_{10}$} sont les particules avec un diamètre compris entre \qty{2,5}{\micro\m} et \qty{10}{\micro\m}.
    \end{listePoints}
  \end{importants}

  Les principales sources de particules fines sont les chauffages individuels, certaines industries, l'agriculture et surtout les voitures.
  À cause de leur taille, \important{on ne peut pas éliminer les particules fines dans l'atmosphère,} mais on peut limiter leurs émissions.
  On peut contrôler la concentration en particules fines avec des détecteurs spécialisés, qu'il faut changer régulièrement, car ils s'encrassent rapidement.
\end{doc}

%%
\begin{doc}{Pollution des eaux par des hormones}{doc:A2_pollution_hormones}
  Les effluents en zones urbaines sont la source principale de \important{libération d'hormones} dans les milieux aquatiques, à cause d'un manque de traitement des eaux rejetées.

  \begin{importants}
    Les hormones rejetées se retrouvent dans nos aliments, l'eau potable et \important{peuvent perturber le fonctionnement du système endocrinien, même à très faible doses,} de l'ordre du \unit{\nano\g\per\litre}.
  \end{importants}
  Il est difficile d'établir un lien de cause à effet entre cette pollution aux hormones et des maladies, mais depuis des décennies il y a une augmentation importante de certaines maladies qui pourraient être lié à ses hormones.

  On ne comprend pas parfaitement les mécanismes d'élimination des hormones en milieu naturel, mais leur concentration évolue de la même manière que la population d'un échantillon radioactif.
  \begin{importants}
    La durée d'élimination d'une hormone est caractérisée par sa \important{demi-vie $t_{1/2}$}, qui est le temps nécessaire pour que la concentration initiale en hormone soit divisée par 2.
  \end{importants}
  
  La demie-vie peut varier de \important{quelques heures à plusieurs semaines,} selon l'hormone et les conditions de dégradations.
\end{doc}

%%
\begin{doc}{Acidification de l'eau et des océans}{doc:A2_acidification}
  En augmentant la concentration en dioxyde de carbone dans l'atmosphère, on augmente aussi la quantité de dioxyde de carbone dissoute dans les océans et les rivières.
  
  \begin{importants}  
    Quand du dioxyde de carbone est dissous dans de l'eau douce ou salée, cela entraine une diminution du pH.
  \end{importants}
  La concentration en ion oxonium \oxonium augmente, ce qui est néfaste pour les mollusques et les coraux, car l'ion oxonium vient dissoudre leur coquilles composée de calcaire, le carbonate de calcium \chemfig{Ca^{2+}} + \chemfig{CO_3^{2-}}.

  \begin{importants}
    Augmenter la concentration en dioxyde de carbone dans l'atmosphère entraine donc la dissolution des coquilles des mollusques, ce qui implique souvent leur mort.
  \end{importants}

  On peut comprendre ce phénomène en regardant les couples acides/bases impliqués.
  \begin{listePoints}
    \item Le dioxyde de carbone dissous dans l'eau, noté \chemfig{CO_2^{*}}, forme un couple acide/base avec l'ion hydrogénocarbonate : \chemfig{CO_2^{*}}/\chemfig{HCO_3^{-}}.
    \item L'ion hydrogénocarbonate forme un couple avec l'ion carbonate \chemfig{HCO_3^{-}}/\chemfig{CO_3^{2-}}.
    \item Ces couples acide/base vont réagir avec le couple acide/base de l'eau \oxonium/\eau.
  \end{listePoints}
\end{doc}


%%%%
\question{
  Écrire la réaction chimique entre l'eau et le dioxyde de carbone dissous dans l'eau.
}{
}{1}


\question{
  Écrire la réaction chimique entre les ions oxonium \oxonium et l'ion carbonate \chemfig{CO_3^{2-}}.
}{
}{1}

\question{
  Écrire la somme de ces deux équations et expliquer pourquoi la présence de dioxyde de carbone entraine la dissolution des coquilles de mollusques.
}{}{2}