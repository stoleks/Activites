%%%%
\teteTermStssImag

%%%% titre
\vspace*{-36pt}
\numeroActivite{5}
\titreActivite{Imagerie par résonance magnétique (IRM)}


%%%% objectifs
\begin{objectifs}
  \item Comprendre le principe de l'IRM et l'intérêt des produits de contraste.
\end{objectifs}

\begin{contexte}
  L'\important{I}magerie par \important{R}ésonance \important{M}agnétique est une technique d'imagerie médicale qui permet d'observer les tissus mous en temps réel.

  \problematique{
    Quels principes physiques permettent de former une image par IRM ?
  }
\end{contexte}


%%%% docs
\begin{doc}{Aimantation des noyaux}{doc:A5_aimantation_noyaux}
  Les noyaux de certains éléments chimiques se comportent comme des petits aimants.
  Les aimants qui sont placés dans un champs magnétique vont s'aligner avec celui-ci, comme le ferait une boussole, on parle \important{d'aimantation.}
  
  Si un noyau est entouré d'électrons, ils vont agir comme un écran et le protéger des champs magnétiques extérieurs.
  % Pour qu'un noyau soit sensible au champ magnétique, il ne doit pas être entouré de trop d'électrons, qui agissent comme un écran vis-à-vis du champ magnétique.

  C'est pourquoi les noyaux d'hydrogènes, les protons, qui se trouvent dans des \important{molécules simples}, comme la molécule d'eau, sont les plus sensibles au champ magnétique et présentent l'aimantation la plus forte.
\end{doc}

\begin{doc}{Principe d'une IRM}{doc:A5_principe_IRM}
  \begin{wrapfigure}[3]{r}{0.1\linewidth}
    \vspace*{-30pt}
    \qrcode{https://www.youtube.com/watch?v=49tgEcMA7kM}
  \end{wrapfigure}
  
  L'Imagerie par Résonance Magnétique est utilisée pour observer les tissus mous riche en molécules d'eau : cerveau, muscle, coeur, moelle épinière, etc.

  Le principe est le suivant : 
  \begin{listePoints}
    \item les tissus observés sont soumis à un fort champ magnétique $\vec{B}$ qui aimante les protons ;
    \item on émet une onde électromagnétique dans le domaine radio qui va être absorbée par les noyaux d'hydrogène, ce qui change l'orientation de leur aimantation ;
    \item en fonction de leur environnement, les noyaux d'hydrogènes vont retrouver plus ou moins rapidement l'orientation du fort champ magnétique $\vec{B}$, on parle de \important{relaxation} ;
    \item en mesurant l'orientation de l'aimantation au cours de cette relaxation, on peut ainsi créer une image contrastée des tissus en fonction de leur composition.
  \end{listePoints}
  
  \begin{tblr}{
      colspec = {X[c,m] X[c,m] X[c,m] X[c,m]},
      row{2} = {l, m},
    }
    \image{1}{images/electricite/IRM0001} &
    \image{1}{images/electricite/IRM0002} &
    \image{1}{images/electricite/IRM0003} &
    \image{1}{images/electricite/IRM0004} \\
    %
    Les noyaux d'hydrogènes ont des environnements différents en fonction des tissus. &
    Les protons s'orientent selon $\vec{B}$, ce qui donne une aimantation plus ou moins forte. &
    L'aimantation est pivotée de \qty{90}{\degree} sous l'action d'une onde radio perpendiculaire à $\vec{B}$. &
    L'aimantation reprend l'orientation de $\vec{B}$ et les protons émettent une onde radio. \\
  \end{tblr}
\end{doc}


\newpage
\vspace*{-20pt}
\question{
  Expliquer pourquoi les noyaux d'hydrogènes dans des molécules complexes comme les lipides ou les protéines s'aimantent moins bien que les noyaux d'hydrogènes dans des molécules d'eau.
}{}{3}

\question{
  Expliquer pourquoi les tissus osseux ne sont pas visible en IRM.
}{}{1}


%%%%
\begin{doc}{Les produits de contraste}{doc:A5_produits_contraste}
  En fonction du diagnostique souhaité, on peut ingérer ou injecter des produits de contraste dans le corps du ou de la patiente.
  Ces produits permettent d'améliorer la sensibilité au champ magnétique d'un tissu particulier, ce qui augmente son aimantation et permet de mieux le visualiser grâce au contraste.
  Par exemple pour visualiser les vaisseaux sanguins et donc détecter d'éventuels thromboses.

  Les produits de contraste utilisent l'ion gadolinium III noté \chemfig{Gd^{3+}}, qui est le plus sensible au champ magnétique.
  L'ion est couplé à des molécule, les ligands, pour former des chélates de gadolinium inoffensifs pour l'organisme.

  \begin{tblr}{X[c,m] X[c,m]}
    \chemfig{
      N (-[::-45, 0.9,,, draw = none] Gd^{3+}) 
        (-[::140] -[::-60] COO^{-}) -[::80] -[ 0] -[::-80]
      N (-[::140] -[::-60] COO^{-}) -[::70] -[-3] -[::-80]
      N (-[::120] -[::-60] COO^{-}) -[::70] -[-6] -[::-80]
      N (-[::140] -[::-60] ^{-}OOC) -[::70] -[ 3] -[::-80, 0.8]
    } &
    \chemfig{
      N (-[::-45, 0.9,,, draw = none] Gd^{3+}) 
        (-[::140] -[::-60] COO^{-})     -[::80] -[ 0] -[::-80]
      N (-[::140] -[0] (-[2] OH) -[-2]) -[::70] -[-3] -[::-80]
      N (-[::120] -[::-60] COO^{-})     -[::70] -[-6] -[::-80]
      N (-[::140] -[::-60] ^{-}OOC)     -[::70] -[ 3] -[::-80, 0.8]
    } \\
    %
    \important{1} & \important{2} \\
  \end{tblr}
  \begin{center}
    \chemfig{
      ^{-}OOC --[-2]
      N (-[-4]-[6] COOH)  -[2]-[0]-[-2]
      N (-[-3, 0.8,,, draw = none] Gd^{3+})
        (-[3]-[2] COO^{-})-[2]-[0]-[-2]
      N (-[-2]- COOH) -[2] - COO^{-}
    } \\[4pt]
    \important{3}
  \end{center}
\end{doc}


\question{
  Identifier le chélate qui contient une fonction alcool.
  Donner son nombre de fonction amine.
}{}{1}

\question{
  Identifier le chélate qui est un ion.
  Est-ce un cation ou un anion ?
}{}{2}

\question{
  Expliquer pourquoi on pourrait utiliser le fer à la place du gadolinium.
}{}{1}
