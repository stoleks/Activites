%%%%
\teteTermStssImag
\vspace*{-24pt}
\titreActivite{Utilisation de la radioactivité en médecine}

%%%% objectifs
\begin{objectifs}
  \item Comprendre la notion de dose absorbée.
  \item Comprendre la notion de dose équivalente.
  \item Connaître quelques utilisation médicale diagnostique et curative.
\end{objectifs}

\begin{contexte}
  La radioactivité est utilisée tous les jours en médecine pour diagnostiquer ou pour soigner des maladies.
  
  \problematique{
    Quelles sont les doses radioactives utilisées pour diagnostiquer ou guérir des maladies ?
  }
\end{contexte}


%%%% docs
\begin{doc}{Dose absorbée et dose équivalente}{doc:A7_dose}
  \begin{importants}
    La \important{dose absorbée $D$} se mesure en Gray noté \unit{\gray}
    \begin{equation*}
      D =
      \dfrac{\text{Énergie reçue pendant la désintégration (\unit{J})}}
      {\text{masse du corps recevant l'énergie (\unit{kg})}}
    \end{equation*}
  \end{importants}
  
  La dose absorbée mesure l'irradiation brute reçue, mais certaines particules sont plus dangereuses que d'autres à cause de leur masse.
  C'est pour ça qu'on introduit la dose équivalente $H$.
  \begin{importants}
    La \important{dose équivalente $H$} se mesure en sievert noté \unit{\sievert}
    \begin{equation*}
      H = w_R \times D
    \end{equation*}
    où $w_R$ est un facteur de pondération. $w_R$ vaut 1 pour les radioactivités $\beta^-$, $\beta^+$ et $\gamma$.
    $w_R$ vaut 20 pour la radioactivité $\alpha$.
  \end{importants}
\end{doc}

\begin{doc}{Échelle de dose reçue}{doc:A7_dose_recue}
  \centering
  \image{0.95}{images/exterieures/donnees/doses_radioactivite}    
\end{doc}

\newpage
\vspace*{-28pt}
\begin{doc}{Réglementation française}{doc:A7_reglementation}
  En France, une dose efficace annuelle $H$ est préconisé pour le grand public, en plus de la radioactivité naturelle et médicale.
  \begin{tableau}{|c |c |}
    Grand public & Personne travaillant avec des sources radioactives \\
    \qty{1}{\milli\sievert}/an & \qty{20}{\milli\sievert}/an
  \end{tableau}
\end{doc}  

\begin{doc}{Utilisation des radioéléments en médecine}{doc:A7_radioelement_medecine}
  \begin{tblr}{
    colspec = {X[l] X[-1, l] c c X[l]},
    column{1} = {couleurSec-50},
    row{1} = {couleurSec-100, c},
    row{1-6} = {m},
    hlines, vlines
  }
     Radioélément & Cible & Dose & Demi-vie & Application \\
     %
     Technétium : $\gamma$ &
     Peu Spécifique & 1 à 10 mSv &
     6 h & \SetCell[r=2]{l} Scintigraphie \\
     %
     Gallium : $\gamma$ &
     Colon, poumons & 30 mSv &
     78 h & Scintigraphie \\
     %
     Fluor : $\beta^+$ et $\gamma$ &
     Détection des cellules cancéreuses. Neurologie. & 7 mSv &
     110 min & PET par détection des rayon $\gamma$ de haute énergie \\
     %
     Samarium $\beta^-$ &
     Os, poumon, prostate, sein & \SetCell[r=2]{c} 2 Sv/séance &
     1,9 jours & \SetCell[r=2]{l} Radiothérapie métabolique \\
     %
     Yttrium $\beta^-$ & Foie & & 2,7 jours &
  \end{tblr}
\end{doc}

\question{
  On considère qu'une source radioactive est inoffensive passé 20 demi-vie.
  Calculer 20 fois la demi-vie pour chaque radioélément utilisé.
}{}{3}

\question{
  Pourquoi utilise-t-on des éléments avec de courtes demi-vie en médecine ?
}{}{3}

\question{
  Est-ce que les examens utilisant des radioéléments sont dangereux ?
}{}{2}

\question{
  Comparer les doses reçues lors d'un examen diagnostique et pendant une radiothérapie.
}{}{3}

\question{
  Chercher comment le personnel médical se protège des radiations.
}{}{3}
