%%%%
\teteTermStssImag

%%%% titre
\vspace*{-36pt}
\numeroActivite{4}
\titreActivite{Radiographie et radiothérapie}


%%%% objectifs
\begin{objectifs}
  \item Comprendre le principe de la radiographie.
  \item Comprendre le principe de la radiothérapie.
\end{objectifs}

\begin{contexte}
  Que ce soit pour diagnostiquer des blessures (radiographie) ou pour traiter des maladies (radiothérapie), les rayons X sont utilisés tous les jours en médecine.

  \problematique{
    Comment et pourquoi les rayons X sont-ils utilisés dans le milieu médical ?
  }
\end{contexte}


%%%% docs
\begin{doc}{Les rayons X}{doc:A4_rayon_X}
  \begin{center}  
    \image{0.75}{images/lumiere/spectre_EM_frequence}
  \end{center}
  
  La lumière est une \important{onde électromagnétique,} dont les propriétés dépendent de la \important{fréquence.}
  Plus la fréquence est élevée, plus les ondes électromagnétiques sont potentiellement dangereuse.
  
  La lumière est visible pour des yeux humains de 380 à \qty{790}{\tera\hertz} (\qty{1}{\tera\hertz} = \qty{e12}{\hertz} = mille milliards de hertz).
  Les autres fréquences sont invisibles.

  Les ondes électromagnétiques sont la propagation d'un champ magnétique et d'un champ électrique à \important{la vitesse de la lumière $\mathbf{c =} \qty{3,00e8}{\m\per\s}$.}
  Elles se propagent dans le vide comme dans les milieux matériels.
  
  \begin{importants}
    Les rayons X sont des ondes électromagnétiques de fréquences très élevées.
    En médecine les rayons X sont utilisés principalement pour
    \begin{listePoints}  
      \item faire de l'imagerie médicale : c'est la \important{radiographie} ;
      \item traiter des cancers : c'est la \important{radiothérapie.}
    \end{listePoints}
  \end{importants}

  Le rayonnement dans le domaine X est dangereux à forte dose, car il est suffisamment énergétique pour détruire des molécules !

  De par leur danger, les rayons X sont manipulés par des spécialistes et il faut limiter la durée d'exposition à ce rayonnement.
\end{doc}

\begin{doc}{Principe de la radiographie}{doc:A4_radiographie}
  La \important{radiographie} est une technique d'imagerie médicale utilisant des rayons X.
  Les rayonnement X sont très énergétiques et traversent plus ou moins la matière en fonction de sa composition et de son épaisseur.

  Pour réaliser une radiographie, il faut placer une plaque X-sensible sous l'objet que l'on veut observer, puis envoyer des rayons X à partir d'une source.
  La plaque X-sensible noircit si elle reçoit des rayons X et reste blanche si elle n'en reçoit pas.

  Si on irradie une main avec des rayons X pendant une durée très courte, alors :
  \begin{itemize}
    \item la peau et les muscles absorbent peu les rayons X, la plaque reçoit peu de rayon et noircit faiblement ;
    \item les os absorbent beaucoup les rayons X, la plaque reçoit très peu de rayon et apparaît presque blanche.
  \end{itemize}
\end{doc}

\begin{doc}{Absorption des rayons X}{doc:A4_absorption}
  \begin{wrapfigure}{r}{0.4\linewidth}
    \vspace*{-34pt}
    \image{0.725}{images/donnees/radiographie}

    \centering Radiographie d'une main    
  \end{wrapfigure}
  
  Les rayons X sont plus absorbés si les atomes qui composent la matière ont des numéro atomique $Z$ élevé.
  \begin{listePoints}
    \item La peau et les muscles sont essentiellement composés d'hydrogène ($Z = 1$), de carbone ($Z = 6$), d'azote ($Z = 7$) et d'oxygène  ($Z = 8$).
    Ils absorbent donc peu les rayons X et apparaissent gris.
    \item Les os sont essentiellement composés de phosphore ($Z = 15$) et de calcium ($Z = 20$).
    Ils absorbent beaucoup les rayons X et apparaissent presque blanc.
  \end{listePoints}
\end{doc}

\begin{doc}{Principe de la radiothérapie}{doc:A4_radiothérapie}
  La \important{radiothérapie} consiste à irradier suffisamment longtemps les cellules cancéreuses pour les tuer et éviter leur prolifération.
  Quand ils pénètrent dans la matière, les rayons X vont décharger leur énergies a une certaine profondeur que l'on connaît : on peut donc détruire finement un cancer en endommageant au minimum ce qu'il y a autour.

  Pendant une radiothérapie, le ou la patiente est donc soumis localement à des rayonnement X intense et prolongée.
\end{doc}


%%%%
\numeroQuestion
Légender la radio en précisant l'épaisseur et la composition atomique des milieux traversé.

\question{
  Rechercher le numéro atomique de l'or dans le tableau périodique et expliquer pourquoi on observe une ellipse blanche sur la radio.
}{}{3}

\question{
  Expliquer pourquoi la source de rayon X doit être proche de la patiente pendant une radiographie.
}{}{3}

\question{
  Comparer radiographie et radiothérapie.
  Trouver un point commun et deux différences.
}{}{4}

\question{
  Chercher et lister quelques effet néfastes sur la santé des rayons X s'ils sont utilisés à trop fortes doses.
}{}{3}