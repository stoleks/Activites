\teteTermStssImag
\vspace*{-30pt}
\titreActivite{La radioactivité}

\begin{objectifs}
  \item Comprendre le principe de la radioactivité.
  \item Savoir qu'il y a trois types de décomposition pour la radioactivité.
  \item Comprendre la notion d'activité et de temps de demi-vie.
\end{objectifs}

\begin{contexte}
  Les objets qui nous entourent sont composés d'atomes, qui peuvent être radioactifs.
  Par exemple les bananes sont naturellement radioactives.
  
  \problematique{
    Qu'est-ce que la radioactivité ?
    Comment la radioactivité évolue au cours du temps ?
  }
\end{contexte}


%%%% docs
\begin{doc}{Constitution d'un noyau et isotope}
  Un noyau atomique est constitué de neutrons et de protons, qui sont des nucléons.
  Un élément chimique est noté \isotope{X}{Z}{A}
  \begin{listePoints}
    \item A est le nombre de nucléons (protons + neutrons) ;
    \item Z est le numéro atomique ou le \important{nombre de charges} ;
    \item X est le symbole de l'élément (H, O, C, etc.).
  \end{listePoints}

  \begin{wrapfigure}{r}{0.4\linewidth}
    \vspace*{-60pt}
    \phantom{b}
    \image{1}{images/atomes/Carbone_12-13-14}
  \end{wrapfigure}
  Un même élément chimique peut avoir plusieurs noyaux différents, on parle \important{d'isotopes.}
  
  Par exemple le carbone a 3 isotopes
\end{doc}

\begin{doc}{Désintégration des noyaux instables}[\label{doc:noyaux_instables}]
  Le carbone 12 et le carbone 13 sont \important{stables}, tandis que le carbone 14 est \important{instable.}
  Les noyaux instables sont dit \important{radioactifs.} 
  Cette instabilité se traduit par une \important{désintégration inéluctable} du noyau.

  \begin{wrapfigure}{r}{0.4\linewidth}
    \centering
    \vspace*{-20pt}
    \image{1}{images/atomes/desintegration_carbone14}

    Désintégration du carbone 14 en azote 14.
  \end{wrapfigure}

  \begin{importants}  
    La désintégration est un phénomène aléatoire, on ne peut  donc pas prédire quand un noyau va se désintégrer.
  \end{importants}
  
  Par contre on sait avec certitude que le noyau instable, appelé noyau père, va se transformer en un autre noyau, appelé noyau fils, en libérant une particule au passage.
  
  Le noyau fils a généralement une énergie trop élevée, il est dans un état excité, noté avec un astérisque en exposant ($X^*$).
  Son énergie est abaissée par l'émission d'un rayonnement électromagnétique de très courte longueur d'onde, appelé rayonnement $\gamma$ (gamma).
  On parle de \important{désexcitation gamma.}

  \begin{tableau}{|c |c |}
    Type de désintégration & Particule émise \\
    $\alpha$ (alpha) & Noyau d'hélium \isotope{He}{2}{4} \\
    $\beta^-$ (beta moins) & Électron \isotope{e}{-1}{0} \\
    $\beta^+$ (beta plus) & Positron \isotope{e}{+1}{0}
  \end{tableau}

  \attention Vous devez être capable de reconnaître la particule émise au cours d'une désintégration.
\end{doc}


\newpage
\vspace*{-28pt}
\question{
  Noter les 3 noyaux isotopes du carbone en utilisant l'écriture \isotope{A}{Z}{X}.
}{}[1]

\question{
  Donner le sens du mot « inéluctable » utilisé dans le document~\ref{doc:noyaux_instables}.
}{}[2]

\question{
  Identifier les équations suivantes en indiquant s'il s'agit de désintégration $\alpha$, $\beta^-$, $\beta^+$ ou d'une désexcitation $\gamma$ :

  \vspace*{-20pt}
  \begin{equation*}
    a)\; \isotope{F}{9}{18} \rightarrow \isotope{O}{8}{18} + \isotope{e}{+1}{0}
    \qq{} 
    %
    b)\; \isotope{Xe}{54}{131}^* \rightarrow \isotope{Xe}{54}{131} + \gamma 
    \qq{} 
    %
    c)\; \isotope{C}{6}{14} \rightarrow \isotope{N}{7}{14} + \isotope{e}{-1}{0} 
    \qq{} 
    %
    d)\; \isotope{U}{92}{238} \rightarrow \isotope{Th}{90}{234} + \isotope{4}{2}{He} 
    \qq{}
    %
  \end{equation*}
  \vspace*{-16pt}
}{}[4]

\begin{doc}{Activité d'un échantillon de matière}
  L'activité représente le nombre de désintégration dans un échantillon de matière pendant une seconde.
  Elle se note $A$ et s'exprime en \unit{\becquerel}.

  \exemple Si $A = \qty{400}{\becquerel}$, alors chaque secondes 400 noyaux radioactifs se désintègrent.
\end{doc}

\begin{doc}{Évolution de la radioactivité au cours du temps}
  Les noyaux radioactifs instables forment des noyaux stables, non radioactifs, en se désintégrant.
  Au sein d'un échantillon de matière radioactive, l'activité ne fait donc que diminuer au cours du temps.

  \begin{wrapfigure}{r}{0.5\linewidth} 
    \vspace*{-24pt}
    \image{1}{images/atomes/decroissance_radioactive}
  \end{wrapfigure}
  \phantom{b}\vspace*{-12pt}
  
  \begin{importants}
    La \important{période radioactive $T$} ou \important{demi-vie radioactive $t_{1/2}$} est la durée nécessaire 
    \begin{listePoints}
      \item pour que la moitié des noyaux radioactifs dans un échantillon se désintègrent.
      \item donc pour que l'activité soit divisée par 2.
    \end{listePoints}
  \end{importants}  
  Quelques exemples de périodes radioactives \\[4pt]
  \centering
  \begin{tblr}{colspec = {l c c c}, vlines, hlines, row{1} = {couleurSec-100}}
    Noyau & \isotope{C}{}{14} & \isotope{I}{}{131} & \isotope{Po}{}{191} \\
    $T$ & \num{5730} ans & 8 jours & \qty{22}{\ms} 
  \end{tblr}
\end{doc}

\begin{multicols}{2}
  \begin{doc}{Variation de l'activité avec la nature de la source et sa masse}
    \begin{tableau}{|l |c |c |}
                    & Lait & Granite \\
      $A$ pour 1 kg & 70   & 1000 \\
      $A$ pour 2 kg & 140  & 2000 \\
    \end{tableau}
  \end{doc}
  
  \question{
    Calculer $2^{20}$ et indiquer pourquoi on peut considérer qu'il n'y a plus de radioactivité dans un échantillon au bout de 20 périodes radioactive.
  }{}[2]
\end{multicols}
