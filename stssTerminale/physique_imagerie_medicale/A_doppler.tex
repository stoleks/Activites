%%%%
\teteTermStssImag

%%%% titre
\numeroActivite{2}
\titreActivite{Principe d'une échographie doppler}


%%%% objectifs
\begin{objectifs}
  \item Comprendre le principe d'une échographie doppler.
\end{objectifs}

\begin{contexte}
  Pour vérifier que le sang circule normalement dans les vaisseaux sanguins, de nos jours on utilise l'échographie doppler, qui est précise et non-intrusive.
  
  \problematique{
    Comment fonctionne une échographie doppler ?
  }
\end{contexte}


%%%% docs
\begin{doc}{L'effet doppler}{doc:A2_effet_doppler}
  \begin{wrapfigure}[4]{r}{0.4\linewidth}
    \centering
    \vspace*{-35pt}
    \image{0.9}{images/acoustique/effet_doppler}
  \end{wrapfigure}
  
  Quand deux personnes sont à la même distance d'une source sonore immobile, elles entendent le même son.
  Mais si la source est en mouvement, chaque personne perçoit un son différent :
  \begin{listePoints}
    \item si la source se rapproche, le son parait plus aigu : \important{la fréquence de l'onde augmente} ;
    \item si la source s'éloigne, le son parait plus grave : \important{la fréquence de l'onde diminue}.
  \end{listePoints}
  
  \begin{importants}  
    C'est \important{l'effet doppler} : la fréquence de l'onde émise change lorsqu'il y a un mouvement relatif entre la source d'émission et la personne qui écoute.
  \end{importants}
\end{doc}


%%%%
\question{
  Yasmine entend la sirène d'une ambulance de plus en plus aiguë.
  L'ambulance se rapproche ou s'éloigne de Yasmine ?
}{
}{2}

%%
\begin{doc}{L'échographie doppler}{doc:A2_echographie_doppler}
  L'échographie doppler utilise aussi le phénomène d'écho, comme l'échographie simple.
  Une sonde est posée sur la peau recouverte d'un gel et émet des ultrasons.
  Les ultrasons se propagent dans le corps et sont réfléchis par les globules rouges dans les vaisseaux sanguins.

  \begin{wrapfigure}[10]{l}{0.5\linewidth}
    \centering
    \vspace*{-10pt}
    \image{1}{images/acoustique/echographie_doppler}
  \end{wrapfigure}

  Après réflexion, les ultrasons sont reçues par la sonde.
  La fréquence de l'onde sonore réfléchie varie en fonction de la fréquence de l'onde émise et de la vitesse de déplacement des globules rouges.

  Mesurer le décalage en fréquence $\Delta f$ entre la fréquence de l'onde émise et la fréquence de l'onde reçue, permet donc de déterminer la vitesse et le sens d'écoulement du sang dans les vaisseaux.
  \vspace*{13pt}

  \begin{importants}
    Le décalage se calcule avec la relation suivante :
    \begin{equation*}
      \Delta f = f_R - f_E = \dfrac{2vf_E \cos(\theta)}{c}
    \end{equation*}
  \end{importants}
  
  \begin{listePoints}[2]
    \item $f_E$ est la fréquence de l'onde émise en \unit{\hertz} ;
    \item $f_R$ est la fréquence de l'onde réfléchie en \unit{\hertz} ;
    \item $c$ est la célérité du son dans le corps en \unit{\m\per\s} ;
    \item $v$ est la vitesse des globules rouges en \unit{\m\per\s} ;
    \item $\theta$ est l'angle entre l'axe de la sonde et l'axe du vaisseau sanguin.
  \end{listePoints}
\end{doc}


%%%%
\question{
  Indiquer quels partie du corps humain réfléchit le son dans une échographie doppler et quelle est la grandeur mesurée.
}{
}{3}


\begin{doc}{Échographie doppler d'une artère}{doc:A2_doppler_artere}
  Les échographies doppler servent notamment à vérifier que les patient-es ne présentent pas de \important{sténose aortique}, c'est-à-dire une diminution du diamètre d'une artère.
  On peut exprimer cette diminution du diamètre en pourcentage par rapport à une taille normale.
  On a alors une évolution du signal mesuré avec une échographie doppler en fonction de l'avancée de la sténose.
  \begin{center}
    \image{0.6}{images/acoustique/doppler_stenose}
    
    Signal doppler visible en fonction de l'avancée de la sténose.
  \end{center}

  On compare deux images d'échographie doppler d'une artère rénale : celle d'un-e patient-e sans pathologie et celle d'un-e patient-e souffrant d'une sténose aortique.
  
  \begin{center}
    \image{0.8}{images/acoustique/doppler_stenose_artere}
    
    \separationBlocs{
      \hspace{4cm}
      \legende{A}
    }{
      \hspace{2cm}
      \legende{B}
    }
  \end{center}
\end{doc}


%%%%
\question{
  Analyser les échographies et en déduire qui de A ou B souffre de sténose aortique.
}{}{3}
