%%%%
\teteTermStssImag
\titreActivite{Diagnostiquer une hémochromatose}

\begin{objectifs}
  \item Utiliser les principes de l'échographie pour mener un diagnostique médical.
\end{objectifs}

\begin{contexte}
  L’hémochromatose est une maladie qui perturbe l’absorption du fer et nécessite la surveillance particulière du foie.

  \problematique{
    Comment diagnostiquer une hémochromatose avec une échographie ?
  }
\end{contexte}


%%%% docs
\begin{doc}{L'hémochromatose}
  Les hémochromatoses sont un groupe de maladies héréditaires autosomiques, 
  récessives dans l'immense majorité des cas,
  concernant le métabolisme du fer,
  et se caractérisant par une surcharge de fer dans l'organisme.
  À long terme, les dépôts ferriques engendrent des lésions anatomiques et fonctionnelles irréversibles.
  
  Une augmentation de la taille du foie peut être un symptôme de l'hémochromatose.
  Une échographie du foie permet de mesurer sa taille.
\end{doc}


\begin{doc}{Suivi échographique}[\label{doc:suivi_echo}]
  L’épaisseur d’un foie est normalement comprise entre \qty{8}{\cm} et \qty{12}{\cm}.
  Une sonde échographique, placée sur le ventre d’un patient,
  émet des ondes  ultrasonores de fréquence $f = \qty{6,0}{\mega\hertz}$.
  Les signaux reçus par la sonde sont représentés sur la figure de droite ci-dessous.

  \begin{center}
    \image{0.48}{images/acoustique/principe_echo_foie}
    \image{0.48}{images/acoustique/mesure_echo_foie}
  \end{center}
  
  L’instant $t = \qty{0}{\micro\s}$ correspond à l’émission du signal

  \begin{donnees}
    \item \qty{1}{\mega\hertz} = \qty{e6}{\hertz}.
    \item \qty{1}{\micro\s} = \qty{e-6}{\s}.
    \item Fréquences des ondes sonores audibles : de \qty{20}{\hertz} à \qty{20000}{\hertz}.
    \item Vitesse des ultrasons dans le corps humain $c = \qty{1540}{\m\per\s}$.
  \end{donnees}
\end{doc}


%%%%
\question{
  Justifier que les ondes utilisées sont des ultrasons.
}{
  Les ondes sonores utilisées ont une fréquence de $\qty{6,0}{\mega\hertz} = \qty{6e6}{\hertz} > \qty{20000}{\hertz}$, ce sont donc des ultrasons.
}[3]
\pasCorrection{\break}

\question{
  Rappeler le principe de l'échographie en précisant les phénomène physiques mis en jeu.
}{
  On émet une onde sonore, qui va traverser le corps humain.
  En le traversant, elle va se réfléchir sur les organes internes ou les os. 
  En mesurant les signaux réfléchis, comme on connait la vitesse du son dans le corps humain,
  on peut calculer des distances ou produire des images.
}[4]

\question{
  Expliquer la présence des deux signaux 1 et 2 reçus par la sonde et représentés dans le document~\ref{doc:suivi_echo}.
}{
  On a un signal qui est la réfléxion de l'onde sonore quand elle pénètre dans le foie,
  et on a un signal qui est la réfléxion de l'onde sonore quand elle sort du foie.
}[4]

\question{
  Montrer, à l’aide du document~\ref{doc:suivi_echo}, que la durée de propagation des ultrasons
  pour parcourir l’épaisseur $d$ du foie est $\Delta t = \qty{55}{\micro\s}$.
}{
  La durée entre la réception de ces deux signaux vaut $190 - \qty{80}{\micro\s} = \qty{110}{\micro\s}$.
  Cette durée correspond au temps mis par l'onde ultrasonore pour faire un aller-retour dans le foie,
  il faut donc diviser par 2 et on retrouve bien $\Delta t = \qty{55}{\micro\s}$.
}[4]

\question{
  Déterminer si le foie du patient a une épaisseur normale.
}{
  L'épaisseur du foie vaut, en faisant attention à convertir les microsecondes en secondes,
  \begin{align*}
    d &= c \times \Delta t \\
    &= \qty{1540}{\m\per\s} \times \qty{55e-6}{\s} \\
    &= \qty{0.085}{\m} \\
    &= \qty{8,5}{\cm}
  \end{align*}
  Le foie a donc une épaisseur normale comprise entre 8 et \qty{12}{\cm}.
}[5]

\question{
  Déterminer la distance entre la sonde et la paroi du foie la plus proche de la sonde.
}{
  L'onde ultrasonore a mis \qty{40}{\micro\s} pour aller de la sonde à la première paroi du foie, soit une distance $D$ qui vaut
  \begin{equation*}
    D = \qty{1540}{\m\per\s} \times \qty{40}{\micro\s} = \qty{6,1}{\cm}
  \end{equation*}
}[5]
