%%%%
\teteTermStssImag

%%%% titre
\numeroActivite{3}
\titreActivite{Diagnostiquer une hémochromatose}


%%%% objectifs
\begin{objectifs}
  \item Utiliser les principes de l'échographie pour mener un diagnostique médical.
\end{objectifs}

\begin{contexte}
  L’hémochromatose est une maladie qui perturbe l’absorption du fer et nécessite la surveillance particulière du foie.

  \problematique{
    Comment diagnostiquer une hémochromatose avec une échographie ?
  }
\end{contexte}


%%%% docs
\begin{doc}{L'hémochromatose}{doc:A3_hemochromatose}
  Les hémochromatoses sont un groupe de maladies héréditaires autosomiques, 
  récessives dans l'immense majorité des cas,
  concernant le métabolisme du fer,
  et se caractérisant par une surcharge de fer dans l'organisme.
  À long terme, les dépôts ferriques engendrent des lésions anatomiques et fonctionnelles irréversibles.
  
  Une augmentation de la taille du foie peut être un symptôme de l'hémochromatose.
  Une échographie du foie permet de mesurer sa taille.
\end{doc}


\begin{doc}{Suivi échographique}{doc:A3_suivi_echo}
  L’épaisseur d’un foie est normalement comprise entre \qty{8}{\cm} et \qty{12}{\cm}.
  Une sonde échographique, placée sur le ventre d’un patient,
  émet des ondes  ultrasonores de fréquence $f = \qty{6,0}{\mega\hertz}$.
  Les signaux reçus par la sonde sont représentés sur la figure de droite ci-dessous.

  \begin{center}
    \image{0.48}{images/acoustique/principe_echo_foie}
    \image{0.48}{images/acoustique/mesure_echo_foie}
  \end{center}
  
  L’instant $t = \qty{0}{\micro\s}$ correspond à l’émission du signal

  \begin{donnees}
    \item \qty{1}{\mega\hertz} = \qty{e6}{\hertz}.
    \item \qty{1}{\micro\s} = \qty{e-6}{\s}.
    \item Fréquences des ondes sonores audibles : de \qty{20}{\hertz} à \qty{20000}{\hertz}.
    \item Vitesse des ultrasons dans le corps humain $c = \qty{1540}{\m\per\s}$.
  \end{donnees}
\end{doc}


%%%%
\question{
  Justifier que les ondes utilisées sont des ultrasons.
}{
}{3}
\break

\question{
  Rappeler le principe de l'échographie en précisant les phénomène physiques mis en jeu.
}{}{4}

\question{
  Expliquer la présence des deux signaux 1 et 2 reçus par la sonde et représentés dans le document~\ref{doc:A3_suivi_echo}.
}{}{4}

\question{
  Montrer, à l’aide du document~\ref{doc:A3_suivi_echo}, que la durée de propagation des ultrasons
  pour parcourir l’épaisseur $d$ du foie est $\Delta t = \qty{55}{\micro\s}$.
}{}{4}

\question{
  Déterminer si le foie du patient a une épaisseur normale.
}{}{5}

\question{
  Déterminer la distance entre la sonde et la paroi du foie la plus proche de la sonde.
}{}{5}