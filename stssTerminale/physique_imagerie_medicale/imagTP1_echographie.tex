%%%%
\teteTermStssImag

%%%% titre
\numeroActivite{1}
\titreTP{Réalisation pratique d'une échographie}


%%%% objectifs
\begin{objectifs}
  \item Utiliser une démarche expérimentale pour comprendre le principe d'imagerie par échographie.
\end{objectifs}

\begin{contexte}
  En envoyant des ultrasons sur un corps humain, on observe que les ondes sont plus ou moins réfléchies en fonction des tissus rencontré.
  Si on détecte beaucoup d'ultrasons réfléchis sur des tissus dur, la zone correspondante apparaît blanche sur l’image (os).
  Si on ne détecte pas ou peu d’ultrasons réfléchis, c’est qu’ils se sont propagé dans des tissus mou : la zone apparaît sombre sur l’image (liquide ou membrane).
  
  \problematique{
    Comment reconstruire une image à partir des données mesurée pendant une échographie ?
  }
\end{contexte}


%%%%
\begin{doc}{Onde ultrasonore et matériau}{doc:TP1_ultrason_materiau}
  Quand une onde sonore dans le domaine des ultrasons arrive sur une surface,
  elle peut être \important{transmise}, \important{absorbée} ou \important{réfléchie} en fonction des propriétés du matériau.
  \begin{listePoints}
    \item \important{Transmission :} l'onde traverse le matériau ;
    \item \important{Absorption :} l'onde est absorbée par le matériau (son amplitude diminue) ;
    \item \important{Réflexion :} l'onde est réfléchie comme sur un miroir.
  \end{listePoints}

  En général, plus un matériau est dense et dur, plus il réfléchira bien les ondes ultrasonore.
\end{doc}

\begin{doc}{Matériel disponible}{doc:TP1_materiel}
  On dispose 
  \begin{listeTirets}
    \item d'un générateur \qty{12}{\volt} ;
    \item d'un émetteur d'ultrasons (noté E), qui émet autour de \qty{40}{\kilo\hertz} ;
    \item d'un récepteur d'ultrasons (noté R), sensible autour de \qty{40}{\kilo\hertz} ;
    \item d'un oscilloscope ;
    \item de câbles BNC et de câbles banane.
  \end{listeTirets}
\end{doc}

\begin{doc}{Protocole de mise en place}{doc:TP1_mise_en_place}
  \begin{protocole}
    \item Alimenter l’émetteur (E) d’ultrasons en \qty{12}{\volt} en mode salve et le relier à la voie 1 de l’oscilloscope.
    \item Placer le récepteur (R) à environ \qty{15}{\cm} en face de l'émetteur que l'on relie à la voie 2 de l'oscilloscope.
    \item Allumer et régler l'oscilloscope pour qu'il affiche les signaux des deux voies.
    \item Changer le calibre de la voie 2 pour augmenter la sensibilité verticale (en Volt par division : V/div) et obtenir 2 signaux de taille similaire à l’écran.
  \end{protocole}
\end{doc}

%%
\mesure Réaliser le protocole du document~\ref{doc:TP1_mise_en_place}, appeler le professeur en cas de soucis.

% \question{
%   Comparer la fréquence des signaux émis par l’émetteur (E) et reçus par le récepteur (R).
%   Comparer la tension maximale des signaux émis par l’émetteur (E) et reçus par le récepteur (R). Proposer une explication possible de ces résultats.
% }{
% }{2}

\mesure Placer une plaque entre l'émetteur et le récepteur.
Compléter le tableau concernant la capacité de transmission des différents matériaux avec les adjectifs : fort, moyen, faible, nul.

\smallskip
\begin{tblr}{
    colspec = {c X[c] X[c] X[c] X[c] X[c]},
    hlines, vlines
    column{1} = { couleurPrim!20 },
  }
  Matériau & & & & & \\
  Capacité de transmission & & & & & \\
\end{tblr}
\bigskip


\mesure 
Dans une échographie, l'émetteur et le récepteur sont côte à côte.
Placer l'émetteur et le récepteur côte à côte, puis placer des obstacles devant l'ensemble pour remplir le tableau suivant avec les adjectifs fort, moyen, faible, nul.


\smallskip
\begin{tblr}{
    colspec = {c X[c] X[c] X[c] X[c] X[c]},
    hlines, vlines,
    column{1} = { couleurPrim!20 },
  }
  Matériau & & & & & \\
  Capacité de réflexion & & & & & \\
\end{tblr}
\bigskip

\question{
  Mesurer la durée $\Delta t$ en seconde mise par les ultrasons pour faire l'aller-retour.
}{}{2}

\question{
  Trouver la relation entre la célérité $c$ de l'onde ultrasonore, le temps $\Delta t$ que met l'onde à faire l'aller-retour et la distance $d$ entre l'émetteur-récepteur et l'obstacle.
}{}{3}

\question{
  Calculer d, sachant que $c = \qty{340}{\m\per\s}$ dans l'air.
}{}{3}

\question{
  Vérifier cette mesure avec une règle.
}{}{2}

\mesure Pour comprendre le fonctionnement de l'échographie médicale, on utilise le dispositif précédant.
Une boîte en carton (ventre) contient un objet (foetus), que l'on va chercher à imager.

\question{
  Proposer et réaliser une démarche pour identifier la position de l'objet dans la boite.
}{}{6}