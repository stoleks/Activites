%%%%
\teteTermStssOrga

%%%% titre
\numeroActivite{2}
\vspace*{-34pt}
\titreActivite{Fonctions organiques et nomenclature}

%%%% objectifs
\begin{objectifs}
  \item Rappeler les familles organiques et la nomenclature
\end{objectifs}

\begin{contexte}
  Il existe des millions de molécules organiques, certaines avec des propriétés similaires

  \problematique{
    Comment classer, décrire et nommer ces molécules selon leur propriétés ?
  }
\end{contexte}


%%
\vspace*{-8pt}
\titreSection{Les fonctions organiques}

%%
\vspace*{-8pt}
\begin{doc}{Fonctions organiques}{doc:OA2_fonction_organique}
  Certaines séquences d'éléments donnent des \textbf{propriétés} spécifiques aux molécules organiques que l’on classe en différentes familles : alcane, alcène, alcyne, alcool, aldéhyde, cétone, acide carboxylique, ester, éther, amine, amide, etc.

  $R_1,$ $R_2$ et $R_3$ sont des chaînes carbonées appelées \og radicaux alkyles \fg.
  
  \begin{tblr}{
    width = \linewidth,
    colspec = {|c |c |c |X |}, hlines,
    column{2} = {couleurPrim!20},
    row{1} = {gray!20},
    cell{3}{1} = {r=2}{c},
    rows = {m}, columns = {c}
  }
    Groupe caractéristique & Famille fonctionnelle & Formule & Exemple \\
    %
    Hydroxyle & Alcool
    & \chemfig{R_1 - \textcolor{couleurQuat}{OH}} 
    & {\chemfig{-[1] -[-1] OH} \\[1pt] éthanol} \\
    %
    Carbonyle & Cétone
    & \chemfig{\textcolor{couleurQuat}{C} !\alkyleG !\cetoneCouleur R_2}
    & {\chemfig{-[1] !\carbonyle -[1]} \\[1pt] butan-2-one} \\
    
    %
    & Aldéhyde
    & \chemfig{\textcolor{couleurQuat}{C} !\alkyleG !\cetoneCouleur \textcolor{couleurQuat}{H}}
    & {\chemfig{!\carbonyle H} \\[1pt] méthanal ou formaldéhyde } \\
    %
    Carboxyle & Acide carboxylique
    & \chemfig{\textcolor{couleurQuat}{C} !\alkyleG !\cetoneCouleur \textcolor{couleurQuat}{OH}}
    & {\chemfig{-[-1] -[1] !\carboxyle} \\[1pt] acide propanoïque} \\
    %
    Ester & Ester
    & \chemfig{R_1 -[1] \textcolor{couleurQuat}{C} !\cetoneCouleur \textcolor{couleurQuat}{O} -[1] R_2}
    & {\chemfig{-[1] -[-1] -[1] !\ester -[1] -[-1]} \\[1pt] butanoate d'éthyle} \\
    %
    Éther-oxyde & Éther
    & \chemfig{R_1 -[1,,,,couleurQuat] \textcolor{couleurQuat}{O} -[-1,,,,couleurQuat] R_2}
    & {\chemfig{-[-1] -[1] O -[-1] -[1]} \\[1pt] éthoxyéthane} \\
    %
    Amine & Amine
    & \chemfig{R_1 - \textcolor{couleurQuat}{NH_2}}
    & {\chemfig{-[1] -[-1] -[1] NH_2} \\[1pt] propan-1-amine} \\
    %
    Amide & Amide
    & \chemfig{\textcolor{couleurQuat}{C} !\alkyleG !\cetoneCouleur \textcolor{couleurQuat}{N} (-[-3] R_3) - R_2}
    & {\chemfig{-[-1] -[1] !\amide H_2} \\[1pt] propanamide}
  \end{tblr}
\end{doc}

%%
\newpage
\vspace*{-24pt}
\titreSection{La nomenclature}

\begin{doc}{Principe de la nomenclature}{doc:OA2_principe_nomenclature}
  \begin{importants}  
    La \important{nomenclature} est l'ensemble des règles établies pour nommer les molécules organiques.
  \end{importants}
   
  La nomenclature moderne repose sur deux principes :
  \begin{listePoints}
    \item décrire la \textbf{géométrie} de la molécule nommée ;
    \item indiquer les \textbf{fonction organiques} présentes dans la molécule.
  \end{listePoints}
\end{doc}

%%
\begin{doc}{Nommer une chaîne carbonée}{doc:OA2_chaine_carbonee}
  Toute molécule organique possède au moins une chaîne carbonée.
  Pour nommer une chaîne carbonée, on va associer un \textbf{préfixe} avec un \textbf{suffixe}.
  Le suffixe dépend de la fonction organique, mais le préfixe est déterminé par le nombre de carbones qui composent la chaîne.
  \begin{importants}
  \begin{center}
    \begin{tblr}{
      columns = {c}, vlines, hlines,
      row{1} = {couleurPrim!20!white},
      column{1} = {gray!20}
    }
      Nombre de carbone \chemfig{C} 
      & 1 & 2 & 3 & 4 & 5 & 6 & 7 & 8\\
      Préfixe
      & meth- & éth- & prop- & but- & pent- & hex- & hept- & octa- \\
    \end{tblr}
  \end{center}  
  \end{importants}
\end{doc}

%%
\titreSousSection{Règles pour les alcanes, alcènes ou alcynes}

\begin{doc}{Les alcanes}{doc:OA2_alcanes}
  \separationBlocs{
    \begin{importants}
      Une molécule d'alcane est un \important{hydrocarbure} composé de \textbf{liaisons simples}.
    \end{importants}
    Pour nommer un alcane, il faut déterminer la chaîne carbonée la plus longue qui compose la molécule. \\
    On écrit alors le préfixe lié à la longueur de la chaîne et on ajoute le suffixe \og \textbf{-ane} \fg. \\
    Un alcane a toujours une formule brute de la forme \chemfig{ C_{n} H_{2(n + 1)} }.
  }{
    \vspace*{-26pt}
    \begin{boite}
      \vspace*{-6pt}
      \begin{importants}
        Un \important{hydrocarbure} est une molécule qui ne contient que des éléments carbones et hydrogènes.
      \end{importants}
      \begin{importants}
        Un hydrocarbure est \important{saturé} (en hydrogène) s'il ne comporte que des \textbf{liaisons simples}. \\
        Si l'hydrocarbure comporte des \textbf{liaisons doubles} ou \textbf{triples}, on dit qu'il est \important{insaturé.}
      \end{importants}
      \vAligne{-38pt}
    \end{boite}
  }
  
  \vspace*{4pt}
  \exemple \chemfig{H_3C - CH_2 - CH_3} trois carbones dans la chaîne, donc prop- $+$ -ane : propane.
\end{doc}

%%%% Question
\question{
  Nommer les molécules suivantes :
  \begin{equation*}  
    \chemfig{H_3C -CH_2 -CH_2 -CH_3} \qq{}
    \chemfig{-[1] -[-1] -[1] -[-1] -[1]} \qq{}
    \chemfig{H -C !\paireH -C !\saturationH}
  \end{equation*}
}{
  Butane, hexane et éthane.
}{2}


%%
\begin{doc}{Les alcènes}{doc:OA_2alcenes}
  \begin{importants}
    Les alcènes sont des hydrocarbures avec au moins une liaison double.
    Le suffixe \og -ane \fg, devient \og \textbf{-ène} \fg.
    On indique le (ou les) numéro de la liaison double avant le suffixe, de sorte que \textbf{le numéro soit le plus petit possible}.
  \end{importants}
  \exemple \chemfig{H_3C- CH - CH = CH_2 -CH_3} quatre carbones dans la chaîne (but-) et la liaison double se trouve en position 3 ou 2 (si on compte depuis la droite).
  Donc but $+$ 2 $+$ ène : but-2-ène.
\end{doc}

%%
\begin{doc}{les alcynes}{doc:OA2_alcynes}
  \begin{importants}
    Les alcynes sont des hydrocarbures avec au moins une liaison triple.
    Le suffixe \og -ane \fg, devient \og \textbf{-yne} \fg.
    On indique le (ou les) numéro de la liaison triple avant le suffixe, de sorte que \textbf{le numéro soit le plus petit possible}, comme pour les alcènes.
  \end{importants}
  \exemple \chemfig{-[1] ~[-1]} : trois carbones dans la chaîne (prop-) et la liaison triple se trouve en position 1.
  Donc prop-1-yne ou propyne (le 1 est implicite).
\end{doc}


%%
\titreSousSection{Règles pour les ramifications}

\begin{doc}{Ramification à la chaîne principale}{doc:OA2_ramification}
  \begin{importants}  
    Une \important{ramification} est un substituant qui remplace un hydrogène sur la chaîne principale.
  \end{importants}
  Si le substituant est un \important{alkyle} (un hydrocarbure), son nom prend le suffixe \og \textbf{-yl} \fg.

  \exemples \chemfig{CH_3 -[6]} : méthyl, \chemfig{CH_2 (-[6]) -CH_3} éthyl.
\end{doc}

\begin{doc}{Nommer une ramification}{doc:OA2_nom_ramification}
  \begin{importants}
  Pour nommer une molécule contenant des ramifications, il faut :
  \begin{listePoints}
    \item trouver la \textbf{plus longue chaîne carbonée} pour déterminer son nom.
    \item \textbf{Numéroter} la chaîne carbonée afin que la ramification ait le numéro le plus \textbf{petit possible}, comme pour les alcènes ou les alcynes.
    \item Placer le \textbf{numéro} et le \textbf{nom} de l'alkyle avant le nom de la chaîne.
    \item S'il y a plusieurs ramifications, leurs noms sont placés par ordre alphabétique.
  \end{listePoints}
  \end{importants}
\end{doc}

\question{
  Nommer les molécules suivantes :
  \begin{equation*}  
    \chemfig{H_3C- CH (-[3]CH_3) - CH (-[-3]CH_2 -CH_3) -CH_3} \qq{}
    \chemfig{H_3C- CH (-[3]CH_3) - CH (-[-3]CH_3) -CH_2 -CH_3}
  \end{equation*}
}{
  Pour la molécule 1 : la chaîne principale a 4 atomes, donc -butane.
  Deux ramifications sont en position 2 (avec un méthyl) et 3 (avec un éthyl).
  Donc le nom de cette molécule est 3-éthyl-2-méthyl-butane.

  Pour la molécule 2 : la chaîne principale a 5 atomes, donc -pentane.
  Deux ramifications méthyl sont en position 2 et 3.
  Donc le nom de cette molécul est 2,3-méthyl-pentane.
}{2}

%%
\newpage
\vspace*{-28pt}
\titreSousSection{Règles pour les groupes caractéristiques}

\begin{doc}{Groupes caractéristiques}{doc:OA2_nom_groupe_carac}
  \vspace*{-4pt}
  \begin{wrapfigure}[5]{r}{0.58\linewidth}
    \vspace*{-30pt}
    \centering
    \begin{tikzpicture}[help lines/.style={thin,draw=black!50}]
      % chaine principale et carbone fonctionnel
      \large
      \node[draw] at (3,3) { \chemfig{
        H_3C-CH-CH_2 -\textcolor{couleurSec}{\textsf{\textbf{C}}} H-CH_3
        }
      };
      \draw (5, 2.25) node[right] {\textbf{chaîne principale}};
      \draw[couleurSec] (3.7, 3.7) node[right] {\textbf{carbone fonctionnel}};
      % Ramification
      \draw[very thick, couleurPrim] (1.51, 2.79) -- (1.51, 2.29);
      \draw[couleurPrim] (2.5, 1.3)  node[left] {\textbf{ramification}};
      \node[draw, couleurPrim] at (1.8, 2) { \chemfig{CH_3} };
      % Alcool
      \draw[very thick, violet] (4.11, 2.79) -- (4.11, 2.29);
      \draw[violet] (3.6, 1.3)  node[right] {\textbf{groupe caractéristique}};
      \node[draw, violet] at (4.28, 2){ \chemfig{OH_{}} };
    \end{tikzpicture}
  \end{wrapfigure}
  %
  Pour nommer les molécules contenant des groupes caractéristiques, on utilise les règles décrites dans le tableau ci-dessous, en respectant la priorité des fonctions organiques.
  
  \begin{importants}
    Le \important{carbone fonctionnel} désigne le carbone contenant la fonction de la molécule.
  \end{importants}
  
  Pour les cétones, alcools et amines, le numéro est celui du \textbf{carbone fonctionnel}, comme pour les ramifications il \textbf{doit être le plus petit possible}.
  
  ($R_1$) et ($R_2$) représentent les noms des chaînes carbonées auxquels les groupes caractéristiques sont attachées. 

  \vspace*{4pt}
  \begin{tblr}{
    width = \linewidth,
    colspec = {|c |c |c |X |}, hlines,
    column{2} = {couleurPrim!20},
    row{1} = {gray!20},
    rows = {m}, columns = {c}
  }
    Priorité & Famille fonctionnelle & Formule & Nom si prioritaire \\
    %
    1 & Acide carboxylique
    & \chemfig{\textcolor{couleurQuat}{C} !\alkyleG !\cetoneCouleur \textcolor{couleurQuat}{OH}}
    & acide ($R_1$)-oïque \\
    %
    2 & Ester
    & \chemfig{\textcolor{couleurQuat}{C} !\alkyleG !\cetoneCouleur \textcolor{couleurQuat}{O} -[1] R_2}
    & ($R_1$)-oate de ($R_2$)-yle \\
    %
    3 & Amide
    & \chemfig{\textcolor{couleurQuat}{C} !\alkyleG !\cetoneCouleur \textcolor{couleurQuat}{N} H_2}
    & ($R_1$)-amide \\
    %
    4 & Aldéhyde
    & \chemfig{\textcolor{couleurQuat}{C} !\alkyleG !\cetoneCouleur \textcolor{couleurQuat}{H}}
    & ($R_1$)-al \\
    %
    5 & Cétone
    & \chemfig{\textcolor{couleurQuat}{C} !\alkyleG !\cetoneCouleur R_2}
    & ($R_1$)-(numéro)-one \\
    %
    6 & Alcool
    & \chemfig{R_1 - \textcolor{couleurQuat}{OH}}
    & ($R_1$)-(numéro)-ol \\
    %
    7 & Amine & \chemfig{R_1 - \textcolor{couleurQuat}{NH_2}}
    & ($R_1$)-(numéro)-amine \\
    %
    8 & Éther
    & \chemfig{R_1 -[1,,,,couleurQuat] \textcolor{couleurQuat}{O} -[-1,,,,couleurQuat] R_2}
    & ($R_1$)-oxy-($R_2$) \\
  \end{tblr}

  \vspace*{4pt}
  \attention Pour ces 8 familles organiques, vous devez savoir :
  \begin{listePoints}
    \item les noms de chacune des familles ;
    \item les reconnaître dans une molécule si on vous en donne une représentation ;
    \item le reconnaître si on vous donne le nom d'une molécule.
  \end{listePoints}
\end{doc}

\question{
  Nommer la molécule du document~\ref{doc:OA2_nom_groupe_carac}.
}{
  4-méthyl-pent-2-ol
}{1}