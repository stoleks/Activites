%%%%
\teteTermStssMedi
\titreActivite{Atténuer la douleur avec des médicaments}

\begin{objectifs}
  \item Comprendre qu'un médicament est composé d'un principe actif et d'un excipient.
  \item Comprendre le principe global des nanomédicaments.
\end{objectifs}

\begin{contexte}
  Au moins depuis l'antiquité, les humains se servent des plantes qui les entoure pour soigner leur douleur, notamment de décoction de feuilles de saules.
  
  \problematique{
    Quelle est l'origine de l'aspirine ?
  }
\end{contexte}


%%%% Aspirine
\begin{doc}{Premiers usage de l'acide salicylique}{doc:A1_acide_salicylique}
  Des traces d'utilisation médicinale de décoctions à base d'écorce de saule blanc, pour calmer les douleurs, ont été retrouvés dans tous les continents, dès l'antiquité.
  L'écorce de saule blanc contient de \important{l'acide salicylique,} « salix » signifiant « saule » en latin.

  Au cours du \siecle{19}\!, de nombreux médecins et chimistes essayent d'extraire l'acide salicylique de l'écorce des saules. 
  L'extraction est difficile et l'acide salicylique obtenu provoque de graves brûlure d'estomac, malgré son efficacité pour soulager la douleur.
\end{doc}

\begin{doc}{Première synthèse de l'aspirine}{doc:A1_synthese_aspirine}
  En 1897 un chimiste allemand travaillant chez Bayer, Felix Hoffman, parvient à synthétiser une forme pure et stable de\important{ l'acide acétylsalicylique.}
  L'acide acétylsalicylique est tout aussi efficace que l'acide salicylique pour calmer les douleurs, mais le tube digestif est bien moins irrité.

  L'industriel Bayer commence alors la production massive de l'acide acétylsalicylique, sous la forme d'un \important{médicament} breveté en 1899, \important{l'aspirine.}
\end{doc}

\begin{doc}{Action de l'acide acétylsalicylique dans le corps humain}{doc:A1_action_aspirine}
  \begin{wrapfigure}[10]{r}{0.45\linewidth}
    \vspace*{-12pt}
    \centering 
    \chemfig{!\acideSalicylique}
    \chemfig{!\aspirine}
    \smallskip

    \legende{
      Formules topologique de l'acide salicylique (gauche) et de l'acide acétylsalicylique (droite).
    }
  \end{wrapfigure}
  L'action de la molécule d'acide acétylsalicylique n'est expliqué qu'en 1971 par Pricilla Piper et John Vane, des biochimistes suédois.
  La molécule inhibe la formation de \important{prostaglandines,} des molécules importante pour la perception de la douleur dans notre corps.

  L'acide acétylsalicylique est le \important{principe actif} de l'aspirine, un des médicaments les plus utilisé dans le monde.
  L'aspirine est \important{antalgique} (calme la douleur), \important{antipyrétique} (diminue la fièvre), \important{anti-inflammatoire} (diminue les inflammations locales) et \important{antiagrégant plaquettaire} (limite la formation de caillot sanguin), ce qui la rend utile pour lutter contre certaines maladie cardio-vasculaires.
\end{doc}


%%%% 
\begin{doc}{Les médicaments}{doc:A1_medicaments}
  \begin{importants}
    Un \important{médicament} est « [...] toutes substances ou compositions présentées comme possédant des propriétés curatives ou préventives à l'égard des maladies humaines ou animales. » (loi du 26 février 2007).
  \end{importants}

  \begin{importants}  
    Un médicament est composé 
    \begin{listePoints}
      \item de un ou plusieurs \important{principes actifs,} une molécule dont la structure lui donne des propriétés thérapeutique ;
      \item \important{d'excipients,} qui permettent d'améliorer l'efficacité ou l'acceptabilité d'un médicament (goût, odeur, conservation, etc.).
    \end{listePoints}
  \end{importants}

  \begin{wrapfigure}[8]{r}{0.4\linewidth}
    \centering
    \vspace*{-10pt}
    \chemfig{!\paracetamol}
    \medskip

    \legende{Formule topologique du paracétamol}
  \end{wrapfigure}

  On peut donc avoir différents médicaments avec le même principe actif, mais différents excipients.
  C'est qui fait la différence entre un médicament breveté (comme l'aspirine) et un médicament générique qui est dans le domaine public.

  Un autre exemple est le doliprane (médicament breveté), dont le principe actif est le paracétamol (médicament générique).
  Contrairement à l'aspirine qui est fabriqué à partir d'acide salicylique extraite de l'écorce de saule blanc, le paracétamol est une molécule artificielle, fabriquée par des chimistes.
  \attention 
  Les médicaments génériques possèdent la même efficacité que les médicaments brevetés, mais ils sont moins cher. 
\end{doc}

\question{
  En comparant les structure de l'acide salicylique, de l'acide acétylsalicylique est-ce qu'on peut expliquer que leur action soit similaire ?
  Peut-on faire la même chose pour le paracétamol ?
}{}[3]


\begin{doc}{Les nanomédicaments}{doc:A1_nanomedicaments}
  \begin{wrapfigure}[6]{r}{0.4\linewidth}
    \vspace*{-20pt}
    \begin{center}
      \image{0.9}{images/sante/nanomedicament} \\
      \legende{Structure d'un nanomédicament}
    \end{center}
  \end{wrapfigure}

  Les \important{nanomédicaments} sont l'association d'un principe actif et d'un \important{vecteur,} qui va \important{transporter} le principe actif jusqu'à une zone précise du corps humain.

  \begin{importants}
    La \important{vectorisation} permet
    \begin{itemize}[label=\pointCyan, leftmargin=*]
      \item de cibler précisément certaines cellules malades ;
      \item de protéger le principe actif lors de son transport ;
      \item de diminuer la concentration de principe actif dans le médicament.
    \end{itemize}
  \end{importants}

  Les premiers nanomédicaments ont été développés au début des années 2000 pour lutter contre les cancers.
  Aujourd'hui une de leur utilisation majeure sont les vaccins de dernières génération à ARN, pour protéger et transporter l'ARN jusqu'aux cellules pour y fabriquer des protéines qui vont permettre de défendre notre organisme.
\end{doc}

\question{
  Quelle protéine naturelle permettant de transporter le cholestérol a une structure similaire à un nanomédicament ?
}{}[1]