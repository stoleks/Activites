\teteTermStssAlim
\titreActivite{Hydrolyse des triglycérides}

\begin{objectifs}
  \item Connaître la réaction modélisant l'hydrolyse d'un triglycérides
\end{objectifs}

\begin{contexte}
  Comme les glucides, les triglycérides sont une source importante d'énergie pour notre organisme.
  La dégradation des acides gras constituants les triglycérides permet de produire de \important{l'adénosine triphosphate,} ou \important{ATP,} qui fournit l'énergie nécessaire aux réactions chimiques du métabolisme cellulaire.
  
  \problematique{
    Quelle réaction chimique permet de séparer les triglycéride en acide gras ?  
  }
\end{contexte}

%%%% docs
\begin{doc}{Hydrolyse de l'oléine}[\label{doc:hydrolyse_oleine}]
  \begin{importants}
    \important{L'hydrolyse} (du grec « hydro » : eau et « lysis » : briser) est une réaction chimique enzymatique dans laquelle une liaison covalente est rompue par action d'une molécule d'eau.
  \end{importants}

  L'oléine est un triglycéride constituant \qty{80}{\percent} de l'huile d'olive.

  \begin{importants}
        Un \important{triglycéride} est un triester du glycérol avec trois acides gras. Un triglycéride est \important{saturé} si les trois acides gras qui le compose sont saturés. Il est \important{insaturé} sinon.
  \end{importants}
  
  Au cours de son absorption par l’organisme, \important{l’oléine} est \important{hydrolysée} à l'aide de la lipase pour former de \important{l'acide
oléique} selon l'équation suivante :
  \begin{center}
    \separationTroisBlocs{
      \centering
      \chemfig[atom sep = 15pt]{[:-40]!\oleine}
      
      + 3 \texteTrou[0.3]{\eau}
    }[0.48]{
      \reaction
    }[0.1]{
      \centering
      \chemfig[atom sep = 16pt]{!\oleique}
      
      + \chemfig[atom sep = 16pt]{!\oleique} 
      
      + \chemfig[atom sep = 16pt]{!\oleique} \\[8pt]
      
      + \chemfig[atom sep = 20pt]{!\glycerol}
    }[0.3]
  \end{center}
  La masse molaire de l'oléine est $M_\text{oléine} = \qty{884}{\g\per\mole}$.
\end{doc}

\question{
  Donner le nom des deux molécules formées au cours de la réaction d'hydrolyse.
}{
  On a du glycérol, formé une fois, et de l'acide oléique, formé trois fois.
}[2]

\schematisation
Dans le document~\ref{doc:hydrolyse_oleine}, entourer les groupes caractéristiques de l'oléine et d'un acide oléique.

\newpage
\vspace*{-28pt}
\question{
  Préciser si l'acide oléique est un acide gras saturé ou insaturé. Justifier.
}{
  C'est un acide gras insaturé, car il possède une double liaison entre deux carbones.
}[2]

L'organisme hydrolyse une masse d'oléine $m_\text{oléine} = \qty{8,84}{\g}$.

\question{
  La réaction est supposée totale. Calculer la quantité de matière d'eau $n_\text{eau}$ qui a été transformé au cours de la réaction.
}{}[5]

\question{
  Donner la quantité de matière d'acide oléique produite au cours de la réaction.
}{}[1]

%%
\begin{doc}{La palmitine}
  La palmitine est un des triglycérides les plus présents chez les être vivants, animaux ou végétaux.
  \begin{center}
    \separationBlocs{
      \centering
      \chemname{\chemfig[atom sep = 18pt]{!\palmitine}}
      { Formule topologique de la palmitine }
    }[0.6]{
      \centering
      \chemname{\chemfig{!\palmitineSemiDev}}
      { Formule semi-développée de la palmitine }
    }[0.38]
  \end{center}
\end{doc}


%%%%
\question{
  Indiquer si la palmitine est un triglycéride saturé ou insaturé. Justifier.
}{
  Elle est saturée, car elle ne possède pas de double liaison entre deux carbones.
}[2]

\question{
  Donner la réaction d'hydrolyse de la palmitine.
}{
  \begin{center}
    \separationTroisBlocs{
      \centering
      \chemfig[atom sep = 15pt]{[:-40]!\palmitine}
      
      + 3 \eau
    }[0.48]{
      \reaction
    }[0.1]{
      \centering
      \chemfig[atom sep = 16pt]{!\palmitique}
      
      + \chemfig[atom sep = 16pt]{!\palmitique} 
      
      + \chemfig[atom sep = 16pt]{!\palmitique} \\[8pt]
      
      + \chemfig[atom sep = 20pt]{!\glycerol}
    }[0.3]
  \end{center}
}[7]
