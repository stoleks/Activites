%%%%
\teteTermStssAlim
\titreActivite{Hydrolyse des triglycérides}


%%%% objectifs
\begin{objectifs}
  \item Connaître la réaction modélisant l'hydrolyse d'un triglycérides
\end{objectifs}

\begin{contexte}
  Les triglycérides sont une source importante d'énergie pour notre organisme.
  La dégradation des acides gras constituants les triglycérides permet de produire de \important{l'adénosine triphosphate,} ou \important{ATP,} qui fournit l'énergie nécessaire aux réactions chimiques du métabolisme cellulaire.
  
  \problematique{
    Quelle réaction chimique permet de séparer les triglycéride en acide gras ?  
  }
\end{contexte}
\vspace*{4pt}

%%%% docs
\begin{doc}{Hydrolyse de l'oléine}{doc:A2_hydrolyse_oleine}
  \begin{importants}
    \important{L'hydrolyse} (du grec \og hydro \fg: eau et \og lysis \fg : briser) est une réaction chimique et enzymatique dans laquelle une liaison covalente est rompue par action d'une molécule d'eau.
  \end{importants}

  L'oléine est un triglycéride constituant \qty{80}{\percent} de l'huile d'olive.
  
  Au cours de son absorption par l’organisme, \important{l’oléine} est \important{hydrolysée} à l'aide de la lipase pour former de \important{l'acide
oléique} selon l'équation suivante :
  \begin{center}
    \separationTroisBlocs{
      \chemfig[atom sep = 14pt]{!\trioleine}
      + 3 \texteTrou[0.3]{\eau}
    }[0.3]{
      \reaction
    }[0.1]{
      \centering
      \chemfig[atom sep = 14pt]{H!\oleique}
      + \chemfig[atom sep = 14pt]{H!\oleique} 
      + \chemfig[atom sep = 14pt]{H!\oleique}
      + \chemfig{!\glycerol}
    }[0.4]
  \end{center}
  La masse molaire de l'oléine est $M_\text{oléine} = \qty{884}{\g\per\mole}$.
\end{doc}

\numeroQuestion
Dans le document~\ref{doc:A2_hydrolyse_oleine}, entourer les groupes caractéristiques de la molécule d'oléine et d'une molécule d'acide oléique.

\question{
  Donner le nom des deux molécules formées au cours de la réaction d'hydrolyse.
}{}{2}


\newpage
\vspace*{-24pt}
\question{
  Préciser si l'acide oléique est un acide gras saturé ou insaturé. Justifier.
}{}{2}


L'organisme hydrolyse une masse d'oléine $m_\text{oléine} = \qty{8,84}{\g}$.

\question{
  La réaction est supposée totale. Calculer la quantité de matière d'eau $n_\text{eau}$ qui a été transformé au cours de la réaction.
}{}{5}

\question{
  Donner la quantité de matière d'acide oléique produite au cours de la réaction.
}{}{1}



%%
\begin{doc}{La palmitine}{doc:A2_hydrolyse_palmitine}
  La palmitine est un des triglycérides les plus présents chez les être vivants, animaux ou végétaux.
  \begin{center}
    \separationBlocs{
      \centering
      \chemfig[atom sep = 18pt]{[:-60] !\tripalmitine} \\[8pt]
      Formule topologique de la palmitine
    }[0.6]{
      \chemfig{
        H C (!\teteAcideDev C_{15} H_{31}) 
        (-[3,1.7,2,2] H_2C (!\teteAcideDev C_{15} H_{31}))
        -[-3,1.7,2,2] H_2 C (!\teteAcideDev C_{15} H_{31})
      } \\[4pt]
      
      Formule semi-développée de la palmitine
    }[0.3]
  \end{center}
\end{doc}


%%%%
\question{
  Indiquer si la palmitine est un triglycéride saturé ou insaturé. Justifier.
}{}{2}

\question{
  Donner la réaction d'hydrolyse de la palmitine.
}{}{7}
