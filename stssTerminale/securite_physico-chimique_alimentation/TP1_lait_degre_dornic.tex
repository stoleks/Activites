%%%%
\teteTermStssAlim

%%%% titre
\numeroActivite{1}
\titreTP{Contrôler la fraicheur d'un lait}


%%%% objectifs
\begin{objectifs}
  \item Déterminer la fraîcheur d’un lait conformément aux normes de santé publique.
\end{objectifs}

\begin{contexte}
  En tant qu’inspecteur-ice Hygiène et Sécurité, vous avez prélevé du lait dans le réfrigérateur d’un restaurant.
  
  \problematique{
    Les client-es du restaurant peuvent-ils consommer sans risque ce lait ?
  }
\end{contexte}


%%%% docs
\begin{doc}{Le degré Dornic}{doc:TP1_degre_dornic}
  Pierre Dornic, ingénieur agronome du XIXème siècle, a effectué de nombreuses recherches sur le lait et ses constituants.
  Dans l’industrie laitière, l’acidité d’un lait n’est pas exprimée par son pH mais par son degré Dornic : un degré Dornic (\unit{\dornic}) correspond à une concentration de \qty{0,1}{\g} d’acide lactique par litre de lait.
\end{doc}

%%
\begin{doc}{Degré Dornic et fraîcheur du lait}{doc:TP1_dornic_fraicheur}
  Le pH du lait dépend de son état de fraîcheur.
  Il est d’environ \num{6,7} pour un lait frais puis il diminue au cours du temps.
  L’acidité naturelle du lait est due à la présence de nombreuses espèces chimiques,
  comme la caséine ou l’acide lactique.
  
  Les bactéries qui prolifèrent dans le lait transforment le lactose,
  un sucre présent dans le lait, en acide lactique de formule \chemfig{C_3 H_6 O_3}.
  Si la quantité d’acide lactique présente est trop grande alors le lait n’est plus frais et ne doit plus être consommé.
  
  \important{Pour éviter les troubles digestifs, le lait que nous consommons doit être frais.}
  
  Lorsque l’acidité augmente, la caséine (protéine) coagule : on dit que « le lait tourne ».
  L’acidité croît avec le temps, c’est donc un bon critère d’évaluation de la fraîcheur.
  La fraîcheur d’un lait est caractérisée par son degré Dornic \unit{\dornic}.

  \begin{tblr}{
    colspec = {X[c] | X[c] | X[c] | X[c] | X[c]},
    vline{2} = {1}{text = \clap{\qty{18}{\dornic}}},
    vline{3} = {1}{text = \clap{\qty{35}{\dornic}}},
    vline{4} = {1}{text = \clap{\qty{50}{\dornic}}},
    vline{5} = {1}{text = \clap{\qty{100}{\dornic}}},
  }
    & & & & \\ \hline
    %
    Lait frais &
    Le lait caille en chauffant &
    Le lait caille à température ambiante &
    Le yaourt tourne &
    Yaourt « bulgare »
  \end{tblr}
\end{doc}

%%%%
\numeroQuestion Lire les documents et compléter le schéma du document~\ref{doc:TP1_dispositif_experimental}.

\mesure Préparer le dispositif expérimental en suivant les consignes du document~\ref{doc:TP1_preparation}.

\mesure Réaliser un \important{premier dosage rapide :}
Verser \unit{\ml} par \unit{\ml} la solution titrante de soude dans l'erlenmeyer.
Quand la solution change de couleur et que la couleur persiste après agitation, noter la valeur approximative $V_1$ du volume de soude versé.

\mesure Réaliser un \important{deuxième dosage précis :}
\begin{protocole}
  \item Toujours en agitant, recommencer le dosage de zéro.
  \item Verser rapidement jusqu'à $V_1 - \qty{1}{\ml}$, puis verser goutte-à-goutte.
  \item Arrêter de verser au changement de couleur et noter le volume de soude versé $V_e =$ \texteTrou[0.1]{\qty{4,5}{\ml}}
\end{protocole}

\question{
  Les ions hydroxydes \hydroxyde et l'acide lactique \chemfig{C_3 H_6 O_3} sont présents initialement dans le milieu réactionnel lors du dosage.
  En déduire l’équation de la réaction acido-basique de dosage. 
  
  \important{Couples acide/base :} \chemfig{C_3 H_6 O_3}/\chemfig{C_3 H_5 O_3^{-}} et \eau/\hydroxyde
}{
}{1}


%%
\begin{doc}{Dispositif expérimental}{doc:TP1_dispositif_experimental}
  \begin{center}
    \separationBlocs{
      \image{1}{images/chimie/montage_dosage}
    }{
      \texteTrou{Burette} \\[5.4cm]
      Solution de soude $c_b = \qty{0,05}{\mol\per\litre}$ \\[2.5cm]
      \texteTrou[0.5]{Erlen meyer} \\[1.4cm]
      Lait $V_a = \qty{10}{\ml}$ + eau \qty{75}{\ml} + indicateur \\[0.15cm]
      \texteTrou[0.5]{Barreau aimantée} \\
      \texteTrou[0.5]{Agitateur magnétique}
    }    
  \end{center}
  
  L'acide lactique est complètement consommé par la solution de soude quand le pH de la solution augmente rapidement.
  C'est-à-dire quand l'indicateur coloré change de couleur.
\end{doc}

%%
\begin{doc}{Préparation du dispositif expérimental}{doc:TP1_preparation}
  \begin{protocole}
    \item Préparation de la burette, à compléter avant tout dosage :
    Introduire la solution titrante de soude (\ionSodium + \hydroxyde) de concentration $c_b = \qty{0,05}{\mol\per\litre}$ (rincer au préalable la burette avec de l’eau distillée puis avec un peu de solution de soude).
    \item Prélever un volume $V_a = \qty{10}{\ml}$ de lait de concentration molaire $c_a$ inconnue en acide lactique à l’aide d’une pipette jaugée (rincée avec le lait ) et les mettre dans un erlenmeyer de \qty{250}{\ml}.
    \item Ajouter \qty{75}{\ml} d’eau distillée et quelques gouttes
    % de phénolphtaléine. La phénolphtaléine est un indicateur coloré incolore en milieu acide et fuschia lorsque $\text{pH} > 10$.
    de bleu de bromothymol. Le bleu de bromothymol est un indicateur coloré jaune en milieu acide et bleu lorsque $\text{pH} > \num{7,6}$
    \item Mettre l’agitateur magnétique en marche.
  \end{protocole}
\end{doc}


\begin{doc}{Équivalence}{doc:TP1_equivalence}
  Lorsque tous l'acide lactique a été consommé, on dit qu'on a atteint \important{l'équivalence.}
  La quantité de matière d’ions hydroxyde \hydroxyde versée est alors égale au nombre de mole
  d’acide lactique contenu dans $V_a = \qty{10}{\ml}$ de lait.
\end{doc}

\question{
  Donner la relation entre la quantité de matière $n_a$ d'acide lactique et la quantité de matière $n_b$ d'ions hydroxyde.
}{}{1}

\question{
  Exprimer $n_a$ en fonction de $c_a$ et $V_a$.
}{}{2}

\question{
  Exprimer $n_b$ en fonction de $c_b$ et $V_e$.
}{}{2}

\question{
  En utilisant les 3 relations précédentes, donner l'expression littérale de $c_a$ en fonction de $c_b$, $V_e$ et $V_a$.
  Calculer la valeur de $c_a$.
}{}{3}

\question{
  Calculer la quantité de matière $n_a$ d'acide lactique contenue dans \qty{1}{\litre} de lait.
}{}{2}

\question{
  Calculer la masse d'acide lactique contenue dans \qty{1}{\litre} de lait.
  \important{Donnée :} $\text{M(acide lactique)} = \qty{90}{\g\per\mole}$.
}{}{2}

\question{
  En déduire le degré Dornic du lait. Ce lait est-il frais ?
}{}{4}