\teteTermStssAlim
\vspace*{-36pt}
\titreActivite{Conservation des huiles végétales}
\vspace*{-12pt}

\begin{objectifs}
  \item Revoir la structure des acides gras.
  \item Revoir la définition d'un triglycéride.
  \item Connaître les facteurs responsables de la dégradation d'une huile.
\end{objectifs}

\begin{contexte}
  Les vergetures sont des petites stries pouvant apparaître sur la peau, particulièrement au moment de la grossesse. 
  Pour les prévenir, des massages à l’huile sont recommandés afin de nourrir la peau en profondeur et d’en conserver l’élasticité.
  
  \problematique{
    Comment conserver les huiles végétales utilisées ?
  }
\end{contexte}


%%%% docs
\begin{doc}{Les huiles végétales}[\label{doc:huiles_vegetales}]
  \begin{importants}
    Une huile végétale est composée de \important{triesters de glycérol} et \important{d'acides gras} \important{saturés} ou \important{insaturés.}
  \end{importants}
  
  \begin{wrapfigure}{r}{0.22\linewidth}
    \centering
    \chemfig{!\glycerol} \\[4pt]
    {\small Glycérol}
  \end{wrapfigure}
  
  Par exemple, l'huile de coco est composée majoritairement de triesters \important{d'acide laurique} et \important{d'acide myristique}, et en quantité plus faible, d'autres acides tels que \important{l'acide oléique.}
  L'huile d'amande douce est composée en grande majorité de triesters \important{acides oléique} et \important{linoléique.}
  Comme elle rancit facilement, contrairement à l'huile de coco, il est nécessaire de l'acheter en petite quantité.
  
  \begin{center}
    Modèles moléculaires de quelques acides gras :
  \end{center}
  \vspace*{-24pt}
  \begin{multicols}{2}
    \centering
    \image{1}{images/molecules/biomolecules/acide_laurique} \\
    {\small Acide laurique}
    
    \image{1}{images/molecules/biomolecules/acide_oleique} \\
    {\small Acide oléique}
    
    \image{1}{images/molecules/biomolecules/acide_mystirique} \\
    {\small Acide myristique}
    
    \image{1}{images/molecules/biomolecules/acide_linoleique} \\
    {\small Acide linoléique}
  \end{multicols}
\end{doc}

\question{
  Préciser l'autre nom des triesters de glycérol et d'acides gras.
}{}[1]

\question{
  À partir des modèles moléculaires des acides gras représentés dans le document~\ref{doc:huiles_vegetales}, justifier le nom d'acide donné à ces espèces.
}{}[2]

\newpage
\vspace*{-30pt}
\question{
  Classer les acides gras du document~\ref{doc:huiles_vegetales} en acide gras saturés et insaturés. Justifier.
}{}[4]


%%
\begin{doc}{Dégradation des huiles végétales}
  Si les acides gras contenus dans une huile se dégradent, l’huile perd une partie de ses propriétés, change de couleur et développe une odeur de rance.
  \begin{importants}  
    \important{L’oxydation} est le principal phénomène à l’origine de cette dégradation.
  \end{importants}
  Le rancissement ne s'observe qu'avec des huiles contenant des graisses insaturées, car l'oxydation se fait au niveau des doubles liaisons carbone-carbone.
  \begin{importants}  
    Certains facteurs accélèrent cette oxydation comme l’exposition au dioxygène de l’air, à des température élevée, à la lumière (UV), etc.
  \end{importants}
  \begin{importants}
    Au contraire certains facteur ralentissent cette oxydation, comme \texteTrou(2){des température faibles, l'obscurité, l'absence de dioxygène, etc.}
  \end{importants}
\end{doc}


%%%%
\question{
  Expliquer la différence de comportement d'une huile d'amande et d'une huile de coco face au rancissement.
}{}[3]

\begin{doc}{Emballage d'une huile}
  L'emballage contenant un flacon d'huile d'amande douce mentionne \og Précaution de stockage : Conserver à l’abri de la chaleur et de la lumière \fg.
\end{doc}

\question{
  Justifier ces recommandations de stockage.
}{}[3]

\question{
  Justifier alors que le flacon en verre soit de couleur brune.
}{}[2]

\begin{doc}{Indice d’iode d’une huile végétale}[\label{doc:indice_iode}]
  \important{L'indice d'iode $I_\text{iode}$} d'une huile est la masse de diiode \diiode, exprimée en gramme, se fixant sur les doubles liaisons des acides gras contenus dans \qty{100}{g} d’huile.
  
  L'indice d'iode d'un acide gras saturé est donc nul.
  On modélise la réaction du diiode \diiode sur un acide gras insaturé possédant une seule double liaison par l’équation :
  \begin{center}
    \chemfig{R_1- CH= CH- R_2} + \diiode \reaction \chemfig{R_1- CHI- CHI- R_2}
  \end{center}
  C'est-à-dire que chaque double liaison réagit avec une molécule de diiode.

  \begin{importants}
    L'indice d'iode permet de déterminer le degré d'insaturation d'un acide gras.
  \end{importants}
\end{doc}

\begin{doc}{Indices d'iodes de l'huile de coco et d'amande}
  L'indice d'iode d'une huile de coco est compris entre \num{6} et \qty{11}{\g} de diiode \diiode pour \qty{100}{\g} d'huile,
  alors que celui d'une huile d'amande douce est compris entre \num{92} et \qty{109}{\g} pour \qty{100}{\g} d'huile.
\end{doc}

\question{
  Justifier qualitativement cette différence entre les deux huiles.
}{}[2]

\question{
  En utilisant le document~\ref{doc:indice_iode}, déterminer la quantité de matière de diiode \diiode qui peut réagir avec une mole d'acide linoléique.
}{}[2]

\question{
  Une quantité de matière $n(\text{linoléique}) = \qty{0,010}{\mole}$ d'acide linoléique réagit
  avec une masse $m(\diiode) = \qty{5,1}{\g}$ de diiode \diiode.
  Calculer la quantité de matière de diiode et vérifier qu'on retrouve bien le nombre de doubles liaisons que contient une molécule d’acide linoléique.
  
  \important{Données :} M(\diiode) = \qty{254,0}{\g\per\mole}
}{
  On utilise la relation $n(\diiode) = \dfrac{m(\diiode)}{M(\diiode)}$, soit
  \begin{equation*}
    n(\diiode) = \dfrac{5,1 \unit{g}}{254,0 \unit{\g\per\mole}}
    = \qty{0,020}{\mole}
  \end{equation*}

  On trouve donc $n(\diiode) = 2\times n(\text{linoléique})$, soit 2 doubles liaisons.
}[4]


\question{
  Une quantité de matière $n(\text{linolénique}) = \qty{0,020}{\mole}$ d'acide $\alpha$-linolénique réagit avec une masse $m(\diiode) = \qty{15,2}{\g}$ de diiode \diiode.
  Calculer le nombre de double liaisons que contient une molécule d'acide $\alpha$-linolénique.
}{
  On calcule la quantité de matière de diiode qui a réagit $n(\diiode) = m(\diiode) / M(\diiode) = \qty{0,060}{\mole}$. 

  On trouve que $n(\diiode) = 3 \times n(\text{linolénique})$, il y a donc 3 liaisons doubles dans l'acide $\alpha$-linolénique.
}[4]


\newpage
\textit{Pour les plus rapides}

\question{
  Donner les formules brutes des acide laurique, myristique, oléique et linoléique.
}{}[4]

\question{
  Donner la formule topologique des acides laurique, myristique, oléique et linoléique.
}{}[8]
