%%%%
\teteTermStssOrga

%%%% titre
\numeroActivite{1}
\vspace*{-30pt}
\titreActivite{Représenter des molécules organiques}

%%%% objectifs
\begin{objectifs}
  \item Rappeler les règles de formation des molécules et la valence d'un atome
  \item Rappeler les différentes représentations des molécules organiques
\end{objectifs}

\begin{contexte}
  Les atomes de carbones peuvent se lier entre eux pour former des \important{chaînes carbonées,} de formes et de tailles variées.
  Ces chaînes carbonées, une fois liée à des atomes d'hydrogène, d'oxygène ou d'azote, forment des \important{molécules organiques.}
  Il existe ainsi des millions de molécules organiques différentes.

  \problematique{
    Comment peut-on représenter ces molécules ?
  }
\end{contexte}


%%
\vspace*{-8pt}
\titreSection{La valence}
\vspace*{-8pt}

%%
\begin{doc}{Éléments composant un corps humain}{doc:A1_element_corps_humain}
  Le corps humain est composé majoritairement de 4 éléments chimiques :
  \vspace*{-4pt}
  \begin{multicols}{2}
  \begin{listePoints}
    \item l'oxygène   \oxygene (\qty{65}{\percent} en masse),
    \item le carbone  \carbone (\qty{18}{\percent}),
    \item l'hydrogène \hydrogene (\qty{10}{\percent})
    \item et l'azote  \chemfig{N} (\qty{3}{\percent}).
  \end{listePoints}
  \end{multicols}
  
  \begin{importants}
    \important{Numéro atomique :} il correspond au nombre de protons d'un atome et est noté $Z$ : \isotope{A}{Z}{X} (\hspace{-8pt}\exemple \isotope{12}{6}{C})
    Par neutralité de l'atome, c'est aussi son nombre d'électrons.
  \end{importants}
\end{doc}

%%
\begin{doc}{Liaison moléculaire}{doc:A1_liaison_molecule}
  %
  À partir du numéro atomique d'un atome, on peut déterminer sa structure électronique en couche (1, 2 ou 3) et sous-couche (s ou p), puis sa \important{valence} (mono, bi, tri ou tétravalent).
  %
  \begin{importants}
    Pour former des molécules, les atomes partagent les électrons de leur couche externe pour former des \important{liaison covalentes.}
    Chaque liaison covalente apporte 1 électron à l'atome.
    La \important{valence} est le nombre de liaisons formées par l'atome.
  \end{importants}
  %
  \begin{importants}
    La couche 1 contient au maximum \important{2 électrons} et les couches 2 et 3 contiennent jusqu'à \important{8 électrons.}

    Les atomes cherchent à remplir leur couche externe : c'est la règle du \important{duet} (couche 1) ou de \important{l'octet} (couche 2 ou 3).
  \end{importants}
  %
  Pour connaître la valence d'un atome, il suffit donc de compter combien d'électrons il lui manque pour remplir sa couche externe.

  \exemple \isotope{}{6}{C} : $1^2\; 2^4$,
  il lui manque \important{4} électrons pour compléter sa couche externe et respecter la règle de \important{l'octet.}
  Il fera donc \important{4} liaisons, il est \important{tétravalent.}
\end{doc}


%% questions
\question{%
  Indiquer la configuration électronique de l'oxygène \isotope{}{8}{O}, combien d'électrons il lui manque pour respecter la règle du duet ou de l'octet, le nombre de liaisons ainsi formées et sa valence.
}{%
  \isotope{}{8}{O} : $1^2\; 2^6$,
  il lui manque 2 électrons pour respecter la règle de l'octet, il formera donc 2 liaisons. Il est bivalent.
}{1}

%
\newpage
\vspace*{-30pt}
\question{%
  Même question pour l'azote \isotope{}{7}{N} et l'hydrogène \isotope{}{1}{H} .
}{%
  Il manque 3 électrons à l'azote pour respecter la règle de l'octet, l'azote formera donc 3 liaisons. Il est trivalent. \\
  Il manque 1 électron à l'hydrogène pour respecter la règle du duet, l'hydrogène formera donc 1 liaison. Il est monovalent.
}{4}


%%
\begin{doc}{Liaisons multiples}{doc:A1_liaisons_multiples}
  %
  \begin{importants}
    Pour compléter leur couche externe et respecter la règle de l'octet, deux atomes peuvent se lier en formant 2 ou 3 liaisons covalentes.
    
    On dit qu'il y a une \texteTrou[0.3]{liaison double} ou une \texteTrou[0.3]{liaison triple}
  \end{importants}
\end{doc}

%
\question{
  Indiquer si les liaisons sont simples, triples ou doubles sur les molécules suivantes :
\begin{equation*}
  \chemfig{N ~N} \qq{}
  \chemfig{O =C =O} \qq{}
  \chemfig{H -C ~N}
\end{equation*}
\vspace*{1cm}
}{
  Liaison triple, liaison double et double, liaison simple et triple
}{0}


%%
\titreSection{Représentation des molécules}

%%
\vspace*{-12pt}
\titreSousSection{La formule brute}

\vspace*{-8pt}
\begin{doc}{Formule brute}{doc:A1_formule_brute}
  \begin{importants}
    Elle indique le nombre de chaque atomes présents dans la molécule.
  \end{importants}
  Elle permet de calculer facilement les \important{masses molaires} et de vérifier si deux molécules sont \important{isomères.}
  Par contre elle \important{ne permet pas} de déterminer la géométrie d'une molécule.

  \begin{importants}
    Deux molécules sont \important{isomères} si elles ont la même formule brute, mais un agencement des atomes différents.
  \end{importants}

  \exemple Le butane \chemfig{C_4 H_{10}}, l'éthanol \bruteCHO{2}{6}{} ou l'acide carbonique \bruteCHO{}{2}{3}
\end{doc}

L'oxybenzone est une molécule utilisée pour protéger des UVA et B issu du soleil.
Sa formule brute est \bruteCHO{14}{12}{3} .

\question{%
  Indiquer le nombre d'élément d'hydrogène, d'oxygène et de carbone dans la molécule d'oxybenzone.
}{%
  Il y a 12 hydrogènes, 3 oxygènes et 14 carbones.
}{1}

\vspace*{8pt}
\begin{wrapfigure}[2]{r}{0.23\linewidth}
  \centering
  \vspace*{-22pt}
  \image{0.8}{images/organique/taurine.png}
\end{wrapfigure}

La taurine est un acide aminé produit naturellement dans le corps humain.
Sa représentation avec un modèle moléculaire est présentée ci-contre.

\question{%
  Donner la formule brute de la taurine.
}{%
  \chemfig{C_2 H_7 O_3 N}
}{1}


%%
\newpage
\vspace*{-40pt}
\titreSousSection{La formule développée}

\vspace*{-8pt}
\begin{doc}{Formule développée}{doc:A1_formule_developpee}
  \begin{importants}  
    Elle représente tous les éléments chimiques et toutes les liaisons dans le même plan.
  \end{importants}

  \exemple*
  \vspace*{-18pt}
  \begin{center}
    \chemname{\chemfig{H -C (-[3]O (-[5]H)) (-[-3]H) -C !\saturationH}}{éthanol}
    %
    \qq{}
    \chemname{\chemfig{Cl -C !\paireH -Si !\saturationH}}{chlorométhylsilane}
    %
    \qq{}
    \chemname{
      \chemfig[atom sep = 19pt]{[:-30]
        !\paracetamolDev
        % H -O -[1]C *6 (
        %   -C(-H) =C(-H) -C(-N (-[3]H) (-[-1]C (=[-3]O) (-C!\saturationH))) =C(-H) -C(-H) =
        % )
      }
    }{paracétamol}
  \end{center}
\end{doc}

Pour les molécules linéaire, on peut passer de la formule brute à la formule développée \important{en comptant les liaisons formées par chaque éléments} composant la molécule.

\question{Compléter le tableau suivant :}{}{0}

\vspace*{8pt}
\begin{tblr}{
  width = \linewidth, hlines, vlines,
  colspec = {X[0.5] X X X X},
  columns = {l}, rows = {m},
  column{1} = {couleurPrim!10, c},
  row{1} = {couleurPrim!20, c}, 
  row{4} = {c}
}
  Formule brute &
  Méthane \chemfig{CH_4} &
  Propane \chemfig{C_3 H_8} &
  Eau oxygénée \chemfig{H_2 O_2} &
  Méthanol \bruteCHO{}{4}{} \\
  %
  Nombre d'éléments &
  {Carbone : 1 \\ Hydrogène : 4} &
  {\carbone :   \texteTrou{3} \\ \hydrogene : 8} &
  {\hydrogene : \texteTrou{2} \\ \oxygene :   \texteTrou{2}} &
  {\carbone :   \texteTrou{1} \\ \hydrogene : 4 \\ \oxygene : \texteTrou{1}} \\
  %
  Nombre de liaisons & 
  {\carbone : \textbf{4 liaisons} \\ \hydrogene : \textbf{1 liaison}} &
  {\carbone : \texteTrou{4}liaisons \\ \hydrogene : \texteTrou{1} liaison} &
  {\hydrogene : \texteTrou{1} liaison  \\ \oxygene : \hspace{-2pt}\texteTrou{2} liaisons} &
  {\carbone : \texteTrou{4} liaisons \\ \hydrogene : \texteTrou{1} liaison \\ \oxygene : \hspace{-2pt}\texteTrou{2} liaisons } \\
  %
  Formule développée &
  \chemfig[atom sep = 20pt]{H - C !\saturationH} & & & \\
\end{tblr}


%%
\titreSousSection{La formule semi-développée}

\begin{doc}{Formule semi-développée}{doc:A1_formule_semi_developpee}
  \begin{importants}
    Comme la formule développée, elle représente tous les éléments chimiques, mais elle ne détaille pas les liaisons des éléments \important{hydrogènes.}
  \end{importants}

  \exemple*
  \vspace*{-8pt}
  \begin{center}
    \chemname{\chemfig{HO -CH_2 -CH_3}}{éthanol}
    %
    \qq{}
    \chemname{\chemfig{Cl -CH_2 -SiH_3}}{chlorométhylsilane}
    %
    \qq{}
    \chemname{\chemfig{!\paracetamolSemiDev}}{paracétamol}
  \end{center}
\end{doc}

\newpage
\vspace*{-20pt}
Pour passer de la formule développée à la formule semi-développée, il suffit de 
\begin{listeFleche}
  \item surligner tous les hydrogènes et leur liaison ;
  \item recopier \important{à l'identique} tous ce qui n'est pas surligné ;
  \item indiquer les hydrogènes et leur nombres à côté de l'élément auxquels ils sont liés.
\end{listeFleche}

\question{Compléter le tableau suivant :}{}{0}

\vspace*{8pt}
\begin{tblr}{
  width = \linewidth, rows = {66pt, m},
  colspec = {X[0.25] c c c c}, hlines, vlines,
  column{1} = {couleurPrim!20, c}
}
  Écriture développée &
  \chemfig{H -C !\paireH -C !\saturationH} &
  \chemfig{N (-[5] H) (-[-5] H) - N !\paireSatH} &
  \chemfig{C (-[5] H) (-[-5] H) = C (-[1] O-H) (-[-1] H)} &
  \chemfig{H - C!\paireH -C (=[1] O) (-[-1] O-H)} \\
  %
  Écriture semi-développée &
  \chemfig{H_3C - CH_3} & \vAligne{50pt} & & \\
  %
\end{tblr}

%%
\titreSousSection{La formule topologique}

\begin{doc}{Formule topologique}{doc:A1_formule_topologique}
  \begin{importants}  
    Elle représente les liaisons \important{carbone-carbone \chemfig{C - C}} par des segments formant des angles.
    Chacune des extrémités d'un segment représente un carbone, sauf si un autre élément chimique y est attaché.
    Les éléments \important{carbones} et les \important{hydrogènes} qui sont attachés aux carbones \important{ne sont pas représentés.}
    Tous les autres éléments chimiques sont représentés normalement.
  \end{importants}

  \exemple*
  \vspace*{-10pt}
  
  \begin{center}
    \chemname{\chemfig{HO-[1]-[-1]}}{éthanol}
    %
    \qq{}
    \chemname{\chemfig{Cl -[1] -[-1]SiH_3}}{chlorométhylsilane}
    %
    \qq{}
    \chemname{\chemfig{!\paracetamol}}{paracétamol}
  \end{center}
  %
  \begin{center}
    \chemname{\chemfig{!\acideAcetylsalicylique}}{Acide acétylsalicylique}
    %
    \qq{}
    \chemname{\chemfig{!\glucose}}{glucose}
    %
    \qq{}
    \chemname{\chemfig{!\fructose}}{fructose}
  \end{center}
\end{doc}