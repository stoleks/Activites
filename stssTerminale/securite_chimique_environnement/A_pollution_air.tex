%%%%
\teteTermStssEnvi
\titreActivite{Pollution de l'air}

\begin{objectifs}
  \item Connaître les principales sources de pollution de l'air : voiture thermique et électrique, agriculture conventionnelle, chauffage et industrie thermique, produits artificiels contenant des COV
\end{objectifs}

\begin{contexte}
  Certaines activités humaines sont source de pollution pour l'air.
  Elles émettent des \important{polluants nocifs} pour la vie ou qui sont des \important{gaz à effet de serre}, responsable du \important{dérèglement climatique.}
  
  \problematique{ 
    Quelles sont les activités humaines sources de pollution ?
  }
\end{contexte}


%%%% Pollution & polluant
\begin{doc}{Les polluants de l'air et leurs conséquences sur la santé}{doc:A4_polluant_air}
  La pollution de l'air à des conséquences très négatives sur la santé et l'environnement.
  En France, la pollution de l'air représente
  \begin{listePoints}
    \item $\sim \num{40000}$ décès par an.
    \item $\sim 100$ milliards d'euros de dépense de santé par an 
  \end{listePoints}
  À titre de comparaison, le budget de toute l'éducation nationale représente 60 milliards d'euros par an. 
  Il y a \num{660 000} morts par an en France, donc $\num{40000}/\num{660000} = \qty{6}{\percent}$ des morts en France sont dus à la pollution de l'air : c'est l'équivalent d'une crise covid chaque année !

  \begin{importants}
    Il existe deux catégorie de polluants
    \begin{listePoints}
      \item les \important{polluants primaires} : monoxyde de carbone \chemfig{CO}, dioxyde de soufre \chemfig{SO_2}, monoxyde d'azote \chemfig{NO}, particules fines et métaux lourds, etc.
      \item les \important{polluants secondaires} : ozone \chemfig{O_3}, dioxyde d'azote \chemfig{NO_2}, etc.
    \end{listePoints}
    On note souvent \chemfig{NO_x} les oxyde d'azote \chemfig{NO} et \chemfig{NO_2}.
  \end{importants}
  
  En France la qualité de l'air s'améliore, mais reste un enjeu majeur de société.

  \begin{center}
    \image{0.9}{images/sante/humain_organes}
  \end{center}
\end{doc}


\begin{doc}{Les activités humaines sources de polluants}{doc:A4_source_pollution}
  \begin{listePoints}
    \item les \important{transports carbonés} (voiture, deux roues et camion thermique) émettent \qty{61}{\percent} des oxydes d'azote \chemfig{NO_x}.
    %
    \item Les roues des véhicules de transports émettent des \important{particules fines,} nocives pour la santé.
    %
    \item Les \important{chauffages carbonés} (fioul, bois et charbon) et la \important{production d'électricité carbonée} (pétrole et charbon) génèrent des oxydes d'azotes \chemfig{NO_x} et des oxyde de carbone \chemfig{CO} et \chemfig{CO_2}.
    %
    \item Les \important{produits phytosanitaires} utilisé par \important{l'agriculture conventionnelle} émettent également des oxydes d'azote \chemfig{NO_x} et représentent \qty{97}{\percent} de la pollution à l'ammoniac \chemfig{NH_3} à cause des déjections animales.
    %
    \item L'air intérieur est pollué par \important{les cosmétiques, solvants, peintures...} qui émettent des « Composé Organique Volatil » COV.
    Les COV sont des molécules organiques provenant à \qty{46}{\percent} des particuliers et \qty{40}{\percent} des industries.
    %
    \item Le secteur \important{médical, pharmaceutique et l'élevage} génère des pollutions aux antibiotiques et aux hormones.
    %
    \item Enfin l'ozone est un polluant secondaire, produit par des réactions entre les \chemfig{NO_x} et les COV.
  \end{listePoints}
\end{doc}

\question{
  Rechercher les définitions des termes ischémie cérébrale, inflammation, ischémie myocardique, athérosclérose, vasoconstriction, thrombose.
  Les résumer en quelque mots.
}{}[6]

\question{
  Chercher le sens de « produit phytosanitaire » et « d'agriculture conventionnelle ».
}{}[6]

\question{
  Sur quels activités polluantes pouvez-vous agir ou être attentive au quotidien ?
}{}[6]

