%%%%
\teteTermStssEnvi

%%%% titre
%\vspace*{-32pt}
\numeroActivite{4}
\titreActivite{Pollution de l'air et dérèglement climatique}
\vspace*{-8pt}


%%%% objectifs
\begin{objectifs}
  \item Connaître les trois principaux gaz à effet de serre : \chemfig{CO_2}, \chemfig{CH_4} et \chemfig{H_2O}.
  \item Comprendre le principe de « l'effet de serre ».
  \item Connaître les principales sources de pollutions et d'émission de gaz à effet de serre.
\end{objectifs}

\begin{contexte}
  Certaines activités humaines sont source de pollution pour l'air.
  Elles émettent des \important{polluants nocifs} pour la vie ou qui sont des \important{gaz à effet de serre}, responsable du \important{dérèglement climatique.}
  
  \problematique{ 
    Quelles sont les activités sources de pollution ?
    Quels sont les gaz à effets de serre et comment réchauffent-ils l'atmosphère ?
  }
\end{contexte}


%%%% Pollution & polluant
\begin{doc}{Les polluants de l'air et leurs conséquences sur la santé}{doc:A4_polluant_air}
  La pollution de l'air à des conséquences très négatives sur la santé et l'environnement.
  En France, la pollution de l'air représente
  \begin{listePoints}
    \item $\sim \num{40000}$ décès par an.
    \item $\sim 100$ milliards d'euros de dépense de santé par an 
  \end{listePoints}
  À titre de comparaison, le budget de toute l'éducation nationale représente 59 milliards d'euros par an. 
  Il y a \num{660 000} morts par an en France, donc $\num{40000}/\num{660000} = \qty{6}{\percent}$ des morts en France sont dus à la pollution de l'air : c'est l'équivalent d'une crise covid chaque année !

  \begin{importants}
    Il existe deux catégorie de polluants
    \begin{listePoints}
      \item les \important{polluants primaires} : monoxyde de carbone \chemfig{CO}, dioxyde de soufre \chemfig{SO_2}, monoxyde d'azote \chemfig{NO}, particules fines et métaux lourds, etc.
      \item les \important{polluants secondaires} : ozone \chemfig{O_3}, dioxyde d'azote \chemfig{NO_2}, etc.
    \end{listePoints}
    On note souvent \chemfig{NO_x} les oxyde d'azote \chemfig{NO} et \chemfig{NO_2}.
  \end{importants}
  
  En France la qualité de l'air s'améliore, mais reste un enjeu majeur de société.

  \begin{center}
    \image{1}{images/donnees/humain_organes}
  \end{center}
\end{doc}


\begin{doc}{Les activités humaines sources de polluants}{doc:A4_source_pollution}
  \begin{listePoints}
    \item les \important{transports carbonés} (voiture, deux roues et camion thermique) émettent \qty{61}{\percent} des oxydes d'azote \chemfig{NO_x}.
    %
    \item Les \important{chauffages carbonés} (fioul, bois et charbon) et la \important{production d'électricité carbonée} (pétrole et charbon) génèrent des \chemfig{NO_x} et des oxyde de carbone \chemfig{CO} et \chemfig{CO_2}.
    %
    \item Les \important{produits phytosanitaires} utilisé par \important{certains types d'agricultures} émettent également des oxydes d'azote \chemfig{NO_x} et représentent \qty{97}{\percent} de la pollution à l'ammoniac \chemfig{NH_3} à cause des déjections animales.
    %
    \item L'air intérieur est pollué par \important{les cosmétiques, solvants, peintures...} qui émettent des « Composé Organique Volatil » COV.
    Les COV sont des molécules organiques provenant à \qty{46}{\percent} des particuliers et \qty{40}{\percent} des industries.
    %
    \item Le secteur \important{médical, pharmaceutique et l'élevage} génère des pollutions aux antibiotiques et aux hormones.
    %
    \item Enfin l'ozone est un polluant secondaire, produit par des réactions entre les \chemfig{NO_x} et les COV.
  \end{listePoints}
\end{doc}

\question{
  Rechercher les définitions des termes ischémie cérébrale, inflammation, ischémie myocardique, athérosclérose, vasoconstriction, thrombose.
  Les résumer en quelque mots.
}{}{6}


%%%% Effet de serre
\begin{doc}{Le trou dans la couche d'ozone}{doc:A4_trou_ozone}
  La couche d'ozone protège la vie sur Terre des rayon UV mutagène du Soleil.
  Sans la couche d'ozone, la vie serait restée dans les océans.

  Dans les année 1970, les scientifiques ont alerté la société mondiale : un trou se formait dans la couche d'ozone à cause des réfrigérateurs qui émettait des ChloroFluoroCarbures CFC, destructeur d'ozone \chemfig{O_3}.

  En 1987, les pays de l'ONU signèrent un protocole interdisant les CFC.
  30 ans plus tard, le trou s'est résorbé, mais une nouvelle molécule, le protoxyde d'azote \chemfig{N_2O}, inquiète la communauté scientifique, car elle attaque la couche d'ozone.
\end{doc}

\begin{doc}{L'augmentation de « l'effet de serre »}{doc:A4_effet_serre}
  \begin{wrapfigure}[3]{r}{0.1\linewidth}
    \vspace*{-18pt}
    \qrcode{https://youtu.be/ewc8FBtEKPs}
  \end{wrapfigure}
  La présence de Gaz à Effet de Serre noté GES rend la Terre habitable en augmentant sa température moyenne au niveau de la mer, \qty{15}{\degreeCelsius} au lieu de \qty{-18}{\degreeCelsius}.

  \begin{importants}  
    Les principaux gaz à effet de serre sont le dioxyde de carbone \chemfig{CO_2},
    le méthane \chemfig{CH_4} et l'eau \chemfig{H_2O}.
  \end{importants}

  Certaines activités humaines au cours du siècle dernier ont entraîné une augmentation de la quantité de \chemfig{CO_2} et de \chemfig{CH_4} dans l'atmosphère, ce qui accentue le déséquilibre entre l'énergie radiative reçue du Soleil et l'énergie radiative émise par la Terre.

  Ce déséquilibre entraine une augmentation des température moyenne au niveau de la mer : actuellement +\qty{1}{\degreeCelsius} et jusqu'à +\qty{4}{\degreeCelsius} en 2100.

  \begin{importants}
    L'augmentation de la température moyenne sur Terre entraine un dérèglement du climat, avec plus d'événements climatiques extrêmes : inondation, canicule, incendie, sécheresse, tempête, ouragan, etc.
  \end{importants}

  Avec +\qty{2}{\degreeCelsius} certaines partie de la Terre seront inhabitable en été à cause des température et de l'humidité trop élevée ! 
  Au delà de ce seuil de \qty{2}{\degreeCelsius}, les climatologues prévoient le pire.
\end{doc}


\begin{doc}{Les activités humaines source de gaz à effet de serre}{doc:A4_source_GES}
  \begin{wrapfigure}[3]{r}{0.1\linewidth}
    \vspace*{-26pt}
    \qrcode{https://ourworldindata.org/ghg-emissions-by-sector}
  \end{wrapfigure}
  En 2015 un accord a été signé à Paris, les pays membres de l'ONU s'étant engagé à diminuer leur émissions de GES pour rester sous la barre des \qty{2}{\degreeCelsius} d'augmentation.
  Pourtant les politiques tardent à se mettre en place, voir vont carrément dans le mauvais sens en favorisant les secteurs d'activités les plus polluants.

  \begin{center}
    \image{0.9}{images/donnees/emission_GES_secteur}

    \image{0.9}{images/donnees/emission_GES_secteur_EU}

    Émission européenne de gaz à effet de serre par secteurs. 
    
    \textit{Sources : ministère de la transition écologique.}
  \end{center}
\end{doc}

\numeroQuestion Compléter le tableau suivant

\begin{tableau}{l| X[c]| X[c]}
  Phénomène                   & Couche d'ozone & Effet de serre \\
  Gaz impliqués               & & \\
  Problème lié à la pollution & & \\
  {Secteurs responsables \\
  du problème}                & & \\
\end{tableau}

\question{
  Expliquer pourquoi il a été plus simple de régler le problème de la couche d'ozone que du réchauffement climatique.
}{}{2}

\question{
  Est-ce qu'en Europe et dans le monde la répartition des émissions de gaz à effet de serre par secteur est la même ?
}{}{2}


\begin{doc}{La part de l'alimentation dans les émissions de GES}{doc:A4_alim_GES}
  \begin{wrapfigure}{r}{0.1\linewidth}
    \qrcode{https://ourworldindata.org/food-ghg-emissions}
  \end{wrapfigure}
  Estimer les émissions secteurs par secteurs est un exercice de classification difficile.
  Si on regarde la nourriture par exemple, on pourrait uniquement compter les émissions due à l'agriculture,
  mais aussi compter les émissions liée au transport et au stockage des denrées.

  D'une source à l'autre, la répartition par secteur peuvent donc varier.
  Une estimation courante est que l'alimentation représente 1/4 des émissions mondiales.
\end{doc}

\question{
  En utilisant le lien fourni dans le document, calculer la proportion des émissions de GES qui sont liées à la production de viande et de poisson.
}{}{2}

\question{
  En déduire la part des émissions mondiale liée à la consommation de viande et de poisson.
}{}{2}