%%%%
\teteTermStssEnvi

%%%% titre
\vspace*{-36pt}
\numeroActivite{2}
\titreActivite{Potabilité et pollution des eaux}


%%%% objectifs
\begin{objectifs}
  \item Comprendre la modélisation de de la dissolution des composés ioniques.
  \item Comprendre le lien entre présence d'ions et conductivité électrique.
\end{objectifs}

\begin{contexte}
  Les activités industrielles humaines polluent les fleuves et les rivières, composés d'eau douce, mais aussi les océans et les mers, composés d'eau salée.
  Ainsi, dans les 50 dernières années, la qualité des eaux s'est considérablement dégradée.
  
  \problematique{
    Quels paramètres physico-chimiques permettent de contrôler la qualité d'une eau ?
  }
\end{contexte}


%%%% docs
\begin{doc}{Concentration en ions}{doc:A2_concentration_eau}
  Pour mesurer la qualité d'une eau, il faut pouvoir décrire sa composition en ions.
  Pour ça, on va utiliser la \important{concentration ionique en masse}
  et la \important{concentration ionique en quantité de matière}.

  \begin{importants}  
    La \important{concentration ionique en masse} (« massique ») d'un ion en solution
    %
    \begin{center}
      \begin{tblr}{X[c] | X[l]}
        $c_{m,X} = \dfrac{m_X}{V}$ &
        { $m_X$ est la masse d'ion en \unit{\g} \\
        $V$ est le volume de la solution en \unit{\litre} }
      \end{tblr}
    \end{center}
    L'unité de $c_{m,X}$ est \texteTrou[0.1]{\unit{\g\per\litre}}
  \end{importants}
  %
  \begin{importants}  
    La \important{concentration ionique en quantité de matière} (« molaire ») d'un ion en solution
    %
    \begin{center}
      \begin{tblr}{X[c] | X[l]}
        $[X] = \dfrac{n_X}{V}$ &
        { $n_X$ est la quantité de matière d'ion en \unit{\mole} \\
        $V$ est le volume de la solution en \unit{\litre} }
      \end{tblr}
    \end{center}
    L'unité de $[X]$ est \texteTrou[0.1]{\unit{\mole\per\litre}}
  \end{importants}
\end{doc}

\numeroQuestion Indiquer les unités de $c_{m,X}$ et de $[X]$ dans le document~\ref{doc:A2_concentration_eau}.

\question{
  Donner la relation littérale qui permet de calculer la masse d'ion en fonction de la concentration ionique massique et du volume d'eau.
}{}{2}


%%
\begin{doc}{Potabilité d'une eau}{doc:A2_potabilite_eau}
  \begin{wrapfigure}[3]{r}{0.3\linewidth}
    \vspace*{-24pt}
    \begin{boite}
      \pointCyan \important{Organoleptique :} \\ qui affecte les sens.
    \end{boite}
  \end{wrapfigure}

  Pour qu'une eau soit désignée comme potable, elle doit répondre à plusieurs critères.
  Ces critères sont établis par l'Organisation Mondiale de la Santé, l'OMS.

  \begin{listePoints}
    \item \important{Qualité microbiologique :} l'eau est consommable s'il n'y a pas de germes témoins de matières fécales, de parasites et de micro-organismes pathogènes.
    Il n'y a pas de seuil de tolérance pour la qualité bactériologique de l'eau : c'est une \important{limite de potabilité.}
    \item \important{Qualités organoleptiques :} l'eau ne doit pas avoir d'odeur, ni de saveur particulière. L'eau doit être limpide.
    \item \important{Critères physico-chimiques} que l'eau doit respecter
  \end{listePoints}

  \centering
  \begin{tblr}{
    colspec = {|l |l |}, hlines,
    row{1} = { couleurPrim!20, m, c },
  }
    Paramètres & Limite de potabilité \\
    \ionChlorure & \qty{200}{\mg\per\litre} \\
    \ionSodium   & \qty{200}{\mg\per\litre} \\
    \ionSulfate  & \qty{250}{\mg\per\litre} \\
    pH           & Entre \num{6,5} et \num{9} \\
    Température  & < \qty{25}{\degreeCelsius} \\
    Conductivité & Entre \num{200} et 
    \qty{1100}{\micro\siemens\per\cm} à \qty{25}{\degreeCelsius} \\
  \end{tblr}
  
  \faArrowUp\; Exemple de critères physico-chimiques de potabilité d'une eau.
\end{doc}

\question{
  Lister les critères physico-chimiques que doit respecter une eau potable.
}{
}{5}


%%
\begin{doc}{Etiquettes d'eau minérale}{doc:A2_eau_minerale}
  \begin{multicols}{3}
    \centering
    \textbf{Vichy St Yorre}
     \begin{tableau}{l | r}
      \SetCell[c=2]{c} Minéralisation : \unit{\mg} pour \qty{1}{\litre} \\
      \ionBicarbonate & \num{4368} \\
      \ionChlorure    & \num{322}  \\
      \ionSodium      & \num{1708} \\
      \ionSulfate     & \num{174}  \\
      \ionPotassium   & \num{110}  \\
      \ionCalcium     & \num{90}   \\
      \ionFluorure    & \num{1}    \\
      \ionMagnesium   & \num{11}   \\
    \end{tableau}
    %
    
    \textbf{Mont Roucous}
    \begin{tableau}{l | r}
      \SetCell[c=2]{c} Minéralisation : \unit{\mg} pour \qty{1}{\litre} \\
      \ionBicarbonate & \num{1} \\
      \ionChlorure    & \num{2}  \\
      \ionSodium      & \num{3,2}  \\
      \ionSulfate     & \num{6,9}  \\
      \ionFluorure    & < \num{0,1}  \\
      \ionCalcium     & \num{2,7}  \\
      \ionNitrate     & \num{1,8}  \\
      \ionMagnesium   & \num{0,3}  \\
    \end{tableau}
    %
    
    \textbf{Cristalline}
    \begin{tableau}{l | r}
      \SetCell[c=2]{c} Minéralisation : \unit{\mg} pour \qty{1}{\litre} \\
      \ionBicarbonate & \num{228} \\
      \ionChlorure    & \num{15}    \\
      \ionSodium      & \num{8,4}  \\
      \ionSulfate     & \num{11}  \\
      \ionPotassium   & \num{2,3}     \\
      \ionCalcium     & \num{549}   \\
      \ionNitrate     & < \num{1}   \\
      \ionMagnesium   & \num{6,9}   \\
    \end{tableau}
  \end{multicols}
\end{doc}

\question{
  Indiquer quelle eau du document~\ref{doc:A2_eau_minerale} est potable selon les critères de l'OMS.
}{}{4}


%%%%
\begin{doc}{Impact des activités humaines sur la qualité chimique de l'eau}{doc:A2_impact_humain}
  Depuis les années 1950, la qualité des eaux s'est fortement dégradée.
  Cette dégradation met en péril les écosystème en milieux aquatiques et marins.

  Dans les océans, on constate :
  \begin{listePoints}
    \item une diminution du pH des océans due à l'absorption de dioxyde de carbone émis par les activités humaines ;
    \item une salinité (concentration ionique en sodium et chlorure) accrue ;
    \item une augmentation de la température des eaux ;
    \item une augmentation de la présence de matières plastiques et d'hydrocarbures.
  \end{listePoints}

  \begin{wrapfigure}[3]{r}{0.1\linewidth}
    \vspace*{-35pt}
    \qrcode{https://www.eaufrance.fr/la-qualite-des-rivieres}
  \end{wrapfigure}
  Dans les rivières, leur pollution dépend de leur localisation. 
  En fonction des industries et des villes à proximité, leur qualité est plus ou moins altérée. 
  Le site eauFrance permet de suivre la qualité des rivière en France (scanner le QR code pour y accéder).

  \begin{tblr}{
    colspec = {|X[l] |X[l] |X[l] |}, hlines,
    cell{1}{1} = {r=2}{c}, cell{1}{2} = {c=2}{c},
    row{1,2} = {couleurPrim!20, c}
  }
    \textbf{Macropolluants} & \textbf{Micropolluant} & \\
    & \textbf{Organiques} & \textbf{Inorganiques} \\
    Matières en suspension, nitrates, phosphates, \ldots &
    Hydrocarbures, plastifiant, pesticides, détergents, médicaments, \ldots &
    Métaux et autre éléments lourds : \chemfig{Pb}, \chemfig{Hg}, \chemfig{Cd}, \chemfig{Cu}, \chemfig{Fe}, \chemfig{Zn}, \chemfig{Co}, \ldots
  \end{tblr}

  \faArrowUp\; Types de polluants qu'on retrouve dans les eaux.
\end{doc}

\question{
  Aller sur le site eauFrance et indiquer quelle est la proportion des cours d'eau qui sont dans un bon état chimique en 2015. 
  Cette proportion a-t-elle augmentée par rapport à 2010 ?
}{}{3}

\question{
  Quels sont les zones de la France ou les cours d'eau sont dans le moins bon état chimique ?
  Et le moins bon état biologique ?
}{}{3}

\question{
  En cherchant sur le site eauFrance, expliquer ce qu'est le phénomène \important{d'eutrophisation des milieux.}
}{}{4}
