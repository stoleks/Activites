%%%%
\teteTermStssEnvi
\titreActivite{Loi des gaz parfait et stockage des gaz}
\vspace*{-8pt}


%%%% objectifs
\begin{objectifs}
  \item Connaître la loi des gaz parfait.
  \item Savoir utiliser la loi des gaz parfaits.
\end{objectifs}

\begin{contexte}
  Les gaz ont des propriétés physiques macroscopiques qui permettent de les décrire : pression $P$, volume $V$, température $T$, quantité de matière $n$.
  Ces quatre grandeurs sont reliés par la \important{loi des gaz parfait.}
  
  \problematique{ 
    Quelles sont les conséquences de la loi des gaz parfait ?
  }
\end{contexte}


%%%%
\begin{doc}{La loi des gaz parfaits}{doc:A3_gaz_parfait}
  Un gaz parfait est un modèle des gaz où l'on représente les molécules par des particules ponctuelles et les molécules n'interagissent entre elles qu'au cours de collisions, comme un grand jeu de billard.
  \important{Ce modèle est valide pour tous les gaz qui nous entourent au quotidien.}
  
  \begin{importants}
    La \important{loi des gaz parfait} est :
    \begin{equation*}
      PV = nRT
    \end{equation*}
    où $R$ est la \important{constante des gaz parfaits} et vaut $R = \qty{8,314}{\pascal \m\cubed \per\mole \per\kelvin}$ ;
    \begin{listePoints}
      \item $P$ est la pression du gaz en pascal \unit{\pascal} ;
      \item $V$ est le volume occupé par le gaz en \unit{\m\cubed} ;
      \item $n$ est le nombre de molécules dans le gaz en \unit{\mole}.
      \item $T$ est la température du gaz en kelvin \unit{\kelvin} ;
    \end{listePoints}
  \end{importants}

  Il y a trois règles de conversion à connaître :
  \begin{listePoints}
    \item \qty{1}{\litre} = \qty{e-3}{\m\cubed} ;
    \item \qty{1}{\degreeCelsius} = (1 + \num{273,15}) \unit{\kelvin}
    \item \qty{1}{\bar} = \qty{e5}{\pascal}
  \end{listePoints}

  Sauf précision contraire, en général on travaille avec une
  \begin{itemize}
    \item température ambiante $T_\text{amb} = \qty{20}{\degreeCelsius} 
      = (20 + \num{273,15}) \unit{\kelvin}
      = \qty{293,15}{\kelvin}$ ;
    \item pression atmosphérique $P_\text{atm} = \qty{1,013e5}{\pascal} = \qty{1013}{\hecto\pascal}$.
  \end{itemize}
  
  
\end{doc}


\begin{doc}{Oxygénothérapie}{doc:A3_oxygenotherapie}
  L'oxygénothérapie consiste à administrer du dioxygène à un-e patient-e, pour maintenir ou rétablir une concentration normale en oxygène dans le sang.

  Le dioxygène est conservé dans des bouteilles de $V_b = \qty{2,0}{\litre}$ à $P_b = \qty{200}{\bar}$, dans des locaux à température ambiante $T_\text{amb}$.
  Pendant la thérapie, le dioxygène s'échappe de la bouteille et retrouve une pression atmosphérique $P_\text{atm}$.
\end{doc}

\question{
  Convertir la pression régnant dans la bouteille en pascal.
}{}[2]

\newpage
\question{
  En utilisant la loi des gaz parfait, calculer la quantité de matière de dioxygène contenue dans la bouteille.
}{}[5]

\medskip
On laisse le gaz sortir lentement de la bouteille.
Dans ces conditions la température $T_\text{amb}$ reste constante.

\question{
  Calculer le volume de gaz libérable $V_l$ dans cette bouteille.
}{}[5]

\medskip
Une patiente nécessite \qty{1,0}{\litre\per\minute} de dioxygène en permanence.
Elle souhaite se promener avec son fauteuil roulant pendant \qty{2}{\hour} et possède plusieurs bouteilles de \qty{2,0}{\litre} à \qty{200}{\bar}.

\question{
  Calculer le nombre de bouteilles dont elle aura besoin pour sa balade.
}{}[6]
