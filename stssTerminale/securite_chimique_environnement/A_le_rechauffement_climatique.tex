\teteTermStssEnvi
\titreActivite{Le réchauffement climatique}

\begin{objectifs}
  \item Comprendre le principe de « l'effet de serre ».
  \item Connaître les principales sources de pollutions et d'émission de gaz à effet de serre.
  \item Connaître les trois principaux gaz à effet de serre : \dioxydeDeCarbone, \methane et \eau.
\end{objectifs}

\begin{contexte}
  Certaines activités humaines sont sources de \important{gaz à effet de serre}, responsable du \important{dérèglement climatique.}
  
  \problematique{
    Quels sont les gaz à effets de serre et comment réchauffent-ils l'atmosphère ?
    Quels activités humaines émettent des gaz à effet de serre ?
  }
\end{contexte}

%%%% Effet de serre
\begin{doc}{Le trou dans la couche d'ozone}{doc:A4_trou_ozone}
  La couche d'ozone protège la vie sur Terre des rayon UV mutagène du Soleil.
  Sans la couche d'ozone, la vie serait restée dans les océans.

  Dans les année 1970, les scientifiques ont alerté la société mondiale : un trou se formait dans la couche d'ozone à cause des réfrigérateurs qui émettait des ChloroFluoroCarbures CFC, destructeur d'ozone \chemfig{O_3}.

  En 1987, les pays de l'ONU signèrent un protocole interdisant les CFC.
  30 ans plus tard, le trou s'est résorbé, mais une nouvelle molécule, le protoxyde d'azote \chemfig{N_2O}, inquiète la communauté scientifique, car elle réagit avec la couche d'ozone.
  
  Depuis 2020 un autre phénomène attaque la couche d'ozone : les rentrées atmosphériques des satellites starlink.
  Ces satellites tombent sur Terre et libèrent des métaux lourd dans l’atmosphère en la traversant. Ces métaux réagissent fortement avec l'ozone et risque de récréer un trou.
\end{doc}

\question{
  Expliquer pourquoi régler le problème de la couche d'ozone a été relativement simple.
}{}[2]


\begin{doc}{L'augmentation de « l'effet de serre »}{doc:A4_effet_serre}
  \qrcodeCote[3]{https://www.youtube.com/watch?v=R6eywXdssMw}
  
  La présence de Gaz à Effet de Serre noté GES rend la Terre habitable en augmentant sa température moyenne au niveau de la mer, \qty{15}{\degreeCelsius} au lieu de \qty{-18}{\degreeCelsius}.

  \begin{importants}  
    Les principaux gaz à effet de serre sont le dioxyde de carbone \dioxydeDeCarbone,
    le méthane \methane et l'eau \eau.
  \end{importants}

  Certaines activités humaines au cours du siècle dernier ont entraîné une augmentation de la quantité de \dioxydeDeCarbone et de \methane dans l'atmosphère, ce qui accentue le déséquilibre entre l'énergie radiative reçue du Soleil et l'énergie radiative émise par la Terre.

  Ce déséquilibre entraine une augmentation des température moyenne au niveau de la mer : actuellement +\qty{1}{\degreeCelsius} et jusqu'à +\qty{4}{\degreeCelsius} en 2100.

  \begin{importants}
    L'augmentation de la température moyenne sur Terre entraine un dérèglement du climat, lié à l'intensification du cycle de l'eau, avec plus d'événements climatiques extrêmes : \important{inondation, canicule, incendie, sécheresse, tempête, ouragan, etc.}
  \end{importants}
  \begin{importants}
    Mais elle a aussi des conséquences sur la \important{montée des eaux} et sur la \important{fonte des neiges éternelles} (banquises, glaciers), qui sont d'importantes réservoirs d'eau douce autour du globe.
  \end{importants}

  Avec +\qty{2}{\degreeCelsius} certaines parties de la Terre seront inhabitables en été à cause des température et de l'humidité trop élevée ! 
  Au delà de ce seuil de \qty{2}{\degreeCelsius}, les climatologues prévoient le pire.
\end{doc}


\begin{doc}{Les activités humaines source de gaz à effet de serre}{doc:A4_source_GES}
  \qrcodeCote{https://ourworldindata.org/ghg-emissions-by-sector}
  
  En 2015 un accord a été signé à Paris, les pays membres de l'ONU s'étant engagé à diminuer leur émissions de GES pour rester sous la barre des \qty{2}{\degreeCelsius} d'augmentation.
  Pourtant les politiques tardent à se mettre en place, voir vont carrément dans le mauvais sens en favorisant les secteurs d'activités les plus polluants.

  \begin{center}
    % \image{0.9}{images/donnees/emission_GES_secteur}

    \image{0.75}{images/donnees/emission_GES_secteur_EU}

    \legende{Émission européenne de gaz à effet de serre par secteurs.}
    
    \textit{Sources : ministère de la transition écologique.}
  \end{center}
\end{doc}

\question{
  En utilisant le lien fourni dans le document~\ref{doc:A4_source_GES}, indiquer si en Europe et dans le monde la répartition des émissions de gaz à effet de serre par secteur est la même.
}{}[2]

\question{
  En utilisant le lien fourni dans le document~\ref{doc:A4_source_GES}, indiquer la part des émissions qui sont dues à l'utilisation de voitures individuelles.
}{}[2]


\begin{doc}{La part de l'alimentation dans les émissions de GES}{doc:A4_alim_GES}
  \qrcodeCote{https://ourworldindata.org/food-ghg-emissions}
  
  Estimer les émissions secteurs par secteurs est un exercice de classification difficile.
  Si on regarde la nourriture par exemple, on pourrait uniquement compter les émissions due à l'agriculture,
  mais aussi compter les émissions liée au transport et au stockage des denrées.

  D'une source à l'autre, la répartition par secteur peuvent donc varier.
  Une estimation courante est que l'alimentation représente 1/4 (\qty{25}{\percent}) des émissions mondiales.
\end{doc}

\question{
  En utilisant le lien fourni dans le document~\ref{doc:A4_alim_GES}, calculer la proportion des émissions de gaz à effet de serre qui sont liées à la production de viande et de poisson.
}{}[3]