\teteTermStssEnvi
\vspace*{-36pt}
\titreActivite{Propriétés de l'eau}

\begin{objectifs}
  \item Comprendre la modélisation de de la dissolution des composés ioniques.
  \item Comprendre le lien entre présence d'ions et conductivité électrique.
\end{objectifs}

\begin{contexte}
  L'eau sous forme liquide est un très bon solvant pour les entités chimiques polaires ou ioniques.
  Les solutions aqueuses avec des espèces chimiques ioniques sont de bonnes conductrices électriques.
  
  \problematique{
    Comment expliquer les propriétés des solutions aqueuses ?
  }
\end{contexte}


%%%% docs
\separationBlocs{
  \begin{doc}{La molécule H$_2$O}
    \begin{center}
      \image{0.5}{images/molecules/eau_polarite}
    \end{center}
    \vspace*{-8pt}
    
    L'eau est une molécule \important{polaire.} 
    C'est comme si elle était composé de charges séparées $\delta^-$ et $\delta^+$ ($\delta$ est un nombre compris entre $0$ et $1$).
  \end{doc}
}[0.39]{
  \begin{doc}{Un peu de vocabulaire}{doc:vocabulaire}
    \begin{importants}
      \pointCyan \important{Liaison ionique :}
      liaison entre un cation (positif) et un anion (négatif).
      
      \pointCyan \important{Solvatation :}
      dissolution d'une espèce ionique dans le solvant.
      
      \pointCyan \important{Solubilité :}
      masse maximale d'une espèce chimique que l'on peut dissoudre dans un liquide, exprimée en \unit{\g\per\litre}.
    \end{importants}
  \end{doc}
}[0.6]

%%
\begin{doc}{Solubilité des espèces ioniques dans l'eau}
  \qrcodeCote{https://edurl.fr/BHUqx4FD}
  %https://www.youtube.com/watch?v=aKGJm6OGJNs}
  
  Une espèce ionique et composé d'un \important{cation} et d'un \important{anion}, relié par une \important{liaison ionique.}
  En contact avec de l'eau liquide, l'espèce ionique se sépare en deux espèces chimiques.
  Les anions et les cations sont entourés par des molécules d'eau à cause de leur polarité, les charges $+$ sont attirées par les charges $-$.
  
  Cette modélisation s'appelle la \important{solvatation.}
  Ce modèle permet d'expliquer la \important{solubilité} de certaines espèces ioniques dans l'eau.

  \begin{center}
    \image{0.23}{images/chimie/solvatation_chlore}~
    \image{0.23}{images/chimie/solvatation_sodium}

    \legende{
      Interaction entre les molécules d'eau et les ions d'une espèce ionique, le sel \chemfig{Na^+ Cl^{-}}.
    }
  \end{center}
\end{doc}

\question{
  Expliquer avec vos mots la solubilité des espèces ioniques dans l'eau.
}{

}[4]

%%
\begin{doc}{Lien entre conductivité et ions dissous}[\label{doc:conductivite_ions}]
  Les solutions aqueuses avec des espèces ioniques sont de \important{bons conducteurs électriques.}

  \begin{importants}    
    La \important{conductivité électrique $\mathbf{\sigma}$} (sigma) se mesure avec un \important{conductimètre}.
    Son unité est le \important{siemens par mètre \unit{\siemens\per\metre}.}
  \end{importants}

  La conductivité électrique dépend de la composition de la solution aqueuse.
  \vspace*{2pt}

  \centering
  \begin{tblr}{
    colspec = {l c c c c c c c}, hlines, vlines,
    column{1} = { couleurSec-50 },
    row{1} = { couleurSec-100, m, c },
  }
    Concentration en ion (\unit{\mg/\litre}) &
    \chemfig{Ca^{+}}    &
    \sulfate & 
    \ionMagnesium   &
    \bicarbonate &
    \chemfig{K^{+}}     &
    \chlorure    &
    {Conductivité $\sigma$ \\ à \qty{25}{\degreeCelsius}} \\
    %
    Eau distillée                & 0 & 0 & 0 & 0 & 0 & 0        & \important{0} \\
    Eau 1                        & 202 & 306 & 36 & 402 & 0 & 0 & \important{\num{0,1567}} \\
    Eau 2                        & 78 & 10 & 24 & 357 & 1 & 4,5 & \important{\num{0,0640}} \\
    Eau saturée en \chemfig{KCl} & 0 & 0 & 0 & 0 & 391 & 355    & \important{\num{0,1502}} \\
  \end{tblr}
\end{doc}


%%%%

\question{
  Donner le nom de la grandeur qui permet d'évaluer si une solution conduit bien l'électricité. 
  Donner aussi l'unité et l'appareil qui permet de mesurer cette grandeur.
}{}[2]

\question{
  Expliquer la conductivité des 4 solutions présentées dans le tableau du document~\ref{doc:conductivite_ions}.
}{}[5]


%%
\begin{doc}{Eau déminéralisée ou distillée}
  \centering
  \begin{tblr}{
    colspec = {l c c}, hlines, vlines,
    column{1} = { couleurSec-50 },
    row{1} = { couleurSec-100 },
  }
                 & Eau déminéralisée & Eau distillée \\
    Description  & Eau sans ions     & Eau pure avec quelques gaz dissous \\
    Utilisation  & Chimie, ménage    & Chimie, médical \\
    Conductivité & faible            & faible \\
    Micro-organismes (dont bactéries) & Présents & Absents \\
  \end{tblr}
\end{doc}

\question{
  Expliquer la différences entre une eau déminéralisée et une eau distillée.
}{}[2]

\question{
  Expliquer pourquoi on n'utilise pas une eau déminéralisée dans le domaine médical.
}{}[2]