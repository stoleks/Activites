\teteTermStssEnvi
\titreTP{Composition de l'air}
\vspace*{-8pt}

\begin{objectifs}
  \item Connaître la composition de l'air.
  \item Connaître quelques tests d'identification de gaz présents dans l'air.
  \item Savoir calculer une fraction molaire.
\end{objectifs}

\begin{contexte}
  L'air qui nous entoure et qui nous permet de respirer est un mélange composé de plusieurs molécules.
  
  \problematique{
    Quelle est la composition de l'air et comment la mesurer ?
  }
\end{contexte}

%%%%
\begin{doc}{Composition de l'air}[\label{doc:composition_air}]
  \begin{center}
    \begin{tblr}{
        colspec = {|l l| r|}, hlines, cell{10}{3} = {c},
        row{1} = {l, couleurSec-100, font = \sffamily\bfseries},
      }
      Constituant & & Proportion molaire \\
      Diazote            & \diazote          & \qty{78,08}{\percent} \barrePourcentage{0.7808} \\
      Dioxygène          & \dioxygene        & \qty{20,95}{\percent} \barrePourcentage{0.2095} \\
      Argon              & \chemfig{Ar}      & \qty{0,93}{\percent} \barrePourcentage{0.01} \\
      Dioxyde de carbone & \dioxydeDeCarbone & \qty{0,04}{\percent} \barrePourcentage{0} \\
      Néon               & \chemfig{Ne}      & \qty{18,2}{\ppm} \barrePourcentage{0} \\
      Hélium             & \chemfig{He}      & \qty{5,2}{\ppm}  \barrePourcentage{0} \\
      Monoxyde d'azote   & \chemfig{NO}      & \qty{5,0}{\ppm}  \barrePourcentage{0} \\
      Méthane            & \methane          & \qty{1,9}{\ppm}  \barrePourcentage{0} \\
      Eau                & \eau              & Variable \\ 
    \end{tblr}
  \end{center}

  \vspace*{-80pt}
  \begin{center}
    \begin{tikzpicture}
      % Barres
      \fill[color = couleurSec]     (0.0 , 0.0)        rectangle (0.7808*90ex, 3ex);
      \fill[color = couleurSec-300] (0.7808*90ex, 0.0) rectangle (0.9903*90ex, 3ex);
      \fill[color = couleurSec-100] (0.9903*90ex, 0.0) rectangle        (90ex, 3ex);
      % Valeurs
      \node at (0.7808*45ex, 1.5ex) {\important[white]{\qty{78,08}{\percent}}};
      \node at (0.8855*90ex, 1.5ex) {\important[black]{\qty{20,95}{\percent}}};
      \node at        (91ex, 1.5ex) {\important[black]{\qty{0,97} {\percent}}};
      % Carré légende
      \fill[color = couleurSec]     (75ex, 15ex) rectangle (78ex, 12ex);
      \fill[color = couleurSec-300] (75ex, 11ex) rectangle (78ex, 8ex);
      \fill[color = couleurSec-100] (75ex, 7ex)  rectangle (78ex, 4ex);
      % Légende
      \node[right] at (78ex, 13.5ex) {\important[black]{Diazote}};
      \node[right] at (78ex, 9.5ex)  {\important[black]{Dioxygène}};
      \node[right] at (78ex, 5.5ex)  {\important[black]{Autres gaz}};
    \end{tikzpicture}
  \end{center}

  \qty{1}{\percent} signifie qu'il y a 1 molécule sur un total de 100 molécules.
  
  \qty{1}{\ppm} signifie qu'il y a 1 molécule sur un total de \num{1000000} de molécules.
\end{doc}


\begin{doc}{Fraction molaire}
  La \important{fraction molaire} est le rapport entre la quantité de matière du constituant considéré et la quantité de matière totale dans le mélange étudié.

  La fraction molaire est noté $x_i$ pour le constituant $i$. Elle varie entre 0 et 1 et se calcule avec la relation :
  \begin{equation*}
    x_i = \dfrac{n_i}{n_\text{tot}}
  \end{equation*}

  $n_i$ est la quantité de matière du constituant $i$.

  $n_\text{tot}$ est la quantité de matière totale dans le mélange.
\end{doc}

\question{
  Arrondir les proportion des 4 premiers éléments, puis les ramener à des fractions entières les plus simple possible (exemple : \qty{78,08}{\percent} $\simeq$ \num{80}/\num{100} = \num{4}/\num{5}).
}{}[4]


\begin{doc}{Quelques tests pour identifier des espèces chimiques}
  \pointCyan L'eau de chaux est une solution saturée en hydroxyde de calcium \chemfig{Ca}(\chemfig{OH})$_2$.
  En présence de dioxyde de carbone \dioxydeDeCarbone, l'eau de chaux se trouble. C'est dû à la formation d'un précipité blanc de carbonate de calcium \chemfig{CaCO_3}.

  \pointCyan Le sulfate de cuivre anhydre \chemfig{CuSO_4} est une poudre blanche.
  En contact avec des molécules d'eau \eau la poudre bleuit. C'est dû à la formation d'un complexe pentahydrate \chemfig{CuSO_4, 5H_2O}.

  \pointCyan La combustion d'une allumette nécessite un combustible, la cellulose de formule brute \bruteCHO{6}{10}{5} du bois de l'allumette, et un comburant, le dioxygène \dioxygene.
  Cette réaction chimique forme du dioxyde de carbone \dioxydeDeCarbone et de la vapeur d'eau \eau.
\end{doc}

\question{
  Pour chacun des 3 tests, établir l'équation de la réaction chimique mise en jeu.
}{}[3]


%%%%
\begin{doc}{Combustion d'une bougie}[\label{doc:combustion_bougie}]
  Le combustible d'une bougie est l'acide stéarique de formule brute \bruteCHO{18}{36}{2}
  L'équation de la réaction de combustion d'une bougie est
  \begin{center}
    \bruteCHO{18}{36}{2}\sol + 26\dioxygene\gaz \reaction
    18\dioxydeDeCarbone\gaz + 18\eau\gaz
  \end{center}

  \important{Matériel :} un cristallisoir, une bougie, une éprouvette graduée.

  \important{Protocole :} remplir le cristallisoir d'eau.
  Placer et allumer la bougie au centre du cristallisoir.
  Recouvrir la bougie avec l'éprouvette.

  Le dioxyde de carbone se dissout dans l'eau dès sa formation et la vapeur d'eau se condense rapidement, ce qui laisse un vide dans le récipient où la combustion a lieue.
\end{doc}

\begin{doc}{Volume molaire des gaz}
  Le volume molaire des gaz vaut $V_m = \qty{24,1}{\litre\per\mole}$ à \qty{20}{\degreeCelsius} sous pression atmosphérique.
  Cette valeur est la même pour tous les gaz, donc \important{la fraction volumique est égale à la fraction molaire pour les gaz.}
\end{doc}

\mesure Réaliser l'expérience du document~\ref{doc:combustion_bougie} et mesurer le volume d'eau déplacé.

\question{
  En déduire la fraction volumique de dioxygène, puis la fraction molaire de dioxygène.
}{}[4]

\question{
  Comparer cette valeur avec celle fournie dans le document~\ref{doc:composition_air}.
}{}[1]