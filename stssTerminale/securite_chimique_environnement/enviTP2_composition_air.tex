%%%%
\teteTermStssEnvi

%%%% titre
%\vspace*{-32pt}
\numeroActivite{2}
\titreTP{Composition de l'air}
\vspace*{-8pt}


%%%% objectifs
\begin{objectifs}
  \item Connaître la composition de l'air.
  \item Connaître quelques tests d'identification de gaz présents dans l'air.
  \item Savoir calculer une fraction molaire.
\end{objectifs}

\begin{contexte}
  L'air qui nous entoure et qui nous permet de respirer est un mélange composé de plusieurs molécules.
  
  \problematique{
    Quelle est la composition de l'air et comment la mesurer ?
  }
\end{contexte}


%%%%
\begin{doc}{Composition de l'air}{doc:TP2_composition_air}
  \begin{tableau}{|l|c|}
    Constituant & Proportion molaire \\
    Diazote \chemfig{N_2}             & \phantom{0}\qty{78,08}{\percent} \barrePourcentage{0.7808} \\
    Dioxygène \chemfig{O_2}           & \phantom{0}\qty{20,95}{\percent} \barrePourcentage{0.2095} \\
    Argon \chemfig{Ar}                & \phantom{00}\qty{0,93}{\percent} \barrePourcentage{0.0093} \\
    Dioxyde de carbone \chemfig{CO_2} & \phantom{00}\qty{0,04}{\percent} \barrePourcentage{0.0004} \\
    Néon \chemfig{Ne}                 & \qty{18,2}{\ppm} \barrePourcentage{0.0000182} \\
    Hélium \chemfig{He}               & \phantom{0}\qty{5,2}{\ppm} \barrePourcentage{0.0000052} \\
    Monoxyde d'azote \chemfig{NO}     & \phantom{0}\qty{5,0}{\ppm} \barrePourcentage{0.0000050} \\
    Méthane \chemfig{CH_4}            & \phantom{0}\qty{1,9}{\ppm} \barrePourcentage{0.0000019} \\
    Eau \eau                          & Variable \\ 
  \end{tableau}

  \vspace*{-80pt}
  \begin{center}
    \begin{tikzpicture}
      % Barres
      \fill[color = couleurPrim]    (0.0 , 0.0)        rectangle (0.7808*90ex, 3ex);
      \fill[color = couleurPrim!50] (0.7808*90ex, 0.0) rectangle (0.9903*90ex, 3ex);
      \fill[color = couleurPrim!25] (0.9903*90ex, 0.0) rectangle        (90ex, 3ex);
      % Valeurs
      \node[white] at (0.7808*45ex, 1.5ex) {\textbf{\qty{78,08}{\percent}}};
      \node[black] at (0.8855*90ex, 1.5ex) {\textbf{\qty{20,95}{\percent}}};
      \node[black] at        (91ex, 1.5ex) {\textbf{\qty{0,97} {\percent}}};
      % Carré légende
      \fill[color = couleurPrim]    (75ex, 15ex) rectangle (78ex, 12ex);
      \fill[color = couleurPrim!50] (75ex, 11ex) rectangle (78ex, 8ex);
      \fill[color = couleurPrim!25] (75ex, 7ex)  rectangle (78ex, 4ex);
      % Légende
      \node[black,right] at (78ex, 13.5ex) {\textbf{Diazote}};
      \node[black,right] at (78ex, 9.5ex)  {\textbf{Dioxygène}};
      \node[black,right] at (78ex, 5.5ex)  {\textbf{Autres gaz}};
    \end{tikzpicture}
  \end{center}

  \qty{1}{\percent} signifie qu'il y a 1 molécule sur un total de 100 molécules.
  
  \qty{1}{\ppm} signifie qu'il y a 1 molécule sur un total de \num{1000000}.
\end{doc}


\begin{doc}{Fraction molaire}{doc:TP2_fraction_molaire}
  La \important{fraction molaire} est le rapport entre la quantité de matière du constituant considéré et la quantité de matière totale dans le mélange étudié.

  La fraction molaire est noté $x_i$ pour le constituant $i$. Elle varie entre 0 et 1 et se calcule avec la relation :
  \begin{equation*}
    x_i = \dfrac{n_i}{n_\text{tot}}
  \end{equation*}

  $n_i$ est la quantité de matière du constituant $i$.

  $n_\text{tot}$ est la quantité de matière totale dans le mélange.
\end{doc}

\question{
  Convertir les proportions molaire des 5 premiers constituants de l'air en fraction molaire pour pouvoir les comparer.
}{}{4}


\begin{doc}{Quelques tests pour identifier des espèces chimiques}{doc:TP2_tests_identification}
  \pointCyan L'eau de chaux est une solution saturée en hydroxyde de calcium \chemfig{Ca} $(\chemfig{OH})_2$.
  En présence de dioxyde de carbone \chemfig{CO_2}, l'eau de chaux se trouble, suite à la formation d'un précipité blanc de carbonate de calcium \chemfig{CaCO_3}.

  \pointCyan Le sulfate de cuivre anhydre \chemfig{CuSO_4} est une poudre blanche.
  En contact avec des molécules d'eau \chemfig{H_2O} la poudre bleuit, suite à la formation d'un complexe pentahydrate \chemfig{CuSO_4, 5H_2O}.

  \pointCyan La combustion d'une allumette nécessite un combustible, la cellulose de formule brute \bruteCHO{6}{10}{5} du bois de l'allumette, et un comburant, le dioxygène \chemfig{O_2}.
  Cette réaction chimique forme du dioxyde de carbone \chemfig{CO_2} et de la vapeur d'eau \chemfig{H_2O}.
\end{doc}

\question{
  Pour chacun des 3 tests, établir l'équation de la réaction chimique mise en jeu.
}{
}{3}


%%%%
\begin{doc}{Combustion d'une bougie}{doc:TP2_combustion_bougie}
  Le combustible d'une bougie est l'acide stéarique de formule brute \bruteCHO{18}{36}{2}
  L'équation de la réaction de combustion d'une bougie est
  \begin{center}
    \bruteCHO{18}{36}{2}(s) + 26\chemfig{O_2}(g) \reaction
    18\chemfig{CO_2}(g) + 18\chemfig{H_2O}(g)
  \end{center}

  \textbf{Matériel :} un cristallisoir, une bougie, une éprouvette graduée.

  \textbf{Protocole :} remplir la coupelle d'eau.
  Placer et allumer la bougie au centre de la coupelle.
  Recouvrir la bougie avec l'éprouvette.

  Le dioxyde de carbone se dissout dans l'eau dès sa formation et la vapeur d'eau se condense rapidement, ce qui laisse un vide dans le récipient où à lieu la combustion.
\end{doc}

\begin{doc}{Volume molaire des gaz}{doc:TP2_volume_molaire}
  Le volume molaire des gaz vaut $V_m = \qty{24,1}{\litre\per\mole}$ à \qty{20}{\degreeCelsius} sous pression atmosphérique.
  Cette valeur est la même pour tous les gaz, donc \textbf{la fraction volumique est égale à la fraction molaire pour les gaz.}
\end{doc}

\mesure Réaliser l'expérience du document~\ref{doc:TP2_combustion_bougie} et mesurer le volume d'eau déplacé.

\question{
  En déduire la fraction volumique de dioxygène, puis la fraction molaire de dioxygène.
}{}{4}

\question{
  Comparer cette valeur avec celle fournie dans le document~\ref{doc:TP2_composition_air}.
}{}{1}