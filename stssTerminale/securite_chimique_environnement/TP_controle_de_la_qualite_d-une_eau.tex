\teteTermStssEnvi
\vspace*{-36pt}
\titreTP{Contrôle de la qualité d'une eau du robinet}
\vspace*{-12pt}

%%%% objectifs
\begin{objectifs}
  \item Comprendre le principe d'un dosage conductimétrique
  \item Savoir lire un volume équivalent à partir de l'évolution de la conductivité
\end{objectifs}

\begin{contexte}
  La teneur en ions \chlorure d'une eau destinée à la consommation ne doit pas dépasser \qty{250}{\mg\per\litre}.
  
  \problematique{
    Comment contrôler cette limite de qualité ?
  }
\end{contexte}


%%%%
\begin{doc}{Principe d'un dosage conductimétrique}
  Pour réaliser un dosage conductimétrique, on utilise une espèce chimique ionique pour qu'elle réagisse avec l'ion dont on veut mesurer la concentration pour former un solide.

  Par exemple, on veut mesurer les ions sulfate \sulfate présents dans de l'eau.
  On va utiliser une \important{solution titrante} d'ions baryum \chemfig{Ba^{2+}} pour doser, car les ions sulfate et baryum se transforment en une espèce solide :
  \begin{equation*}
    \sulfate\aq + \chemfig{Ba^{2+}}\aq \reaction \chemfig{BaSO_4}\sol
  \end{equation*}
  Comme on transforme des ions en solide, la solution voit d'abord sa conductivité baisser, avant de remonter quand tous les ions \sulfate présents au départ ont été transformés.

  \begin{importants}
    Cette variation de la conductivité permet de mesurer précisément \important{l'équivalence.}
    À l'équivalence, on a introduit la même quantité de matière de \chemfig{Ba^{2+}} qu'il y avait d'ions \sulfate initialement
    \begin{equation*}
      n(\chemfig{Ba^{2+}}) = n(\sulfate)
    \end{equation*}
  \end{importants}
\end{doc}

\begin{doc}{Protocole d'un dosage conductimétrique}
  En pratique pour réaliser un dosage conductimétrique, il faut :
  \begin{protocole}
    \item verser millilitre par millilitre la \important{solution titrante} dans l'eau à doser ;
    \item mesurer la \important{conductivité $\sigma$} pour chaque millilitre versé avec un \important{conductimètre} ;
    \item tracer la conductivité en fonction du volume de la solution titrante versé $V$ ;
    \item tracer les deux droites de la conductivité et repérer leur point d'intersection (l'équivalence).
  \end{protocole}

  \vspace*{-10pt}
  \begin{wrapfigure}[2]{r}{0.5\linewidth}
    \hspace*{-60pt}
    \image{0.85}{images/chimie/montage_dosage_conductimetrie}
  \end{wrapfigure}

  \vAligne{-24pt}
  \begin{importants}  
    À l'intersection des droites, on a \important{l'équivalence,} associé au \important{volume d'équivalence $V_\text{eq}$}.
    L'équivalence permet de calculer la concentration de l'espèce qui nous intéresse.
  \end{importants}
  \vAligne{3.2cm}
  \phantom{b}
\end{doc}

\newpage
\begin{doc}{Dosage d'une eau du robinet}
  \begin{wrapfigure}{r}{0.4\linewidth}
    \vspace*{-36pt}
    \centering
    \image{1}{images/exterieures/donnees/conductivite_dosage}
  \end{wrapfigure}
  Pour contrôler la qualité de l'eau du robinet, on réalise un dosage conductimétrique.

  On prélève $V_\text{eau} = \qty{200}{\mL}$ d'eau du robinet dont on veut déterminer la concentration en ions  \chlorure.
  On dose cette eau avec une solution de nitrate d'argent (\ionArgent, \nitrate), de concentration en ions $[\ionArgent] = \qty{1,0e-2}{\mol\per\litre}$.

  Une fois mélangés, les ions argent et les ions  se transforment pour former un précipité blanc de  d'argent \chlorureDArgent.

  On obtient la courbe de dosage présentée à droite, avec les points mesurées et les deux droite de tendances.
  Les deux droites se croisent pour $V = \qty{16,5}{\ml}$.
\end{doc}

%%
\question{
  Donner l'espèce chimique titrante et l'espèce que l'on veut doser dans l'eau du robinet.
}{}[3]

\question{
  Écrire la réaction de dosage entre l'espèce titrante et l'espèce à doser.
}{}[2]

\question{
  Donner la valeur du volume équivalent $V_\text{eq}$.
}{}[2]

\question{
  Calculer la concentration molaire $[\chlorure]$, en montrant que $[\ionArgent] \times V_\text{eq} = [\chlorure] \times V_\text{eau}$.
}{}[6]

\question{
  Calculer la concentration massique en ions  dans cet échantillon d'eau du robinet.
  Conclure si cette eau du robinet est propre à la consommation ou non.

  \important{Donnée :} \masseMol*{Cl} = \qty{35,5}{\g\per\mole}
}{}[3]