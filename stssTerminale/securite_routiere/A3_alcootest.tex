%%%%
\teteTermStssRout

%%%% titre
\numeroActivite{3}
\titreActivite{Principe de fonctionnement d'un alcootest}


%%%% objectifs
\begin{objectifs}
  \item Comprendre le principe d'un alcootest
  \item Revoir les réaction d'oxydoréduction
\end{objectifs}


%%%% docs
\begin{doc}{Principe de l'alcootest}{doc:A3_alcootest}
  L'alcootest est constitué d'un tube en verre dans lequel on fait circuler l'air préalablement expiré dans un ballon en plastique de 1 litre.
  
  L'air expiré traverse une zone constituée de grains jaune-orangé de dichromate de potassium.

  Si l'haleine contient de l'alcool, le solide jaune-orangé devient vert.

  Un repère situé à peu près au premier tiers de la zone de détection indique la limite à ne pas dépasser.

  \centering
  \image{1}{images/photos/alcootest}
\end{doc}

%%
\begin{doc}{Dichromate de potassium}{doc:A3_dichromate}
  Le dichromate de potassium \chemfig{K_2 Cr_2 O_7} est un solide ionique constitué de cations potassium \chemfig{K^+} incolores et d'anions dichromate responsables de la couleur jaune-orangé.
  
  Le dichromate est un oxydant et les ions \chemfig{K^+} n'interviennent pas : ils sont spectateurs.

  L'anion dichromate est très toxique, cancérigène et nuit à l'environnement.
  \begin{center}
    \image{0.1}{images/securite/picto_comburant}
    \image{0.1}{images/securite/picto_corrosif}
    \image{0.1}{images/securite/picto_toxique}
    \image{0.1}{images/securite/picto_environnement}
    \image{0.1}{images/securite/picto_reprotoxique}
  \end{center}
\end{doc}

\begin{doc}{Réaction d'oxydo-réduction dans un alcootest}{doc:A3_reaction_chim_alcootest}
  L'alcootest exploite une réaction chimique d'oxydoréduction.
  L'éthanol \chemfig{C_2H_6O} contenu dans l'air expiré par une personne alcoolisée constitue le réducteur destiné à être oxydé en acide éthanoïque \chemfig{C_2H_4O_2} par l'ion dichromate \chemfig{Cr_2 O_7^{2-}} contenu dans le tube de test.
  \smallskip

  \begin{tblr}{
    vlines, hlines, column{1} = {couleurSec-100},
    colspec = {l X[c] X[c]}
  }
    Couple Ox/Red & \chemfig{Cr_2 O_7^{2-}}/\chemfig{Cr^{3+}} & \chemfig{C_2H_4O_2}/\chemfig{C_2H_6O} \\ 
    Couleurs & orange/vert & incolore/incolore \\
    %
    Demi-équation &
    {\chemfig{Cr_2 O_7^{2-}} 14\chemfig{H^+} + 6\chemfig{e^{–}} \\ = 2 \chemfig{Cr^{3+}} + 7 \chemfig{H_2O}} &
    {\chemfig{C_2H_4O_2} + 4\chemfig{H^+} + 4\chemfig{e^{–}} \\ = \chemfig{C_2H_6O} + \chemfig{H_2O}}
  \end{tblr}
\end{doc}


\question{
  Qui est l'oxydant dans le couple formé par l'ion dichromate et l'ion chromique \chemfig{Cr_2 O_7^{2-}}/\chemfig{Cr^{3+}} ?
  Même question pour l'éthanol et l'acide éthanoïque \chemfig{C_2H_4O_2}/\chemfig{C_2H_6O}.
}{}[3]


\begin{doc}{Démarche pour établir l'équation d'une réaction redox}{doc:A3}
  Pour établir l'équation d'une réaction d'oxydoréduction il faut
  \begin{listePoints}
    \item Identifier les deux réactifs $Ox_1$ et $Red_2$.
    \item Écrire, l'une sous l'autre, les deux demi-équations en mettant les réactifs à gauche.
    \item Ajuster les coefficients des deux demi-équations pour obtenir le même nombre d'électrons.
    \item Additionner côté par côté les deux demi-équations.
    \item Vérifier que les charges et les éléments sont conservés, puis supprimer les électrons.
  \end{listePoints}
\end{doc}

%%
\question{
  Établir l’équation de la réaction d’oxydoréduction entre l'éthanol et  sous la forme $Ox_1 + Red_2 \reaction Red_1 + Ox_2$.
}{}[1]

\question{
  Interpréter les changements de couleurs observés lorsque l’alcootest est positif.
}{
}[4]