\teteTermStssRout
\titreActivite{Principe de fonctionnement d'un airbag}

%%%% objectifs
\begin{objectifs}
  \item Étudier le fonctionnement d'un airbag
\end{objectifs}


%%%% docs
\begin{doc}{Utilité d'un airbag}
  \begin{wrapfigure}{r}{0.5\linewidth}  
    \centering
    \vspace*{-24pt}
    \image{1}{images/exterieures/securite/ouverture_airbag} \\[4pt]
    \legende{Ouverture d'un airbag}
  \end{wrapfigure}
  
  Les airbags sont utilisés dans les voitures, pour protéger les passagers en cas de choc violent.
  L'airbag permettrait d'obtenir jusqu'à \qty{25}{\percent} de personnes tuées en moins sur les routes.
  Pour protéger efficacement les passagers, l'airbag doit se gonfler en une fraction de seconde.
\end{doc}

%%
\begin{doc}{Accéléromètre}
  Un accéléromètre est un dispositif qui permet de détecter des variations de vitesse.
  En cas de choc la voiture passe de sa vitesse de croisière à une vitesse nulle en très peu de temps.
  La décélération, alors très importante, est détectée par l'accéléromètre, qui transmet un signal électrique au détonateur, ce qui déclenche une suite de réactions chimiques dont l'un des produits est le gaz servant à gonfler l'airbag.
  Ce processus est extrêmement rapide : il prend environ \qty{0,1}{\s}.
\end{doc}

\begin{doc}{Réaction chimiques servant à gonfler un airbag}
  Le détonateur enclenche d'abord la décomposition extrêmement rapide (explosive) de l'azoture de sodium solide $\azotureDeSodium\sol$ en sodium solide $\chemfig{Na}\sol$ et diazote gazeux $\diazote\gaz$ selon la réaction (1) d'équation :
  \begin{equation}
    2 \chemfig{NaN_3}\sol \reaction 2\chemfig{Na}\sol + 3\diazote\gaz
  \end{equation}
  
  Le sodium, dangereux, est éliminé au fur et à mesure de sa formation par le nitrate de potassium, selon la réaction (2) d'équation :
  \begin{equation}
    10\chemfig{Na}\sol + 2\chemfig{KNO_3}\sol \reaction \chemfig{K_2O}\sol + 5 \chemfig{Na_2O}\sol + \diazote\gaz
  \end{equation}

  Puis l'oxyde de potassium $\chemfig{K_2O}\sol$ et l'oxyde de sodium $\chemfig{Na_2O}\sol$ sont consommés à leur tour par la silice \chemfig{SiO_2} selon les réactions (3) et (4) d'équations :
  \begin{align}
    \chemfig{K_2O}\sol + \chemfig{Na_2O}\sol + \chemfig{SiO_2}\sol &\reaction \chemfig{K_2 Na_2 SiO_4}\sol \\
    2\chemfig{Na_2 O}\sol + \chemfig{SiO_2}\sol &\reaction \chemfig{Na_4SiO_4}\sol
  \end{align}
\end{doc}


\begin{doc}{Danger des espèces chimiques intervenant dans le gonflement d'un airbag}
  \begin{tblr}{
    hlines, vlines, row{1} = {couleurSec-100}, width = \linewidth,
    colspec = {X[c,m] X[2,c,m] X[c,m] X[c,m] X[c,m] X[c,m] X[1, c,m] X[c,m]}
  }
    \chemfig{NaN_3} & \chemfig{Na} &
    \diazote & \chemfig{KNO_3} &
    \chemfig{K_2O} et \chemfig{Na_2O} & \chemfig{SiO_2} &
    \chemfig{K_2Na_2} \chemfig{SiO_4} & \chemfig{Na_4 SiO_4} \\
    %
    très toxique & s'enflamme au contact l'eau &
    inoffensif & irritant &
    corrosif & inoffensif &
    inoffensif & inoffensif 
  \end{tblr} \\[4pt]
  
  Tant que l’airbag ne s’est pas gonflé, l'azoture de sodium est inaccessible, donc sans danger.
  À la fin du gonflage, tous les produits restants sont inoffensifs.
\end{doc}


%%%%
\question{
  Donner le nom et la formule brute du gaz utilisé pour gonfler un airbag.
}{}[2]

\question{
  Préciser l'intérêt de la réaction chimique 2.
}{}[2]

\question{
  Préciser la nécessité d'utiliser la silice.
}{}[2]

\question{
  L'airbag du conducteur contient \qty{70}{\litre} de gaz dans des conditions de températures et de pressions telles que le volume molaire est égal à $V_m = \qty{30}{\litre\per\mole}$. \\
  Calculer en moles la quantité de matière de gaz produit pour remplir l'airbag.  
}{}[4]

\question{
  Sachant que les équations des réaction 1 et 2 peuvent être réduite à l'équation suivante
  \begin{equation*}
    10 \chemfig{NaN_3}\sol + 2 \chemfig{KNO_3}\sol \reaction \chemfig{K_2 O}\sol + 5\chemfig{Na_2 O}\sol + 16\diazote\gaz
  \end{equation*}
  En déduire la quantité de matière initiale d'azoture de sodium qu'il a fallu introduire pour obtenir les \qty{70}{\litre} de gaz du coussin d'air.
}{
}[8]

\question{
  En déduire la masse de \chemfig{NaN_3} présente initialement dans l’airbag.
  
  \begin{donnees}[3]
    \item \masseMol{N}  = \qty{14}{\g/\mole}, 
    \item \masseMol{O}  = \qty{16}{\g/\mole},
    \item \masseMol{Na} = \qty{23}{\g/\mole},
    \item \masseMol{K}  = \qty{39,1}{\g/\mole},
    \item \masseMol{Si} = \qty{28,1}{\g/\mole}
  \end{donnees} \phantom{b}\vspace*{-20pt}
}{
}[3]