%%%%
\teteTermStssRout
\titreActivite{Principe de fonctionnement d'un éthylotest}

%%%% objectifs
\begin{objectifs}
  \item Comprendre le principe d'un éthylotest
  \item Revoir les réaction d'oxydoréduction
\end{objectifs}


%%%% docs
\begin{doc}{Principe de l'éthylotest}{doc:A3_alcootest}
  L'éthylotest est constitué d'un tube en verre dans lequel on fait circuler l'air préalablement expiré dans un ballon en plastique de 1 litre.  
  L'air expiré traverse une zone constituée de grains jaune-orangé de dichromate de potassium.
  Si l'haleine contient de l'alcool, le solide jaune-orangé devient vert.

  Un repère situé au premier tiers de la zone de détection indique la limite légale à ne pas dépasser, qui correspond à \qty{0,25}{\milli\g\per\litre} d’air expiré ou \qty{0.5}{\g\per\litre} dans le sang, soit deux verres standard d'alcool.

  \begin{center}
    \image{1}{images/photos/alcootest}
  \end{center}
\end{doc}

%%
\begin{doc}{Dichromate de potassium}{doc:A3_dichromate}
  Le dichromate de potassium \chemfig{K_2 Cr_2 O_7} est un solide ionique constitué de cations potassium \ionPotassium incolores et d'anions dichromate responsables de la couleur jaune-orangé.
  
  Le dichromate est un oxydant et les ions \ionPotassium n'interviennent pas : ils sont spectateurs.

  L'anion dichromate est très toxique, cancérigène et nuit à l'environnement.
  \begin{center}
    \image{0.1}{images/securite/picto_comburant}
    \image{0.1}{images/securite/picto_corrosif}
    \image{0.1}{images/securite/picto_toxique}
    \image{0.1}{images/securite/picto_environnement}
    \image{0.1}{images/securite/picto_reprotoxique}
  \end{center}
\end{doc}

\begin{doc}{Rappel sur les réaction d'oxydo-réduction}{doc:A_reaction_oxydoreduction}
  \begin{importants}  
    Un \important{oxydant} est une espèce chimique capable d'\important{obtenir} un ou plusieurs \important{électrons.}

    Un \important{réducteur} est une espèce chimique capable de \important{relâcher} un ou plusieurs \important{électrons.}
  \end{importants}

  Un oxydant et un réducteur forment un couple Oxydant/Réducteur, si l'on peut passer de l'un à l'autre par le gain ou la perte d'électrons.
  Le couple est noté Ox/Réd. \exemple{\chemfig{Zn^{2+}}/\chemfig{Zn}}.
  
  \begin{importants}
    Une réaction \important{d'oxydoréduction} a lieu quand on met en contact un oxydant et un réducteur de deux couples différents.
  \end{importants}
  
  Elle met donc en jeu deux couples oxydant/réducteur.
  Par exemple avec un couple du fer : \ionFerIII/\chemfig{Fe} ; et un couple de l'oxygène : \dioxygene/\ionOxygene

  Le gaz \dioxygene va réagir avec le solide \chemfig{Fe}, pour se transformer en ion \ionFerIII et en ion \ionOxygene, qui vont se combiner pour former de l'hématite solide \chemfig{Fe_2 O_3} (la rouille).
  
  \begin{equation*}
    4\chemfig{Fe}\sol + 3\dioxygene\gaz \reaction 4\ionFerIII + 6\ionOxygene \reaction 2\chemfig{Fe_2 O_3}\sol
  \end{equation*}
\end{doc}

\begin{doc}{Réaction d'oxydo-réduction dans un éthylotest}{doc:A3_reaction_chim_alcootest}
  L'éthylotest exploite une réaction chimique d'oxydoréduction.
  L'éthanol \chemfig{C_2H_6O} contenu dans l'air expiré par une personne alcoolisée constitue le réducteur destiné à être oxydé en acide éthanoïque \chemfig{C_2H_4O_2} par l'ion dichromate \chemfig{Cr_2 O_7^{2-}} contenu dans le tube de test.
  \smallskip

  \begin{tblr}{
    vlines, hlines, column{1} = {couleurSec-100},
    colspec = {l X[c] X[c]}
  }
    Couple Ox/Red & \chemfig{Cr_2 O_7^{2-}}/\chemfig{Cr^{3+}} & \chemfig{C_2H_4O_2}/\chemfig{C_2H_6O} \\ 
    Couleurs & orange/vert & incolore/incolore \\
    %
    Demi-équation &
    {\chemfig{Cr_2 O_7^{2-}} 14\ionHydrogene + 6\chemfig{e^{–}} \\ = 2 \chemfig{Cr^{3+}} + 7 \eau} &
    {\chemfig{C_2H_4O_2} + 4\ionHydrogene + 4\chemfig{e^{–}} \\ = \chemfig{C_2H_6O} + \eau}
  \end{tblr}
\end{doc}


\question{
  Qui est l'oxydant dans le couple formé par l'ion dichromate et l'ion chromique \chemfig{Cr_2 O_7^{2-}}/\chemfig{Cr^{3+}} ?
  Même question pour l'éthanol et l'acide éthanoïque \chemfig{C_2H_4O_2}/\chemfig{C_2H_6O}.
}{}[3]


\begin{doc}{Démarche pour établir l'équation d'une réaction d'oxydoréduction}{doc:A3}
  Pour établir l'équation d'une réaction d'oxydoréduction il faut
  \begin{listePoints}
    \item Identifier les deux réactifs $\text{Oxydant}_1$ et $\text{Réducteur}_2$.
    \item Écrire, l'une sous l'autre, les deux demi-équations en mettant les réactifs à gauche.
    \item Ajuster les coefficients des deux demi-équations pour obtenir le même nombre d'électrons.
    \item Additionner côté par côté les deux demi-équations.
    \item Vérifier que les charges et les éléments sont conservés, puis supprimer les électrons.
  \end{listePoints}
\end{doc}

%%
\question{
  Établir l’équation de la réaction d’oxydoréduction entre l'éthanol et  sous la forme $\text{Oxydant}_1 + \text{Réducteur}_2 \reaction \text{Réducteur}_1 + \text{Oxydant}_2$.
}{}[7]

\question{
  Interpréter les changements de couleurs observés lorsque l’éthylotest est positif.
}{
}[4]