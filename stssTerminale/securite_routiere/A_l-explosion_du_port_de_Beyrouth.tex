\teteTermStssRout
\titreActivite{L'explosion du port de Beyrouth}

\begin{objectifs}
  \item Faire un bilan de matière à partir d'une équation de réaction fournie.
  \item Utiliser la relation entre le volume et le volume molaire $V = n \times V_m$.
\end{objectifs}

\begin{contexte}
  Le 4 août 2020, une terrible explosion a fait voler en éclats le port de Beyrouth, blessant plus de \num{6500} personnes et causant \num{190} décès.
  La cause, découverte récemment, indique qu'un incendie se serait déclaré dans un entrepôt de nitrate d’ammonium.
  
  \problematique{
    Comment expliquer l’ampleur de l’explosion dans ce hangar ?
  }
\end{contexte}


%%%%
\titreSection{Le stockage}

%%
\begin{doc}{Description du stockage à Beyrouth}[\label{doc:description_stockage}]
  Le conseil supérieur de la défense indique qu'un incendie s’est déclaré dans un hangar de \qty{50000}{\cubic\metre} dans lequel étaient stockés \qty{2750e3}{\kg} de nitrate d’ammonium de formule brute \nitrateAmmonium.
\end{doc}

%%
\begin{doc}{Tableau descriptif des espèces chimiques}[\label{doc:descriptif_especes_chimiques}]
  \centering
  \begin{tblr}{
    colspec = {X X X X X X}, width = \linewidth,
    hlines, vlines, row{1} = {c, m, couleurSec-100}, row{2} = {c},
  }
    Espèce chimique & Nitrate d'ammonium & diazote & dioxygène & eau \\
    Formule brute & \nitrateAmmonium & \diazote & \dioxygene & \eau \\
    Propriétés physico-chimiques & 
    Solide à \qty{20}{\degreeCelsius}. \newline (poudre). Légèrement nocif. &
    Gazeux à \qty{20}{\degreeCelsius}. Gaz incolore inerte présent dans l’air. &
    Gazeux à \qty{20}{\degreeCelsius}. Gaz incolore oxydant présent dans l’air. Comburant. &
    Liquide à \qty{20}{\degreeCelsius}. Amphotère. 
  \end{tblr}    
\end{doc}

\begin{donnees}[2]
  \item \masseMol{C} = \qty{12,0}{\g\per\mole}
  \item \masseMol{N} = \qty{14,0}{\g\per\mole}
  \item \masseMol{O} = \qty{16,0}{\g\per\mole}
  \item \masseMol{H} = \qty{1,0}{\g\per\mole}
\end{donnees}

\question{
  Donner le nom et la formule brute de l’espèce chimique entreposée dans le port de Beyrouth responsable de l’explosion.
}{
  C'est le nitrate d'ammonium \nitrateAmmonium.
}[2]

\question{
  Après avoir converti la masse de cette espèce chimique en gramme, calculer sa masse molaire notée \masseMol{NH_4 NO_3}.
}{
  \begin{equation*}
    m(\nitrateAmmonium) = \qty{2.75e9}{\g}
  \end{equation*}
  \begin{align*}
    \masseMol{NH_4 NO_3}
    &= 2\masseMol{N} + 4\masseMol{H} + 3\masseMol{O} \\
    &= (2\times\num{14.0} + 4\times\num{1.0} + 3\times\num{16.0}) \unit{\g\per\mole} \\
    &= \qty{80.0}{\g\per\mole}
  \end{align*}
}[4]


%%
\begin{doc}{calcul de quantité de matière (solide et gaz)}[\label{doc:calcul_mole_sol_gaz}]
  La relation utilisée pour calculer la quantité de matière dépend de l'état physique de l'espèce chimique.
  \begin{multicols}{2}
    Espèces chimique à l'état solide
    \begin{equation*}
      n = \dfrac{m}{M}
    \end{equation*}
    \begin{listePoints}
      \item $n$ la quantité de matière en \unit{\mole}
      \item $m$ la masse en \unit{\g}
      \item $M$ la masse molaire en \unit{\g\per\mole}
    \end{listePoints}
    La masse molaire se calcule en additionnant les masse molaire atomique des entités chimiques qui composent la molécule.
    
    Espèce chimique à l'état gazeux
    \begin{equation*}
      n = \dfrac{V}{V_m}
    \end{equation*}
    \begin{listePoints}
      \item $n$ la quantité de matière en \unit{\mole}
      \item $V$ le volume en \unit{\litre}
      \item $V_m$ le volume molaire en \unit{\litre\per\mole}
    \end{listePoints}
    Le volume molaire d'un gaz est une constante $V_m = \qty{24,0}{\litre\per\mole}$ (à \qty{20}{\degreeCelsius} et sous pression atmosphérique).
  \end{multicols}
\end{doc}

\question{
  En déduire, à l'aide du document~\ref{doc:description_stockage} et~\ref{doc:calcul_mole_sol_gaz}, la quantité de matière $n_1$ de nitrate d’ammonium entreposée dans le hangar.
}{
  \begin{align*}
    n_1 
    &= \dfrac{m(\nitrateAmmonium)} {\masseMol{NH_4 NO_3}} \\
    &= \dfrac{\qty{2.75e9}{\g}} {\qty{80.0}{\g\per\mole}} \\
    &= \qty{3.44e7}{\mole}
  \end{align*}
}[4]


%%%%
\titreSection{La réaction produite par l’incendie}

%%
\begin{doc}{Rappels sur la réaction chimique}
  On réalise une transformation chimique lorsqu’on mélange des espèces chimiques et que de nouvelles espèces chimiques apparaissent.
  \begin{importants}
    Pour modéliser une transformation chimique on écrit une \important{réaction chimique} entre entités chimiques.
  \end{importants}
  équation de la transformation chimie produite lors de l’incendie dans le hangar à \qty{300}{\degreeCelsius} :
  \begin{center}
    \important{2}\nitrateAmmonium\sol \reaction
    \important{2}\diazote\gaz + \dioxygene\gaz + \important{4}\eau\liq
  \end{center}

  \begin{multicols}{2}
    \begin{importants}
      Les espèces chimiques qui sont transformées au cours de la réaction chimique sont les \important{réactifs.}
      Les réactifs sont à gauche dans la réaction.
    \end{importants}
    \begin{importants}
      Les espèces chimiques qui sont produites au cours de la réaction chimique sont les \important{produits.}
      Les produits sont à droite dans la réaction.
    \end{importants}
  \end{multicols}
\end{doc}

%%
\pasCorrection{\newpage}
\begin{doc}{Faire un bilan de matière}[\label{doc:bilan_matiere}]
  L'équation de la réaction est comme une recette de cuisine :
  \begin{center}
    \important{2}\nitrateAmmonium\sol \reaction
    \important{2}\diazote\gaz + \dioxygene\gaz + \important{4}\eau\liq
  \end{center}
  Si je mélange \important{deux} \nitrateAmmonium, il se forme \important{deux} \diazote,
  \important{un} \dioxygene\; et \important{quatre} \eau.
  
  Si je mélange 4 \nitrateAmmonium, il se forme 4 \diazote, 2 \dioxygene\; et 8 \eau.
  
  Si je mélange 6 \nitrateAmmonium, il se forme \texteTrou{6} \diazote, \texteTrou{3} \dioxygene\; et \texteTrou{12} \eau.
  
  Si je mélange \qty{2,4}{\mole} de \nitrateAmmonium, il se forme
  \texteTrou{$\qty{2,4}{\mole}$} \diazote,
  \texteTrou{$\qty{1,2}{\mole}$} \dioxygene\; et \\
  \texteTrou{$\qty{4,8}{\mole}$} \eau.
\end{doc}

\begin{donnees}[2]
  \item $\qty{1}{\cubic\metre} = \qty{e3}{\litre}$
  \item $\qty{1}{\kelvin} = 1 + \qty{273}{\degreeCelsius}$
\end{donnees}

\question{
  Réécrire l’équation de la réaction produite lors de l’incendie.
  À partir du document~\ref{doc:bilan_matiere}, nommer les réactifs et les produits de cette réaction chimique. 
  En vous aidant du document~\ref{doc:descriptif_especes_chimiques}, indiquer si ces espèces sont dangereuses.
}{
  Réactifs : nitrate d'ammonium (légèrement nocif) ;\par
  Produits : diazote, dioxygène et eau (sans dangers)
}[4]

\documentaire
La chaleur apportée par l’incendie a permis à la réaction de se produire.
Compléter la première ligne « \important{avant l'incendie} » et la deuxième ligne « \important{après l'incendie} », du tableau ci-dessous, en vous aidant du document~\ref{doc:bilan_matiere}

\vspace*{8pt}
\begin{tblr}{
  vlines, hlines,
  colspec = {c X[l,m] X[l,m] X[l,m] X[l,m]},
  hline{1,2} = {1-5}{dashed}, vline{1,6} = {1}{dashed},
  vline{2} = {1}{text = \clap{:}}, vline{4,5} = {1}{text = \clap{+}},
  row{1} = {couleurPrim-100, c}, row{3,4} = {l, 36pt}
}
  Équation de la réaction &
  \SetCell[c=2]{c} 2 \nitrateAmmonium\sol \reaction 2 \diazote\gaz \qq{}\qq{} & &
  \dioxygene\gaz &
  4 \eau\liq \\
  %
  État du système & \SetCell[c=4]{c} \important{Quantités de matières (\unit{\mole})} \\
  %
  \important{Avant l’incendie} &
  $n_1 =$ \correction{\num{3.44e7}} &
  $n(\diazote) =$ \correction{\num{0}} &
  $n(\dioxygene) =$ \correction{\num{0}} &
  $n(\eau) =$ \correction{\num{0}} \\
  %
  \important{Après l’incendie} &
  $n_{f,1} =$ \correction{\num{0}} &
  $n_f(\diazote) =$ \correction{$n_1$} &
  $n_f(\dioxygene) =$ \correction{$\dfrac{1}{2} n_1$}
  $n_f(\eau) =$ \correction{$2 n_1$}
\end{tblr}

\vspace*{8pt}
\question{
  En utilisant le document~\ref{doc:calcul_mole_sol_gaz} et le tableau ci-dessus, calculer (dans les conditions normales), le volume de diazote $V(\diazote)$,
  de dioxygène $V(\dioxygene)$ et de vapeur d’eau $V(\eau)$ produit.
}{
  On multiplie la quantité de matière par le volume molaire pour trouver le volume occupé par chaque gaz :
  \begin{align*}
    V(\diazote) 
    &= n(\diazote) \times V_m
     = \qty{3.44e7}{\mole} \times \qty{24.0}{\litre\per\mole}
     = \qty{8.25e8}{\litre} \\
    %
    V(\dioxygene) 
    &= n(\dioxygene) \times V_m
     = \qty{1.71e7}{\mole} \times \qty{24.0}{\litre\per\mole}
     = \qty{4.13e8}{\litre} \\
    %
    V(\eau) 
    &= n(\eau) \times V_m
     = \qty{6.88e7}{\mole} \times \qty{24.0}{\litre\per\mole}
     = \qty{1.65e9}{\litre}
  \end{align*}
}[6]

\pasCorrection{\newpage}
\question{
  Soit $n$ la quantité de matière produite totale avec 
  $n = n_f(\diazote) + n_f(\dioxygene) + n_f(\eau)$
  et $V$ le volume totale 
  $V = V(\diazote) + V(\dioxygene) + V(\eau)$.
  Calculer $n$ et $V$.
}{
  $n = n_1 + \dfrac{1}{2}n_1 + 2n_1
  = \dfrac{7}{2}n_1 = \qty{1.20e8}{\mole}$

  $V = \qty{2.89e9}{\litre}$
}[6]

\question{
  Conclure sur la valeur de $V$ par rapport à celle du hangar
}{
  Le volume du hangar est de $\qty{50000}{\m\cubed} = \qty{5.00e7}{\litre}$, soit un volume 58 fois plus petit que le volume de gaz libéré.
}[4]

\question{
  \textit{Pour les plus rapides.}
  La relation des gaz parfait est la suivante :
  \begin{equation*}
    PV = nRT
  \end{equation*}
  avec $R = \qty{8,31}{\pascal\cubic\metre\per\mole\per\kelvin}$, $n$ la quantité totale de gaz et ici $V = \qty{5e4}{\m\cubed}$ représente le volume du hangar.
  
  Sachant que la température dans le hangar était de \qty{873}{\kelvin} après la réaction, calculer la pression produite par la réaction.

  La comparer avec la pression atmosphérique $P_\text{atm} = \qty{100}{\kilo\pascal}$ et la pression dans un pneu de vélo $P_\text{pneu} = \qty{300}{\kilo\pascal}$.
}{
  On calcule la pression à l'aide de la loi des gaz parfaits
  \begin{align*}
    P &= \dfrac{nRT}{V} \\
    &= \dfrac{\qty{1.20e8}{\mole} \times \qty{873}{\kelvin} \times \qty{8,31}{\pascal\cubic\metre\per\mole\per\kelvin}} {\qty{2.89e9}{\litre}} \\
    &= \qty{1.74e7}{\pascal} 
    = \qty{1.74e4}{\kilo\pascal} \\
    &= \qty{17400}{\kilo\pascal}
  \end{align*}
  Soit une pression 58 fois plus élevée que dans un pneu atteinte en quelques secondes !
}[8]
