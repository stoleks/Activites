%%%%
\teteTermStssRout

% \termStssRout

%%%% titre
\numeroActivite{1}
\titreActivite{L'explosion du port de Beyrouth}


%%%% objectifs
\begin{objectifs}
  \item Faire un bilan de matière à partir d'une équation de réaction fournie
  \item Utiliser la relation entre le volume et le volume molaire $V = n \times V_m$
\end{objectifs}

\begin{contexte}
  Le 4 août 2020, une terrible explosion a fait voler en éclats le port de Beyrouth, blessant plus de \num{6500} personnes et causant \num{190} décès.
  La cause, découverte récemment, indique qu'un incendie se serait déclaré dans un entrepôt de nitrate d’ammonium.
  
  \problematique{
    Comment expliquer l’ampleur de l’explosion dans ce hangar ?
  }
\end{contexte}


%%%%
\titreSousSection{Le stockage}

%%
\begin{doc}{Description du stockage à Beyrouth}{doc:A1_description_stockage}
  Le conseil supérieur de la défense indique qu'un incendie s’est déclaré dans un hangar de \qty{50000}{\cubic\metre} dans lequel étaient stockés \qty{2750e3}{\kg} de nitrate d’ammonium de formule brute \chemfig{NH_4 NO_3}.
\end{doc}

%%
\begin{doc}{Tableau descriptif des espèces chimiques}{doc:A1_descriptif_especes_chimiques}
  \centering
  \begin{tblr}{
    colspec = {X X X X X X}, width = \linewidth,
    hlines, vlines, row{1} = {c, m, couleurSec100}, row{2} = {c},
  }
    Espèce chimique & Nitrate d'ammonium & diazote & dioxygène & eau \\
    Formule brute & \chemfig{NH_4 NO_3} & \chemfig{N_2} & \dioxygene & \eau \\
    Propriétés physico-chimiques & 
    Solide à \qty{20}{\degreeCelsius}. \newline (poudre). Légèrement nocif. &
    Gazeux à \qty{20}{\degreeCelsius}. Gaz incolore inerte présent dans l’air. &
    Gazeux à \qty{20}{\degreeCelsius}. Gaz incolore oxydant présent dans l’air. Comburant. &
    Liquide à \qty{20}{\degreeCelsius}. Amphotère. 
  \end{tblr}    
\end{doc}

\begin{donnees}[2]
  \item \masseMol{C} = \qty{12,0}{\g\per\mole}
  \item \masseMol{N} = \qty{14,0}{\g\per\mole}
  \item \masseMol{O} = \qty{16,0}{\g\per\mole}
  \item \masseMol{H} = \qty{1,0}{\g\per\mole}
\end{donnees}

\question{
  Donner le nom et la formule brute de l’espèce chimique entreposée dans le port de Beyrouth responsable de l’explosion.
}{

}[2]

\question{
  Après avoir converti la masse de cette espèce chimique en gramme, calculer sa masse molaire notée \masseMol{NH_4 NO_3}.
}{

}[4]


%%
\begin{doc}{calcul de quantité de matière (solide et gaz)}{doc:A1_calcul_mole_sol_gaz}
  La relation utilisée pour calculer la quantité de matière dépend de l'état physique de l'espèce chimique.
  \begin{multicols}{2}
    Espèces chimique à l'état solide
    \begin{equation*}
      n = \dfrac{m}{M}
    \end{equation*}
    \begin{listePoints}
      \item $n$ la quantité de matière en \unit{\mole}
      \item $m$ la masse en \unit{\g}
      \item $M$ la masse molaire en \unit{\g\per\mole}
    \end{listePoints}
    La masse molaire se calcule en additionnant les masse molaire atomique des entités chimiques qui composent la molécule.
    
    Espèce chimique à l'état gazeux
    \begin{equation*}
      n = \dfrac{V}{V_m}
    \end{equation*}
    \begin{listePoints}
      \item $n$ la quantité de matière en \unit{\mole}
      \item $V$ le volume en \unit{\litre}
      \item $V_m$ la volume molaire en \unit{\litre\per\mole}
    \end{listePoints}
    Le volume molaire d'un gaz est une constante $V_m = \qty{24}{\litre\per\mole}$ (à \qty{20}{\degreeCelsius} et sous pression atmosphérique)
  \end{multicols}
\end{doc}

\question{
  En déduire, à l'aide du document~\ref{doc:A1_description_stockage} et~\ref{doc:A1_calcul_mole_sol_gaz}, la quantité de matière $n_1$ de nitrate d’ammonium entreposée dans le hangar.
}{
  
}[4]


%%%%
\titreSousSection{La réaction produite par l’incendie}

%%
\begin{doc}{Rappels sur la réaction chimique}{doc:A1_rappel_reaction_chimique}
  On réalise une transformation chimique lorsqu’on mélange des espèces chimiques et que de nouvelles espèces chimiques apparaissent.
  \begin{importants}
    Pour modéliser une transformation chimique on écrit une \important{réaction chimique} entre entités chimiques.
  \end{importants}
  équation de la transformation chimie produite lors de l’incendie dans le hangar à \qty{300}{\degreeCelsius} :
  \begin{center}
    \important{2}\chemfig{NH_4 NO_3}\sol \reaction
    \important{2}\chemfig{N_2}\gaz + \dioxygene\gaz + \important{4}\eau\liq
  \end{center}

  \begin{multicols}{2}
    \begin{importants}
      Les espèces chimiques qui sont transformées au cours de la réaction chimique sont les \important{réactifs.}
      Les réactifs sont à gauche dans la réaction.
    \end{importants}
    \begin{importants}
      Les espèces chimiques qui sont produites au cours de la réaction chimique sont les \important{produits.}
      Les produits sont à droite dans la réaction.
    \end{importants}
  \end{multicols}
\end{doc}

%%
\begin{doc}{Faire un bilan de matière}{doc:A1_bilan_matiere}
  L'équation de la réaction est comme une recette de cuisine :
  \begin{center}
    \important{2}\chemfig{NH_4 NO_3}\sol \reaction
    \important{2}\chemfig{N_2}\gaz + \dioxygene\gaz + \important{4}\eau\liq
  \end{center}
  Si je mélange \important{deux} \chemfig{NH_4 NO_3}, il se forme \important{deux} \chemfig{N_2},
  \important{un} \dioxygene et \important{quatre} \eau.
  
  Si je mélange 4 \chemfig{NH_4 NO_3}, il se forme 4 \chemfig{N_2}, 2 \dioxygene et 8 \eau.
  
  Si je mélange 6 \chemfig{NH_4 NO_3}, il se forme \texteTrou{6} \chemfig{N_2}, \texteTrou{3} \dioxygene et \texteTrou{12} \eau.
  
  Si je mélange \qty{2,4}{\mole} de \chemfig{NH_4 NO_3}, il se forme
  \texteTrou{$\qty{2,4}{\mole}$} \chemfig{N_2},
  \texteTrou{$\qty{1,2}{\mole}$} \dioxygene et \\
  \texteTrou{$\qty{4,8}{\mole}$} \eau.
\end{doc}

\begin{donnees}
  \item $\qty{1}{\cubic\metre} = \qty{e3}{\litre}$
  \item $\qty{1}{\kelvin} = \qty{273}{\degreeCelsius}$
\end{donnees}

\question{
  Réécrire l’équation de la réaction produite lors de l’incendie.
  A partir du document~\ref{doc:A1_bilan_matiere}, nommer les réactifs et les produits de cette réaction chimique. 
  En vous aidant du document~\ref{doc:A1_descriptif_especes_chimiques}, indiquer si ces espèces sont dangereuses.
}{
  
}[4]

\numeroQuestion
La chaleur apportée par l’incendie a permis à la réaction de se produire.
Compléter la première ligne « \important{avant l'incendie} » et la deuxième ligne « \important{après l'incendie} », du tableau ci-dessous, en vous aidant du document~\ref{doc:A1_bilan_matiere}

\vspace*{8pt}
\begin{tblr}{
  vlines, hlines,
  colspec = {c X[l,m] X[l,m] X[l,m] X[l,m]},
  hline{1,2} = {1-5}{dashed}, vline{1,6} = {1}{dashed},
  vline{2} = {1}{text = \clap{:}}, vline{4,5} = {1}{text = \clap{+}},
  row{1} = {couleurPrim100, c}, row{3,4} = {l, 36pt}
}
  Équation de la réaction &
  \SetCell[c=2]{c} 2 \chemfig{NH_4 NO_3}\sol \reaction 2 \chemfig{N_2}\gaz \qq{}\qq{} & &
  \dioxygene\gaz &
  4 \eau\liq \\
  %
  État du système & \SetCell[c=4]{c} \important{Quantités de matières (\unit{\mole})} & & & \\
  %
  \important{Avant l’incendie} &
  $n_1 =$               \correction{\num{2}} &
  $n(\chemfig{N_2}) =$  \correction{\num{0}} &
  $n(\dioxygene) =$  \correction{\num{0}} &
  $n(\eau) =$ \correction{\num{0}} \\
  %
  \important{Après l’incendie} &
  $n_{f,1} =$             \correction{\num{0}} &
  $n_f(\chemfig{N_2}) =$  \correction{\num{}} &
  $n_f(\dioxygene) =$  \correction{\num{2}} &
  $n_f(\eau) =$ \correction{\num{2}}
\end{tblr}

\vspace*{8pt}
\question{
  En utilisant le document~\ref{doc:A1_calcul_mole_sol_gaz} et le tableau ci-dessus, calculer (dans les conditions normales), le volume de diazote $V(\chemfig{N_2})$,
  de dioxygène $V(\dioxygene)$ et de vapeur d’eau $V(\eau)$ produit.
}{

}[6]

\newpage
\question{
  Soit $n$ la quantité de matière produite totale avec 
  $n = n_f(\chemfig{N_2}) + n_f(\dioxygene) + n_f(\eau)$
  et $V$ le volume totale 
  $V = V(\chemfig{N_2}) + V(\dioxygene) + V(\eau)$.
  Calculer $n$ et $V$.
}{

}[6]

\question{
  Conclure sur la valeur de $V$ par rapport à celle du hangar
}{

}[4]

\question{
  \textit{Pour les plus rapide.}
  La relation des gaz parfait est la suivante :
  \begin{equation*}
    PV = nRT
  \end{equation*}
  avec $R = \qty{8,31}{\pascal\cubic\metre\per\mole\per\kelvin}$. \\
  Sachant que la température dans le hangar était de \qty{600}{\degreeCelsius} après la réaction, calculer la pression produite par la réaction.
  La comparer avec la pression atmosphérique $P_\text{atm} = \qty{100}{\kilo\pascal}$ et la pression dans un pneu de vélo $P_\text{pneu} = \qty{300}{\kilo\pascal}$.
}{

}[6]