%%%%
\teteTermStssBiom
\titreTP{Saponification d'une matière grasse}


%%%% objectifs
\begin{objectifs}
  \item Comprendre la réaction de saponification.
  \item Calculer le rendement d'une réaction de saponification.
\end{objectifs}

\begin{contexte}
  En ajoutant une base forte comme la soude dans une matière grasse, on peut transformer la matière grasse en savon, c'est la \important{saponification.} 
  
  \problematique{
    Quelle réaction décrit la formation d'un savon à partir d'une matière grasse ?
  }
\end{contexte}


%%%%
\begin{doc}{Réaction de saponification}{doc:A6_reaction_savon}
  \begin{importants}
    La \important{saponification} est l'hydrolyse d'un triglycéride en \important{milieu basique.}
    Cette réaction va transformer le triester en glycérol et en 3 ions carboxylate.
  \end{importants}
  \begin{center}
    \chemnameinit{\chemfig{!\oleineSemiDev}}
    \schemestart
    \chemname{
      \chemfig[atom sep = 1.75em]{!\oleineSemiDev}
    }{triglycéride}
    +
    \chemname{\hspace{14pt}3 \hydroxyde \hspace{10pt}}{ions hydroxydes}
    %
    \reaction
    %
    \chemname{\chemfig{!\glycerolSemiDev}}{glycérol}
    +
    \chemname{
      \hspace{10pt}3 \chemfig{!\oleateSemiDev} \hspace{8pt}
    }{ions carboxylates (savon)}
    \schemestop
  \end{center}

  \begin{importants}  
    Les ions carboxylate \chemfig{R-CO_2^{-}} sont \important{amphiphile} : ils possèdent une tête ionique hydrophile et une chaîne carbonée hydrophobe.
  \end{importants}

  Grâce à leur caractère amphiphile, les ions carboxylate vont venir se placer à l'interface entre les matière grasse et l'eau, ce qui permet d'enlever les résidus graisseux et en fait de bon détergents.
\end{doc}


\begin{doc}{Réalisation pratique}{doc:A5_realisation_savon}
  \begin{wrapfigure}{r}{0.4\linewidth}
    \centering
    \vspace*{-32pt}
    \image{0.9}{images/chimie/montage_reflux}
  \end{wrapfigure}
  En pratique on réalise du savon en mélangeant une \important{base forte} avec une \important{matière grasse} et en chauffant à reflux.
  Le \important{chauffage à reflux} à deux intérêts :
  \begin{listePoints}
    \item chauffer le milieu réactionnel pour accélérer la réaction de saponification ;
    \item ne pas perdre de matière en condensant les vapeurs qui s'échappent du ballon avec le réfrigérant.
  \end{listePoints}

  Le plus souvent on utilise de la soude \chemfig{NaOH} ou de la potasse \chemfig{KOH} comme base forte.

  D'après la réaction de saponification du document~\ref{doc:A6_reaction_savon}, pour 1 mole de triglycéride, on produira 3 moles de savon, soit
  \begin{equation*}
    n_\text{savon} = 3\times n_\text{triglycéride}
  \end{equation*}

  En pratique la réaction n'est pas totale et on perd un peu de matière avec le chauffage au cours de la réaction, on a donc un \important{rendement $\eta$} (« eta »)  inférieur à \qty{100}{\percent}.
  On calcule le rendement en divisant la quantité de matière obtenue sur celle attendue, ou de manière équivalente avec les masses :
  \begin{equation*}
    \eta = \dfrac{n_\text{savon obtenu}}{n_\text{savon théorique}}
    \qq{} \text{ou} \qq{}
    \eta = \dfrac{m_\text{savon obtenu}}{m_\text{savon théorique}}
  \end{equation*}
\end{doc}

\begin{doc}{Savon de Marseille et huile d'olive}{doc:A6_huile_olive}
  Le savon de Marseille est fabriquée à partir d'huile d'olive historiquement.
  
  On va considérer que l'huile d'olive est constituée à \qty{70}{\percent} en masse de trioléine, un acide gras composé de trois acides oléique.

  La masse molaire de la trioléine est $M = \qty{884}{\g\per\mole}$.
\end{doc}

\question{
  Expliquer pourquoi on utilise un chauffage à reflux, au lieu de laisser les vapeurs s'échapper du ballon.
}{}[3]

\question{
  On introduit \qty{10}{\g} d'huile d'olive dans un ballon.
  Calculer la masse de trioléine $m_o$ introduite dans le ballon.
}{}[3]

\question{
  Calculer la quantité de matière de trioléine $n_o$ introduite dans le ballon.
}{}[3]

\question{
  Calculer la quantité de matière de savon $n_\text{savon}$ que l'on s'attendrait à produire avec la réaction de saponification de la trioléine.
}{}[3]

\question{
  Expérimentalement, on obtient une quantité de matière de savon $n_\text{exp} = \qty{1,9e-2}{\mol}$.
  Calculer le rendement $\eta$ de cette réaction.
}{}[2]

\question{
  Proposer une hypothèse qui expliquerait pourquoi le rendement n'est pas de \qty{100}{\percent}.
}{}[2]