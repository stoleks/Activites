%%%%
\teteTermStssBiom

%%%% titre
\vspace*{-40pt}
\numeroActivite{1}
\titreActivite{Structure des acides $\mathbf{\alpha}$-aminés}

%%%% objectifs
\begin{objectifs}
  \item Connaître la structure des acides aminés.
  \item Comprendre la notion de molécule énantiomère et de chiralité.
  \item Comprendre les différentes représentations des acides aminés.
  \item Comprendre le principe de la liaison peptidique.
\end{objectifs}

\begin{contexte}
  Un \important{acide aminé} est une molécule organique comportant à la fois une fonction acide carboxylique \chemfig{COOH} et une fonction amine \chemfig{NH_2}.

  Les acides $\alpha$-aminés sont les briques de bases des \important{protéines,} qui permettent à nos molécules de fonctionner. 
  
  \problematique{
    Quel est la structure des acides aminés et comment les représenter ?
  }
\end{contexte}


%%%% docs
\begin{doc}{Les acides aminé}{doc:A1_acide_amine}
  \begin{wrapfigure}[3]{r}{0.3\linewidth}
    \centering
    \vspace*{-24pt}
    \chemname{
      \chemfig{R- C^{\alpha} !\branche{NH_2}{H} -C !\carboxyleDev}
    }{
      Exemple d'acide $\alpha$-aminé
    }
  \end{wrapfigure}
  Un acide aminé est une molécule organique comportant un groupe carboxyle et un groupe amine.

  \begin{importants}
    On parle \important{d'acide $\alpha$-aminé,} si les groupes \important{amine} et \important{carboxyle} sont porté par le même carbone, numéroté $\alpha$.
  \end{importants}
  
  \begin{wrapfigure}{l}{0.3\linewidth}
    \centering
    \vspace*{-10pt}
    \chemname{
      \chemfig{H_3C -C^{*} !\branche{NH_2}{H} -C !\carboxyleDev}
    }{Molécule d'alanine}
  \end{wrapfigure}

  Ici \important{R} est une chaîne d'éléments appelée \important{résidu.}
  
  \begin{importants}  
    Un \important{carbone asymétrique} est un carbone avec 4 liaisons simples, lié à quatre élément ou quatre groupements différents.
    On le note \chemfig{C^{*}}.
  \end{importants}

  Repérer les carbones asymétriques d'une molécule permet de déterminer si elle est \important{chirale.}
\end{doc}

\question{
  Justifier que le carbone en $\alpha$ de l'alanine est bien asymétrique.
}{}[2]

\begin{doc}{Chiralité et énantiomère}{doc:A1_chiral_enantiomere}
  \begin{wrapfigure}{r}{0.5\linewidth}
    % \centering
    \vspace*{-18pt}
    \image{1}{images/molecules/biomolecules/alanine_enantiomere_persp}
    \legende{
      Molécules d'alanine et son image dans un miroir. Ces deux molécules ne sont pas superposables. Elles sont \important{énantiomères.}
    }
  \end{wrapfigure}
  \strut\vspace*{-18pt}
  
  \begin{importants}
    Une \important{molécule} est \important{chirale} si elle n'est pas superposable à son image dans un miroir.
    Une molécule est chirale si elle possède \important{au moins un carbone asymétrique.}
  \end{importants}

  \begin{importants}
    Si deux molécules sont images l'une de l'autre dans un miroir et ne sont pas superposable, alors ce sont des \important{énantiomères.}
  \end{importants}
  On parle alors \important{d'isomérie de configuration.}
\end{doc}

\newpage
\vspace*{-20pt}
\question{
  Donner des exemples d'objet chiraux dans la vie quotidienne.
}{}[2]


\begin{doc}{Acides $\alpha$-aminé produit par le vivant}{doc:A1_acide_amine_vivant}
  Sur Terre, plus de 500 acides $\alpha$-aminés sont produit naturellement,
  mais chez les eucaryotes, seuls 20 acides $\alpha$-aminés sont utilisés et synthétisés.
  On parle d'acide $\alpha$-aminé \important{protéinogène} (« qui donne naissance aux protéines »). 
  Les êtres humains peuvent synthétiser \important{11 acides aminés.}
  \begin{importants}
    Les 9 acides $\alpha$-aminés qui ne peuvent pas être synthétisé dans nos corps sont les acides aminés \important{essentiels.}
  \end{importants}

  \centering
  \begin{tblr}{
    columns = {c, m}, hlines, vlines,
    row{1} = {couleurSec-100}, row{2} = {couleurSec-50}
  }
    Isoleucine & Leucine & Méthionine & Valine \\
    %
    Ile & Leu & Met & Val \\
    %
    \chemfig{!\isoleucine} &
    \chemfig{!\leucine}    &
    \chemfig{!\methionine} &
    \chemfig{!\valine} \\   
    %
  \end{tblr}
  
  \legende{Quelques acides aminés essentiels}
\end{doc}


\numeroQuestion
Entourer les carbones asymétriques dans les quatre exemples d'acide aminés essentiels donnés dans le document~\ref{doc:A1_acide_amine_vivant}.

\question{
  La molécule de glycine est le seul acide $\alpha$-aminé qui \important{n'est pas} chiral.
  \begin{center}
    Glycine : \chemfig{H-C !\branche{NH_2}{H} -C !\carboxyleDev}
  \end{center}
  \vspace*{-4pt}
  Expliquer pourquoi.
}{}[1]


%%%%
\begin{doc}{Représentation des acide aminés}{doc:A1_representations}
  
  Pour représenter en 2D des molécules 3D, on utilise la représentation de Cram en chimie et la représentation de Fischer en biologie.

  \setchemfig{cram width = 5pt}
  \begin{wrapfigure}[3]{r}{0.25\linewidth}
    \centering
    \chemfig{C !\cram{H}{H} (-[::90] COOH) -[::-30] NH_2}
  \end{wrapfigure}
  \phantom{b}\vspace*{-12pt}

  \begin{importants}
    \important{Représentation de Cram :} c'est une représentation en perspective avec trois conventions
    \begin{listePoints}
      \item \chemfig{-} liaison dans le plan ;
      \item \chemfig{>:} liaison en arrière du plan (qui s'éloigne de nous) ;
      \item \chemfig{>} liaison en avant du plan (qui s'approche de nous).
    \end{listePoints}
  \end{importants}
  Cette représentation permet de distinguer deux formes énantiomères.

  \begin{importants}
    \important{Représentation de Fischer :} la molécule d'acide aminé est projetée dans le plan et représentée sous formes de croix, comme si on l'avait aplati
    \begin{listePoints}
      \item l'atome de carbone \chemfig{C^{*}} asymétrique est celui sur lequel est centrée la représentation ;
      \item le groupe carboxyle \chemfig{COOH} est placé au dessus du carbone asymétrique ;
      \item le résidu \chemfig{R} est placé en dessous du carbone asymétrique ;
      \item \chemfig{H} et le groupe amine \chemfig{NH_2} sont placés horizontalement, à droite ou à gauche du carbone asymétrique.
    \end{listePoints}
  \end{importants}
  
  \begin{wrapfigure}[2]{r}{0.4\linewidth}
    \centering
    \chemfig{C !\cram{H_2N}{H} (-[::90] !\couleur{COOH}) -[::-30] !\couleur{R}}
    \reaction
    \chemfig{H_2N- (-[::90] !\couleur{COOH}) (-[::-90] !\couleur{R}) -H}
  \end{wrapfigure}
  \chemfigParDefaut
  
  Il y a deux positions possibles pour le groupe amine.
  \begin{importants}
    \begin{listePoints}
      \item Si le groupe amine est à \important{gauche}, l'énantiomère est dit \important{énantiomère L.}
      \item Si le groupe amine est à \important{droite}, l'énantiomère est dit \important{énantiomère D.}
    \end{listePoints}
  \end{importants}

  Dans le vivant, seuls les acides aminés de configuration L existent.
  Le nom des acides aminés sont précédés de la lettre L ou D.
\end{doc}

\numeroQuestion
Dans le vivant on trouve de la L-valine.
Donner la représentation de Fischer de la D-valine.
\smallskip

\begin{minipage}[T]{0.48\linewidth}
  \centering
  \chemname{ \chemfig{!\valineL} }{L-valine}
\end{minipage}
\smallskip


%%%% Liaisons peptidiques
\begin{doc}{Liaison peptidique}{doc:A1_liaison_peptidique}
  \begin{wrapfigure}[3]{r}{0.3\linewidth}
    \centering
    \chemfig{- C (=[::90] O) -N (-[::-90] H) -}

    \legende{Liaison peptidique}
  \end{wrapfigure}
  
  Pour former une protéine, il faut assembler des acides aminés entre eux avec des \important{liaisons peptidiques.}

  \begin{importants}
    La \important{liaison peptidique} est un groupe amide particulier.
    L'azote du groupe amide est monosubstitué, c'est-à-dire qu'il n'est relié qu'à un seul H.
  \end{importants}

  Le groupe amide se forme au cours d'une réaction de \important{condensation} entre un acide carboxylique et un amine
  \vspace*{-4pt}
  
  \begin{center}
    \chemfig{R- C !\carboxyleDev} +   
    \chemfig{H-N (-[::-90] H) -R'}
    \reaction
    \chemfig{R- C (=[::90] O) -N (-[::-90] H) -R'} +
    \eau
  \end{center}
  \vspace*{-4pt}

  Comme tous les acides aminés possèdent un \important{groupe amine} et un \important{groupe carboxyle}, cette réaction de condensation peut avoir lieue avec n'importe quelle paire d'acides aminés.

  \begin{importants}
    Un \important{dipeptide} est la molécule formée par deux acides aminées liés par une liaison peptidique.
    
    Pour nommer les \important{dipeptides} obtenus par réaction de condensation, on colle les abréviations des 2 acides aminés.
  \end{importants}

  Dans un mélange \important{équimolaire} d'alanine et de valine, 4 dipeptides vont être formés, car chaque groupe amine peut réagir avec chaque groupe carboxyle :
  Ala-Val, Ala-Ala, Val-Ala et Val-Val.

  \begin{importants}  
    À partir des dipeptides, on peut former des tri-, quadri-, etc. peptides.
    Quand la chaîne peptidique atteint un certain nombre d'acides aminés (plus d'une cinquantaine), on parle de \important{polypeptide} et si ce polypeptide remplit une fonction biologique, c'est une \important{protéine.}
  \end{importants}
\end{doc}

\numeroQuestion
Donner les formules topologiques de l'alanine et de la valine.
\correction{
  \centering
  \chemfig{!\valine}  \qq{} \chemfig{!\alanine}
}
\pasCorrection{\vspace*{4cm}}

\numeroQuestion
Donner les formules topologiques des dipeptides Val-Ala, Met-Ala et Val-Leu.
\vfill

\begin{doc}{Acide $\mathbf{\alpha}$-aminés et alimentation}{doc:A1_alpha_amine_alimentation}
  \begin{importants}
    Les 9 acides $\alpha$-aminés qui ne peuvent pas être synthétisé dans nos corps sont les acides aminés \important{essentiels.}
    Ils doivent être fournis par l'alimentation, en mangeant des protéines végétales ou animales.
  \end{importants}
  
  On parle de \important{protéines complètes} si l'aliment contient tous les acide $\alpha$-aminé essentiels, comme le poulet, le saumon, le tempeh ou le tofu.
  La plupart des végétaux et certains produits animaux sont des \important{protéines incomplètes,} il faut donc les combiner pour avoir tous les acides $\alpha$-aminés nécessaires aux bon fonctionnement de notre corps.
\end{doc}
