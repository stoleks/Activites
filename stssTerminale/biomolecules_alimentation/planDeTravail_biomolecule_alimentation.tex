\teteTermStssBiom

\vspace*{-40pt}
\titre{Plan de Travail -- \termStssBiom}
\vspace*{-8pt}

%\begin{importants}
  % Le plan de travail est un cadre de travail collectif où tu as la liberté d'avancer, seul-e ou en groupe, à ton rythme.
  Ce document présente les activités et travaux pratiques à réaliser pendant les 4,5 semaines du chapitre.
  À chaque séance (en classe entière ou demi-groupe), tu es libre de choisir quelle activité ou TP réaliser avec ton groupe.
  Tous les documents sont imprimés sur le bureau du professeur.
  % Au début de la 2ème et 3ème semaine, une courte interrogation sera réalisé sur certaines activités.
%\end{importants}


%%%% Activités
\titre{Activités à réaliser}
\vspace*{-20pt}

\begin{multicols}{2}
  \begin{activite}{Structure des acides $\mathbf{\alpha}$-aminés}[3 h]{acides_amines}
    \begin{prerequis}
      \item Identifier les groupes carboxyles et amines.
    \end{prerequis}
    \begin{objectifs}  
      \item Voir la structure des acides aminés.
      \item Apprendre la notion de chiralité.
      \item Les représentations de Cram et Fischer.
      \item Voir la liaison peptidique.
    \end{objectifs}
  \end{activite}

  \begin{activite}{Structure des pro-téines}{structure_proteine}
    \begin{prerequis}
      \item Structure des acides $\alpha$-aminé.
    \end{prerequis}
    \begin{objectifs}
      \item Voir la structure 3D des protéines.
      \item Voir les actions biologiques des protéines.
    \end{objectifs}
  \end{activite}

  \setcounter{activiteNum}{4}
  \begin{activite}{Additifs alimentaires et arômes}{additifs_aromes}
    \begin{objectifs}
      \item Comprendre l'intérêt et les défauts des additifs alimentaires.
    \end{objectifs}
  \end{activite}

  \setcounter{activiteNum}{2}
  \begin{activite}{Les vitamines}{vitamines}
    \begin{objectifs}
      \item Définir « hydro- » et « liposoluble ».
      \item Relier les besoins journaliers avec leur solubilité.
      \item Analyser la structure des vitamines A, C et D pour déterminer leur solubilité.
    \end{objectifs}
  \end{activite}
  
  \begin{activite}{Lipides et alimen-tation}{lipides_alimentation}
    \begin{prerequis}
      \item Reconnaître une molécule hydrosoluble et liposoluble.
    \end{prerequis}
    \begin{objectifs}
      \item Revoir la structure des acides gras et des triglycérides.
      \item Comprendre l'impact du cholestérol et des lipoprotéines sur l'organisme.
    \end{objectifs}
  \end{activite}
  
  \begin{TP}{Saponification d'une matière grasse}[2h]{saponification}
    \begin{prerequis}
      \item Structure d'un triglycéride.
      \item Hydrolyse d'un triglycéride.
    \end{prerequis}
    \begin{objectifs}
      \item Voir la réaction de saponification.
      \item Calculer le rendement d'une saponification.
    \end{objectifs}
  \end{TP}  
\end{multicols}


\vspace*{-2cm}
\begin{tikzpicture}[
  overlay, remember picture, line width=1.5mm, draw=couleurQuat
]
    \draw[->] (acides_amines) to (structure_proteine);
\end{tikzpicture}



%%%% Progression
\newpage
\nomPrenomClasse
\titre{Progression des activités}
\vspace*{12pt}

\flecheProgression{3}
\vspace*{-353pt}

\begin{programmeSeance}
  \seance{2 h}{}
  \seance{1 h}{}
  \seance{2 h}{}
\end{programmeSeance}

\begin{programmeSeance}
  \seance{1 h}{}
  \seance{2 h}{}
  \seance{1 h}{}
\end{programmeSeance}

\begin{programmeSeance}[2](0)
  \seance{2 h}{ \important{Tâche finale} }
  \seance{1 h}{ Présentation orale du poster réalisé. }
\end{programmeSeance}


%%%% Tâche finale
\begin{tacheFinale}
  Préparer une affiche par \important{groupe de 4} sur un type de biomolécules (lipide, acide $\alpha$-aminé, protéine, vitamine), qui présente de manière synthétique sa structure générale (s'il y en a une), quelques propriétés chimiques, quelques propriétés biologiques et comment avoir une alimentation saine du point de vue de cette biomolécule.
\end{tacheFinale}


%%%% Evaluation
\titre{Évaluation de l'autonomie}

\important{Les différents degrés d'autonomie}

\begin{enumerate}[label = \Alph*]
  \item Je planifie librement mon apprentissage, je coopère avec mes camarades et je sollicite de l'aide pour valider les travaux réalisés.
  \item Je travaille seul-e ou avec mes camarades à partir des documents et je sollicite régulièrement de l'aide pour avancer.
  \item J'avance uniquement quand le professeur est là pour m'aider, je n'arrive pas à planifier mon travail ou je ne fais que recopier les réponses d'un-e de mes camarades.
  \item J'utilise des stratégies pour éviter d'apprendre et je refuse d'essayer de faire les activités.
\end{enumerate}

\begin{tableauCompetences}
  AUTO & Travailler de manière autonome \\
\end{tableauCompetences}