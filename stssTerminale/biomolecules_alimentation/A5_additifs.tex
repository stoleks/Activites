%%%%
\teteTermStssBiom

%%%% titre
\numeroActivite{5}
\titreActivite{Additifs alimentaires et arômes}


%%%% objectifs
\smallskip
\begin{objectifs}
  \item Comprendre l'intérêt et les défauts des additifs alimentaires.
\end{objectifs}

\begin{contexte}
  Les additifs alimentaires sont ajoutés dans les aliments industriels pour les améliorer (arôme, texture, couleur, etc.) ou pour mieux les conserver.
  Ils sont utilisés massivement par l'industrie agro-alimentaire pour créer des produits appétissants et qui se conservent longtemps.
  
  \problematique{
    Quelles sont les différents types d'additifs alimentaires ?
  }
\end{contexte}
\smallskip


%%%%
\begin{doc}{Les additifs alimentaires}{doc:A6_additifs}
  \begin{wrapfigure}[21]{r}{0.5\linewidth}
    \vspace*{-32pt}
    \centering
    \begin{tblr}{
      colspec = {l X[l] X[1.28,l]}, hlines, vlines,
      column{1} = {couleurSec-50},
      row{1} = {c, couleurSec-100},
      %width = 0.75\linewidth,
    }
      Additif & Nom & Propriétés \\
      E100    & Curcumine & Colorant, pigment du curcuma \\
      E220-8  & Sulfites & Conservateur. DJA = \qty{0,7}{\mg\per\kg} \\
      E320    & BHA & Antioxydant. DJA = \qty{1}{\mg\per\kg} \\
      E330    & Acide citrique & Antioxydant \\
      E450i   & Diphosphate disodique & Émulsifiant. DJA = \qty{70}{\mg\per\kg} \\
      E461    & Méthylcellulose & Stabilisant, épaississant \\
      E471    & Mono et diglycérides d'acide gras & Émulsifiant \\
      E1403   & Amidon transformé & Stabilisant, émulsifiant, liant, épaississant \\
    \end{tblr}
    \smallskip
    
    \legende{Quelques additifs alimentaires.}
  \end{wrapfigure}
  
  Les \important{additifs alimentaires} sont des produits ajoutés volontairement et en faible quantité dans un aliment pour remplir une fonction précise.
  En Europe, il existe une liste d'additifs alimentaires autorisés et ils doivent obligatoirement être indiqué sur les étiquettes des aliments qui en contiennent.

  Les additifs autorisés sont codé avec 4 symboles qui commencent par « E » en Europe et dont le chiffre suivant indique le type d'additifs.
  Ils se décomposent en 4 grands groupes :
  \begin{listePoints}
    \item les \important{colorants} codés \important{E1} ;
    \item les \important{conservateurs} codés \important{E2} ;
    \item les \important{antioxydants} codés \important{E3} ;
    \item les \important{agents de texture} codés \important{E4}.
  \end{listePoints}
  Comme la liste des additifs autorisés est régulièrement mise à jour, certains additifs ne respecte pas ce code.

  Certains additifs possèdent une Dose Journalière Admissible (DJA) ou une Dose Journalière Tolérable (DJT), il faut donc faire attention à ne pas consommer trop de produits qui en contiennent.
  Il est important de noter qu'un aliment vendu ne peut pas dépasser la DJA ou la DJT des additifs qu'il contient, mais la quantité d'additifs n'est pas indiqué sur l'étiquette.
  Il est donc difficile de savoir si on dépasse la DJA ou la DJT d'un additifs en mangeant plusieurs aliment qui en contiennent...
\end{doc}


\begin{doc}{L'arôme de vanille}{doc:A6_aromes}
  Les arômes naturels sont un mélange de molécules aromatiques.
  C'est ce mélange de molécules qui donnent aux aliments leur goût et leur odeur, même si une molécule est souvent majoritaire dans le mélange.
  Pour la vanille, l'arôme majoritaire est la \important{vanilline}. 
  On peut donc donner à un aliment un arôme similaire à la vanille en y ajoutant de la vanilline.

  La production mondiale de vanille ne suit pas les demandes des industriels alimentaires : on extrait à peine 50 tonnes de vanillines naturelles, alors qu'on en consomme plusieurs milliers de tonnes.
  Pour répondre à cette demande, on peut synthétiser la vanilline de manière artificielle,
  ou on peut synthétiser une molécule similaire, mais avec un pouvoir aromatisant 5 fois plus puissant : \important{l'éthylvanilline.}

  \begin{center}
    \schemestart
    \chemname{
      \chemfig{
        O=[-1] (-[1] H) -[-3]
        *6(-=-(-OH) = (-O-[1]) -=)
      }
    }{Vanilline,\\ extraite depuis 1854, synthétisée depuis 1874}
    \hspace{80pt}
    \chemname{
      \chemfig{
        O=[-1] (-[1] H) -[-3]
        *6(-=-(-OH) = (-O-[1]-[-1]) -=)
      }
    }{Éthylvanilline,\\ synthétisée depuis 1930}
    \schemestop
  \end{center}
  
  \begin{importants}
    On dit que l'éthylvanilline est un \important{arôme artificiel,} cette molécule donne les même sensation que la molécule naturelle de vanilline.

    La vanilline synthétisée est un \important{arôme synthétique,} elle ne vient pas d'une plante, mais la molécule est identique à celle trouvée naturellement.
  \end{importants}

  On notera qu'utiliser un arôme synthétique peut être plus écologique que d'utiliser sa version naturelle, si l'exploitation des ressources naturelles est trop intense et si la synthèse est peu coûteuse en matières premières.

  \medskip
  \centering
  \begin{tblr}{
    colspec = {c c c}, hlines, vlines,
    column{1} = {couleurSec-50},
    row{1} = {couleurSec-100},
  }
    Arôme & Origine & Coût \\
    \SetCell[r=2]{c} Vanilline & Naturelle & $\sim$\num{12500} euro le \unit{\kg} \\
    %
    & Synthétique & $\sim$\num{10} euro le \unit{\kg} \\
    %
    Éthylvanilline & Artificielle & $\sim$\num{10} euro le \unit{\kg} \\
  \end{tblr}
\end{doc}

\question{
  Entourer et nommer les fonctions organiques présentes dans les molécules de vanilline et d'éthylvanilline.
}{}{3}

\question{
  Donner au moins 2 raisons pour préférer l'utilisation d'arômes synthétiques.
}{}{3}

\question{
  Expliquer pourquoi l'éthylvanilline est moins cher à l'usage que la vanilline synthétique.
}{}{3}