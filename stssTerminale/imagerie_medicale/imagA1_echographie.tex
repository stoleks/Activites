%%%%
\teteTermStssImag

%%%% titre
\numeroActivite{1}
\titreActivite{Principe d'une échographie}


%%%% objectifs
\begin{objectifs}
  \item Comprendre le principe d'une échographie.
\end{objectifs}

\begin{contexte}
  Pendant les grossesses on peut visualiser l'embryon à l'aide d'une \important{échographie,} ce qui permet de vérifier son bon développement.
  
  \problematique{
    Comment fonctionne une échographie ?
  }
\end{contexte}


%%%% docs
\begin{doc}{Fréquence de propagation et réflexion des ultrasons}{doc:A2_propriete_ultrason}
  Les ultrasons sont des ondes sonores inaudibles par une oreille humaine.
  La fréquence d'un ultrason est supérieure à \qty{20000}{\hertz}.

  Les ultrasons sont des ondes mécaniques et donc
  \begin{listePoints}
    \item elles ne se propagent pas dans le vide ;
    \item la vitesse des ultrasons dépend du milieu traversé ;
    \item elles peuvent être réfléchie sur un obstacle (écho).
  \end{listePoints}

  \begin{tableau}{|l |c |c |c |}
    \SetCell[c=4]{c} Vitesse de propagation d'un ultrason dans un milieu & & & \\
    Matériau & Air à \qty{20}{\degreeCelsius} & Eau à \qty{37}{\degreeCelsius} & Sang \\
    Vitesse de propagation & \qty{340}{\m\per\s} & \qty{1530}{\m\per\s} & \qty{1560}{\m\per\s} \\
  \end{tableau}

  Dans un corps humain, la célérité moyenne d'un ultrason est $c = \qty{1540}{\m\per\s}$.
\end{doc}

%%
\begin{doc}{L'échographie}{doc:A2_echographie}
  L'échographie utilise le phénomène d'écho, comme son nom l'indique.
  Une sonde est posée sur la peau recouverte d'un gel et émet des ultrasons.
  Les ultrasons se propagent dans le corps et sont réfléchis quand ils rencontrent un changement de milieu.
  Par exemple, pendant le passage d'un tissus musculaire à un tissus osseux.

  Après réflexion, les ultrasons sont reçues par la sonde.
  La durée $\delta t$ qui sépare l'émission et la réception des ultrasons est alors mesurée et permet de calculer des distances.

  En comparant les durée de réception et avec un traitement numérique, on peut obtenir une image contrastée des tissus biologiques dans le corps humain.

  \separationBlocs{
    \centering
    \image{0.9}{images/acoustique/echographie}
    
    \faArrowUp\; Schéma d'une échographie.
  }{
    \centering
    \image{0.9}{images/acoustique/reception_ultrason}
    
    \faArrowUp\; Ultrasons reçus lors d'une échographie.
  }
\end{doc}

\begin{doc}{Surveillance d'une grossesse avec l'échographie}{doc:A2_surveillance_grossesse}
  Afin de suivre la croissance du foetus, une surveillance est réalisée par échographie.
  Elle permet d’effectuer différentes mesures, notamment celle du diamètre bipariétal BIP
  (largeur de la tête entre les deux oreilles),
  qui fournit de précieuses informations sur le développement cérébral du foetus.
  
  On note A la position de la première oreille et B la position de la seconde oreille.
\end{doc}


%%%%
\question{
  Exprimer la distance $d_A$ entre la sonde et la première oreille,
  en fonction de la célérité des ultrasons $c$
  et du temps de détection des ultrasons réfléchis $\Delta t_A$.
}{
}{3}

\question{
  Exprimer de même pour la distance $d_B$ entre la sonde et la seconde oreille,
  en fonction de la célérité des ultrasons $c$
  et du temps de détection des ultrasons réfléchis $\Delta t_B$.
}{
}{3}

\question{
  Calculer les valeurs de $d_A$ et $d_B$. 
  En déduire le diamètre bipariétal.

  \textbf{Données :} $\Delta t_A = \qty{120}{\micro\s}$, $\Delta t_B = \qty{185}{\micro\s}$, $\qty{1}{\micro\s} = \qty{e-6}{\s}$.
}{}{5}

\question{
  La patiente est examinée lors de la 21ème semaine d’aménorrhée.
  Les valeurs normales du diamètre bipariétal se situent alors entre
  \qty{46}{\milli\m} et \qty{57}{\milli\m}.
  Indiquer si l’examen permet de suspecter un retard de croissance du foetus.
}{}{5}
