%%%%
\teteTermStssDosa

%%%% titre
\vspace*{-32pt}
\titreTP{Préparer une solution de glucose}

%%%% objectifs
\begin{objectifs}
  \item Revoir le principe d'une dissolution et d'une dilution
  \item Réaliser une dissolution et une dilution.
\end{objectifs}

\begin{contexte}
  Les patient-es en hypoglycémie ont besoin d'un apport en glucose contrôlé.
  Pour ça, à l’hôpital on prépare des solution avec une concentration en glucose précise.
  
  \problematique{
    Comment préparer une solution avec une concentration donnée en réalisant une dissolution ou une dilution ?
  }
\end{contexte}


%%%%
\begin{doc}{Solution glucosée}{doc:TP1_solution_glucose}
  Une solution de glucose à \qty{5}{\percent} signifie qu'elle contient \qty{5,0}{\g} de glucose pour \qty{100}{\g} de solution, soit \qty{100}{\ml} de solution aqueuse.
\end{doc}
    
\begin{doc}{Protocole de dissolution dans le désordre}{doc:TP1_protocole_disso}
  \vspace*{-20pt}
  \begin{tblr}{
    colspec = {X[c] X[c] X[c] X[c] X[c] X[c]},
  }
    \image{1}{images/chimie/protocoles/dissoDilu0002.png} &
    \image{1}{images/chimie/protocoles/dissoDilu0004.png} &
    \image{1}{images/chimie/protocoles/dissoDilu0001.png} &
    \image{1}{images/chimie/protocoles/dissoDilu0005.png} &
    \image{1}{images/chimie/protocoles/dissoDilu0003.png} &
    \image{1}{images/chimie/protocoles/dissoDilu0006.png} \\
    \nombreCercle{1} Verser le solide dans la fiole jaugée &
    \nombreCercle{2} Agiter la fiole pour dissoudre le solide &
    \nombreCercle{3} Peser le glucose en poudre &
    \nombreCercle{4} Remplir avec de l'eau distillée jusqu'au trait de jauge &
    \nombreCercle{5} Verser de l'eau distillée jusqu'au 2/3 de la fiole&
    \nombreCercle{6} Agiter de nouveau pour homogénéiser \\
  \end{tblr}
\end{doc}

\begin{doc}{Protocole de dilution dans le désordre}{doc:TP1_protocole_dilu}
  \vspace*{-20pt}
  \begin{tblr}{
    colspec = {X[c] X[c] X[c] X[c] X[c] X[c]},
  }
    %
    \image{1}{images/chimie/protocoles/dissoDilu0010.png} &
    \image{1}{images/chimie/protocoles/dissoDilu0007.png} &
    \image{1}{images/chimie/protocoles/dissoDilu0009.png} &
    \image{1}{images/chimie/protocoles/dissoDilu0012.png} &
    \image{1}{images/chimie/protocoles/dissoDilu0008.png} &
    \image{1}{images/chimie/protocoles/dissoDilu0011.png} \\
    \nombreCercle{1} Agiter la fiole pour homogénéiser &
    \nombreCercle{2} Prélever la solution mère à diluer &
    \nombreCercle{3} Verser de l'eau distillée jusqu'au 2/3 &
    \nombreCercle{4} Agiter de nouveau pour homogénéiser &
    \nombreCercle{5} Verser la solution mère dans la fiole jaugée  &
    \nombreCercle{6} Remplir avec de l'eau distillée jusqu'au trait de jauge \\
  \end{tblr}
\end{doc}

\begin{doc}{Facteur de dilution}{doc:TP1_facteur_dilution}
  Le \important{facteur de dilution} est le rapport du volume de la solution fille sur le volume de la solution mère prélevée
  \begin{equation*}
    F = \frac{V_\text{1}}{V_\text{0}}.
  \end{equation*}

  C'est aussi le rapport de la concentration mère sur la concentration fille
  \begin{equation*}
    F = \frac{c_0}{c_1}. 
  \end{equation*}
  
  On dit qu'on a dilué $F$ fois une solution.
\end{doc}

%%
\question{
  Il faut préparer une solution de glucose de \qty{50}{\ml} à \qty{15}{\percent} pour une patiente hypoglycémique.
  Calculer la masse de glucose en poudre à prélever pour la solution.
}{}[4]

\question{
  Remettre dans l'ordre les étapes du protocole de dissolution.
}{}[2]

\mesure Réaliser la solution de glucose à \qty{15}{\percent}.

\question{
  Pour un autre patient, il faut réaliser une solution de \qty{50}{\ml}, 10 fois moins concentrée que la première.
  Calculer le volume à prélever de la première solution (solution mère) afin de réaliser une deuxième solution de \qty{50}{\ml} (solution fille).
}{}[4]

\question{
  Calculer la concentration en glucose de la deuxième solution.
}{}[4]

\question{
  Remettre dans l'ordre les étapes du protocole de dilution.
}{}[2]

\mesure Réaliser la deuxième solution de glucose à l'aide d'une dilution.