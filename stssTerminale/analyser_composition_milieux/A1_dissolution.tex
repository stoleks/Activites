%%%%
\teteTermStssDosa

%%%% titre
\numeroActivite{1}
\titreActivite{Dissolution et concentration des ions en solution}


%%%% objectifs
\begin{objectifs}
  \item Comprendre le lien entre un composé ionique et les ions qui le constitue.
  \item Savoir calculer la concentration d'un ion à partir de la concentration d'un composé ionique.
\end{objectifs}

\begin{contexte}
  De nombreuses analyses médicales repose sur le fait de mesurer la concentration d'espèces dissoutes dans le sang.
  
  \problematique{
    Quels sont les principes chimiques qui permettent d'étudier la composition ionique d'une solution ?
  }
\end{contexte}


%%%% docs
\begin{doc}{Neutralité des solutions}{doc:A1_neutralite_solution}
  \begin{wrapfigure}{l}{0.4\linewidth}
    \vspace*{-18pt}
    \image{1}{images/chimie/dissolution}
  \end{wrapfigure}
  Une solution est \important{électriquement neutre,} que le soluté soit moléculaire ou neutre !

  Une molécule est composée d'atomes électriquement neutres, un soluté moléculaire ne possède donc pas de charge électrique.
  
  Un ion est électriquement chargé. Pour qu'une solution ionique soit électriquement neutre, il faut une proportion d'anions (ions négatifs) et de cations (ions positifs) conduisant à une charge électrique totale nulle

  \textcolor{couleurPrim}{\faLongArrowLeft} Dissolution du chlorure de sodium
\end{doc}

%%
\begin{doc}{Équation de dissolution et calcul de concentration}{doc:A1_equation_dissolution}
  Pour déterminer la concentration \important{des espèces ioniques} dissoutes à partir de la concentration \important{en soluté,} on utilise la concentration en quantité de matière (\unit{\mole\per\litre}).
  \medskip

  \exemple Pour la dissolution du chlorure d'aluminium, une entité de chlorure d'aluminium \chemfig{Al Cl_3} produit 1 ion aluminium III \chemfig{Al^{3+}} et 3 ions chlorure \chemfig{Cl^{-}}.

  Donc, si on a une concentration en chlorure d'aluminium de \qty{2,0}{\mole\per\litre}, on aura une concentration de \qty{2,0}{\mole\per\litre} d'ion aluminium III et une concentration de $3\times\qty{2,0}{\mole\per\litre} = \qty{6,0}{\mole\per\litre}$ d'ion chlorure.
\end{doc}

%%
\begin{doc}{Quelques composés ioniques}{doc:A1_composes_ioniques}
  \begin{tableau}{|c |c |c |c |}
    Composé ionique      & Formule brute & Cation & Anion \\
    %
    Sulfate de magnésium & \chemfig{MgSO_4} &
    Ion magnésium \chemfig{Mg^{2+}} &
    Ion sulfate \chemfig{SO_4^{2-}} \\
    %
    Chlorure de sodium & \chemfig{NaCl} &
    Ion sodium \chemfig{Na^{+}} &
    Ion chlorure \chemfig{Cl^{-}} \\
    %
    Hydroxyde de sodium & \chemfig{NaOH} &
    Ion sodium \chemfig{Na^{+}} &
    Ion hydroxyde \chemfig{HO^{-}} \\
    %
    Chlorure de fer II & \chemfig{FeCl_2} &
    Ion fer II \chemfig{Fe^{2+}} &
    Ion chlorure \chemfig{Cl^{-}} \\
  \end{tableau}
\end{doc}


%%%%
\question{
  Écrire l'équation de dissolution du chlorure de fer \chemfig{Fe Cl_2}.
}{
}{2}

\newpage
\question{
  \qty{6,0}{\g} de chlorure de fer sont dissous dans \qty{1,0}{\litre} d'eau.
  Calculer la concentration en masse $c_m$ de chlorure de fer dans la solution. 
}{}{2}

\question{
  Calculer la concentration en quantité de matière en utilisant la relation suivante 
  \begin{equation*}  
    c = \dfrac{c_m}{M(\chemfig{FeCl_2})}
  \end{equation*}
}{}{4}

\begin{donnees}
  \item $M(\chemfig{Fe}) = \qty{55,8}{\g\per\mole}$
  \item $M(\chemfig{Cl}) = \qty{35,5}{\g\per\mole}$
\end{donnees}


\question{
  En déduire la concentration en quantité de matière des ions fer notés $[\chemfig{Fe^{2+}}]$ et des ions chlorure notés $[\chemfig{Cl^{-}}]$.
}{}{3}