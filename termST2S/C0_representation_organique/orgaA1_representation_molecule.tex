%%%%
\teteTermStssOrga

%%%% titre
\numeroActivite{1}
\vspace*{-30pt}
\titreActivite{Représenter des molécules organiques}

%%%% objectifs
\begin{objectifs}
  \item Rappeler les règles de formation des molécules et la valence d'un atome
  \item Rappeler les différentes représentations des molécules organiques
\end{objectifs}

\begin{contexte}
  Les atomes de carbones peuvent se lier entre eux pour former des \textbf{chaînes carbonées}, de formes et de tailles variées.
  Ces chaînes carbonées, une fois liée à des atomes d'hydrogène, d'oxygène ou d'azote, forment des \textbf{molécules organiques}.
  Il existe ainsi des millions de molécules organiques différentes.

  \problematique{
    Comment peut-on représenter ces molécules ?
  }
\end{contexte}


%%
\vspace*{-8pt}
\titreSection{La valence}
\vspace*{-8pt}

%%
\begin{doc}{Éléments composant un corps humain}{doc:element_corps_humain}
  Le corps humain est composé majoritairement de 4 éléments chimiques :
  \vspace*{-4pt}
  \begin{multicols}{2}
  \begin{listePoints}
    \item l'oxygène   \chemfig{O} (\qty{65}{\percent} en masse),
    \item le carbone  \chemfig{C} (\qty{18}{\percent}),
    \item l'hydrogène \chemfig{H} (\qty{10}{\percent})
    \item et l'azote  \chemfig{N} (\qty{3}{\percent}).
  \end{listePoints}
  \end{multicols}
  
  \begin{encart}
    \important{Numéro atomique :} il correspond au nombre de protons d'un atome et est noté $Z$ : \isotope{A}{Z}{X} (\hspace{-8pt}\exemple \isotope{12}{6}{C})
    Par neutralité de l'atome, c'est aussi son nombre d'électrons.
  \end{encart}
\end{doc}

%%
\begin{doc}{Liaison moléculaire}{doc:liaison_molecule}
  %
  À partir du numéro atomique d'un atome, on peut déterminer sa structure électronique en couche (1, 2 ou 3) et sous-couche (s ou p), puis sa \important{valence} (mono, bi, tri ou tétravalent).
  %
  \begin{encart}
    Pour former des molécules, les atomes partagent les électrons de leur couche externe pour former des \important{liaison covalentes}.
    Chaque liaison covalente apporte 1 électron à l'atome.
    La \important{valence} est le nombre de liaisons formées par l'atome.
  \end{encart}
  %
  \begin{encart}
    La couche 1 contient au maximum \textbf{2 électrons} et les couches 2 et 3 contiennent jusqu'à \textbf{8 électrons}.

    Les atomes cherchent à remplir leur couche externe : c'est la règle du \important{duet} (couche 1) ou de \important{l'octet} (couche 2 ou 3).
  \end{encart}
  %
  Pour connaître la valence d'un atome, il suffit donc de compter combien d'électrons il lui manque pour remplir sa couche externe.

  \exemple \isotope{}{6}{C} : $1^2 2^4$,
  il lui manque \textbf{4} électrons pour compléter sa couche externe et respecter la règle de \textbf{l'octet.}
  Il fera donc \textbf{4} liaisons, il est \textbf{tétravalent}.
\end{doc}


%% questions
\question{%
  Indiquer la configuration électronique de l'oxygène \isotope{}{8}{C}, combien d'électrons il lui manque pour respecter la règle du duet ou de l'octet, le nombre de liaisons ainsi formées et sa valence.
}{%
  \isotope{}{8}{C} : $1^2; 2^6$,
  il lui manque 2 électrons pour respecter la règle de l'octet, il formera donc 2 liaisons. Il est bivalent.
}{2}

%
\question{%
  Même question pour l'azote \isotope{}{7}{N} et l'hydrogène \isotope{}{1}{H}.
}{%
}{4}


%%
\begin{doc}{Liaisons multiples}{doc:liaisons_multiples}
  %
  \begin{encart}
    Pour compléter leur couche externe et respecter la règle de l'octet, deux atomes peuvent se lier en formant 2 ou 3 liaisons covalentes.
    
    On dit qu'il y a une \texteTrou{liaison double}{0.3} ou une \texteTrou{liaison triple}{0.3}
  \end{encart}
\end{doc}

%
\numeroQuestion
Indiquer si les liaisons sont simples, triples ou doubles sur les molécules suivantes :
\begin{equation}
  \chemfig{
    N~N
  } \qq{}
  \chemfig{
    O=C=O
  } \qq{}
  \chemfig{
    H-C~N
  }
\end{equation}


%%
\titreSection{Les représentations des molécules}

%%
\titreSousSection{La formule brute}

\begin{doc}{Formule brute}{doc:formule_brute}
  \begin{encart}
    Elle indique le nombre de chaque élément présent dans la molécule.
  \end{encart}
  Elle permet de calculer facilement les \important{masses molaires} et de vérifier si deux molécules sont \important{isomères}.

  \begin{encart}
    Deux molécules sont \important{isomères} si elles ont la même formule brute, mais un agencement des atomes différents.
  \end{encart}

  \exemple Le butane \chemfig{C_4 H_{10}}. L'éthanol \chemfig{C_2 H_6 O}. 
\end{doc}

L'oxybenzone est une molécule utilisée pour protéger des UVA et B issu du soleil.
Sa formule brute est \chemfig{C_14 H_12 O_3}.

\question{%
  Indiquer le nombre d'élément d'hydrogène, d'oxygène et de carbone dans la molécule d'oxybenzone.
}{%
  Il y a 12 hydrogènes, 3 oxygènes et 14 carbones.
}{1}


La taurine est un acide aminé produit naturellement dans le corps humain.
Sa représentation avec un modèle moléculaire est présentée à gauche.

\question{%
  Donner la formule brute de la taurine.
}{%
  \chemfig{C_2 H_7 O_3 N}
}{1}


%%
\titreSousSection{La formule développée}

\begin{doc}{Formule développée}
  \label{doc:formule_developpee}
  %
  Elle représente tous les éléments chimiques et toutes les liaisons dans le même plan.

  \exemples
  \begin{equation*}
    \chemfig{
      H 
      - C (-[3] O (-[5] H)) (-[9] H)
      - C (-[3] H) (-[9] H)
      - H
    }
    %éthanol
    \qq{}
    \chemfig{
      Cl
      - C (-[3] H) (-[9] H)
      - Si (-[3] H) (-[9] H)
      - H
    }
    %chlorométhylsilane
    \qq{}
    \chemfig{
      *6 (C (-[7] O (-[5] H))
        - C (-[9] H)
        = C (-[-1] H)
        - C (-[1] N 
          (-[3] H) (-[-1] C 
            (=[1] O) (-[10] C (-[0] H) (-[6] H) (-[9] H))
          )
        )
        = C (-[3] H)
        - C (-[5] H) (=[9,0.75])
      )
    }
    %paracétamol
  \end{equation*}
\end{doc}

%%
\titreSousSection{La formule semi-développée}

\begin{doc}{Formule semi-développée}
  \label{doc:formule_semi_developpee}

  Comme la formule développée, elle représente tous les éléments chimiques, mais elle ne détaille pas les liaisons des éléments \textbf{hydrogènes}.

  \exemples
  \begin{equation*}
    \chemfig{CH_2 (-[3] OH) - CH_3}
    %éthanol
    \qq{}
    \chemfig{Cl - CH_2 - Si H_3}
    %chlorométhylsilane
    \qq{}
    \chemfig{
      *6 (C (-[7] HO)
        - CH
        = CH
        - C (-[1] NH
          (-[-1] C (=[1] O) (-[9] CH_3))
        )
        = CH
        - CH (=[9,0.75,2])
      )
    }
    %paracétamol
  \end{equation*}
\end{doc}


%%
\titreSousSection{La formule topologique}

\begin{doc}{Formule topologique}
  \label{doc:formule_topologique}

  Elle représente les liaisons \textbf{carbone-carbone \chemfig{C - C}} par des segments formant un angle de \qty{30}{\degree}.
  Les éléments \textbf{carbones} et \textbf{hydrogènes} qui sont attachés aux carbones \textbf{ne sont pas représentés}.
  Tous les autres éléments chimiques sont représentés normalement.

  \exemples
  \begin{equation*}
    \chemfig{HO -[1] -[11]}
    %éthanol
    \qq{}
    \chemfig{Cl-[1] -[11] Si H_3}
    %chlorométhylsilane
    \qq{}
    \chemfig{
      *6 ((-[7] HO)
        -=- (-[1] NH
          (-[-1] (=[1] O) (-[9]))
        )
        =-=
      )
    }
    %paracétamol
  \end{equation*}
\end{doc}

%
\question{%
 bla
}{%
 bla
}{2}
