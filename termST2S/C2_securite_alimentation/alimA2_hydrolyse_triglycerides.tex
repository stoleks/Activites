%%%%
\teteTermStssAlim

%%%% titre
\numeroActivite{2}
\titreActivite{Hydrolyse des triglycérides}


%%%% objectifs
\begin{objectifs}
  \item Connaître la réaction modélisant l'hydrolyse d'un triglycérides
\end{objectifs}

\begin{contexte}
  
  \problematique{
    
  }
\end{contexte}


%%%% docs
\begin{doc}{Hhydrolyse de la trioléine}{doc:A1_hydrolyse_lio}
  \begin{center}
    \begin{tblr}{colspec = {c c c}}
      \SetCell[r=4]{c, m} \chemfig[atom sep = 14pt]{!\trioleine} & 
      \SetCell[r=5]{c, m} \reaction & 
      \chemfig[atom sep = 14pt]{H!\oleique} \\
      %
      & & + \chemfig[atom sep = 14pt]{H!\oleique} \\
      & & + \chemfig[atom sep = 14pt]{H!\oleique} \\
      & & + \chemfig[atom sep = 16pt]{!\glycerol} \\
      + \texteTrou{3} \chemfig{H_2O} & & 
    \end{tblr}
  \end{center}
\end{doc}

%%
\begin{doc}{}{doc:A1_}
\end{doc}


%%%%
\question{
}{
}{}

\numeroQuestion
