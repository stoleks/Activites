%%%%
\teteTermStssAlim

%%%% titre
\numeroActivite{2}
\titreActivite{Hydrolyse des triglycérides}


%%%% objectifs
\begin{objectifs}
  \item Connaître la réaction modélisant l'hydrolyse d'un triglycérides
\end{objectifs}

\begin{contexte}
  Les triglycérides sont une source importante d'énergie pour notre organisme.
  La dégradation des acides gras constituants les triglycérides permet de produire de \important{l'adénosine triphosphate}, ou \important{ATP}, qui fournit l'énergie nécessaire aux réactions chimiques du métabolisme cellulaire.
  
  \problematique{
    Quelle réaction chimique permet de séparer les triglycéride en acide gras ?  
  }
\end{contexte}


%%%% docs
\begin{doc}{Hydrolyse de l'oléine}{doc:A2_hydrolyse_oleine}
  \begin{encart}
    \important{L'hydrolyse} (du grec \og hydro \fg: eau et \og lysis \fg : briser) est une réaction chimique et enzymatique dans laquelle une liaison covalente est rompue par action d'une molécule d'eau.
  \end{encart}

  L'oléine est un triglycéride constituant \qty{80}{\percent} de l'huile d'olive.
  
  Au cours de son absorption par l’organisme, \textbf{l’oléine} est \important{hydrolysée} pour former de \textbf{l’acide
oléique} selon l’équation suivante :
  \begin{center}
    \begin{tblr}{colspec = {c c c}}
      \SetCell[r=4]{c, m} \chemfig[atom sep = 14pt]{!\trioleine} & 
      \SetCell[r=5]{c, m} \reaction & 
      \chemfig[atom sep = 14pt]{H!\oleique} \\
      %
      & & + \chemfig[atom sep = 14pt]{H!\oleique} \\
      & & + \chemfig[atom sep = 14pt]{H!\oleique} \\
      & & + \chemfig[atom sep = 16pt]{!\glycerol} \\
      + \texteTrou{3} \chemfig{H_2O} & & 
    \end{tblr}
  \end{center}
\end{doc}

\numeroQuestion
Dans le document~\ref{doc:A2_hydrolyse_oleine}, entourer les groupes caractéristiques de la molécule d'oléine et d'une molécule d'acide oléique.

%%
\begin{doc}{}{doc:A2_}
\end{doc}


%%%%
\question{
  Bla bla
}{
}{1}

\numeroQuestion
