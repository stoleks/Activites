%%%% Pour avoir les accents et autre caractère français
\usepackage[french]{babel}
\usepackage[T1]{fontenc}
\usepackage[utf8]{inputenc}

%%%% Paquets utilisé
\usepackage{ifthen} % pour programmer avec des boucle et des conditions
%% Images/dessin
\usepackage{subcaption} % pour les légendes des figures
\usepackage{graphicx} % pour insérer des images
\usepackage[european, straightvoltages, RPvoltages]{circuitikz} % pour dessiner des circuits électrique
\usepackage{pdfpages} % pour inclure des fichiers pdf
\usepackage{wrapfig} % pour entourer les images par du texte 
\usepackage{chemfig} % pour dessiner des formules chimiques
\usepackage{fontawesome} % pour dessiner de jolies icônes
%% Mise en page
\usepackage{geometry} % définition des marges
\usepackage{dashundergaps} % pour avoir générer des textes à compléter
\usepackage{fancyhdr} % pour faire des en-tête
\usepackage[most]{tcolorbox} % pour faire de jolie boîtes colorée
\usepackage{enumitem} % pour pouvoir définir des listes personnalisées
\usepackage{hyperref} % pour insérer des liens
\usepackage{multicol} % pour avoir plusieurs colonnes côte-à-côte
\usepackage{listings} % pour insérer du code
\usepackage{marginnote} % pour insérer des notes sur le côté
%% Tableau
\usepackage{tabularray} % pour avoir de meilleurs tableaux
%% QR code
\usepackage{qrcode} % Note : il faut que le qrcode soit sur une ligne séparé...
%% Math
\usepackage{amsmath} % symboles mathématiques
\usepackage{amssymb} % symboles mathématiques en gras
\usepackage{wasysym} % pour avoir des symbole d'intégrale
\usepackage{accents} % pour les notations mathématiques avec une barre
\usepackage{physics} % pour les dérivées, les bra, les kets, etc.
\usepackage{esvect} % pour faire de jolis vecteurs
\usepackage{siunitx} % pour avoir de jolie grandeurs avec des unités
\usetikzlibrary{calc} % pour faire des opérations dans tikz
\usepackage{xspace}


%%%% Commandes prédéfinies
%%%% quelque couleurs maison
\definecolor{vertHerbe}   {RGB} {124, 179,  66}
\definecolor{vertSapin}   {RGB} {  0,  95,  17}
\definecolor{vertSombre}  {RGB} { 14,  84,  60}
\definecolor{cyan}        {RGB} {  0, 140, 128}
\definecolor{cyanSombre}  {RGB} {  0,  98, 116}
\definecolor{bleuPale}    {RGB} { 39,  76, 167}
\definecolor{jauneClair}  {RGB} {218, 173,   0}
\definecolor{jauneSombre} {RGB} {213, 145,   2}
\definecolor{orangeSombre}{RGB} {174,  82,   0}
\definecolor{rougeClair}  {RGB} {224,  59,  54}
\definecolor{rougeSombre} {RGB} {148,  31,   0}


%%%% Couleurs flexoki : https://stephango.com/flexoki
\definecolor{red600}    {HTML} {AF3029}
\definecolor{orange600} {HTML} {BC5215}
\definecolor{yellow600} {HTML} {AD8301}
\definecolor{green600}  {HTML} {66800B}
\definecolor{cyan600}   {HTML} {24837B}
\definecolor{blue600}   {HTML} {205EA6}
\definecolor{purple600} {HTML} {5E409D}
\definecolor{magenta600}{HTML} {A02F6F}
%%%%%%%%%%%%%%%%%%%%%%%%%%%%%%%%%%%%%%%%%%%%%%%%%%%%%%%%%%%%%%%%%%%%%%%%%%
%%%% quelque couleurs
\definecolor{vertSombre}  {RGB} {  0,  95,  17}
\definecolor{vertForet}   {RGB} {124, 179,  66}
%
\definecolor{jauneSombre} {RGB} {213, 145,   2}
\definecolor{orangeSombre}{RGB} {174,  82,   0}
\definecolor{cyanSombre}  {RGB} {  0, 140, 128}
\definecolor{bleuPale}    {RGB} { 39,  76, 167}
%
\definecolor{jauneClair} {RGB} {218, 173,   0}
\definecolor{rougeSombre}{RGB} {148,  31,   0}
\definecolor{rougeClair} {RGB} {224,  39,  34}

%%% quelques couleurs dérivées des couleurs choisie
\newcommand{\couleurPrimSombre}{couleurPrim!60!black}

%%%% rectangle coloré
\NewDocumentCommand{\rectangle}{O{couleurPrim} m m}{%
  \shorthandoff{;}
  \tikz \node (rect) [draw, fill, color=#1,
              minimum width=#2,
              minimum height=#3] {};
  \shorthandon{;}
}


%%%%%%%%%%%%%%%%%%%%%%%%%%%%%%%%%%%%%%%%%%%%%%%%%%%%%%%%%%%%%%%%%%%%%%%%%%%
%%%% une simple boite vide
\newtcolorbox{boite}[1][]{
  breakable, enhanced jigsaw, % pour s'étendre sur plusieurs pages
  arc = 0mm, % les lignes de la boites sont droites
  colback = white, colframe = black, % fond blanc et traits noirs
  #1
}

%%%% boite colorée
\newtcolorbox{boiteColoree}[1][]{
  breakable, enhanced jigsaw, % pour s'étendre sur plusieurs pages
  arc = 2mm, % les lignes de la boites sont droites
  colback = couleurPrim, colframe = white, % fond et traits colorés
  coltext = white,
  #1
}

%%%% document
\newcounter{documentNum}
\newtcolorbox{doc}[3][]{
  before title = {\refstepcounter{documentNum}},
  breakable, enhanced jigsaw, % pour s'étendre sur plusieurs pages
  colback = white, % fond blanc
  colframe = couleurPrim!25!black, % couleur de la boite
  coltitle = black, % couleur du titre
  boxrule = 0.5mm, arc = 0.5mm, % largeur et arrondi des traits de la boite
  titlerule = 0mm, top = 0mm, % pour ne pas avoir de séparation titre/boite
  colbacktitle = white, % fond pour le titre blanc
  fonttitle = \bfseries\sffamily,
  title = {Document \arabic{documentNum} -- #2\strut \label{#3}},
  #1
}

%%%% Passage important à connaître
\newtcolorbox{encart}[1][]{
  breakable, enhanced jigsaw, % pour s'étendre sur plusieurs pages
  frame hidden, sharp corners, boxrule = 0mm, % pas de contours
  colback = couleurPrim!10, % fond
  borderline west={4pt}{0pt}{couleurPrim}, % barre gauche
  #1
}

%%%% Boite de correction
\newtcolorbox{boiteCorrection}[1][]{
  breakable, enhanced jigsaw, % pour s'étendre sur plusieurs pages
  frame hidden, sharp corners, boxrule = 0mm, % pas de contours
  colback = couleurPrim!10, % fond
  #1
}

%%%% contexte
\newtcolorbox{contexte}[1][]{
  breakable, enhanced jigsaw, % pour s'étendre sur plusieurs pages
  boxrule = 3pt, sharp corners, % contours droits
  colframe = couleurPrim, % couleur des contours
  colback = couleurPrim!5, % fond
  title = {Contexte :}, % titre
  colbacktitle = couleurPrim!5, % couleur du fond du titre
  fonttitle = \bfseries\sffamily, coltitle = black, %
  titlerule = 0mm, top = 0mm, % pour ne pas avoir de séparation titre/boite
  detach title, before upper={\vspace{2pt}\hspace{-8pt}\tcbtitle\;},
  #1
}

%%%% Pour les objectifs
\newtcolorbox{boiteObjectifs}[2][]{
  empty, % pas de boite automatique
  attach boxed title to top left = {yshift=-2.5mm}, % position
  boxed title style = {empty, size = small, top = 1mm, bottom = 0.5mm},
  frame code = { % tracé de la boite
    \path (title.east |- frame.north) coordinate (aux);
    \path [draw=couleurPrim, line width = 3pt]
    (frame.west) |- ([xshift=-4mm] title.north east)
    to[out=0, in=180] ([xshift=10mm] aux) -| % définit la courbe
    (frame.east) |- (frame.south) -| cycle; % trace la boite
  },
  coltitle = black, % couleur du titre
  fonttitle = \bfseries\sffamily, % police du titre
  title = {#2},
  #1 
}
\newenvironment{objectifs}{
  \begin{boiteObjectifs}{Objectifs de la séance :}
    \begin{listeObjectifs}
}{
    \end{listeObjectifs}
  \end{boiteObjectifs}
}

%%%% Espace pour un coup de pouce
\newcounter{coupDePouceNum}
\newtcolorbox{coupDePouce}[1][]{
  before title = {\refstepcounter{coupDePouceNum}},
  breakable, enhanced jigsaw, % pour s'étendre sur plusieurs pages
  arc = 0mm, % les lignes de la boites sont droites
  colback = white, colframe = black, % fond blanc et traits noirs
  fonttitle = \bfseries, coltitle = black, % couleur et police du titre
  titlerule = 0mm, top = 0mm, % pour ne pas avoir de séparation titre/boite
  colbacktitle = white, % fond pour le titre blanc
  title = {
    \textcolor{couleurPrim}{\faThumbsUp}
    Coup de pouce \arabic{coupDePouceNum} :
    \flushright \vspace*{-26pt}\faSquareO
  },
  #1
}

%%%% Espace pour une appréciation
\newcommand{\appreciation}[1]{
  \pasCorrection{
    \begin{boite}
      \vspace*{-4pt}
      \sousTitre{Appréciation et remarques}
      
      \vspace*{#1 cm}
      \phantom{b}
    \end{boite}
  }
}

%%%% Espace fiche TP
\newtcolorbox{boiteMateriel}[2][]{
  colback = white, colframe = black, % fond blanc et traits noirs
  coltitle = white, % couleur du titre
  boxrule = 0.5mm, arc = 0.5mm, % largeur et arrondi des traits de la boite
  titlerule = 0mm, top = 0mm, % pour ne pas avoir de séparation titre/boite
  colbacktitle = couleurPrim, % fond pour le titre blanc
  fonttitle = \bfseries\sffamily, % type de titre
  title = {\centering \large #2}, % titre
  #1
}

%%%%%%%%%%%%%%%%%%%%%%%%%%%%%%%%%%%%%%%%%%%%%%%%%%%%%%%%%%%%%%%%%%%%%%%%%%
%%%% pagination et sections
\newcommand{\titre}[1]{
  \begin{center}
    \textsf{\bfseries \Large #1}
  \end{center}
}
\newcommand{\sousTitre}[1]{
  \textsf{\bfseries #1}
}
\newcommand{\pasDePagination}{
  \thispagestyle{empty}
}
\newcommand{\feuilleBlanche}{
  \newpage
  \phantom{b}
  \pasDePagination
}
    

%%%% activité ou TP
\newcounter{activiteNum}
\newcommand{\titreActi}[2]{
  \refstepcounter{activiteNum}
  \titre{#1 \arabic{section}.\arabic{activiteNum} -- #2}
}
\newcommand{\titreTP}[1]{
  \titreActi{TP}{#1}
  % \titreActi{Activité expérimentale}{#1}
}
\newcommand{\titreActivite}[1]{
  \titreActi{Activité}{#1}
}
\NewDocumentCommand{\titreEvaluation}{o m}{
  \IfNoValueTF {#1}{
    \titre{Évaluation \arabic{section} -- #2}
  }{
    \titre{Évaluation #1 -- #2}
  }
  % reset du numéro de page et d'exercices
  \setcounter{page}{1}
  \numeroActivite{1}
}
\newcounter{exerciceNum}
\newcommand{\exercice}[1]{
  \refstepcounter{exerciceNum}
  \sousTitre{\large Exercice \arabic{exerciceNum} : #1}
  % reset des numéros de questions
  \setcounter{questionNum}{0}
  \setcounter{documentNum}{0}
}


%%%% chapitre, section et sous-section
\newcommand{\titreChapitre}[1]{
  \titre{Chapitre \arabic{section} : #1}
}
\newcommand{\titrePartie}[1]{
  \vspace*{24pt}
  \refstepcounter{subsection}
  \rectangle{40pt}{1pt}
  \sousTitre{\Large \Roman{subsection} -- #1}
  \rectangle{40pt}{1pt}
  \vspace*{10pt}
}
\newcounter{sousSectionNum}
\newcommand{\titreSection}[1]{
  \vspace*{16pt}
  \refstepcounter{subsubsection}
  \setcounter{sousSectionNum}{0}
  \rectangle{30pt}{4pt}
  \sousTitre{\large \arabic{subsubsection} -- #1}
  \vspace*{10pt}
}
\newcommand{\titreSousSection}[1]{
  \vspace*{12pt}
  \refstepcounter{sousSectionNum}
  \sousTitre{\Alph{sousSectionNum} -- #1}
  \vspace*{8pt}
}

%%%% fixe le numéro de l'activité
\newcommand{\numeroActivite}[1]{
  % fixe les compteurs LaTeX
  \setcounter{page}{1}
  \setcounter{subsection}{0}
  \setcounter{subsubsection}{0}
  \setcounter{figure}{0}
  % fixe les compteurs internes
  \setcounter{qcmNum}{0}
  \setcounter{documentNum}{0}
  \setcounter{questionNum}{0}
  \setcounter{coupDePouceNum}{0}
  \setcounter{sousSectionNum}{0}
  \setcounter{activiteNum}{#1 - 1}
}
% fixe le numéro de partie (#1) et le numéro de la page (#2)
\newcommand{\numeroPartieCours}[2]{
  \newpage
  \setcounter{subsection}{#1 - 1}
  \setcounter{page}{#2}
}

%%%% lignes
\newcommand{\ligne}{
  \par\noindent\rule{\textwidth}{0.4pt}
}
\newcommand{\lignePointillee}[1]{
  \makebox[#1\linewidth]{\dotfill}
}


%%%%%%%%%%%%%%%%%%%%%%%%%%%%%%%%%%%%%%%%%%%%%%%%%%%%%%%%%%%%%%%%%%%%%%%%%%
%%%% Paramètre par défaut pour l'en-tête
\newcommand{\annee}{Réglez avec \textbackslash renewcommand\{\textbackslash annee\}\{2023 -- 2024\}}
\newcommand{\classe}{Réglez avec \textbackslash renewcommand\{\textbackslash classe\}\{Seconde\}}
\newcommand{\etablissement}{Réglez avec \textbackslash renewcommand\{\textbackslash etablissement\}\{Lycée\}}

%%%% en-tête
\newcommand{\teteGauche}[2]{
  \lhead{
    \textbf{\footnotesize #1}
    \newline
    \footnotesize #2
  }
}
\newcommand{\teteDroite}[2]{
  \rhead{
    \hfill \textbf{\footnotesize #1}
    \newline \hfill
    \footnotesize #2
  }
}
\newcommand{\enTete}[3]{
  \pagestyle{fancy}
  \setcounter{section}{#3}
  \setcounter{subsection}{0}
  \setcounter{sousSectionNum}{0}
  \teteGauche{\etablissement{}}{Chapitre \arabic{section} -- #1} % left header
  \chead{} % central header
  \teteDroite{\annee{}}{#2} % right header
}


%%%%%%%%%%%%%%%%%%%%%%%%%%%%%%%%%%%%%%%%%%%%%%%%%%%%%%%%%%%%%%%%%%%%%%%%%%
%%%% exercice
% définit un booléen pour entrer ou sortir du mode correction
\newboolean{modeProf}
\setboolean{modeProf}{false}
\newcommand{\modeCorrection}{
  \setboolean{modeProf}{true}
  \TeacherModeOn
}

% Pour afficher le numéro d'une question avec choix du compteur
\NewDocumentCommand{\numeroQuestion}{O{questionNum} O{16}}{
  \refstepcounter{#1}
  \setcounter{sousQuestionNum}{0}
  \vspace*{2pt}
  \ifnum \thequestionNum > 9
    \hspace{6 pt}
  \else
    \hspace{#2 pt}
  \fi
  \textcolor{couleurSec}{
    \textbf{\arabic{#1}} {\small\faMinus}
  }
}

\newcommand{\numeroSousQuestion}{
  \refstepcounter{sousQuestionNum}
  \hspace{16 pt}
  \textcolor{couleurSec}{
    \textbf{\arabic{questionNum}.\arabic{sousQuestionNum}.}
  }
}


% trace des lignes pointillées pour répondre aux questions
% \lignessDeReponse* complète la ligne actuelle par des pointillées
% \lignesDeReponse commence à la ligne suivante
\newcounter{ligneNum}
\NewDocumentCommand{\lignesDeReponse}{s m}{
  % Trace la fin de la ligne, ou pas
  \IfBooleanTF{#1}{ % Version *
    \espaceReponse \dotfill\phantom{bb}
    \ifnum #2 < 1
      \newline
    \fi
  }{}
  % Trace le bon nombre de lignes
  \setcounter{ligneNum}{-1}
  \loop
    \stepcounter{ligneNum}
    \ifnum \value{ligneNum} < #2
      \\[8pt] \lignePointillee{0.98}
  \repeat
  \vspace*{1pt}
}


% définit une commande pour afficher une question 
% #1 : question
% #2 : réponse
% #3 : nombres de lignes pour répondre
\newcounter{questionNum}
\newcounter{sousQuestionNum}
\newcommand{\question}[3]{
  \numeroQuestion \!#1
  % pointille ou correction
  \ifthenelse {\boolean{modeProf}} { % prof
    \begin{boiteCorrection}
      #2
    \end{boiteCorrection}
  }{ % eleve
    \lignesDeReponse{#3}
  }
}

% Affiche le contenu en mode correction
\newcommand{\correction}[1]{
  \ifthenelse {\boolean{modeProf}} { % correction
    #1
  }{}
}

% Affiche le contenu si on est pas en mode correction
\newcommand{\pasCorrection}[1]{
  \ifthenelse{\boolean{modeProf}} {}{ % pas correction
    #1
  }
}

% Point associé à une question
\newcommand{\points}[1]{
  \marginnote{#1}
}


% sous questions
\newcommand{\sousQuestion}[2]{
  \hspace{16pt}
  \textcolor{couleurSec}{\textbullet} #1
  
  \vspace*{8pt}
  \reponse{#2}
}

% question QCM
\newcommand{\QCM}[2]{
  \numeroQuestion[qcmNum][0] #1
  \begin{qcm}
    #2
  \end{qcm}
}

% À ajouter devant la bonne réponse dans un qcm
\newcounter{qcmNum}
\newcommand{\reponseQCM}{
  \correction{
    \hspace*{-15pt}$\checkmark$\hspace*{-12pt}
  } % Note : trace une croix à la bonne position
}

%%%% Pour afficher les compétences
\newcommand{\competence}[1]{
  ~{\footnotesize\textit{(#1)}}
}

%%%% Espace pour indiquer nom, prénom et classe
\newcommand{\nomPrenomClasse}{
  \pasCorrection{
    \vspace*{-24pt}
    Nom : \lignePointillee{0.3}
    Prénom : \lignePointillee{0.3}
    Classe : \dotfill
  }
}
\newcommand{\nomPrenom}{
  \pasCorrection{
    \vspace*{-24pt}
    Nom : \lignePointillee{0.3}
    Prénom : \lignePointillee{0.3}
  }
}


%%%%%%%%%%%%%%%%%%%%%%%%%%%%%%%%%%%%%%%%%%%%%%%%%%%%%%%%%%%%%%%%%%%%%%%%%%
% texte à trou avec option pour régler la largeur
\NewDocumentCommand{\texteTrou}{o +m}{
  \ifthenelse {\boolean{modeProf}}{ % prof
    \important[black]{#2}
  }{ % élève
    \IfValueTF{#1}{ % Si la largeur est réglée, on utilise des lignes
      \espaceReponse
      \lignePointillee{#1}
      \hspace*{-12pt}
    }{ % Sinon on utilise dash undergap pour la version automatique
      \espaceReponse \hspace*{0.1pt}
      \gap{#2}
    }
  }
}

% texte à trou avec option pour laisser plusieurs lignes
\NewDocumentCommand{\texteTrouLignes}{O{0} +m}{
  \ifthenelse {\boolean{modeProf}} {% prof
    \important[black]{#2}
  }{% élève
    \lignesDeReponse*{#1}
  }
}

% espace vertical pour la réponse
\newcommand{\espaceReponse}{
  \phantom{$\dfrac{1}{1}$} % espace vertical
  \hspace*{-38pt} \phantom{b} % ajuste l'espace horizontal
}


%%%%%%%%%%%%%%%%%%%%%%%%%%%%%%%%%%%%%%%%%%%%%%%%%%%%%%%%%%%%%%%%%%%%%%%%%%
%%%% Pour choisir parmi deux sujets
\newboolean{sujetA}
\setboolean{sujetA}{true}
\newcommand{\sujetB}{
  \setboolean{sujetA}{false}
}
\newcommand{\sujetA}{
  \setboolean{sujetA}{true}
}

%%%% Pour faire plusieurs sujets en parallèle
\newcommand{\variationSujet}[2]{
  \hspace*{-6pt}
  \ifthenelse {\boolean{sujetA}}{#1}{#2}
  \hspace*{-6pt}
}


%%%%%%%%%%%%%%%%%%%%%%%%%%%%%%%%%%%%%%%%%%%%%%%%%%%%%%%%%%%%%%%%%%%%%%%%%%
%%%% Tableau générique avec la première ligne bleue
\NewDocumentEnvironment{tableau}{m}{
  \begin{center}
    \begin{tblr}{
      hlines,
      colspec = #1,
      row{1} = {couleurPrim!20},
    }
}{
    \end{tblr}
  \end{center}
}

%%%% Tableau de competence
\newenvironment{tableauCompetences}{
  \centering
  \begin{tblr}{
    colspec = {| c | X[l] | c | c | c | c |},
    rows = {m}, hlines,
    row{1} = {couleurPrim!20}
  }
    \textbf{Compétences} & \centering \textbf{Items} 
    & \textbf{D} & \textbf{C} & \textbf{B} & \textbf{A} \\
}{
  \end{tblr}
}

%%%% Tableau de connaissances sans exercices
\newenvironment{tableauConnaissances}{
  \centering
  \begin{tblr}{
    colspec = {Q[t,wd=0.7\textwidth] c c c},
    rows = {m}, hlines, vlines,
    column{4} = {0.2},
    row{1} = {couleurPrim!20, c}
  }
    \textbf{Connaissances et capacités exigibles} & \ok & \pasOk & \textbf{En classe} \\
}{ 
  \end{tblr}
}


%%%% Alignement dans un tableau
\newcommand{\vAligne}[1]{
  \strut \\ \vspace*{#1}
}


%%%%%%%%%%%%%%%%%%%%%%%%%%%%%%%%%%%%%%%%%%%%%%%%%%%%%%%%%%%%%%%%%%%%%%%%%%
%%%% symboles : chevron, flèche, attention, etc.
\NewDocumentCommand{\chevron}{O{couleurPrim}}{
  \textcolor{#1}{\small \faChevronRight}
}
%
\NewDocumentCommand{\fleche}{O{couleurPrim}}{
  \textcolor{#1}{\faCaretRight}
}
%
\NewDocumentCommand{\attention}{O{couleurPrim}}{
  \textcolor{#1}{\faExclamationTriangle}
}
%
\NewDocumentCommand{\flecheLongue}{O{couleurPrim}}{
  \textcolor{#1}{\faLongArrowRight}
}
%
\NewDocumentCommand{\ok}{O{couleurPrim}}{
  \textcolor{#1}{\faCheckCircle}
}
%
\NewDocumentCommand{\pasOk}{O{couleurPrim}}{
  \textcolor{#1}{\faTimesCircle}
}
%
\NewDocumentCommand{\pointCyan}{O{couleurPrim}}{
  \textcolor{#1}{\textbullet}
}
%
\NewDocumentCommand{\mesure}{O{couleurPrim}}{
  \hspace{15pt}
  %\numeroQuestion
  \textcolor{couleurSec}{\faWrench\faFlask}
}
% pictogramme sécurité
\newcommand{\picto}[2]{
  \image{#1}{images/pictogrammes/picto_#2}
}
% Nombre dans un cercle
\newcommand*\nombreCercle[1]{
  % \tikz[baseline=(char.base)]{
  %   \node [shape=circle, draw filled, inner sep=1.2pt, color=couleurSec!20] (char) {\textcolor{black}{#1};
  % }
  \important[couleurSec]{#1}
}
% Pour légender une image
\newcommand{\legende}[1]{
  \textcolor{couleurPrim}{\faArrowUp} \; #1
}

%%%%%%%%%%%%%%%%%%%%%%%%%%%%%%%%%%%%%%%%%%%%%%%%%%%%%%%%%%%%%%%%%%%%%%%%%%
%%%% emphase
\newcommand{\emphase}[1]{
  \textcolor{couleurSec}{\textsf{\bfseries \large #1}}
}
%
\NewDocumentCommand{\important}{O{couleurSec!75!black} m}{
  \!\textcolor{#1}{\textsf{\bfseries #2}}\!\!
}
%
\newcommand{\exemple}{
  \flecheLongue \textit{Exemple :}
}
\newcommand{\exemples}{
  \flecheLongue \textit{Exemples :}
}
%
\newcounter{compteAppelProf}
\newcommand{\appelProf}{
  \refstepcounter{compteAppelProf}
  \hspace{24pt} \faHandPaperO \hspace{2pt}
  \textbf{Appel n$^\circ$ \arabic{compteAppelProf} :}
}

%
\newcommand{\extrait}[2]{
  « #1 »
  
  \vspace*{-12pt}
  \begin{flushright}
    \textit{#2}
  \end{flushright}
  \vspace*{-12pt}
}


%%%% image
\newcommand{\image}[2]{
  \includegraphics[width=#1\linewidth]{#2}
}

%%%% qr code en insert sur la droite
\NewDocumentCommand{\QRCode}{o m}{
  \IfNoValueTF{#1} {
    \begin{wrapfigure}{r}{0.1\linewidth}
      \vspace*{-16pt}
      \qrcode{#2}
    \end{wrapfigure}
  }{
    \begin{wrapfigure}[#1]{r}{0.1\linewidth}
      \vspace*{-16pt}
      \qrcode{#2}
    \end{wrapfigure}
  }
}


%%%%%%%%%%%%%%%%%%%%%%%%%%%%%%%%%%%%%%%%%%%%%%%%%%%%%%%%%%%%%%%%%%%%%%%%%%
%%%% qcm
\newlist{qcm}{itemize}{2}
\setlist[qcm]{label=$\square$, leftmargin=2cm}

%%%% liste d'objectif
\newlist{listeObjectifs}{itemize}{2}
\setlist[listeObjectifs]{label = \chevron}

%%%% protocole
\newlist{protocole}{itemize}{2}
\setlist[protocole]{label = {\footnotesize \fleche[couleurSec]}}

%%%% liste de points
\newlist{listePoints}{itemize}{2}
\setlist[listePoints]{label = \pointCyan}

%%%% liste tirets
\newlist{listeTirets}{itemize}{2}
\setlist[listeTirets]{label = \textcolor{couleurPrim}{\small\faMinus}}

%%%% liste avec des flèches
\newlist{listeFleche}{itemize}{2}
\setlist[listeFleche]{label = \textbf{\flecheLongue}}

%%%% jeu de données
\newenvironment{donnees}{
  
  \textbf{Données :}
  \vspace*{-8pt}
  \begin{multicols}{2}
    \begin{listeTirets}
}{
    \end{listeTirets}
  \end{multicols}
}

%%%% problematique
\newcommand{\problematique}[1]{
  \hspace{8pt}
  \flecheLongue
  \textbf{#1}
}

%%%% liste avec chiffre
\newlist{enumeration}{enumerate}{2}
\setlist[enumeration]{label = \textcolor{\couleurPrimSombre}{\textbf{\arabic*.}} }


%%%%%%%%%%%%%%%%%%%%%%%%%%%%%%%%%%%%%%%%%%%%%%%%%%%%%%%%%%%%%%%%%%%%%%%%%%
%%%% Séparation de la page en blocs
\newcommand{\separationTroisBlocs}[3]{
  \begin{minipage}[T]{0.3\linewidth}
    #1
  \end{minipage}
  ~
  \begin{minipage}[T]{0.3\linewidth}
    #2
  \end{minipage}
  ~
  \begin{minipage}[T]{0.3\linewidth}
    #3
  \end{minipage}
}
%%%% Separation en deux blocs
\NewDocumentCommand{\separationBlocs}{+m O{0.48} +m O{0.48}}{
  \begin{minipage}[T]{#2\linewidth}
    #1
  \end{minipage}
  \hfill
  \begin{minipage}[T]{#4\linewidth}
    #3
  \end{minipage}
}


%%%%%%%%%%%%%%%%%%%%%%%%%%%%%%%%%%%%%%%%%%%%%%%%%%%%%%%%%%%%%%%%%%%%%%%%%%
%% nombre algébrique, réaction
\newcommand{\algebrique}[1]{
  \overline{\mathrm{#1}}
}
\newcommand{\reaction}{
  \!\!\schemestart \arrow(.mid east--.mid west){->}[, 0.9, ultra thick] \schemestop\!\!
}

%% Pour simplifier l'écriture des formules brutes
\newcommand{\bruteCHO}[3]{
  \chemfig{C_{#1} H_{#2} O_{#3}}
}

%% pour les masse molaire et atomique
\newcommand{\masseMol}[1]{
  M(\chemfig{#1})
  % M_{\chemfig{#1}}
}
\newcommand{\masseAtom}[1]{
  m(\chemfig{#1})
  % m_{\chemfig{#1}}
}


%% Unités
\DeclareSIUnit{\dioptre}{\text{$\delta$}}
\DeclareSIUnit{\dornic}{\text{\textdegree D}}
\DeclareSIUnit{\ppm}{\text{ppm}}
\DeclareSIUnit{\COeq}{\text{kgCO$_{2}$e}}
\DeclareSIUnit{\jour}{\text{jour}}
% \DeclareSIUnit{}{\text{}}


%% atome ou isotope #1: Z, #2: A, #3: X
\makeatletter
\newcommand{\isotope}[3]{%
   \settowidth\@tempdimb{\ensuremath{\scriptstyle#1}}%
   \settowidth\@tempdimc{\ensuremath{\scriptstyle#2}}%
   \ifnum\@tempdimb>\@tempdimc%
       \setlength{\@tempdima}{\@tempdimb}%
   \else%
       \setlength{\@tempdima}{\@tempdimc}%
   \fi%
  \begingroup%
  \ensuremath{
    ^{\makebox[\@tempdima][r]{\ensuremath{\scriptstyle#1}}}
    _{\makebox[\@tempdima][r]{\ensuremath{\scriptstyle#2}}}
    \chemfig{#3}
  }%
  \endgroup%
}%
\makeatother

%% element chimique dans le tableau périodique
\makeatletter
\newcommand{\element}[2]{%
   \settowidth\@tempdimb{\ensuremath{\footnotesize #1}}%
  \begingroup%
  \ensuremath{
    _{\makebox[\@tempdimb][r]{\ensuremath{\small #1}}} 
    \chemfig[atom style={scale=1.3}]{#2}
  }%
  \endgroup%
}%
\makeatother

%% siècle
\newcommand{\siecle}[1]{
  \textsc{\romannumeral #1}\textsuperscript{e}~siècle
}

%% texte avec une boite autour
\NewDocumentCommand{\texteEncadre}{m O{black}}{
  \textcolor{#2}{
    \frame{
      \vphantom{$\dfrac{1}{10}$} \textcolor{black}{\text{#1}}
    }
  }
}

%% case cochée
\newcommand{\caseCochee}{
  $\text{\rlap{$\checkmark$}}\square$
}


%%%%%%%%%%%%%%%%%%%%%%%%%%%%%%%%%%%%%%%%%%%%%%%%%%%%%%%%%%%%%%%%%%%%%%%%%%
%%%% Couleur pour le code
\definecolor{vertCode}  {rgb}{0.2,0.6,0}
\definecolor{grisCode}  {rgb}{0.5,0.5,0.5}
\definecolor{violetCode}{rgb}{0.58,0,0.82}
\definecolor{fondCode}  {rgb}{0.95,0.95,0.92}
%%%% Style python
\lstdefinestyle{codePython}{
  backgroundcolor=\color{fondCode},
  commentstyle=\color{magenta},
  keywordstyle=\color{vertCode},
  numberstyle=\tiny\color{grisCode},
  stringstyle=\color{violetCode},
  basicstyle=\ttfamily\footnotesize,
  breakatwhitespace=false,
  breaklines=true,
  captionpos=b,
  keepspaces=true,
  numbers=left,
  numbersep=5pt, 
  showspaces=false,
  showstringspaces=false,
  showtabs=false, 
  tabsize=2
}
\def\inline{\lstinline[style=codePython,language=python]}


%%%%%%%%%%%%%%%%%%%%%%%%%%%%%%%%%%%%%%%%%%%%%%%%%%%%%%%%%%%%%%%%%%%%%%%%%%
%%%% circuit tikz
\NewDocumentCommand{\fixedvlen}{O{0.5cm} m m O{}}{% [semilength]{node}{label}[extra options]
  % get the center of the standard arrow
  \coordinate (#2-Vcenter) at ($(#2-Vfrom)!0.5!(#2-Vto)$);
  % draw an arrow of a fixed size around that center and on the same line
  \draw[-Triangle, #4] ($(#2-Vcenter)!#1!(#2-Vfrom)$) -- ($(#2-Vcenter)!#1!(#2-Vto)$);
  % position the label as in the normal voltages
  \node[anchor=\ctikzgetanchor{#2}{Vlab}, #4] at (#2-Vlab) {#3};
}

%%%%%%%%%%%%%%%%%%%%%%%%%%%%%%%%%%%%%%%%%%%%%%%%%%%%%%%%%%%%%
%% grandeurs récurrentes
% Physique
\newcommand{\ISS}{\text{ISS}}
\newcommand{\Terre}{\text{Terre}}
\newcommand{\inertie}{\text{inertie}}
\newcommand{\Tfus}{T_\text{f}}
\newcommand{\Teb}{T_\text{éb}}
% Chimie
\newcommand{\solute}{\text{soluté}}
\newcommand{\solution}{\text{solution}}
\newcommand{\espece}{\text{espèce}}
\newcommand{\avogadro}{\num{6,02e23}}
% ions
\newcommand{\ionFerII}      {Fer II      \chemfig{Fe^{2+}}   }
\newcommand{\ionFerIII}     {Fer III     \chemfig{Fe^{3+}}   }
\newcommand{\ionSodium}     {Sodium      \chemfig{Na^{+}}    }
\newcommand{\ionCuivreII}   {Cuivre II   \chemfig{Cu^{2+}}   }
\newcommand{\ionCalcium}    {Calcium     \chemfig{Ca^{2+}}   }
\newcommand{\ionSulfate}    {Sulfate     \chemfig{SO_4^{2-}} }
\newcommand{\ionNitrate}    {Nitrate     \chemfig{NO_3^{-}}  }
\newcommand{\ionChlorure}   {Chlorure    \chemfig{Cl^{-}}    }
\newcommand{\ionFluorure}   {Fluorure    \chemfig{F^{-}}     }
\newcommand{\ionMagnesium}  {Magnésium   \chemfig{Mg^{2+}}   }
\newcommand{\ionPotassium}  {Potassium   \chemfig{K^{+}}     }
\newcommand{\ionBicarbonate}{Bicarbonate \chemfig{CO_3^{2-}} }

%% vecteurs
\newcommand{\FBsurA}{F_{B/A}}
\newcommand{\FAsurB}{F_{A/B}}
\newcommand{\vvFAsurB}{\vv{F}_{A/B}}
\newcommand{\vvFBsurA}{\vv{F}_{B/A}}
%%%%%%%%%%%%%%%%%%%%%%%%%%%%%%%%%%%%%%%%%%%%%%%%%%%%%%%%%%%%%
%%%% figures simples
\newcommand{\tkzRect}[4]{
  \fill[color=#1] (#2,#4) -- (-#2,#4) -- (-#2,#3) -- (#2,#3);
}
\newcommand{\tkzEllipse}[4]{
  \fill[color=#1] (0,#3) ellipse (#2 and #4);
}

% \tkzCercle {x}{y} {couleur} {rayon}
\newcommand{\tkzCercle}[4]{
  \filldraw [#3] (#1, #2) circle (#4pt);
}
% \tkzCercleLigne {x}{y} {couleurFond}{couleurTrait} {rayon}
\newcommand{\tkzCercleLigne}[5]{
  \filldraw [color = #4, fill = #3, very thick] (#1, #2) circle (#5pt);
}

%%%% tube à essais
\newcommand{\tkzTubeEssais}[3]{
  \draw[thick] (#1,#2) -- (#1,0) arc (0:-180:#1) -- (-#1,#2);
  \draw[thick] (0,#2) ellipse (#1 and #3);
}
\newcommand{\tkzBasTubeEssais}[5]{
  \fill[color=#1] (-#2,#3) -- (#2,#3) arc (0:-180:#2);
  \tkzRect{#1}{#2}{#3 - 0.01}{#4}
  \tkzEllipse{#1!85!black}{#2}{#4}{#5}
}
\newcommand{\tkzPhaseTubeEssais}[5]{
  \tkzRect{#1}{#2}{#3}{#4}
  \tkzEllipse{#1}{#2}{#3}{#5}
  \tkzEllipse{#1!85!black}{#2}{#4}{#5}
}

%%%% Point et vecteurs
\newcommand{\tkzLabel}[3]{
  \node at (#1, #2) {#3};
}
\newcommand{\tkzPointLabel}[3]{
  \filldraw (#1, #2) circle (2pt) node[above] {#3};
}
% \tkzVecteur [couleur] (x) [longueur x] (y) [longueur y] {legende} [position legende] 
% ajouter une * à la fin transforme la flèche en double flèche <->
\NewDocumentCommand{\tkzVecteur}{O{black} r() O{0} r() O{0} m O{right} s}{
  \IfBooleanTF{#8}{
    \draw[#1, <->, very thick] (#2, #4) -- (#2 + #3, #4 + #5) node[#7] {#6};
  }{
    \draw[#1, ->, very thick] (#2, #4) -- (#2 + #3, #4 + #5) node[#7] {#6};
  }
}
% \tkzLegende (x) (y) [longueur fleche] {légende} 
% ajouter une * passe de la version gauche -> à la version droite <-
\NewDocumentCommand{\tkzLegende}{O{black} r() r() O{1.25} m s}{
  \IfBooleanTF{#6}{
    \draw[#1, ->, very thick] (#2 + #4, #3) node[right] {#5} -- (#2, #3);
  }{
    \draw[#1, ->, very thick] (#2, #3) node[left] {#5} -- (#2 + #4, #3);
  }
}

\newcommand{\barrePourcentage}[1]{%
  \begin{tikzpicture}
    \fill[color=couleurSec]    (0.0,    0.0) rectangle (#1*8ex, 1.5ex);
    \fill[color=couleurSec!20] (#1*8ex, 0.0) rectangle (8.0ex,  1.5ex);
  \end{tikzpicture}
}

\newcommand{\flecheProgression}[1]{%
  \begin{center}
    \begin{tikzpicture}
      \draw[
        -{Triangle [width = 36pt, length = 16pt]}, 
        line width = 20pt,
        rounded corners = 10mm,
        color = couleurSec,
      ]
      #1
    \end{tikzpicture}
  \end{center}
}


%%%%%%%%%%%%%%%%%%%%%%%%%%%%%%%%%%%%%%%%%%%%%%%%%%%%%%%%%%%%%
%%%% plan de classe
\NewDocumentCommand{\texteCadre}{O{black} r() O{2} r() O{2} m}{
  \filldraw [fill=white, draw=#1, ultra thick] (#2, #4) rectangle (#2 + #3, #4 + #5);
  \node at (#2 + #3/2, #4 + #5/2) [font=\sffamily] {\textbf{#6}};
}

%% place dans la classe
\NewDocumentCommand{\place}{r() r() m}{
  \texteCadre(#1)[3](#2)[2] {#3}
}
\NewDocumentCommand{\places}{r()r() r[] d[] d[] d[]}{
  \place(#1)(#2) {#3}
  \IfValueT{#4}{ \place(#1 + 1*3)(#2) {#4} }
  \IfValueT{#5}{ \place(#1 + 2*3)(#2) {#5} }
  \IfValueT{#6}{ \place(#1 + 3*3)(#2) {#6} }
}

%% rangée de classe ou de TP
\NewDocumentCommand{\rangee}{m r()r() r()r()r()d() r()r()}{
  \places(0)(0 - 3*#1) [#2][#3]
  \IfValueTF{#7}{
    \places(7) (0 - 3*#1) [#4][#5][#6][#7]
    \places(20)(0 - 3*#1) [#8][#9]
  }{
    \places(8.5)(0 - 3*#1) [#4][#5][#6]
    \places(20) (0 - 3*#1) [#8][#9]
  }
}

\NewDocumentCommand{\rangeeTP}{m r[]r[]r[] r()r()r()d()}{
  \places(3)(0 - 3*#1) [#2][#3][#4]
  \IfValueTF{#8}{
    \places(14) (0 - 3*#1) [#5][#6][#7][#8]
  }{
    \places(14) (0 - 3*#1) [#5][#6][#7]
  }
}


%%%% tube à essai de sang
\newcommand{\tubeEssaisSolution}[1]{
  \begin{tikzpicture}
    \tkzBasTubeEssais{#1}{0.25}{0}{0.75}{0.1} % contenu du tube
    \tkzTubeEssais{0.25}{1.5}{0.1} % tube
  \end{tikzpicture}
}

\newcommand{\tubeEssaisSangCentrifuge}[3]{
  \begin{tikzpicture}
    % phases dans le tube à essai
    \tkzBasTubeEssais{rougeSombre!75!white} {0.35}{0}{#1}{0.1}
    \tkzPhaseTubeEssais{gray!10!white}      {0.35}{#1}{#2}{0.1}
    \tkzPhaseTubeEssais{jauneClair!75!white}{0.35}{#2}{#3}{0.1}
    \tkzTubeEssais{0.35}{#3 + 1}{0.1}
    % Légende
    \tkzLegende(0.4)(#3 - 0.1) [1]{Plasma}*
    \tkzLegende(0.4)(#2 - 0.08)[1]{Globules blancs}*
    \tkzLegende(0.4)(-0.1)     [1]{Globules rouges}*
  \end{tikzpicture}
}
%%%% Ce fichier sert à déclarer les titres des chapitres des différents niveaux

%% Commun
\newcommand{\methode} {\chapitre{Outils pratiques}}

%% Seconde
%%%% Chapitre
\newcommand{\snd}{Seconde}
\newcommand{\sndCorp} {\chapitre{Corps purs et mélanges}}
\newcommand{\sndSolu} {\chapitre{Solutions}}
\newcommand{\sndMouv} {\chapitre{Mouvement et interactions}}
\newcommand{\sndAtom} {\chapitre{Structure de l'atome}}
\newcommand{\sndMole} {\chapitre{Des atomes à la matière}}
\newcommand{\sndLumi} {\chapitre{Ondes lumineuses et optique}}
\newcommand{\sndTran} {\chapitre{Transformations de la matière}}
\newcommand{\sndChim} {\chapitre{Transformations chimiques}}
\newcommand{\sndSign} {\chapitre{Signaux et capteurs}}

%%%% en-tête correspondant
\newcommand{\teteSndMeth} {\enTete[\snd]{\methode}}
\newcommand{\teteSndCorp} {\enTete[\snd]{\sndCorp}[1]}
\newcommand{\teteSndSolu} {\enTete[\snd]{\sndSolu}[2]}
\newcommand{\teteSndMouv} {\enTete[\snd]{\sndMouv}[3]}
\newcommand{\teteSndAtom} {\enTete[\snd]{\sndAtom}[4]}
\newcommand{\teteSndMole} {\enTete[\snd]{\sndMole}[5]}
\newcommand{\teteSndLumi} {\enTete[\snd]{\sndLumi}[6]}
\newcommand{\teteSndTran} {\enTete[\snd]{\sndTran}[7]}
\newcommand{\teteSndChim} {\enTete[\snd]{\sndChim}[8]}
\newcommand{\teteSndSign} {\enTete[\snd]{\sndSign}[9]}


%% Première ST2S
%%%% Chapitres
\newcommand{\premStss}{Première ST2S}
\newcommand{\premStssChim} {\chapitre{Sécurité chimique dans l'habitat}}
\newcommand{\premStssVisi} {\chapitre{Propagation de la lumière et vision}}
\newcommand{\premStssRedo} {\chapitre{Antiseptique et désinfectant, oxydoréduction}}
\newcommand{\premStssLumi} {\chapitre{Les infrarouges et leurs applications}}
\newcommand{\premStssStru} {\chapitre{Molécules d'intérêt biologique}}
\newcommand{\premStssBiom} {\chapitre{Biomolécules dans l’organisme}}
\newcommand{\premStssRout} {\chapitre{Sécurité routière}}
\newcommand{\premStssAlim} {\chapitre{Gestion des ressources naturelles et alimentation}}
\newcommand{\premStssElec} {\chapitre{Sécurité électrique dans l'habitat}}
\newcommand{\premStssPres} {\chapitre{Propriétés des fluides et pression sanguine}}
\newcommand{\premStssSono} {\chapitre{Ondes sonores et audition}}

%%%% en-tête
\newcommand{\tetePremStssMeth} {\enTete[\premStss]{\methode}     }
\newcommand{\tetePremStssChim} {\enTete[\premStss]{\premStssChim}[1]}
\newcommand{\tetePremStssVisi} {\enTete[\premStss]{\premStssVisi}[2]}
\newcommand{\tetePremStssRedo} {\enTete[\premStss]{\premStssRedo}[3]}
\newcommand{\tetePremStssLumi} {\enTete[\premStss]{\premStssLumi}[4]}
\newcommand{\tetePremStssStru} {\enTete[\premStss]{\premStssStru}[5]}
\newcommand{\tetePremStssBiom} {\enTete[\premStss]{\premStssBiom}[6]}
\newcommand{\tetePremStssRout} {\enTete[\premStss]{\premStssRout}[7]}
\newcommand{\tetePremStssAlim} {\enTete[\premStss]{\premStssAlim}[8]}
\newcommand{\tetePremStssElec} {\enTete[\premStss]{\premStssElec}[9]}
\newcommand{\tetePremStssPres} {\enTete[\premStss]{\premStssPres}[10]}
\newcommand{\tetePremStssSono} {\enTete[\premStss]{\premStssSono}[11]}


%% Terminale ST2S
%%%% Chapitres
\newcommand{\termStss}{Terminale ST2S}
\newcommand{\termStssOrga} {\chapitre{Représentation des molécules organiques}}
\newcommand{\termStssAlim} {\chapitre{Sécurité physico-chimique dans l'alimentation}}
\newcommand{\termStssImag} {\chapitre{La physique de l'imagerie médicale}}
\newcommand{\termStssBiom} {\chapitre{Biomolécules et alimentation}}
\newcommand{\termStssMedi} {\chapitre{De la molécule au médicament}}
\newcommand{\termStssEnvi} {\chapitre{Sécurité chimique dans l'environnement}}
\newcommand{\termStssDosa} {\chapitre{Analyser la composition d'un milieu}}
\newcommand{\termStssRout} {\chapitre{Sécurité routière}}
\newcommand{\termStssCosm} {\chapitre{L'usage responsable des cosmétiques}}

%%%% en-tête
\newcommand{\teteTermStssMeth} {\enTete[\termStss]{\methode}}
\newcommand{\teteTermStssOrga} {\enTete[\termStss]{\termStssOrga}[1]}
\newcommand{\teteTermStssRout} {\enTete[\termStss]{\termStssRout}[8]}
\newcommand{\teteTermStssAlim} {\enTete[\termStss]{\termStssAlim}[2]}
\newcommand{\teteTermStssEnvi} {\enTete[\termStss]{\termStssEnvi}[6]}
\newcommand{\teteTermStssImag} {\enTete[\termStss]{\termStssImag}[3]}
\newcommand{\teteTermStssDosa} {\enTete[\termStss]{\termStssDosa}[7]}
\newcommand{\teteTermStssBiom} {\enTete[\termStss]{\termStssBiom}[4]}
\newcommand{\teteTermStssMedi} {\enTete[\termStss]{\termStssMedi}[5]}
\newcommand{\teteTermStssCosm} {\enTete[\termStss]{\termStssCosm}[9]}

%%%%%%%%%%%%%%%%%%%%%%%%%%%%%%%%%%%%%%%%%%%%%%%%%%%%%%%%%%%%%
%%%% Réglage de chemfig
\newcommand{\chemfigParDefaut}{
  \setchemfig{
    atom sep= 24pt,
    bond style = {line width = 1pt},
    cram width = 2.2pt,
    angle increment = 30
  }
}
\chemfigParDefaut

%%%%%%%%%%%%%%%%%%%%%%%%%%%%%%%%%%%%%%%%%%%%%%%%%%%%%%%%%%%%%
%% Pour faire des parenthèses dans les molécules 
\def\parentheseG{\llap{$\left(\strut\right.$}}
\def\parentheseD{\rlap{$\left.\strut\right)$}}

%% Pour avoir des molécules en gras dans un texte
\newcommand{\moleculesGras}{ \renewcommand*\printatom[1]{\ensuremath{\mathbf{##1}}} }
\newcommand{\moleculesNormale}{ \renewcommand*\printatom[1]{\ensuremath{\mathrm{##1}}} }

%%%% Éléments récurrents 
\newcommand{\hydrogene}{\chemfig{H}\xspace}
\newcommand{\carbone}  {\chemfig{C}\xspace}
\newcommand{\oxygene}  {\chemfig{O}\xspace}
\newcommand{\azote}    {\chemfig{N}\xspace}
\newcommand{\phosphore}{\chemfig{P}\xspace}
\newcommand{\electron} {\chemfig{e^{-}}\xspace}
%%%% Molécules récurrentes
\newcommand{\dioxygene}          {\chemfig{O_2}\xspace}
\newcommand{\dihydrogene}        {\chemfig{H_2}\xspace}
\newcommand{\diazote}            {\chemfig{N_2}\xspace}
\newcommand{\dioxydeDeCarbone}   {\chemfig{CO_2}\xspace}
\newcommand{\eau}                {\chemfig{H_2O}\xspace}
\newcommand{\methane}            {\chemfig{CH_4}\xspace}
\newcommand{\ammoniac}           {\chemfig{NH_3}\xspace}
\newcommand{\diiode}             {\chemfig{I_2}\xspace}
\newcommand{\acideCarbonique}    {\chemfig{H_2CO_3}\xspace}
\newcommand{\carbonateDeCalcium} {\chemfig{CaCO_3}\xspace}
\newcommand{\bicarbonateDeSodium}{\chemfig{NaHCO_3}\xspace}
\newcommand{\azotureDeSodium}    {\chemfig{NaN_3}\xspace}
\newcommand{\chlorureDArgent}    {\chemfig{AgCl}\xspace}
%%%% Ions récurrents
\newcommand{\oxonium}     {\chemfig{H_3O^{+}}\xspace}
\newcommand{\hydroxyde}   {\chemfig{HO^{-}}\xspace}
\newcommand{\ionHydrogene}{\chemfig{H^{+}}\xspace}
\newcommand{\ammonium}    {\chemfig{NH_4^{+}}\xspace}
\newcommand{\nitrate}     {\chemfig{NO_3^{-}}\xspace}
\newcommand{\nitrite}     {\chemfig{NO_2^{-}}\xspace}
\newcommand{\sulfate}     {\chemfig{SO_4^{2-}}\xspace}
\newcommand{\chlorure}    {\chemfig{Cl^{-}}\xspace}
\newcommand{\fluorure}    {\chemfig{F^{-}}\xspace}
\newcommand{\carbonate}   {\chemfig{CO_3^{2-}}\xspace}
\newcommand{\bicarbonate} {\chemfig{HCO_3^{-}}\xspace}
\newcommand{\ionOxygene}  {\chemfig{O^{2-}}\xspace}
\newcommand{\ionFerII}    {\chemfig{Fe^{2+}}\xspace}
\newcommand{\ionFerIII}   {\chemfig{Fe^{3+}}\xspace}
\newcommand{\ionSodium}   {\chemfig{Na^{+}}\xspace}
\newcommand{\ionArgent}   {\chemfig{Ag^{+}}\xspace}
\newcommand{\hypochlorite}{\chemfig{ClO^{-}}\xspace}
\newcommand{\ionCuivreII} {\chemfig{Cu^{2+}}\xspace}
\newcommand{\ionCalcium}  {\chemfig{Ca^{2+}}\xspace}
\newcommand{\ionMagnesium}{\chemfig{Mg^{2+}}\xspace}
\newcommand{\ionPotassium}{\chemfig{K^{+}}\xspace}
\newcommand{\ionPhosphate}{\chemfig{HPO_4^{2-}}\xspace}

%%%% État physique
\newcommand{\aq} { \ensuremath{_\text{(aq)}} }
\newcommand{\sol}{ \ensuremath{_\text{(s)}} }
\newcommand{\liq}{ \ensuremath{_\text{(l)}} }
\newcommand{\gaz}{ \ensuremath{_\text{(g)}} }

%%%%%%%%%%%%%%%%%%%%%%%%%%%%%%%%%%%%%%%%%%%%%%%%%%%%%%%%%%%%%
%%%% Pour simplifier certaines molécules
\definesubmol\vide1{ -[#1,,,, draw = none] } % liaison invisible avec angle et longueur réglable
\definesubmol\lh   { -[::60] }  % C-C vers le haut (liaison haut)
\definesubmol\lb   { -[::-60] } % "              " (liaison bas)
\definesubmol\lhb  { -[::60] -[::-60] } % Liaison C-C-C ^ (liaison haut bas)
\definesubmol\lbh  { -[::-60] -[::60] } % "           " v (liaison bas haut)
\definesubmol\llh  { =[::60] }  % Double liaison = vers le haut
\definesubmol\llb  { =[::-60] } % "                      " bas
\definesubmol\cis  { -[::60] =[::-60] -[::-60] } % Liaison -C=C- cis
\definesubmol\trans{ -[::60] =[::-30] -[::-30] } % Liaison -C=C- trans
\definesubmol\ldh  { -[::50] }  % liaison développée vers le haut
\definesubmol\ldb  { -[::-50] } % "                        " bas
\definesubmol\lldh { =[::50] }  % double liaison développée vers le haut
\definesubmol\lldb { =[::-50] } % "                               " bas  ""
\definesubmol\cram2{
  (>:[::-150] #1) 
  (<#(4pt,4pt)[::-100,1.2] #2)
}
\definesubmol\branche2  {
  (-[::-90] #1)
  (-[::90] #2)
}
\definesubmol\triesterDev2{
  HC                    (#2)
  (-[::90,1.7,2,2] H_2C (#1))
  -[::-90,1.7,2,2] H_2C
}  
\definesubmol\triesterSat2{
  [:-90] O (-[::180] !\carbonyle #1) !\lhb 
  (!\lh O !\lh !\carbonyle #2) !\lbh
  O !\lh (!\llb O) !\lh 
}
\definesubmol\triester3{
  O *6( (-!\carbonyle #1) -- 
  (-O !\lh !\carbonyle #2) --
  O (- !\carbonyle #3) )
}
\definesubmol\glycero1{ O !\lbh (!\lh O #1) !\lbh O }

%% Hydrogènes saturés
\definesubmol\HH {(-[::90] H) (-[::-90] H)} % paire H- R -H
\definesubmol\HHH{!\HH (-[::0] H)} % triplet H- RH -H
\definesubmol\NH {\chembelow{N}{H}}
\definesubmol\HN {\chemabove{N}{H}}
\definesubmol\CH {\chembelow{C}{H}}
\definesubmol\HC {\chemabove{C}{H}}
%% Quelques groupes caractéristiques
\definesubmol\tete{ !\vide{-90,0.01} } % Pour passer en tête les groupes carboxyle développée
\definesubmol\carboxyle   [ HO-[:30] (!\llh O) !\lb ]{ (!\llh O) (!\lb OH) }
\definesubmol\carboxyleDev[ !\vide{:0,0.7} -O-C (=[::90] O) - ]{ (!\lldh O) (!\ldb OH) }
\definesubmol\carbonyle   { (!\llh O) !\lb }
\definesubmol\carbonyleDev{ (\lldh O) !\ldb }
\definesubmol\ester       { (!\llh O) !\lb O}
\definesubmol\ether       { !\lh O !\lb}
\definesubmol\amide       { (!\llh O) !\lb N}
\definesubmol\phosphate   { P (=[::-90] O) (-[::90] \charge{45:1.5pt=$\scriptstyle-$}{O}) -[::0] O}

%%%% Pour des molécules colorées
\definesubmol\cCouleur1 {-[:: #1,,,, couleurQuat-700, line width = 2.pt]}
\definesubmol\chCouleur {-[ ::60,,,, couleurQuat-700, line width = 2.pt]}
\definesubmol\cbCouleur {-[::-60,,,, couleurQuat-700, line width = 2.pt]}
\definesubmol\cchCouleur{=[ ::60,,,, couleurQuat-700, line width = 2.pt]}
\definesubmol\ccbCouleur{=[::-60,,,, couleurQuat-700, line width = 2.pt]}
\definesubmol\couleur1  {\textcolor{couleurQuat-700}{#1}}


%%%%%%%%%%%%%%%%%%%%%%%%%%%%%%%%%%%%%%%%%%%%%%%%%%%%%%%%%%%%%
%% Pour l'utilisation dans les triglycérides
\definesubmol\tricaproique     { !\lhb !\lhb !\lh }
\definesubmol\trilaurique      { !\lhb !\lhb !\lhb !\lhb !\lhb}
\definesubmol\tripalmitique    { !\lhb !\lhb !\lhb !\lhb !\lhb !\lhb !\lh }
\definesubmol\trilinolenique   { !\lhb !\lhb !\lhb !\cis !\cis !\cis !\lh }
\definesubmol\trioleique       { !\lhb !\lhb !\lhb !\lh!\llh!\lh !\lhb !\lhb !\lb!\lb!\lb }
\definesubmol\trilinoleique    { !\lhb !\lhb !\lhb !\cis !\cis !\lhb !\lhb }
\definesubmol\trieicosapenta   { !\lh !\lh !\cis !\cis !\cis !\cis !\cis !\lh }
\definesubmol\triarachidonique { !\lh !\lh !\cis !\cis !\cis !\cis !\lh !\lhb !\lh }
\definesubmol\tridocosahexa    { !\lh !\cis !\cis !\lh !\llh !\lh !\cis !\cis !\cis !\lh }
%% Formes semi-developpées
\definesubmol\tristeraiqueSemiDev { !\tete !\carboxyleDev C_{17} H_{35} }
\definesubmol\tricaproiqueSemiDev { !\tete !\carboxyleDev CH_2 - CH_2 - CH_2 - CH_2 - CH_3 }
\definesubmol\trioleiqueSemiDev   { !\tete !\carboxyleDev C_{17} H_{33} }
\definesubmol\tripalmitiqueSemiDev{ !\tete !\carboxyleDev C_{15} H_{31} }
%% Acides gras
\definesubmol\caproique         { !\tete !\carboxyle !\tricaproique }
\definesubmol\palmitique        { !\tete !\carboxyle !\tripalmitique !\lb }
\definesubmol\linolenique       { !\tete !\carboxyle !\trilinolenique }
% version trans !\lhb !\lhb !\lhb !\trans !\trans !\trans !\lh
\definesubmol\oleique           { !\tete !\carboxyle !\trioleique }
\definesubmol\linoleique        { !\tete !\carboxyle !\trilinoleique }
\definesubmol\arachidonique     { !\tete !\carboxyle !\triarachidonique }
\definesubmol\eicosaPentaenoique{ !\tete !\carboxyle !\trieicosapenta }
\definesubmol\docosaHexanoique  { !\tete !\carboxyle !\tridocosahexa }
%% Formes semi-développées
\definesubmol\oleiqueSemiDev  { C_{17} H_{33} -C !\carboxyleDev }
\definesubmol\oleateSemiDev   { C_{17} H_{33} -C (!\lldh O) (!\ldb \phantom{'}O^{-}) }
\definesubmol\steraiqueSemiDev{ C_{17} H_{35} -C !\carboxyleDev }
\definesubmol\caproiqueSemiDev{ CH_2 - CH_2 - CH_2 - CH_2 - CH_2 - C !\carboxyleDev }

%%%%%%%%%%%%%%%%%%%%%%%%%%%%%%%%%%%%%%%%%%%%%%%%%%%%%%%%%%%%%
%% Lipide topo
\definesubmol\oleine{
  !\triester{!\trioleique} {!\trioleique} {!\trioleique} 
}
\definesubmol\palmitine{ 
  !\triesterSat{!\lb !\tripalmitique} {!\tripalmitique !\lb} !\lb !\tripalmitique 
}
\definesubmol\arachidonine{
  !\triester{!\triarachidonique} {!\triarachidonique} {!\triarachidonique} 
}
\definesubmol\phosphatidylcholine{
  % choline
  -[::-30] \charge{90:4pt=$+$}{N} (-[::-30])(-[::-90]) !\lhb
  % phosphate 
  !\lh O !\lb P  (=[::-20] O)(-[::-100] \charge{140:2pt=$-$}{O}) !\lh
  % diglycéride
  !\glycero{!\lb !\carbonyle !\trioleique} !\lb (!\llb O) !\trilaurique
}
% Lipide semi-dev
\definesubmol\oleineSemiDev{
  !\triesterDev {!\trioleiqueSemiDev} {!\trioleiqueSemiDev} !\trioleiqueSemiDev
}
\definesubmol\caproineSemiDev{
  !\triesterDev {!\tricaproiqueSemiDev} {!\tricaproiqueSemiDev} !\tricaproiqueSemiDev
}
\definesubmol\palmitineSemiDev{
  !\triesterDev {!\tripalmitiqueSemiDev} {!\tripalmitiqueSemiDev} !\tripalmitiqueSemiDev
}

%% glycérol
\definesubmol\glycerol{[:30] H !\glycero{H} H }
\definesubmol\glycerolSemiDev{
  HC (-OH)
  (-[3,,2,2] H_2C (-OH))
  -[-3,,2,2] H_2C (-OH)
}


%%%%%%%%%%%%%%%%%%%%%%%%%%%%%%%%%%%%%%%%%%%%%%%%%%%%%%%%%%%%%
%% Stérols
\definesubmol\sterol6{
  *6(#1 % 1er cycle
    *6(#2 % 2ème cycle
      *6(- % 3ème cycle
        *5(-- #3) % 4ème cycle
        #4) % 3
      #5) % 2
    #6) % 1
}
\definesubmol\cholesterol{
  HO-[:30] !\sterol {--} {=--} {
    -(-[::-35] (!\lh) !\lb !\lb !\lhb (!\lb) !\lh) -
  } {- (-[::0]) ---} {---} {- (-[::0]) ---}
}
% Glucocorticoïdes
\definesubmol\cortisol{
  O=[:30] !\sterol {-=} {---} {
    -(-[::-100] OH) (-[::-35] (!\llh O) !\lb!\lh OH) -
  }{
    -(-[::0]) -- (-HO) -
  } {--} {- (-[::0]) ---}
}
\definesubmol\corticosterone{
  O=[:30] !\sterol {-=} {---} {
    - (-(!\llh O) !\lb!\lh OH) -
  }{
    -(-[::0] !\llh O) -- (-HO) -
  } {--} {- (-[::0]) ---}
}
% Minéralocorticoïdes
\definesubmol\aldosterone{
  O=[:30] !\sterol {-=} {---} {
    - (-(!\llh O) !\lb!\lh OH) -
  }{
    -(-[::0]) -- (-HO) -
  } {--} {- (-[::0]) ---}
}
% Oestrogènes
\definesubmol\estrone{
  HO-[:30] !\sterol {-=} {---} {-(=O)-} {- (-[::0]) ---} {--} {-=-=}
}
\definesubmol\estriol{
  HO-[:30] !\sterol {-=} {---} {(-OH)-(-OH)-} {- (-[::0]) ---} {--} {-=-=}
}
\definesubmol\estradiol{
  HO-[:30] !\sterol {-=} {---} {-(-OH)-} {- (-[::0]) ---} {--} {-=-=}
}
% Androgènes
\definesubmol\testosterone{
  O=[:30] !\sterol {-=} {---} {-(-OH)-} {- (-[::0]) ---} {---} {- (-[::0]) ---}
}
\definesubmol\dihydrotestosterone{
  O=[:30] !\sterol {--} {---} {-(-OH)-} {- (-[::0]) ---} {---} {- (-[::0]) ---}
}
\definesubmol\androstenedione{
  O=[:30] !\sterol {-=} {---} {-(=O)-} {- (-[::0]) ---} {---} {- (-[::0]) ---}
}
\definesubmol\DHEA{
  O=[:30] !\sterol {--} {=--} {-(=O)-} {- (-[::0]) ---} {---} {- (-[::0]) ---}
}
\definesubmol\DHEAS{
  HO-[:30] S (=[::20] O) (=[::100] O) !\lb O !\lh 
  !\sterol {--} {=--} {-(=O)-} {- (-[::0]) ---} {---} {- (-[::0]) ---}
}
% Progestatif
\definesubmol\progesterone{
  O=[:30] !\sterol {-=} {---} {-(- (!\lh) !\llb O)-} {- (-[::0]) ---} {--} {- (-[::0]) ---}
}


%%%%%%%%%%%%%%%%%%%%%%%%%%%%%%%%%%%%%%%%%%%%%%%%%%%%%%%%%%%%%
%%%% Glucides
%% Pour la représentation de Haworth du glucose ou du fructose
\definesubmol\hexaOseHaw1{
  !\vide{:90,0.01} % Pour avoir la bonne orientation
  %% Bas du cycle
  <[::-140,0.9] (-[::140,0.7] OH) 
  -[::50,1.1,,,line width=3pt] (-[::-90,0.7] OH)
  >[::45,0.9]
  %% haut du cycle
  -[::90,0.9]O -[::45] (#1) -[::40,0.9]
  %% pour retourner sur la droite du cycle
  !\vide{::180,0.9} !\vide{::-40} !\vide{::-45,0.9}
}
\definesubmol\pentaOseHaw2{
  !\vide{:-90,0.01} % Pour avoir la bonne orientation
  ? <[::30] (#1) -[::60,1.3,,,line width= 3pt] (#2) >[::60] -[::90,1.35]O ?
  % on repart à l'envers de l'oxygène pour pouvoir ajouter une chaîne à gauche
  !\vide{::180,1.35}
}
\definesubmol\CHHOH{ -[::-90,0.5] -[::60,0.7,,2] HO }
\definesubmol\gluHaw{ !\hexaOseHaw{!\CHHOH} }
\definesubmol\polymere{ \cdots }

%%%% Amidon
\definesubmol\amylopectineHaw{
  !\polymere - !\gluHaw -O-
  !\hexaOseHaw{-[::-90,0.8] !\lh O -[::-60,1.2] !\gluHaw -O- !\gluHaw -!\polymere}
  -O- !\gluHaw -!\polymere
}

%% Pour faciliter l'écriture d'un sucre en formule développée
\definesubmol{\ose} { -[::0] C (-[::-90] H) (-[::90] OH) }

%% glucose
\definesubmol\glucoseHaw{
  HO -[::90,0.9,2] !\gluHaw -[::135,0.7,,1] OH
}
\definesubmol\glucoseCycle{
  HO -[::30] *6 (-(-OH) -(-OH) -(-OH) -O- (- !\lb OH)-)
}
\definesubmol\glucose{
  H -[::30] (!\llh O) !\lb (!\lb OH) !\lh (!\lh OH) !\lb (!\lb OH) !\lh (!\lh OH) !\lb !\lh OH
}
\definesubmol\glucoseSemiDev{
  C (-[::120] H) (=[::-120] O) !\ose !\ose !\ose !\ose !\ose (-[::0] OH)
}

%% galactose
\definesubmol\galactoseHaw{
  !\vide{::90,0.95} % alignement vertical
  (-[::0,0.9,,2] HO) !\gluHaw -[::135,0.7,,1] OH
}

%% fructose
\definesubmol\fructoseHaw{
  HO -[::90,0.9,2] !\hexaOseHaw{} (-[::135,0.7,,1] OH) -[::-45,0.7] -[::60,0.7,,1] OH
}
\definesubmol\fructofuranoseHaw{
  !\vide{:90,2} % alignement vertical
  HO -[::-120,0.7] -[::-60,0.8]
  !\pentaOseHaw{-[::-30,0.7,,2] HO}{-[::90,0.8,,2] HO\phantom{I}}
  (-[::120,0.7] OH) -[::-60,0.7] -[::60,0.7,,1] OH
}
\definesubmol\fructoseCycle{
  HO -[::30] *6 (-(-OH) -(-OH) -(-[::0] OH) (-[::-90] !\lh OH) -O--)
}
\definesubmol\fructose{
  HO -[::30] !\lb (!\llb O) !\lh (!\lh OH) !\lb (!\lb OH) !\lh (!\lh OH) !\lb !\lh OH
}
\definesubmol\fructoseSemiDev{
  OH -[::0]  C!\HH -[::0] C (=[::90] O) !\ose !\ose !\ose !\ose (-[::0] H)
}

%% Saccharose
\definesubmol\saccharoseHaw{
  % glucose
  HO -[::90,0.9,2] !\gluHaw -[::20] O -[::50] 
  % fructose
  (-[::70,0.5] -[::60,0.7,,2] HO)
  !\pentaOseHaw{-[::-30,0.7,,2] HO}{-[::90,0.8,,2] HO\phantom{I}}
  -[::-60,0.7] -[::60,0.7,,1] OH
}

%%%% Ribose
\definesubmol\ribose{
  % liaison à droite et cycle
  -[::-30] !\lb *5([::70]- (-HO) -(-OH) --O-)
  % imite les angles du cycle pour pouvoir ajouter une chaîne à gauche
  !\vide{::124} !\vide{::-72} -[::128]
}
\definesubmol\riboseHaw{
  -[::-30] !\lb !\pentaOseHaw{!\lb HO}{!\lb OH} -[::120]
}
\definesubmol\desoxyribose{
  % liaison à droite et cycle
  -[::-30] !\lb *5([::70]- (-HO) ---O-)
  % imite les angles du cycle pour pouvoir ajouter une chaîne à gauche
  !\vide{::124} !\vide{::-72} -[::128]
}
\definesubmol\desoxyriboseHaw{
  -[::-30] !\lb !\pentaOseHaw{!\lb HO}{} -[::120]
}

%%%%%%%%%%%%%%%%%%%%%%%%%%%%%%%%%%%%%%%%%%%%%%%%%%%%%%%%%%%%%
%%%% Base nucléique
\definesubmol\adenine { *5(- *6(-N=-N= (-[,,,1]NH_2) -) =-N=-) }
\definesubmol\guanine { *5(- *6(-N= (-[,,,1]NH_2) -[,,,1] NH -[,,1] (=O)-) =-N=-) }
\definesubmol\thymine { *6(- (=O) -[,,,1] NH -[,,1] (=O) -(-)=-) }
\definesubmol\uracile { *6(- (=O) -[,,,1] NH -[,,1] (=O) -=-) }
\definesubmol\cytosine{ *6(- (=O) -N= (-[,,,1]NH_2) -=-) }

%%% Ribonucléoside
\definesubmol\adenosine{ !\ribose N !\adenine }
\definesubmol\cytidine { !\ribose N !\cytosine }
\definesubmol\guanosine{ !\ribose N !\guanine }
\definesubmol\thymidine{ !\ribose N !\thymine }
\definesubmol\uridine  { !\ribose N !\uracile }
%%
\definesubmol\adenosineHaw{ !\riboseHaw N !\adenine }
\definesubmol\cytidineHaw { !\riboseHaw N !\cytosine }
\definesubmol\guanosineHaw{ !\riboseHaw N !\guanine }
\definesubmol\thymidineHaw{ !\riboseHaw N !\thymine }
\definesubmol\uridineHaw  { !\riboseHaw N !\uracile }
%% Desoxyribonucléoside
\definesubmol\desoxyAdenosine{ !\desoxyribose N !\adenine }
\definesubmol\desoxyCytidine { !\desoxyribose N !\cytosine }
\definesubmol\desoxyGuanosine{ !\desoxyribose N !\guanine }
\definesubmol\desoxyThymidine{ !\desoxyribose N !\thymine }
\definesubmol\desoxyUridine  { !\desoxyribose N !\uracile }
%%
\definesubmol\desoxyAdenosineHaw{ !\desoxyriboseHaw N !\adenine }
\definesubmol\desoxyCytidineHaw { !\desoxyriboseHaw N !\cytosine }
\definesubmol\desoxyGuanosineHaw{ !\desoxyriboseHaw N !\guanine }
\definesubmol\desoxyThymidineHaw{ !\desoxyriboseHaw N !\thymine }
\definesubmol\desoxyUridineHaw  { !\desoxyriboseHaw N !\uracile }

%%%% Adenosine Tri-Phosphate et Adenosine Di-Phosphate
\definesubmol\tetePhosphate{ \charge{45:1.5pt=$\scriptstyle -$}{O} -!\phosphate }
\definesubmol\ADP{ !\tetePhosphate -!\phosphate !\adenosine }
\definesubmol\ATP{ !\tetePhosphate -!\phosphate -!\phosphate !\adenosine}
\definesubmol\ADPHaw{ !\tetePhosphate -!\phosphate !\adenosineHaw }
\definesubmol\ATPHaw{ !\tetePhosphate -!\phosphate -!\phosphate !\adenosineHaw}

%%%%%%%%%%%%%%%%%%%%%%%%%%%%%%%%%%%%%%%%%%%%%%%%%%%%%%%%%%%%%
%% Acides alpha aminés, formules topologiques
\definesubmol\acideAmine1{ H_2N -[::30] (#1) !\lb !\carboxyle }
\definesubmol\arginine      { !\acideAmine{!\lh!\lhb!\lh HN -[::-60,,2] (!\lb NH_2) !\llh HN} }
\definesubmol\histidine     { !\acideAmine{!\lh!\lh *5(-N=-HN-=)} }
\definesubmol\lysine        { !\acideAmine{!\lh!\lhb!\lhb NH_3^{+}} }
\definesubmol\aspartique    { !\acideAmine{!\lh!\lh !\carbonyle O^{-}} }
\definesubmol\glutamique    { !\acideAmine{!\lh!\lhb !\carbonyle O^{-}} }
\definesubmol\serine        { !\acideAmine{!\lh!\lh HO} }
\definesubmol\threonine     { !\acideAmine{!\lh (!\lh HO) !\lb} }
\definesubmol\asparagine    { !\acideAmine{!\lh!\lh (!\llb O) !\lh H_2N} }
\definesubmol\glutamine     { !\acideAmine{!\lh!\lhb (!\llb O) !\lh NH_2} }
\definesubmol\cysteine      { !\acideAmine{!\lh!\lh HS} }
\definesubmol\selenocysteine{ !\acideAmine{!\lh!\lh HSe} }
\definesubmol\glycine       { !\acideAmine{} }
\definesubmol\proline       { !\vide{:90} !\vide{:6,0.01} *5(-\!\NH- (- !\carboxyle) ---) }
\definesubmol\alanine       { !\acideAmine{!\lh} }
\definesubmol\valine        { !\acideAmine{!\lh (!\lb) !\lh} }
\definesubmol\isoleucine    { !\acideAmine{!\lh (!\lb) !\lhb} }
\definesubmol\leucine       { !\acideAmine{!\lh!\lh (!\lb) !\lh} }
\definesubmol\methionine    { !\acideAmine{!\lh!\lh!\lb S !\lh} }
\definesubmol\phenylalanine { !\acideAmine{!\lh!\lh *6(=-=-=-)} }
\definesubmol\tyrosine      { !\acideAmine{!\lh!\lh *6(=-=(-OH)-=-)} }
\definesubmol\tryptophane   { !\acideAmine{!\lh!\lh *5(- *6(-=-=-) =-HN-=)} }
%% Acides alpha aminés, formules semi-developpée
\definesubmol\acideAmineSD{ CH (-[::-90] NH_2) -C !\carboxyleDev }
\definesubmol\alanineSemiDev   { H_3C- !\acideAmineSD }
\definesubmol\asparagineSemiDev{ C (=[::-120] O) (-[::120] H_2N) -CH_2 -CH_2 - !\acideAmineSD }
\definesubmol\glycineSemiDev   { H_2C (-[::-90,,2] NH_2) -C !\carboxyleDev }
\definesubmol\cysteineSemiDev  { HS -CH_2 - !\acideAmineSD }
\definesubmol\valineSemiDev    { HC (-[::90,,2,2] H_3C) (-[::-90,,2,2] H_3C) - !\acideAmineSD }
\definesubmol\isoleucineSemiDev{ HC (-[::90,,2] CH_2 -[::90] H_3C) (-[::-90,,2,2] H_3C) - !\acideAmineSD }
%% Acides alpha aminés, représentation de fischer gauche (FL) ou droite (FD)
\definesubmol\acideAmineFL{COOH-[::0] (-[::-90] NH_2) (-[::90] H) -[::0]}
\definesubmol\acideAmineFD{COOH-[::0] (-[::90] NH_2) (-[::-90] H) -[::0]}
\definesubmol\alanineL {[:-90] !\acideAmineFL CH_3}
\definesubmol\alanineD {[:-90] !\acideAmineFD CH_3}
\definesubmol\valineL  {[:-90] !\acideAmineFL !\branche{H_3C}{H} -[::0] CH_3}
\definesubmol\valineD  {[:-90] !\acideAmineFD !\branche{H_3C}{H} -[::0] CH_3}

%% Pour faire des polypeptides en formule topologique H = haut, B = bas
\definesubmol\acideAmineH1{ !\lh (#1) !\lb (!\llb O) !\lh }
\definesubmol\acideAmineB1{ !\lb (#1) !\lh (!\llh O) !\lb }
\definesubmol\arginineH      { !\acideAmineH{!\lh!\lhb!\lh HN -[::-60,,2] (!\lb NH_2) !\llh HN} }
\definesubmol\arginineB      { !\acideAmineB{!\lb!\lbh!\lb HN -[::60,,2]  (!\lh H_2N) !\llb NH} }
\definesubmol\histidineH     { !\acideAmineH{-[::70]-[::-45] *5(-N=-\!\HN-=)} }
\definesubmol\histidineB     { !\acideAmineB{-[::-70]-[::45] *5(-N=-NH-=)} }
\definesubmol\lysineH        { !\acideAmineH{!\lh!\lhb!\lhb NH_3^{+}} }
\definesubmol\lysineB        { !\acideAmineB{!\lb!\lbh!\lbh H_3N^{+}} }
\definesubmol\aspartiqueH    { !\acideAmineH{!\lh!\lb !\carbonyle O^{-}} }
\definesubmol\aspartiqueB    { !\acideAmineB{!\lb!\lh !\carbonyle O^{-}} }
\definesubmol\glutamiqueH    { !\acideAmineH{!\lh!\lhb !\carbonyle O^{-}} }
\definesubmol\glutamiqueB    { !\acideAmineB{!\lb!\lbh !\carbonyle O^{-}} }
\definesubmol\serineH        { !\acideAmineH{!\lh!\lb HO} }
\definesubmol\serineB        { !\acideAmineB{!\lb!\lh OH} }
\definesubmol\threonineH     { !\acideAmineH{!\lh (!\lb HO) !\lh} }
\definesubmol\threonineB     { !\acideAmineB{!\lb (!\lh OH) !\lb} }
\definesubmol\asparagineH    { !\acideAmineH{!\lh!\lb (!\llb O) !\lh NH_2} }
\definesubmol\asparagineB    { !\acideAmineB{!\lb!\lh (!\llh O) !\lb NH_2} }
\definesubmol\glutamineH     { !\acideAmineH{!\lh!\lhb (!\llh O) !\lb H_2N} }
\definesubmol\glutamineB     { !\acideAmineB{!\lb!\lbh (!\llb O) !\lh H_2N} }
\definesubmol\cysteineH      { !\acideAmineH{!\lh!\lb HS} }
\definesubmol\cysteineB      { !\acideAmineB{!\lb!\lh SH} }
\definesubmol\selenocysteineH{ !\acideAmineH{!\lh!\lb HSe} }
\definesubmol\selenocysteineB{ !\acideAmineB{!\lb!\lh SeH} }
\definesubmol\glycineH       { !\acideAmineH{} }
\definesubmol\glycineB       { !\acideAmineB{} }
\definesubmol\prolineH       { *5(-----) !\vide{::54}  !\lh (!\llh O) !\lb}
\definesubmol\prolineB       { *5(-----) !\vide{::-54} !\lb (!\llb O) !\lh}
\definesubmol\alanineH       { !\acideAmineH{!\lh} }
\definesubmol\alanineB       { !\acideAmineB{!\lb} }
\definesubmol\valineH        { !\acideAmineH{!\lh (!\lb) !\lh} }
\definesubmol\valineB        { !\acideAmineB{!\lb (!\lb) !\lh} }
\definesubmol\isoleucineH    { !\acideAmineH{!\lh (!\lb) !\lbh} }
\definesubmol\isoleucineB    { !\acideAmineB{!\lb (!\lb) !\lhb} }
\definesubmol\leucineH       { !\acideAmineH{!\lh!\lb (!\lb) !\lh} }
\definesubmol\leucineB       { !\acideAmineB{!\lb!\lh (!\lb) !\lh} }
\definesubmol\methionineH    { !\acideAmineH{!\lh!\lb!\lh S !\lb} }
\definesubmol\methionineB    { !\acideAmineB{!\lb!\lh!\lb S !\lh} }
\definesubmol\phenylalanineH { !\acideAmineH{!\lh!\lb *6(=-=-=-)} }
\definesubmol\phenylalanineB { !\acideAmineB{!\lb!\lh *6(=-=-=-)} }
\definesubmol\tyrosineH      { !\acideAmineH{-[::70]-[::-45] *6(=-=(-OH)-=-)} }
\definesubmol\tyrosineB      { !\acideAmineB{-[::-70]-[::45] *6(=-=(-HO)-=-)} }
\definesubmol\tryptophaneH   { !\acideAmineH{-[::70]-[::-45] *5(- *6(-=-=-) =-\!\HN-=)} }
\definesubmol\tryptophaneB   { !\acideAmineB{-[::-70]-[::45] *5(- *6(-=-=-) =-NH-=)} }
%% Pour faire des polypeptides en formule semi-développée
%% SD = semi-dev ; H = haut ; B = bas
\definesubmol\acideAmineSDH1{-CH (#1) -C (=[::-90] O)-}
\definesubmol\acideAmineSDB1{-CH (#1) -C (=[::90] O)-}
\definesubmol\cysteineSemiDevH  { !\acideAmineSDH{-[::90]  CH_2 -[::0] HS } }
\definesubmol\cysteineSemiDevB  { !\acideAmineSDB{-[::-90] CH_2 -[::0] HS } }
\definesubmol\glycineSemiDevH   { -CH_2 -C (=[::-90] O)- }
\definesubmol\glycineSemiDevB   { -CH_2 -C (=[::90] O)- }
\definesubmol\alanineSemiDevH   { !\acideAmineSDH{-[::90]  CH_3} }
\definesubmol\alanineSemiDevB   { !\acideAmineSDB{-[::-90] CH_3} }
\definesubmol\isoleucineSemiDevH{ !\acideAmineSDH{-[::90]  CH (!\ldh H_2C -[::-50,,2,2]H_3C) !\ldb CH_3 } }
\definesubmol\isoleucineSemiDevB{ !\acideAmineSDB{-[::-90] CH (!\ldh CH_2 !\ldb CH_3) !\ldb H_3C } }
\definesubmol\valineSemiDevH    { !\acideAmineSDH{-[::90]  CH (!\ldh H_3C) !\ldb CH_3} }
\definesubmol\valineSemiDevB    { !\acideAmineSDB{-[::-90] CH (!\ldb H_3C) !\ldh CH_3} }

%%%% Heme
\definesubmol\hemeB {
  [:-60] !\llb
  !\pyrrole{-N (!\ldb) --(-!\lhb (!\llh O) !\lb HO) =(-) -} % bas droite
  !\llh !\lb
  !\pyrrole{=N (-[::-50,0.9,,,dotted]) --(-) =(-!\lhb (!\llh O) !\lb HO) -} % bas gauche
  !\llh !\lb 
  !\pyrrole{-N (-[::-50,1.4]Fe^{2+}) -=(-!\llh) -(-) =} % haut gauche
  !\lh !\llb 
  !\pyrrole{-N (-[::-50,0.9,,,dotted]) =-(-!\llh) =(-) -} % haut droite
  !\lh
}


%%%%%%%%%%%%%%%%%%%%%%%%%%%%%%%%%%%%%%%%%%%%%%%%%%%%%%%%%%%%%
%% Hormones
\definesubmol\creatinine{
  O= *5(-N (-[-3,0.5]H) -(=NH) -N (-) --)
}
\definesubmol\DOPA{
  HO -[:30] *6(= (-OH) -= (-!\lb (!\lb NH_2) !\lh!\carboxyle) -=-)
}
\definesubmol\DOPAH{
  HO -[:30] *6(= (-OH) -= (-!\lb (!\lb NH_3^+) !\lh!\carboxyle) -=-)
}
\definesubmol\prostaglandine{
  HO-[::75] *5(
    - (-!\llh !\lb (!\lb HO) !\lhb !\lhb !\lh)
    - (-[::-65] !\lbh !\lbh !\lbh !\carboxyle)
    - (=O)
    --
  )
}

%%%%%%%%%%%%%%%%%%%%%%%%%%%%%%%%%%%%%%%%%%%%%%%%%%%%%%%%%%%%%
%% Produit de contraste
\definesubmol\chelate{-[::-70] -[::80] -[::80]} 
\definesubmol\chelateCOO{-[::-120] !\lb (!\llb O) !\lh O^{-}}
\definesubmol\ionChelate{
  [:-10] N (!\vide{::70, 0.7} Gd^{3+})
    (!\chelateCOO) !\chelate
  N (!\chelateCOO) !\chelate
  N (!\chelateCOO) !\chelate 
  N (!\chelateCOO) -[::-70] -[::80] -[::80,0.75]
}
\definesubmol\chelateAlcool{
  [:-10] N (!\vide{::70, 0.7} Gd^{3+})
    (!\chelateCOO) !\chelate
  N (!\chelateCOO) !\chelate
  N (-[::-120] !\lb (!\lb) !\lh OH) !\chelate 
  N (!\chelateCOO) -[::-70] -[::80] -[::80,0.75]
}


%%%%%%%%%%%%%%%%%%%%%%%%%%%%%%%%%%%%%%%%%%%%%%%%%%%%%%%%%%%%%
%% Vitamines
\definesubmol\cret{ !\llb !\lh }
\definesubmol\retinol{ % Vitamine A
  *6( % cycle
    --(-[::0]) (-[::-80]) = ( % chaine
      !\lb !\cret (!\lh) !\cret !\cret (!\lh) !\cret !\lb OH
    ) % chaine
    -(-)--
  ) % cycle
}
\definesubmol\acideAscorbique{ % Vitamine C
  HO-[::-30] !\lh (!\lh OH) !\lb *5(-(-OH) =(-OH) -(=O) -O-)
}
\definesubmol\cholecarciferol{ % Vitamine D
  OH-[::-30]
  *6( % 1er cycle
    ---(=)- ( % ramification
      = !\lb !\llh *6(- % 2eme
        *5(
          --- (-(!\lh) !\lb !\lhb !\lh(!\lh) !\lb) --
        ) % 3eme
        -(-[::0])----
      ) % 2eme
    ) % ramification
    --
  ) % 1er
}
% vitamine B1 thiamine
\definesubmol\thiamine{
  -[::30]  *6(
    -N=- (
      - !\lb N  *5(-(-)= (- !\lh !\lb OH) -S-=) % 2nd cycle
    )
    = (-NH_2) -N=
  ) % 1er cycle
}
% vitamine B2 riboflavine
\definesubmol\riboflavine{
  -[::30]  *6(
    =- ( *6(
        - N ( !\lbh (!\lh OH) !\lb (!\lb OH) !\lh (!\lh OH) !\lb !\lh OH )
      - ( *6(=N- (=O) -NH- (= O)-) )
      -=N-
    ))
    =-= (-) -
  )
}
% vitamine B3 acide nicotinique
\definesubmol\acideNicotinique{
   *6(-N=- (- (!\llh O) !\lb OH) =-=)
}
\definesubmol\nicotinamide{
   *6(-N=- (- (!\llh O) !\lb NH_2) =-=)
}
% vitamine B5 acide panthotenique
\definesubmol\acidePantothenique{
  HO -[::30] !\lb (-[::-90]) (-[::-30])
  !\lh ( !\lh OH )
  !\lb (!\llb O )
  !\lh \!\HN !\lbh !\lb (!\llb O)
  !\lh OH
}
% vitamine B6 pyroxidine
\definesubmol\pyroxidine{
  HO -[::30] !\lb  *6(=-N= (-)- (-OH)= (!\lb OH) -)
}
% vitamine B8 biotine
\definesubmol\biotine{
  \!\NH *5(
    - *5(--S- (- !\lbh !\lb) -)
    --\!\HN- (= HO) -
  )
}
% vitamine B9 acide folique
\definesubmol\acideFolique{
  H_2N -[::30]  *6(
    =N- *6(
      -N=- (
        - !\lb \!\NH !\lh  *6(
          =-= (
            - (!\llh O)
            !\lb \!\NH !\lh (!\lh (!\lh HO) !\llb O)
            !\lb !\lbh (!\lh OH) !\llb O
          )
          -=-
        ) % 3eme cycle
      )
      =N-
    ) % 2nd cycle
    =- (=O) =N-
  ) % 1er cycle
}
% vitamine B12 cyanocobalamine
\definesubmol\pyrrole1{ *5(#1) !\vide{::-54} !\vide{::72} !\vide{::-108,0.01} }
\definesubmol\propanamide[!\lb!\lbh (!\lh H_2N) !\llb O ]{ !\lb!\lbh (!\lh NH_2) !\llb O}
\definesubmol\ethanamide[!\lb!\lb (!\lh H_2N) !\llb O ]{ !\lb!\lb (!\lb NH_2) !\llh O}
\definesubmol\cyanocobalamine{
  [:-60] !\llb
  !\pyrrole{ % bas droite
    -N (-[::-50,0.9,,,->]) =-(!\ldb !\ldh !\ldh (!\lb NH_2) !\llh O) -(-[::-30])(-[::-90]) -
  }
  !\lh (!\lh) !\llb 
  !\pyrrole{ % bas gauche
    -N (-[::-50,1.4]Co^{+} -[::45,,1,1] CN) 
    --(!\ethanamide) -(
      -[::-100] !\lhb (!\llb O) !\lh NH -[::-60,,1] !\lh (!\lb) 
      !\lh O -[::0] !\phosphate !\ldh *5( % cycle du bas ether
        - (-!\lh OH) -O- (-N *5(- *6(=-(-)=(-)-=)
        --N (-[::-39,4.5,,,dotted]) =-)) % Cycle du bas amidine
      -(-HO)-)
    ) (-[::0]) -
  }
  -[::28] -[::0,0.68] !\vide{::-28,0.01} (-[::150]) % longue liaison à gauche
  !\pyrrole{ % haut gauche
    -N (-[::-60,0.9,,,->]) =-(!\propanamide) -(!\ethanamide)(-[::0]) -
  }
  !\lh (!\lh)  !\llb 
  !\pyrrole{ % haut droite
    -N (-[::-50,0.9,,,->]) =-(!\propanamide) -(!\ethanamide)(-[::0]) -
  }
  !\lh
}
% vitamine E
\definesubmol\isoprene{!\llh (!\lh) !\lb}
\definesubmol\phytyle{!\lbh (!\lh) !\lb}
\definesubmol\tocopherol{ % alpha
  -[::30]  *6(
    = (-) -  *6(-O- (
        -!\lh !\phytyle !\lh !\phytyle !\lh !\phytyle
      ) (-[::0]) ---
    ) % 2nd cycle
    =- (-) = (-HO) -
  ) % 1er cycle
}
\definesubmol\tocotrienol{ % alpha
  -[::30]  *6(
    = (-) -  *6(-O- (
        -!\lhb !\isoprene !\lhb !\isoprene !\lhb !\isoprene
      ) (-[::0]) ---
    ) % 2nd cycle
    =- (-) = (-HO) -
  ) % 1er cycle
}
% vitamine K1
\definesubmol\menatetrenone{
  *6(-= 
    *6(- (=O)- (-)= (
      - !\lb !\isoprene !\lhb !\isoprene !\lhb !\isoprene !\lhb !\isoprene
    )- (=O) -)
  -=-=)
}
\definesubmol\phylloquinone{
  *6(-= 
    *6(- (=O)- (-)= (
      - !\lb !\isoprene !\lh !\phytyle !\lh !\phytyle !\lh !\phytyle
    )- (=O) -)
  -=-=)
}

%%%%%%%%%%%%%%%%%%%%%%%%%%%%%%%%%%%%%%%%%%%%%%%%%%%%%%%%%%%%%
%% Aspirine
\definesubmol\aspirineSemiDev{
  HC *6( % cycle
    -\CH =CH -C (
      -O !\lb C (!\llb O) !\lh CH_3 % chaine latérale
    )
    =C (-C !\carboxyle) -[,,,2]HC =[,,2]
  ) % cyle
}
\definesubmol\aspirine{
  *6 (-=- (-O !\lb (!\llb O) !\lh) = (-!\carboxyle) -=)
}
\definesubmol\acideSalicylique{
  *6 (-=- (-OH) = (-!\carboxyle) -=)
}

%% Paracétamol
\definesubmol\paracetamol{
  HO -[::30] *6(
    -=- (
      -\!\HN (!\lb (!\llb O) !\lh) % amide
    )
    =-=
  )
}
\definesubmol\paracetamolSemiDev{
  HO -[::30] C *6(
    -\CH =CH -C (
      -\!\HN (!\lb C (!\llb O) -CH_3) % amide
    )
    =\HC -HC =[,,2]
  )
}
\definesubmol\paracetamolDev{
  H -[::0] O -[::30] C *6(
    -C (-H) =C (-H) -C (
      -N (!\lh H) (!\lb C (!\llb O) (-C !\HHH)) % amide
    )
    =C (-H) -C (-H) =
  )
}

%% Aspartame
\definesubmol\aspartame{
  HO -[::90,,2] (!\llh O) !\lb!\lb (!\lb NH_2) % chaine latérale
  !\lh (!\llh O) !\lb \!\NH !\lh % amide
  *6(- (=O) -O (-) !\vide{} *6(-=-=-) =--) % double cycle
}

%%%%%%%%%%%%%%%%%%%%%%%%%%%%%%%%%%%%%%%%%%%%%%%%%%%%%%%%%%%%%
%%%% Molécules odorantes
\definesubmol\geraniol{
  -[::30] (!\lb) !\llh *6( % pied
    !\vide{} !\vide{} (- !\lb OH) % chaine latérale
    =(-)--- %fin du cycle
  )
}
\definesubmol\geraniolSemiDev{
  H_3C - C (- CH_3) =[::90] CH % pied
  !\lh H_2C -[::-60,,2,2] H_2C !\lb C % "cycle"
  (!\lh CH_3) !\llb !\CH -[::30] CH_2 -OH % chaînes latérales
}
\definesubmol\vanilline{
  HO -[::90,,2] *6(= (-O !\lh) -= (- !\carbonyle H) -=-)
}
\definesubmol\ethylvanilline{
  HO -[::90,,2] *6(= (-O !\lhb) -= (- !\carbonyle H) -=-)
}
\definesubmol\oxyphenylone{
  HO -[::30] *6(-=- (-!\lbh (!\llh O) !\lb) =-=)
}
\definesubmol\limonene{ 
  -[:30] (!\llb) !\lh *6(--=(-)---)
}
\definesubmol\limoneneSemiDev{
  H_3C -[:30] C (!\llb CH_3) !\lh \HC 
  *6(-CH_2 -CH =C (-CH_3) -[,,,2]H_2C -[,,2,2]H_2C -)
}
\definesubmol\acetateIsoamyle{
  -[:30] (!\lh) !\lbh !\lb O !\lh (!\llh O) !\lb
}

%%%%%%%%%%%%%%%%%%%%%%%%%%%%%%%%%%%%%%%%%%%%%%%%%%%%%%%%%%%%%
%%%% Drogues
\definesubmol\THC{
  -[::30] *6( % cyle ether
    (-[::120]) -O- *6( % cycle chaine latérale et alcool
      -= (!\lbh !\lbh !\lb) -= (-OH) -
    )
    =- *6(-=(-)---) -- % cycle supérieur
  )
}
\definesubmol\cocaineHaw{
  ? 
    <[::60,0.7] (!\lh N !\lh)
    -[::-60,,,,line width = 3pt]
    >[::-30,0.7] ( % ether-phenyl
      !\lh O !\lb (!\llb O) !\lh *6(=-=-=-)
    )
    -[::130,0.7] ( % ester
      !\ldb (!\llh O)
      !\lb O !\lh 
    )
    -[::80, 0.9] (-[::-30, 0.85]) -[::60, 0.7] 
  ?
}
\definesubmol\bisphenolA{
  HO-[:30] *6(-=- (- (-[::0]) (-[::120]) !\lb *6(-=- (-OH) =-=)) =-=)
}
\definesubmol\bisphenolASemiDev{
  HO-[:30] C *6(-\CH =CH -[,,,1]C (
    -C (-[::25]CH_3) (-[::95]H_3C) !\lb C 
      *6(-[,,,2]HC =\CH -C (-OH) =CH -[,,,1]\HC =C)
  ) =\HC -HC =[,,2])
}

\newcommand{\largeurCaseTableauPeriodique}{1.5}

%%%% Pour afficher un élément dans le tableau périodique
\NewDocumentCommand{\elementTexteCharge}{m m m o}
{
  \begin{minipage}{\largeurCaseTableauPeriodique cm}
    \begin{center}
      \IfValueTF{#4}{ \vAligne{-20pt} }{ \vAligne{-34pt} } % position du nom
      {\small #3} \\[2pt] % nom de l'élément
      {\ensuremath\footnotesize \textbf{#1}} \\[6pt] % nombre atomique
      \chemfig[atom style={scale = 1.8}]{#2} % symbole atomique
      % \element{#1}{#2} % element symbol and atomic number
      \IfValueT{#4}{
        \\ {\small \qty{#4}{\g/\mole}}
      }
    \end{center}
  \end{minipage}
}

%%%% Pour afficher un élément dans le tableau périodique
\NewDocumentCommand{\elementElectroneg}{m m}
{
  \begin{minipage}{\largeurCaseTableauPeriodique cm}
    \begin{center}
      {\Large \important[black]{#1} \\[2pt]} % symbole atomique
      {\small $\chi = \num{#2}$} % électronégativité
    \end{center}
  \end{minipage}
}


%%%% Pour afficher un tableau périodique
%% #1 : largeur ; #2 : hauteur ; #3 : élements
\NewDocumentCommand{\tableauPeriodique}{O{2.6} O{2.7} m}{
\begin{tikzpicture}[font=\sffamily, scale=0.75, transform shape]

%% Type d'élément, par famille
  \tikzstyle{Alcali} = [Element, fill=green-200]
  \tikzstyle{Alcalo} = [Element, fill=green-150]
  \tikzstyle{Metaux} = [Element, fill=green-100]
  \tikzstyle{Metoid} = [Element, fill=orange-100]
  \tikzstyle{NoMeta} = [Element, fill=orange-150]
  \tikzstyle{Haloge} = [Element, fill=orange-200]
  \tikzstyle{GazRar} = [Element, fill=red-150]

%% Type d'élément, par électronégativité
 \tikzstyle{elec1} = [Element, fill=green-50]
 \tikzstyle{elec2} = [Element, fill=green-100!80]
 \tikzstyle{elec3} = [Element, fill=yellow-100]
 \tikzstyle{elec4} = [Element, fill=orange-100]
 \tikzstyle{elec5} = [Element, fill=orange-150]
 \tikzstyle{elec6} = [Element, fill=orange-200]
 \tikzstyle{elec7} = [Element, fill=red-200]
 \tikzstyle{elec8} = [Element, fill=red-300]
  
%% Style des éléments
  \tikzstyle{Element} = [
    draw=black, cyan-800!50!black,
    minimum width  = #1 cm, % Largeur de la case
    node distance  = #1 cm, % Espace entre deux case
    minimum height = #2 cm, % Hauteur de la case
  ]

%% Période, groupe et titre
  \tikzstyle{Period} = [font={\sffamily\LARGE}, node distance=2cm]
  \tikzstyle{Groupe} = [font={\sffamily\LARGE}, minimum width=2.5cm, node distance=2cm]
  \tikzstyle{Titre}  = [font={\sffamily\Huge\bfseries}]

%% Place des éléments
  #3
\end{tikzpicture}
}


%%%% Pour faciliter l'utilisation du tableau périodique
\newcommand{\elementH} {\elementTexteCharge{1} {H} {Hydrogène}[1,00]}
\newcommand{\elementHe}{\elementTexteCharge{2} {He}{Hélium}   [4,00]}
\newcommand{\elementLi}{\elementTexteCharge{3} {Li}{Lithium}  [6,94]}
\newcommand{\elementBe}{\elementTexteCharge{4} {Be}{Béryllium}[9,01]}
\newcommand{\elementB} {\elementTexteCharge{5} {B} {Bore}     [10,8]}
\newcommand{\elementC} {\elementTexteCharge{6} {C} {Carbone}  [12,0]}
\newcommand{\elementN} {\elementTexteCharge{7} {N} {Azote}    [14,0]}
\newcommand{\elementO} {\elementTexteCharge{8} {O} {Oxygène}  [16,0]}
\newcommand{\elementF} {\elementTexteCharge{9} {F} {Fluor}    [19,0]}
\newcommand{\elementNe}{\elementTexteCharge{10}{Ne}{Néon}     [20,2]}
\newcommand{\elementNa}{\elementTexteCharge{11}{Na}{Sodium}   [23,0]}
\newcommand{\elementMg}{\elementTexteCharge{12}{Mg}{Magnésium}[24,3]}
\newcommand{\elementAl}{\elementTexteCharge{13}{Al}{Aluminium}[27,0]}
\newcommand{\elementSi}{\elementTexteCharge{14}{Si}{Silicium} [28,1]}
\newcommand{\elementP} {\elementTexteCharge{15}{P} {Phosphore}[31,0]}
\newcommand{\elementS} {\elementTexteCharge{16}{S} {Soufre}   [32,1]}
\newcommand{\elementCl}{\elementTexteCharge{17}{Cl}{Chlore}   [35,5]}
\newcommand{\elementAr}{\elementTexteCharge{18}{Ar}{Argon}    [39,9]}
\newcommand{\elementK} {\elementTexteCharge{19}{K} {Potassium}[39,1]}
\newcommand{\elementCa}{\elementTexteCharge{20}{Ca}{Calcium}  [40,0]}


%%%% Palettes de couleur
\palette{couleurPrim}{cyan}
\palette{couleurSec} {blue}
\palette{couleurTer} {purple}
\palette{couleurQuat}{red}


%%%% Réglages de la taille des indentations et des sauts de paragraphes
\setlength{\parskip}{0cm}
\setlength{\parindent}{0cm}
\renewcommand{\baselinestretch}{1}
% réglage du niveau (sous-section) ou s'arrête la table des matières
\setcounter{tocdepth}{2}


%%%% Réglage de la géométrie des pages
\geometry{
  a4paper, % format
  left=1.3cm, right=1.3cm, % marge horizontale
  top=2.2cm, bottom=2.1cm % marge verticale
}


%%% Apparence (couleur) des liens
\hypersetup{
  colorlinks=true,
  linkcolor=black, % lien type table des matière
  citecolor=black, % citation
  filecolor=black, 
  urlcolor=couleurPrim!10!black % lien internet
}


%%%% Réglage de tikz (flèche et caractères)
\usetikzlibrary{babel}
\tikzset{>=latex}


%%%% Réglage des en-tête
\renewcommand{\headrulewidth}{0.4pt}
\setlength{\headheight}{22.50113pt}


%%%% Réglage de dashundergaps pour avoir des points et pas de numération
\dashundergapssetup{
  gap-numbers = false,
  gap-format = dot,
  gap-widen,
  gap-extend-percent
}


%%%% Réglage de siunit
\sisetup{
  locale = FR, % français
	 group-minimum-digits = 4, % groupage des chiffres par millier
  inter-unit-product = \ensuremath { { } \cdot { } }, % point médian entre les unités,
  detect-weight, propagate-math-font = true, reset-math-version = false % pour avoir les versions grasse des typo
}
\AtBeginDocument{\RenewCommandCopy\qty\SI} % Pour "écraser" la commande \qty du package physics
%% Pour rendre silencieux le warning précisant d'ajouter la commande au dessus
\ExplSyntaxOn
\msg_redirect_name:nnn { siunitx } { physics-pkg } { none }
\ExplSyntaxOff
