%%%% french character
\usepackage[french]{babel}
\usepackage[T1]{fontenc}
\usepackage[utf8]{inputenc}

%%%% useful packages
\usepackage[a4paper, left=1.25cm, right=1.25cm, top=2.2cm, bottom=2.3cm]{geometry}
\usepackage{subcaption} % for figure caption
\usepackage{graphicx} % image
\usepackage{tabularx} % table
\usepackage[table]{xcolor} % color in table
\usepackage{amsmath} % math
\usepackage{amssymb} % bold math
\usepackage{wasysym} % integral
\usepackage[many]{tcolorbox} % colored box
\usepackage{fancyhdr} % headers
\usepackage{enumitem} % for bullet in itemize
\usepackage[colorlinks=true,linkcolor=black,citecolor=black,filecolor=black,urlcolor=black]{hyperref} % for link
\usepackage{accents} % for complex notation
\usepackage[european, straightvoltages, RPvoltages]{circuitikz} % for electronic circuit
\usepackage{multicol} % to use several columns
\usepackage{fontawesome} % awesome icons
\usepackage{ifthen} % for loop and boolean in commands
\usepackage{pdfpages} % to include pdf
\usepackage{wrapfig} % to wrap text around figures
\usepackage{chemfig} % to draw chemistry formula
\usepackage{multirow} % for vertically merged cells
\usepackage{makecell} % to format cell in tables
\usepackage{physics} % for derivatives, braket, etc.
\usepackage{esvect} % for large vectors
\usepackage{listings} % for code
% dyslexia friendly font (need to be compiled in xetex)
%\usepackage{fontspec}
%\setmainfont{OpenDyslexic}


%%%% settings
\setlength{\parskip}{0cm}
\setlength{\parindent}{0cm}
\renewcommand{\baselinestretch}{1}
\setcounter{tocdepth}{2}


%%%% tikz configuration
\usetikzlibrary{babel}
\tikzset{>=latex}


%%%% header
\renewcommand{\headrulewidth}{0.4pt}
\setlength{\headheight}{22.50113pt}


%%%% Table
\renewcommand{\tabularxcolumn}[1]{m{#1}}
\setlength{\extrarowheight}{8pt}


%%%% Chemfig configuration
\setchemfig{
  atom sep=20pt,
  bond style={line width=1pt},
  angle increment=30
}