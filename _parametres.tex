%%%% Pour avoir les accents et autre caractère français
\usepackage[french]{babel}
\usepackage[T1]{fontenc}
\usepackage[utf8]{inputenc}

%%%% Paquets utilisé
\usepackage{ifthen} % pour programmer avec des boucle et des conditions
%% Images/dessin
\usepackage{subcaption} % pour les légendes des figures
\usepackage{graphicx} % pour insérer des images
\usepackage[european, straightvoltages, RPvoltages]{circuitikz} % pour dessiner des circuits électrique
\usepackage{pdfpages} % pour inclure des fichiers pdf
\usepackage{wrapfig} % pour entourer les images par du texte 
\usepackage{chemfig} % pour dessiner des formules chimiques
\usepackage{fontawesome} % pour dessiner de jolies icônes
%% Mise en page
\usepackage{geometry} % définition des marges
\usepackage{dashundergaps} % pour avoir générer des textes à compléter
\usepackage{fancyhdr} % pour faire des en-tête
\usepackage[many]{tcolorbox} % pour faire de jolie boîtes colorée
\usepackage{enumitem} % pour pouvoir définir des listes personnalisées
\usepackage{hyperref} % pour insérer des liens
\usepackage{multicol} % pour avoir plusieurs colonnes côte-à-côte
\usepackage{listings} % pour insérer du code
%% Tableau
\usepackage{tabularray} % pour avoir de meilleurs tableaux
%% Math
\usepackage{amsmath} % symboles mathématiques
\usepackage{amssymb} % symboles mathématiques en gras
\usepackage{wasysym} % pour avoir des symbole d'intégrale
\usepackage{accents} % pour les notations mathématiques avec une barre
\usepackage{physics} % pour les dérivées, les bra, les kets, etc.
\usepackage{esvect} % pour faire de jolis vecteurs
\usepackage{siunitx} % pour avoir de jolie grandeurs avec des unités 
%% Décommenter pour avoir une police spéciale dysléxie (doit être compilé en XeTeX)
% \usepackage{fontspec}
% \setmainfont{OpenDyslexic}


%%%% Réglages de la taille des indentations et des sauts de paragraphes
\setlength{\parskip}{0cm}
\setlength{\parindent}{0cm}
\renewcommand{\baselinestretch}{1}
% réglage du niveau (sous-section) ou s'arrête la table des matières
\setcounter{tocdepth}{2}


%%%% Réglage de la géométrie des pages
\geometry{
  a4paper, % format
  left=1.3cm, right=1.3cm, % marge horizontale
  top=2.2cm, bottom=2.3cm % marge verticale
}


%%%% Réglage de chemfig
\setchemfig{
  atom sep=26pt,
  bond style={line width=1pt},
  angle increment=30
}


%%% Apparence (couleur) des liens
\hypersetup{
  colorlinks=true,
  linkcolor=black, % lien type table des matière
  citecolor=black, % citation
  filecolor=black, 
  urlcolor=blue!90!black % lien internet
}


%%%% Réglage de tikz (flèche et caractères)
\usetikzlibrary{babel}
\tikzset{>=latex}


%%%% Réglage des en-tête
\renewcommand{\headrulewidth}{0.4pt}
\setlength{\headheight}{22.50113pt}


%%%% Réglage de dashundergaps pour avoir des points et pas de numération
\dashundergapssetup{
  gap-numbers = false,
  gap-format = dot,
  gap-widen,
  gap-extend-percent
}


%%%% Réglage de siunit
\sisetup{
  locale = FR, % français
	 group-minimum-digits = 4, % groupage des chiffres par millier
  inter-unit-product = \ensuremath { { } \cdot { } } % point médian entre les unités
}
\AtBeginDocument{\RenewCommandCopy\qty\SI} % Pour "écraser" la commande \qty du package physics