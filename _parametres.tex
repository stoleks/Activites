%%%% Pour avoir les accents et autre caractère français
\usepackage[french]{babel}
\usepackage[T1]{fontenc}
\usepackage[utf8]{inputenc}

%%%% Paquets utilisé
\usepackage{ifthen} % pour programmer avec des boucle et des conditions
%% Images/dessin
\usepackage{subcaption} % pour les légendes des figures
\usepackage{graphicx} % pour insérer des images
\usepackage[european, straightvoltages, RPvoltages]{circuitikz} % pour dessiner des circuits électrique
\usepackage{pdfpages} % pour inclure des fichiers pdf
\usepackage{wrapfig} % pour entourer les images par du texte 
\usepackage{chemfig} % pour dessiner des formules chimiques
\usepackage{fontawesome} % pour dessiner de jolies icônes
%% Mise en page
\usepackage{geometry} % définition des marges
\usepackage{dashundergaps} % pour avoir générer des textes à compléter
\usepackage{fancyhdr} % pour faire des en-tête
\usepackage[many]{tcolorbox} % pour faire de jolie boîtes colorée
\usepackage{enumitem} % pour pouvoir définir des listes personnalisées
\usepackage{hyperref} % pour insérer des liens
\usepackage{multicol} % pour avoir plusieurs colonnes côte-à-côte
\usepackage{listings} % pour insérer du code
%% Tableau
\usepackage{tabularray} % pour avoir de meilleurs tableaux
%% QR code
\usepackage{qrcode}
%% Math
\usepackage{amsmath} % symboles mathématiques
\usepackage{amssymb} % symboles mathématiques en gras
\usepackage{wasysym} % pour avoir des symbole d'intégrale
\usepackage{accents} % pour les notations mathématiques avec une barre
\usepackage{physics} % pour les dérivées, les bra, les kets, etc.
\usepackage{esvect} % pour faire de jolis vecteurs
\usepackage{siunitx} % pour avoir de jolie grandeurs avec des unités 


%%%% Commandes prédéfinies
%%%%%%%%%%%%%%%%%%%%%%%%%%%%%%%%%%%%%%%%%%%%%%%%%%%%%%%%%%%%%%%%%%%%%%%%%%
%%%% quelque couleurs
\definecolor{vertSombre}  {RGB} {  0,  95,  17}
\definecolor{vertForet}   {RGB} {124, 179,  66}
%
\definecolor{jauneSombre} {RGB} {213, 145,   2}
\definecolor{orangeSombre}{RGB} {174,  82,   0}
\definecolor{cyanSombre}  {RGB} {  0, 140, 128}
\definecolor{bleuPale}    {RGB} { 39,  76, 167}
%
\definecolor{jauneClair} {RGB} {218, 173,   0}
\definecolor{rougeSombre}{RGB} {148,  31,   0}
\definecolor{rougeClair} {RGB} {224,  39,  34}

%%% quelques couleurs dérivées des couleurs choisie
\newcommand{\couleurPrimSombre}{couleurPrim!60!black}

%%%% rectangle coloré
\NewDocumentCommand{\rectangle}{O{couleurPrim} m m}{%
  \shorthandoff{;}
  \tikz \node (rect) [draw, fill, color=#1,
              minimum width=#2,
              minimum height=#3] {};
  \shorthandon{;}
}


%%%%%%%%%%%%%%%%%%%%%%%%%%%%%%%%%%%%%%%%%%%%%%%%%%%%%%%%%%%%%%%%%%%%%%%%%%%
%%%% une simple boite vide
\newtcolorbox{boite}[1][]{
  breakable, enhanced jigsaw, % pour s'étendre sur plusieurs pages
  arc = 0mm, % les lignes de la boites sont droites
  colback = white, colframe = black, % fond blanc et traits noirs
  #1
}

%%%% boite colorée
\newtcolorbox{boiteColoree}[1][]{
  breakable, enhanced jigsaw, % pour s'étendre sur plusieurs pages
  arc = 2mm, % les lignes de la boites sont droites
  colback = couleurPrim, colframe = white, % fond et traits colorés
  coltext = white,
  #1
}

%%%% document
\newcounter{documentNum}
\newtcolorbox{doc}[3][]{
  before title = {\refstepcounter{documentNum}},
  breakable, enhanced jigsaw, % pour s'étendre sur plusieurs pages
  colback = white, % fond blanc
  colframe = couleurPrim!25!black, % couleur de la boite
  coltitle = black, % couleur du titre
  boxrule = 0.5mm, arc = 0.5mm, % largeur et arrondi des traits de la boite
  titlerule = 0mm, top = 0mm, % pour ne pas avoir de séparation titre/boite
  colbacktitle = white, % fond pour le titre blanc
  fonttitle = \bfseries\sffamily,
  title = {Document \arabic{documentNum} -- #2\strut \label{#3}},
  #1
}

%%%% Passage important à connaître
\newtcolorbox{encart}[1][]{
  breakable, enhanced jigsaw, % pour s'étendre sur plusieurs pages
  frame hidden, sharp corners, boxrule = 0mm, % pas de contours
  colback = couleurPrim!10, % fond
  borderline west={4pt}{0pt}{couleurPrim}, % barre gauche
  #1
}

%%%% Boite de correction
\newtcolorbox{boiteCorrection}[1][]{
  breakable, enhanced jigsaw, % pour s'étendre sur plusieurs pages
  frame hidden, sharp corners, boxrule = 0mm, % pas de contours
  colback = couleurPrim!10, % fond
  #1
}

%%%% contexte
\newtcolorbox{contexte}[1][]{
  breakable, enhanced jigsaw, % pour s'étendre sur plusieurs pages
  boxrule = 3pt, sharp corners, % contours droits
  colframe = couleurPrim, % couleur des contours
  colback = couleurPrim!5, % fond
  title = {Contexte :}, % titre
  colbacktitle = couleurPrim!5, % couleur du fond du titre
  fonttitle = \bfseries\sffamily, coltitle = black, %
  titlerule = 0mm, top = 0mm, % pour ne pas avoir de séparation titre/boite
  detach title, before upper={\vspace{2pt}\hspace{-8pt}\tcbtitle\;},
  #1
}

%%%% Pour les objectifs
\newtcolorbox{boiteObjectifs}[2][]{
  empty, % pas de boite automatique
  attach boxed title to top left = {yshift=-2.5mm}, % position
  boxed title style = {empty, size = small, top = 1mm, bottom = 0.5mm},
  frame code = { % tracé de la boite
    \path (title.east |- frame.north) coordinate (aux);
    \path [draw=couleurPrim, line width = 3pt]
    (frame.west) |- ([xshift=-4mm] title.north east)
    to[out=0, in=180] ([xshift=10mm] aux) -| % définit la courbe
    (frame.east) |- (frame.south) -| cycle; % trace la boite
  },
  coltitle = black, % couleur du titre
  fonttitle = \bfseries\sffamily, % police du titre
  title = {#2},
  #1 
}
\newenvironment{objectifs}{
  \begin{boiteObjectifs}{Objectifs de la séance :}
    \begin{listeObjectifs}
}{
    \end{listeObjectifs}
  \end{boiteObjectifs}
}

%%%% Espace pour un coup de pouce
\newcounter{coupDePouceNum}
\newtcolorbox{coupDePouce}[1][]{
  before title = {\refstepcounter{coupDePouceNum}},
  breakable, enhanced jigsaw, % pour s'étendre sur plusieurs pages
  arc = 0mm, % les lignes de la boites sont droites
  colback = white, colframe = black, % fond blanc et traits noirs
  fonttitle = \bfseries, coltitle = black, % couleur et police du titre
  titlerule = 0mm, top = 0mm, % pour ne pas avoir de séparation titre/boite
  colbacktitle = white, % fond pour le titre blanc
  title = {
    \textcolor{couleurPrim}{\faThumbsUp}
    Coup de pouce \arabic{coupDePouceNum} :
    \flushright \vspace*{-26pt}\faSquareO
  },
  #1
}

%%%% Espace pour une appréciation
\newcommand{\appreciation}[1]{
  \pasCorrection{
    \begin{boite}
      \vspace*{-4pt}
      \sousTitre{Appréciation et remarques}
      
      \vspace*{#1 cm}
      \phantom{b}
    \end{boite}
  }
}

%%%% Espace fiche TP
\newtcolorbox{boiteMateriel}[2][]{
  colback = white, colframe = black, % fond blanc et traits noirs
  coltitle = white, % couleur du titre
  boxrule = 0.5mm, arc = 0.5mm, % largeur et arrondi des traits de la boite
  titlerule = 0mm, top = 0mm, % pour ne pas avoir de séparation titre/boite
  colbacktitle = couleurPrim, % fond pour le titre blanc
  fonttitle = \bfseries\sffamily, % type de titre
  title = {\centering \large #2}, % titre
  #1
}

%%%%%%%%%%%%%%%%%%%%%%%%%%%%%%%%%%%%%%%%%%%%%%%%%%%%%%%%%%%%%%%%%%%%%%%%%%
%%%% pagination et sections
\newcommand{\titre}[1]{
  \begin{center}
    \textsf{\bfseries \Large #1}
  \end{center}
}
\newcommand{\sousTitre}[1]{
  \textsf{\bfseries #1}
}
\newcommand{\pasDePagination}{
  \thispagestyle{empty}
}
\newcommand{\feuilleBlanche}{
  \newpage
  \phantom{b}
  \pasDePagination
}
    

%%%% activité ou TP
\newcounter{activiteNum}
\newcommand{\titreActi}[2]{
  \refstepcounter{activiteNum}
  \titre{#1 \arabic{section}.\arabic{activiteNum} -- #2}
}
\newcommand{\titreTP}[1]{
  \titreActi{TP}{#1}
  % \titreActi{Activité expérimentale}{#1}
}
\newcommand{\titreActivite}[1]{
  \titreActi{Activité}{#1}
}
\NewDocumentCommand{\titreEvaluation}{o m}{
  \IfNoValueTF {#1}{
    \titre{Évaluation \arabic{section} -- #2}
  }{
    \titre{Évaluation #1 -- #2}
  }
  % reset du numéro de page et d'exercices
  \setcounter{page}{1}
  \numeroActivite{1}
}
\newcounter{exerciceNum}
\newcommand{\exercice}[1]{
  \refstepcounter{exerciceNum}
  \sousTitre{\large Exercice \arabic{exerciceNum} : #1}
  % reset des numéros de questions
  \setcounter{questionNum}{0}
  \setcounter{documentNum}{0}
}


%%%% chapitre, section et sous-section
\newcommand{\titreChapitre}[1]{
  \titre{Chapitre \arabic{section} : #1}
}
\newcommand{\titrePartie}[1]{
  \vspace*{24pt}
  \refstepcounter{subsection}
  \rectangle{40pt}{1pt}
  \sousTitre{\Large \Roman{subsection} -- #1}
  \rectangle{40pt}{1pt}
  \vspace*{10pt}
}
\newcounter{sousSectionNum}
\newcommand{\titreSection}[1]{
  \vspace*{16pt}
  \refstepcounter{subsubsection}
  \setcounter{sousSectionNum}{0}
  \rectangle{30pt}{4pt}
  \sousTitre{\large \arabic{subsubsection} -- #1}
  \vspace*{10pt}
}
\newcommand{\titreSousSection}[1]{
  \vspace*{12pt}
  \refstepcounter{sousSectionNum}
  \sousTitre{\Alph{sousSectionNum} -- #1}
  \vspace*{8pt}
}

%%%% fixe le numéro de l'activité
\newcommand{\numeroActivite}[1]{
  % fixe les compteurs LaTeX
  \setcounter{page}{1}
  \setcounter{subsection}{0}
  \setcounter{subsubsection}{0}
  \setcounter{figure}{0}
  % fixe les compteurs internes
  \setcounter{qcmNum}{0}
  \setcounter{documentNum}{0}
  \setcounter{questionNum}{0}
  \setcounter{coupDePouceNum}{0}
  \setcounter{sousSectionNum}{0}
  \setcounter{activiteNum}{#1 - 1}
}
% fixe le numéro de partie (#1) et le numéro de la page (#2)
\newcommand{\numeroPartieCours}[2]{
  \newpage
  \setcounter{subsection}{#1 - 1}
  \setcounter{page}{#2}
}

%%%% lignes
\newcommand{\ligne}{
  \par\noindent\rule{\textwidth}{0.4pt}
}
\newcommand{\lignePointillee}[1]{
  \makebox[#1\linewidth]{\dotfill}
}


%%%%%%%%%%%%%%%%%%%%%%%%%%%%%%%%%%%%%%%%%%%%%%%%%%%%%%%%%%%%%%%%%%%%%%%%%%
%%%% Paramètre par défaut pour l'en-tête
\newcommand{\annee}{Réglez avec \textbackslash renewcommand\{\textbackslash annee\}\{2023 -- 2024\}}
\newcommand{\classe}{Réglez avec \textbackslash renewcommand\{\textbackslash classe\}\{Seconde\}}
\newcommand{\etablissement}{Réglez avec \textbackslash renewcommand\{\textbackslash etablissement\}\{Lycée\}}

%%%% en-tête
\newcommand{\teteGauche}[2]{
  \lhead{
    \textbf{\footnotesize #1}
    \newline
    \footnotesize #2
  }
}
\newcommand{\teteDroite}[2]{
  \rhead{
    \hfill \textbf{\footnotesize #1}
    \newline \hfill
    \footnotesize #2
  }
}
\newcommand{\enTete}[3]{
  \pagestyle{fancy}
  \setcounter{section}{#3}
  \setcounter{subsection}{0}
  \setcounter{sousSectionNum}{0}
  \teteGauche{\etablissement{}}{Chapitre \arabic{section} -- #1} % left header
  \chead{} % central header
  \teteDroite{\annee{}}{#2} % right header
}


%%%%%%%%%%%%%%%%%%%%%%%%%%%%%%%%%%%%%%%%%%%%%%%%%%%%%%%%%%%%%%%%%%%%%%%%%%
%%%% exercice
% définit un booléen pour entrer ou sortir du mode correction
\newboolean{modeProf}
\setboolean{modeProf}{false}
\newcommand{\modeCorrection}{
  \setboolean{modeProf}{true}
  \TeacherModeOn
}

% Pour afficher le numéro d'une question avec choix du compteur
\NewDocumentCommand{\numeroQuestion}{O{questionNum} O{16}}{
  \refstepcounter{#1}
  \setcounter{sousQuestionNum}{0}
  \vspace*{2pt}
  \ifnum \thequestionNum > 9
    \hspace{6 pt}
  \else
    \hspace{#2 pt}
  \fi
  \textcolor{couleurSec}{
    \textbf{\arabic{#1}} {\small\faMinus}
  }
}

\newcommand{\numeroSousQuestion}{
  \refstepcounter{sousQuestionNum}
  \hspace{16 pt}
  \textcolor{couleurSec}{
    \textbf{\arabic{questionNum}.\arabic{sousQuestionNum}.}
  }
}


% trace des lignes pointillées pour répondre aux questions
% \lignessDeReponse* complète la ligne actuelle par des pointillées
% \lignesDeReponse commence à la ligne suivante
\newcounter{ligneNum}
\NewDocumentCommand{\lignesDeReponse}{s m}{
  % Trace la fin de la ligne, ou pas
  \IfBooleanTF{#1}{ % Version *
    \espaceReponse \dotfill\phantom{bb}
    \ifnum #2 < 1
      \newline
    \fi
  }{}
  % Trace le bon nombre de lignes
  \setcounter{ligneNum}{-1}
  \loop
    \stepcounter{ligneNum}
    \ifnum \value{ligneNum} < #2
      \\[8pt] \lignePointillee{0.98}
  \repeat
  \vspace*{1pt}
}


% définit une commande pour afficher une question 
% #1 : question
% #2 : réponse
% #3 : nombres de lignes pour répondre
\newcounter{questionNum}
\newcounter{sousQuestionNum}
\newcommand{\question}[3]{
  \numeroQuestion \!#1
  % pointille ou correction
  \ifthenelse {\boolean{modeProf}} { % prof
    \begin{boiteCorrection}
      #2
    \end{boiteCorrection}
  }{ % eleve
    \lignesDeReponse{#3}
  }
}

% Affiche le contenu en mode correction
\newcommand{\correction}[1]{
  \ifthenelse {\boolean{modeProf}} { % correction
    #1
  }{}
}

% Affiche le contenu si on est pas en mode correction
\newcommand{\pasCorrection}[1]{
  \ifthenelse{\boolean{modeProf}} {}{ % pas correction
    #1
  }
}

% Point associé à une question
\newcommand{\points}[1]{
  \marginnote{#1}
}


% sous questions
\newcommand{\sousQuestion}[2]{
  \hspace{16pt}
  \textcolor{couleurSec}{\textbullet} #1
  
  \vspace*{8pt}
  \reponse{#2}
}

% question QCM
\newcommand{\QCM}[2]{
  \numeroQuestion[qcmNum][0] #1
  \begin{qcm}
    #2
  \end{qcm}
}

% À ajouter devant la bonne réponse dans un qcm
\newcounter{qcmNum}
\newcommand{\reponseQCM}{
  \correction{
    \hspace*{-15pt}$\checkmark$\hspace*{-12pt}
  } % Note : trace une croix à la bonne position
}

%%%% Pour afficher les compétences
\newcommand{\competence}[1]{
  ~{\footnotesize\textit{(#1)}}
}

%%%% Espace pour indiquer nom, prénom et classe
\newcommand{\nomPrenomClasse}{
  \pasCorrection{
    \vspace*{-24pt}
    Nom : \lignePointillee{0.3}
    Prénom : \lignePointillee{0.3}
    Classe : \dotfill
  }
}
\newcommand{\nomPrenom}{
  \pasCorrection{
    \vspace*{-24pt}
    Nom : \lignePointillee{0.3}
    Prénom : \lignePointillee{0.3}
  }
}


%%%%%%%%%%%%%%%%%%%%%%%%%%%%%%%%%%%%%%%%%%%%%%%%%%%%%%%%%%%%%%%%%%%%%%%%%%
% texte à trou avec option pour régler la largeur
\NewDocumentCommand{\texteTrou}{o +m}{
  \ifthenelse {\boolean{modeProf}}{ % prof
    \important[black]{#2}
  }{ % élève
    \IfValueTF{#1}{ % Si la largeur est réglée, on utilise des lignes
      \espaceReponse
      \lignePointillee{#1}
      \hspace*{-12pt}
    }{ % Sinon on utilise dash undergap pour la version automatique
      \espaceReponse \hspace*{0.1pt}
      \gap{#2}
    }
  }
}

% texte à trou avec option pour laisser plusieurs lignes
\NewDocumentCommand{\texteTrouLignes}{O{0} +m}{
  \ifthenelse {\boolean{modeProf}} {% prof
    \important[black]{#2}
  }{% élève
    \lignesDeReponse*{#1}
  }
}

% espace vertical pour la réponse
\newcommand{\espaceReponse}{
  \phantom{$\dfrac{1}{1}$} % espace vertical
  \hspace*{-38pt} \phantom{b} % ajuste l'espace horizontal
}


%%%%%%%%%%%%%%%%%%%%%%%%%%%%%%%%%%%%%%%%%%%%%%%%%%%%%%%%%%%%%%%%%%%%%%%%%%
%%%% Pour choisir parmi deux sujets
\newboolean{sujetA}
\setboolean{sujetA}{true}
\newcommand{\sujetB}{
  \setboolean{sujetA}{false}
}
\newcommand{\sujetA}{
  \setboolean{sujetA}{true}
}

%%%% Pour faire plusieurs sujets en parallèle
\newcommand{\variationSujet}[2]{
  \hspace*{-6pt}
  \ifthenelse {\boolean{sujetA}}{#1}{#2}
  \hspace*{-6pt}
}


%%%%%%%%%%%%%%%%%%%%%%%%%%%%%%%%%%%%%%%%%%%%%%%%%%%%%%%%%%%%%%%%%%%%%%%%%%
%%%% Tableau générique avec la première ligne bleue
\NewDocumentEnvironment{tableau}{m}{
  \begin{center}
    \begin{tblr}{
      hlines,
      colspec = #1,
      row{1} = {couleurPrim!20},
    }
}{
    \end{tblr}
  \end{center}
}

%%%% Tableau de competence
\newenvironment{tableauCompetences}{
  \centering
  \begin{tblr}{
    colspec = {| c | X[l] | c | c | c | c |},
    rows = {m}, hlines,
    row{1} = {couleurPrim!20}
  }
    \textbf{Compétences} & \centering \textbf{Items} 
    & \textbf{D} & \textbf{C} & \textbf{B} & \textbf{A} \\
}{
  \end{tblr}
}

%%%% Tableau de connaissances sans exercices
\newenvironment{tableauConnaissances}{
  \centering
  \begin{tblr}{
    colspec = {Q[t,wd=0.7\textwidth] c c c},
    rows = {m}, hlines, vlines,
    column{4} = {0.2},
    row{1} = {couleurPrim!20, c}
  }
    \textbf{Connaissances et capacités exigibles} & \ok & \pasOk & \textbf{En classe} \\
}{ 
  \end{tblr}
}


%%%% Alignement dans un tableau
\newcommand{\vAligne}[1]{
  \strut \\ \vspace*{#1}
}


%%%%%%%%%%%%%%%%%%%%%%%%%%%%%%%%%%%%%%%%%%%%%%%%%%%%%%%%%%%%%%%%%%%%%%%%%%
%%%% symboles : chevron, flèche, attention, etc.
\NewDocumentCommand{\chevron}{O{couleurPrim}}{
  \textcolor{#1}{\small \faChevronRight}
}
%
\NewDocumentCommand{\fleche}{O{couleurPrim}}{
  \textcolor{#1}{\faCaretRight}
}
%
\NewDocumentCommand{\attention}{O{couleurPrim}}{
  \textcolor{#1}{\faExclamationTriangle}
}
%
\NewDocumentCommand{\flecheLongue}{O{couleurPrim}}{
  \textcolor{#1}{\faLongArrowRight}
}
%
\NewDocumentCommand{\ok}{O{couleurPrim}}{
  \textcolor{#1}{\faCheckCircle}
}
%
\NewDocumentCommand{\pasOk}{O{couleurPrim}}{
  \textcolor{#1}{\faTimesCircle}
}
%
\NewDocumentCommand{\pointCyan}{O{couleurPrim}}{
  \textcolor{#1}{\textbullet}
}
%
\NewDocumentCommand{\mesure}{O{couleurPrim}}{
  \hspace{15pt}
  %\numeroQuestion
  \textcolor{couleurSec}{\faWrench\faFlask}
}
% pictogramme sécurité
\newcommand{\picto}[2]{
  \image{#1}{images/pictogrammes/picto_#2}
}
% Nombre dans un cercle
\newcommand*\nombreCercle[1]{
  % \tikz[baseline=(char.base)]{
  %   \node [shape=circle, draw filled, inner sep=1.2pt, color=couleurSec!20] (char) {\textcolor{black}{#1};
  % }
  \important[couleurSec]{#1}
}
% Pour légender une image
\newcommand{\legende}[1]{
  \textcolor{couleurPrim}{\faArrowUp} \; #1
}

%%%%%%%%%%%%%%%%%%%%%%%%%%%%%%%%%%%%%%%%%%%%%%%%%%%%%%%%%%%%%%%%%%%%%%%%%%
%%%% emphase
\newcommand{\emphase}[1]{
  \textcolor{couleurSec}{\textsf{\bfseries \large #1}}
}
%
\NewDocumentCommand{\important}{O{couleurSec!75!black} m}{
  \!\textcolor{#1}{\textsf{\bfseries #2}}\!\!
}
%
\newcommand{\exemple}{
  \flecheLongue \textit{Exemple :}
}
\newcommand{\exemples}{
  \flecheLongue \textit{Exemples :}
}
%
\newcounter{compteAppelProf}
\newcommand{\appelProf}{
  \refstepcounter{compteAppelProf}
  \hspace{24pt} \faHandPaperO \hspace{2pt}
  \textbf{Appel n$^\circ$ \arabic{compteAppelProf} :}
}

%
\newcommand{\extrait}[2]{
  « #1 »
  
  \vspace*{-12pt}
  \begin{flushright}
    \textit{#2}
  \end{flushright}
  \vspace*{-12pt}
}


%%%% image
\newcommand{\image}[2]{
  \includegraphics[width=#1\linewidth]{#2}
}

%%%% qr code en insert sur la droite
\NewDocumentCommand{\QRCode}{o m}{
  \IfNoValueTF{#1} {
    \begin{wrapfigure}{r}{0.1\linewidth}
      \vspace*{-16pt}
      \qrcode{#2}
    \end{wrapfigure}
  }{
    \begin{wrapfigure}[#1]{r}{0.1\linewidth}
      \vspace*{-16pt}
      \qrcode{#2}
    \end{wrapfigure}
  }
}


%%%%%%%%%%%%%%%%%%%%%%%%%%%%%%%%%%%%%%%%%%%%%%%%%%%%%%%%%%%%%%%%%%%%%%%%%%
%%%% qcm
\newlist{qcm}{itemize}{2}
\setlist[qcm]{label=$\square$, leftmargin=2cm}

%%%% liste d'objectif
\newlist{listeObjectifs}{itemize}{2}
\setlist[listeObjectifs]{label = \chevron}

%%%% protocole
\newlist{protocole}{itemize}{2}
\setlist[protocole]{label = {\footnotesize \fleche[couleurSec]}}

%%%% liste de points
\newlist{listePoints}{itemize}{2}
\setlist[listePoints]{label = \pointCyan}

%%%% liste tirets
\newlist{listeTirets}{itemize}{2}
\setlist[listeTirets]{label = \textcolor{couleurPrim}{\small\faMinus}}

%%%% liste avec des flèches
\newlist{listeFleche}{itemize}{2}
\setlist[listeFleche]{label = \textbf{\flecheLongue}}

%%%% jeu de données
\newenvironment{donnees}{
  
  \textbf{Données :}
  \vspace*{-8pt}
  \begin{multicols}{2}
    \begin{listeTirets}
}{
    \end{listeTirets}
  \end{multicols}
}

%%%% problematique
\newcommand{\problematique}[1]{
  \hspace{8pt}
  \flecheLongue
  \textbf{#1}
}

%%%% liste avec chiffre
\newlist{enumeration}{enumerate}{2}
\setlist[enumeration]{label = \textcolor{\couleurPrimSombre}{\textbf{\arabic*.}} }


%%%%%%%%%%%%%%%%%%%%%%%%%%%%%%%%%%%%%%%%%%%%%%%%%%%%%%%%%%%%%%%%%%%%%%%%%%
%%%% Séparation de la page en blocs
\newcommand{\separationTroisBlocs}[3]{
  \begin{minipage}[T]{0.3\linewidth}
    #1
  \end{minipage}
  ~
  \begin{minipage}[T]{0.3\linewidth}
    #2
  \end{minipage}
  ~
  \begin{minipage}[T]{0.3\linewidth}
    #3
  \end{minipage}
}
%%%% Separation en deux blocs
\NewDocumentCommand{\separationBlocs}{+m O{0.48} +m O{0.48}}{
  \begin{minipage}[T]{#2\linewidth}
    #1
  \end{minipage}
  \hfill
  \begin{minipage}[T]{#4\linewidth}
    #3
  \end{minipage}
}


%%%%%%%%%%%%%%%%%%%%%%%%%%%%%%%%%%%%%%%%%%%%%%%%%%%%%%%%%%%%%%%%%%%%%%%%%%
%% nombre algébrique, réaction
\newcommand{\algebrique}[1]{
  \overline{\mathrm{#1}}
}
\newcommand{\reaction}{
  \!\!\schemestart \arrow(.mid east--.mid west){->}[, 0.9, ultra thick] \schemestop\!\!
}

%% Pour simplifier l'écriture des formules brutes
\newcommand{\bruteCHO}[3]{
  \chemfig{C_{#1} H_{#2} O_{#3}}
}

%% pour les masse molaire et atomique
\newcommand{\masseMol}[1]{
  M(\chemfig{#1})
  % M_{\chemfig{#1}}
}
\newcommand{\masseAtom}[1]{
  m(\chemfig{#1})
  % m_{\chemfig{#1}}
}


%% Unités
\DeclareSIUnit{\dioptre}{\text{$\delta$}}
\DeclareSIUnit{\dornic}{\text{\textdegree D}}
\DeclareSIUnit{\ppm}{\text{ppm}}
\DeclareSIUnit{\COeq}{\text{kgCO$_{2}$e}}
\DeclareSIUnit{\jour}{\text{jour}}
% \DeclareSIUnit{}{\text{}}


%% atome ou isotope #1: Z, #2: A, #3: X
\makeatletter
\newcommand{\isotope}[3]{%
   \settowidth\@tempdimb{\ensuremath{\scriptstyle#1}}%
   \settowidth\@tempdimc{\ensuremath{\scriptstyle#2}}%
   \ifnum\@tempdimb>\@tempdimc%
       \setlength{\@tempdima}{\@tempdimb}%
   \else%
       \setlength{\@tempdima}{\@tempdimc}%
   \fi%
  \begingroup%
  \ensuremath{
    ^{\makebox[\@tempdima][r]{\ensuremath{\scriptstyle#1}}}
    _{\makebox[\@tempdima][r]{\ensuremath{\scriptstyle#2}}}
    \chemfig{#3}
  }%
  \endgroup%
}%
\makeatother

%% element chimique dans le tableau périodique
\makeatletter
\newcommand{\element}[2]{%
   \settowidth\@tempdimb{\ensuremath{\footnotesize #1}}%
  \begingroup%
  \ensuremath{
    _{\makebox[\@tempdimb][r]{\ensuremath{\small #1}}} 
    \chemfig[atom style={scale=1.3}]{#2}
  }%
  \endgroup%
}%
\makeatother

%% siècle
\newcommand{\siecle}[1]{
  \textsc{\romannumeral #1}\textsuperscript{e}~siècle
}

%% texte avec une boite autour
\NewDocumentCommand{\texteEncadre}{m O{black}}{
  \textcolor{#2}{
    \frame{
      \vphantom{$\dfrac{1}{10}$} \textcolor{black}{\text{#1}}
    }
  }
}

%% case cochée
\newcommand{\caseCochee}{
  $\text{\rlap{$\checkmark$}}\square$
}


%%%%%%%%%%%%%%%%%%%%%%%%%%%%%%%%%%%%%%%%%%%%%%%%%%%%%%%%%%%%%%%%%%%%%%%%%%
%%%% Couleur pour le code
\definecolor{vertCode}  {rgb}{0.2,0.6,0}
\definecolor{grisCode}  {rgb}{0.5,0.5,0.5}
\definecolor{violetCode}{rgb}{0.58,0,0.82}
\definecolor{fondCode}  {rgb}{0.95,0.95,0.92}
%%%% Style python
\lstdefinestyle{codePython}{
  backgroundcolor=\color{fondCode},
  commentstyle=\color{magenta},
  keywordstyle=\color{vertCode},
  numberstyle=\tiny\color{grisCode},
  stringstyle=\color{violetCode},
  basicstyle=\ttfamily\footnotesize,
  breakatwhitespace=false,
  breaklines=true,
  captionpos=b,
  keepspaces=true,
  numbers=left,
  numbersep=5pt, 
  showspaces=false,
  showstringspaces=false,
  showtabs=false, 
  tabsize=2
}
\def\inline{\lstinline[style=codePython,language=python]}


%%%%%%%%%%%%%%%%%%%%%%%%%%%%%%%%%%%%%%%%%%%%%%%%%%%%%%%%%%%%%%%%%%%%%%%%%%
%%%% circuit tikz
\NewDocumentCommand{\fixedvlen}{O{0.5cm} m m O{}}{% [semilength]{node}{label}[extra options]
  % get the center of the standard arrow
  \coordinate (#2-Vcenter) at ($(#2-Vfrom)!0.5!(#2-Vto)$);
  % draw an arrow of a fixed size around that center and on the same line
  \draw[-Triangle, #4] ($(#2-Vcenter)!#1!(#2-Vfrom)$) -- ($(#2-Vcenter)!#1!(#2-Vto)$);
  % position the label as in the normal voltages
  \node[anchor=\ctikzgetanchor{#2}{Vlab}, #4] at (#2-Vlab) {#3};
}

%%%%%%%%%%%%%%%%%%%%%%%%%%%%%%%%%%%%%%%%%%%%%%%%%%%%%%%%%%%%%
%% grandeurs récurrente
% Physique
\newcommand{\gISS}{g_\text{ISS}}
\newcommand{\MTerre}{M_\text{Terre}}
\newcommand{\RTerre}{R_\text{Terre}}
\newcommand{\inertie}{\text{inertie}}
\newcommand{\Tfus}{ T_{\text{f}} }
\newcommand{\Teb}{ T_{\text{éb}} }
% Chimie
\newcommand{\solute}{\text{soluté}}
\newcommand{\solution}{\text{solution}}
\newcommand{\espece}{\text{espèce}}

%% vecteurs
\newcommand{\FBsurA}{F_{B/A}}
\newcommand{\FAsurB}{F_{A/B}}
\newcommand{\vvFAsurB}{\vv{F}_{A/B}}
\newcommand{\vvFBsurA}{\vv{F}_{B/A}}

%%%%%%%%%%%%%%%%%%%%%%%%%%%%%%%%%%%%%%%%%%%%%%%%%%%%%%%%%%%%%
%%%% figures simples
\newcommand{\tkzRect}[4]{
  \fill[color=#1] (#2,#4) -- (-#2,#4) -- (-#2,#3) -- (#2,#3);
}
\newcommand{\tkzEllipse}[4]{
  \fill[color=#1] (0,#3) ellipse (#2 and #4);
}
\newcommand{\tkzLegende}[4]{
  \draw[black, ->, very thick] (#2 + #4, #3) node[right] {#1} -- (#2, #3);
}
\newcommand{\tkzCercle}[4]{
  \filldraw [#3] (#1, #2) circle (#4pt);
}
\newcommand{\tkzCercleLigne}[5]{
  \filldraw [color = #4, fill = #3, very thick] (#1, #2) circle (#5pt);
}

%%%% tube à essais
\newcommand{\tkzTubeEssai}[3]{
  \draw[thick] (#1,#2) -- (#1,0) arc (0:-180:#1) -- (-#1,#2);
  \draw[thick] (0,#2) ellipse (#1 and #3);
}
\newcommand{\tkzBasTubeEssai}[5]{
  \fill[color=#1] (-#2,#3) -- (#2,#3) arc (0:-180:#2);
  \tkzRect{#1}{#2}{#3 - 0.01}{#4}
  \tkzEllipse{#1!85!black}{#2}{#4}{#5}
}
\newcommand{\tkzPhaseTubeEssai}[5]{
  \tkzRect{#1}{#2}{#3}{#4}
  \tkzEllipse{#1}{#2}{#3}{#5}
  \tkzEllipse{#1!85!black}{#2}{#4}{#5}
}

%%%% Point et vecteurs
\newcommand{\tkzLabel}[3]{
  \node at (#1, #2) {#3};
}
\newcommand{\tkzPointLabel}[3]{
  \filldraw (#1, #2) circle (2pt) node[above] {#3};
}
\newcommand{\tkzVecteur}[6]{
  \draw[black, ->, very thick] (#1, #2) -- (#1 + #3, #2 + #4) node[#6] {#5};
}
\newcommand{\tkzEquiv}[6]{
  \draw[black, <->, thick] (#1, #2) -- (#1 + #3, #2 + #4);
  \draw[black] (#1 + #3/2, #2 + #4/2) node[#6] {#5};
}
\newcommand{\tkzVecteurX}[4]{
  \draw[black, ->, very thick] (#1, #2) -- (#1 + #3, #2) node[above] {#4};
}
\newcommand{\tkzVecteurY}[4]{
  \draw[black, ->, very thick] (#1, #2) -- (#1, #2 + #3) node[right] {#4};
}
\newcommand{\tkzEquivY}[5]{
  \draw[#1, <->, thick] (#2, #3) -- (#2, #3 + #4);
  \draw[#1] (#2, #3 + #4/2) node[right] {#5};
}
\newcommand{\tkzEquivX}[5]{
  \draw[#1, <->, thick] (#2, #3) -- (#2 + #4, #3);
  \draw[#1] (#2 + #4/2, #3) node[below] {#5};
}


%%%%%%%%%%%%%%%%%%%%%%%%%%%%%%%%%%%%%%%%%%%%%%%%%%%%%%%%%%%%%
%%%% plan de classe
\newcommand{\rectangleTexte}[5]{
  \filldraw [fill=white, draw=black, ultra thick] (#1, #2) rectangle (#1 + #3, #2 + #4);
  \node at (#1 + #3/2, #2 + #4/2) [font=\sffamily] {\textbf{\large #5}};
}
% place dans la classe
\newcommand{\place}[3]{
  \rectangleTexte{#1}{#2}{3}{2}{#3}
}
\newcommand{\deuxPlaces}[4]{
  \place{#1}{#2}{#3}
  \place{#1 + 3}{#2}{#4}
}
\newcommand{\troisPlaces}[5]{
  \place{#1}{#2}{#3}
  \place{#1 + 3}{#2}{#4}
  \place{#1 + 6}{#2}{#5}
}
\newcommand{\quatrePlaces}[6]{
  \place{#1}{#2}{#3}
  \place{#1 + 3}{#2}{#4}
  \place{#1 + 6}{#2}{#5}
  \place{#1 + 9}{#2}{#6}
}
%%%% rangée
\newboolean{quatrePlace}
\setboolean{quatrePlace}{false}
\newcommand{\avecQuatrePlaces}{ \setboolean{quatrePlace}{true} }
\newcommand{\avecTroisPlaces} { \setboolean{quatrePlace}{false} }
\newcommand{\rangee}[9]{
  \ifthenelse {\boolean{quatrePlace}} {
    \deuxPlaces  {0}{#1} {#2}{#3}
    \quatrePlaces{7}{#1} {#4}{#5}{#6}{#7}
    \deuxPlaces  {20}{#1}{#8}{#9}
  }{
    \deuxPlaces {#1}{#2}     {#3}{#4}
    \troisPlaces{#1 + 7}{#2} {#5}{#6}{#7}
    \deuxPlaces {#1 + 17}{#2}{#8}{#9}
  }
}
\newcommand{\rang}[9]{
  \ifthenelse {\boolean{quatrePlace}} {
    \deuxPlaces  {0} {12 - 3*#1} {#2}{#3}
    \quatrePlaces{7} {12 - 3*#1} {#4}{#5}{#6}{#7}
    \deuxPlaces  {20}{12 - 3*#1} {#8}{#9}
  }{
    \deuxPlaces {0} {12 - 3*#1} {#2}{#3}
    \troisPlaces{7} {12 - 3*#1} {#4}{#5}{#6}
    \deuxPlaces {17}{12 - 3*#1} {#7}{#8}
  }
}

%%%% TP
\newcommand{\paillasse}[3]{
  \rectangleTexte{#1}{#2}{4}{2}{#3}
}
\newcommand{\rangeeTP}[6]{
  \paillasse{#1}{#2}     {#3}
  \paillasse{#1 + 4}{#2} {#4}
  \paillasse{#1 + 12}{#2}{#5}
  \paillasse{#1 + 16}{#2}{#6}
}
%%%% Ce fichier sert à déclarer les titres des chapitres des différents niveaux

%% Seconde
%%%% Chapitre 
\newcommand{\sndCorp} {Corps purs et mélanges}
\newcommand{\sndSolu} {Solutions}
\newcommand{\sndMouv} {Mouvement et interactions}
\newcommand{\sndAtom} {Structure de l'atome}
\newcommand{\sndMole} {Des atomes à la matière}
\newcommand{\sndLumi} {Ondes lumineuses et optique}
\newcommand{\sndTran} {Transformations de la matière et nucléaires}
\newcommand{\sndChim} {Transformations chimiques}
\newcommand{\sndSign} {Signaux et capteurs}

%%%% en-tête correspondant
\newcommand{\teteSndCorp} {\newpage \enTete{\sndCorp}{1} }
\newcommand{\teteSndSolu} {\newpage \enTete{\sndSolu}{2} }
\newcommand{\teteSndMouv} {\newpage \enTete{\sndMouv}{3} }
\newcommand{\teteSndAtom} {\newpage \enTete{\sndAtom}{4} }
\newcommand{\teteSndMole} {\newpage \enTete{\sndMole}{5} }
\newcommand{\teteSndLumi} {\newpage \enTete{\sndLumi}{6} }
\newcommand{\teteSndTran} {\newpage \enTete{\sndTran}{7} }
\newcommand{\teteSndChim} {\newpage \enTete{\sndReac}{8} }
\newcommand{\teteSndSign} {\newpage \enTete{\sndSign}{9} }


%% Première ST2S
%%%% Chapitres
\newcommand{\premStssElec} {Sécurité chimique et électrique dans l'habitat}
\newcommand{\premStssRout} {Sécurité routière}
\newcommand{\premStssSono} {Ondes sonores et audition}
\newcommand{\premStssVisi} {Propagation de la lumière et vision}
\newcommand{\premStssPres} {Propriétés des fluides et pression sanguine}
\newcommand{\premStssComp} {Contrôle de la composition des milieux biologiques}
\newcommand{\premStssEner} {Besoins énergétiques et alimentation}
\newcommand{\premStssBiom} {Biomolécules dans l’organisme}
\newcommand{\premStssAlim} {Gestion des ressources naturelles et alimentation}

%%%% en-tête
\newcommand{\tetePremStssElec} {\newpage \enTete{\premStssElec}{1} }
\newcommand{\tetePremStssRout} {\newpage \enTete{\premStssRout}{2} }
\newcommand{\tetePremStssSono} {\newpage \enTete{\premStssSono}{3} }
\newcommand{\tetePremStssVisi} {\newpage \enTete{\premStssVisi}{4} }
\newcommand{\tetePremStssPres} {\newpage \enTete{\premStssPres}{5} }
\newcommand{\tetePremStssComp} {\newpage \enTete{\premStssComp}{6} }
\newcommand{\tetePremStssEner} {\newpage \enTete{\premStssEner}{7} }
\newcommand{\tetePremStssBiom} {\newpage \enTete{\premStssBiom}{8} }
\newcommand{\tetePremStssAlim} {\newpage \enTete{\premStssAlim}{9} }



%% Terminale ST2S
%%%% Chapitres
\newcommand{\termStssOrga} {Représentation des molécules organiques}
\newcommand{\termStssRout} {Sécurité routière}
\newcommand{\termStssAlim} {Sécurité physico-chimique dans l'alimentation}
\newcommand{\termStssEnvi} {Sécurité chimique dans l'environnement}
\newcommand{\termStssImag} {La physique au service de l'imagerie médicale}
\newcommand{\termStssDosa} {Contrôle de la composition des milieux naturels}
\newcommand{\termStssBiom} {Biomolécules et alimentation}
\newcommand{\termStssMedi} {De la molécule au médicament}
\newcommand{\termStssCosm} {L'usage responsable des cosmétiques}

%%%% en-tête
\newcommand{\teteTermStssOrga} {\newpage \enTete{\termStssOrga}{0} }
\newcommand{\teteTermStssRout} {\newpage \enTete{\termStssRout}{1} }
\newcommand{\teteTermStssAlim} {\newpage \enTete{\termStssAlim}{2} }
\newcommand{\teteTermStssEnvi} {\newpage \enTete{\termStssEnvi}{3} }
\newcommand{\teteTermStssImag} {\newpage \enTete{\termStssImag}{4} }
\newcommand{\teteTermStssDosa} {\newpage \enTete{\termStssDosa}{5} }
\newcommand{\teteTermStssBiom} {\newpage \enTete{\termStssBiom}{6} }
\newcommand{\teteTermStssMedi} {\newpage \enTete{\termStssMedi}{7} }
\newcommand{\teteTermStssCosm} {\newpage \enTete{\termStssCosm}{8} }

%%%%%%%%%%%%%%%%%%%%%%%%%%%%%%%%%%%%%%%%%%%%%%%%%%%%%%%%%%%%%
%%%% Bouts de molécules fréquement utilisés
%% Hydrogène saturés
\definesubmol\paireH{(-[::90] H) (-[::-90] H)}
\definesubmol\paireSatH{(-[::30] H) (-[::-30] H)}
\definesubmol\saturationH{(-[::90] H) (-[::-90] H) (-[::0] H)}

%% Quelques groupes caractéristiques
\definesubmol\carboxyle{(=[:90] O) (-[:-30] OH)}
\definesubmol\carbonyle{(=[::60] O) -[::-60]}
\definesubmol\ester{(=[:90] O) -[:-30] O}
\definesubmol\ether{-[:30] O -[:-30]}
\definesubmol\amide{(=[:90] O) -[:-30] N}

%% parties colorées
\definesubmol\cetoneCouleur{(=[3,,,,couleurQuat] \textcolor{couleurQuat}{O}) -[-1,,,,couleurQuat]}
%% ramification
\definesubmol\alkyleG{(-[-5] R_1)}
\definesubmol\alkyleD{(-[-1] R_2)}

%%%% Élément récurrent, pour faciliter la lecture
\newcommand{\hydrogene}{\chemfig{H}}
\newcommand{\carbone}{\chemfig{C}}
\newcommand{\oxygene}{\chemfig{O}}
\newcommand{\azote}{\chemfig{N}}
\newcommand{\eau}{\chemfig{H_2O}}
\newcommand{\oxonium}{\chemfig{H_3O^+}}
\newcommand{\hydroxyde}{\chemfig{HO^{-}}}
\newcommand{\azoture}{\chemfig{NaN_3}}


%%%% Molécule courante
\definesubmol\paracetamol{
  *6((-HO)-=-(-NH (-[::-60] (=[::-60]O)-[::60]))=-=)
}


%%%% Acide gras
\definesubmol\cc{
  -[::60] -[::-60]
}
\definesubmol\teteAcide{
  O-[::30] (=[::60]O) -[::-60]
}
\definesubmol\teteAcideDev{
  - O - C (=[::90] O) -
}
\definesubmol\cis{
  -[::60] =[::-60] -[::-60]
}
\definesubmol\trans{
  -[::60] =[::-30] -[::-30]
}
\definesubmol\palmitique{
  !\carbonyle !\cc !\cc !\cc !\cc !\cc !\cc !\cc
}
\definesubmol\oleique{
  !\teteAcide !\cc !\cc !\cc !\cis -[::60] !\cc !\cc !\cc
}
\definesubmol\trioleique{
  !\carbonyle !\cc !\cc !\cc -[::60] =[::60] -[::60] !\cc !\cc -[::-60] -[::-60] -[::-60]
}
\definesubmol\alphaLinoleique{
  !\teteAcide !\cc !\cc !\cc !\trans !\trans !\trans -[::60]
}
\definesubmol\alphaLinolenique{
  !\teteAcide !\cc !\cc !\cc !\cis !\cis !\cis -[::60]
}
\definesubmol\steraique{
  !\teteAcideDev C_{17}H_{35}
}
\definesubmol\caproique{
  !\teteAcideDev - CH_2 - CH_2 - CH_2 - CH_2 - CH_3
}
\definesubmol\trioleine{
   (-[::150] -[::60] O-[::-60] !\trioleique)
   (-[::-90] -[::-60] O-[::60] !\trioleique)
   -[::30] O-[::60] !\trioleique
}
\definesubmol\tripalmitine{
   (-[::150] -[::60] O-[::-60] (=[::60] O) -[::-60] -[::-60] !\cc !\cc !\cc !\cc !\cc !\cc -[::60]) % gauche
   (-[::-90] -[::60] O-[::-60] (=[::-60]O) -[::60] -[::60] !\cc !\cc !\cc !\cc !\cc !\cc -[::60]) % droite
   -[::30] O-[::60] !\palmitique % bas
}
%% glycerol
\definesubmol\glycerol{HO -[-1] -[1] (-[3] OH) -[-1] -[1] OH}

%%%%
\newcommand{\elementText}[3]
{
  \begin{minipage}{2.2cm}
    \begin{center}
      \element{#1}{#2} % element symbol and atomic number
      \\[8 pt]
      \small{#3} % element name
    \end{center}
  \end{minipage}
}

%%%%
\newcommand{\tableauPeriodique}{
\begin{tikzpicture}[font=\sffamily, scale=0.75, transform shape]
%% Fill Color Styles
  \tikzstyle{elementFill}     = [fill=yellow!30]
  \tikzstyle{alkaliFill}      = [fill=red!45]
  \tikzstyle{alkaliEarthFill} = [fill=red!30]
  \tikzstyle{metalFill}       = [fill=red!15]
  \tikzstyle{metalloidFill}   = [fill=yellow!15]
  \tikzstyle{nonmetalFill}    = [fill=orange!15]
  \tikzstyle{halogenFill}     = [fill=orange!30]
  \tikzstyle{nobleGasFill}    = [fill=orange!45]

%% Element Styles
  \tikzstyle{Element} = [
    draw=black, elementFill,
    minimum width=2.25cm, minimum height=2.51cm,
    node distance=2.52cm
  ]
  \tikzstyle{Alkali}      = [Element, alkaliFill]
  \tikzstyle{AlkaliEarth} = [Element, alkaliEarthFill]
  \tikzstyle{Metal}       = [Element, elementFill]
  \tikzstyle{Metalloid}   = [Element, elementFill]
  \tikzstyle{Nonmetal}    = [Element, elementFill]
  \tikzstyle{Halogen}     = [Element, halogenFill]
  \tikzstyle{NobleGas}    = [Element, nobleGasFill]
  \tikzstyle{PeriodLabel} = [font={\sffamily\LARGE}, node distance=2cm]
  \tikzstyle{GroupLabel}  = [font={\sffamily\LARGE}, minimum width=2.5cm, node distance=2cm]
  \tikzstyle{TitleLabel}  = [font={\sffamily\Huge\bfseries}]

%% Group 1 - IA
  \node[name=H, Element]              {\elementText{1} {H} {Hydrogène}};
  \node[name=Li, below of=H, Alkali]  {\elementText{3} {Li}{Lithium}};
  \node[name=Na, below of=Li, Alkali] {\elementText{11}{Na}{Sodium}};
%% Group 2 - IIA
  \node[name=Be, right of=Li, AlkaliEarth] {\elementText{4} {Be}{Béryllium}};
  \node[name=Mg, below of=Be, AlkaliEarth] {\elementText{12}{Mg}{Magnésium}};
%% Group 13 - IIIA
  \node[name=B,  right of=Be, Metalloid] {\elementText{5} {B} {Bore}};
  \node[name=Al, below of=B,  Metal]     {\elementText{13}{Al}{Aluminium}};
%% Group 14 - IVA
  \node[name=C,  right of=B, Nonmetal]  {\elementText{6} {C} {Carbone}};
  \node[name=Si, below of=C, Metalloid] {\elementText{14}{Si}{Silicium}};
%% Group 15 - VA
  \node[name=N, right of=C, Nonmetal] {\elementText{7} {N}{Azote}};
  \node[name=P, below of=N, Nonmetal] {\elementText{15}{P}{Phosphore}};
%% Group 16 - VIA
  \node[name=O, right of=N, Nonmetal] {\elementText{8} {O}{Oxygène}};
  \node[name=S, below of=O, Nonmetal] {\elementText{16}{S}{Soufre}};
%% Group 17 - VIIA
  \node[name=F,  right of=O, Halogen] {\elementText{9} {F} {Fluor}};
  \node[name=Cl, below of=F, Halogen] {\elementText{17}{Cl}{Chlore}};
%% Group 18 - VIIIA
  \node[name=Ne, right of=F,  NobleGas] {\elementText{10}{Ne}{Néon}};
  \node[name=Ar, below of=Ne, NobleGas] {\elementText{18}{Ar}{Argon}};
  \node[name=He, above of=Ne, NobleGas] {\elementText{2} {He}{Helium}};

%% Period
  \node[name=Period1, left of=H,  PeriodLabel] {\normalsize{1}};
  \node[name=Period2, left of=Li, PeriodLabel] {\normalsize{2}};
  \node[name=Period3, left of=Na, PeriodLabel] {\normalsize{3}};
\end{tikzpicture}
}



%%%% Couleurs réglables
\colorlet{couleurPrim}{cyanSombre}
\colorlet{couleurSec}{bleuPale}
\colorlet{couleurTer}{vertSombre}
\colorlet{couleurQuat}{orangeSombre}


%%%% Réglages de la taille des indentations et des sauts de paragraphes
\setlength{\parskip}{0cm}
\setlength{\parindent}{0cm}
\renewcommand{\baselinestretch}{1}
% réglage du niveau (sous-section) ou s'arrête la table des matières
\setcounter{tocdepth}{2}


%%%% Réglage de la géométrie des pages
\geometry{
  a4paper, % format
  left=1.3cm, right=1.3cm, % marge horizontale
  top=2.2cm, bottom=2.3cm % marge verticale
}


%%%% Réglage de chemfig
\setchemfig{
  atom sep=24pt,
  bond style={line width=1pt},
  angle increment=30
}


%%% Apparence (couleur) des liens
\hypersetup{
  colorlinks=true,
  linkcolor=black, % lien type table des matière
  citecolor=black, % citation
  filecolor=black, 
  urlcolor=couleurPrim!10!black % lien internet
}


%%%% Réglage de tikz (flèche et caractères)
\usetikzlibrary{babel}
\tikzset{>=latex}


%%%% Réglage des en-tête
\renewcommand{\headrulewidth}{0.4pt}
\setlength{\headheight}{22.50113pt}


%%%% Réglage de dashundergaps pour avoir des points et pas de numération
\dashundergapssetup{
  gap-numbers = false,
  gap-format = dot,
  gap-widen,
  gap-extend-percent
}


%%%% Réglage de siunit
\sisetup{
  locale = FR, % français
	 group-minimum-digits = 4, % groupage des chiffres par millier
  inter-unit-product = \ensuremath { { } \cdot { } } % point médian entre les unités
}
\AtBeginDocument{\RenewCommandCopy\qty\SI} % Pour "écraser" la commande \qty du package physics
