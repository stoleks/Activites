%%%% Pour avoir les accents et autre caractère français
\usepackage[french]{babel}
\usepackage[T1]{fontenc}
\usepackage[utf8]{inputenc}

%%%% Paquets utilisés
%\usepackage{ghsystem} % pictogramme de sécurité
\usepackage{subcaption} % pour les légendes des figures
\usepackage[european, straightvoltages, RPvoltages]{circuitikz} % pour dessiner des circuits électrique
\usepackage{pdfpages} % pour inclure des fichiers pdf
\usepackage{geometry} % définition des marges
%% Paquets persos
\usepackage{biomolecules} % pour dessiner des formules chimiques
\usepackage{profSciences} % mise en page et autre


%%%% Commandes prédéfinies
%%%%%%%%%%%%%%%%%%%%%%%%%%%%%%%%%%%%%%%%%%%%%%%%%%%%%%%%%%%%%
%% grandeurs récurrentes
% Physique
\newcommand{\ISS}{\text{ISS}}
\newcommand{\Terre}{\text{Terre}}
\newcommand{\inertie}{\text{inertie}}
\newcommand{\Tfus}{T_\text{f}}
\newcommand{\Teb}{T_\text{éb}}
% Chimie
\newcommand{\solute}{\text{soluté}}
\newcommand{\solution}{\text{solution}}
\newcommand{\espece}{\text{espèce}}
\newcommand{\avogadro}{\num{6,02e23}}
% ions
\newcommand{\ionFerII}      {Fer II      \chemfig{Fe^{2+}}   }
\newcommand{\ionFerIII}     {Fer III     \chemfig{Fe^{3+}}   }
\newcommand{\ionSodium}     {Sodium      \chemfig{Na^{+}}    }
\newcommand{\ionCuivreII}   {Cuivre II   \chemfig{Cu^{2+}}   }
\newcommand{\ionCalcium}    {Calcium     \chemfig{Ca^{2+}}   }
\newcommand{\ionSulfate}    {Sulfate     \chemfig{SO_4^{2-}} }
\newcommand{\ionNitrate}    {Nitrate     \chemfig{NO_3^{-}}  }
\newcommand{\ionChlorure}   {Chlorure    \chemfig{Cl^{-}}    }
\newcommand{\ionFluorure}   {Fluorure    \chemfig{F^{-}}     }
\newcommand{\ionMagnesium}  {Magnésium   \chemfig{Mg^{2+}}   }
\newcommand{\ionPotassium}  {Potassium   \chemfig{K^{+}}     }
\newcommand{\ionBicarbonate}{Bicarbonate \chemfig{CO_3^{2-}} }

%% vecteurs
\newcommand{\FBsurA}{F_{B/A}}
\newcommand{\FAsurB}{F_{A/B}}
\newcommand{\vvFAsurB}{\vv{F}_{A/B}}
\newcommand{\vvFBsurA}{\vv{F}_{B/A}}
%%%% Ce fichier sert à déclarer les titres des chapitres des différents niveaux

%% Commun
\newcommand{\methode} {\chapitre{Outils pratiques}}

%% Seconde
%%%% Chapitre
\newcommand{\snd}{Seconde}
\newcommand{\sndCorp} {\chapitre{Corps purs et mélanges}}
\newcommand{\sndSolu} {\chapitre{Solutions}}
\newcommand{\sndMouv} {\chapitre{Mouvement et interactions}}
\newcommand{\sndAtom} {\chapitre{Structure de l'atome}}
\newcommand{\sndMole} {\chapitre{Des atomes à la matière}}
\newcommand{\sndLumi} {\chapitre{Ondes lumineuses et optique}}
\newcommand{\sndTran} {\chapitre{Transformations de la matière}}
\newcommand{\sndChim} {\chapitre{Transformations chimiques}}
\newcommand{\sndSign} {\chapitre{Signaux et capteurs}}

%%%% en-tête correspondant
\newcommand{\teteSndMeth} {\enTete[\snd]{\methode}}
\newcommand{\teteSndCorp} {\enTete[\snd]{\sndCorp}[1]}
\newcommand{\teteSndSolu} {\enTete[\snd]{\sndSolu}[2]}
\newcommand{\teteSndMouv} {\enTete[\snd]{\sndMouv}[3]}
\newcommand{\teteSndAtom} {\enTete[\snd]{\sndAtom}[4]}
\newcommand{\teteSndMole} {\enTete[\snd]{\sndMole}[5]}
\newcommand{\teteSndLumi} {\enTete[\snd]{\sndLumi}[6]}
\newcommand{\teteSndTran} {\enTete[\snd]{\sndTran}[7]}
\newcommand{\teteSndChim} {\enTete[\snd]{\sndChim}[8]}
\newcommand{\teteSndSign} {\enTete[\snd]{\sndSign}[9]}


%% Première ST2S
%%%% Chapitres
\newcommand{\premStss}{Première ST2S}
\newcommand{\premStssChim} {\chapitre{Sécurité chimique dans l'habitat}}
\newcommand{\premStssVisi} {\chapitre{Propagation de la lumière et vision}}
\newcommand{\premStssRedo} {\chapitre{Antiseptique et désinfectant, oxydoréduction}}
\newcommand{\premStssLumi} {\chapitre{Les infrarouges et leurs applications}}
\newcommand{\premStssStru} {\chapitre{Molécules d'intérêt biologique}}
\newcommand{\premStssBiom} {\chapitre{Biomolécules dans l’organisme}}
\newcommand{\premStssRout} {\chapitre{Sécurité routière}}
\newcommand{\premStssAlim} {\chapitre{Gestion des ressources naturelles et alimentation}}
\newcommand{\premStssElec} {\chapitre{Sécurité électrique dans l'habitat}}
\newcommand{\premStssPres} {\chapitre{Propriétés des fluides et pression sanguine}}
\newcommand{\premStssSono} {\chapitre{Ondes sonores et audition}}

%%%% en-tête
\newcommand{\tetePremStssMeth} {\enTete[\premStss]{\methode}     }
\newcommand{\tetePremStssChim} {\enTete[\premStss]{\premStssChim}[1]}
\newcommand{\tetePremStssVisi} {\enTete[\premStss]{\premStssVisi}[2]}
\newcommand{\tetePremStssRedo} {\enTete[\premStss]{\premStssRedo}[3]}
\newcommand{\tetePremStssLumi} {\enTete[\premStss]{\premStssLumi}[4]}
\newcommand{\tetePremStssStru} {\enTete[\premStss]{\premStssStru}[5]}
\newcommand{\tetePremStssBiom} {\enTete[\premStss]{\premStssBiom}[6]}
\newcommand{\tetePremStssRout} {\enTete[\premStss]{\premStssRout}[7]}
\newcommand{\tetePremStssAlim} {\enTete[\premStss]{\premStssAlim}[8]}
\newcommand{\tetePremStssElec} {\enTete[\premStss]{\premStssElec}[9]}
\newcommand{\tetePremStssPres} {\enTete[\premStss]{\premStssPres}[10]}
\newcommand{\tetePremStssSono} {\enTete[\premStss]{\premStssSono}[11]}


%% Terminale ST2S
%%%% Chapitres
\newcommand{\termStss}{Terminale ST2S}
\newcommand{\termStssOrga} {\chapitre{Représentation des molécules organiques}}
\newcommand{\termStssAlim} {\chapitre{Sécurité physico-chimique dans l'alimentation}}
\newcommand{\termStssImag} {\chapitre{La physique de l'imagerie médicale}}
\newcommand{\termStssBiom} {\chapitre{Biomolécules et alimentation}}
\newcommand{\termStssMedi} {\chapitre{De la molécule au médicament}}
\newcommand{\termStssEnvi} {\chapitre{Sécurité chimique dans l'environnement}}
\newcommand{\termStssDosa} {\chapitre{Analyser la composition d'un milieu}}
\newcommand{\termStssRout} {\chapitre{Sécurité routière}}
\newcommand{\termStssCosm} {\chapitre{L'usage responsable des cosmétiques}}

%%%% en-tête
\newcommand{\teteTermStssMeth} {\enTete[\termStss]{\methode}}
\newcommand{\teteTermStssOrga} {\enTete[\termStss]{\termStssOrga}[1]}
\newcommand{\teteTermStssRout} {\enTete[\termStss]{\termStssRout}[8]}
\newcommand{\teteTermStssAlim} {\enTete[\termStss]{\termStssAlim}[2]}
\newcommand{\teteTermStssEnvi} {\enTete[\termStss]{\termStssEnvi}[6]}
\newcommand{\teteTermStssImag} {\enTete[\termStss]{\termStssImag}[3]}
\newcommand{\teteTermStssDosa} {\enTete[\termStss]{\termStssDosa}[7]}
\newcommand{\teteTermStssBiom} {\enTete[\termStss]{\termStssBiom}[4]}
\newcommand{\teteTermStssMedi} {\enTete[\termStss]{\termStssMedi}[5]}
\newcommand{\teteTermStssCosm} {\enTete[\termStss]{\termStssCosm}[9]}

\newcommand{\largeurCaseTableauPeriodique}{1.5}

%%%% Pour afficher un élément dans le tableau périodique
\NewDocumentCommand{\elementTexteCharge}{m m m o}
{
  \begin{minipage}{\largeurCaseTableauPeriodique cm}
    \begin{center}
      \IfValueTF{#4}{ \vAligne{-20pt} }{ \vAligne{-34pt} } % position du nom
      {\small #3} \\[2pt] % nom de l'élément
      {\ensuremath\footnotesize \textbf{#1}} \\[6pt] % nombre atomique
      \chemfig[atom style={scale = 1.8}]{#2} % symbole atomique
      % \element{#1}{#2} % element symbol and atomic number
      \IfValueT{#4}{
        \\ {\small \qty{#4}{\g/\mole}}
      }
    \end{center}
  \end{minipage}
}

%%%% Pour afficher un élément dans le tableau périodique
\NewDocumentCommand{\elementElectroneg}{m m}
{
  \begin{minipage}{\largeurCaseTableauPeriodique cm}
    \begin{center}
      {\Large \important[black]{#1} \\[2pt]} % symbole atomique
      {\small $\chi = \num{#2}$} % électronégativité
    \end{center}
  \end{minipage}
}


%%%% Pour afficher un tableau périodique
%% #1 : largeur ; #2 : hauteur ; #3 : élements
\NewDocumentCommand{\tableauPeriodique}{O{2.6} O{2.7} m}{
\begin{tikzpicture}[font=\sffamily, scale=0.75, transform shape]

%% Type d'élément, par famille
  \tikzstyle{Alcali} = [Element, fill=green-200]
  \tikzstyle{Alcalo} = [Element, fill=green-150]
  \tikzstyle{Metaux} = [Element, fill=green-100]
  \tikzstyle{Metoid} = [Element, fill=orange-100]
  \tikzstyle{NoMeta} = [Element, fill=orange-150]
  \tikzstyle{Haloge} = [Element, fill=orange-200]
  \tikzstyle{GazRar} = [Element, fill=red-150]

%% Type d'élément, par électronégativité
 \tikzstyle{elec1} = [Element, fill=green-50]
 \tikzstyle{elec2} = [Element, fill=green-100!80]
 \tikzstyle{elec3} = [Element, fill=yellow-100]
 \tikzstyle{elec4} = [Element, fill=orange-100]
 \tikzstyle{elec5} = [Element, fill=orange-150]
 \tikzstyle{elec6} = [Element, fill=orange-200]
 \tikzstyle{elec7} = [Element, fill=red-200]
 \tikzstyle{elec8} = [Element, fill=red-300]
  
%% Style des éléments
  \tikzstyle{Element} = [
    draw=black, cyan-800!50!black,
    minimum width  = #1 cm, % Largeur de la case
    node distance  = #1 cm, % Espace entre deux case
    minimum height = #2 cm, % Hauteur de la case
  ]

%% Période, groupe et titre
  \tikzstyle{Period} = [font={\sffamily\LARGE}, node distance=2cm]
  \tikzstyle{Groupe} = [font={\sffamily\LARGE}, minimum width=2.5cm, node distance=2cm]
  \tikzstyle{Titre}  = [font={\sffamily\Huge\bfseries}]

%% Place des éléments
  #3
\end{tikzpicture}
}


%%%% Pour faciliter l'utilisation du tableau périodique
\newcommand{\elementH} {\elementTexteCharge{1} {H} {Hydrogène}[1,00]}
\newcommand{\elementHe}{\elementTexteCharge{2} {He}{Hélium}   [4,00]}
\newcommand{\elementLi}{\elementTexteCharge{3} {Li}{Lithium}  [6,94]}
\newcommand{\elementBe}{\elementTexteCharge{4} {Be}{Béryllium}[9,01]}
\newcommand{\elementB} {\elementTexteCharge{5} {B} {Bore}     [10,8]}
\newcommand{\elementC} {\elementTexteCharge{6} {C} {Carbone}  [12,0]}
\newcommand{\elementN} {\elementTexteCharge{7} {N} {Azote}    [14,0]}
\newcommand{\elementO} {\elementTexteCharge{8} {O} {Oxygène}  [16,0]}
\newcommand{\elementF} {\elementTexteCharge{9} {F} {Fluor}    [19,0]}
\newcommand{\elementNe}{\elementTexteCharge{10}{Ne}{Néon}     [20,2]}
\newcommand{\elementNa}{\elementTexteCharge{11}{Na}{Sodium}   [23,0]}
\newcommand{\elementMg}{\elementTexteCharge{12}{Mg}{Magnésium}[24,3]}
\newcommand{\elementAl}{\elementTexteCharge{13}{Al}{Aluminium}[27,0]}
\newcommand{\elementSi}{\elementTexteCharge{14}{Si}{Silicium} [28,1]}
\newcommand{\elementP} {\elementTexteCharge{15}{P} {Phosphore}[31,0]}
\newcommand{\elementS} {\elementTexteCharge{16}{S} {Soufre}   [32,1]}
\newcommand{\elementCl}{\elementTexteCharge{17}{Cl}{Chlore}   [35,5]}
\newcommand{\elementAr}{\elementTexteCharge{18}{Ar}{Argon}    [39,9]}
\newcommand{\elementK} {\elementTexteCharge{19}{K} {Potassium}[39,1]}
\newcommand{\elementCa}{\elementTexteCharge{20}{Ca}{Calcium}  [40,0]}


%%%% Réglages de la taille des indentations et des sauts de paragraphes
\setlength{\parskip}{0cm}
\setlength{\parindent}{0cm}
\renewcommand{\baselinestretch}{1}
% réglage du niveau (sous-section) ou s'arrête la table des matières
\setcounter{tocdepth}{2}


%%%% Réglage de la géométrie des pages
\geometry{
  a4paper, % format
  left=1.3cm, right=1.3cm, % marge horizontale
  top=2.2cm, bottom=2.1cm % marge verticale
}

%%%% Réglage des en-tête
\renewcommand{\headrulewidth}{0.4pt}
\setlength{\headheight}{22.50113pt}
