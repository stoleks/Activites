\strut \vspace*{-40pt}
\titre{Communication non violente et gestion des émotions}

% \begin{objectifs}
%   \item Reconnaître et exprimer ses émotions et sentiments, agréables ou désagréables
%   \item Enrichir son vocabulaire pour affiner l'expression de ses émotions
%   \item Exprimer ses besoins et se sentir en confiance ensemble
% \end{objectifs}

\begin{contexte}
  Ce n'est pas toujours facile de parler de ses émotions et de ce que l'on ressent.
  C'est pourquoi on va s'engager collectivement à respecter les règles suivantes :
  \begin{listePoints}
    \item on s'écoute et \important{on ne coupe pas la parole} ;
    \item on essaie \important{d'être empathique,} on \important{ne se moque pas} et on \important{ne juge pas} ;
    \item on peut décider \important{de ne pas parler} si on préfère rester silencieux ou silencieuse ;
    \item ce qui est dit pendant l'heure restera \important{confidentiel,} on ne vas pas le répéter à ses camarades.
  \end{listePoints}
\end{contexte}

%%%%
\titreSection{Gestion des émotions}

\titreSousSection{Vocabulaire associée aux émotions}

\begin{doc}{Les six émotions principales}
  En étudiant les sociétés humaines, les ethnologues et psychologues ont établi une liste de six émotions communes à toutes les sociétés humaine (et à la plupart des mammifère d'ailleurs) :
  \begin{listePoints}[2]
    \item la \important{colère} ;
    \item la \important{tristesse} ;
    \item la \important{peur} ;
    \item la \important{joie} ;
    \item le \important{dégoût} ;
    \item la \important{surprise}.
  \end{listePoints}

  Ces six émotions principales peuvent se combiner et s'exprimer avec différents niveaux d'intensités pour donner une vaste palette de sentiments et d'émotions secondaires.
\end{doc}

\documentaire Associer chacun des adjectifs désignant une émotion à une des six émotions principales.

\titreSousSection{Retour sur un moment agréable}

\begin{tblr}{
  colspec = {X[c] X[c] X[c]},
  hlines, vlines,
  row{2} = {ht = 7cm}
}
  Moment agréable & Émotion(s) & Besoin(s) satisfait(s) \\
  & & 
\end{tblr}

\documentaire Réfléchissez à un moment agréable et notez le dans le tableau.

\documentaire Réfléchissez aux émotions que vous avez ressenti pendant ce moment agréable et utiliser les cartes « émotions » déjà distribuées pour les noter.

\documentaire Prenez ensuite les cartes « besoins » et notez les besoins qui ont été remplis pendant ce moment agréable.


%%%%
\pagebreak
\titreSection{Communication non violente}

\begin{doc}{Le principe de la communication non violente}
  Dans notre société, la communication a souvent tendance à se baser sur la domination en utilisant la violence, la manipulation, en accusant l'autre d'être responsable de notre état émotionnel ou en imposant sa volonté à l'autre.
  
  Au contraire, la communication non violente essaye de privilégier l'empathie et l'écoute des autres et de soi-même, en cherchant à clarifier ses besoins et ses sentiments.
  En pratique, elle repose sur quatre étapes :
  \begin{enumeration}
    \item Description des faits : on décrit une situation initiale en essayant de ne pas y porter de jugement ou d'interpréter les actions de l'autre.
    \item Expression de ses sentiments sans faire porter leur responsabilité sur l'autre.
    \item Reconnaître ses besoins en s'aidant de ses émotions et les formuler clairement.
    \item Formuler une demande claire, concrète et négociable, pour que l'autre puisse y répondre en l'adaptant à ses besoins.
  \end{enumeration}

  Prenons l'exemple d'un-e jeune adulte qui ne range pas l'appartement familial. Un exemple de demande de rangement basée sur la communication non violente pourrait être la suivante :
  \extrait{
    Quand tes habits traînent dans le salon [observation], je suis énervé [expression sans culpabilisation], car j’ai besoin que l'espace commun soit mieux rangé [besoin précisé]. Est-ce que tu pourrais, s’il te plaît, mettre tes affaires dans ta chambre ? [demande précise et concrète]
  }

  En communiquant plus clairement ses besoins et ses sentiments, on évite de mettre la personne en face de soit dans une situation inconfortable, ce qui mène souvent à de la colère et de la frustration.
\end{doc}

\question{
  Trouver et noter une situation ou à un comportement qui vous à gêné récemment.
}{}[3]

\question{
  Comment avez-vous géré cette situation ? Est-ce que la situation s'est améliorée après ?
}{}[3]

\question{
  Formulez une demande en respectant les 4 étapes de la communication non violente.
}{}[4]