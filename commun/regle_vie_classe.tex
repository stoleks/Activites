\newpage
\pasDePagination

%%%%
\titre{Règles de vie en classe}

\large

\begin{listePoints}
  \item \important{Le règlement intérieur s'applique en classe.}
  \item Pas de prise de parole sans autorisation.
  \item Pas de déplacement sans autorisation.
  \item Téléphone silencieux et posés sur la table du professeur.
  \item Pas de retards acceptés passé 5 minutes.
  \item Les documents du chapitre en cours doivent \important{tous} être amenés.
\end{listePoints}

\begin{center}
  \important{Objectif : avoir un cadre de travail serein et agréable.}
\end{center}

\vspace*{-0.2cm}
\ligne

%%%
\titre{Règles d'évaluation}


\important{Les devoirs sur table}

\begin{listePoints}
  \item 1 ou 2 devoirs sur table par trimestre.
  \item \important{Date et sujet communiqués une ou deux semaines en avance.}
  \item Coefficient 3.
\end{listePoints}


\important{Les travaux pratiques}

\begin{listePoints}
  \item Évaluation des compte-rendus.
  \item Évaluation de certaines activités réalisées en classe.
  \item Évaluation de la maîtrise des gestes expérimentaux.
  \item Coefficient 2.
\end{listePoints}


\important{La progression}

\begin{listePoints}
  \item Évaluation variable (classeur, DM, interrogation).
  \item Coefficient 0,5 ou 1.
\end{listePoints}

% \bigskip
% \normalsize
% \important{Exemple :} Avec $8/20$, $9/20$, $11/20$ en devoir sur table et $10/20$ en travaux pratiques, un ou une élève assidue ($> 15/20$ en progression), s'assure une moyenne supérieure à $10/20$ (contre $9,\!5/10$ sans la progression).

% \bigskip
% \normalsize
% \important{Signature élève :}