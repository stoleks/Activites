\begin{doc}{Les pictogrammes de sécurités}{doc:TP0_picto_secu}
  Les pictogrammes de sécurités sont à connaître par c\oe{}ur ! \\[4pt]

  %% Tableau avec les pictogrammes
  \NewDocumentCommand{\pictoTableau}{m O{65}}{
    \hspace{2pt}
    \image{0.7}{images/securite/picto_#1} \vAligne{-#2pt}
  }
  \begin{tblr}{
    colspec = {Q[h, m, wd = 0.15\linewidth] Q[m, wd = 0.8\linewidth]},
    hlines, vlines,
    row{1} = {couleurPrim!20, c}
  }
    \textbf{Pictogramme} & \textbf{Signification} \\
    %
    \pictoTableau{corrosif} &
    {\texteTrou{Corrosif.} \\
    Je peux attaquer ou détruire les métaux.
    Je ronge la peau et/ou les yeux en cas de contact.} \\
    %
    \pictoTableau{nocif} &
    {\texteTrou{Toxique, irritant, narcotique.} \\
    J'empoisonne à forte dose.
    J'irrite la peau, les yeux et/ou les voies respiratoires.
    Je peux provoquer des allergies, de la somnolence ou des vertiges.} \\
    %
    \pictoTableau{toxique}[55] &
    {\texteTrou{Toxique.} \\
    J’empoisonne rapidement, même à faible dose. \\} \\
    %
    \pictoTableau{explosif} &
    {\texteTrou{Explosif.} \\
    Je peux exploser au contact d’une flamme, d’une étincelle, d’électricité statique, sous l’effet de la chaleur, de frottements ou d’un choc.} \\
    %
    \pictoTableau{combustible} &
    {\texteTrou{Inflammable.} \\
    Je peux m’enflammer au contact d’une flamme, d’une étincelle, d’électricité statique, sous l’effet de la chaleur, de frottements ou au contact de l’air ou de l’eau.} \\
    %
    \pictoTableau{comburant} &
    {\texteTrou{Comburant.} \\
    Je peux provoquer ou aggraver un incendie ou même provoquer une explosion en présence de produits inflammables.} \\
    %
    \pictoTableau{gaz_pression} &
    {\texteTrou{Gaz sous pression.} \\
    Je peux exploser sous l’effet de la chaleur.
    Je peux causer des brûlures ou blessures liées au froid.} \\
    %
    \pictoTableau{environnement} &
    {\texteTrou{Dangereux pour l'environnement.} \\
    Je provoque des effets néfastes sur les organismes du milieu aquatique, sur les êtres vivants.} \\
    %
    \pictoTableau{reprotoxique} &
    {\texteTrou{Mutagène, cancerogène, reprotoxique.} \\
    Je peux provoquer le cancer, modifier l’ADN, nuire à la fertilité ou au f\oe{}tus, altérer le fonctionnement des organes.
    Je peux être mortel en cas d’ingestion dans les voies respiratoires.}
    %
  \end{tblr}
\end{doc}