%\small

%%%% Quelques liens utiles
{\large \important{Quelques liens utiles en classe}}
\bigskip

\qrcodeCote[1]{https://nosgestesclimat.fr/tutoriel}
Nos gestes climats.
\\[22pt]

\qrcodeCote[1]{https://www.youtube.com/watch?v=OrT6vTS47f4}
Vidéos sur les VELI (pour parler du réchauffement climatique)
\\[22pt]

\qrcodeCote[1]{https://bsky.app/profile/laydgeur.bsky.social/post/3lwhxcsfeus2y}
Thread sur bsky sur l'adaptation des villes aux vagues de chaleurs

\qrcodeCote[1]{https://www.youtube.com/watch?v=z_zSd4GhOUo}
Présentation d'une étude sur la formation d'ubiquinone par des protéines dans les bactéries et de leur évolution, CNRS
\\[8pt]

\qrcodeCote[1]{https://www.youtube.com/watch?v=TEl4jeETVmg}
Présentation de la mole, avec des ordres de grandeurs, en anglais.
\\[22pt]


%%%% Lumière
\qrcodeCote[1]{https://www.youtube.com/watch?v=eaibXuVXo5s}
Vidéo de scienceclic sur les ondes électromagnétiques, complet et visuel, mais très rapide...
\\[22pt]

\qrcodeCote[1]{https://www.youtube.com/watch?v=FvbNrwjIrNU}
Présentation de la perception des couleurs chez les humains, science étonnante.
\\[22pt]

\qrcodeCote[1]{https://www.youtube.com/watch?v=w7y-1eY0mcE}
Présentation des ondes électromagnétique, et de leur utilisation dans le domaine radio, CNRS.
\\[8pt]

\qrcodeCote[1]{https://fr.science-questions.org/experiences/65/Faire_un_mirage_artificiel/?comments_page=2}
Réalisation d'un "mirage artificiel" avec un laser et une solution d'eau sucrée hétérogène.
\\[22pt]


%%%% Nucléaire
\qrcodeCote[1]{https://youtu.be/pFgTPZpjiqs?t=15}
Schématisation du fonctionnement d'une centrale nucléaire, ENGIE.
\\[22pt]

\qrcodeCote[1]{https://www.youtube.com/watch?v=ScP-uPIEpl8}
Présentation détaillée du fonctionnement d'une centrale nucléaire, le réveilleur.
\\[22pt]

\qrcodeCote[1]{https://www.youtube.com/watch?v=1MUcizMqVAc}
Présentation de la différence entre fission et fusion, science étonnante.
\\[22pt]


%%%% Son
\qrcodeCote[1]{https://www.youtube.com/watch?v=Q58ns2rLXx8}
Présentation du son, c'est pas sorcier.
\\[22pt]

\qrcodeCote[1]{https://www.youtube.com/watch?v=8YGQmV3NxMI}
Vibration d'une corde de guitare filmée avec la bonne fréquence.
\\[22pt]

% https://www.syndicalistes.org/syndiquer-en-zone-rurale