\titre{Règles du jeu « Méth »}

\setcounter{section}{1}
\setlength{\parskip}{8pt}

\titreSection{Objectif du jeu}\label{regle:objectif}

À chaque manche, les joueur-ses doivent se débarrasser de toutes leurs cartes. La première personne qui vide sa main gagne la manche et marque autant de points qu'il y a de cartes dans les main de ses camarades de jeu. 
La première personne qui arrive à 200 points gagne la partie.


\titreSection{Préparation du jeu}\label{regle:preparation}

Pour la préparation du jeu, il faut que :
\begin{listePoints}
  \item chaque personne tire une carte. Celle avec la molécule qui a le plus de carbones distribue ;
  \item toutes les cartes sont mélangées et 6 cartes sont distribuées à chaque personne ;
  \item le reste des cartes est mis au centre et forme la pioche ;
  \item la première carte de la pioche est retournée et forme le talon (la défausse), qui sert de base au jeu.
\end{listePoints}


\titreSection{Déroulement du jeu}\label{regle:deroulement}

La personne à gauche de celle qui a distribué commence la partie.
Lors de son tour, on doit recouvrir le talon avec une carte contenant une molécule du même \important{groupe fonctionnel} ou avec le \important{même nombre de carbone.}

Par exemple, si la carte visible est du méthanol (1 carbone avec un groupe hydroxyle), la personne qui joue peut poser une molécule avec 1 carbone ou une molécule avec un groupe hydroxyle.
Certaines cartes spéciales permettent en plus d'effectuer des actions, cf la section~\ref{regle:speciales}.

Si on ne peut pas jouer de carte pendant son tour, on en prend une dans la pioche. Si la carte piochée est jouable, elle peut être jouée immédiatement. Sinon, la personne passe son tour.

Si la personne qui a joué s'est trompé (mauvais groupe ou mauvais nombre de carbones), n'importe qui peut faire remarquer l'erreur et la personne qui a joué pioche alors 2 cartes.
Sa carte reste sur la défausse.


\titreSection{Fin de manche}\label{regle:fin}

Lorsque qu'une personne ne possède plus qu'une carte en main, elle doit dire « méth », ce qui signifie « un » en chimie organique.
Si elle ne le fait pas ou qu'elle se trompe de mot, elle doit piocher 2 cartes.

Quand une personne vide sa main, la manche s'arrête et les point sont comptés.
Chaque carte encore en main des autres personnes rapporte 1 point par carbone contenu à la personne qui a vidé sa main. Les acides gras rapportent gros !

Si quelqu'un atteint 200 points, la partie est finie.
Sinon la partie continue, avec une nouvelle manche en redistribuant les cartes, cf la section~\ref{regle:preparation}.

Si la pioche se vide à un moment du jeu, il faut prendre toutes les cartes de la défausse (sauf la dernière) et les mélanger pour reconstituer une pioche.


\titreSection{Cartes spéciales}\label{regle:speciales}

\titreSousSection{Acides $\mathbf{\alpha}$-aminés}

Quand un acide $\alpha$-aminé est joué sur un groupe « carboxyle » ou « amine » en le nommant correctement à voix haute, le jeu change de sens.
Si la personne se trompe en nommant le groupe utilisé, elle pioche 2 cartes.
Un acide aminé ne peut être joué que sur une molécule comportant un de ses groupes ou ayant le même nombre de carbones.

\important{Si la partie démarre avec un acide aminé,} le jeu commence dans le sens anti-horaire, avec la personne à droite de celle qui a distribué.

\titreSousSection{Acide gras}

Un acide gras peut être joué sur n'importe quelle molécule et compte comme n'importe quel nombre de carbone.
Si un acide gras est joué sur un groupe carboxyle en disant « carboxyle » à voix haute, la personne suivante passe son tour.

\important{Si la partie démarre avec un acide gras,} la première personne à jouer peut poser n'importe quelle molécule dessus.

\titreSousSection{Glucides + 2}

Quand un glucide est joué sur un groupe « hydroxyle » ou un groupe « carbonyle » en le nommant correctement à voix haute, la personne suivante passe son tour et pioche 2 cartes.
Si la personne qui joue le glucide se trompe en nommant le groupe utilisé, elle pioche 2 cartes.
Un glucide ne peut être joué que sur une molécule comportant un groupe carbonyle ou un groupe hydroxyle.

\important{Si la partie finit avec un glucide bien nommé,} les deux cartes sont piochées et comptent pour le score de la manche.

\titreSousSection{Molécules synthétiques + 4}

Une molécule synthétique peut être jouée sur n'importe quelle molécule et compte comme n'importe quel nombre de carbone.
Si elle est jouée sur une molécule qui contient un de ses groupe fonctionnel et qu'il est nommé correctement à voix haute, la personne suivante pioche 4 cartes et passe son tour.

\important{Si la partie démarre avec une molécule synthétique,} elle est remise dans la pioche et une nouvelle carte est tirée pour former le talon.

\important{Si la partie finit sur une molécule synthétique et que les conditions sont remplies pour faire piocher,} la personne suivant pioche 4 cartes et elles comptent pour le score de la manche.