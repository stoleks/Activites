\titreSection{Homogénéité d'une relation}
\vspace*{-8pt}

\begin{doc}{Relation homogène}
  \begin{importants}  
    Une relation entre grandeurs ne peut être correcte que si elle est \important{homogène.}
    C'est-à-dire si les membres à droite et à gauche de l'égalité s'exprime avec les \important{même unités.}
  \end{importants}
  
  Toute égalité entre deux grandeurs qui ne peuvent pas s'exprimer avec les mêmes unités est donc forcément \important{fausse.}
  On dit \important{qu'elle n'est pas homogène.}
  Vérifier l'homogénéité d'une équation c'est faire de \important{l'analyse dimensionnelle.}

  \begin{importants}
    L'homogénéité permet de \important{retrouver} la relation entre trois grandeurs en regardant l'unité de la grandeur que l'on cherche à calculer.

    \begin{wrapfigure}{r}{0.25\linewidth}
      \vspace*{-20pt}
      \begin{equation*}
        M \left(\dfrac{\unit{\g}} {\unit{ \mole}}\right) = \dfrac{m (\unit{\g})} {n (\unit{\mole})}
      \end{equation*}
    \end{wrapfigure}

    \exemple Pour calculer la masse molaire (en \unit{\g\per\mole}), à partir d'une masse (\unit{\g}) et d'une quantité de matière (\unit{\mole}), il faut diviser la masse par la quantité de matière pour avoir une égalité homogène.
  \end{importants}
\end{doc}

\question{
  Donner l'égalité qui permet de calculer une distance $d (\unit{\m})$ à partir d'un temps $t (\unit{\s})$ et d'une vitesse $v (\unit{\m\per\s})$.
}{}[2]

\question{
  Donner l'égalité permettant de calculer une concentration molaire $c (\unit{\mole\per\l})$ à partir d'une concentration massique $c_m (\unit{\g\per\l})$ et d'une masse molaire $M (\unit{\g\per\mole})$.
}{}[2]
