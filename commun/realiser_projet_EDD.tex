\titre{Réaliser un projet en lien avec le développement durable}

\begin{tcolorbox}[
    boite cassable,
    titre sans separation = 0.5mm,
    title = Avant de réaliser une action
  ]
  Pour organiser une action, il y a plusieurs questions qu'il faut se poser en amont.
  \begin{listePoints}
    \item Identifier le cadre du projet : 
    \begin{listeTirets}
      \item que veux-t-on faire et pourquoi le fait-on ? 
      \item quels objectifs du développement durables sont visés ?
      \item se renseigner sur les actions similaires qui ont déjà été menées.
    \end{listeTirets}
    \item Trouver des partenaires (associations, services sociaux, services municipaux, établissements publics, etc.) avec qui travailler pour réaliser ses projets.
    \item Contacter des associations pour s'informer et organiser des actions communes. 
    \item Planifier la ou les actions précisément : 
    \begin{listeTirets}
      \item lieu,
      \item date,
      \item budget (est-ce qu'il faut financer le déplacement ou l'achat de matériel ?),
      \item qui fait quoi (attribution des rôles : coordinateur-ice, responsable du matériel, communication, etc.).
    \end{listeTirets}
  \end{listePoints}
\end{tcolorbox}

\begin{tcolorbox}[
    boite cassable,
    titre sans separation = 0.5mm,
    title = Apprendre à se connaître
  ]
  Avant de commencer à travailler ensemble, c'est bien de se connaitre un minimum.
  Pour ça, il peut être utile de faire un \important{tour de table,} ou chacun-e parle à tour de rôle, \important{sans se couper la parole,} dans le sens des aiguilles d'une montre.
  Lors de son tour, on se présente rapidement (prénom, classe, pourquoi on a voulu être éco-délégué, etc.).
  En dehors de son tour, on écoute attentivement ses camarades.
\end{tcolorbox}

\begin{tcolorbox}[
    boite cassable,
    titre sans separation = 0.5mm,
    title = Planifier une action
  ]
  Bien s'organiser et planifier permet de gagner du temps et de s'éviter de mauvaises surprises.
  Pour s'organiser à plusieurs, des règles simples permettent de gagner du temps et de s'assurer que tous le monde fasse quelque chose :
  \begin{listePoints}
    \item \important{S'écouter avec attention, sans se couper la parole.}
    \item Se répartir des rôles, avec une personne qui :
    \begin{listeTirets}
      \item gère la répartition de la parole : attribution de qui parle et surveillance du temps de parole ($\sim$ 2-3 minutes par interventions) ;
      \item gère le temps restant pendant la réunion ;
      \item prend des notes pour garder une trace de ce qui a été dit (x 2) ;
      \item est responsable du suivi des tâches à faire en dehors des heures de réunion, s'il y en a.
    \end{listeTirets}
    \item Faire une sessions de réflexion pour noter des idées sur le projet.
    \item \important{Se renseigner sur le projet à faire en utilisant plusieurs sources fiables.}
    \item S'assurer que l'on fait quelque chose \important{en lien avec le développement durable.}
    \item Se répartir clairement les tâches à faire pour la prochaine fois, par exemple s'il faut faire des recherches ou contacter une association.
    \item Il peut être utile d'avoir une conversation (sur pronote, signal, whatsapp ou snapchat) avec tous les membres du groupe, pour faciliter la communication sur le projet.
  \end{listePoints}
\end{tcolorbox}