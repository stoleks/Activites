\begin{objectifs}
  \item Comprendre sommairement les mécanismes de mémorisation et d'attention.
  \item Apprendre quelques techniques de mémorisation efficace.
\end{objectifs}

\begin{contexte}
  Au quotidien, notre cerveau reçoit un très grand nombre d'informations, qu'il doit trier pour nous permettre de nous concentrer sur des éléments importants de notre environnement.
  Parmi ce flux d'informations, notre cerveau va en conserver certaines, c'est le mécanisme de \important{mémorisation.} Il est au cœur de tout apprentissage, manuel ou théorique.

  \problematique{
    Quelles méthodes permettent de maximiser nos capacités d'apprentissage ?
  }
\end{contexte}

\begin{doc}{Les 4 piliers de l'apprentissage}
  \qrcodeCote{https://edurl.fr/Bn5FmpTk}
  %https://www.youtube.com/watch?v=TJSeinBVUXk}
  Pour bien apprendre, il y a 4 piliers essentiels : 
  \begin{listePoints}
    \item \important{L'attention,} qui nécessite d'écouter les explications ;
    \item \important{L'engagement actif,} qui nécessite de réfléchir aux questions posées ;
    \item \important{Le retour sur l'erreur,} qui nécessite d'écouter les corrections ;
    \item \important{La consolidation,} qui nécessite de \important{la mémorisation} et \important{du repos.}
  \end{listePoints}

  \qrcodeCote{https://edurl.fr/GinLIuAN}(-32pt)
  %https://www.youtube.com/watch?v=vJG698U2Mvo}(-32pt)
  L'attention est particulièrement mise à rude épreuve à votre âge, c'est pour ça qu'il est important de \important{ne pas utiliser son téléphone,} et de \important{ne pas bavarder} en classe.
\end{doc}

\begin{doc}{Le processus de mémorisation}
  Nous n'avons pas une mémoire, mais plusieurs types de mémoires !
  \begin{listePoints}
    \item \important{La mémoire sensorielle,} inconsciente et liée à nos sens (toucher, odorat, vision, etc.).
    \item \important{La mémoire de travail,} consciente et limitée en capacité, qui permet de retenir quelques informations sur une courte durée (salle de cours, document à lire, liste des courses, etc.)
    \item \important{La mémoire à long terme,} qui contient des souvenirs, des savoirs ou des automatismes (marcher, parler, lire, écrire, etc.).
  \end{listePoints}
  
  \begin{tblr}{
    colspec = {X[c, m] X[c, m] X[c, m]},
    hlines = {white}, vlines = {white},
    cell{1-2}{1} = {couleurPrim-100},
    cell{1-2}{2} = {couleurSec-50},
    cell{1-2}{3} = {couleurTer-100},
    row{1} = {font = \sffamily\bfseries},
  }
    Mémoire sensorielle & Mémoire de travail & Mémoire à long terme \\
    Capacité : $< \qty{1}{\s}$ & Capacité : court terme et limitée & Capacité : long terme, illimité \\
  \end{tblr}

  \centering
  \image{0.8}{images/processus_memorisation}
  
  \legende{Schématisation simplifié du processus de mémorisation à partir d'une stimulation sensorielle (lecture d'une feuille par exemple), qui va être encodée, c'est-à-dire stockée dans la mémoire.}
\end{doc}

\begin{doc}{La courbe de l'oubli}
  Quand notre cerveau sauvegarde une information dans la mémoire à long terme, elle va être oublié progressivement.
  Pour ne pas oublier, il faut \important{raviver la mémoire} en révisant.
  
  Un souvenir ravivé va être oublié moins rapidement, mais il faut le revoir de manière répété et espacé dans le temps pour le mémoriser sur le long terme.
  On peut schématiser ça avec une \important{courbe de l'oubli,} qui indique comment on se souvient d'une connaissance au cours du temps.

  \centering
  \image{0.9}{images/courbe_memorisation}
\end{doc}

\begin{doc}{Techniques de mémorisation}[\label{doc:techniques_memorisation}]
  Quelques techniques de mémorisation :
  \begin{listePoints}[3]
    \item relire sa leçon plein de fois ;
    \item surligner le cours ;
    \item faire des fiches de révisions ;
    \item regarder une vidéo ;
    \item recopier sa leçon ;
    \item réciter sa leçon par cœur ;
    \item réviser juste avant une évaluation ;
    \item refaire les exercices ;
    \item se tester ;
    \item espacer les révisions dans le temps ;
    \item savoir expliquer la leçon ;
    \item alterner les cours révisés.
  \end{listePoints}
\end{doc}

\numeroQuestion Remplir le tableau suivant avec les différentes techniques de mémorisation du document~\ref{doc:techniques_memorisation}.
\smallskip

\begin{tblr}{
    colspec = {X[c] X[c] X[c]}, vlines,
    cell{1}{1} = {red-300}, cell{1}{2} = {orange-100}, cell{1}{3} = {green-300},
    vline{1} = {red-600}, vline{2} = {orange-600}, vline{3} = {green-600}, vline{4} = {cyan-600},
    row{1} = {font = \sffamily\bfseries},
    row{2} = {ht = 180pt},
  }
  Inefficaces & Partiellement efficaces & Efficaces \\
  \correction{Relire sa leçon plein de fois, surligner le cours, regarder une vidéo, recopier sa leçon.} &
  \correction{Réciter sa leçon par cœur, réviser juste avant une évaluation, refaire les exercices, savoir expliquer la leçon, faire des fiches de révisions.} &
  \correction{Se tester, espacer les révisions dans le temps, alterner les cours révisés.} \\
\end{tblr}
