%% Molécule d'exemple
\vspace*{-4pt}
\begin{wrapfigure}[5]{r}{0.58\linewidth}
  \vspace*{-30pt}
  \centering
  \begin{tikzpicture}[help lines/.style={thin,draw=black!50}]
    % chaine principale et carbone fonctionnel
    \large
    \node[draw] at (3,3) { \chemfig{
      H_3C-CH-CH_2 -\textcolor{couleurSec}{\textsf{\textbf{C}}} H-CH_3
      }
    };
    \draw (5, 2.25) node[right] {\textbf{chaîne principale}};
    \draw[couleurSec] (3.7, 3.7) node[right] {\textbf{carbone fonctionnel}};
    % Ramification
    \draw[very thick, couleurPrim] (1.48, 2.79) -- (1.48, 2.29);
    \draw[couleurPrim] (2.5, 1.3)  node[left] {\textbf{ramification}};
    \node[draw, couleurPrim] at (1.8, 2) { \chemfig{CH_3} };
    % Alcool
    \draw[very thick, couleurTer] (4.11, 2.79) -- (4.11, 2.29);
    \draw[violet] (3.6, 1.3)  node[right] {\textbf{groupe caractéristique}};
    \node[draw, couleurTer] at (4.28, 2){ \chemfig{OH} };
  \end{tikzpicture}
\end{wrapfigure}
%

Pour nommer les molécules contenant des groupes caractéristiques, on utilise les règles décrites dans le tableau ci-dessous, en respectant la priorité des fonctions organiques.

\begin{importants}
  Le \important{carbone fonctionnel} désigne le carbone contenant la fonction de la molécule.
\end{importants}

Pour les cétones, alcools et amines, le numéro est celui du \important{carbone fonctionnel,} comme pour les ramifications il \important{doit être le plus petit possible.}
($R_1$) et ($R_2$) représentent les noms des chaînes carbonées auxquels les groupes caractéristiques sont attachés. 

%%%% Tableau avec les groupes caractéristiques
\vspace*{2pt}
\begin{center}
\begin{tblr}{
  width = \linewidth, hlines, vlines,
  colspec = {c c c Q[wd=0.4\linewidth]},
  column{2,4} = {couleurPrim!10},
  row{1} = {couleurPrim!20},
  rows = {c,m}
}
  Priorité & Famille fonctionnelle & Formule & Nom si famille prioritaire \\
  %
  1 & Acide carboxylique
  & \hspace{-24pt} \chemfig{!\vide{:-30} R_1 !\lh !\couleur C (!\llhCouleur !\couleur O) !\lbCouleur !\couleur{OH}}
  & acide ($R_1$)-oïque \\
  %
  2 & Ester 
  & \hspace{-24pt}\chemfig{!\vide{:-30} R_1 !\lh !\couleur C (!\llhCouleur !\couleur O) !\lbCouleur !\couleur O !\lh R_2}
  & ($R_1$)-oate de ($R_2$)-yle \\
  %
  { 3 \\ \phantom{B} } & { Amide \\ \phantom{B} }
  & \chemfig{[:-30] R_1 !\lh !\couleur C (!\llhCouleur !\couleur O) !\lbCouleur !\couleur{NH_2}}
  & ($R_1$)-amide \\
  %
  4 & Aldéhyde
  & \hspace{-24pt}\chemfig{!\vide{:-30} R_1 !\lhCouleur !\couleur C (!\llhCouleur !\couleur O) !\lbCouleur !\couleur H} 
  & ($R_1$)-al \\
  %
  5 & Cétone
  & \hspace{-24pt} \chemfig{!\vide{:-30} R_1 !\lh !\couleur C (!\lb R_2) (!\llhCouleur !\couleur O)}
  & ($R_1$)-(numéro)-one \\
  %
  6 & Alcool
  & \chemfig{R_1- !\couleur{OH}} 
  & ($R_1$)-(numéro)-ol \\
  %
  7 & Amine 
  & \chemfig{R_1- !\couleur{NH_2}}
  & ($R_1$)-(numéro)-amine \\
  %
  8 & Éther
  & \hspace{-24pt} \chemfig{!\vide{:-30} R_1 !\lhCouleur !\couleur O !\lbCouleur R_2}
  & ($R_1$)-oxy-($R_2$) \\
\end{tblr}
\end{center}

\vspace*{2pt}
\begin{importants}
  \attention Pour ces 8 familles organiques, vous devez savoir :
  \begin{listePoints}
    \item les noms de chacune des familles et leur groupes fonctionnels ;
    \item les reconnaître dans une molécule si on vous en donne une représentation.
  \end{listePoints}
\end{importants}
