\begin{doc}{Fonctions organiques}{doc:A_fonction_organique}
  Certaines séquences d'éléments donnent des \important{propriétés} spécifiques aux molécules organiques que l’on classe en différentes familles ou fonctions organiques ou encore famille fonctionnelle.

  \begin{importants}  
    En ST2S on étudie 8 familles : \important{alcool, aldéhyde, cétone, acide carboxylique, ester, éther, amine et amide.}
  \end{importants}
  $R_1,$ $R_2$ et $R_3$ sont des chaînes carbonées appelées \important{« radicaux alkyles ».}

  \begin{center}
    \begin{tblr}{
      colspec = {Q[c,m] Q[c,m] Q[c,m] Q[t, wd=0.26\linewidth]}, hlines, vlines,
      column{2, 3} = {couleurSec-50}, row{1} = {couleurSec-100},
      cell{3}{1} = {r=2}{c}, rows = {m}, columns = {c}
    }
      Groupe caractéristique & Famille organique & Formule & Exemple \\
      %
      Hydroxyle & Alcool
      & \chemfig{R_1 - !\couleur{OH}} 
      & \chemfig{H_3C - OH} méthanol \\
      %
      Carbonyle & {Cétone \\ \phantom{B}}
      & \hspace{-24pt} \chemfig{!\vide{:-30} R_1 !\cu !\couleur C (!\cd R_2) (!\ccuCouleur !\couleur O)}
      & \chemname[-12pt]{ \chemfig{-[1] (=[3,0.8]O) !\cd -[1]} }{ butan-2-one } \\
      %
      & {Aldéhyde \\ \phantom{B}}
      & \hspace{-24pt}\chemfig{!\vide{:-30} R_1 !\cuCouleur !\couleur C (!\ccuCouleur !\couleur O) !\cdCouleur !\couleur H} 
      & \chemname[-12pt]{ \chemfig{H-C (=[3,0.9]O) -H} }{ méthanal } \\
      %
      Carboxyle & Acide carboxylique
      & \hspace{-24pt} \chemfig{!\vide{:-30} R_1 !\cu !\couleur C (!\ccuCouleur !\couleur O) !\cdCouleur !\couleur{OH}}
      & \chemname[1pt]{ \chemfig{-[-1] -[1] !\carboxyle} }{ acide propanoïque} \\
      %
      Ester & Ester
      & \hspace{-24pt} \chemfig{!\vide{:-30} R_1 !\cu !\couleur C (!\ccuCouleur !\couleur O) !\cdCouleur !\couleur O !\cu R_2}
      & \chemname[1pt]{ \chemfig{-[-1] -[1] !\ester -[1] -[-1]} }{ propanoate d'éthyle} \\
      %
      {\phantom{B} \\ Éther-oxyde \\ \phantom{B}} & Éther
      & \hspace{-24pt} \chemfig{!\vide{:-30} R_1 !\cuCouleur !\couleur O !\cdCouleur R_2}
      & \chemname[1pt]{ \chemfig{-[-1] -[1] O -[-1] -[1]} }{ éthoxyéthane } \\
      %
      {\phantom{B} \\ Amine \\ \phantom{B}} & Amine
      & \chemfig{R_1 -[,1.5] !\couleur{NH_2}}
      & \chemname[-12pt]{ \chemfig{H_3C -CH_2 -NH_2} }{ ethan-1-amine } \\
      %
      Amide & Amide
      & \chemfig{[:-30] R_1 !\cu !\couleur C (!\ccuCouleur !\couleur O) !\cdCouleur !\couleur N (!\cu R_3) !\cd R_2}
      & \chemname[1pt]{ \chemfig{-[-1] -[1] !\amide H_2} }{ propanamide}
    \end{tblr}
  \end{center}

  \begin{importants}
    Pour trouver les groupes caractéristiques d'une molécule, il faut repérer tous les éléments qui ne sont ni des carbones, ni des hydrogènes.
  \end{importants}
\end{doc}
