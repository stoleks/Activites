\documentclass[12pt]{extarticle}

%%%% paramètres généraux et commandes prédéfinies
\usepackage[french]{babel}
\usepackage[T1]{fontenc}
\usepackage[utf8]{inputenc}

\usepackage{geometry} % définition des marges
\usepackage{profSciences} % mise en page et autre
\usepackage{biomolecules} % pour dessiner des formules chimiques
\usepackage{lmodern}
\usepackage{lipsum}

\setlength{\parskip}{0cm}
\setlength{\parindent}{0cm}
\renewcommand{\baselinestretch}{1}
\setcounter{tocdepth}{2}

\geometry{
  a4paper, % format
  left=1.3cm, right=1.3cm, % marge horizontale
  top=1.6cm, bottom=2cm % marge verticale
}


%%%% doc
\begin{document}
\titre{Prof sciences}
\begin{center}
  Quelques commandes que j'utilise au jour le jour pour mettre en page mes activités.
\end{center}

\tableofcontents
\newpage

%------
\palette{couleurPrim}{red}
\palette{couleurSec}{orange}
\palette{couleurTer}{yellow}
\palette{couleurQuat}{purple}

%------
\renewcommand{\etablissement}{Documentation}
\enTete[CTAN]{ProfSciences}[1]
\begin{boiteCodeTex}{}
  \renewcommand{\etablissement}{Documentation}
  \enTete[CTAN]{ProfSciences}[1]
\end{boiteCodeTex}

\titrePartie{Gestion des couleurs}

\titreSection{Couleurs flexoki}

J'utilise la palette flexoki pour la couleur \url{https://stephango.com/flexoki}. 
Les huits couleurs et leurs 13 nuances sont définis en suivant la convention de flexoki : <nom de la couleur>-<teinte>.
\begin{boiteCodeTex}{}
  \bfseries \textcolor{blue-200}{Un}
  \textcolor{blue-400}{texte}
  \textcolor{blue-600}{en}
  \textcolor{blue-800}{dégradé}.
\end{boiteCodeTex}


\titreSection{Palette de couleurs}

Pour pouvoir changer facilement de couleurs et avoir une palette cohérente, les couleurs \lstinline|couleurPrim| (couleur primaire), \lstinline|couleurSec| (secondaire), \lstinline|couleurTer| (tertiaire) et \lstinline|couleurQuat| (quaternaire) sont définies avec la commande palette \lstinline|\palette{<alias>}{<couleur>}|.

ce qui permet d'appeler ces alias comme des couleurs flexoki normale, avec la teinte 600 comme défaut
\begin{boiteCodeTex}{}
  \bfseries
  \textcolor{couleurPrim}{Un}
  \textcolor{couleurSec-700}{texte}
  \textcolor{couleurTer}{légèrement}
  \textcolor{couleurQuat}{coloré}.
\end{boiteCodeTex}
En théorie n'importe quelle couleur définie avec le format <couleur>-<50, 100, 150, 200, ..., 900, 950> peut être utilisé avec la commande palette.

Pour changer de thème coloré, il suffit donc d'appeler la commande palette avec la couleur souhaité pour changer toutes les couleurs des commandes internes.
En guise d'illustration, chaque partie de ce document a son propre thème coloré.

%%%%
\begin{boiteCodeTex}{listing only}
  \palette{couleurPrim}{cyan}
  \palette{couleurSec}{blue}
  \palette{couleurTer}{purple}
  \palette{couleurQuat}{red}
\end{boiteCodeTex}
\palette{couleurPrim}{cyan}
\palette{couleurSec}{blue}
\palette{couleurTer}{purple}
\palette{couleurQuat}{red}
\newpage
\titrePartie{Mise en page}

%%
\titreSection{Titre et sections}

\addtocounter{part}{-1}
\addtocounter{section}{-1}
Les commandes de section sont définies en parallèle des commandes classique \lstinline|\section, \subsection|, etc. :
\begin{boiteCodeTex}{}
  \titrePartie*{Une partie d'exemple}

  \titreSection*{Une section d'exemple}

  \titreSousSection{Une sous-section d'exemple}
\end{boiteCodeTex}

Il existe aussi des commandes pour afficher les titres des activités ou des TP, ou bien des exercices :
\begin{boiteCodeTex}{}
  \titre{Un simple titre sans numérotation}
  \titreActivite{Une activité}
  \titreActivite*{Une activité avec moins de numérotation}
  \titreTP{Un TP}
  \titreTP*{Une activité expérimentale}

  \exercice{Un exercice à faire}
\end{boiteCodeTex}
Note : la commande \lstinline|\exercice| entraîne la remise à zéro des compteurs de document et de questions.

%%
\titreSection{Boîtes variées de mise en page}

Plusieurs boites sont définies pour permettre d'avoir une mise en page consistante entre différentes activités.
\begin{boiteCodeTex}{}
  \begin{contexte}
    Une boîte pour introduire une activité.
  \end{contexte}

  \begin{prerequis}
    \item Il faut connaître leur noms,
  \end{prerequis}

  \begin{objectifs}
    \item mais on peut vite faire un document riche,
    \item avec les bonnes commandes.
  \end{objectifs}

  \begin{doc}{Le titre}{le_label}
    Le contenu du document
  \end{doc}

  \begin{doc}{Un deuxième document}{le_label_ii}
    La numérotation est automatique
  \end{doc}

  \begin{importants}
    Pour mettre en valeur des éléments importants de l'activité ou du cours !
  \end{importants}

  \extrait[Source]{Une citation ou un extrait.}
  \extrait{Un extrait, mais sans source associée.}
  C'est aussi possible de n'avoir que la mise en page pour la source.
  \sourceExtrait{Moi même}
\end{boiteCodeTex}

Les courbes des boîtes d'objectifs et de prérequis sont réglées pour ``s'emboiter'' dans un plan de travail, cf.~\ref{plan_de_travail}.

En plus de ces boîtes, il y aussi des boîtes simples sans but précis
\begin{boiteCodeTex}{}
  \begin{boite}
    Une boite toute simple.
  \end{boite}

  \begin{boiteColoree}
  \end{boiteColoree}
  
  \begin{boiteColoree}[yellow-150]
    Une simple boite avec des couleurs.
  \end{boiteColoree}

  \begin{boiteColoree}[magenta-200]
    Bla bla.

    Pratique sur plusieurs lignes aussi !
  \end{boiteColoree}

  \begin{importants}
    Pour mettre en valeur un passage à retenir ou important.

    Le texte peut être cassé en paragraphes.
  \end{importants}
\end{boiteCodeTex}
\newpage
\titrePartie{Rédaction de fiches avec leur corrections}

\titreSection{Texte à trou et question à remplir}

Plusieurs commandes permettent de générer plusieurs fichiers .pdf à partir d'un seul document .tex, en appelant la commande \lstinline|\modeCorrection| pour l'affichage de la correction ou \lstinline|\modeEleve| pour repasser en version ``fiche à remplir''. 
C'est le mode élève à remplir qui est le statut par défaut dans un document.

La première commande permet d'afficher des textes à trous, avec réglage automatique de la largeur, réglage manuel en fraction de la largeur de la page ou avec un nombre de lignes à remplir.
\begin{boiteCodeTex}{}
Voilà un \texteTrou*{texte à trou}. % largeur automatique, texte gras
Suivit d'un autre \texteTrou[0.5]{texte à trou}. % largeur = la moitié de la page
Et encore un \texteTrou(1){texte à trou} ! % complète la ligne actuelle et trace une ligne

  
Un remarque importante : le nombre de lignes prend toujours le pas sur la largeur,
si les deux sont réglées \texteTrou[0.3](0){comme ici.}
Indiquer (0) revient à demander à compléter la ligne actuelle.
\end{boiteCodeTex}

On notera que la commande ajoute un espacement vertical pour faciliter le remplissage pour les personnes qui écrivent gros (j'écris assez gros et cet espacement me convient).
Et voilà la même boite, mais en mode correction :
\begin{boiteCodeTex}{}
\modeCorrection
Voilà un \texteTrou*{texte à trou}. % largeur automatique, texte gras
Suivit d'un autre \texteTrou[0.5]{texte à trou}. % largeur = la moitié de la page
Et encore un \texteTrou(1){texte à trou} ! % complète la ligne actuelle et trace une ligne

  
Un remarque importante : le nombre de lignes prend toujours le pas sur la largeur,
si les deux sont réglées \texteTrou[0.3](0){comme ici.}
Indiquer (0) revient à demander à compléter la ligne actuelle.
\modeEleve
\end{boiteCodeTex}

La seconde commande permet d'afficher des questions, avec un nombre de lignes réglables pour les réponses :
\begin{boiteCodeTex}{}
  \question{La question.}{La réponse.}
  
  \question{Une question avec de la place pour répondre}{Une autre réponse.}[2]

  \modeCorrection \setcounter{questionNum}{0}
  \question{La question.}{La réponse.}
  \question{Une question avec de la place pour répondre}{Une autre réponse.}[2]
  \modeEleve
\end{boiteCodeTex}

Enfin il y a une paire de commande qui permet d'afficher du contenu uniquement pour la correction (\lstinline|\correction{}|) ou uniquement pour la fiche à remplir (\lstinline|\pasCorrection{}|) :
\begin{boiteCodeTex}{}
  \correction{Uniquement dans le corrigé.}
  \pasCorrection{Uniquement dans la fiche à remplir.}

  \modeCorrection
  \correction{Uniquement dans le corrigé.}
  \pasCorrection{Uniquement dans la fiche à remplir.}
  \modeEleve
\end{boiteCodeTex}

%%
\titreSection{Différents types de questions}

Pour rester cohérent d'une activité à une autre, j'ai plusieurs commandes pour afficher certains types de questions récurrentes.
\begin{boiteCodeTex}{}
  \question{Une question numérotée classique.}{Et sa correction}

  \numeroQuestion Si on veut juste le numéro de la question.

  \mesure Quand il faut réaliser une expérience.

  \programmation Quand il faut programmer.

  \schematisation Quand il faut schématiser.

  \documentaire Quand il faut rédiger un compte-rendu.

  \telechargement Quand il faut scanner un QR code pour télécharger une appli ou regarder une vidéo.

  \mesure* La version étoilée permet de supprimer l'indentation.
\end{boiteCodeTex}

%%
\titreSection{Variation entre deux sujets}

Pour varier les sujets entre deux élèves, on peut appeler la commande \lstinline|\variationSujet{A}{B}|, puis les commandes \lstinline|\sujetA| ou \lstinline|\sujetB| pour passer d'un sujet à un autre.
\begin{boiteCodeTex}{}
  \sujetA
  Pour le sujet \variationSujet{A}{B}, on peut avoir des variation entre certaines \variationSujet{valeurs}{grandeurs}.
  
  \sujetB
  Pour le sujet \variationSujet{A,}{B,} on peut avoir des variation entre certaines \variationSujet{valeurs.}{grandeurs.}
\end{boiteCodeTex}
Pour tirer pleinement partie de ces commandes, c'est plus pratique de définir son sujet dans un fichier séparé et de l'inclure ensuite avec la commande \lstinline|\input|
\begin{boiteCodeTex}{listing only}
  \sujetA \input{mon_super_sujet}
  \sujetB \input{mon_super_sujet}
\end{boiteCodeTex}
en faisant comme ça, on a en plus un fichier avec une alternance naturelle entre les deux sujets si on l'imprime, ce qui facilite la distribution.

%%
\titreSection{Compétences et appréciations}

Pour afficher une grille d'évaluation des compétences, on peut utiliser l'environnement \lstinline|tableauCompetence| qui est un tableau basé sur \lstinline|tabularray| :
\begin{boiteCodeTex}{}
  \begin{tableauCompetences}
    RCO & Connaître son cours \\
    APP & Extraire une information \\
    COM & On peut en mettre autant qu'on veut à la suite... les cases sont centrées verticalement.
  \end{tableauCompetences}
\end{boiteCodeTex}

Et pour les appréciations il y a une simple boite à taille réglable et qui ne s'affiche qu'en mode élève :
\begin{boiteCodeTex}{}
  \appreciation{4 cm}
  \modeCorrection
  \appreciation{4 cm}
  \modeEleve
\end{boiteCodeTex}

%%%%
\begin{boiteCodeTex}{listing only}
  \palette{couleurPrim}{orange}
  \palette{couleurSec}{red}
  \palette{couleurTer}{magenta}
  \palette{couleurQuat}{green}
\end{boiteCodeTex}
\palette{couleurPrim}{orange}
\palette{couleurSec}{red}
\palette{couleurTer}{magenta}
\palette{couleurQuat}{green}
\newpage
\titrePartie{Schémas}

Une règle générale pour les commandes qui permettent de tracer des schémas, c'est que si la commande a pour préfixe tikz, elle doit être utilisée dans un environnement tikzpicture.

Commande pour tracer des rectangles colorés de tailles variables :
\begin{boiteCodeTex}{}
  \rectangle
  \rectangle[largeur = 4cm, hauteur = 1.5cm]
  \rectangle[couleur = couleurSec-300]
  \rectangle[largeur = 8cm, hauteur = 2cm, couleur = couleurTer-400]
\end{boiteCodeTex}

\titreSection{Schémas de planètes, points et vecteurs}

Commande pour tracer des ``planètes'' :
\begin{boiteCodeTex}{}
  \begin{tikzpicture}
    \tikzSchemaPlanete
  \end{tikzpicture}
  \begin{tikzpicture}
    \tikzSchemaPlanete[satellite = B, rayon = 30 pt, orbite = 40 pt, remplissage = couleurQuat-100]
  \end{tikzpicture}
  \begin{tikzpicture}
    \tikzSchemaPlanete[centre = T, satellite = La lune, orbite = 40 pt, contour = couleurSec]
  \end{tikzpicture}
\end{boiteCodeTex}

Quelques commandes pour tracer des points avec un nom et des vecteurs, pour les vecteurs il faut préciser l'origine et la longueur du vecteur selon x et y :
\begin{boiteCodeTex}{cote a cote = 3cm}
  \begin{tikzpicture}
    \tikzLabel (0, 0) {$A$} (-0.1, 0.3)
    \tikzLabel(2, 1) {$B$} (2, 1.3)
    \tikzVecteur[couleurQuat] (0, 0) (2, 1) {$\vv{AB}$} [below = 6pt]
    \tikzVecteur* (0, -0.3) (2, 0) {}
    \tikzLabel* (1, -0.7) {$d$}
  \end{tikzpicture}
\end{boiteCodeTex}

\titreSection{Tube à essais}

Pour tracer des tubes à essais :
\begin{boiteCodeTex}{}
  \begin{tikzpicture}
    \tikzTubeEssais
  \end{tikzpicture}
  %
  \tubeEssaisSolution{couleurTer-400}
  %
  \begin{tikzpicture}
    \tikzPhaseBasTubeEssais [couleur = red-200, hauteur = 0.9]
    \tikzTubeEssais
  \end{tikzpicture}
  %
  \begin{tikzpicture}
    \tikzPhaseBasTubeEssais[hauteur = 0.4, couleur = yellow-200]
    \tikzPhaseTubeEssais[hauteur = 0.4, phase = 0.7, couleur = blue-300]
    \tikzPhaseTubeEssais[hauteur = 1, phase = 1.25, couleur = red-150]
    \tikzTubeEssais
  \end{tikzpicture}
\end{boiteCodeTex}

\titreSection{plan de classe}

Pour faire un plan de classe :
\begin{boiteCodeTex}{}
  \begin{center}
    \begin{tikzpicture}
      \texteRectangle(0, 0) (9, 2) {\large Tableau}
    \end{tikzpicture}
    \bigskip

    \rang (A, B) (C, D, E)
    \rang (F, G) (H) (I, J)
    \rang [largeur = 2, hauteur = 1.5] () () ()
  \end{center}
\end{boiteCodeTex}

%%%%
\begin{boiteCodeTex}{listing only}
  \palette{couleurPrim}{green}
  \palette{couleurSec}{cyan}
  \palette{couleurTer}{blue}
  \palette{couleurQuat}{orange}
\end{boiteCodeTex}
\palette{couleurPrim}{green}
\palette{couleurSec}{cyan}
\palette{couleurTer}{blue}
\palette{couleurQuat}{orange}
\newpage
\titrePartie{Plan de travail}\label{plan_de_travail}

\titreSection{Commandes spéciales}

Plusieurs commandes sont définies pour faciliter l'écriture d'une fiche élève pour un plan de travail.

La première catégorie de commande permet d'afficher des boîtes pour organiser le contenu par activités
\begin{boiteCodeTex}{}
  \begin{activite}{titre = Mon super titre, label = mon_activite, duree = 4h}
    Une simple boite d'activité, avec titre et durée réglables. Le label permet de tracer des flèches entre les activités.
  \end{activite}

  \begin{activite}{}
    Boite par défaut.
  \end{activite}

  \begin{TP}{titre = un TP, label = un_TP}
  \end{TP}

  \begin{TPNumerique}{titre = TP python, label = python}
  \end{TPNumerique}

  \phantom{b}
\end{boiteCodeTex}

L'idée de ces boîtes étant de présenter le contenu simplifié de chaque activité, en indiquant leur prérequis et objectifs.
Il y a aussi une boîte pour la tâche finale (une simple boîte avec un titre)
\begin{boiteCodeTex}{}
  \begin{tacheFinale} \end{tacheFinale}
\end{boiteCodeTex}

Pour la progression, il y a trois commandes utilisables, \lstinline|\flecheProgression| qui permet d'afficher une flèche stylisée de progression ; l'environnement \lstinline|programmeSeance|, qui permet de contenir des blocs de séances et enfin \lstinline|\seance| qui permet d'afficher une boîte par séance pour le suivi des élèves.

\begin{boiteCodeTex}{}
  \flecheProgression{boucles = 2}
  
  \begin{programmeSeance}[nombre = 2]
    \seance \seance
  \end{programmeSeance}
  \begin{programmeSeance}[nombre = 2, distance = 0pt]
    \seance \seance[Une information utile pour les élèves.]
  \end{programmeSeance}
\end{boiteCodeTex}
Les commandes sont faites pour être utilisée en tandem, car \lstinline|\flecheProgression| retire un espace vertical pour que les blocs de séances soient bien positionnés sur les flèches.
L'ajout d'une étoile permet d'éviter ce réglage interne de longueur.

\begin{boiteCodeTex}{}
  \vspace*{-36 pt}
  \flecheProgression*{boucles = 1}

  Pas d'espace négatif ajouté !
\end{boiteCodeTex}

Note : n'importe quelle boîte basée sur \lstinline|tcolorbox| peut être utilisée dans \lstinline|programmeSeance|.

\newpage
\titreSection{Exemple complet}

Voilà un exemple complet de mise en page avec toutes les commandes :

\begin{boiteCodeTex}{}
  \titre{Activités à réaliser}

  \begin{importants}
    Ce document, \important{qui sera ramassé et évalué,} présente les activités et travaux pratiques à réaliser pendant les 4 semaines du chapitre.
    À chaque séance (classe entière ou demi-groupe), tu es libre de choisir quelle activité ou TP réaliser avec ton groupe.
    Tous les documents sont sur le bureau du professeur.
  \end{importants}

  \setcounter{activiteNum}{0}
  \setcounter{TPNum}{0}
  \begin{multicols}{2}
    \begin{activite}{titre = Ordres de grandeur, label = ordre_grandeur}
      \begin{objectifs}  
        \item Revoir les puissances de 10.
        \item Apprendre à raisonner en ordres de grandeur.
      \end{objectifs}
    \end{activite}
    %
    \begin{TP}{titre = Le modèle de l'atome, label = modele_atome}
      \begin{objectifs}
          \item Découvrir la méthode scientifique.
          \item Utiliser la méthode scientifique pour étudier l'évolution du modèle de l'atome.
      \end{objectifs}
    \end{TP}
  \end{multicols}
  \begin{multicols}{2}
    %
    \begin{activite}{titre = Cortège électronique, duree = 1 h 30, label = cortege_electrons}
      \begin{prerequis}
        \item Connaître la structure d'un atome.
        \item Savoir qu'un atome a autant d'électrons qu'il a de protons.
      \end{prerequis}
      %
      \begin{objectifs}
        \item Comprendre que les électrons s'organisent en couches électroniques.
        \item Comprendre la règle de remplissage des couches électroniques.
      \end{objectifs}
    \end{activite}
    %
    \begin{TP}{titre = Fabriquer un atome, duree = 1 h 30, label = atome}
      \begin{objectifs}
        \item Étudier la composition d'un atome.
        \item Comprendre que le nombre de protons définit un élément chimique.
        \item Savoir distinguer un ion d'un atome.
        \item Comprendre la notion d'éléments isotopes.
      \end{objectifs}
    \end{TP}
  \end{multicols}

  \nomPrenomClasse*
  \titre{Progression des activités} \smallskip
  
  \flecheProgression{boucles = 3}
  
  \setcounter{seanceNum}{0}
  \begin{programmeSeance}
    \seance \seance \seance
  \end{programmeSeance}
  
  \begin{programmeSeance}
    \seance[\small Courte évaluation sur la structure d'un atome.]
    \seance \seance
  \end{programmeSeance}
  
  \begin{programmeSeance}[nombre = 2, distance = 0 pt]
    \seance[\important{Tâche finale}]
    \seance[\important{Évaluation du chapitre}]
  \end{programmeSeance}
  
  \begin{tacheFinale}
    \important{Par groupe de 4,} choisir un élément du tableau périodique et réaliser sa case au format A4 $\num{29,7} \times \qty{21,0}{\cm\squared}$.
    La case devra contenir des informations microscopique (structure électronique) et des informations macroscopique (dans quels objets on trouve l'élément, sous quels formes naturelles l'élément se trouve sur Terre, des propriétés remarquables ou amusantes, etc.)
  \end{tacheFinale}
  \titre{Évaluation de l'autonomie}

  \important{Les différents degrés d'autonomie}
  
  \begin{enumerate}[label = \Alph*]
    \item Je planifie librement mon apprentissage, je coopère avec mes camarades et je sollicite de l'aide pour valider les travaux réalisés.
    \item Je travaille seul-e ou avec mes camarades à partir des documents et je sollicite régulièrement de l'aide pour avancer.
    \item J'avance uniquement quand le professeur est là pour m'aider, je n'arrive pas à planifier mon travail ou je ne fais que recopier les réponses d'un de mes camarades.
    \item J'utilise des stratégies pour éviter d'apprendre et je refuse d'essayer de faire les activités.
  \end{enumerate}
  
  \begin{tableauCompetences}
    AUTO & Travailler de manière autonome \\
  \end{tableauCompetences}
\end{boiteCodeTex}

\end{document}
