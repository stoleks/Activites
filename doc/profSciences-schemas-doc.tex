\titrePartie{Schémas}

Une règle générale pour les commandes qui permettent de tracer des schémas, c'est que si la commande a pour préfixe tikz, elle doit être utilisée dans un environnement tikzpicture.

Une commande pour tracer des rectangles colorés de tailles variables :
\begin{boiteCodeTex}{}
  \rectangle
  \rectangle[largeur = 4cm, hauteur = 1.5cm]
  \rectangle[couleur = couleurSec-300]
  \rectangle[largeur = 8cm, hauteur = 2cm, couleur = couleurTer-400]
\end{boiteCodeTex}

Une commande pour tracer des ``planètes'' :
\begin{boiteCodeTex}{}
  \begin{tikzpicture}
    \tikzSchemaPlanete
  \end{tikzpicture}
  \begin{tikzpicture}
    \tikzSchemaPlanete[centre = A, satellite = B, rayon = 50 pt, orbite = 60 pt, remplissage = couleurQuat-100]
  \end{tikzpicture}
  \begin{tikzpicture}
    \tikzSchemaPlanete[satellite = La lune, rayon = 15 pt, orbite = 60 pt, contour = couleurSec]
  \end{tikzpicture}
\end{boiteCodeTex}

%%%%
\begin{boiteCodeTex}{listing only}
  \palette{couleurPrim}{green}
  \palette{couleurSec}{cyan}
  \palette{couleurTer}{yellow}
  \palette{couleurQuat}{orange}
\end{boiteCodeTex}
\palette{couleurPrim}{green}
\palette{couleurSec}{cyan}
\palette{couleurTer}{blue}
\palette{couleurQuat}{orange}