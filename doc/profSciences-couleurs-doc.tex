\titrePartie{Gestion des couleurs}

\titreSection{Couleurs flexoki}

J'utilise la palette flexoki pour la couleur \url{https://stephango.com/flexoki}. 
Les huits couleurs et leurs 13 nuances sont définis en suivant la convention de flexoki : <nom de la couleur>-<teinte>.
\begin{boiteCodeTex}{}
  \bfseries \textcolor{blue-200}{Un}
  \textcolor{blue-400}{texte}
  \textcolor{blue-600}{en}
  \textcolor{blue-800}{dégradé}.
\end{boiteCodeTex}


\titreSection{Palette de couleurs}

Pour pouvoir changer facilement de couleurs et avoir une palette cohérente, les couleurs \lstinline|couleurPrim| (couleur primaire), \lstinline|couleurSec| (secondaire), \lstinline|couleurTer| (tertiaire) et \lstinline|couleurQuat| (quaternaire) sont définies avec la commande palette \lstinline|\palette{<alias>}{<couleur>}|.

ce qui permet d'appeler ces alias comme des couleurs flexoki normale, avec la teinte 600 comme défaut
\begin{boiteCodeTex}{}
  \bfseries
  \textcolor{couleurPrim}{Un}
  \textcolor{couleurSec-700}{texte}
  \textcolor{couleurTer}{légèrement}
  \textcolor{couleurQuat}{coloré}.
\end{boiteCodeTex}
En théorie n'importe quelle couleur définie avec le format <couleur>-<50, 100, 150, 200, ..., 900, 950> peut être utilisé avec la commande palette.

Pour changer de thème coloré, il suffit donc d'appeler la commande palette avec la couleur souhaité pour changer toutes les couleurs des commandes internes.
En guise d'illustration, chaque partie de ce document a son propre thème coloré.

%%%%
\begin{boiteCodeTex}{listing only}
  \palette{couleurPrim}{cyan}
  \palette{couleurSec}{blue}
  \palette{couleurTer}{purple}
  \palette{couleurQuat}{red}
\end{boiteCodeTex}
\palette{couleurPrim}{cyan}
\palette{couleurSec}{blue}
\palette{couleurTer}{purple}
\palette{couleurQuat}{red}