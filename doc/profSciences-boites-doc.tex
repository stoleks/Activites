\titrePartie{Mise en page}

%%
\titreSection{Titre et sections}

\addtocounter{part}{-1}
\addtocounter{section}{-1}
Les commandes de section sont définies en parallèle des commandes classique \lstinline|\section, \subsection|, etc. :
\begin{boiteCodeTex}{}
  \titrePartie*{Une partie d'exemple}

  \titreSection*{Une section d'exemple}

  \titreSousSection{Une sous-section d'exemple}
\end{boiteCodeTex}

Il existe aussi des commandes pour afficher les titres des activités ou des TP, ou bien des exercices :
\begin{boiteCodeTex}{}
  \titre{Un simple titre sans numérotation}
  \titreActivite{Une activité}
  \titreActivite*{Une activité avec moins de numérotation}
  \titreTP{Un TP}
  \titreTP*{Une activité expérimentale}

  \exercice{Un exercice à faire}
\end{boiteCodeTex}
Note : la commande \lstinline|\exercice| entraîne la remise à zéro des compteurs de document et de questions.

%%
\titreSection{Boîtes variées de mise en page}

Plusieurs boites sont définies pour permettre d'avoir une mise en page consistante entre différentes activités.
\begin{boiteCodeTex}{}
  \begin{contexte}
    Une boîte pour introduire une activité.
  \end{contexte}

  \begin{prerequis}
    \item Il faut connaître leur noms,
  \end{prerequis}

  \begin{objectifs}
    \item mais on peut vite faire un document riche,
    \item avec les bonnes commandes.
  \end{objectifs}

  \begin{doc}{Le titre}{le_label}
    Le contenu du document
  \end{doc}

  \begin{doc}{Un deuxième document}{le_label_ii}
    La numérotation est automatique
  \end{doc}

  \begin{importants}
    Pour mettre en valeur des éléments importants de l'activité ou du cours !
  \end{importants}

  \extrait[Source]{Une citation ou un extrait.}
  \extrait{Un extrait, mais sans source associée.}
  C'est aussi possible de n'avoir que la mise en page pour la source.
  \sourceExtrait{Moi même}
\end{boiteCodeTex}

Les courbes des boîtes d'objectifs et de prérequis sont réglées pour ``s'emboiter'' dans un plan de travail, cf.~\ref{plan_de_travail}.

En plus de ces boîtes, il y aussi des boîtes simples sans but précis
\begin{boiteCodeTex}{}
  \begin{boite}
    Une boite toute simple.
  \end{boite}

  \begin{boiteColoree}
  \end{boiteColoree}
  
  \begin{boiteColoree}[yellow-150]
    Une simple boite avec des couleurs.
  \end{boiteColoree}

  \begin{boiteColoree}[magenta-200]
    Bla bla.

    Pratique sur plusieurs lignes aussi !
  \end{boiteColoree}

  \begin{importants}
    Pour mettre en valeur un passage à retenir ou important.

    Le texte peut être cassé en paragraphes.
  \end{importants}
\end{boiteCodeTex}