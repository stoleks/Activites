\titrePartie{Mise en page}

%%
\titreSection{Titre et sections}

\addtocounter{part}{-1}
\addtocounter{section}{-1}
Les commandes de section sont définies en parallèle des commandes classique \lstinline|\section, \subsection|, etc. :
\begin{boiteCodeTex}{}
  \titrePartie*{Une partie d'exemple}

  \titreSection*{Une section d'exemple}

  \titreSousSection{Une sous-section d'exemple}
\end{boiteCodeTex}

Il existe aussi des commandes pour afficher les titres des activités ou des TP, ou bien des exercices :
\begin{boiteCodeTex}{}
  \titre{Un simple titre sans numérotation}
  \titreActivite{Une activité}
  \titreActivite*{Une activité avec moins de numérotation}
  \titreTP{Un TP}
  \titreTP*{Une activité expérimentale}

  \exercice{Un exercice à faire}
\end{boiteCodeTex}
Note : la commande \lstinline|\exercice| entraîne la remise à zéro des compteurs de document et de questions.

%%
\titreSection{Boîtes variées de mise en page}

Plusieurs boites sont définies pour permettre d'avoir une mise en page consistante entre différentes activités.
\begin{boiteCodeTex}{}
  \begin{contexte}
    Une boîte pour introduire une activité.
  \end{contexte}

  \begin{prerequis}
    \item Connaître LaTeX.
  \end{prerequis}

  \begin{objectifs}
    \item Savoir utiliser ce paquet.
  \end{objectifs}

  \begin{doc}{Le titre}
    Le contenu du document
  \end{doc}

  \begin{doc}{Un deuxième document}[\label{label_doc}]
    La numérotation est automatique
  \end{doc}
  Le label permet de faire référence au document~\ref{label_doc}.

  \begin{importants}
    Pour mettre en valeur des éléments importants de l'activité ou du cours !
  \end{importants}

  \begin{boiteMateriel}{Matériel élève}
  \end{boiteMateriel}

  \extrait[Source]{Une citation ou un extrait.}
  \extrait{Un extrait, mais sans source associée.}
  C'est aussi possible de n'avoir que la mise en page pour la source.
  \sourceExtrait{Moi même}
\end{boiteCodeTex}

Les courbes des boîtes d'objectifs et de prérequis sont réglées pour ``s'emboiter'' dans un plan de travail, cf.~\ref{plan_de_travail}.

En plus de ces boîtes, il y aussi des boîtes simples sans but précis
\begin{boiteCodeTex}{}
  \begin{boite}
    Une boite toute simple.
  \end{boite}

  \begin{boiteColoree}
  \end{boiteColoree}
  
  \begin{boiteColoree}[yellow-150]
    Une simple boite avec des couleurs.
  \end{boiteColoree}

  \begin{boiteColoree}[magenta-200]
    Bla bla.

    Pratique sur plusieurs lignes aussi !
  \end{boiteColoree}

  \begin{importants}
    Pour mettre en valeur un passage à retenir ou important.

    Le texte peut être cassé en paragraphes.
  \end{importants}

  On peut aussi simplement \important{insister sur un passage précis.}
\end{boiteCodeTex}

%%
\titreSection{Différents outils de mise en page}

\begin{boiteCodeTex}{}
  \separationBlocs
    { Pour séparer une page en deux blocs. }
    { Ils sont de tailles égales par défaut. }

  \separationBlocs
    { \lipsum[1-2] }[0.6]
    { \lipsum[1] }[0.35]

  \separationTroisBlocs{ Bloc 1 }{ Bloc 2 }{ Bloc 3 }
\end{boiteCodeTex}

\begin{boiteCodeTex}{}
  \centering
  \image{0.5}{images/atomes/modele_quantique}
  
  \legende{Les électrons sont délocalisés autour du noyau.}
\end{boiteCodeTex}

\begin{boiteCodeTex}{}
  \qrcodeCote{https://forge.apps.education.fr/jedrecyalexandre/LaTex_physique-chimie_lycee_GT}

  \lipsum[1]
\end{boiteCodeTex}

%%%%
\begin{boiteCodeTex}{listing only}
  \palette{couleurPrim}{blue}
  \palette{couleurSec}{green}
  \palette{couleurTer}{cyan}
  \palette{couleurQuat}{magenta}
\end{boiteCodeTex}
\palette{couleurPrim}{blue}
\palette{couleurSec}{green}
\palette{couleurTer}{cyan}
\palette{couleurQuat}{magenta}
