\titrePartie{Rédaction de fiches avec leur corrections}

\titreSection{Texte à trou et question à remplir}

Plusieurs commandes permettent de générer plusieurs fichiers .pdf à partir d'un seul document .tex, en appelant la commande \lstinline|\modeCorrection| pour l'affichage de la correction ou \lstinline|\modeEleve| pour repasser en version ``fiche à remplir''. 
C'est le mode élève à remplir qui est le statut par défaut dans un document.

La première commande permet d'afficher des textes à trous, avec réglage automatique de la largeur, réglage manuel en fraction de la largeur de la page ou avec un nombre de lignes à remplir.
\begin{boiteCodeTex}{}
Voilà un \texteTrou*{texte à trou}. % largeur automatique, texte gras
Suivit d'un autre \texteTrou[0.5]{texte à trou}. % largeur = la moitié de la page
Et encore un \texteTrou(1){texte à trou} ! % complète la ligne actuelle et trace une ligne

  
Un remarque importante : le nombre de lignes prend toujours le pas sur la largeur,
si les deux sont réglées \texteTrou[0.3](0){comme ici.}
Indiquer (0) revient à demander à compléter la ligne actuelle.
\end{boiteCodeTex}

On notera que la commande ajoute un espacement vertical pour faciliter le remplissage pour les personnes qui écrivent gros (j'écris assez gros et cet espacement me convient).
Et voilà la même boite, mais en mode correction :
\begin{boiteCodeTex}{}
\modeCorrection
Voilà un \texteTrou*{texte à trou}. % largeur automatique, texte gras
Suivit d'un autre \texteTrou[0.5]{texte à trou}. % largeur = la moitié de la page
Et encore un \texteTrou(1){texte à trou} ! % complète la ligne actuelle et trace une ligne

  
Un remarque importante : le nombre de lignes prend toujours le pas sur la largeur,
si les deux sont réglées \texteTrou[0.3](0){comme ici.}
Indiquer (0) revient à demander à compléter la ligne actuelle.
\modeEleve
\end{boiteCodeTex}

La seconde commande permet d'afficher des questions, avec un nombre de lignes réglables pour les réponses :
\begin{boiteCodeTex}{}
  \question{La question.}{La réponse.}
  
  \question{Une question avec de la place pour répondre}{Une autre réponse.}[2]

  \modeCorrection \setcounter{questionNum}{0}
  \question{La question.}{La réponse.}
  \question{Une question avec de la place pour répondre}{Une autre réponse.}[2]
  \modeEleve
\end{boiteCodeTex}

Enfin il y a une paire de commande qui permet d'afficher du contenu uniquement pour la correction (\lstinline|\correction{}|) ou uniquement pour la fiche à remplir (\lstinline|\pasCorrection{}|) :
\begin{boiteCodeTex}{}
  \correction{Uniquement dans le corrigé.}
  \pasCorrection{Uniquement dans la fiche à remplir.}

  \modeCorrection
  \correction{Uniquement dans le corrigé.}
  \pasCorrection{Uniquement dans la fiche à remplir.}
  \modeEleve
\end{boiteCodeTex}

%%
\titreSection{Différents types de questions}

Pour rester cohérent d'une activité à une autre, j'ai plusieurs commandes pour afficher certains types de questions récurrentes.
\begin{boiteCodeTex}{}
  \question{Une question numérotée classique.}{}

  \mesure Quand il faut réaliser une expérience.

  \programmation Quand il faut programmer.

  \schematisation Quand il faut schématiser.

  \documentaire Quand il faut rédiger un compte-rendu.

  \telechargement Quand il faut scanner un QR code pour télécharger une appli ou regarder une vidéo.

  \mesure* La version étoilée permet de supprimer l'indentation.
\end{boiteCodeTex}

%%
\titreSection{Variation entre deux sujets}

Pour varier les sujets entre deux élèves, on peut appeler la commande \lstinline|\variationSujet{A}{B}|, puis les commandes \lstinline|\sujetA| ou \lstinline|\sujetB| pour passer d'un sujet à un autre.
\begin{boiteCodeTex}{}
  \sujetA
  Pour le sujet \variationSujet{A}{B}, on peut avoir des variation entre certaines \variationSujet{valeurs}{grandeurs}.
  
  \sujetB
  Pour le sujet \variationSujet{A,}{B,} on peut avoir des variation entre certaines \variationSujet{valeurs.}{grandeurs.}
\end{boiteCodeTex}
Pour tirer pleinement partie de ces commandes, c'est plus pratique de définir son sujet dans un fichier séparé et de l'inclure ensuite avec la commande \lstinline|\input|
\begin{boiteCodeTex}{listing only}
  \sujetA \input{mon_super_sujet}
  \sujetB \input{mon_super_sujet}
\end{boiteCodeTex}
en faisant comme ça, on a en plus un fichier avec une alternance naturelle entre les deux sujets si on l'imprime, ce qui facilite la distribution.

%%
\titreSection{Compétences et appréciations}

Pour afficher une grille d'évaluation des compétences, on peut utiliser l'environnement \lstinline|tableauCompetence| qui est un tableau basé sur \lstinline|tabularray| :
\begin{boiteCodeTex}{}
  \begin{tableauCompetences}
    RCO & Connaître son cours \\
    APP & Extraire une information \\
    COM & On peut en mettre autant qu'on veut à la suite... les cases sont centrées verticalement.
  \end{tableauCompetences}
\end{boiteCodeTex}

Et pour les appréciations il y a une simple boite à taille réglable et qui ne s'affiche qu'en mode élève :
\begin{boiteCodeTex}{}
  \appreciation{4 cm}
  \modeCorrection
  \appreciation{4 cm}
  \modeEleve
\end{boiteCodeTex}

%%%%
\begin{boiteCodeTex}{listing only}
  \palette{couleurPrim}{orange}
  \palette{couleurSec}{red}
  \palette{couleurTer}{magenta}
  \palette{couleurQuat}{green}
\end{boiteCodeTex}
\palette{couleurPrim}{orange}
\palette{couleurSec}{red}
\palette{couleurTer}{magenta}
\palette{couleurQuat}{green}