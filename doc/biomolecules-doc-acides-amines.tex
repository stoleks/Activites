\section{Acides alpha aminés et protéines}

%%
\subsection{Formules topologiques}

\begin{boiteCodeTex}{}
  \chemfig{!\arginine}
  \chemfig{!\histidine}
  \chemfig{!\lysine}
  \chemfig{!\aspartique}
\end{boiteCodeTex}
  
\begin{boiteCodeTex}{}
  \chemfig{!\glutamique}
  \chemfig{!\serine}
  \chemfig{!\threonine}
  \chemfig{!\asparagine}
\end{boiteCodeTex}
  
\begin{boiteCodeTex}{}
  \chemfig{!\glutamine}
  \chemfig{!\cysteine}
  \chemfig{!\selenocysteine}
  \chemfig{!\glycine}
\end{boiteCodeTex}
  
\begin{boiteCodeTex}{}
  \chemfig{!\proline}
  \chemfig{!\alanine}
  \chemfig{!\valine}
  \chemfig{!\isoleucine}
  \chemfig{!\leucine}
\end{boiteCodeTex}
  
\begin{boiteCodeTex}{}
  \chemfig{!\methionine}
  \chemfig{!\phenylalanine}
  \chemfig{!\tyrosine}
  \chemfig{!\tryptophane}
\end{boiteCodeTex}

%%
\subsection{Formules semi-développées, représentation de Fischer et de Cram}

\begin{boiteCodeTex}{}
  \chemfig{!\alanineSemiDev} \qq{}
  \chemfig{!\asparagineSemiDev} \qq{}
  \chemfig{!\glycineSemiDev} \\[8pt]
  \chemfig{!\cysteineSemiDev} \\[8pt]
\end{boiteCodeTex}

\begin{boiteCodeTex}{}
  \chemfig{!\alanineL} \quad
  \chemfig{!\alanineD} \quad
  \chemfig{!\valineL} \quad
  \chemfig{!\valineD}
\end{boiteCodeTex}

%%
\subsection{Polypeptides et groupements prosthétiques}

\begin{boiteCodeTex}{}
  \chemfig{ [:-30] H_2N !\alanineH !\HN !\glycineB !\NH !\cysteineH !\HN !\isoleucineB !\NH !\valineH OH }
\end{boiteCodeTex}

\begin{boiteCodeTex}{}
  \chemfig[atom sep = 18pt]{!\hemeB}
\end{boiteCodeTex}
