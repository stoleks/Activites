\documentclass[12pt]{extarticle}

%%%% paramètres généraux et commandes prédéfinies
\usepackage[french]{babel}
\usepackage[T1]{fontenc}
\usepackage[utf8]{inputenc}

\usepackage{lmodern} % pour étirer fontawesome
\usepackage{geometry} % définition des marges
\usepackage{profSciences} % mise en page et autre
\usepackage{biomolecules} % pour dessiner des formules chimiques

\input{../_couleurs}
\input{../_abbreviations}
%%%%%%%%%%%%%%%%%%%%%%%%%%%%%%%%%%%%%%%%%%%%%%%%%%%%%%%%%%%%%
%%%% figures simples
\newcommand{\tkzRect}[4]{
  \fill[color=#1] (#2,#4) -- (-#2,#4) -- (-#2,#3) -- (#2,#3);
}
\newcommand{\tkzEllipse}[4]{
  \fill[color=#1] (0,#3) ellipse (#2 and #4);
}

% \tkzCercle {x}{y} {couleur} {rayon}
\newcommand{\tkzCercle}[4]{
  \filldraw [#3] (#1, #2) circle (#4pt);
}
% \tkzCercleLigne {x}{y} {couleurFond}{couleurTrait} {rayon}
\newcommand{\tkzCercleLigne}[5]{
  \filldraw [color = #4, fill = #3, very thick] (#1, #2) circle (#5pt);
}

%%%% tube à essais
\newcommand{\tkzTubeEssais}[3]{
  \draw[thick] (#1,#2) -- (#1,0) arc (0:-180:#1) -- (-#1,#2);
  \draw[thick] (0,#2) ellipse (#1 and #3);
}
\newcommand{\tkzBasTubeEssais}[5]{
  \fill[color=#1] (-#2,#3) -- (#2,#3) arc (0:-180:#2);
  \tkzRect{#1}{#2}{#3 - 0.01}{#4}
  \tkzEllipse{#1!85!black}{#2}{#4}{#5}
}
\newcommand{\tkzPhaseTubeEssais}[5]{
  \tkzRect{#1}{#2}{#3}{#4}
  \tkzEllipse{#1}{#2}{#3}{#5}
  \tkzEllipse{#1!85!black}{#2}{#4}{#5}
}

%%%% Point et vecteurs
\newcommand{\tkzLabel}[3]{
  \node at (#1, #2) {#3};
}
\newcommand{\tkzPointLabel}[3]{
  \filldraw (#1, #2) circle (2pt) node[above] {#3};
}
% \tkzVecteur [couleur] (x) [longueur x] (y) [longueur y] {legende} [Positionition legende] 
% ajouter une * à la fin transforme la flèche en double flèche <->
\NewDocumentCommand{\tkzVecteur}{O{black} r() O{0} r() O{0} m O{right} s}{
  \IfBooleanTF{#8}{
    \draw[#1, <->, very thick] (#2, #4) -- (#2 + #3, #4 + #5) node[#7] {#6};
  }{
    \draw[#1, ->, very thick] (#2, #4) -- (#2 + #3, #4 + #5) node[#7] {#6};
  }
}
% \tkzLegende (x) (y) [longueur fleche] {légende} 
% ajouter une * passe de la version gauche -> à la version droite <-
\NewDocumentCommand{\tkzLegende}{O{black} r() r() O{1.25} m s}{
  \IfBooleanTF{#6}{
    \draw[#1, ->, very thick] (#2 + #4, #3) node[right] {#5} -- (#2, #3);
  }{
    \draw[#1, ->, very thick] (#2, #3) node[left] {#5} -- (#2 + #4, #3);
  }
}

\newcommand{\barrePourcentage}[1]{%
  \begin{tikzpicture}
    \fill[color=couleurSec]    (0.0,    0.0) rectangle (#1*8ex, 1.5ex);
    \fill[color=couleurSec!20] (#1*8ex, 0.0) rectangle (8.0ex,  1.5ex);
  \end{tikzpicture}
}

%% Trace une flèche de progression pour les plans de travail
% \flecheProgression {<nombre de boucles>} [<largeur>] [<espacement vertical>]
\NewDocumentCommand{\flecheProgression}{m O{17} O{2.5}}{%
  \strut\vspace*{8pt}
  \begin{center}
    \begin{tikzpicture}
      \def\max{\numexpr((#1-1)*2)}
      % Premier bout pour l'alignement
      \draw[
        line width = 20pt,
        rounded corners = 10mm,
        color = couleurSec,
      ]
        (0, {(\max-0)*#3}) -- (1, {(\max-0)*#3});
      % Partie centrale répétée
      \draw[
        line width = 20pt,
        rounded corners = 10mm,
        color = couleurSec,
      ]
        \foreach \x in {0,2,...,\max}  {
          \ifnum \x < \max
            ( 1, {(\max-\x)  *#3}) -- (#2,  {(\max-\x)  *#3}) --
            (#2, {(\max-\x-1)*#3}) -- ( 0,  {(\max-\x-1)*#3}) --
            ( 0, {(\max-\x-2)*#3}) -- (1.1, {(\max-\x-2)*#3})
          \fi
        };
      % Flèche finale
      \draw[
        -{Triangle [width = 36pt, length = 16pt]}, 
        line width = 20pt,
        color = couleurSec
      ]
        (0.8, 0) -- (#2, 0);
    \end{tikzpicture}
  \end{center}
}

%%%%%%%%%%%%%%%%%%%%%%%%%%%%%%%%%%%%%%%%%%%%%%%%%%%%%%%%%%%%%
%%%% plan de classe
% Trace un texte centré dans un cadre (x, x+l) -- (y, y+h)
% #1 couleur cadre ; #2 Positionition x ;
% #3 largeur l ;     #4 Positionition y ;
% #5 hauteur h ;     #6 texte.
\NewDocumentCommand{\texteCadre}{O{black} r() O{2} r() O{2} m}{
  \filldraw [fill=white, draw=#1, ultra thick] (#2, #4) rectangle (#2 + #3, #4 + #5);
  \node at (#2 + #3/2, #4 + #5/2) [font=\sffamily] {\textbf{#6}};
}

%% place dans la classe
\NewDocumentCommand{\place}{r() m}{
  \texteCadre(#1)[3](0)[2] {#2}
}

%% Pour tracer une rangée d'élève avec 2 ou 3 colonnes
% \rang {<numero rangee>} {<eleves>} {<eleves>} [<eleves>]
\ExplSyntaxOn
% Position de la place horizontale
\int_new:N \l_rangPositionX_int
\NewDocumentCommand{\rang}{m >{\SplitList{,}} m >{\SplitList{,}} m >{\SplitList{,}} d[]}{
  \begin{tikzpicture}
    % Initialisation de la position horizontale
    \int_set:Nn \l_rangPositionX_int {0}
    % Première rangée
    \ProcessList{#2}{\rangImpl}
    \int_add:Nn \l_rangPositionX_int { 1 }
    % Deuxième rangée
    \ProcessList{#3}{\rangImpl}
    % Troisième rangée
    \IfValueT{#4}{
      \int_add:Nn \l_rangPositionX_int { 1 }
      \ProcessList{#4}{\rangImpl}
    }
  \end{tikzpicture}
  \bigskip
}
\NewDocumentCommand{\rangImpl}{m}{
  \int_add:Nn \l_rangPositionX_int { 3 }
  \place(\l_rangPositionX_int){#1} %
}
\ExplSyntaxOff


%%%% tube à essai de sang
\newcommand{\tubeEssaisSolution}[1]{
  \begin{tikzpicture}
    \tkzBasTubeEssais{#1}{0.25}{0}{0.75}{0.1} % contenu du tube
    \tkzTubeEssais{0.25}{1.5}{0.1} % tube
  \end{tikzpicture}
}

\newcommand{\tubeEssaisSangCentrifuge}[3]{
  \begin{tikzpicture}
    % phases dans le tube à essai
    \tkzBasTubeEssais{rougeSombre!75!white} {0.35}{0}{#1}{0.1}
    \tkzPhaseTubeEssais{gray!10!white}      {0.35}{#1}{#2}{0.1}
    \tkzPhaseTubeEssais{jauneClair!75!white}{0.35}{#2}{#3}{0.1}
    \tkzTubeEssais{0.35}{#3 + 1}{0.1}
    % Légende
    \tkzLegende(0.4)(#3 - 0.1) [1]{Plasma}*
    \tkzLegende(0.4)(#2 - 0.08)[1]{Globules blancs}*
    \tkzLegende(0.4)(-0.1)     [1]{Globules rouges}*
  \end{tikzpicture}
}
%%%% Ce fichier sert à déclarer les titres des chapitres des différents niveaux

%% Commun
\newcommand{\methode} {\chapitre{Outils pratiques}}

%% Seconde
%%%% Chapitre
\newcommand{\snd}{Seconde}
\newcommand{\sndCorp} {\chapitre{Corps purs et mélanges}}
\newcommand{\sndSolu} {\chapitre{Solutions}}
\newcommand{\sndMouv} {\chapitre{Mouvement et interactions}}
\newcommand{\sndAtom} {\chapitre{Structure de l'atome}}
\newcommand{\sndMole} {\chapitre{Des atomes à la matière}}
\newcommand{\sndLumi} {\chapitre{Ondes lumineuses et optique}}
\newcommand{\sndTran} {\chapitre{Transformations de la matière}}
\newcommand{\sndChim} {\chapitre{Transformations chimiques}}
\newcommand{\sndSign} {\chapitre{Signaux et capteurs}}

%%%% en-tête correspondant
\newcommand{\teteSndMeth} {\enTete[\snd]{\methode}}
\newcommand{\teteSndCorp} {\enTete[\snd]{\sndCorp}[1]}
\newcommand{\teteSndSolu} {\enTete[\snd]{\sndSolu}[2]}
\newcommand{\teteSndMouv} {\enTete[\snd]{\sndMouv}[3]}
\newcommand{\teteSndAtom} {\enTete[\snd]{\sndAtom}[4]}
\newcommand{\teteSndMole} {\enTete[\snd]{\sndMole}[5]}
\newcommand{\teteSndLumi} {\enTete[\snd]{\sndLumi}[6]}
\newcommand{\teteSndTran} {\enTete[\snd]{\sndTran}[7]}
\newcommand{\teteSndChim} {\enTete[\snd]{\sndChim}[8]}
\newcommand{\teteSndSign} {\enTete[\snd]{\sndSign}[9]}


%% Première ST2S
%%%% Chapitres
\newcommand{\premStss}{Première ST2S}
\newcommand{\premStssChim} {\chapitre{Sécurité chimique dans l'habitat}}
\newcommand{\premStssVisi} {\chapitre{Propagation de la lumière et vision}}
\newcommand{\premStssRedo} {\chapitre{Antiseptique et désinfectant, oxydoréduction}}
\newcommand{\premStssLumi} {\chapitre{Les infrarouges et leurs applications}}
\newcommand{\premStssStru} {\chapitre{Molécules d'intérêt biologique}}
\newcommand{\premStssBiom} {\chapitre{Biomolécules dans l’organisme}}
\newcommand{\premStssRout} {\chapitre{Sécurité routière}}
\newcommand{\premStssAlim} {\chapitre{Gestion des ressources naturelles et alimentation}}
\newcommand{\premStssElec} {\chapitre{Sécurité électrique dans l'habitat}}
\newcommand{\premStssPres} {\chapitre{Propriétés des fluides et pression sanguine}}
\newcommand{\premStssSono} {\chapitre{Ondes sonores et audition}}

%%%% en-tête
\newcommand{\tetePremStssMeth} {\enTete[\premStss]{\methode}     }
\newcommand{\tetePremStssChim} {\enTete[\premStss]{\premStssChim}[1]}
\newcommand{\tetePremStssVisi} {\enTete[\premStss]{\premStssVisi}[2]}
\newcommand{\tetePremStssRedo} {\enTete[\premStss]{\premStssRedo}[3]}
\newcommand{\tetePremStssLumi} {\enTete[\premStss]{\premStssLumi}[4]}
\newcommand{\tetePremStssStru} {\enTete[\premStss]{\premStssStru}[5]}
\newcommand{\tetePremStssBiom} {\enTete[\premStss]{\premStssBiom}[6]}
\newcommand{\tetePremStssRout} {\enTete[\premStss]{\premStssRout}[7]}
\newcommand{\tetePremStssAlim} {\enTete[\premStss]{\premStssAlim}[8]}
\newcommand{\tetePremStssElec} {\enTete[\premStss]{\premStssElec}[9]}
\newcommand{\tetePremStssPres} {\enTete[\premStss]{\premStssPres}[10]}
\newcommand{\tetePremStssSono} {\enTete[\premStss]{\premStssSono}[11]}


%% Terminale ST2S
%%%% Chapitres
\newcommand{\termStss}{Terminale ST2S}
\newcommand{\termStssOrga} {\chapitre{Représentation des molécules organiques}}
\newcommand{\termStssAlim} {\chapitre{Sécurité physico-chimique dans l'alimentation}}
\newcommand{\termStssImag} {\chapitre{La physique de l'imagerie médicale}}
\newcommand{\termStssBiom} {\chapitre{Biomolécules et alimentation}}
\newcommand{\termStssMedi} {\chapitre{De la molécule au médicament}}
\newcommand{\termStssEnvi} {\chapitre{Sécurité chimique dans l'environnement}}
\newcommand{\termStssDosa} {\chapitre{Analyser la composition d'un milieu}}
\newcommand{\termStssRout} {\chapitre{Sécurité routière}}
\newcommand{\termStssCosm} {\chapitre{L'usage responsable des cosmétiques}}

%%%% en-tête
\newcommand{\teteTermStssMeth} {\enTete[\termStss]{\methode}}
\newcommand{\teteTermStssOrga} {\enTete[\termStss]{\termStssOrga}[1]}
\newcommand{\teteTermStssRout} {\enTete[\termStss]{\termStssRout}[8]}
\newcommand{\teteTermStssAlim} {\enTete[\termStss]{\termStssAlim}[2]}
\newcommand{\teteTermStssEnvi} {\enTete[\termStss]{\termStssEnvi}[6]}
\newcommand{\teteTermStssImag} {\enTete[\termStss]{\termStssImag}[3]}
\newcommand{\teteTermStssDosa} {\enTete[\termStss]{\termStssDosa}[7]}
\newcommand{\teteTermStssBiom} {\enTete[\termStss]{\termStssBiom}[4]}
\newcommand{\teteTermStssMedi} {\enTete[\termStss]{\termStssMedi}[5]}
\newcommand{\teteTermStssCosm} {\enTete[\termStss]{\termStssCosm}[9]}

\input{../_tableau_periodique}

\palette{couleurPrim}{cyan}
\palette{couleurSec} {blue}
\palette{couleurTer} {purple}
\palette{couleurQuat}{red}

\setlength{\parskip}{0cm}
\setlength{\parindent}{0cm}
\renewcommand{\baselinestretch}{1}
% réglage du niveau (sous-section) ou s'arrête la table des matières
\setcounter{tocdepth}{2}

\geometry{
  a4paper, % format
  left=1.3cm, right=1.3cm, % marge horizontale
  top=2.2cm, bottom=2.1cm % marge verticale
}

%% Ces commandes sont adaptées du très bon paquet ProfLycée de Cedric Pierquet 
%% https://ctan.org/pkg/proflycee
\tcbset{
  style code Tex/.style = {%
    listing engine = listings,%
    listing options = {%
      breaklines = true,%
      breakatwhitespace = true,%
      style = tcblatex, basicstyle = \footnotesize\ttfamily,%
      tabsize = 2,%
      commentstyle = {\itshape\color{couleurSec-300}},
      keywordstyle = {\color{couleurSec}},%
      classoffset = 0,%
      keywords = {chemfig, definesubmol},%
      alsoletter = {-},%
      keywordstyle = {\color{couleurSec}}%
    }
  },
  cote a cote/.style = {%
    listing side text,%
    righthand width = #1%
  }
}

%% de Cedric Pierquet https://ctan.org/pkg/proflycee
\NewTCBListing{boiteCodeTex}{ O{couleurTer} m }{%
  enhanced, breakable,%
  flush right, boxrule = 1pt, colframe = #1!90,%
  sharp corners, top = 0mm, bottom = 0mm, left = 0.4em, right = 5mm,%
  before skip = \baselineskip, after skip = \baselineskip,%
  colback = white,%
  fontupper = \footnotesize, fontlower = \footnotesize,%
  title = {{\scriptsize\faCogs} Code \LaTeX},%
  fonttitle = \bfseries\footnotesize\sffamily, colbacktitle = #1,%
  style code Tex,%
  #2,%
}


%%%% doc
\begin{document}
  \titre{Biomolécules}
  \begin{center}
    Quelques commandes pour tracer des biomolécules dans le cadre du lycée.
  \end{center}

  \tableofcontents
  \newpage

  \section{Logique interne}

%%
\subsection{Nom des molécules}

Pour tracer une molécule, il suffit d'appeler \lstinline|\chemfig\{!\nomDeLaMolecule}|.
La représentation de base pour les molécules est la formule topologique, il faut ajouter un suffixe au nom pour passer à une autre représentation \important{si elle est définie, ce qui n'est pas du tout toujours le cas.} Les suffixes sont les suivants :

\begin{listePoints}[2]
  \item \lstinline{SemiDev} : formule semi-développée ;
  \item \lstinline{Dev} : formule développée ;
  \item \lstinline{Haw} : représentation de Haworth ;
  \item \lstinline{Cram} : représentation de Cram.
\end{listePoints}
Pour les acides aminés, il existe quatre autres suffixes
\begin{listePoints}[2]
  \item \lstinline{L} : représentation de Fischer gauche ;
  \item \lstinline{H} : pour tracer un polypeptide, la chaîne latérale est vers le haut ;
  \item \lstinline{D} : représentation de Fischer droite ;
  \item \lstinline{B} : pour tracer un polypeptide, la chaîne latérale est vers le bas.
\end{listePoints}

%%
\subsection{Commandes internes pour faciliter l'écriture}

Pour tracer les formules topologiques, 
j'utilise plusieurs commandes pour éviter d'avoir à spécifier en permanence les angles les plus courants
(\qty{60}{\degree}, \qty{50}{\degree}, etc.),
ou pour réutiliser des morceaux de molécules complexes

\begin{boiteCodeTex}{cote a cote = 2cm}
  \chemfig{-!\vide{::30} -} % Pour tracer une liaison invisible (utile pour les cycles incomplets)

  \chemfig{-!\vide{::-30}-}
\end{boiteCodeTex}
%
\begin{boiteCodeTex}{cote a cote = 2cm}
  \chemfig{-[:30] !\lh} % Pour tracer une liaison vers le haut (liaison haut = lh)

  \chemfig{-[:30] !\lb} % Pour tracer une liaison vers le bas (liaison bas = lb)
\end{boiteCodeTex}
%
\begin{boiteCodeTex}{cote a cote = 2cm}
  \chemfig{-[:30]!\lhb} % Pour tracer une liaison vers le haut puis vers le bas

  \chemfig{-[:30]!\lbh} % Pour tracer une liaison vers le bas puis vers le haut
\end{boiteCodeTex}
%
\begin{boiteCodeTex}{cote a cote = 2cm}
  \chemfig{-[:30]!\llh} % Pour tracer une liaison double vers le haut
  
  \chemfig{-[:30]!\llb} % Pour tracer une liaison double vers le bas
\end{boiteCodeTex}
%
\begin{boiteCodeTex}{cote a cote = 3cm}
  \chemfig{-[:-30]!\cis} % Pour tracer une liaison cis
  
  \chemfig{-[:-30]!\trans} % Pour tracer une liaison "trans" aplatie
\end{boiteCodeTex}
%
\begin{boiteCodeTex}{cote a cote = 2cm}
  \chemfig{-[:30]!\ldh} % Pour tracer une liaison développée vers le haut (l'angle est plus faible)
  
  \chemfig{-[:30]!\ldb} % Pour tracer une liaison développée vers le bas
\end{boiteCodeTex}
%
\begin{boiteCodeTex}{cote a cote = 2cm}
  \chemfig{-[:30]!\lldh} % Pour tracer une liaison double développée vers le haut
  
  \chemfig{-[:30]!\lldb} % Pour tracer une liaison double développée vers le bas
\end{boiteCodeTex}
%
\begin{boiteCodeTex}{cote a cote = 2cm}
  \chemfig[cram width = 5pt]{C !\cram{A}{B} (-[::90] R_1) -[::-30] R_2} % Pour tracer deux liaisons de cram autour d'un élément 
  
  \chemfig{-!\branche{A}{B}-} % Pour tracer deux liaisons à \qty{90}{\degree} autour d'un élément chimique
\end{boiteCodeTex}
%
\begin{boiteCodeTex}{cote a cote = 5.5cm}
  \chemfig{A- !\hexaOseHaw{!\lb B} -C} % Pour tracer des isomères du glucose
  
  \chemfig{A- !\pentaOseHaw{!\lb B}{!\lb C} -D} % Pour tracer des isomères du fructofuranose

  \chemfig{-[:30] 
    !\sterol {-A-} {-B--} {C-D-} 
      {-(-[::0] E)---} {---} {-(-[::0] F)---} 
  } % Pour tracer des stérols
\end{boiteCodeTex}

%%
\subsection{Coloriage de fonctions organiques et de parties de molécules}

\begin{boiteCodeTex}{}
\begin{tikzpicture}
  \node (base) at (0,0) {};
  \chemCarboxyle (-140pt, 14pt)
  \chemAmine (-90pt, 17pt)
  \chemAmide (78pt, 10pt)
  \node at (base) {
    \chemfig{!\alanine} $+$
    \chemfig{[:30]H_2N !\cysteineB OH} \reaction
    \chemfig{[:-30] H_2N !\alanineH !\HN !\cysteineB OH}
  };
\end{tikzpicture}
%
\begin{tikzpicture}
  \node (base) at (0,0) {};
  \chemAmide (-32pt,4pt);
  \node at (base) {\chemfig{[:-30] H_2N !\alanineH !\HN !\cysteineB OH}};
\end{tikzpicture}
%
\begin{tikzpicture}
  \node (base) at (0,0) {};
  \chemPolygone [rotation = -11] (36pt, 26pt)
  \node at (base) {\chemfig{!\adenosine}};
\end{tikzpicture}
%
\begin{tikzpicture}
  \node (base) at (0,0) {};
  \chemPolygone [rotation = 180, bords = 5] (9pt, 14pt)
  \chemPolygone [rotation = -11, couleur = couleurTer-200] (45pt, 26pt)
  \chemPentagoneHaw (-48pt,-30pt)
  \node at (base) {\chemfig{!\adenosineHaw}};
\end{tikzpicture}
%
\begin{tikzpicture}
  \node (base) at (0,0) {};
  \chemHexagoneHaw (-27pt,0pt)
  \node at (base) {\chemfig{!\glucoseHaw}};
\end{tikzpicture}
%
\begin{tikzpicture}
  \node (base) at (0,0) {};
  \chemHexagoneHaw[atom sep = 28pt] (-32pt,0pt)
  \node at (base) {\chemfig[atom sep = 28pt]{!\glucoseHaw}};
\end{tikzpicture}
\end{boiteCodeTex}
  \section{Lipides}

%%
\subsection{Acide gras}
  
\begin{boiteCodeTex}{}
  \chemfig{!\palmitique} \\[8pt]
  \chemfig{!\linoleique}
  \chemfig{!\linolenique} \\[8pt]
  \chemfig{!\oleique}
  \chemfig{!\arachidonique} \\[8pt]
  \chemfig{!\eicosaPentaenoique}
  \chemfig{!\docosaHexanoique}
\end{boiteCodeTex}

\begin{boiteCodeTex}{}
  \chemfig{!\steraiqueSemiDev}
  \chemfig{!\oleiqueSemiDev}
  \chemfig{!\oleateSemiDev}
  \chemfig{!\caproiqueSemiDev}
\end{boiteCodeTex}
  
%%  
\subsection{Triglycérides et phospholipides}

\begin{boiteCodeTex}{}
  \chemfig{!\palmitine} \\
  \chemfig[atom sep = 14pt]{[:60]!\oleine}
  \chemfig[atom sep = 14pt]{!\arachidonine}
\end{boiteCodeTex}
  
\begin{boiteCodeTex}{}
  \chemfig{!\oleineSemiDev}
  \chemfig{!\palmitineSemiDev}
  \chemfig{!\caproineSemiDev}
\end{boiteCodeTex}

\begin{boiteCodeTex}{}
  \chemfig{!\phosphatidylcholine}
\end{boiteCodeTex}

%%
\subsection{glycérol et stérols}

\begin{boiteCodeTex}{}
  \chemfig{!\glycerol} \qq{}
  \chemfig{!\glycerolSemiDev}
\end{boiteCodeTex}
  
\begin{boiteCodeTex}{}
  \chemfig{!\cholesterol}
\end{boiteCodeTex}

%%
\subsection{Sous-molécules utiles}
  
\subsubsection{Pour les chaînes dans les triglycérides}

\begin{boiteCodeTex}{}
  \chemfig{[:-30] !\tricaproique}
  \chemfig{[:-30] !\trilaurique} \\
  \chemfig{[:-30] !\tripalmitique}
  \chemfig{[:-30] !\trioleique} \\
  \chemfig{[:-30] !\trilinoleique}
  \chemfig{[:-30] !\trilinolenique} \\
  \chemfig{[:-30] !\trieicosapenta}
  \chemfig{[:-30] !\triarachidonique}
  \chemfig{[:-30] !\tridocosahexa}
\end{boiteCodeTex}
  
\subsubsection{Pour les triglycérides}

\begin{boiteCodeTex}{}
  \chemfig[atom sep = 18pt]{A-[:30] !\glycero{!\lh B} !\lb C }
  \chemfig[atom sep = 18pt]{[:60] !\triester{A}{B}{C}}
  \chemfig[atom sep = 18pt]{!\triesterSat{A}{B}C} \\
  \chemfig[atom sep = 14pt]{!\triester {!\trioleique} {!\tricaproique} {!\trilinolenique}}
  \chemfig[atom sep = 14pt]{!\triesterSat {!\lb !\trioleique} {!\tripalmitique} !\lb !\trilaurique}
\end{boiteCodeTex}

  \section{Glucides}

%%
\subsection{Amidon}

\begin{boiteCodeTex}{}
  \chemfig{!\amylopectineHaw}
\end{boiteCodeTex}

%%
\subsection{Glucose et fructose}

\begin{boiteCodeTex}{}
  \chemfig{!\glucoseHaw}
  \chemfig{!\glucoseCycle} \\
  \chemfig{!\glucose} \\[8pt]
  \chemfig{!\glucoseSemiDev}
\end{boiteCodeTex}

\begin{boiteCodeTex}{}
  \chemfig{!\fructoseHaw}
  \chemfig{!\fructofuranoseHaw}
  \chemfig{!\fructoseCycle} \\
  \chemfig{!\fructose} \\[8pt]
  \chemfig{!\fructoseSemiDev}
\end{boiteCodeTex}

%%
\subsection{Galactose et saccharose}

\begin{boiteCodeTex}{}
  \chemfig{!\galactoseHaw}
  \chemfig{!\saccharoseHaw}
\end{boiteCodeTex}

%%
\subsection{Ribose et desoxyribose}

\begin{boiteCodeTex}{}
  \chemfig{A !\ribose B}
  \chemfig{A !\desoxyribose B}
  \chemfig{A !\riboseHaw B}
  \chemfig{A !\desoxyriboseHaw B}
\end{boiteCodeTex}

  \section{Acides alpha aminés et protéines}

%%
\subsection{Formules topologiques}

\begin{boiteCodeTex}{}
  \chemfig{!\arginine}
  \chemfig{!\histidine}
  \chemfig{!\lysine}
  \chemfig{!\aspartique}
\end{boiteCodeTex}
  
\begin{boiteCodeTex}{}
  \chemfig{!\glutamique}
  \chemfig{!\serine}
  \chemfig{!\threonine}
  \chemfig{!\asparagine}
\end{boiteCodeTex}
  
\begin{boiteCodeTex}{}
  \chemfig{!\glutamine}
  \chemfig{!\cysteine}
  \chemfig{!\selenocysteine}
  \chemfig{!\glycine}
\end{boiteCodeTex}
  
\begin{boiteCodeTex}{}
  \chemfig{!\proline}
  \chemfig{!\alanine}
  \chemfig{!\valine}
  \chemfig{!\isoleucine}
  \chemfig{!\leucine}
\end{boiteCodeTex}
  
\begin{boiteCodeTex}{}
  \chemfig{!\methionine}
  \chemfig{!\phenylalanine}
  \chemfig{!\tyrosine}
  \chemfig{!\tryptophane}
\end{boiteCodeTex}

%%
\subsection{Formules semi-développées, représentation de Fischer et de Cram}

\begin{boiteCodeTex}{}
  \chemfig{!\alanineSemiDev} \qq{}
  \chemfig{!\asparagineSemiDev} \qq{}
  \chemfig{!\glycineSemiDev} \\[8pt]
  \chemfig{!\cysteineSemiDev} \\[8pt]
\end{boiteCodeTex}

\begin{boiteCodeTex}{}
  \chemfig{!\alanineL} \quad
  \chemfig{!\alanineD} \quad
  \chemfig{!\valineL} \quad
  \chemfig{!\valineD}
\end{boiteCodeTex}

%%
\subsection{Polypeptides et groupements prosthétiques}

\begin{boiteCodeTex}{}
  \chemfig{ [:-30] H_2N !\alanineH !\HN !\glycineB !\NH !\cysteineH !\HN !\isoleucineB !\NH !\valineH OH }
\end{boiteCodeTex}

\begin{boiteCodeTex}{}
  \chemfig[atom sep = 18pt]{!\hemeB}
\end{boiteCodeTex}

  \section{Vitamines}

\subsection{Vitamines B et C}

\begin{boiteCodeTex}{}
  \chemfig{!\thiamine}               % B1
  \chemfig{!\riboflavine} \\         % B2
  \chemfig{!\nicotinamide} \qq{}     % B3
  \chemfig{!\acideNicotinique} \qq{} % B3
  \chemfig{!\acidePantothenique}     % B5
\end{boiteCodeTex}{}

\begin{boiteCodeTex}{}
  \chemfig{!\pyroxidine}   % B6
  \chemfig{!\biotine} \\   % B8
  \chemfig[atom sep = 18pt]{!\acideFolique} % B9
\end{boiteCodeTex}

\begin{boiteCodeTex}{}
  \chemfig[atom sep = 18pt]{!\cyanocobalamine} % B12
\end{boiteCodeTex}

\begin{boiteCodeTex}{}
  \chemfig{!\acideAscorbique} % C
\end{boiteCodeTex}

\subsection{Vitamines A, D, E, K$_1$ et K$_2$}

\begin{boiteCodeTex}{}
  \chemfig[atom sep = 18pt]{!\retinol} \\      % A
  \chemfig[atom sep = 18pt]{!\cholecarciferol} % D
\end{boiteCodeTex}
  
\begin{boiteCodeTex}{}
  \chemfig[atom sep = 18pt]{!\tocopherol} \\[8pt] % E
  \chemfig[atom sep = 18pt]{!\tocotrienol}        % E
\end{boiteCodeTex}
  
\begin{boiteCodeTex}{}
  \chemfig[atom sep = 18pt]{!\phylloquinone} \\[8pt] % K1
  \chemfig[atom sep = 18pt]{!\menatetrenone}         % K2
\end{boiteCodeTex}

  \section{Hormones}

\begin{boiteCodeTex}{}
  \chemfig{!\creatinine}
  \chemfig{!\DOPA}
  \chemfig{!\DOPAH} \\[8pt]
  \chemfig{!\prostaglandine}
\end{boiteCodeTex}

%%
\subsection{Corticoïdes et minéralocorticoïdes}

\begin{boiteCodeTex}{}
  \chemfig{!\cortisol} \hspace*{-50pt}
  \chemfig{!\corticosterone} \hspace*{-64pt}
  \chemfig{!\aldosterone}
\end{boiteCodeTex}

%%
\subsection{Oestrogènes}

\begin{boiteCodeTex}{}
  \chemfig{!\estrone} \hspace*{-40pt}
  \chemfig{!\estriol} \hspace*{-56pt}
  \chemfig{!\estradiol}
\end{boiteCodeTex}

%%
\subsection{Androgènes}

\begin{boiteCodeTex}{}
  \chemfig{!\testosterone} \hspace*{-12pt}
  \chemfig{!\dihydrotestosterone} \hspace*{-12pt}
  \chemfig{!\androstenedione}
\end{boiteCodeTex}

\begin{boiteCodeTex}{}
  \chemfig{!\DHEA}
  \chemfig{!\DHEAS}
\end{boiteCodeTex}

%%
\subsection{Progestatives}

\begin{boiteCodeTex}{}
  \chemfig{!\progesterone}
\end{boiteCodeTex}

  \section{Nucléotides}

\subsection{Bases nucléiques}

\begin{boiteCodeTex}{}
  \chemfig{A- !\adenine} \hspace*{-20pt}
  \chemfig{A- !\cytosine}
  \chemfig{A- !\guanine} \hspace*{-20pt}
  \chemfig{A- !\thymine} 
  \chemfig{A- !\uracile} 
\end{boiteCodeTex}

%%
\subsection{Ribonucléosides et désoxyribonucléosides}

\begin{boiteCodeTex}{}
  \chemfig{!\adenosine}
  \chemfig{!\cytidine} 
  \chemfig{!\guanosine} \\[8pt]
  \chemfig{!\thymidine}
  \chemfig{!\uridine}  
\end{boiteCodeTex}

\begin{boiteCodeTex}{}
  \chemfig{!\adenosineHaw}
  \chemfig{!\cytidineHaw} 
  \chemfig{!\guanosineHaw} \\[8pt]
  \chemfig{!\thymidineHaw}
  \chemfig{!\uridineHaw}  
\end{boiteCodeTex}

\begin{boiteCodeTex}{}
  \chemfig{!\desoxyAdenosineHaw}
  \chemfig{!\desoxyCytidineHaw} 
  \chemfig{!\desoxyGuanosineHaw} \\[8pt]
  \chemfig{!\desoxyThymidineHaw}
  \chemfig{!\desoxyUridineHaw}  
\end{boiteCodeTex}

%%
\subsection{Adénosine triphosphate et diphosphate}
\begin{boiteCodeTex}{}
  \chemfig{!\ADP}
  \chemfig{!\ATP}
\end{boiteCodeTex}

\begin{boiteCodeTex}{}
  \chemfig{!\ADPHaw}
  \chemfig{!\ATPHaw}
\end{boiteCodeTex}

  \section{Médicaments et produits de synthèse}

\subsection{Aspirine}

\begin{boiteCodeTex}{}
  \chemfig{!\aspirineSemiDev}
  \chemfig{!\aspirine} \qq{}
  \chemfig{!\acideSalicylique}
\end{boiteCodeTex}
  
\subsection{Paracétamol}

\begin{boiteCodeTex}{}
  \chemfig{!\paracetamol}
  \chemfig{!\paracetamolSemiDev}
  \chemfig{!\paracetamolDev}
\end{boiteCodeTex}

\subsection{Aspartame}

\begin{boiteCodeTex}{}
  \chemfig{!\aspartame}
  \chemfig{[:-30]H_2N !\aspartiqueH !\NH !\phenylalanineB OH}
\end{boiteCodeTex}
  
\subsection{Divers}

\begin{boiteCodeTex}{}
  \chemfig{!\bisphenolA} \qq{}
  \chemfig{!\bisphenolASemiDev}
\end{boiteCodeTex}

  \section{Molécules odorantes}

\begin{boiteCodeTex}{}
  \chemfig{!\geraniol} \quad
  \chemfig{!\geraniolSemiDev} \quad
  \chemfig{!\vanilline} \quad
  \chemfig{!\ethylvanilline}
\end{boiteCodeTex}

\begin{boiteCodeTex}{}
  \chemfig{!\oxyphenylone} \quad
  \chemfig{!\limonene} \quad
  \chemfig{!\limoneneSemiDev} \quad
  \chemfig{!\acetateIsoamyle}
\end{boiteCodeTex}

  \input{biomolecules-doc-divers}
\end{document}
