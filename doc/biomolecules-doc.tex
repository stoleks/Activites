\documentclass[12pt]{extarticle}

%%%% paramètres généraux et commandes prédéfinies
\usepackage[french]{babel}
\usepackage[T1]{fontenc}
\usepackage[utf8]{inputenc}

\usepackage{lmodern} % pour étirer fontawesome
\usepackage{geometry} % définition des marges
\usepackage{profSciences} % mise en page et autre
\usepackage{biomolecules} % pour dessiner des formules chimiques

%%%% quelque couleurs maison
\definecolor{vertHerbe}   {RGB} {124, 179,  66}
\definecolor{vertSapin}   {RGB} {  0,  95,  17}
\definecolor{vertSombre}  {RGB} { 14,  84,  60}
\definecolor{cyan}        {RGB} {  0, 140, 128}
\definecolor{cyanSombre}  {RGB} {  0,  98, 116}
\definecolor{bleuPale}    {RGB} { 39,  76, 167}
\definecolor{jauneClair}  {RGB} {218, 173,   0}
\definecolor{jauneSombre} {RGB} {213, 145,   2}
\definecolor{orangeSombre}{RGB} {174,  82,   0}
\definecolor{rougeClair}  {RGB} {224,  59,  54}
\definecolor{rougeSombre} {RGB} {148,  31,   0}


%%%% Couleurs flexoki : https://stephango.com/flexoki
\definecolor{red600}    {HTML} {AF3029}
\definecolor{orange600} {HTML} {BC5215}
\definecolor{yellow600} {HTML} {AD8301}
\definecolor{green600}  {HTML} {66800B}
\definecolor{cyan600}   {HTML} {24837B}
\definecolor{blue600}   {HTML} {205EA6}
\definecolor{purple600} {HTML} {5E409D}
\definecolor{magenta600}{HTML} {A02F6F}
%%%%%%%%%%%%%%%%%%%%%%%%%%%%%%%%%%%%%%%%%%%%%%%%%%%%%%%%%%%%%
%% grandeurs récurrentes
% Physique
\newcommand{\ISS}{\text{ISS}}
\newcommand{\Terre}{\text{Terre}}
\newcommand{\inertie}{\text{inertie}}
\newcommand{\Tfus}{T_\text{f}}
\newcommand{\Teb}{T_\text{éb}}
% Chimie
\newcommand{\solute}{\text{soluté}}
\newcommand{\solution}{\text{solution}}
\newcommand{\espece}{\text{espèce}}
\newcommand{\avogadro}{\num{6,02e23}}
% ions
\newcommand{\ionFerII}      {Fer II      \chemfig{Fe^{2+}}   }
\newcommand{\ionFerIII}     {Fer III     \chemfig{Fe^{3+}}   }
\newcommand{\ionSodium}     {Sodium      \chemfig{Na^{+}}    }
\newcommand{\ionCuivreII}   {Cuivre II   \chemfig{Cu^{2+}}   }
\newcommand{\ionCalcium}    {Calcium     \chemfig{Ca^{2+}}   }
\newcommand{\ionSulfate}    {Sulfate     \chemfig{SO_4^{2-}} }
\newcommand{\ionNitrate}    {Nitrate     \chemfig{NO_3^{-}}  }
\newcommand{\ionChlorure}   {Chlorure    \chemfig{Cl^{-}}    }
\newcommand{\ionFluorure}   {Fluorure    \chemfig{F^{-}}     }
\newcommand{\ionMagnesium}  {Magnésium   \chemfig{Mg^{2+}}   }
\newcommand{\ionPotassium}  {Potassium   \chemfig{K^{+}}     }
\newcommand{\ionBicarbonate}{Bicarbonate \chemfig{CO_3^{2-}} }

%% vecteurs
\newcommand{\FBsurA}{F_{B/A}}
\newcommand{\FAsurB}{F_{A/B}}
\newcommand{\vvFAsurB}{\vv{F}_{A/B}}
\newcommand{\vvFBsurA}{\vv{F}_{B/A}}
%%%%%%%%%%%%%%%%%%%%%%%%%%%%%%%%%%%%%%%%%%%%%%%%%%%%%%%%%%%%%
%%%% figures simples
\newcommand{\tkzRect}[4]{
  \fill[color=#1] (#2,#4) -- (-#2,#4) -- (-#2,#3) -- (#2,#3);
}
\newcommand{\tkzEllipse}[4]{
  \fill[color=#1] (0,#3) ellipse (#2 and #4);
}

% \tkzCercle {x}{y} {couleur} {rayon}
\newcommand{\tkzCercle}[4]{
  \filldraw [#3] (#1, #2) circle (#4pt);
}
% \tkzCercleLigne {x}{y} {couleurFond}{couleurTrait} {rayon}
\newcommand{\tkzCercleLigne}[5]{
  \filldraw [color = #4, fill = #3, very thick] (#1, #2) circle (#5pt);
}

%%%% tube à essais
\newcommand{\tkzTubeEssais}[3]{
  \draw[thick] (#1,#2) -- (#1,0) arc (0:-180:#1) -- (-#1,#2);
  \draw[thick] (0,#2) ellipse (#1 and #3);
}
\newcommand{\tkzBasTubeEssais}[5]{
  \fill[color=#1] (-#2,#3) -- (#2,#3) arc (0:-180:#2);
  \tkzRect{#1}{#2}{#3 - 0.01}{#4}
  \tkzEllipse{#1!85!black}{#2}{#4}{#5}
}
\newcommand{\tkzPhaseTubeEssais}[5]{
  \tkzRect{#1}{#2}{#3}{#4}
  \tkzEllipse{#1}{#2}{#3}{#5}
  \tkzEllipse{#1!85!black}{#2}{#4}{#5}
}

%%%% Point et vecteurs
\newcommand{\tkzLabel}[3]{
  \node at (#1, #2) {#3};
}
\newcommand{\tkzPointLabel}[3]{
  \filldraw (#1, #2) circle (2pt) node[above] {#3};
}
% \tkzVecteur [couleur] (x) [longueur x] (y) [longueur y] {legende} [position legende] 
% ajouter une * à la fin transforme la flèche en double flèche <->
\NewDocumentCommand{\tkzVecteur}{O{black} r() O{0} r() O{0} m O{right} s}{
  \IfBooleanTF{#8}{
    \draw[#1, <->, very thick] (#2, #4) -- (#2 + #3, #4 + #5) node[#7] {#6};
  }{
    \draw[#1, ->, very thick] (#2, #4) -- (#2 + #3, #4 + #5) node[#7] {#6};
  }
}
% \tkzLegende (x) (y) [longueur fleche] {légende} 
% ajouter une * passe de la version gauche -> à la version droite <-
\NewDocumentCommand{\tkzLegende}{O{black} r() r() O{1.25} m s}{
  \IfBooleanTF{#6}{
    \draw[#1, ->, very thick] (#2 + #4, #3) node[right] {#5} -- (#2, #3);
  }{
    \draw[#1, ->, very thick] (#2, #3) node[left] {#5} -- (#2 + #4, #3);
  }
}

\newcommand{\barrePourcentage}[1]{%
  \begin{tikzpicture}
    \fill[color=couleurSec]    (0.0,    0.0) rectangle (#1*8ex, 1.5ex);
    \fill[color=couleurSec!20] (#1*8ex, 0.0) rectangle (8.0ex,  1.5ex);
  \end{tikzpicture}
}

\newcommand{\flecheProgression}[1]{%
  \begin{center}
    \begin{tikzpicture}
      \draw[
        -{Triangle [width = 36pt, length = 16pt]}, 
        line width = 20pt,
        rounded corners = 10mm,
        color = couleurSec,
      ]
      #1
    \end{tikzpicture}
  \end{center}
}


%%%%%%%%%%%%%%%%%%%%%%%%%%%%%%%%%%%%%%%%%%%%%%%%%%%%%%%%%%%%%
%%%% plan de classe
\NewDocumentCommand{\texteCadre}{O{black} r() O{2} r() O{2} m}{
  \filldraw [fill=white, draw=#1, ultra thick] (#2, #4) rectangle (#2 + #3, #4 + #5);
  \node at (#2 + #3/2, #4 + #5/2) [font=\sffamily] {\textbf{#6}};
}

%% place dans la classe
\NewDocumentCommand{\place}{r() r() m}{
  \texteCadre(#1)[3](#2)[2] {#3}
}
\NewDocumentCommand{\places}{r()r() r[] d[] d[] d[]}{
  \place(#1)(#2) {#3}
  \IfValueT{#4}{ \place(#1 + 1*3)(#2) {#4} }
  \IfValueT{#5}{ \place(#1 + 2*3)(#2) {#5} }
  \IfValueT{#6}{ \place(#1 + 3*3)(#2) {#6} }
}

%% rangée de classe ou de TP
\NewDocumentCommand{\rangee}{m r()r() r()r()r()d() r()r()}{
  \places(0)(0 - 3*#1) [#2][#3]
  \IfValueTF{#7}{
    \places(7) (0 - 3*#1) [#4][#5][#6][#7]
    \places(20)(0 - 3*#1) [#8][#9]
  }{
    \places(8.5)(0 - 3*#1) [#4][#5][#6]
    \places(20) (0 - 3*#1) [#8][#9]
  }
}

\NewDocumentCommand{\rangeeTP}{m r[]r[]r[] r()r()r()d()}{
  \places(3)(0 - 3*#1) [#2][#3][#4]
  \IfValueTF{#8}{
    \places(14) (0 - 3*#1) [#5][#6][#7][#8]
  }{
    \places(14) (0 - 3*#1) [#5][#6][#7]
  }
}


%%%% tube à essai de sang
\newcommand{\tubeEssaisSolution}[1]{
  \begin{tikzpicture}
    \tkzBasTubeEssais{#1}{0.25}{0}{0.75}{0.1} % contenu du tube
    \tkzTubeEssais{0.25}{1.5}{0.1} % tube
  \end{tikzpicture}
}

\newcommand{\tubeEssaisSangCentrifuge}[3]{
  \begin{tikzpicture}
    % phases dans le tube à essai
    \tkzBasTubeEssais{rougeSombre!75!white} {0.35}{0}{#1}{0.1}
    \tkzPhaseTubeEssais{gray!10!white}      {0.35}{#1}{#2}{0.1}
    \tkzPhaseTubeEssais{jauneClair!75!white}{0.35}{#2}{#3}{0.1}
    \tkzTubeEssais{0.35}{#3 + 1}{0.1}
    % Légende
    \tkzLegende(0.4)(#3 - 0.1) [1]{Plasma}*
    \tkzLegende(0.4)(#2 - 0.08)[1]{Globules blancs}*
    \tkzLegende(0.4)(-0.1)     [1]{Globules rouges}*
  \end{tikzpicture}
}
%%%% Ce fichier sert à déclarer les titres des chapitres des différents niveaux

%% Commun
\newcommand{\methode} {\chapitre{Outils pratiques}}

%% Seconde
%%%% Chapitre
\newcommand{\snd}{Seconde}
\newcommand{\sndCorp} {\chapitre{Corps purs et mélanges}}
\newcommand{\sndSolu} {\chapitre{Solutions}}
\newcommand{\sndMouv} {\chapitre{Mouvement et interactions}}
\newcommand{\sndAtom} {\chapitre{Structure de l'atome}}
\newcommand{\sndMole} {\chapitre{Des atomes à la matière}}
\newcommand{\sndLumi} {\chapitre{Ondes lumineuses et optique}}
\newcommand{\sndTran} {\chapitre{Transformations de la matière}}
\newcommand{\sndChim} {\chapitre{Transformations chimiques}}
\newcommand{\sndSign} {\chapitre{Signaux et capteurs}}

%%%% en-tête correspondant
\newcommand{\teteSndMeth} {\enTete[\snd]{\methode}}
\newcommand{\teteSndCorp} {\enTete[\snd]{\sndCorp}[1]}
\newcommand{\teteSndSolu} {\enTete[\snd]{\sndSolu}[2]}
\newcommand{\teteSndMouv} {\enTete[\snd]{\sndMouv}[3]}
\newcommand{\teteSndAtom} {\enTete[\snd]{\sndAtom}[4]}
\newcommand{\teteSndMole} {\enTete[\snd]{\sndMole}[5]}
\newcommand{\teteSndLumi} {\enTete[\snd]{\sndLumi}[6]}
\newcommand{\teteSndTran} {\enTete[\snd]{\sndTran}[7]}
\newcommand{\teteSndChim} {\enTete[\snd]{\sndChim}[8]}
\newcommand{\teteSndSign} {\enTete[\snd]{\sndSign}[9]}


%% Première ST2S
%%%% Chapitres
\newcommand{\premStss}{Première ST2S}
\newcommand{\premStssChim} {\chapitre{Sécurité chimique dans l'habitat}}
\newcommand{\premStssVisi} {\chapitre{Propagation de la lumière et vision}}
\newcommand{\premStssRedo} {\chapitre{Antiseptique et désinfectant, oxydoréduction}}
\newcommand{\premStssLumi} {\chapitre{Les infrarouges et leurs applications}}
\newcommand{\premStssStru} {\chapitre{Molécules d'intérêt biologique}}
\newcommand{\premStssBiom} {\chapitre{Biomolécules dans l’organisme}}
\newcommand{\premStssRout} {\chapitre{Sécurité routière}}
\newcommand{\premStssAlim} {\chapitre{Gestion des ressources naturelles et alimentation}}
\newcommand{\premStssElec} {\chapitre{Sécurité électrique dans l'habitat}}
\newcommand{\premStssPres} {\chapitre{Propriétés des fluides et pression sanguine}}
\newcommand{\premStssSono} {\chapitre{Ondes sonores et audition}}

%%%% en-tête
\newcommand{\tetePremStssMeth} {\enTete[\premStss]{\methode}     }
\newcommand{\tetePremStssChim} {\enTete[\premStss]{\premStssChim}[1]}
\newcommand{\tetePremStssVisi} {\enTete[\premStss]{\premStssVisi}[2]}
\newcommand{\tetePremStssRedo} {\enTete[\premStss]{\premStssRedo}[3]}
\newcommand{\tetePremStssLumi} {\enTete[\premStss]{\premStssLumi}[4]}
\newcommand{\tetePremStssStru} {\enTete[\premStss]{\premStssStru}[5]}
\newcommand{\tetePremStssBiom} {\enTete[\premStss]{\premStssBiom}[6]}
\newcommand{\tetePremStssRout} {\enTete[\premStss]{\premStssRout}[7]}
\newcommand{\tetePremStssAlim} {\enTete[\premStss]{\premStssAlim}[8]}
\newcommand{\tetePremStssElec} {\enTete[\premStss]{\premStssElec}[9]}
\newcommand{\tetePremStssPres} {\enTete[\premStss]{\premStssPres}[10]}
\newcommand{\tetePremStssSono} {\enTete[\premStss]{\premStssSono}[11]}


%% Terminale ST2S
%%%% Chapitres
\newcommand{\termStss}{Terminale ST2S}
\newcommand{\termStssOrga} {\chapitre{Représentation des molécules organiques}}
\newcommand{\termStssAlim} {\chapitre{Sécurité physico-chimique dans l'alimentation}}
\newcommand{\termStssImag} {\chapitre{La physique de l'imagerie médicale}}
\newcommand{\termStssBiom} {\chapitre{Biomolécules et alimentation}}
\newcommand{\termStssMedi} {\chapitre{De la molécule au médicament}}
\newcommand{\termStssEnvi} {\chapitre{Sécurité chimique dans l'environnement}}
\newcommand{\termStssDosa} {\chapitre{Analyser la composition d'un milieu}}
\newcommand{\termStssRout} {\chapitre{Sécurité routière}}
\newcommand{\termStssCosm} {\chapitre{L'usage responsable des cosmétiques}}

%%%% en-tête
\newcommand{\teteTermStssMeth} {\enTete[\termStss]{\methode}}
\newcommand{\teteTermStssOrga} {\enTete[\termStss]{\termStssOrga}[1]}
\newcommand{\teteTermStssRout} {\enTete[\termStss]{\termStssRout}[8]}
\newcommand{\teteTermStssAlim} {\enTete[\termStss]{\termStssAlim}[2]}
\newcommand{\teteTermStssEnvi} {\enTete[\termStss]{\termStssEnvi}[6]}
\newcommand{\teteTermStssImag} {\enTete[\termStss]{\termStssImag}[3]}
\newcommand{\teteTermStssDosa} {\enTete[\termStss]{\termStssDosa}[7]}
\newcommand{\teteTermStssBiom} {\enTete[\termStss]{\termStssBiom}[4]}
\newcommand{\teteTermStssMedi} {\enTete[\termStss]{\termStssMedi}[5]}
\newcommand{\teteTermStssCosm} {\enTete[\termStss]{\termStssCosm}[9]}

\newcommand{\largeurCaseTableauPeriodique}{1.5}

%%%% Pour afficher un élément dans le tableau périodique
\NewDocumentCommand{\elementTexteCharge}{m m m o}
{
  \begin{minipage}{\largeurCaseTableauPeriodique cm}
    \begin{center}
      \IfValueTF{#4}{ \vAligne{-20pt} }{ \vAligne{-34pt} } % position du nom
      {\small #3} \\[2pt] % nom de l'élément
      {\ensuremath\footnotesize \textbf{#1}} \\[6pt] % nombre atomique
      \chemfig[atom style={scale = 1.8}]{#2} % symbole atomique
      % \element{#1}{#2} % element symbol and atomic number
      \IfValueT{#4}{
        \\ {\small \qty{#4}{\g/\mole}}
      }
    \end{center}
  \end{minipage}
}

%%%% Pour afficher un élément dans le tableau périodique
\NewDocumentCommand{\elementElectroneg}{m m}
{
  \begin{minipage}{\largeurCaseTableauPeriodique cm}
    \begin{center}
      {\Large \important[black]{#1} \\[2pt]} % symbole atomique
      {\small $\chi = \num{#2}$} % électronégativité
    \end{center}
  \end{minipage}
}


%%%% Pour afficher un tableau périodique
%% #1 : largeur ; #2 : hauteur ; #3 : élements
\NewDocumentCommand{\tableauPeriodique}{O{2.6} O{2.7} m}{
\begin{tikzpicture}[font=\sffamily, scale=0.75, transform shape]

%% Type d'élément, par famille
  \tikzstyle{Alcali} = [Element, fill=green-200]
  \tikzstyle{Alcalo} = [Element, fill=green-150]
  \tikzstyle{Metaux} = [Element, fill=green-100]
  \tikzstyle{Metoid} = [Element, fill=orange-100]
  \tikzstyle{NoMeta} = [Element, fill=orange-150]
  \tikzstyle{Haloge} = [Element, fill=orange-200]
  \tikzstyle{GazRar} = [Element, fill=red-150]

%% Type d'élément, par électronégativité
 \tikzstyle{elec1} = [Element, fill=green-50]
 \tikzstyle{elec2} = [Element, fill=green-100!80]
 \tikzstyle{elec3} = [Element, fill=yellow-100]
 \tikzstyle{elec4} = [Element, fill=orange-100]
 \tikzstyle{elec5} = [Element, fill=orange-150]
 \tikzstyle{elec6} = [Element, fill=orange-200]
 \tikzstyle{elec7} = [Element, fill=red-200]
 \tikzstyle{elec8} = [Element, fill=red-300]
  
%% Style des éléments
  \tikzstyle{Element} = [
    draw=black, cyan-800!50!black,
    minimum width  = #1 cm, % Largeur de la case
    node distance  = #1 cm, % Espace entre deux case
    minimum height = #2 cm, % Hauteur de la case
  ]

%% Période, groupe et titre
  \tikzstyle{Period} = [font={\sffamily\LARGE}, node distance=2cm]
  \tikzstyle{Groupe} = [font={\sffamily\LARGE}, minimum width=2.5cm, node distance=2cm]
  \tikzstyle{Titre}  = [font={\sffamily\Huge\bfseries}]

%% Place des éléments
  #3
\end{tikzpicture}
}


%%%% Pour faciliter l'utilisation du tableau périodique
\newcommand{\elementH} {\elementTexteCharge{1} {H} {Hydrogène}[1,00]}
\newcommand{\elementHe}{\elementTexteCharge{2} {He}{Hélium}   [4,00]}
\newcommand{\elementLi}{\elementTexteCharge{3} {Li}{Lithium}  [6,94]}
\newcommand{\elementBe}{\elementTexteCharge{4} {Be}{Béryllium}[9,01]}
\newcommand{\elementB} {\elementTexteCharge{5} {B} {Bore}     [10,8]}
\newcommand{\elementC} {\elementTexteCharge{6} {C} {Carbone}  [12,0]}
\newcommand{\elementN} {\elementTexteCharge{7} {N} {Azote}    [14,0]}
\newcommand{\elementO} {\elementTexteCharge{8} {O} {Oxygène}  [16,0]}
\newcommand{\elementF} {\elementTexteCharge{9} {F} {Fluor}    [19,0]}
\newcommand{\elementNe}{\elementTexteCharge{10}{Ne}{Néon}     [20,2]}
\newcommand{\elementNa}{\elementTexteCharge{11}{Na}{Sodium}   [23,0]}
\newcommand{\elementMg}{\elementTexteCharge{12}{Mg}{Magnésium}[24,3]}
\newcommand{\elementAl}{\elementTexteCharge{13}{Al}{Aluminium}[27,0]}
\newcommand{\elementSi}{\elementTexteCharge{14}{Si}{Silicium} [28,1]}
\newcommand{\elementP} {\elementTexteCharge{15}{P} {Phosphore}[31,0]}
\newcommand{\elementS} {\elementTexteCharge{16}{S} {Soufre}   [32,1]}
\newcommand{\elementCl}{\elementTexteCharge{17}{Cl}{Chlore}   [35,5]}
\newcommand{\elementAr}{\elementTexteCharge{18}{Ar}{Argon}    [39,9]}
\newcommand{\elementK} {\elementTexteCharge{19}{K} {Potassium}[39,1]}
\newcommand{\elementCa}{\elementTexteCharge{20}{Ca}{Calcium}  [40,0]}

\palette{couleurPrim}{cyan}
\palette{couleurSec} {blue}
\palette{couleurTer} {purple}
\palette{couleurQuat}{red}

\setlength{\parskip}{0cm}
\setlength{\parindent}{0cm}
\renewcommand{\baselinestretch}{1}
% réglage du niveau (sous-section) ou s'arrête la table des matières
\setcounter{tocdepth}{2}

\geometry{
  a4paper, % format
  left=1.3cm, right=1.3cm, % marge horizontale
  top=2.2cm, bottom=2.1cm % marge verticale
}

%% Ces commandes sont adaptées du très bon paquet ProfLycée de Cedric Pierquet 
%% https://ctan.org/pkg/proflycee
\tcbset{
  style code Tex/.style = {%
    listing engine = listings,%
    listing options = {%
      breaklines = true,%
      breakatwhitespace = true,%
      style = tcblatex, basicstyle = \footnotesize\ttfamily,%
      tabsize = 2,%
      commentstyle = {\itshape\color{couleurSec-300}},
      keywordstyle = {\color{couleurSec}},%
      classoffset = 0,%
      keywords = {chemfig, definesubmol},%
      alsoletter = {-},%
      keywordstyle = {\color{couleurSec}}%
    }
  },
  cote a cote/.style = {%
    listing side text,%
    righthand width = #1%
  }
}

%% de Cedric Pierquet https://ctan.org/pkg/proflycee
\NewTCBListing{boiteCodeTex}{ O{couleurTer} m }{%
  enhanced, breakable,%
  flush right, boxrule = 1pt, colframe = #1!90,%
  sharp corners, top = 0mm, bottom = 0mm, left = 0.4em, right = 5mm,%
  before skip = \baselineskip, after skip = \baselineskip,%
  colback = white,%
  fontupper = \footnotesize, fontlower = \footnotesize,%
  title = {{\scriptsize\faCogs} Code \LaTeX},%
  fonttitle = \bfseries\footnotesize\sffamily, colbacktitle = #1,%
  style code Tex,%
  #2,%
}


%%%% doc
\begin{document}
  \titre{Biomolécules}
  \begin{center}
    Quelques commandes pour tracer des biomolécules dans le cadre du lycée.
  \end{center}

  \tableofcontents
  \newpage

  \section{Logique interne}

%%
\subsection{Nom des molécules}

Pour tracer une molécule, il suffit d'appeler \lstinline|\chemfig\{!\nomDeLaMolecule}|.
La représentation de base pour les molécules est la formule topologique, il faut ajouter un suffixe au nom pour passer à une autre représentation \important{si elle est définie, ce qui n'est pas du tout toujours le cas.} Les suffixes sont les suivants :

\begin{listePoints}[2]
  \item \lstinline{SemiDev} : formule semi-développée ;
  \item \lstinline{Dev} : formule développée ;
  \item \lstinline{Haw} : représentation de Haworth ;
  \item \lstinline{Cram} : représentation de Cram.
\end{listePoints}
Pour les acides aminés, il existe quatre autres suffixes
\begin{listePoints}[2]
  \item \lstinline{L} : représentation de Fischer gauche ;
  \item \lstinline{H} : pour tracer un polypeptide, la chaîne latérale est vers le haut ;
  \item \lstinline{D} : représentation de Fischer droite ;
  \item \lstinline{B} : pour tracer un polypeptide, la chaîne latérale est vers le bas.
\end{listePoints}

%%
\subsection{Commandes internes pour faciliter l'écriture}

Pour tracer les formules topologiques, 
j'utilise plusieurs commandes pour éviter d'avoir à spécifier en permanence les angles les plus courants
(\qty{60}{\degree}, \qty{50}{\degree}, etc.),
ou pour réutiliser des morceaux de molécules complexes

\begin{boiteCodeTex}{cote a cote = 2cm}
  \chemfig{-!\vide{::30} -} % Pour tracer une liaison invisible (utile pour les cycles incomplets)

  \chemfig{-!\vide{::-30}-}
\end{boiteCodeTex}
%
\begin{boiteCodeTex}{cote a cote = 2cm}
  \chemfig{-[:30] !\lh} % Pour tracer une liaison vers le haut (liaison haut = lh)

  \chemfig{-[:30] !\lb} % Pour tracer une liaison vers le bas (liaison bas = lb)
\end{boiteCodeTex}
%
\begin{boiteCodeTex}{cote a cote = 2cm}
  \chemfig{-[:30]!\lhb} % Pour tracer une liaison vers le haut puis vers le bas

  \chemfig{-[:30]!\lbh} % Pour tracer une liaison vers le bas puis vers le haut
\end{boiteCodeTex}
%
\begin{boiteCodeTex}{cote a cote = 2cm}
  \chemfig{-[:30]!\llh} % Pour tracer une liaison double vers le haut
  
  \chemfig{-[:30]!\llb} % Pour tracer une liaison double vers le bas
\end{boiteCodeTex}
%
\begin{boiteCodeTex}{cote a cote = 3cm}
  \chemfig{-[:-30]!\cis} % Pour tracer une liaison cis
  
  \chemfig{-[:-30]!\trans} % Pour tracer une liaison "trans" aplatie
\end{boiteCodeTex}
%
\begin{boiteCodeTex}{cote a cote = 2cm}
  \chemfig{-[:30]!\ldh} % Pour tracer une liaison développée vers le haut (l'angle est plus faible)
  
  \chemfig{-[:30]!\ldb} % Pour tracer une liaison développée vers le bas
\end{boiteCodeTex}
%
\begin{boiteCodeTex}{cote a cote = 2cm}
  \chemfig{-[:30]!\lldh} % Pour tracer une liaison double développée vers le haut
  
  \chemfig{-[:30]!\lldb} % Pour tracer une liaison double développée vers le bas
\end{boiteCodeTex}
%
\begin{boiteCodeTex}{cote a cote = 2cm}
  \chemfig[cram width = 5pt]{C !\cram{A}{B} (-[::90] R_1) -[::-30] R_2} % Pour tracer deux liaisons de cram autour d'un élément 
  
  \chemfig{-!\branche{A}{B}-} % Pour tracer deux liaisons à \qty{90}{\degree} autour d'un élément chimique
\end{boiteCodeTex}
%
\begin{boiteCodeTex}{cote a cote = 5.5cm}
  \chemfig{A- !\hexaOseHaw{!\lb B} -C} % Pour tracer des isomères du glucose
  
  \chemfig{A- !\pentaOseHaw{!\lb B}{!\lb C} -D} % Pour tracer des isomères du fructofuranose

  \chemfig{-[:30] 
    !\sterol {-A-} {-B--} {C-D-} 
      {-(-[::0] E)---} {---} {-(-[::0] F)---} 
  } % Pour tracer des stérols
\end{boiteCodeTex}

  \section{Lipides}

%%
\subsection{Acide gras}
  
\begin{boiteCodeTex}{}
  \chemfig{!\palmitique} \\[8pt]
  \chemfig{!\linoleique}
  \chemfig{!\linolenique} \\[8pt]
  \chemfig{!\oleique}
  \chemfig{!\arachidonique} \\[8pt]
  \chemfig{!\eicosaPentaenoique}
  \chemfig{!\docosaHexanoique}
\end{boiteCodeTex}

\begin{boiteCodeTex}{}
  \chemfig{!\steraiqueSemiDev}
  \chemfig{!\oleiqueSemiDev}
  \chemfig{!\oleateSemiDev}
  \chemfig{!\caproiqueSemiDev}
\end{boiteCodeTex}
  
%%  
\subsection{Triglycérides et phospholipides}

\begin{boiteCodeTex}{}
  \chemfig{!\palmitine} \\
  \chemfig[atom sep = 14pt]{[:60]!\oleine}
  \chemfig[atom sep = 14pt]{!\arachidonine}
\end{boiteCodeTex}
  
\begin{boiteCodeTex}{}
  \chemfig{!\oleineSemiDev}
  \chemfig{!\palmitineSemiDev}
  \chemfig{!\caproineSemiDev}
\end{boiteCodeTex}

\begin{boiteCodeTex}{}
  \chemfig{!\phosphatidylcholine}
\end{boiteCodeTex}

%%
\subsection{glycérol et stérols}

\begin{boiteCodeTex}{}
  \chemfig{!\glycerol} \qq{}
  \chemfig{!\glycerolSemiDev}
\end{boiteCodeTex}
  
\begin{boiteCodeTex}{}
  \chemfig{!\cholesterol}
\end{boiteCodeTex}

%%
\subsection{Sous-molécules utiles}
  
\subsubsection{Pour les chaînes dans les triglycérides}

\begin{boiteCodeTex}{}
  \chemfig{[:-30] !\tricaproique}
  \chemfig{[:-30] !\trilaurique} \\
  \chemfig{[:-30] !\tripalmitique}
  \chemfig{[:-30] !\trioleique} \\
  \chemfig{[:-30] !\trilinoleique}
  \chemfig{[:-30] !\trilinolenique} \\
  \chemfig{[:-30] !\trieicosapenta}
  \chemfig{[:-30] !\triarachidonique}
  \chemfig{[:-30] !\tridocosahexa}
\end{boiteCodeTex}
  
\subsubsection{Pour les triglycérides}

\begin{boiteCodeTex}{}
  \chemfig[atom sep = 18pt]{A-[:30] !\glycero{!\lh B} !\lb C }
  \chemfig[atom sep = 18pt]{[:60] !\triester{A}{B}{C}}
  \chemfig[atom sep = 18pt]{!\triesterSat{A}{B}C} \\
  \chemfig[atom sep = 14pt]{!\triester {!\trioleique} {!\tricaproique} {!\trilinolenique}}
  \chemfig[atom sep = 14pt]{!\triesterSat {!\lb !\trioleique} {!\tripalmitique} !\lb !\trilaurique}
\end{boiteCodeTex}

  \section{Glucides}

%%
\subsection{Amidon}

\begin{boiteCodeTex}{}
  \chemfig{!\amylopectineHaw}
\end{boiteCodeTex}

%%
\subsection{Glucose et fructose}

\begin{boiteCodeTex}{}
  \chemfig{!\glucoseHaw}
  \chemfig{!\glucoseCycle} \\
  \chemfig{!\glucose} \\[8pt]
  \chemfig{!\glucoseSemiDev}
\end{boiteCodeTex}

\begin{boiteCodeTex}{}
  \chemfig{!\fructoseHaw}
  \chemfig{!\fructofuranoseHaw}
  \chemfig{!\fructoseCycle} \\
  \chemfig{!\fructose} \\[8pt]
  \chemfig{!\fructoseSemiDev}
\end{boiteCodeTex}

%%
\subsection{Galactose et saccharose}

\begin{boiteCodeTex}{}
  \chemfig{!\galactoseHaw}
  \chemfig{!\saccharoseHaw}
\end{boiteCodeTex}

%%
\subsection{Ribose et desoxyribose}

\begin{boiteCodeTex}{}
  \chemfig{A !\ribose B}
  \chemfig{A !\desoxyribose B}
\end{boiteCodeTex}

\begin{boiteCodeTex}{}
  \chemfig{A !\riboseHaw B}
  \chemfig{A !\desoxyriboseHaw B}
\end{boiteCodeTex}

  \section{Acides alpha aminés et protéines}

%%
\subsection{Formules topologiques}

\begin{boiteCodeTex}{}
  \chemfig{!\arginine}
  \chemfig{!\histidine}
  \chemfig{!\lysine}
  \chemfig{!\aspartique}
\end{boiteCodeTex}
  
\begin{boiteCodeTex}{}
  \chemfig{!\glutamique}
  \chemfig{!\serine}
  \chemfig{!\threonine}
  \chemfig{!\asparagine}
\end{boiteCodeTex}
  
\begin{boiteCodeTex}{}
  \chemfig{!\glutamine}
  \chemfig{!\cysteine}
  \chemfig{!\selenocysteine}
  \chemfig{!\glycine}
\end{boiteCodeTex}
  
\begin{boiteCodeTex}{}
  \chemfig{!\proline}
  \chemfig{!\alanine}
  \chemfig{!\valine}
  \chemfig{!\isoleucine}
  \chemfig{!\leucine}
\end{boiteCodeTex}
  
\begin{boiteCodeTex}{}
  \chemfig{!\methionine}
  \chemfig{!\phenylalanine}
  \chemfig{!\tyrosine}
  \chemfig{!\tryptophane}
\end{boiteCodeTex}

%%
\subsection{Formules semi-développées, représentation de Fischer et de Cram}

\begin{boiteCodeTex}{}
  \chemfig{!\alanineSemiDev} \qq{}
  \chemfig{!\asparagineSemiDev} \qq{}
  \chemfig{!\glycineSemiDev} \\[8pt]
  \chemfig{!\cysteineSemiDev} \\[8pt]
\end{boiteCodeTex}

\begin{boiteCodeTex}{}
  \chemfig{!\alanineL} \quad
  \chemfig{!\alanineD} \quad
  \chemfig{!\valineL} \quad
  \chemfig{!\valineD}
\end{boiteCodeTex}

%%
\subsection{Polypeptides et groupements prosthétiques}

\begin{boiteCodeTex}{}
  \chemfig{ [:-30] H_2N !\alanineH !\HN !\glycineB !\NH !\cysteineH !\HN !\isoleucineB !\NH !\valineH OH }
\end{boiteCodeTex}

\begin{boiteCodeTex}{}
  \chemfig[atom sep = 18pt]{!\hemeB}
\end{boiteCodeTex}

  \section{Vitamines}

\subsection{Vitamines B et C}

\begin{boiteCodeTex}{}
  \chemfig{!\thiamine}               % B1
  \chemfig{!\riboflavine} \\         % B2
  \chemfig{!\nicotinamide} \qq{}     % B3
  \chemfig{!\acideNicotinique} \qq{} % B3
  \chemfig{!\acidePantothenique}     % B5
\end{boiteCodeTex}{}

\begin{boiteCodeTex}{}
  \chemfig{!\pyroxidine}   % B6
  \chemfig{!\biotine} \\   % B8
  \chemfig[atom sep = 18pt]{!\acideFolique} % B9
\end{boiteCodeTex}

\begin{boiteCodeTex}{}
  \chemfig[atom sep = 18pt]{!\cyanocobalamine} % B12
\end{boiteCodeTex}

\begin{boiteCodeTex}{}
  \chemfig{!\acideAscorbique} % C
\end{boiteCodeTex}

\subsection{Vitamines A, D, E, K$_1$ et K$_2$}

\begin{boiteCodeTex}{}
  \chemfig[atom sep = 18pt]{!\retinol} \\      % A
  \chemfig[atom sep = 18pt]{!\cholecarciferol} % D
\end{boiteCodeTex}
  
\begin{boiteCodeTex}{}
  \chemfig[atom sep = 18pt]{!\tocopherol} \\[8pt] % E
  \chemfig[atom sep = 18pt]{!\tocotrienol}        % E
\end{boiteCodeTex}
  
\begin{boiteCodeTex}{}
  \chemfig[atom sep = 18pt]{!\phylloquinone} \\[8pt] % K1
  \chemfig[atom sep = 18pt]{!\menatetrenone}         % K2
\end{boiteCodeTex}

  \section{Hormones}

\begin{boiteCodeTex}{}
  \chemfig{!\creatinine}
  \chemfig{!\DOPA}
  \chemfig{!\DOPAH} \\[8pt]
  \chemfig{!\prostaglandine}
\end{boiteCodeTex}

%%
\subsection{Corticoïdes et minéralocorticoïdes}

\begin{boiteCodeTex}{}
  \chemfig{!\cortisol} \hspace*{-50pt}%
  \chemfig{!\corticosterone} \hspace*{-64pt}%
  \chemfig{!\aldosterone}
\end{boiteCodeTex}

%%
\subsection{Oestrogènes}

\begin{boiteCodeTex}{}
  \chemfig{!\estrone} \hspace*{-40pt}%
  \chemfig{!\estriol} \hspace*{-56pt}%
  \chemfig{!\estradiol}
\end{boiteCodeTex}

%%
\subsection{Androgènes et progestatives}

\begin{boiteCodeTex}{}
  \chemfig{!\testosterone} \hspace*{-12pt}%
  \chemfig{!\dihydrotestosterone} \hspace*{-12pt}%
  \chemfig{!\androstenedione}
\end{boiteCodeTex}

\begin{boiteCodeTex}{}
  \chemfig{!\DHEA} \hspace*{-12pt}%
  \chemfig{!\DHEAS} \hspace*{-12pt}%
  \chemfig{!\progesterone}
\end{boiteCodeTex}

  \section{Nucléotides}

\subsection{Bases nucléiques}

\begin{boiteCodeTex}{}
  \chemfig{A- !\adenine} \hspace*{-20pt}
  \chemfig{A- !\cytosine}
  \chemfig{A- !\guanine} \hspace*{-20pt}
  \chemfig{A- !\thymine} 
  \chemfig{A- !\uracile} 
\end{boiteCodeTex}

%%
\subsection{Ribonucléosides et désoxyribonucléosides}

\begin{boiteCodeTex}{}
  \chemfig{!\adenosine}
  \chemfig{!\cytidine} 
  \chemfig{!\guanosine} \\[8pt]
  \chemfig{!\thymidine}
  \chemfig{!\uridine}  
\end{boiteCodeTex}

\begin{boiteCodeTex}{}
  \chemfig{!\adenosineHaw}
  \chemfig{!\cytidineHaw} 
  \chemfig{!\guanosineHaw} \\[8pt]
  \chemfig{!\thymidineHaw}
  \chemfig{!\uridineHaw}  
\end{boiteCodeTex}

\begin{boiteCodeTex}{}
  \chemfig{!\desoxyAdenosineHaw}
  \chemfig{!\desoxyCytidineHaw} 
  \chemfig{!\desoxyGuanosineHaw} \\[8pt]
  \chemfig{!\desoxyThymidineHaw}
  \chemfig{!\desoxyUridineHaw}  
\end{boiteCodeTex}

%%
\subsection{Adénosine triphosphate et diphosphate}
\begin{boiteCodeTex}{}
  \chemfig{!\ADP}
  \chemfig{!\ATP}
\end{boiteCodeTex}

\begin{boiteCodeTex}{}
  \chemfig{!\ADPHaw}
  \chemfig{!\ATPHaw}
\end{boiteCodeTex}

  \section{Médicaments et produits de synthèse}

\subsection{Antalgiques}

\begin{boiteCodeTex}{}
  \chemfig{!\aspirineSemiDev}
  \chemfig{!\aspirine} \qq{}
  \chemfig{!\acideSalicylique}
\end{boiteCodeTex}

\begin{boiteCodeTex}{}
  \chemfig{!\paracetamol}
  \chemfig{!\paracetamolSemiDev}
  \chemfig{!\paracetamolDev}
\end{boiteCodeTex}

\begin{boiteCodeTex}{}
  \chemfig{!\aspartame}
  \chemfig{[:-30]H_2N !\aspartiqueH !\NH !\phenylalanineB OH}
\end{boiteCodeTex}
  
\subsection{Divers}

\begin{boiteCodeTex}{}
  \chemfig{!\bisphenolA} \qq{}
  \chemfig{!\bisphenolASemiDev}
\end{boiteCodeTex}

  \section{Molécules odorantes}

\begin{boiteCodeTex}{}
  \chemfig{!\geraniol} \quad
  \chemfig{!\geraniolSemiDev} \quad
  \chemfig{!\vanilline} \quad
  \chemfig{!\ethylvanilline}
\end{boiteCodeTex}

\begin{boiteCodeTex}{}
  \chemfig{!\oxyphenylone} \quad
  \chemfig{!\limonene} \quad
  \chemfig{!\limoneneSemiDev} \quad
  \chemfig{!\acetateIsoamyle}
\end{boiteCodeTex}

  \section{Divers}
  
\subsection{Produits de contraste}

\begin{boiteCodeTex}{}
  \chemfig{!\ionChelate}
  \chemfig{!\chelateAlcool}
\end{boiteCodeTex}

\subsection{Drogues}

\begin{boiteCodeTex}{}
  \chemfig{!\THC} \qq{}
  \chemfig{!\cocaineHaw}
\end{boiteCodeTex}

\end{document}
