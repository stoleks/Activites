%%%%
\teteSndAtom

%%%% titre
\vspace*{-44pt}
\numeroActivite{3}
\titreActivite{L'élément chimique}


%%%% Objectifs
\begin{objectifs}
  \item Apprendre la composition d'un atome.
  \item Comprendre la différence entre ion et atome.
\end{objectifs}

\begin{contexte}
  Au cours du \siecle{19}, la communauté scientifique considérait que l'atome était la plus petite \og brique \fg\, de la matière.
  Au début du \siecle{20}, deux expériences vont montrer que l'atome est composé de particules plus élémentaires :
  \begin{listePoints}
    \item en 1897, Thomson montre que l'on peut arracher des particules de charges négatives d'un atome ;
    \item en 1911, Rutherford montre que l'atome possède un noyau très petit devant la taille d'un atome, avec une charge positive.
  \end{listePoints}
  
  \problematique{
    Quelles entités composent les atomes ?
  }
\end{contexte}


%%%% Documents
\begin{doc}{Fabriquer des élément chimique}
  L'université du Colorado a créé une animation pour fabriquer des atomes à l'aide de leurs constituants.
  
  \important{Lancer l'application \og Symbole \fg\, sur : \url{https://tinyurl.com/nrenzhzh}}
\end{doc}


%%%%
\titreSection{L'atome}

%%%%
\question{
  Légender cette représentation d'un atome en utilisant les mots proton, neutron, électron, nucléons et noyau.
}{0}

\begin{flushleft}  
  \image{0.4}{images/atome/atome.jpg}
\end{flushleft}

\vspace*{-30pt}
\question{
  Dans l'application le cadre \og symbole \fg\, indique l'élément chimique fabriqué.
  Que faut-il ajouter pour changer d'élément chimique ?
}{2}

\question{
  Pour distinguer les atomes on utilise la notation \isotope{A}{Z}{X}. Compléter l'encadré ci-dessous.
}{0}

\vspace*{-8pt}
\begin{encart}
  \begin{listePoints}
    \item \chemfig{X} est le symbole de l'atome considéré.
    \item $Z$ est le nombre de \lignePointillee{0.2}, appelé \important{numéro atomique.}
    \item $A$ est le nombre de \lignePointillee{0.2}, appelé \important{nombre de masse.}
  \end{listePoints}
\end{encart}

\question{
  \isotope{23}{11}{Na} : le sodium \chemfig{Na} possède \lignePointillee{0.05} protons, \lignePointillee{0.05} nucléons, \lignePointillee{0.05} neutrons.
}{0}


%%%%
\titreSection{Les ions}

\question{
  Vérifier que la case \og Neutralité/Ionisation\fg\, est cochée.
  Dans quel cas un élément chimique est un atome neutre ?
  Comment appelle-t-on cet élément sinon ?
}{2}

\question{
  Que signifie le \og + \fg\, de \chemfig{Na^+} ? Donner la composition de l'élément.
}{2}

\question{
  Que peut-on dire de l'ion chlorure \chemfig{Cl^{-}} et de l'ion cuivrique \chemfig{Cu^{2+}} ?
}{1}


%%%%
\titreSection{Les isotopes}

\question{
  Vérifier que la case \og Stabilité/Instabilité \fg\, est cochée.
  Deux atomes du même élément peuvent-ils avoir des noyaux différents ?
}{2}

\question{
  Que manque-t-il à l'élément \isotope{2}{2}{He} pour être stable ?
}{1}