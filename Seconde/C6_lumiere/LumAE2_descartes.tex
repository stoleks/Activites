%%%%
\teteSndLumi
\nomPrenomClasse

%%%% titre
\numeroActivite{2}
\titreTP{La réfraction}

\begin{tableauCompetences}
  \centering REA &
  Réaliser une série de mesures avec précision.
  & & & &
\end{tableauCompetences}


%%%% Objectifs
\begin{objectifs}
  \item Comprendre le phénomène de réfraction
  \item Découvrir la loi de Snell-Descartes
\end{objectifs}

\begin{contexte}
  La lumière se propage en ligne droite dans un même milieu.
  
  Lorsque la lumière passe d'un milieu à un autre sa direction de propagation change : on dit qu'elle est \textbf{réfractée}.
  
  Pour notre cerveaux, la lumière se déplace toujours en ligne droite : c'est pour cela que les objets immergés dans de l'eau nous apparaissent déformés.
  
  \problematique{
    Comment décrire mathématiquement le phénomène de réfraction ?
  }
\end{contexte}


%%%% docs
\begin{doc}{Indice de réfraction}
  Quand la lumière se propage dans un milieu, sa vitesse est réduite.
  
  \begin{encart}
    La capacité d'un milieu à réduire la vitesse de la lumière est mesurée par un nombre que l'on appelle \important{l'indice de réfraction} et que l'on note $n_\text{milieu}$.
    
    Dans le milieu, la vitesse de la lumière est
    \begin{equation*}
      c_\text{milieu} = \Frac{c}{n_\text{milieu}}
    \end{equation*}
  \end{encart}
  
  \exemples
  \begin{listePoints}
    \item L'air a un indice de réfraction $n_\text{air} = 1,\!00$ et donc $c_\text{air} = c = 3,\!00 \times 10^8 \unit{m.s}^{-1}$.
    \item L'eau a un indice de réfraction $n_\text{eau} = 1,\!33$ et donc $c_\text{eau} = 2,\!26 \times 10^8 \unit{m.s}^{-1}$.
  \end{listePoints}
\end{doc}

\begin{doc}{La proportionnalité}
  Deux grandeurs ($a$ et $b$ par exemple) sont \textbf{proportionnelles} si le graphique représentant la grandeur $a$ en fonction de la grandeur $b$ est une droite passant par l'origine du repère.
  Ces deux grandeurs $a$ et $b$ sont alors reliées par l'égalité 
  \begin{equation*}
    a = k\times b
  \end{equation*}
  Dans cette égalité, $k$ est une constante, le coefficient directeur de la droite.
\end{doc}

\newpage
\vspace*{-36pt}
\begin{doc}{Schéma expérimental}
  \label{doc:schema_exp_disque_optique}
  \begin{wrapfigure}[2]{r}{0.52\linewidth}
    \vspace*{-54pt}
    \centering
    \image{1}{images/lumière/disque_optique_refraction.png}
  \end{wrapfigure}
  \textbf{Matériel disponible :}
  \begin{itemize}
    \item 1 source de lumière alimentée en 6-12 V continu ;
    \item 1 demi-cylindre de plexiglas sur son disque-support gradué en degrés.
  \end{itemize}
  \bigskip
  
  \phantom{b}
  % \attention Il ne faut \textbf{jamais} pointer le laser vers son oeil ou l'oeil de quelqu'un d'autre.
  % Une exposition directe du laser sur l'oeil risque de l'endommager de manière définitive.
\end{doc}


%%%%
\mesure
  Brancher la source lumineuse. Appeler le professeur pour vérifier le branchement.

\mesure
  Mesurer la valeur de l'angle de réfraction $i_2$ pour dix angles d'incidence $i_1$ différents.
  Noter les deux valeurs $i_1$ et $i_2$ pour chaque mesure.
  Appeler le professeur pour une des mesures.

\question{
  Utiliser le premier programme python pour tracer $i_1$ en fonction de $i_2$.
  Ces grandeurs sont-elles proportionnelles ?
}{
  ...
}{1}

\question{
  Utiliser le second programme python pour tracer $\sin(i_1)$ en fonction de $\sin(i_2)$.
  Ces grandeurs sont-elles proportionnelles ?
}{
  ...
}{1}


%%%%
\vspace*{-8pt}
\begin{doc}{Loi de Snell-Descartes}
  \vspace*{-24pt}
  \begin{encart}
    Lorsque la lumière passe d'un milieu d'indice $n_1$ à un milieu d'indice $n_2$, alors
    \begin{listePoints}
      \item le rayon incident, le rayon réfracté et la normale sont \dotfill %dans le même plan.
      \item \dotfill
      %$n_1 \sin(i_1) = n_2 \sin(i_2)$ pour la réfraction.
    \end{listePoints}
    
    Cette relation entre l'angle d'incidence $i_1$ et l'angle de réfraction $i_2$ s'appelle la \textbf{loi de Snell-Descartes}.
  \end{encart}
  
  On retrouve bien la relation de proportionnalité mesurée :
  \begin{equation*}
    \sin(i_1) = \Frac{n_2}{n_1} \times \sin(i_2)
  \end{equation*}
\end{doc}

\question{
  En utilisant la valeur du coefficient directeur 
  $k = n_\text{plexiglas} / n_\text{air}$
  calculée par le second programme python, calculer la valeur de l'indice de réfraction $n_\text{plexiglas}$.
}{
  ...
}{1}