%%%%
\teteSndLumi
\vspace*{-6pt}
\nomPrenomClasse

%%%% titre
\numeroActivite{2}
\titreActivite{Spectre d'une lampe}


%%%% Objectifs
\vspace*{-12pt}
\begin{objectifs}
  \item Analyser le spectre d'émission de raies d'une lampe pour déterminer les entités chimiques qui le composent.
\end{objectifs}


%%%% evaluation
\begin{tableauCompetences}
  \centering APP &
  Rechercher l'information, schématiser une situation.
  & & & &
  \\ \hline
  %
  \centering VAL &
  Comparer avec des valeurs de références.
  & & & &
\end{tableauCompetences}

\begin{contexte}
  Les gaz atomiques excitées émettent des raies d'émissions avec des longueurs d'onde précises.
  Chaque raie correspond à une onde monochromatique. 
  
  \problematique{
    Comment utiliser le spectre d'émission d'une lampe pour déterminer sa composition en entités chimiques ?
  }
\end{contexte}


%%%% docs
\begin{doc}{Spectre de raies de quelques éléments chimiques}
  \label{doc:spectre_H_Hg_Ne}
  Les éléments chimiques ont des spectres d'émission de raies qui leur sont propres.
  En regardant le spectre d'une source lumineuse, on peut donc déterminer les éléments chimiques qui composent la source.
  \vspace*{-8pt}
  \begin{center}
    \image{0.65}{images/lumière/spectre_gaz.jpg}
  \end{center}
\end{doc}


%%%%
\begin{boite}
  \textbf{Question version \og expert \fg}
\end{boite}
\numeroQuestion
  En utilisant le spectroscope, rédiger un rapport sur la composition du gaz se trouvant dans les tubes fluorescentes (aussi appelée \og tube néon \fg) éclairant la salle de classe.
  Ce rapport devra être argumenté à partir des données de spectre d'émission fournies dans les documents.


%%%%
\newpage
\begin{boite}
  \textbf{Questions version \og intermédiaire \fg}
\end{boite}

\numeroQuestion
  Observer avec le spectroscope la lumière provenant d'un tube fluorescent installé au plafond.
  Schématiser son spectre ci-dessous en respectant les graduations :
\begin{center}
  \image{0.75}{images/lumière/spectre_lumineux_graduee.jpg}
\end{center}  

\question{
  Décrire le spectre obtenu en choisissant parmi les mots suivants : \og raies \fg, \og continu \fg, \og polychromatique \fg, \og monochromatique \fg.
}{
  On obtient un spectre polychromatique, composé de plusieurs raies d'émissions. Chaque raie correspond à une longueur d'onde bien définie.
}{4}

\question{
  À l’aide des spectres du document~\ref{doc:spectre_H_Hg_Ne}, donner la composition du gaz contenu dans le tube fluorescent
}{
  Un élément chimique est présent dans le tube fluorescent, si toute les raies du spectre d'émission de l'élément chimique se retrouvent dans le spectre d'émission du tube fluorescent. \\
  Ici le tube fluorescent contient du mercure \chemfig{Hg}, car toutes ces raies peuvent être observée dans le spectre d'émission du tube.
}{4}

\question{
  Est-il approprié d'appeler \og tube néon \fg\; ce type d'éclairage ?
  Justifier à l'aide de vos observations.
}{
  Non, car il ne contient pas de néon.
}{4}