%%%%
\teteSndLumi

%%%% titre
\vspace*{-32pt}
\numeroActivite{1}
\titreActivite{Onde lumineuse}


%%%% Objectifs
\begin{objectifs}
  \item Connaître la vitesse de la lumière.
  \item Comprendre la notion de longueur d'onde.
  \item Comprendre la notion de rayonnement monochromatique.
\end{objectifs}

\begin{contexte}
  La lumière est en fait une onde électromagnétique, constitué d'un champs électrique et d'un champs magnétique.
  
  \problematique{
    Quelles sont les propriétés de cette onde électromagnétique ?
  }
\end{contexte}


%%%% docs
\begin{doc}{Onde électromagnétique}
  \vspace*{-24pt}
  \begin{encart}
    Une onde est une perturbation qui se propage.
  \end{encart}
  
  Une onde électromagnétique a un certain nombre de propriétés qui la définisse.
  Cette année on va se concentrer sur sa \important{vitesse de propagation} et sur sa \important{longueur d'onde,} notée $\lambda$.
  
  \begin{encart}
    Une onde est dite \important{monochromatique} (une couleur) si elle a une longueur d'onde bien définie.
    
    Une onde est dite \important{polychromatique} (plusieurs couleurs) si elle est la superposition de plusieurs ondes monochromatique.
  \end{encart}
\end{doc}

%%
\begin{doc}{Vitesse de propagation}
  \vspace*{-24pt}
  \begin{encart}
    Dans le vide, une onde électromagnétique se propage à la vitesse de la lumière notée $c$
    \begin{equation*}
      c = 3,\!00 \times 10^8 \unit{m.s}^{-1}
    \end{equation*}
  \end{encart}
\end{doc}

%%
\begin{doc}{Spectre électromagnétique}
  Le spectre électromagnétique est le classement des ondes électromagnétique par longueur d'onde. 
  \begin{center}
    \image{1}{images/lumière/spectre_EM.jpg}
  \end{center}
  Le domaine visible se trouve entre \textbf{380 nm (bleu)} et \textbf{700 nm (rouge)} de longueur d'onde et représente une petite partie du spectre électromagnétique.
\end{doc}


%%%%
\numeroQuestion
  Le soleil est une source de lumière qui émet une onde électromagnétique
\vspace*{-2pt}
\begin{qcm}
  \item monochromatique, avec une longueur d'onde.
  \item polychromatique, avec plusieurs longueurs d'onde.
\end{qcm}

Pour mieux visualiser la vitesse de la lumière, on va la comparer avec la vitesse d'un TGV.
Un TGV a une vitesse de pointe de $300 \unit{km.h}^{-1} = 83,\!3 \metreParSeconde$.
  
\question{
  Calculer le temps que met le TGV pour parcourir $1000 \unit{km} = 10^6 \unit{m}$ (distance Paris-Marseille).
}{
  \begin{align*}
     t_{\text{TGV}} &= \frac{d_\text{Paris-Marseille}}{v_\text{TGV}} \\
       &= \frac{10^6 \unit{m}}{83,\!3 \metreParSeconde} \\
       &= 1,\!20 \times 10^4 \unit{s}
  \end{align*}
  \vspace*{-24pt}
  \phantom{b}
}{2}

\question{
  Calculer le temps que met la lumière pour parcourir $10^6 \unit{m}$.
  Comparer les deux temps de parcours.
}{
  \begin{align*}
    t_{\text{lumière}} &= \frac{d_\text{Paris-Marseille}}{c} \\
      &= \frac{10^6 \unit{m}}{3,\!0 \times 10^8 \metreParSeconde} \\
      &= 3,\!3 \times 10^{-3} \unit{s}
  \end{align*}
  \phantom{b}\\[-24pt]
  La lumière est beaucoup plus rapide qu'un TGV : le temps que le TGV arrive à Marseille, la lumière aura fait 2 millions de fois l'aller-retour !
}{5}


%%
\begin{doc}{Longueur d'onde et énergie}
  L'énergie d'une onde électromagnétique est liée à sa longueur d'onde.
  Plus la longueur d'onde est petite et plus l'énergie d'une onde électromagnétique est élevée. 
  Il peut être dangereux d'être exposé à une onde électromagnétique avec une énergie élevée, qui pourrait endommager les tissus vivants.
  
  Une onde électromagnétique très énergétique, dans le domaine des rayons X, peut briser les liaisons covalentes d'une molécules ou arracher des électrons d'un atome, ce qui peut tuer des cellules vivantes.
\end{doc}

\question{
  Expliquer pourquoi un laser rouge est moins dangereux qu'un laser bleu.
}{
  Un laser rouge émet une onde électromagnétique avec une longueur d'onde plus élevée qu'un laser bleu. L'énergie de cette onde électromagnétique est donc plus faible et le laser rouge est moins dangereux.
}{3}