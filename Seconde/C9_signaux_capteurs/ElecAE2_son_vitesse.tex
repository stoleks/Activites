%%%%
\teteSndSign
\numeroActivite{2}

%%%% titre
\vspace*{-24pt}
\titreTP{Les sons et leur propagation}


%%%% Objectifs
\begin{objectifs}
  \item Découvrir les caractéristique d'un signal sonore
  \item Mesurer la vitesse du son dans l'air
\end{objectifs}


%%%% docs
\begin{doc}{Signal sonore}
  \vspace*{-36pt}
  \begin{wrapfigure}[5]{r}{0.6\linewidth}
    \vspace*{-28pt}
    \begin{center}
      \image{1}{images/son_emission_perception}
    \end{center}
  \end{wrapfigure}
  %
  \begin{encart}
    Un \important{signal sonore} est une \important{onde} de pression : c'est une perturbation qui se propage sans transport de matière.
  \end{encart}
  %
  Un son est la mise en vibration des entités chimiques d'un milieu matériel, comme l'air ambiant ou de l'eau. 
  Dans ce milieu matériel, il n'y a pas de déplacement de matière et la vitesse de propagation du son dépend de ce milieu.
\end{doc}


%%
\begin{doc}{Caractéristique d'un signal sonore}
  \vspace*{10pt}
  \begin{wrapfigure}[3]{l}{0.5\linewidth}
    \vspace*{-60pt}
    \begin{center}
      \image{1}{images/son_exemple_periode}
    \end{center}
  \end{wrapfigure}
  %
  Un signal sonore, ou un son, est caractérisé par son \important{intensité sonore} et sa \important{fréquence}.
  
  %
  \vspace*{40pt}
  \begin{encart}
    La fréquence $f$ est exprimée en hertz noté Hz, c'est l'inverse de la période de vibration $T$
    \begin{equation*}
      f = \frac{1}{T}
    \end{equation*}
  \end{encart}
\end{doc}


%%
\begin{doc}{Son et oreille}
  Un son est dit \important{audible} s'il peut être perçu par une oreille.
  Un son est audible si :
  \begin{listePoints}
    \item son niveau d'intensité sonore, mesuré en décibel noté dB, est suffisant.
    \item sa fréquence se trouve dans le domaine de sensibilité de l'oreille.
  \end{listePoints}
  \begin{equation*}
      \ldots < f_\text{audible} < \ldots
  \end{equation*}
\end{doc}



%%
\begin{doc}{Capteurs et smartphone}
  \vspace*{-18pt}
  \begin{wrapfigure}[5]{r}{0.2\linewidth}
    \vspace*{-24pt}
    \image{1}{images/QR_fizziq}
  \end{wrapfigure}
  On va chercher à mesurer la vitesse du son dans l'air.
  Pour ça on va utiliser l'application FizziQ, téléchargeable ici :
  
  Cette application permet d'utiliser les \important{capteurs} présent sur un smartphone pour réaliser des expériences de physique.
  
  %
  \begin{encart}
    Un \important{capteur} est un dispositif qui permet de transformer une grandeur physique mesurable en une grandeur exploitable.
  \end{encart}
  %
  La grandeur exploitable est, de nos jours, très souvent une tension électrique.
\end{doc}

\question{
  Citer des exemples de capteurs avec les grandeurs mesurées et exploitées.
}{
}{2}

\numeroQuestion Télécharger l'application FizziQ.

\question{
  En utilisant deux smartphone, la fonction déclencheurs de Fizziq et le microphone comme capteur, développer un protocole pour mesurer la vitesse du son dans l'air.
}{
}{10}

\mesure Mesurer la vitesse du son dans l'air avec votre protocole.