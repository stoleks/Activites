%%%%
\teteSndCorp

%%%% titre
\vspace*{-36pt}
\numeroActivite{0}
\titreTP{Découverte du laboratoire}


%%%% objectifs
\begin{objectifs}
  \item Connaître les pictogrammes de sécurité
  \item Connaître la verrerie de base en chimie
\end{objectifs}


%%%% docs
\begin{doc}{Les pictogrammes de sécurités}{doc:TP0_picto_secu}
  Les pictogrammes de sécurités sont à connaitre par coeur !
  \begin{encart}
  \end{encart}
\end{doc}

%%
\begin{doc}{Verrerie}{doc:TP0_verrerie}
  \begin{encart}
    La \important{verrerie} désigne l'ensemble des contenants utilisés pour réaliser des manipulations en chimie.
  \end{encart}
  La majorité de ces contenants sont en verre, c'est pour ça qu'on parle de \textit{verre}rie.
  
  Bécher x,
  tube à essai x,
  pipette jaugée,
  pipette graduée,
  éprouvette graduée x,
  coupelle de pesée x,
  erlenmeyer x,
  fiole jaugée x,
  propipette,
  poire x
\end{doc}


%%
\numeroQuestion Associer à chaque schéma de verrerie son nom.