%%%% début de la page
\teteSndCorp


%%%% titre
\numeroActivite{2}
\titreTP{Répression des fraudes}


%%%% objectifs
\begin{objectifs}
  \item Déterminer la masse volumique d'un échantillon.
  \item Mettre en oeuvre un protocole expérimental.
  \item Rédiger une problématique, un protocole et une conclusion.
\end{objectifs}


%%%% contexte
\begin{contexte}
  La \textsf{DGCCRF} (Direction générale de la concurrence, de la consommation et de la répression des fraudes) dispose de 11 laboratoires répartis dans tout le pays. 
  Les personnes qui travaillent dans ces laboratoires sont sollicitées pour vérifier la pureté de certains échantillons.
\end{contexte}

\important{\fleche Deux missions vous sont confiées par la DGCCRF.}

Pour chaque mission \textbf{vous devrez rédiger un rapport avec :}
\begin{listePoints}
  \item Le problème que l'on cherche à résoudre (problématique).
  \item Les protocoles et schémas des expériences réalisées.
  \item Les calculs et les mesures réalisées, avec les causes d'erreurs possibles.
  \item Une conclusion argumentée en utilisant les données fournies par les documents.
\end{listePoints}

\flecheLongue Pour la rédaction, faites en sorte que chaque rapport soit compréhensible par un élève de seconde qui ne connaîtrait pas le sujet.


%%%% document
\begin{doc}{Masse volumique}{doc:TP2_masse_volumique}
  \begin{encart}
    Chaque espèce chimique possède une masse volumique $\rho$ qui lui est propre.
    Pour un échantillon, elle est définie par le rapport entre la masse $m$ et le volume $V$ de cet échantillon : 
    \begin{equation*}
      \rho = \dfrac{m}{V}
    \end{equation*}
  \end{encart}
  
  \begin{listePoints}
    \item La masse s'exprime en \unit{\g}.
    Le volume s'exprime en \unit{\ml} ou \unit{\litre}.
    \item La masse volumique s'exprime en \unit{\g/\ml} ou \unit{\g/\litre}.
  \end{listePoints}
  Pour mesurer une masse volumique, il faut donc mesurer la masse et le volume d'un échantillon.
\end{doc}

%%
\begin{doc}{Glucose}{doc:TP2_glucose}
  \begin{wrapfigure}{r}{0.3\linewidth}
    \vspace{-30pt}
    \centering
    \chemfig[atom sep = 18pt]{
      *6(-(-OH) -(-OH) -(-OH) -O -(- -[3]OH) -) (-[-5]OH)
    }
  \end{wrapfigure}
  
  Le glucose est un composé chimique de formule brute\bruteCHO{6}{12}{6}.
  Le sucre que l'on consomme tous les jours et un corps pur composé de glucose.
  Il se présente sous la forme d'un solide blanc inodore.
  De par son côté addictif, le sucre est utilisé dans de nombreuses préparation agro-alimentaire.
\end{doc}


%%
\begin{doc}{Écart relatif}{doc:TP2_ecart_relatif}
  Pour comparer une valeur mesurée et une valeur théorique, on calcule l'écart relatif $ER$ entre ces deux valeurs en \unit{\percent}
  \begin{equation*}
    ER = \dfrac{|\text{mesurée} - \text{théorique}|}{\text{théorique}} \times 100
  \end{equation*}
  Si cet écart est faible, typiquement $ER \leq \qty{5}{\percent}$, on a un bon accord entre théorie et expérience.
\end{doc}

%%
\begin{multicols}{2}
  \begin{doc}{Matériel disponible}{doc:TP2_materiel_exp}
    Vous disposez de
    \begin{listePoints}
      \item 1 balance
      \item 1 pipette jaugée de \qty{10}{\ml}
      \item 1 éprouvette graduée de \qty{50}{\ml}
      \item 1 bécher de \qty{50}{\ml}
    \end{listePoints}
  \end{doc}
    
  \begin{doc}{Mesure d'un volume}{doc:TP2_mesure_volume_menisque}
    \begin{wrapfigure}{l}{0.48\linewidth}  
      \centering
      \vspace*{-18pt}
      \image{1}{images/chimie/mesure_volume_menisque}
    \end{wrapfigure}
    On mesure toujours le volume d'un liquide en repérant le bas du ménisque (la courbe) formé par le liquide.
  \end{doc}
\end{multicols}

\begin{multicols}{2}
  \begin{doc}{Mélange eau-glucose}{doc:TP2_densite_eau_sucre}
    Le glucose peut être dissous dans l'eau.
    La masse volumique du mélange eau-glucose dépend de la masse de sucre dissoute.
    \begin{center}
      \image{1}{images/donnees/densite_glucose.png}
    \end{center}
  \end{doc}  
  
  \begin{doc}{Mélange eau-éthanol}{doc:TP2_densite_ethanol_eau}
    L'eau et l'éthanol sont deux liquides miscibles.
    La masse volumique du mélange eau-éthanol dépend du pourcentage d'éthanol.
    \begin{center}
      \image{1}{images/donnees/densite_ethanol.png}
    \end{center}
  \end{doc}
\end{multicols}

%%%% questions
\emphase{Mission 1 : Alcool pharmaceutique}

L'entreprise \og SHACOL \fg, fabricant de solution hydroalcoolique, accuse son fournisseur de lui avoir donné de l'alcool pharmaceutique avec une fraction volumique d'éthanol inférieur à \num{0,70}.

Vous disposez d'un flacon d'alcool transmis par le fournisseur.

\begin{center}
  \textbf{Rédiger un rapport pour établir quelle entreprise a raison.}
\end{center}


%%%%
\emphase{Mission 2 : Sirop }

L'association de consommateur \og UFC-que choisir \fg, soupçonne une marque de sirop de mentir sur la quantité de sucre présente dans un sirop.
La marque annonce que le sirop contient une masse de sucre dissoute de \qty{20}{\g}.

Vous disposez d'un flacon du sirop de la marque.

\begin{center}
  \textbf{Rédiger un rapport pour établir si la marque a menti.}
\end{center}