%%%% début de la page
\teteSndCorp


%%%% titre
\numeroActivite{2}
\titreTP{Répression des fraudes}


%%%% objectifs
\begin{objectifs}
  \item Déterminer la masse volumique d'un échantillon.
  \item Mettre en oeuvre un protocole expérimental.
  \item Rédiger une problématique, un protocole et une conclusion.
\end{objectifs}


%%%% contexte
\begin{encart}
  \emphase{Contexte :}
  
  La \textsf{DGCCRF} (Direction générale de la concurrence, de la consommation et de la répression des fraudes) dispose de 11 laboratoires répartis dans tout le pays. 
  Les personnes qui travaillent dans ces laboratoires sont sollicitées pour vérifier la pureté de certains échantillons.
  
  Trois missions vous sont confiées par la \textsf{DGCCRF}.
\end{encart}

\textbf{\large \fleche Pour chaque mission vous devrez rédiger un rapport avec :}
\begin{itemize}
  \item Une problématique (problème que l'on cherche à résoudre).
  \item Le protocole expérimental utilisé et les mesures réalisées.
  \item Une conclusion argumentée en utilisant les données fournies par les documents.
\end{itemize}
{\large \fleche} Pour la rédaction, faites en sorte que chaque rapport soit compréhensible par un élève de la classe qui ne connaîtrait pas le sujet.


%%%% document
\begin{doc}{Masse volumique}
  Chaque espèce chimique possède une masse volumique $\rho$ qui lui est propre.
  Pour un échantillon, elle est définie par le rapport entre la masse $m$ et le volume $V$ de cet échantillon : 
  \begin{equation*}
    \rho = \Frac{m}{V}
  \end{equation*}
  \begin{itemize}
      \item La masse s'exprime en $\unit{g}$.
      \item Le volume s'exprime en $\unit{mL}$ ou $\unit{L}$.
      \item La masse volumique s'exprime en $\unit{g/mL}$ ou $\unit{g/L}$.
  \end{itemize}
\end{doc}

\begin{doc}{}
  \vspace*{-1.2cm}
  \begin{wrapfigure}{r}{0.2\linewidth}
    \image{1}{images/molecules/glycerol_chimie.png}
  \end{wrapfigure}
  
  Le glycérol est un composé chimique de fomule \chemfig{C_3H_8O_3}. C'est un liquide incolore, inodore et visqueux, avec un goût sucré. 
  Le glycérol est faiblement toxique et donc utilisé dans de nombreuses préparations pharmaceutiques.
  
  Masse volumique du glycérol $\rho = 1,26 \unit{g / mL}$.
\end{doc}


\begin{multicols}{2}
%%%%
\begin{doc}{Écart relatif}
  Pour comparer une valeur mesurée $\text{Mes}$ et une valeur théorique $\text{Théo}$, on calcule l'écart relatif entre ces deux valeurs en $\%$
  \begin{equation*}
    ER = \Frac{|\text{Mes} - \text{Théo}|}{\text{Théo}} \times 100
  \end{equation*}
  Si cet écart est faible, on a un bon accord entre théorie et expérience.
\end{doc}
%%%%

%%%%
\begin{doc}{Température de fusion}
  Acide benzoïque $\Tfus \approx 122\degree\unit{C}$.
  
  Glucose $\Tfus \approx 146\degree\unit{C}$
\end{doc}
%%%%

%%%%
\begin{doc}{}
  L'eau et l'éthanol forment un mélange homogène, dont la masse volumique varie en fonction de la quantité d'éthanol présente.
  \begin{center}
    \image{0.9}{images/donnees/masse_volumique_ethanol_eau.png}
  \end{center}
\end{doc}
%%%%
\end{multicols}


\begin{doc}{Mesure d'une température de fusion avec un banc Köfler}
  \vspace*{-1.2cm}
  \begin{wrapfigure}[10]{r}{0.6\linewidth}
    \centering
    \image{1}{images/instruments_chimie/banc_kofler.png}
    {\scriptsize Source : A.-S. Bernard et al., Techniques expérimentales en chimie}
  \end{wrapfigure}
  
  Un banc Köfler permet de mesurer la température de fusion d'un échantillon solide.
  Il est constitué d'une surface métallique inoxydable, qui est chauffée pour garantir une croissance continue de la température sur la longueur du banc.
  
  En déposant un échantillon solide directement sur le banc, on peut repérer quand le solide fond.
  Un index mobile permet de lire la température de fusion correspondante.
  
  La température à la surface du banc pouvant varier, pour régler le banc, on utilise des solides de référence dont on connaît la température de fusion.
\end{doc}

\newpage
%%%% questions
\emphase{Mission 1 : Glycérol pharmaceutique}

L'entreprise \og SHACOL \fg, fabricant de solution hydroalcoolique, affirme que son fournisseur ne lui a pas fourni du glycérol pur.

Vous disposez d'une bouteille de glycérol transmis par le fournisseur.

\begin{center}
\textbf{Rédiger un rapport pour établir quelle entreprise a raison.}
\end{center}


%%%%
\emphase{Mission 2 : Alcool pharmaceutique}

L'entreprise \og SHACOL \fg, accuse un autre fournisseur de lui avoir fourni de l'alcool pharmaceutique avec un pourcentage volumique d'éthanol inférieur à $70 \%$.

Vous disposez d'un flacon d'alcool transmis par le fournisseur.

\begin{center}
\textbf{Rédiger un rapport pour établir quelle entreprise a raison.}
\end{center}


%%%%
\emphase{Mission 3 : Glucose et acide benzoïque}

L'entreprise \og Confiture \& Co \fg, productrice de confiture, affirme que son fournisseur lui a fourni de l'acide benzoïque (un conservateur alimentaire) à la place de glucose.

Vous disposez d'un échantillon transmis par le fournisseur. \textbf{Le protocole de mesure et les mesures seront effectuées en démonstration.}

\begin{center}
\textbf{Rédiger un rapport pour établir quelle entreprise a raison.}
\end{center}


\feuilleBlanche
%%%% conclusion
% \begin{encart}
% \end{encart}
