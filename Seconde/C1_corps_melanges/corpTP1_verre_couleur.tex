%%%%
\teteSndCorp

%%%% titre
\vspace*{-36pt}
\numeroActivite{1}
\titreTP{Cocktail et vinaigrette}


%%%% objectifs
\begin{objectifs}
  \item Connaître le vocabulaire associé aux corps purs et mélanges.
  \item Connaître et manipuler la verrerie de base en chimie.
  \item Comprendre la notion de masse volumique.
\end{objectifs}

\begin{contexte}
  En cuisine, mélanger deux liquides peut amener à des résultats différents selon les combinaisons.
  Préparer un cocktail ou une vinaigrette ce n'est pas la même chose !
  
  \problematique{
    Quels notions physico-chimique utilise-t-on pour décrire les propriétés d'un mélange ?
  }
\end{contexte}


%%%% docs
\begin{doc}{Un peu de vocabulaire}{doc:vocabulaire_melange}
  \begin{encart}
    La matière est constituée \important{d'entités chimiques} microscopiques : atomes, molécules, ions.
    Une \important{espèce chimique} est constituée d’un ensemble d’entités chimiques
identiques.
  \end{encart}
    
  \begin{encart}
    \begin{listePoints}
      \item Un \important{corps pur} est constitué d'une seule espèce chimique.
      \item Un \important{mélange} est constitué de plusieurs espèces chimiques.
    \end{listePoints}
  \end{encart}
\end{doc}

%%
\begin{doc}{Type de mélange}{doc:type_melange}
  \begin{encart}
    Un mélange est \important{homogène} si on ne peut pas distinguer ses constituants.
    Un mélange homogène est constitué d'\important{une seule phase}.
  \end{encart}
  
  \begin{encart}
    Un mélange est \important{hétérogène} si on peut distinguer ses constituants.
    Un mélange hétérogène est constitué de \important{plusieurs phases}.
  \end{encart}

  \begin{encart}
    On dit que deux liquides sont \important{miscible} s'ils forment un \important{mélange homogène.}
  \end{encart}
  \begin{encart}
    Inversement, deux liquides sont \important{non miscibles} s'ils forment un \important{mélange hétérogène.}
  \end{encart}
  Miscible vient du latin \og misceo \fg, qui veut dire mélanger.
\end{doc}


%%
\mesure
Sur la paillasse se trouve une pissette d'eau distillée, de l'huile et de l'éthanol.
Dans les tubes à essais, verser :
\vspace*{-8pt}
\begin{center}
  \pointCyan Tube 1 : eau.
  \pointCyan Tube 2 : eau + huile.
  \pointCyan Tube 3 : eau + éthanol.
\end{center}
\vspace*{-8pt}
\attention Il faut faire attention à ne pas remplir les tubes, quelques centimètres suffisent.

\mesure 
Utiliser les bouchons pour agiter les différents mélanges.

%
\numeroQuestion
Attendre un peu, puis schématiser le résultat obtenu dans chaque tube à essais

%
\vspace{5cm}
\question{
  Décrire le contenu des tubes en utilisant le vocabulaire des documents~\ref{doc:vocabulaire_melange} et~\ref{doc:type_melange}.
}{
  Le tube 1 contient un corps pur.
  Le tube 2 contient un mélange hétérogène.
  Le tube 3 contient un mélange homogène.
}{3}

%
\question{
  Indiquer si l'eau et l'huile sont miscibles et si l'éthanol et l'huile sont miscibles.
}{
  L'eau et l'huile ne sont pas miscibles.
}{1}
  

%%
\begin{doc}{Notion de masse volumique}{doc:masse_volumique}
  \begin{encart}
    La \important{masse volumique} est une grandeur qui représente la masse par unité de volume d'un échantillon de matière.
  \end{encart}

  \separationDeuxBlocs{%
  \begin{encart}
    Si l'échantillon a une masse $m$ et un volume $V$, sa masse volumique est définie par
    \vspace*{-8pt}
    \begin{equation*}
      \rho = \Frac{m}{V}
    \end{equation*}
  \end{encart}
  }{%
  \vspace*{-18pt}
  \begin{donnees}
    \item $\rho (\text{eau liquide}) = \qty{1,00}{\g\per\ml}$
    \item $\rho (\text{huile}) = \qty{0,92}{\g\per\ml}$
    \item $\rho (\text{éthanol}) = \qty{0,79}{\g\per\ml}$
  \end{donnees}
  }
\end{doc}

%
\question{
  En utilisant les informations du document~\ref{doc:masse_volumique}, formuler une hypothèse qui expliquerait pourquoi l'huile flotte au dessus de l'eau.
}{
  L'huile flotte au dessus de l'eau, car elle a une masse volumique plus petite que la masse volumique de l'eau.
}{2}

%
\mesure
Vérifier l'hypothèse en versant dans un tube à essais l'huile et l'éthanol.
