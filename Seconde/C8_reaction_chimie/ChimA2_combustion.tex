%%%%
\sndEnTeteSix

%%%% titre
\vspace*{-44pt}
\numeroActivite{2}
\titreActivite{Combustion du méthane}

%%%% Objectifs
\begin{objectifs}
  \item Déterminer les produits d'une réaction chimique à partir de tests d'identification.
  \item Équilibrer une réaction chimique à l'aide de coefficients stoechiométrique.
\end{objectifs}

\begin{contexte}
  Dans les chaudière à gaz (chauffe-eau) ou dans les cuisinières à gaz, on utilise la combustion du méthane pour chauffer des aliments ou de l'eau.
  
  \problematique{
    Quelle est la réaction chimique de la combustion du méthane ?
  }
\end{contexte}


%%%% docs
\begin{doc}{Expérience}
  Le méthane \chemfig{CH_4} réagit avec le dioxygène \chemfig{O_2} lors de sa combustion pour former deux produits.
  On peut identifier ces produits en réalisant l'expérience suivante :
  
  \vspace*{-12pt}
  \begin{center}
    \image{0.4}{images/chimie/combustion_methane}
  \end{center}
\end{doc}


%%%% Questions
\question{
  Lister les réactifs de la réaction de combustion du méthane.
}{
  Le méthane \chemfig{CH_4} et le dioxygène \chemfig{O_2}
}{1}

\question{
  En présence de quelle espèce chimique l’eau de chaux devient-elle trouble ?
  Donner le nom de l'espèce et sa formule chimique.
}{
  Le dioxyde de carbone \chemfig{C_2 O}.
}{1}

\question{
  En présence de quelle espèce chimique le sulfate de cuivre anhydre devient-il bleu ?
  Donner le nom de l'espèce et sa formule chimique.
}{
  L'eau \chemfig{H_2 O}.
}{1}

\question{
  Conclusion : quels sont les produits formés lors de la combustion du méthane ?
}{
  L'eau et le dioxyde de carbone.
}{1}

\question{
  Écrire la réaction chimique de combustion du méthane.
}{
  \begin{center}
      \chemfig{CH_4}(g) + \chemfig{O_2}(g)
      \reaction
      \chemfig{CO_2}(g) + \chemfig{H_2O}(g)
  \end{center}
}{1}

%%
\begin{doc}{Équilibrage d'une réaction}
  \label{doc:equlibrage_reaction}
  Au cours d'une réaction chimique, les éléments chimiques présents dans les réactifs se réarrangent pour former des produits et les liaisons chimiques changent.
  \begin{encart}
    Il y a \important{conservation} 
    \begin{listePoints}
      \item \textbf{des éléments chimiques} ;
      \item \textbf{de la charge électrique} totale.
    \end{listePoints}
  \end{encart}
  \begin{encart}
    Pour assurer cette \important{conservation}, il faut \important{équilibrer} la réaction chimique avec des coefficients devant les éléments chimiques.
    Ces coefficients sont appelés \important{coefficient stoechiométrique.}
  \end{encart}
  
  Exemple de la réaction d'un acide avec du magnésium :
  \vspace*{-10pt}
  \begin{equation*}
    \underset{\text{1 atome de magnésium}}{\chemfig{Mg}(s)}
    + \underset{\text{\important{2} ions hydrogènes}}{\important{2} \;\; \chemfig{H^+ }(aq)}
    \reaction
    \underset{\text{1 ion magnésium II}}{\chemfig{Mg^{2+}}(aq)}
    + \underset{\text{1 molécule de dihydrogène}}{\chemfig{H_2}(g)}
  \end{equation*}
  On vérifie bien qu'il y a le même nombre de charges positives, de magnésium \chemfig{Mg} et d'hydrogène \chemfig{H}, dans l'état initial et dans l'état final.
\end{doc}

\question{
  Équilibrer la réaction de combustion du méthane à l'aide de coefficients stoechiométriques.
  Commencer par équilibrer le nombre d'atomes d'hydrogène.
}{
  \begin{equation*}
    \chemfig{CH_4}(g) + 2\chemfig{O_2}(g)
    \reaction
    \chemfig{CO_2}(g) + 2\chemfig{H_2O}(g)
  \end{equation*}
}{3}


%%%% bonus
\vspace*{-8pt}
\begin{doc}{Le butane}
  Parfois le gaz utilisé pour se chauffer est du butane et non du méthane.
  La formule chimique de la molécule de butane est \chemfig{C_4 H_{10}}.
  Le butane réagit avec le dioxygène et sa combustion forme les mêmes produits que la combustion du méthane.
\end{doc}

\question{
  Écrire la réaction de combustion du butane équilibrée avec des coefficients stoechiométriques.
  Préciser l'état physique des réactifs et des produits.
}{
  ...
}{3}

\vspace*{-8pt}
\begin{doc}{L'eau de chaux}
  L'eau de chaux est une solution aqueuse saturée en ion calcium \chemfig{Ca^{2+}} et en ion hydroxyde \chemfig{HO^{-}}.
  En réagissant avec le dioxyde de carbone \chemfig{CO_2}, l'eau de chaux forme du calcaire \chemfig{Ca CO_3} et de l'eau \chemfig{H_2O}
\end{doc}

\question{
  Écrire la réaction de formation du calcaire dans l'eau de chaux en présence de dioxyde de carbone et l'équilibrer avec des coefficients stoechiométrique.
}{
  ...
}{2}

\numeroQuestion
Équilibrer les réactions chimiques suivantes en écrivant, si nécessaire, les coefficients stoechiométriques devant chaque élément chimique :
\newcommand{\localEcart}{20}
\begin{align*}
  \ldots\chemfig{C}(s) + \ldots\chemfig{O_2}(g)
  &\reaction \ldots\chemfig{CO_2}(g)
  \\[\localEcart pt]
  %
  \ldots\chemfig{Fe}(s) + \ldots\chemfig{H^{+}}(aq)
  &\reaction \ldots\chemfig{Fe^{2+}}(aq) + \ldots\chemfig{H_2}(g)
  \\[\localEcart pt]
  %
  \ldots\chemfig{Fe}(s) + \ldots\chemfig{O_2}(g)
  &\reaction \ldots\chemfig{Fe_2O_3}(s)
  \\[\localEcart pt]
  %
  \ldots\chemfig{C_2H_6O}(l) + \ldots\chemfig{O_2}(g)
  &\reaction \ldots\chemfig{CO_2}(g) + \ldots\chemfig{H_2O}(l)
  \\[\localEcart pt]
  %
  \ldots\chemfig{Cu^{2+}}(aq) + \ldots\chemfig{HO^{-}}(aq)
  &\reaction \ldots\chemfig{Cu{(HO)}_2}(s)
  \\[\localEcart pt]
  %
  \ldots\chemfig{Fe}(s) + \ldots\chemfig{H_2O}(l) + \ldots\chemfig{O_2}(g)
  &\reaction \ldots\chemfig{Fe{(HO)}_2}(s)
  \\[\localEcart pt]
  %
  \ldots\chemfig{Fe{(OH)}_2}(s) + \ldots\chemfig{H_2O}(l) + \ldots\chemfig{O_2}(g)
  &\reaction \ldots\chemfig{Fe{(OH)}_3}(s)
  \\[\localEcart pt]
  %
  \ldots\chemfig{Fe{(OH)}_3}(s)
  &\reaction \ldots\chemfig{Fe_2O_3}(s) + \ldots\chemfig{H_2O}(l)
  %
\end{align*}

%%
\begin{minipage}[t]{0.9\linewidth}\vspace{0pt}
  \numeroQuestion
  Pour travailler les notions vues pendant la séance :
  \url{https://tinyurl.com/5d8j7e6b}
\end{minipage}
\begin{minipage}[t]{0.1\linewidth}\vspace{0pt}
  \centering
  \image{1}{images/chimie/QR_equilibrage_reactions_static}
\end{minipage}

%%
\begin{minipage}[t]{0.9\linewidth}\vspace{0pt}
  \numeroQuestion
  Pour aller plus loin :
  \url{https://tinyurl.com/4cp2924v}
\end{minipage}
\begin{minipage}[t]{0.1\linewidth}\vspace{0pt}
  \centering
  \image{1}{images/chimie/QR_equilibrage_reactions_phet}
\end{minipage}

%%
\feuilleBlanche