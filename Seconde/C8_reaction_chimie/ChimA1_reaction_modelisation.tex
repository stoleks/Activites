%%%%
\teteSndChim

%%%% titre
\vspace*{-40pt}
\numeroActivite{1}
\titreActivite{Réaction chimique}

%%%% Objectifs
\begin{objectifs}
  \item Comprendre qu'une réaction chimique modélise une transformation.
  \item Savoir utiliser l'écriture symbolique d'une réaction chimique.
\end{objectifs}

% \begin{contexte}
  
%   \problematique{
%   }
% \end{contexte}


%%%% docs
\begin{doc}{Observations macroscopiques}
  Pendant une transformation chimiques, des espèces chimiques interagissent, réarrangent leurs atomes, et forment d'autres espèces chimiques.
  Les espèces présentes initialement sont les \important{réactifs}. Celle présentes au final après la transformation sont les \important{produits.}
  
  Pour modéliser la transformation, il faut \important{identifier} les espèces chimiques qui réagissent et celles qui se forment.
  Pour ça, on observe ce qu'il se passe d'un point de vue macroscopique : formation d'un gaz ou d'un solide, disparition d'un solide, changement de couleur, etc.
  %Il est aussi possible d'utiliser des tests d'identification des espèces chimiques.
  
  \begin{encart}
    Les observations expérimentales macroscopiques permettent d'écrire l'équation de la \important{réaction} modélisant la transformation chimique microscopique, en identifiant les \important{réactifs} et les \important{produits.}
  \end{encart}
\end{doc}

\begin{doc}{Modélisation de la réaction}
  L'écriture de la réaction chimique permet de transcrire la transformation des réactifs en produit.
  
  \begin{encart}
    La réaction est symbolisée par une flèche. À gauche de la flèche se trouvent les \important{réactifs} qui se transforment et à droite de la flèche se trouvent les \important{produits} formés :
    \begin{center}
      réactif 1 + réactif 2 + \ldots \reaction produit 1 + produit 2 + \ldots
    \end{center}
  \end{encart}
  
  Au cours d'une réaction chimique, rien ne se perd, rien ne se crée. \textbf{Il doit donc y avoir le même nombre d'atomes et de charges de chaque côté de la réaction}.
  Seuls les liaisons des molécules peuvent être modifiées pendant une réaction chimique.
\end{doc}

\begin{doc}{Notation des états physiques}
  Les réactifs et les produits peuvent se trouver dans différents états physiques.
  Pour indiquer dans quel état se trouve une espèce chimiques, on écrit son état entre parenthèse à côté de sa formule chimique : $(g)$ pour un gaz, $(l)$ pour un liquide, $(s)$ pour un solide et $(aq)$ pour des solutés en solution aqueuse.
\end{doc}

\newpage
\vspace*{-40pt}
\begin{doc}{Combustion du charbon}
  On modélise la combustion du charbon avec du dioxygène par la réaction chimique suivante :
  \begin{equation*}
    \chemfig{C}(s) + \chemfig{O_2}(g) \reaction \chemfig{CO_2}(g)
  \end{equation*}
  On vérifie bien qu'il y a le même nombre d'atome de carbone et d'oxygène des deux côté de la réaction chimique.
\end{doc}


%%%% Questions
\question{
  Lister les réactifs et les produits pour la combustion du charbon en présence d'oxygène, en indiquant leurs état physique.
}{
  Réactifs : carbone solide et dioxygène gazeux.
  
  Produits : dioxyde de carbone gazeux.
}{2}


%%
\vspace*{-8pt}
\begin{doc}{Pile Daniell}
  La pile Daniell est une des premières pile inventée pour fournir de l'énergie électrique.
  Dans cette pile, des ions cuivre II \chemfig{Cu^{2+}} en solution aqueuse et du zinc solide \chemfig{Zn} réagissent pour former du cuivre solide \chemfig{Cu} et des ions zinc II \chemfig{Zn^{2+}} en solution aqueuse.
  Cette transformation permet de générer une tension électrique.
\end{doc}

\question{
  Lister les réactifs et les produits dans la pile Daniell.
}{
  Réactifs : ion cuivre II, zinc solide.
  
  Produits : cuivre solide, ion zinc II.
}{2}

\question{
  Écrire la réaction chimique modélisant la transformation dans la pile Daniell.
}{
  {\centering
    $\chemfig{Cu^{2+}} + \chemfig{Zn}(s) \reaction \chemfig{Zn^{2+}} + \chemfig{Cu}(s)$
  }
}{1}


%%
\vspace*{-8pt}
\begin{doc}{Test de reconnaissance des ions chlorure}
  En ajoutant du nitrate d'argent \chemfig{AgNO_3}, dans une solution aqueuse contenant des ions chlorure \chemfig{Cl^{-}}, il y a formation d'un précipité blanc qui noircit à la lumière.
\end{doc}

\question{
  Lorsque l'on met du nitrate d'argent en solution aqueuse, il se dissocie en ses ions constitutifs : \chemfig{Ag^+} et \chemfig{NO_3^{-}}.
  Écrire la réaction chimique qui modélise cette dissolution.
}{
  {\centering 
    $\chemfig{AgNO_3}(s) \reaction \chemfig{Ag^{+}} + \chemfig{NO_3^{-}}$
  }
}{1}

\question{
  Écrire la réaction chimique qui modélise la formation du précipité blanc.
}{
  {\centering 
    $\chemfig{Ag^{+}} + \chemfig{Cl^{-}} \reaction \chemfig{AgCl}(s)$
  }
}{1}

\vspace*{-12pt}
\begin{encart}
  Les espèces chimiques qui n'interviennent pas au cours de la réaction sont appelées des \important{espèces spectatrices}.
\end{encart}