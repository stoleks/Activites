%%%%
\teteSndMeth

%%%% titre
\vspace*{-36pt}
\numeroActivite{1}
\titreTP{Découverte du laboratoire}


%%%% objectifs
\begin{objectifs}
  \item Connaître les pictogrammes de sécurité
  \item Connaître la verrerie de base en chimie
\end{objectifs}


%%%% docs
\begin{doc}{Les pictogrammes de sécurités}{doc:TP0_picto_secu}
  Les pictogrammes de sécurités sont à connaître par c\oe{}ur ! \\[8pt]

  %% Tableau avec les pictogrammes
  \NewDocumentCommand{\pictoTableau}{m O{65}}{
    \hspace{16pt} \image{0.45}{images/securite/picto_#1} \vAligne{-#2pt}
  }
  \begin{tblr}{
    colspec = {|X[-1, h, m] |X[4, m] |}, hlines,
    row{1} = {couleurPrim!20, c}
  }
    \textbf{Pictogramme} & \textbf{Signification} \\
    %
    \pictoTableau{ronge} &
    {C\texteTrou{orrosif}. \\
    Je peux attaquer ou détruire les métaux.
    Je ronge la peau et/ou les yeux en cas de contact.} \\
    %
    \pictoTableau{toxique} &
    {T\texteTrou{oxique}, i\texteTrou{rritant}, n\texteTrou{arcotique}. \\
    J'empoisonne à forte dose.
    J'irrite la peau, les yeux et/ou les voies respiratoires.
    Je peux provoquer des allergies, de la somnolence ou des vertiges.} \\
    %
    \pictoTableau{tue}[60] &
    {T\texteTrou{oxique} \\
    J’empoisonne rapidement, même à faible dose.} \\
    %
    \pictoTableau{explose} &
    {E\texteTrou{xplosif}. \\
    Je peux exploser au contact d’une flamme, d’une étincelle, d’électricité statique, sous l’effet de la chaleur, de frottements ou d’un choc.} \\
    %
    \pictoTableau{flambe} &
    {I\texteTrou{inflammable}. \\
    Je peux m’enflammer au contact d’une flamme, d’une étincelle, d’électricité statique, sous l’effet de la chaleur, de frottements ou au contact de l’air ou de l’eau.} \\
    %
    \pictoTableau{flamber} &
    {C\texteTrou{omburant}. \\
    Je peux provoquer ou aggraver un incendie ou même provoquer une explosion en présence de produits inflammables.} \\
    %
    \pictoTableau{pression} &
    {G\texteTrou{az} s\texteTrou{ous} p\texteTrou{ression}. \\
    Je peux exploser sous l’effet de la chaleur.
    Je peux causer des brûlures ou blessures liées au froid.} \\
    %
    \pictoTableau{pollue} &
    {D\texteTrou{angereux} p\texteTrou{our} l'\texteTrou{environnement}. \\
    Je provoque des effets néfastes sur les organismes du milieu aquatique, sur les êtres vivants.} \\
    %
    \pictoTableau{sante} &
    {M\texteTrou{utagène}. C\texteTrou{ancerogène}. R\texteTrou{eprotoxique}. \\
    Je peux provoquer le cancer, modifier l’ADN, nuire à la fertilité ou au f\oe{}tus, altérer le fonctionnement des organes.
    Je peux être mortel en cas d’ingestion dans les voies respiratoires.}
    %
  \end{tblr}
\end{doc}

%%
\begin{doc}{Verrerie}{doc:TP0_verrerie}
  \begin{encart}
    La \important{verrerie} désigne l'ensemble des contenants utilisés pour réaliser des manipulations en chimie.
  \end{encart}
  La majorité de ces contenants sont en verre, c'est pour ça qu'on parle de \textit{verre}rie.
  
  Bécher x,
  tube à essai x,
  pipette jaugée,
  pipette graduée,
  éprouvette graduée x,
  coupelle de pesée x,
  erlenmeyer x,
  fiole jaugée x,
  propipette,
  poire x
\end{doc}


%%
\numeroQuestion Associer à chaque schéma de verrerie son nom.

\separationBlocs{
  \separationBlocs{
    \centering
    \image{1}{images/chimie/verrerie/schema0001} \\[-18pt]
    \pointCyan \\[3cm]
    \pointCyan \\ Bécher
  }{
    \centering
    \image{1}{images/chimie/verrerie/schema0002} \\[-18pt]
    \pointCyan \\[3cm]
    \pointCyan \\ Coupelle de pesée
  }
}{
  \separationBlocs{
    \centering
    \image{1}{images/chimie/verrerie/schema0003} \\[-18pt]
    \pointCyan \\[3cm]
    \pointCyan \\ Ballon monocol
  }{
    \centering
    \image{1}{images/chimie/verrerie/schema0004} \\[-18pt]
    \pointCyan \\[3cm]
    \pointCyan \\ Éprouvette graduée
  }
}
\vspace*{2cm}

\separationBlocs{
  \separationBlocs{
    \centering
    \image{1}{images/chimie/verrerie/schema0005} \\[-6pt]
    \pointCyan \\[3cm]
    \pointCyan \\ Poire 
  }{
    \centering
    \image{1}{images/chimie/verrerie/schema0006} \\[-6pt]
    \pointCyan \\[3cm]
    \pointCyan \\ Verre à pied
  }
}{
  \separationBlocs{
    \centering
    \image{1}{images/chimie/verrerie/schema0007} \\[-6pt]
    \pointCyan \\[3cm]
    \pointCyan \\ Erlenmeyer 
  }{
    \centering
    \image{1}{images/chimie/verrerie/schema0008} \\[-6pt]
    \pointCyan \\[3cm]
    \pointCyan \\ Pipette jaugée
  }
}