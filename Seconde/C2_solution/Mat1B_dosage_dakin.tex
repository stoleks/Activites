%%%% début de la page
\teteSndSolu


%%%% titre
\vspace*{-36pt}
\numeroActivite{4}
\titreTP{Dosage d'un antiseptique}
%\nomPrenomClasse


%%%% objectifs
\begin{objectifs}
  \item Apprendre le vocabulaire sur les solutions.
  \item Comprendre la notion de concentration massique.
  \item Doser la quantité de permanganate de potassium présente dans du Dakin.
\end{objectifs}


%%%% contexte
\begin{encart}
  \emphase{Contexte :}
  
  Le Dakin est une solution antiseptique qui sert à nettoyer des plaies. Le principe actif du Dakin est stabilisé par l'ajout de permanganate de potassium \chemfig{K MnO_4}.
  Le permanganate de potassium donne une teinte violette au Dakin.
\end{encart}

\problematique{Comment mesurer la concentration en permanganate de potassium présente dans le Dakin ?}


%%%% documents
\begin{doc}{Solution, solvant et soluté}
  \label{doc:solution}
  \vspace*{-24pt}
  \begin{encart}
    \chevron Une \important{solution} est un mélange homogène. \\
    Le \important{solvant} est le composant majoritaire du mélange. Le \important{soluté} est l'espèce qui est dispersée dans le solvant.
  \end{encart}
  
  \begin{center}
    \important{Solvant + Soluté(s) = Solution}
  \end{center}
  
  \begin{encart}
    On parle de \important{solution aqueuse} si le solvant est l'eau \chemfig{H_2 O}.
  \end{encart}
\end{doc}


%%%%
\begin{doc}{Concentration en soluté}
  \label{doc:concentration}
  \vspace*{-24pt}
  \begin{encart}
    La \important{concentration massique $\mathbf{c}$} mesure la quantité de soluté présent dans une solution.
    C'est le rapport de la masse $m$ de \textbf{soluté} dissous dans le volume $V$ de la \textbf{solution}
    \begin{equation}
      c = \frac{m_\text{soluté}}{V_\text{solution}}
    \end{equation}
  \end{encart}
  
  \attention Il faut bien distinguer \textbf{concentration massique} et \textbf{masse volumique}.
  La concentration mesure la masse de soluté contenue dans une solution.
  La masse volumique mesure la masse d'un échantillon contenue dans un volume donné.
\end{doc}


%%%%
\newpage
\begin{doc}{Dakin}
  \label{doc:dakin}
  Le Dakin est une solution aqueuse d'hypochlorite de sodium \chemfig{Na ClO}.
  Du permanganate de potassium \chemfig{K MnO_4} est ajouté à la solution, pour qu'elle ne soit pas dégradée par l'exposition au rayonnement UV du Soleil.
  
  \fleche Le constructeur indique que la concentration de \chemfig{KMnO_4} est de l'ordre de $0,\!01 \unit{g/L}$ dans le Dakin.
\end{doc}


%%%%
\begin{doc}{Mesure de concentration}
  \label{doc:dosage}
  \vspace*{-24pt}
  \begin{encart}
    On parle de \important{dosage} quand on mesure la concentration d'une espèce chimique présente dans une solution.
    
    Un \important{dosage par étalonnage} consiste à déterminer la concentration d’une espèce chimique en comparant une grandeur physique caractéristique de la solution, à la même grandeur physique mesurée pour des solutions étalon.
  \end{encart}
  
  \begin{encart}
    Une \important{échelle de teinte} permet de mesurer la concentration d'un soluté coloré.
    \vspace*{-16pt}
  \end{encart}

  La teinte d'une solution est proportionnelle à la concentration en soluté.
  En préparant une série de solutions de concentrations connues, une \textbf{gamme}, et en comparant les teintes, on va pouvoir encadrer la valeur de la concentration de la solution que l'on veut mesurer.
  
  \bigskip
  
  \attention Il faut comparer les teintes avec des verreries identiques, la teinte s'assombrit avec l'épaisseur.
  
  \attention La solution dont on veut mesurer la concentration doit avoir une teinte comprises dans la gamme réalisée !
\end{doc}
 

%%%%
\begin{doc}{Dilution d'une solution}
  \label{doc:dilution}
  \vspace*{-24pt}
  \begin{encart}
    La \important{dilution} est la diminution de la concentration d'une solution par ajout de solvant, sans ajout de soluté.
    La solution est diluée.
  \end{encart}
  On parle de \textbf{solution mère} pour la solution de départ et de \textbf{solution fille} pour la solution obtenue.
  
  Le facteur de dilution est le rapport des concentrations des solutions mère et fille.
  Ce rapport est égal au volume de la solution fille sur le volume de la solution mère
  \begin{equation*}
    F = \frac{c_\text{mère}}{c_\text{fille}}
      = \frac{V_\text{fille}}{V_\text{mère}}
  \end{equation*}
\end{doc}


%%%%
\begin{doc}{Protocole d'une dilution}
  \label{doc:protocole_dilution}
  \begin{center}
    \image{1}{images/protocoles/protocole_dilution.png}
  \end{center}
  
  \vspace*{-24pt}
  
  \begin{enumerate}
    \item Prélever le volume $V_\text{mère}$ de la solution mère à l'aide de la pipette graduée.
    Le bas du ménisque doit atteindre la graduation supérieure.
    \item Introduire la solution prélevée dans la fiole jaugée de volume $V_\text{fille}$.
    \item Ajouter de l'eau distillée dans la fiole jaugée jusqu'aux $2/3$ et agiter doucement. Compléter jusqu'à ce que le bas du ménisque atteigne le trait de jauge.
    \item Fermer la fiole et l'agiter en la retournant plusieurs fois.
    \item Verser la solution fille obtenue dans un bécher.
  \end{enumerate}
\end{doc}


%%% Questions
\vspace*{12pt}
\question{
  \docu{\ref{doc:solution}, \ref{doc:dakin}} Donner le solvant et les solutés de la solution de Dakin.
}{4}


%
\vspace*{-3pt}
\question{
  \docu{\ref{doc:concentration}} On dispose d'une solution mère de volume $V_\text{mère} = 20 \unit{mL}$, avec une concentration de permanganate de potassium $c_\text{mère} = 0,\!04 \unit{g/L}$.
  Calculer la masse de permanganate de potassium dans la solution.
}{2}


%
\question{
  \docu{\ref{doc:dosage}, \ref{doc:dilution}} On souhaite réaliser une échelle de teinte composée de 4 solutions étalon pour mesurer la concentration de permanganate de potassium dans le Dakin.

  \begin{center}
    \setlength{\extrarowheight}{4pt}
    \begin{tabular}{c | c | c | c | c}
      Solution étalon & \phantom{00}1\phantom{00} & \phantom{00}2\phantom{00}& \phantom{00}3\phantom{00} & \phantom{00}4\phantom{00} \\
      \hline
      Concentration (g/L) &  &  &  & 
    \end{tabular}
  \end{center}
  
  Donner le facteur de dilution entre les différentes solutions.
}{1}


%
\question{
  \docu{\ref{doc:dakin}} Justifier l’intervalle des concentrations proposées pour l’échelle de teinte, à partir de la valeur attendue de la concentration en permanganate de potassium.
}{1}


%
\question{
  \docu{\ref{doc:dilution}, \ref{doc:protocole_dilution}}
  Sachant que le volume de la fiole jaugée est de $50\unit{mL}$, donner le volume de la solution mère à prélever pour avoir un facteur de dilution $F = 2$.
}{2}


%
\question{
  \docu{\ref{doc:protocole_dilution}} Réaliser l'échelle de teinte en effectuant trois dilutions successives.
  Verser quelques millilitres de chaque solutions dans des tubes à essais.
}{0}


%
\vspace*{8pt}
\question{
  \docu{\ref{doc:dakin}, \ref{doc:dosage}} Utiliser l'échelle de teinte pour encadrer la valeur de la concentration en permanganate de potassium dans le Dakin.
  Est-elle cohérente avec celle du constructeur ?
}{2}


%
\question{
  Proposer une autre échelle de teinte pour améliorer la précision de la mesure (donner une liste de concentration).
}{1}
