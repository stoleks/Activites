%%%%
\teteSndTran

%%%% titre
\vspace*{-32pt}
\numeroActivite{1}
\titreTP{Fusion de la glace}


%%%% Objectifs
\begin{objectifs}
  \item Comprendre le lien entre énergie et température.
  \item Comprendre la notion de transformation endothermique et exothermique.
\end{objectifs}

\begin{contexte}
  Si on veut refroidir une boisson tiède, on peut la placer dans un réfrigérateur, mais une solution bien plus rapide est de rajouter des glaçons dedans.
  
  \problematique{
    Comment modéliser le changement de température lié à l'ajout des glaçons ?
  }
\end{contexte}


%%%% docs
\begin{doc}{Un peu de vocabulaire}
  Dans ce chapitre on va s'intéresser à l'évolution de la température et des états des objets.
  Cette branche de la physique s'appelle la \textbf{thermodynamique} (\og \textit{thermos} \fg : \textit{chaud} en grec ``thermodynamique'' = ``évolution de la chaleur'').
  Pour pouvoir définir précisément ce que l'on étudie, on utilise un vocabulaire particulier en thermodynamique.
  
  \begin{encart}
    \begin{listePoints}
      \item \important{Corps :} objet macroscopique continu avec des propriétés physiques bien définies (température, pression, état).
      \item \important{Système :} ensemble de corps dont on étudie l'évolution.
      \item \important{Milieu extérieur :} tous les corps qui ne sont pas le système.
    \end{listePoints}
  \end{encart}
  \attention Il faut faire attention à bien définir le système étudié et le milieu extérieur !
\end{doc}

%%
\begin{doc}{Transfert thermique}
  \vspace*{-20pt}
  \begin{encart}
    Un corps chaud en contact avec un corps froid lui transfert de l'énergie, ce qui se traduit par une modification de la température des deux corps : on parle de \important{transfert thermique}.
  \end{encart}
  L'énergie transférée se note $Q$, son unité est le Joule \unit{joule}.
  Un corps qui \textbf{reçoit un transfert thermique positif} ($Q > 0$) voit \textbf{sa température augmenter.}
  
  \begin{encart}
    Sous certaines conditions, ce transfert thermique peut mener un des deux corps à changer d'état (solide à liquide par exemple) : on parle de \important{transformation physique}.
  \end{encart}
  \attention le transfert thermique va \textbf{toujours} du corps chaud vers le corps froid !
  Si un corps pur change d'état, sa température ne varie pas au cours du transfert thermique.
\end{doc}

%%
\begin{doc}{Calorimètre}
  \vspace*{-34pt}
  \begin{wrapfigure}{r}{0.15\linewidth}
    \centering
    \vspace*{-8pt}
    \image{0.9}{images/transformations/calorimetre}
  \end{wrapfigure}
  
  Un calorimètre (\og \textit{calor} \fg : \textit{chaleur} en latin) est un récipient qui sert à mesurer des transferts thermiques.
  \textbf{Un calorimètre est un vase qui isole son contenu de tous transfert thermique avec l'extérieur :} aucune chaleur n'y rentre ni n'en sort.
  Tous les transferts thermiques se passent donc entre les corps que contient le calorimètre.
\end{doc}

%%
\begin{doc}{Protocole de mesure de la variation de température}
  \label{doc:protocole_fusion_eau}
  \vspace*{-20pt}
  \begin{protocole}
    \item Placer le calorimètre sur la balance et appuyer sur ``tare''.
    \item Verser environ \qty{200}{\ml}} d'eau et mesurer la masse $m_\text{eau}$ introduite.
    \item Fermer le calorimètre et introduire le thermomètre. Mesurer la température initiale de l'eau $T_i$.
    \item Mesurer la masse $m_\text{glaçons}$ d'au moins deux glaçons sur une balance.
    \item Introduire rapidement ces glaçons dans le calorimètre et le refermer.
    \item Quand les glaçons ont entièrement fondus, agiter l'eau et mesurer sa température finale $T_f$.
  \end{protocole}
  
  \textbf{Mesures réalisées :}
  \begin{center}
    \newcommand{\tailleTableau}{\hspace{60pt}\phantom{$\Frac{1}{8}$}}
    \begin{tabular}{| c | c | c | c | c |}
      \hline
      \rowcolor{gray!20}
      Grandeur mesurée & $m_\text{eau}$ & $m_\text{glaçons}$ & $T_i$ & $T_f$ 
      \\ \hline
      Mesure & \tailleTableau & \tailleTableau & \tailleTableau & \tailleTableau
      \\ \hline
    \end{tabular}
  \end{center}
\end{doc}

%%
\begin{doc}{Énergie de changement d'état}
  \label{doc:energie_changement_etat}
  Pour faire fondre de la glace, il faut un transfert thermique entre la glace et un autre corps.
  \begin{encart}
    L'énergie nécessaire pour changer d'état s'appelle \important{l'énergie de changement d'état} et on la note $L$, son unité est le Joule$\unit{J}$.
  \end{encart}
  Plus la masse de la glace est élevée et plus l'énergie de changement d'état sera élevée.
  
  \begin{encart}
    On peut définir \important{l'énergie de changement d'état massique} notée $L_m$, qui est propre à chaque corps pur et s'exprime en Joule par gramme \unit{\joule\per\g} :
    \begin{equation*}
      L_m = \Frac{L}{m_\text{glace}}
    \end{equation*}
  \end{encart}
\end{doc}


%%%%
\vspace*{-12pt}
\titreSection{Premier système étudié : l'eau liquide}

\mesure
Réaliser le protocole du document~\ref{doc:protocole_fusion_eau} en notant les valeurs des mesures expérimentales.

\question{
  L'eau liquide a-t-elle gagné ou perdu de l'énergie par transfert thermique ?
}{
  La température de l'eau a diminué, l'eau liquide a donc perdu de l'énergie par transfert thermique.
}{1}

\question{
  On peut calculer le transfert thermique reçu par l'eau liquide à partir de sa masse et de sa variation de température
  \begin{equation*}
    Q = m_\text{eau} \times c_\text{eau} \times (T_f - T_i)
  \end{equation*}
  où $c_\text{eau} = \qty{4,180}{\Joule\per\g\per\degreeCelsius}$ est la capacité calorifique de l'eau.
  Cette constante mesure la quantité d'énergie nécessaire pour augmenter la température de \qty{1}{\degreeCelsius} pour \qty{1}{\g} d'eau.
  
  Calculer la valeur de $Q$ avec vos mesures.
}{
  \vspace*{-18pt}
  \begin{align*}
    Q
    & = c_\text{eau} \times m_\text{eau} \times (T_f - T_i) \\
    & = \qty{4,180}{\Joule\per\g\per\degreeCelsius}
      \times \qty{199}{\g}
      \times (\num{6,3} - \num{17,4}) \unit{\degreeCelsius} \\
    & = \qty{-9233}{\Joule}
  \end{align*}
  \vspace*{-24pt}
}{2}


\titreSection{Second système étudié : les glaçons}

\question{
  \label{exo:energie_L_fusion}
  Comme on utilise un calorimètre, on va considérer que tous le transfert thermique $Q$ fourni par l'eau liquide a servi à faire fondre les glaçons.
  Donner la valeur de $L$ l'énergie de changement d'état de fusion de la glace.
}{
  Toute l'énergie perdue par l'eau est transférée aux glaçons, qui gagne donc une énergie $- Q$.
  Toute cette énergie fait fondre les glaçons, on a donc
  \begin{equation*}
    L = -Q
  \end{equation*}
  \vspace*{-12pt}
}{2}

\question{
  En vous aidant du document~\ref{doc:energie_changement_etat}, calculer la valeur $L_m$ de l'énergie de changement d'état massique de fusion de la glace.
}{
  \vspace*{-18pt}
  \begin{align*}
    L_m
    = \Frac{L}{m_\text{glaçon}}
    = \Frac{\qty{9233}{\Joule}}{\qty{27,92}{\g}}
    = \qty{330,7}{\Joule\g}
  \end{align*}
  \vspace*{-12pt}
}{2}

\question{
  Comparer avec la valeur de référence $L_{m, \text{référence}} = \qty{334}{\Joule\per\g}$.
  L'hypothèse de la question~\ref{exo:energie_L_fusion} vous semble-t-elle valide ?
}{
  On trouve une énergie de changement d'état massique plus petite que la valeur de référence.
  On peut expliquer cette différence par le fait que le glaçon commence à fondre avant d'être placé dans le calorimètre.
}{2}


%%%%
\newpage
\vspace*{-12pt}
\titreSection{Bilan}

On voit que pour fondre, les glaçons ont dû recevoir de l'énergie sous forme de transfert thermique par l'eau liquide autour d'eux.
On parle de \texteTrouMultiLignes{transformation endothermique.}{1}

%%
\begin{doc}{Transformations endothermique et exothermique}
  \vspace*{-20pt}
  \begin{encart}
    \begin{listePoints}
      \item Si l'énergie du système \textbf{augmente}, $Q > 0$, pendant une transformation physique, on parle de \important{transformation endothermique}.
      \item Si l'énergie du système \textbf{diminue}, $Q < 0$, pendant une transformation physique, on parle de \important{transformation exothermique}.
    \end{listePoints}
  \end{encart}
  \begin{center}
     \image{1}{images/transformations/transformation_energie}
  \end{center}
  \attention Attention aux signes !
  
  \begin{listePoints}
    \item Pour une réaction \textbf{endothermique} le système reçoit de l'énergie et $Q > 0$, ce qui implique que le milieu extérieur va se refroidir. 
    \item Au contrainte pour une réaction \textbf{exothermique} le système perd de l'énergie et $Q < 0$, ce qui implique que le milieu extérieur va se réchauffer.
  \end{listePoints}
\end{doc}
