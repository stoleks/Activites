%%%% début de la page
\sndEnTeteDeux

%%%%
\nomPrenomClasse


%%%% titre
\numeroActivite{1}
\titreActivite{Décrire le mouvement}


%%%% objectifs
\begin{objectifs}
  \item Décrire un mouvement.
  \item Comprendre la notion de référentiel.
  \item Comprendre que le mouvement dépend du référentiel.
\end{objectifs}


%%%% evaluation
\begin{tableauCompetences}
  \centering APP &
  Rechercher l'information.
  
  Représenter la situation par un schéma.
  & & & &
  \\ \hline
  \centering ANA/RAI &
  Proposer une stratégie de résolution.
  & & & &
\end{tableauCompetences}

%%%%
\vspace*{6pt}
\begin{doc}{Un peu de vocabulaire}
  \vspace*{-24pt}
  \begin{encart}
    \important{Système} : objet dont on étudie le mouvement.
  \end{encart}
  
  \begin{encart}
    \important{Trajectoire} : ensemble des positions successives occupées par le système.
  \end{encart}
  
  Le \important{mouvement} d'un système est donné par la description de sa trajectoire + l'évolution de sa vitesse.
\end{doc} 


\vspace*{2pt}
\begin{doc}{Type de trajectoires}
  \label{doc:trajectoires}

  Trajectoire \important{rectiligne} : trajectoire représentée par \dotfill \\[4pt]
  Trajectoire \dotfill : trajectoire représentée par un cercle. \\[4pt]
  Trajectoire \important{curviligne} : trajectoire représentée par \dotfill
\end{doc}


\vspace*{2pt}
\begin{doc}{Vitesse et accéleration}
  \label{doc:vitesse}

  Vitesse \important{uniforme} (constante) : le système n’accélère pas. \\[4pt]
  La vitesse augmente : le système \dotfill \\[4pt]
  La vitesse diminue : le système \dotfill \\[4pt]
  Si la vitesse est \dotfill on dit que le système est \important{immobile}.
\end{doc}


%%%%
\newpage

\question{Compléter les documents~\ref{doc:trajectoires} et~\ref{doc:vitesse}.}{0}

\fleche Pour la suite de cette activité, vous allez choisir entre l'étude du mouvement des oies ou de la Lune.
Vous présenterez ensuite les résultats de votre étude au reste de la classe à l'oral.


%%%%
\titreSection{\'Etude du mouvement des oies}

Le compteur du bateau affiche une vitesse $v_\text{bateau} = 3,\!6 \times 10^1 \unit{km/h}$.

\vspace*{6pt}
\question{Pour la personne qui filme les oies, quelle est la vitesse des oies ?}{0}
\question{Pour une personne se trouvant sur la berge, quelle est la vitesse des oies ?}{0}
\question{Schématiser la trajectoire des oies si on les observe depuis la berge.}{0}
\question{Indiquer le mouvement des oies depuis le bateau et la berge.}{0}

\titreSection{\'Etude du mouvement de la Lune}

La Lune tourne autour de la Terre à une vitesse $v_\text{Lune} = 3,\!7 \times 10^3 \unit{km/h}$ et la Terre tourne autour du Soleil à une vitesse $v_\text{Terre} = 1,\!1 \times 10^5 \unit{km/h}$.

\begin{figure}[!ht]
  \begin{subfigure}{0.48\linewidth}
    \centering
    \image{0.8}{images/mouvements/terre_lune.png}
    \caption{Point de vue centré sur la Terre}
    \label{fig:terre_lune}
  \end{subfigure}
  \begin{subfigure}{0.48\linewidth}
    \centering
    \image{0.8}{images/mouvements/terre_lune_soleil.png}
    \caption{Point de vue centré sur le Soleil}
    \label{fig:terre_lune_soleil}
  \end{subfigure}
\end{figure}

\vspace*{-6pt}
\question{Depuis le point de vue centré sur la Terre, quelle est la vitesse de la Lune ?}{0}
\question{Schématiser la trajectoire de la Lune depuis ce point de vue et indiquer son mouvement.}{0}
\question{Peut-on décrire la vitesse de la Lune depuis le point de vue centré sur le Soleil ?}{0}
\question{Schématiser la trajectoire de la Lune depuis ce point de vue.}{0}


\titreSection{Notion de référentiel}

\question{Convertir la vitesse $v_\text{Lune}$ en m/s.
\textit{Rappel : } $1\unit{km} = 10^3\unit{m}$, $1\unit{h} = 3,\!6 \times 10^3 \unit{s}$.}{0}
\question{Quelle distance la Lune parcours pendant 1 seconde ? Comparer avec la longueur de sa trajectoire, qui est de $2,4\times 10^6 \unit{km}$}{0}
\question{Est-ce que la seconde est une durée d'observation suffisante pour décrire la trajectoire de la Lune ?}{0}
\question{Conclusion : pourquoi est-il important de définir le référentiel, qui est l’endroit où on se place et le temps passé à observer, avant d'étudier un mouvement ?}{0}

\feuilleBlanche