%%%% début de la page
\sndEnTeteDeux

%%
\nomPrenomClasse


%%%% titre
\numeroActivite{7}
\titreActivite{Chute d'une goutte d'encre}


%%%% evaluation
\begin{tableauCompetences}
  %
  \centering APP &
  Analyser un programme python.
  & & & &
  \\ \hline
  %
  \centering REA &
  Mesurer des positions avec un logiciel de pointage.
  & & & &
  \\ \hline
  %
  \centering COM &
  Travailler en groupe.
  & & & &
\end{tableauCompetences}


%%%% objectifs
\begin{objectifs}
  \item Utiliser des outils numériques pour analyser un mouvement.
\end{objectifs}


%%%%
\begin{doc}{Mesurer les positions avec Cineris}
  \label{doc:cineris}
  Pour mesurer les positions d'un objet, dans Cineris il faut cliquer sur :
  \begin{enumeration}
    \item \texttt{Montage} (1), puis \texttt{choix du fichier} et ouvrir \url{chute_goutte.avi}.
    \item Cliquer sur \texttt{Traitement manuel} (2) puis sur \texttt{étalonnage} (3)
    \begin{listePoints}
      \item maintenir appuyé le clic-gauche de la souris sur la vidéo
      \item glisser verticalement le long d'un objet de référence et relâcher pour régler l'échelle verticale.
    \end{listePoints}
    \item Cliquer sur le bouton vert dans \texttt{traitement} (4), puis 
    \begin{listePoints}
      \item cliquer sur le centre de l'objet pour mesurer sa position ;
      \item répéter pour chaque instants de la vidéo ;
      \item appuyer sur le bouton rouge pour arrêter le traitement.
    \end{listePoints}
    \item Cliquer sur \texttt{Tableau} (5) pour accéder aux positions mesurée.
  \end{enumeration}
  
  \centering
  \image{0.6}{images/mouvements/logiciel_cineris.jpg}
\end{doc}


%%%%
\newpage
\phantom{b}
\vspace*{-44pt}
\begin{doc}{Programme python pour tracer la trajectoire}
  \label{doc:code_python}
  \vspace*{-20pt}
  \lstset{style=codePython, language=python}
  \begin{lstlisting}
    import numpy as np # bibliotheque de calcul
    import matplotlib.pyplot as plt # bibliotheque d'affichage
    
    # dessine une fleche partant du point 'depart' et allant au point 'fin'
    def traceFleche (depart, fin) :
        taille = 0.1 # taille de la pointe de la fleche
        plt.arrow (depart[0], depart[1], fin[0], fin[1], # coordonnees
                   head_length=taille, head_width=taille) # apparence
    
    # calcul et trace le vecteur vitesse
    def traceVitesses (x, y, Dt) :
        for i in range (1, len (x) - 1) :
            vx = (x[i + 1] - x[i - 1]) / (2*Dt)
            vy = (y[i + 1] - y[i - 1]) / (2*Dt)
            traceFleche ((x[i], y[i]), (vx, vy))
    
    # reglage du graphique
    plt.axis('equal') # pour avoir des vecteurs symetriques
    plt.xlabel (r'$x$ (en cm)') # legende de l'abscisse
    plt.ylabel (r'$y$ (en cm)') # legende de l'ordonnee
    plt.title ("Trajectoire d'une goutte d'encre") # titre du graphique
    
    # definition de la trajectoire
    Dt = 1
    x = [0, 0, 0, 0, 0, 0, 0, 0, 0, 0, 0, 0, 0, 0, 0, 0, 0, 0, 0, 0]
    y = []
    
    # trace les positions et les vecteurs vitesses
    plt.plot (x, y, 'go') # g : vert (green), o : cercle
    traceVitesses (x, y, Dt) # trace les vitesses
    plt.show () # affiche le graphique\end{lstlisting}
  \vspace*{-8pt}
\end{doc}


%%%%
\mesure
  À l'aide du document~\ref{doc:cineris}, mesurer les positions de la goutte d'encre avec Cineris.


%%
\question{
  Dans le programme python du document~\ref{doc:code_python}, repérer la ligne qui indique les positions successives de la goutte d'encre. Compléter cette ligne avec les positions que vous avez mesuré-es sur Cineris, puis exécuter le programme.
}{...}{0}

%%
\question{
  Décrire le mouvement de la goutte d'encre.
}{...}{1}

%%
\question{
  Que pouvez-vous en déduire sur les forces qui s'exercent sur la goutte d'encre ?
}{...}{2}

%%
\question{
  Expliquer pourquoi le programme ne peut pas calculer la vitesse initiale et finale.
}{...}{2}