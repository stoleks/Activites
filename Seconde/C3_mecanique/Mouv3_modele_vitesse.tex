%%%% début de la page
\sndEnTeteDeux

%%
\nomPrenomClasse


%%%% titre
\numeroActivite{3}
\titreActivite{Modéliser le mouvement}


%%%% objectifs
\vspace{-10pt}
\begin{objectifs}
  \item Comprendre les limites du modèle du point matériel.
  \item Comprendre la notion de référentiel.
  \item Comprendre la notion de vecteur.
\end{objectifs}


%%%% evaluation
\begin{tableauCompetences}
  \centering Communication (COM) &
  Travailler en groupe, échanger entre élèves.
  & & & &
\end{tableauCompetences}


%%%%
\titreSection{Système et référentiel}

%%%%
\vspace{-10pt}
\begin{doc}{Modèle du point matériel}
  \vspace*{-24pt}
  \begin{encart}
    \important{Système} : objet dont on étudie le mouvement.
  
    On ne va s'intéresser qu'au mouvement global du système.
    C'est pourquoi on va modéliser le système par \dotfill \\[8pt]
    \reponse{1}
    \vspace*{-12pt}
    %un point de même masse, localisé en son centre de masse.
    %C'est le \important{modèle du point matériel}.
  \end{encart}

  \fleche Le modèle du point matériel revient à oublier toute information sur la géométrie du système étudié. 
  Les éventuelles rotations et déformations ne sont donc pas prises en compte.
\end{doc}


\begin{tabularx}{\linewidth}{ m{0.2\linewidth} | m{0.2\linewidth} | m{0.25\linewidth} | m{0.26\linewidth} }
  \rowcolor{gray!20}
  \centering Système & Centre de masse & \centering Trajectoire & Informations perdues
  \\ \hline
  %
  \centering
  \phantom{\small b} \newline
  \image{0.8}{images/mouvements/point_balle_tennis.png}
  \newline
  Balle de tennis &
  \centering Centre de la balle & &
  \\ \hline
  %
  \centering
  \phantom{\small b} \newline
  \image{0.8}{images/mouvements/point_roue.jpg} \newline
  Roue &
  \centering Centre de la roue & &
  \\ \hline
\end{tabularx}
\newpage

\vspace*{-30pt}
\begin{tabularx}{\linewidth}{ m{0.2\linewidth} | m{0.2\linewidth} | m{0.25\linewidth} | m{0.3\linewidth} }
  \hline
  %
  \centering
  \phantom{\small b} \newline
  \image{0.8}{images/mouvements/point_humain_course.jpg}
  \newline
  Modèle d'humain &
  \centering Nombril & &
  \\ \hline
\end{tabularx}

%%
\begin{doc}{Référentiel}
  Pour décrire le mouvement, il faut pouvoir le repérer dans l’espace et dans le temps, pour ça on utilise un référentiel.
  
  \begin{encart}
    \important{Référentiel} : objet de référence, muni d'un \dotfill, \newline
    par rapport auquel on étudie le mouvement du système.
  \end{encart}
  
  \begin{encart}
    La description du mouvement dépend du \important{référentiel} choisi. On appelle ça la \important{relativité} du mouvement.
  \end{encart}
\end{doc}


%%%%
\titreSection{Vecteur}

%%
\begin{doc}{Vecteur en physique}
  \label{doc:vecteur}
  \vspace*{-24pt}
  \begin{encart}
    \important{Vecteur} : objet mathématique représenté par un segment fléché $\longrightarrow$ et noté avec une lettre surmontée d'une flèche $\vv{v}$.
    
    Un vecteur contient quatre information : 
    \begin{listePoints}
      \item \dotfill
      \item \dotfill 
      \item \dotfill 
      \item \dotfill 
    \end{listePoints}
  
    Un vecteur est \important{constant} si \dotfill  \\[8pt]
    \reponse{1}
    \vspace{-8pt}
  \end{encart}
  
  \fleche En physique on va se servir des vecteurs pour représenter différentes quantités : \\[8pt]
  \reponse{1}
  
  \attention Un vecteur n'est \textbf{jamais} égal à un nombre, qui contient moins d'information.
\end{doc}