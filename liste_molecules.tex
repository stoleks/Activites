\documentclass[12pt]{extarticle}

%%%% paramètres généraux et commandes prédéfinies
\input{_parametres}

%% Issus du très bon package ProfLycée de Cedric Pierquet https://ctan.org/pkg/proflycee
\NewTCBListing{boiteCodeTex}{ O{CouleurPrim} m }{%
  enhanced, width=0.93\linewidth,
  flush right, boxrule=0.75pt, colframe=#1!85!black,%
  sharp corners, top=0mm, bottom=0mm, left=0.4em, right=5mm,%
  before skip=\baselineskip, after skip=\baselineskip,%
  colback=white,
  fontupper=\footnotesize, fontlower=\footnotesize,%
  watermark text={\faCode}, watermark opacity=0.25,watermark zoom=0.50,%
  title={{\scriptsize\faCode} Code \LaTeX},
  lefttitle=0.4em,
  fonttitle=\bfseries\footnotesize\sffamily,colbacktitle=darkgray!50!#1,%
  StyleCodeTex,
  #2,%
  % listing engine=minted,minted style=colorful,minted language=tex,
  % minted options={tabsize=4,fontsize=\footnotesize,autogobble,breaklines=true},
  % overlay={
  %   \draw[#1!85!black] ($(frame.north west) + (-0.035\linewidth,-0.025\linewidth)$) node[scale=1.66] {\faCode} ;
  % }
}

%%%% doc
\begin{document}
  %%%%
  \tableofcontents
  
  %%%%%%%%%%%%%%%%%%%%%%%%%%%%%%%%%%%%%%%%%%%%%%%%%%%%%%%%%%%%%
  \section{Logique interne}
  
  \subsection{Nom des molécules}
  
  Pour tracer une molécule, il suffit d'appeler \lstinline|\chemfig\{!\nomDeLaMolecule}|.
  La représentation de base pour les molécules est la formule topologique, il faut ajouter un suffixe au nom pour passer à une autre représentation \important{si elle est définie, ce qui n'est pas du tout toujours le cas.} Les suffixes sont les suivants :
  \begin{listePoints}[2]
    \item \lstinline{SemiDev} : formule semi-développée ;
    \item \lstinline{Dev} : formule développée ;
    \item \lstinline{Haw} : représentation de Haworth ;
    \item \lstinline{Cram} : représentation de Cram.
  \end{listePoints}
  Pour les acides aminés, il existe quatre autres suffixes
  \begin{listePoints}[2]
    \item \lstinline{L} : représentation de Fischer gauche ;
    \item \lstinline{H} : pour tracer un polypeptide, la chaîne latérale est vers le haut ;
    \item \lstinline{D} : représentation de Fischer droite ;
    \item \lstinline{B} : pour tracer un polypeptide, la chaîne latérale est vers le bas.
  \end{listePoints}
  
  \subsection{Commandes internes pour faciliter l'écriture}
  
  Pour tracer les formules topologiques, j'utilise plusieurs commandes 
  \begin{listePoints}
    \item \lstinline|!\vide{}| pour tracer une liaison invisible (utile pour les cycles incomplets)
    \item \lstinline|!\lh| Pour tracer une liaison vers le haut (liaison haut = lh) \chemfig{-[:30] !\lh}
    \item \lstinline|!\lb| Pour tracer une liaison vers le bas (liaison bas = lb) \chemfig{-[:30] !\lb}
    \item \lstinline|!\lhb| Pour tracer une liaison vers le haut puis vers le bas \chemfig{-[:30] !\lhb}
    \item \lstinline|!\lbh| Pour tracer une liaison vers le bas puis vers le haut \chemfig{-[:30] !\lbh}
    \item \lstinline|!\llh| Pour tracer une liaison double vers le haut \chemfig{-[:30] !\llh}
    \item \lstinline|!\llb| Pour tracer une liaison doubler vers le bas \chemfig{-[:30] !\llb}
    \item \lstinline|!\cis| Pour tracer une liaison cis \chemfig{-[:-30] !\cis !\lb}
    \item \lstinline|!\trans| Pour tracer une liaison "trans" aplatie \chemfig{-[:-30] !\trans !\lh}
    \item \lstinline|!\ldh| Pour tracer une liaison développée vers le haut (l'angle est plus faible) \chemfig{-[:30] !\ldh}
    \item \lstinline|!\ldb| Pour tracer une liaison développée vers le bas \chemfig{-[:30] !\ldb}
    \item \lstinline|!\lldh| Pour tracer une liaison double développée vers le haut \chemfig{-[:30] !\lldh}
    \item \lstinline|!\lldb| Pour tracer une liaison double développée vers le bas \chemfig{-[:30] !\lldb}
    \item \lstinline|!\cram{}{}| Pour tracer deux liaisons de cram autour d'un élément \chemfig[cram width = 5pt]{C !\cram{A}{B} (-[::90] COOH) -[::-30] NH_2}
    \item \lstinline|!\branche{}{}| Pour tracer deux liaisons à \qty{90}{\degree} autour d'un élément \chemfig{- !\branche{A}{B} -}
  \end{listePoints}
  
  %%%%%%%%%%%%%%%%%%%%%%%%%%%%%%%%%%%%%%%%%%%%%%%%%%%%%%%%%%%%%
  \section{Lipides}
  \subsection{Acide gras}
  
  \begin{boiteCodeTex}{}
\chemfig{!\palmitique}
\chemfig{!\oleique}
\chemfig{!\linoleique}
\chemfig{!\linolenique}
\chemfig{!\arachidonique}
\chemfig{!\eicosaPentaenoique}
\chemfig{!\docosaHexanoique}
  \end{boiteCodeTex}
  \chemfig{!\palmitique} \\[8pt]
  \chemfig{!\oleique}
  \chemfig{!\linolenique} \\[8pt]
  \chemfig{!\linoleique}
  \chemfig{!\arachidonique} \\[8pt]
  \chemfig{!\eicosaPentaenoique}
  \chemfig{!\docosaHexanoique}

  \begin{boiteCodeTex}{}
\chemfig{!\steraiqueSemiDev}
\chemfig{!\oleiqueSemiDev}
\chemfig{!\oleateSemiDev} \qq{}
\chemfig{!\caproiqueSemiDev}
  \end{boiteCodeTex}
  \chemfig{!\steraiqueSemiDev}
  \chemfig{!\oleiqueSemiDev}
  \chemfig{!\oleateSemiDev}
  \chemfig{!\caproiqueSemiDev}
  
  \subsection{Pour les chaînes dans les triglycérides}
  \begin{boiteCodeTex}{}
\chemfig{[:-30] !\tripalmitique}
\chemfig{[:-30] !\trioleique}
\chemfig{[:-30] !\trilinoleique}
\chemfig{[:-30] !\trilinolenique}
  \end{boiteCodeTex}
  \chemfig{[:-30] !\tripalmitique}
  \chemfig{[:-30] !\trioleique}

  \chemfig{[:-30] !\trilinoleique}
  \chemfig{[:-30] !\trilinolenique}
  
  
  %%%%%%%%%%%%%%%%%%%%%%%%%%%%%%%%%%%%%%%%%%%%%%%%%%%%%%%%%%%%%
  \subsection{Triglycérides}
  \begin{boiteCodeTex}{}
\chemfig{!\palmitine}
\chemfig[atom sep = 1.8em]{!\oleine}
\chemfig[atom sep = 1.8em]{!\arachidonine}
  \end{boiteCodeTex}
  \chemfig{!\palmitine}
  
  \chemfig[atom sep = 1.8em]{!\oleine}
  
  \chemfig[atom sep = 1.8em]{!\arachidonine}
  
  \begin{boiteCodeTex}{}
\chemfig{!\oleineSemiDev}
\chemfig{!\palmitineSemiDev}
\chemfig{!\caproineSemiDev}
  \end{boiteCodeTex}
  \chemfig{!\oleineSemiDev} \qq{}
  \chemfig{!\palmitineSemiDev} \\[8pt]
  \chemfig{!\caproineSemiDev}

  \subsection{Phospholipide}  
  \begin{boiteCodeTex}{}
\chemfig{!\phosphatidylcholine}
  \end{boiteCodeTex}
  \chemfig{!\phosphatidylcholine}
  
  \subsection{glycérol}
  \begin{boiteCodeTex}{}
\chemfig{!\glycerol}
\chemfig{!\glycerolSemiDev}
  \end{boiteCodeTex}
  \chemfig{[:30] !\glycerol} \qq{}
  \chemfig{!\glycerolSemiDev}

  \subsection{Sous-molécules utiles}
  \begin{boiteCodeTex}{}
\chemfig{[:60] !\triester{A}{B}{C}}
\chemfig{A-[:30] !\glycero{!\lh B} !\lb C }
\chemfig{A-[:30] !\glycero{!\lh B} !\lb C }
  
\chemfig[atom sep = 15pt]{!\triester {!\trioleique} {!\trilinolenique} {!\tricaproique}}
\chemfig[atom sep = 15pt]{!\triesterSat {!\lb !\trioleique} {!\tripalmitique} !\lb !\trilaurique}
  \end{boiteCodeTex}
  \chemfig{[:60] !\triester{A}{B}{C}}
  \chemfig{!\triesterSat{A}{B}C}
  \chemfig{A-[:30] !\glycero{!\lh B} !\lb C }
  
  \chemfig[atom sep = 15pt]{!\triester {!\trioleique} {!\trilinolenique} {!\tricaproique}}  
  \chemfig[atom sep = 15pt]{!\triesterSat {!\lb !\trioleique} {!\tripalmitique} !\lb !\trilaurique}
  
  
  %%%%%%%%%%%%%%%%%%%%%%%%%%%%%%%%%%%%%%%%%%%%%%%%%%%%%%%%%%%%%
  \subsection{Stérols}  
  
  \begin{boiteCodeTex}{}
\chemfig{!\cholesterol}
  \end{boiteCodeTex}
  \chemfig{!\cholesterol}
 

  %%%%%%%%%%%%%%%%%%%%%%%%%%%%%%%%%%%%%%%%%%%%%%%%%%%%%%%%%%%%%
  \section{Glucides}
  \subsection{Amidon}
  \begin{boiteCodeTex}{}\chemfig{!\amylopectineHaw}\end{boiteCodeTex}
  \chemfig{!\amylopectineHaw}
  
  \subsection{glucose}
  \begin{boiteCodeTex}{}
\chemfig{!\glucoseHaw}
\chemfig{!\glucoseCycle}
\chemfig{[:-90] !\glucose}
\chemfig{[:-90] !\glucoseSemiDev}
  \end{boiteCodeTex}
  \chemfig{!\glucoseHaw}
  \chemfig{!\glucoseCycle}
  \chemfig{[:-90] !\glucose}
  \chemfig{[:-90] !\glucoseSemiDev}
  
  \subsection{fructose}
  \begin{boiteCodeTex}{}
\chemfig{!\fructoseHaw}
\chemfig{!\fructofuranoseHaw}
\chemfig{!\fructoseCycle}
\chemfig{[:-90] !\fructose}
\chemfig{[:-90] !\fructoseSemiDev}
  \end{boiteCodeTex}
  \chemfig{!\fructoseHaw}
  \chemfig{!\fructofuranoseHaw}
  \chemfig{!\fructoseCycle}
  \chemfig{[:-90] !\fructose}
  \chemfig{[:-90] !\fructoseSemiDev}

  \subsection{Galactose et saccharose}
  \begin{boiteCodeTex}{}
\chemfig{!\galactoseHaw}
\chemfig{!\saccharoseHaw}
  \end{boiteCodeTex}
  \chemfig{!\galactoseHaw}
  \chemfig{!\saccharoseHaw}

  \subsection{Ribose et desoxyribose}
  \begin{boiteCodeTex}{}
\chemfig{A !\ribose B}
\chemfig{A !\desoxyribose B}
  \end{boiteCodeTex}
  \chemfig{A !\ribose B}
  \chemfig{A !\desoxyribose B}

  \begin{boiteCodeTex}{}
\chemfig{A !\riboseHaw B}
\chemfig{A !\desoxyriboseHaw B}
  \end{boiteCodeTex}
  \chemfig{A !\riboseHaw B}
  \chemfig{A !\desoxyriboseHaw B}
  
  
  %%%%%%%%%%%%%%%%%%%%%%%%%%%%%%%%%%%%%%%%%%%%%%%%%%%%%%%%%%%%%
  \section{Acides alpha aminés}
  \subsection{Formules topologiques}
  \begin{boiteCodeTex}{}
\chemfig{!\arginine}
\chemfig{!\histidine}
\chemfig{!\lysine}
\chemfig{!\aspartique}
  \end{boiteCodeTex}
  \chemfig{!\arginine}
  \chemfig{!\histidine}
  \chemfig{!\lysine}
  \chemfig{!\aspartique}
  
    \begin{boiteCodeTex}{}
\chemfig{!\glutamique}
\chemfig{!\serine}
\chemfig{!\threonine}
\chemfig{!\asparagine}
  \end{boiteCodeTex}
  \chemfig{!\glutamique}
  \chemfig{!\serine}
  \chemfig{!\threonine}
  \chemfig{!\asparagine}
  
  \begin{boiteCodeTex}{}
\chemfig{!\glutamine}
\chemfig{!\cysteine}
\chemfig{!\selenocysteine}
\chemfig{!\glycine}
  \end{boiteCodeTex}
  \chemfig{!\glutamine}
  \chemfig{!\cysteine}
  \chemfig{!\selenocysteine}
  \chemfig{!\glycine}
  
  \begin{boiteCodeTex}{}
\chemfig{!\proline}
\chemfig{!\alanine}
\chemfig{!\valine}
\chemfig{!\isoleucine}
\chemfig{!\leucine}
  \end{boiteCodeTex}
  \chemfig{!\proline}
  \chemfig{!\alanine}
  \chemfig{!\valine}
  \chemfig{!\isoleucine}
  \chemfig{!\leucine}
  
  \begin{boiteCodeTex}{}
\chemfig{!\methionine}
\chemfig{!\phenylalanine}
\chemfig{!\tyrosine}
\chemfig{!\tryptophane}
  \end{boiteCodeTex}
  \chemfig{!\methionine}
  \chemfig{!\phenylalanine}
  \chemfig{!\tyrosine}
  \chemfig{!\tryptophane}
  
  \subsection{Formules semi-développées}
  \begin{boiteCodeTex}{}
\chemfig{!\alanineSemiDev}
\chemfig{!\asparagineSemiDev}
\chemfig{!\glycineSemiDev}
\chemfig{!\cysteineSemiDev}
  \end{boiteCodeTex}
  \chemfig{!\alanineSemiDev} \qq{}
  \chemfig{!\asparagineSemiDev} \qq{}
  \chemfig{!\glycineSemiDev} \\[8pt]
  \chemfig{!\cysteineSemiDev} \\[8pt]
  
  \subsection{Représentation de Fischer}
  \begin{boiteCodeTex}{}
\chemfig{!\alanineL}
\chemfig{!\alanineD}
\chemfig{!\valineL}
\chemfig{!\valineD}
  \end{boiteCodeTex}
  \chemfig{!\alanineL} \quad
  \chemfig{!\alanineD} \quad
  \chemfig{!\valineL} \quad
  \chemfig{!\valineD}
  

  \subsection{Acide alpha aminés pour les polypeptides}
  \begin{boiteCodeTex}{}
\chemfig{
  [:-30] H_2N !\alanineH N !\glycineB N !\cysteineH N !\isoleucineB N !\valineH OH
}
  \end{boiteCodeTex}
  \chemfig{[:-30] H_2N !\alanineH N !\glycineB N !\cysteineH N !\isoleucineB N !\valineH OH}

  \subsection{Heme}
  \begin{boiteCodeTex}{}
\chemfig{!\hemeB}
  \end{boiteCodeTex}
  \chemfig{!\hemeB}
  
  
  %%%%%%%%%%%%%%%%%%%%%%%%%%%%%%%%%%%%%%%%%%%%%%%%%%%%%%%%%%%%%
  \section{Hormones}
  \begin{boiteCodeTex}{}
\chemfig{!\creatinine}
\chemfig{!\DOPA}
\chemfig{!\DOPAH}
\chemfig{!\prostaglandine}
  \end{boiteCodeTex}
  \chemfig{!\creatinine}
  \chemfig{!\DOPA}
  \chemfig{!\DOPAH} \\[8pt]
  \chemfig{!\prostaglandine}

  \subsection{Corticoïdes et minéralocorticoïdes}
  \begin{boiteCodeTex}{}
\chemfig{!\cortisol}
\chemfig{!\corticosterone}
\chemfig{!\aldosterone}
  \end{boiteCodeTex}
  \chemfig{!\cortisol}
  \chemfig{!\corticosterone} \\[8pt]

  % Minéralocorticoïdes
  \chemfig{!\aldosterone}

  \subsection{Oestrogènes}
  \begin{boiteCodeTex}{}
\chemfig{!\estrone}
\chemfig{!\estriol}
\chemfig{!\estradiol}
  \end{boiteCodeTex}
  \chemfig{!\estrone}
  \chemfig{!\estriol} \\[8pt]

  \chemfig{!\estradiol}

  \subsection{Androgènes}
  \begin{boiteCodeTex}{}
\chemfig{!\testosterone}
\chemfig{!\dihydrotestosterone}
\chemfig{!\androstenedione}
  \end{boiteCodeTex}
  \chemfig{!\testosterone}
  \chemfig{!\dihydrotestosterone}
  \chemfig{!\androstenedione}

  \begin{boiteCodeTex}{}
\chemfig{!\DHEA}
\chemfig{!\DHEAS}
  \end{boiteCodeTex}
  \chemfig{!\DHEA}
  \chemfig{!\DHEAS}

  \subsection{Progestatif}
  \begin{boiteCodeTex}{}
\chemfig{!\progesterone}
  \end{boiteCodeTex}
  \chemfig{!\progesterone}

  
  %%%%%%%%%%%%%%%%%%%%%%%%%%%%%%%%%%%%%%%%%%%%%%%%%%%%%%%%%%%%%
  \section{Nucléotides}
  \subsection{Bases nucléiques}
  \begin{boiteCodeTex}{}
\chemfig{A- !\adenine}
\chemfig{A- !\cytosine}
\chemfig{A- !\guanine} 
\chemfig{A- !\thymine} 
\chemfig{A- !\uracile} 
  \end{boiteCodeTex}
  \chemfig{A- !\adenine} \hspace*{-20pt}
  \chemfig{A- !\cytosine}
  \chemfig{A- !\guanine} \hspace*{-20pt}
  \chemfig{A- !\thymine} 
  \chemfig{A- !\uracile} 

  \subsection{Ribonucléosides}
  \begin{boiteCodeTex}{}
\chemfig{!\adenosine}
\chemfig{!\cytidine} 
\chemfig{!\guanosine}
\chemfig{!\thymidine}
\chemfig{!\uridine}  
  \end{boiteCodeTex}
  \chemfig{!\adenosine}
  \chemfig{!\cytidine} 
  \chemfig{!\guanosine} \\[8pt]
  \chemfig{!\thymidine}
  \chemfig{!\uridine}  

  \begin{boiteCodeTex}{}
\chemfig{!\adenosineHaw}
\chemfig{!\cytidineHaw} 
\chemfig{!\guanosineHaw}
\chemfig{!\thymidineHaw}
\chemfig{!\uridineHaw}  
  \end{boiteCodeTex}
  \chemfig{!\adenosineHaw}
  \chemfig{!\cytidineHaw} 
  \chemfig{!\guanosineHaw} \\[8pt]
  \chemfig{!\thymidineHaw}
  \chemfig{!\uridineHaw}  
  
  \subsection{Desoxyribonucléosides}
  \begin{boiteCodeTex}{}
\chemfig{!\desoxyAdenosine}
\chemfig{!\desoxyCytidine} 
\chemfig{!\desoxyGuanosine}
\chemfig{!\desoxyThymidine}
\chemfig{!\desoxyUridine}  
  \end{boiteCodeTex}
  \chemfig{!\desoxyAdenosine}
  \chemfig{!\desoxyCytidine} 
  \chemfig{!\desoxyGuanosine} \\[8pt]
  \chemfig{!\desoxyThymidine}
  \chemfig{!\desoxyUridine}  

  \begin{boiteCodeTex}{}
\chemfig{!\desoxyAdenosineHaw}
\chemfig{!\desoxyCytidineHaw} 
\chemfig{!\desoxyGuanosineHaw}
\chemfig{!\desoxyThymidineHaw}
\chemfig{!\desoxyUridineHaw}  
  \end{boiteCodeTex}
  \chemfig{!\desoxyAdenosineHaw}
  \chemfig{!\desoxyCytidineHaw} 
  \chemfig{!\desoxyGuanosineHaw} \\[8pt]
  \chemfig{!\desoxyThymidineHaw}
  \chemfig{!\desoxyUridineHaw}  

  \subsection{Adénosine triphosphate et adénoside diphosphate}
  \begin{boiteCodeTex}{}
\chemfig{!\ADP}
\chemfig{!\ATP}
  \end{boiteCodeTex}
  \chemfig{!\ADP}
  \chemfig{!\ATP}

  \begin{boiteCodeTex}{}
\chemfig{!\ADPHaw}
\chemfig{!\ATPHaw}
  \end{boiteCodeTex}
  \chemfig{!\ADPHaw}
  \chemfig{!\ATPHaw}
  
  
  %%%%%%%%%%%%%%%%%%%%%%%%%%%%%%%%%%%%%%%%%%%%%%%%%%%%%%%%%%%%%
  \section{Produits de contraste}
  \begin{boiteCodeTex}{}
\chemfig{!\ionChelate}
\chemfig{!\chelateAlcool}
  \end{boiteCodeTex}
  \chemfig{!\ionChelate}
  \chemfig{!\chelateAlcool}
  
  
  %%%%%%%%%%%%%%%%%%%%%%%%%%%%%%%%%%%%%%%%%%%%%%%%%%%%%%%%%%%%%
  \section{Vitamines}
  \subsection{Vitamines B}
  \begin{boiteCodeTex}{}
\chemfig{!\thiamine}           % B1
\chemfig{!\riboflavine}        % B2
\chemfig{!\nicotinamide}       % B3
\chemfig{!\acideNicotinique}   % B3
\chemfig{!\acidePantothenique} % B5
\chemfig{!\pyroxidine}         % B6
\chemfig{!\biotine}            % B8
\chemfig{!\acideFolique}       % B9
\chemfig{!\cyanocobalamine}    % B12
  \end{boiteCodeTex}
  \chemfig{!\thiamine}
  \chemfig{!\riboflavine}
  
  \chemfig{!\nicotinamide} \qq{}
  \chemfig{!\acideNicotinique} \qq{}
  \chemfig{!\acidePantothenique}
  
  \chemfig{!\pyroxidine}
  \chemfig{!\biotine}
  
  \chemfig{!\acideFolique}
  
  \chemfig{!\cyanocobalamine}
  
  \subsection{Vitamine C}
  \begin{boiteCodeTex}{}
\chemfig{!\acideAscorbique} % C
  \end{boiteCodeTex}
  \chemfig{!\acideAscorbique} \qq{}
  
  \subsection{Vitamines A, D, E, $K_1$ et $K_2$}
  \begin{boiteCodeTex}{}
\chemfig{!\retinol}         % A
\chemfig{!\cholecarciferol} % D
  \end{boiteCodeTex}
  \chemfig{!\retinol} \\[8pt]
  \chemfig{!\cholecarciferol}
  
  \begin{boiteCodeTex}{}
\chemfig{!\tocopherol}  % E
\chemfig{!\tocotrienol} % E
  \end{boiteCodeTex}
  \chemfig{!\tocopherol} \\[8pt]
  \chemfig{!\tocotrienol}
  
  \begin{boiteCodeTex}{}
\chemfig{!\phylloquinone} % K1
\chemfig{!\menatetrenone} % K2
  \end{boiteCodeTex}
\chemfig{!\phylloquinone} \\[8pt]
\chemfig{!\menatetrenone}
  
  
  %%%%%%%%%%%%%%%%%%%%%%%%%%%%%%%%%%%%%%%%%%%%%%%%%%%%%%%%%%%%%
  \section{Médicaments et produits de synthèse}
  \subsection{Aspirine}
  \begin{boiteCodeTex}{}
\chemfig{!\aspirine}
\chemfig{!\aspirineSemiDev}
\chemfig{!\acideSalicylique}
  \end{boiteCodeTex}
  \chemfig{!\aspirineSemiDev}
  \chemfig{!\aspirine} \qq{}
  \chemfig{!\acideSalicylique}
  
  \subsection{Paracétamol}
  \begin{boiteCodeTex}{}
\chemfig{!\paracetamol}
\chemfig{!\paracetamolSemiDev}
\chemfig{!\paracetamolDev}
  \end{boiteCodeTex}
  \chemfig{!\paracetamol}
  \chemfig{!\paracetamolSemiDev}
  \chemfig{!\paracetamolDev}

  \subsection{Aspartame}
  \begin{boiteCodeTex}{}
\chemfig{!\aspartame}
  \end{boiteCodeTex}
  \chemfig{!\aspartame}
  
  \subsection{Divers}
  \begin{boiteCodeTex}{}
\chemfig{!\bisphenolA}
\chemfig{!\bisphenolASemiDev}
  \end{boiteCodeTex}
  \chemfig{!\bisphenolA} \qq{}
  \chemfig{!\bisphenolASemiDev}
  
  %%%%%%%%%%%%%%%%%%%%%%%%%%%%%%%%%%%%%%%%%%%%%%%%%%%%%%%%%%%%%
  \section{Molécules odorantes}
  \begin{boiteCodeTex}{}
\chemfig{!\geraniol}
\chemfig{!\geraniolSemiDev}
\chemfig{!\vanilline}
\chemfig{!\ethylvanilline}
  \end{boiteCodeTex}
  \chemfig{!\geraniol} \quad
  \chemfig{!\geraniolSemiDev} \quad
  \chemfig{!\vanilline} \quad
  \chemfig{!\ethylvanilline}
  \begin{boiteCodeTex}{}
\chemfig{!\oxyphenylone}
\chemfig{!\limonene}
\chemfig{!\limoneneSemiDev}
\chemfig{!\acetateIsoamyle}
  \end{boiteCodeTex}
  \chemfig{!\oxyphenylone} \quad
  \chemfig{!\limonene} \quad
  \chemfig{!\limoneneSemiDev} \quad
  \chemfig{!\acetateIsoamyle}
  
  %%%%%%%%%%%%%%%%%%%%%%%%%%%%%%%%%%%%%%%%%%%%%%%%%%%%%%%%%%%%%
  \section{Drogues}
  \begin{boiteCodeTex}{}
\chemfig{!\THC}
\chemfig{!\cocaineHaw}
  \end{boiteCodeTex}
  \chemfig{!\THC} \qq{}
  \chemfig{!\cocaineHaw}
  
\end{document}
