%%%%%%%%%%%%%%%%%%%%%%%%%%%%%%%%%%%%%%%%%%%%%%%%%%%%%%%%%%%%%%%%%%%%%%%%%%
%% rectangle coloré
\NewDocumentCommand{\rectangle}{O{couleurPrim} m m}{%
  \shorthandoff{;}
  \tikz \node (rect) [draw, fill, color=#1, minimum width=#2, minimum height=#3] {};
  \shorthandon{;}
}

%% Rectangle adapté à la taille d'un texte
\makeatletter % @ désactivé
\newdimen\@mHauteurBoite
\newdimen\@mLargeurBoite
\newdimen\@mProfondeurBoite
\newdimen\@mTailleBoite
\newcommand{\boiteRectangle}[2][]{
    \settoheight{\@mHauteurBoite}{#2}    % Hauteur du texte
    \settowidth {\@mLargeurBoite}{#2}    % Largeur du texte
    \settodepth {\@mProfondeurBoite}{#2} % Profondeur du texte
    \pgfmathsetlength{\@mTailleBoite}{max(\@mLargeurBoite, (\@mHauteurBoite + \@mProfondeurBoite))}
    \tikz \node [inner sep=2.5\pgflinewidth, minimum width=\@mTailleBoite, minimum height=2\@mHauteurBoite, #1] {#2};
}%
\makeatother
%%

%% Trace un rectangle autour du numéro (#2) et un trait sous le titre (#3)
\makeatletter % @ désactivé
\newdimen\@mHauteurChiffre
\newdimen\@mLargeurChiffre
\newdimen\@mProfondeurChiffre
\newdimen\@mTailleChiffre
\newdimen\@mHauteurTexte
\newdimen\@mLargeurTexte
\newdimen\@mHauteurMax
\newcommand{\boiteTitre}[3][]{
    \settoheight{\@mHauteurChiffre}{#2}    % Hauteur du numéro
    \settowidth {\@mLargeurChiffre}{#2}    % Largeur du numéro
    \settodepth {\@mProfondeurChiffre}{#2} % Profondeur du numéro
    \settoheight{\@mHauteurTexte}{#3}      % Hauteur du texte
    \settowidth {\@mLargeurTexte}{#3}      % Largeur du texte
    \pgfmathsetlength{\@mHauteurMax}{max(\@mHauteurTexte, \@mHauteurChiffre)}
    \pgfmathsetlength{\@mTailleChiffre}{max(\@mLargeurChiffre, (\@mHauteurChiffre + \@mProfondeurChiffre))}
    %% Trace le numéro
    \tikz \node [inner sep=2.5\pgflinewidth, minimum width=\@mTailleChiffre, minimum height=2\@mHauteurMax, rounded corners = 2pt, fill = #1] {#2};
    %% Trace le titre
    \hspace*{-12pt}\noindent \tikz \node [minimum width = \@mLargeurTexte, minimum height = \@mHauteurMax] {#3 };
    %% Trace la boite de fin
    \hspace*{-12pt}\noindent \tikz \fill [y = 2\@mHauteurMax, rounded corners = 2pt, fill = #1] 
      (0, 0) -- (1, 0) -- (1, 0.3) -- (0.2, 1) -- (0, 1);
}%
\makeatother
%%

%%%%%%%%%%%%%%%%%%%%%%%%%%%%%%%%%%%%%%%%%%%%%%%%%%%%%%%%%%%%%%%%%%%%%%%%%%%
%%
\tcbset{
  boite cassable/.style = {
    breakable, enhanced jigsaw, % pour s'étendre sur plusieurs pages
  },
  %
  couleur fond/.style = {
    colback = #1, % fond blanc
    colbacktitle = #1, % fond pour le titre blanc
  },
  %
  titre sans separation/.style = {
    couleur fond = white,
    coltitle = black, % couleur du titre
    colframe = couleurSec-800, % couleur de la boite
    boxrule = #1, arc = 0.5mm, % largeur et arrondi des traits de la boite
    titlerule = 0mm, top = 0mm, % pour ne pas avoir de séparation titre/boite
    fonttitle = \bfseries\sffamily, % titre en gras et sans serif
  },
  %
  boite pleine/.style = {
    frame hidden, sharp corners, boxrule = 0mm, % pas de contours
    colback = #1, % fond
  },
  %
  titre sans boite/.style = {
    empty, % pas de boite automatique
    fonttitle = \bfseries\sffamily, coltitle = black, % paramètre du titre
    attach boxed title to top left = {yshift=-2.5mm}, % position
    boxed title style = {empty, size = small, top = 1mm, bottom = 0.5mm},
    title = #1,
  },
}

%% une simple boite vide
\newtcolorbox{boite}[1][]{
  boite cassable, colback = white, top = 4pt,
  #1
}

%%%% Boite colorée avec 
% \begin{boiteColoree}{couleur} contenu \end{boiteColoree}
\newtcolorbox{boiteColoree}[2][]{
  boite cassable, boite pleine = #2,
  #1
}

%%%% Boite pour les documents des activités, avec le format suivant
% \begin{doc}{titre}{label} contenu \end{doc}
\newcounter{documentNum}
\newtcolorbox{doc}[3][]{
  boite cassable,
  titre sans separation = 0.5mm,
  before title = {\refstepcounter{documentNum}},
  title = {Document \arabic{documentNum} -- #2\strut \label{#3}},
  #1
}

%%%% Boîtes pour les activités et TP pour une séquence en plan de travail
% Boite pour afficher la durée recommandée en bas à gauche
\newtcbox{\dureeActivite}[1][]{
  arc = 2mm, % courbes des bords
  colback = couleurTer-700, colframe = white, % couleurs boite
  coltext = white, % couleur texte
}
% Boite de base pour les activités ou TP, avec le format suivant 
% \begin{boiteActivite}{titre activité}{durée}{label}{compteur}{format titre}
%   contenu
% \end{boiteActivite}
\newtcolorbox{boiteActivite}[6][]{
  boite cassable,
  titre sans separation = 0.5mm,
  before title = {\refstepcounter{#5}},
  title = {#6 \arabic{section}.\arabic{#5} -- #2\strut \label{#4}},
  enlarge bottom by = 12pt,
  overlay= {
    \node at ([xshift = -36pt, yshift = -6pt] frame.south east) {
      \dureeActivite{
        {\small \important[white]{#3}}
      }
    };
  }, % affiche la durée de l'activité dans une petite boite
  remember as = #4,
  #1
}

% Boite pour les activités/TP en plan de travail.
% \begin{activite ou TP}{titre activité}[durée]{label} ; durée = 1h acti, 2h TP par défaut
%   contenu
% \end{activite ou TP}
\NewDocumentEnvironment{activite}{m O{1 h} m}{%
  \begin{boiteActivite}{#1}{#2}{#3}
    {activiteNum}{\documentaire* \hspace{-4pt} Activité}
}{
  \end{boiteActivite}
}
% Compteur pour les TP
\newcounter{TPNum} 
\NewDocumentEnvironment{TP}{m O{2 h} m}{%
  \begin{boiteActivite}{#1}{#2}{#3}
    {TPNum}{\mesure* \hspace{-4pt} TP}
}{
  \end{boiteActivite}
}
% Pour réferencer une activité
\newcommand{\reference}[1]{\arabic{section}.\ref{#1}}

% Boîte pour la tâche finale
\newtcolorbox{tacheFinale}[1][]{
  boite cassable,
  titre sans separation = 0.5mm,
  title = {Tâche finale}, % titre
  #1
}

% Boîte pour l'organisation des séances
\newcounter{seanceNum}
\newtcbox{\seance}[2][]{
  boite cassable,
  titre sans separation = 0.5mm,
  before title = {\refstepcounter{seanceNum}},
  title = {Séance \arabic{seanceNum} \hfill (#2)}, % titre,
  halign=center, valign=center, % pour centrer le contenu
  height = 0.13\textheight, % hauteur de la boite
  #1
}
% Pour afficher 3/2/1 séances dans la programmation
\NewDocumentEnvironment{programmeSeance}{O{3} D(){34}}{
  \begin{tcbraster}[
    raster columns = #1, 
    raster width center = (\linewidth - 3cm - 5cm*(3 - #1))
  ]
}{
  \end{tcbraster}
  \vspace*{#2 pt}
}


%%%% Passage important à connaître
\newtcolorbox{importants}[1][]{
  boite cassable,
  boite pleine = couleurPrim-50, % fond
  shadow = {-4pt}{0mm}{0mm}{couleurPrim}, % barre gauche
  #1
}

%%%% Boîte pour le contexte
\newtcolorbox{contexte}[1][]{
  boite cassable, 
  titre sans separation = 3pt,
  couleur fond = couleurPrim-50!50, 
  colframe = couleurPrim, sharp corners, %  contours
  title = {Contexte :}, % titre
  detach title, before upper={\vspace{2pt}\hspace{-8pt}\tcbtitle\;}, % titre "en ligne"
  #1
}

%%%% Pour les objectifs
\newenvironment{objectifs}{ %
  \begin{listeObjectifs}
} {\end{listeObjectifs}}
%
\tcolorboxenvironment{objectifs}{
  titre sans boite = {Objectifs :},
  frame code = { % tracé de la boite
    \path [draw=couleurPrim, line width = 3pt]
    (frame.west) |- ([xshift=6mm] title.north east)
    to[out=0, in=180] ([xshift=20mm] title.east) -| % définit la courbe
    (frame.east) |- (frame.south) -| cycle; % trace la boite
  },
}

%% Pour les pré-requis
\newenvironment{prerequis}{ %
  \begin{listeObjectifs}
} {\end{listeObjectifs}}
%
\tcolorboxenvironment{prerequis}{
  titre sans boite = {Prérequis :},
  frame code = { % tracé de la boite
    \path [draw=couleurPrim, line width = 3pt]
    (frame.south east) |- ([xshift=-34mm, yshift=-4mm] frame.south east)
    to[out=180, in=0] ([xshift=-47mm] frame.south east) -| % définit la courbe
    (frame.west) |- (title.north) -| cycle; % trace la boite
  },
}

%%%% Espace pour un coup de pouce
\newcounter{coupDePouceNum}
\newtcolorbox{coupDePouce}[1][]{
  boite cassable, 
  titre sans separation = 0.5mm,
  before title = {\refstepcounter{coupDePouceNum}},
  title = {
    \textcolor{couleurPrim}{\faThumbsUp}
    Coup de pouce \arabic{coupDePouceNum} :
    \flushright \vspace*{-24pt}\faSquareO
  },
  #1
}

%%%% Espace pour une appréciation
\newcommand{\appreciation}[1]{
  \pasCorrection{
    \begin{boite}
      \vspace*{-4pt}
      \important[black]{Appréciation et remarques}
      \vspace*{#1 cm} \phantom{b}
    \end{boite}
  }
}

%%%% Espace fiche TP
\newtcolorbox{boiteMateriel}[2][]{
  titre sans separation = 0.5mm,
  coltitle = white, colbacktitle = couleurPrim, % fond pour le titre blanc
  title = {\centering \large \phantom{À} #2 \phantom{g}}, % titre
  #1
}

%%%%%%%%%%%%%%%%%%%%%%%%%%%%%%%%%%%%%%%%%%%%%%%%%%%%%%%%%%%%%%%%%%%%%%%%%%
%%%% pagination et sections
\NewDocumentCommand{\titre}{O{black} m}{
  \begin{center}
    \textcolor{#1} {\textsf{\bfseries \Large #2}}
  \end{center}
}
\newcommand{\pasDePagination}{
  \thispagestyle{empty}
}
\newcommand{\feuilleBlanche}{
  \newpage
  \phantom{b}
  \pasDePagination
}
\NewDocumentCommand{\inclusActivite}{O{1} m}{
  \numeroActivite{#1}
  \input{#2}
}
    

%%%% activité ou TP
\newcounter{activiteNum}
\newcommand{\titreActi}[2]{
  \refstepcounter{activiteNum}
  \titre{#1 \arabic{section}.\arabic{activiteNum} -- #2}
}
\NewDocumentCommand{\titreTP}{s m}{
  \IfBooleanTF{#1}{ % Version *
    \titreActi{Activité expérimentale}{#2}
  }{
    \titreActi{TP}{#2}
  }
}
\NewDocumentCommand{\titreActivite}{s m O{Activité}}{
  \IfBooleanTF{#1}{ % Version * sans le numéro du chapitre
    \refstepcounter{activiteNum}
    \titre{#3 \arabic{activiteNum} -- #2}
  }{
    \titreActi{#3}{#2}
  }
}
\NewDocumentCommand{\titreEvaluation}{o m}{
  \IfNoValueTF{#1}{
    \titre{Évaluation \arabic{section} -- #2}
  }{
    \titre{Évaluation #1 -- #2}
  }
  % reinitialisation du numéro de page et d'exercices
  \reinitialiseCompteur
}
\newcounter{exerciceNum}
\newcommand{\exercice}[1]{
  \refstepcounter{exerciceNum}
  \important[black]{\large Exercice \arabic{exerciceNum} : #1}
  % reset des numéros de questions
  \setcounter{questionNum}{0}
  \setcounter{documentNum}{0}
  \setcounter{sousSectionNum}{0}
}


%%%% chapitre, partie, section et sous-section
\newcommand{\chapitre}[1]{ % Réglage du titre du chapitre
  Chapitre \arabic{section} -- #1
}
\newcommand{\titreChapitre}[1]{ % Affichage du titre du chapitre
  \titre{Chapitre \arabic{section} -- #1}
}
\newcommand{\titrePartie}[1]{
  \vspace*{12pt}
  \refstepcounter{subsection}
  %% Tracé du numéro de partie
  \setul{0.8ex}{0.2ex}
  \setulcolor{couleurPrim}
  \boiteTitre [couleurPrim] {
    \important[white]{\Large \Roman{subsection} -- }
  }{
    \important[black]{\Large\ul{ #1 }}
  }
  \vspace*{1pt}
}
\newcounter{sousSectionNum}
\newcommand{\titreSection}[1]{
  \vspace*{12pt}
  \refstepcounter{subsubsection}
  \setcounter{sousSectionNum}{0}
  %% Tracé du numéro de section
  \boiteRectangle[fill = couleurPrim, rounded corners = 2pt]{
    \important[white]{\large \arabic{subsubsection} -- }
  }
  \hspace*{-20pt}
  %% Tracé du titre de la section
  \boiteRectangle{\important[black]{\large #1 }}
  \vspace*{1pt}
}
\newcommand{\titreSousSection}[1]{
  \vspace*{12pt}
  \refstepcounter{sousSectionNum}
  \important[black]{\Alph{sousSectionNum} -- #1}
  \vspace*{4pt}
}

%%%% fixe le numéro de l'activité
\newcommand{\reinitialiseCompteur}{
    % fixe les compteurs LaTeX
  \setcounter{subsection}{0}
  \setcounter{subsubsection}{0}
  \setcounter{figure}{0}
  % fixe les compteurs internes
  \setcounter{documentNum}{0}
  \setcounter{questionNum}{0}
  \setcounter{coupDePouceNum}{0}
  \setcounter{sousSectionNum}{0}
  \setcounter{seanceNum}{0}
  \setcounter{exerciceNum}{0}
}
\newcommand{\numeroActivite}[1]{
  \reinitialiseCompteur
  \setcounter{activiteNum}{#1 - 1}
}
% fixe le numéro de partie (#1) et le numéro de la page (#2)
\newcommand{\numeroPartieCours}[2]{
  \newpage
  \setcounter{subsection}{#1 - 1}
  \setcounter{page}{#2}
}

%%%% lignes
\newcommand{\ligne}{
  \par\noindent\rule{\textwidth}{0.4pt}
}
\NewDocumentCommand{\lignePointillee}{o}{
  \IfValueTF{#1}{
    \makebox[#1\linewidth]{\dotfill}
  }{
    \phantom{b}\hspace*{-12pt} \dotfill
  }
}


%%%%%%%%%%%%%%%%%%%%%%%%%%%%%%%%%%%%%%%%%%%%%%%%%%%%%%%%%%%%%%%%%%%%%%%%%%
%%%% Paramètre par défaut pour l'en-tête
\newcommand{\classe}{Réglez avec \textbackslash renewcommand\{\textbackslash classe\}\{Seconde\}}
\newcommand{\etablissement}{Réglez avec \textbackslash renewcommand\{\textbackslash etablissement\}\{Lycée\}}

%%%% en-tête
\newcommand{\teteGauche}[2]{
  \lhead{
    \textbf{\footnotesize #1}
    \newline
    \footnotesize #2
  }
}
\NewDocumentCommand{\teteDroite}{m o}{
  \rhead{
    \IfValueT{#2}{
      \hfill \textbf{\footnotesize #2}
    }
    \newline 
    \hfill \footnotesize #1
  }
}
%% \enTete [compteur] {titre} [numéro de section] ; * = version simplifié sans pagination
\NewDocumentCommand{\enTete}{s o m O{0}}{
  % reset des compteurs
  \newpage
  \reinitialiseCompteur
  \setcounter{section}{#4}
  
  \IfBooleanTF{#1}{ % affichage de l'entête version fiche réussite (*)
    \pasDePagination
    \phantom{b} \vspace*{-70pt}
    \titre{#3}
  }{ % affichage de l'entête version activité
    \pagestyle{fancy}
    \teteGauche{\etablissement{}}{#3} % left header
    \chead{} % central header
    \teteDroite{#2} % right header
  }
}


%%%%%%%%%%%%%%%%%%%%%%%%%%%%%%%%%%%%%%%%%%%%%%%%%%%%%%%%%%%%%%%%%%%%%%%%%%
%%%% exercice
% définit un booléen pour entrer ou sortir du mode correction
\newboolean{modeProf}
\setboolean{modeProf}{false}
\newcommand{\modeCorrection}{
  \setboolean{modeProf}{true}
  \TeacherModeOn
}

%% Affiche le numéro d'une question avec choix du compteur et de l'espacement
\NewDocumentCommand{\numeroQuestion}{s O{questionNum} O{16}}{
  \refstepcounter{#2}
  \setcounter{sousQuestionNum}{0}
  \vspace*{2pt}
  \IfBooleanTF{#1}{}{ % cas non étoilé
    \ifnum \thequestionNum > 9
      \hspace{6 pt}
    \else
      \hspace{#3 pt}
    \fi
  }
  \textcolor{couleurSec}{
    \textbf{\arabic{#2}} {\small\faMinus}
  }
}
%% Sous question sour la forme 1.2
\NewDocumentCommand{\numeroSousQuestion}{O{16}}{
  \refstepcounter{sousQuestionNum}
  \ifnum \thesousQuestionNum > 9
    \hspace{6 pt}
  \else
    \hspace{#1 pt}
  \fi
  \textcolor{couleurSec}{
    \textbf{\arabic{questionNum}.\arabic{sousQuestionNum}.}
  }
}


%% trace des lignes pointillées pour répondre aux questions
% \lignesDeReponse* complète la ligne actuelle par des pointillées
% \lignesDeReponse commence à la ligne suivante
\newcounter{ligneNum}
\NewDocumentCommand{\lignesDeReponse}{s m}{
  % Trace la fin de la ligne, ou pas
  \IfBooleanTF{#1}{ % Version *
    \espaceReponse \dotfill\phantom{bb}
    \ifnum #2 < 1
      \newline
    \fi
  }{}
  % Trace le bon nombre de lignes
  \setcounter{ligneNum}{-1}
  \loop
    \stepcounter{ligneNum}
    \ifnum \value{ligneNum} < #2
      \\[8pt] \lignePointillee
  \repeat
  \vspace*{1pt}
}


%% définit une commande pour afficher une question 
% \qestion {question} {réponse} [nombre de lignes] ; * = pas d'alinéa
\newcounter{questionNum}
\newcounter{sousQuestionNum}
\NewDocumentCommand{\question}{s +m +m O{0}}{
  \IfBooleanTF{#1}{ % étoile
    \numeroQuestion* \!#2
  }{ % 
    \numeroQuestion \!#2
  }
  % pointille ou correction
  \ifthenelse {\boolean{modeProf}} { % prof
    \begin{boiteColoree}{couleurPrim-50}
      #3
    \end{boiteColoree}
  }{ % eleve
    \lignesDeReponse{#4}
  }
}

% Affiche le contenu en mode correction
\newcommand{\correction}[1]{
  \ifthenelse {\boolean{modeProf}} { % correction
    #1
  }{}
}

% Affiche le contenu si on est pas en mode correction
\newcommand{\pasCorrection}[1]{
  \ifthenelse{\boolean{modeProf}} {}{ % pas correction
    #1
  }
}

% Point associé à une question
\newcommand{\points}[1]{
  \marginnote{#1}
}


% sous questions
\newcommand{\sousQuestion}[2]{
  \hspace{16pt}
  \textcolor{couleurSec}{\textbullet} #1
  
  \vspace*{8pt}
  \reponse{#2}
}

%%%% qcm
\newlist{QCM}{itemize}{2}
\setlist[QCM]{label = $\square$, leftmargin = 2cm}
%% #1 : question, 
%% #2 : réponses
\NewDocumentEnvironment{qcm}{+m +m}{
  \numeroQuestion #1
  \begin{QCM}
    #2
}{
  \end{QCM}
}

% À ajouter devant la bonne réponse dans un qcm
\newcommand{\reponseQCM}{
  \correction{
    \hspace*{-21pt}
    {\large\textcolor{couleurSec}{\faCheck}}
    \hspace*{-12pt}
  } % Note : trace une croix à la bonne position
}

%%%% Pour afficher les compétences
\newcommand{\competence}[1]{
  ~{\footnotesize\textit{(#1)}}
}

%%%% Espace pour indiquer nom, prénom et classe
\newcommand{\nomPrenomClasse}{
  \setcounter{page}{1}
  \pasCorrection{
    \vspace*{-24pt}
    Nom : \lignePointillee[0.3]
    Prénom : \lignePointillee[0.3]
    Classe : \dotfill
  }
}
\newcommand{\nomPrenom}{
  \pasCorrection{
    \vspace*{-24pt}
    Nom : \lignePointillee[0.3]
    Prénom : \lignePointillee[0.3]
  }
}


%%%%%%%%%%%%%%%%%%%%%%%%%%%%%%%%%%%%%%%%%%%%%%%%%%%%%%%%%%%%%%%%%%%%%%%%%%
% texte à trou avec option pour régler la largeur
\NewDocumentCommand{\texteTrou}{s o +m}{
  \ifthenelse {\boolean{modeProf}}{ % prof
    \IfBooleanTF{#1}
    {#3}
    {\important[black]{#3}}
  }{ % élève
    \IfValueTF{#2}{ % Si la largeur est réglée, on utilise des lignes
      \espaceReponse
      \lignePointillee[#2]
      \hspace*{-12pt}
    }{ % Sinon on utilise dash undergap pour la version automatique
      \espaceReponse \hspace*{0.1pt}
      \gap{#3}
    }
  }
}

% texte à trou avec option pour laisser plusieurs lignes
\NewDocumentCommand{\texteTrouLignes}{O{0} +m}{
  \ifthenelse {\boolean{modeProf}} {% prof
    \important[black]{#2}
  }{% élève
    \lignesDeReponse*{#1}
  }
}

% espace vertical pour la réponse
\newcommand{\espaceReponse}{
  \phantom{$\dfrac{1}{1}$} % espace vertical
  \hspace*{-38pt} \phantom{b} % ajuste l'espace horizontal
}


%%%%%%%%%%%%%%%%%%%%%%%%%%%%%%%%%%%%%%%%%%%%%%%%%%%%%%%%%%%%%%%%%%%%%%%%%%
%%%% Pour choisir parmi deux sujets
\newboolean{sujetA}
\setboolean{sujetA}{true}
\newcommand{\sujetB}{
  \setboolean{sujetA}{false}
}
\newcommand{\sujetA}{
  \setboolean{sujetA}{true}
}

%%%% Pour faire plusieurs sujets en parallèle
\newcommand{\variationSujet}[2]{
  \hspace*{-6pt}
  \ifthenelse{\boolean{sujetA}}{#1}{#2}
  \hspace*{-6pt}
}


%%%%%%%%%%%%%%%%%%%%%%%%%%%%%%%%%%%%%%%%%%%%%%%%%%%%%%%%%%%%%%%%%%%%%%%%%%
%%%% Tableau générique avec la première ligne bleue
\NewDocumentEnvironment{tableau}{m}{
  \begin{center}
    \begin{tblr}{
      hlines,
      colspec = #1,
      row{1} = {couleurSec-100},
    }
}{
    \end{tblr}
  \end{center}
}

%%%% Tableau de competence
\newenvironment{tableauCompetences}{
  \begin{center}
    \begin{tblr}{
      colspec = {c X[l] c c c c},
      rows = {m}, hlines, vlines,
      row{1} = {c, couleurSec-100, font = \bfseries}
    }
      Comp. & Items & D & C & B & A \\
}{
    \end{tblr}
  \end{center}
}

%%%% Tableau de connaissances pour les fiches de révisions
\newenvironment{tableauConnaissances}{
  \begin{center}
  \begin{tblr}{
    colspec = {Q[t, wd=0.7\textwidth] c c c},
    rows = {m}, hlines, vlines,
    column{4} = {0.2},
    row{1} = {couleurSec-100, c}
  }
    \important{Connaissances et capacités exigibles} & \ok & \pasOk & \important{En classe} \\
}{ 
  \end{tblr}
  \end{center}
}

%%%% Tableau de mémorisation pour les fiches de mémorisation
% Question | | | Réponse
\newenvironment{tableauMemorisation}{
  \begin{center}
 \begin{tblr}{
    colspec = {
      Q[l, wd=0.3\textwidth] % Question
      X[c] X[c] % Auto-évaluation
      Q[l, wd=0.3\textwidth] % Réponse
      Q[c, wd=0.045\textwidth] Q[c, wd=0.045\textwidth] Q[c, wd=0.045\textwidth] % Répétitions
    }, rows = {m}, hlines, vlines,
    row{1,2} = {couleurSec-100, c}
  }
    \SetCell[r=2]{c} \important{Questions} & 
    \SetCell[c=2]{c} \important{Auto-évaluation} & &
    \SetCell[r=2]{c} \important{Réponses} & 
    \SetCell[r=2]{c} \important{J-7} &
    \SetCell[r=2]{c} \important{J-30} &
    \SetCell[r=2]{c} \important{J-180} \\
    %
    & % questions
    \ok & \pasOk & % auto evaluation
    & % réponses 
    & & \\ % répétitions
    %
}{ 
  \end{tblr}
  \end{center}
}


%%%% Alignement vertical dans un tableau
\newcommand{\vAligne}[1]{
  \strut \\ \vspace*{#1}
}


%%%%%%%%%%%%%%%%%%%%%%%%%%%%%%%%%%%%%%%%%%%%%%%%%%%%%%%%%%%%%%%%%%%%%%%%%%
%%%% symboles : chevron, flèche, attention, etc.
\NewDocumentCommand{\chevron}{O{couleurPrim}}{
  \textcolor{#1}{\small \faChevronRight}
}
%
\NewDocumentCommand{\fleche}{O{couleurPrim}}{
  \textcolor{#1}{\faCaretRight}
}
%
\NewDocumentCommand{\attention}{O{couleurPrim}}{
  \textcolor{#1}{\faExclamationTriangle}
}
%
\NewDocumentCommand{\flecheLongue}{O{couleurPrim}}{
  \textcolor{#1}{\faLongArrowRight}
}
%
\NewDocumentCommand{\ok}{O{couleurPrim}}{
  \textcolor{#1}{\faCheckCircle}
}
%
\NewDocumentCommand{\pasOk}{O{couleurPrim}}{
  \textcolor{#1}{\faTimesCircle}
}
%
\NewDocumentCommand{\pointCyan}{O{couleurPrim}}{
  \textcolor{#1}{\textbullet}
}

%% Commande générale pour colorer un texte avec une couleur par défaut
% L'intérêt de cette commande est d'éviter des répétion de O{couleurSec}
\NewDocumentCommand{\questionSpeciale}{O{couleurSec} m}{
  \textcolor{#1}{#2}
}
% Quelques questions spéciales avec couleur réglable et un espace horizontal
\NewDocumentCommand{\mesure}{s o}{
  \IfBooleanTF{#1}{}{ \hspace{7pt} }
  \questionSpeciale[#2]{\faFlask\hspace{1pt} \faWrench\!}
}
%
\NewDocumentCommand{\telechargement}{s o}{
  \IfBooleanTF{#1}{}{ \hspace{7pt} }
  \questionSpeciale[#2]{\faDownload\, \faMobile\;}
}
%
\NewDocumentCommand{\documentaire}{s o}{
  \IfBooleanTF{#1}{}{ \hspace{7pt} }
  \questionSpeciale[#2]{\faFileText\hspace{1pt} \faBook}
}

%% pictogramme de sécurité \picto {largeur} {pictogramme}
\newcommand{\picto}[2]{
  \image{#1}{images/pictogrammes/picto_#2}
}
% Nombre dans un cercle A REVOIR
\newcommand*\nombreCercle[1]{
  % \tikz[baseline=(char.base)]{
  %   \node [shape=circle, draw filled, inner sep=1.2pt, color=couleurSec-50] (char) {\textcolor{black}{#1};
  % }
  \important[couleurSec]{#1}
}
% Pour légender une image
\newcommand{\legende}[1]{
  \vspace*{4pt}
  \textcolor{couleurPrim}{\faArrowUp} \; #1
}

%%%%%%%%%%%%%%%%%%%%%%%%%%%%%%%%%%%%%%%%%%%%%%%%%%%%%%%%%%%%%%%%%%%%%%%%%%
%%%% points importants
\NewDocumentCommand{\important}{O{couleurSec-800} +m}{
  \!\textcolor{#1}{\textsf{\bfseries #2}}\!\!
}
%%%% Pour donner des exemples. 
% \exemple est au singulier, \exemple* est au pluriel.
\NewDocumentCommand{\exemple}{s}{
  \fleche
  \IfBooleanTF{#1}{
    \textit{Exemples :}
  }{
    \textit{Exemple :}
  }
}

%%%% Citation (#1) avec la source (#2)
\newcommand{\extrait}[2]{
  « #1 »
  
  \vspace*{-12pt}
  \begin{flushright}
    \textit{#2}
  \end{flushright}
  \vspace*{-12pt}
}
\newcommand{\sourceExtrait}[1]{
  \vspace*{-12pt}
  \begin{flushright}
    \textit{#1}
  \end{flushright}
  \vspace*{-12pt}
}

%%%% image avec la largeur réglée par rapport à celle de la ligne
\newcommand{\image}[2]{
  \includegraphics[width=#1\linewidth]{#2}
}

%%%% qr code en insert sur la droite
\NewDocumentCommand{\qrcodeCote}{o +m D(){-16pt} O{1.5cm}}{
  \IfNoValueTF{#1} {
    \begin{wrapfigure}{r}{0.1\linewidth}
      \vspace*{#3}
      \qrcode[height = #4]{#2}
    \end{wrapfigure}
  }{
    \begin{wrapfigure}[#1]{r}{0.1\linewidth}
      \vspace*{#3}
      \qrcode[height = #4]{#2}
    \end{wrapfigure}
  }
}


%%%%%%%%%%%%%%%%%%%%%%%%%%%%%%%%%%%%%%%%%%%%%%%%%%%%%%%%%%%%%%%%%%%%%%%%%%
%%%% protocole, avec plusieurs colonnes possibles
\NewDocumentEnvironment{protocole}{o +m}{
  \IfValueTF{#1}{
    \vspace*{-8pt}
    \begin{multicols}{#1}
  }{}
  \begin{itemize}[label = {\footnotesize \fleche[couleurSec]}]
    #2
}{
  \end{itemize}
  \IfValueTF{#1}{ \end{multicols} }{}
}

%%%% liste de points, avec plusieurs colonnes possibles
\NewDocumentEnvironment{listePoints}{o +m}{
  \IfValueTF{#1}{
    \vspace*{-8pt}
    \begin{multicols}{#1}
  }{}
  \begin{itemize}[label = \pointCyan]
    #2
}{
  \end{itemize}
  \IfValueTF{#1}{ \end{multicols} }{}
}

%%%% jeu de données, avec plusieurs colonnes possibles
\NewDocumentEnvironment{donnees}{o +m}{  
  \important{Données :}
  \IfValueTF{#1}{
    \vspace*{-8pt}
    \begin{multicols}{#1}
  }{}
  \begin{listeTirets}
    #2
}{
  \end{listeTirets}
  \IfValueTF{#1}{ \end{multicols} }{}
}

%%%% liste d'objectif
\newlist{listeObjectifs}{itemize}{2}
\setlist[listeObjectifs]{leftmargin=6pt, label=\fleche}

%%%% liste tirets
\newlist{listeTirets}{itemize}{2}
\setlist[listeTirets]{label = \textcolor{couleurPrim}{\small\faMinus}}

%%%% liste avec des flèches
\newlist{listeFleche}{itemize}{2}
\setlist[listeFleche]{label = \textbf{\flecheLongue}}

%%%% liste avec chiffre
\newlist{enumeration}{enumerate}{2}
\setlist[enumeration]{label = \textcolor{couleurSec}{\textbf{\arabic*.}} }

%%%% problématique 
\newcommand{\problematique}[1]{
  \hspace*{-12pt}
  \flecheLongue[couleurTer]
  \hspace*{-10pt}
  \important[couleurTer]{#1}
}


%%%%%%%%%%%%%%%%%%%%%%%%%%%%%%%%%%%%%%%%%%%%%%%%%%%%%%%%%%%%%%%%%%%%%%%%%%
%%%% Séparation de la page en trois blocs
\NewDocumentCommand{\separationTroisBlocs}{+m O{0.3} +m O{0.3} +m O{0.3}}{
  \begin{minipage}[T]{#2\linewidth}
    #1
  \end{minipage}
  ~
  \begin{minipage}[T]{#4\linewidth}
    #3
  \end{minipage}
  ~
  \begin{minipage}[T]{#6\linewidth}
    #5
  \end{minipage}
}
%%%% Separation en deux blocs
\NewDocumentCommand{\separationBlocs}{+m O{0.48} +m O{0.48}}{
  \begin{minipage}[T]{#2\linewidth}
    #1
  \end{minipage}
  \hfill
  \begin{minipage}[T]{#4\linewidth}
    #3
  \end{minipage}
}


%%%%%%%%%%%%%%%%%%%%%%%%%%%%%%%%%%%%%%%%%%%%%%%%%%%%%%%%%%%%%%%%%%%%%%%%%%
%% nombre algébrique et réaction chimique
\newcommand{\algebrique}[1]{
  \ensuremath{\overline{\mathrm{#1}}}
}
\newcommand{\reaction}{
  \;\text{\faLongArrowRight}\; % flèche courte et jolie
}

%% Pour simplifier l'écriture des formules brutes
% #1 : # carbone, #2 : # hydrogène, #3 : oxygène
\newcommand{\bruteCHO}[3]{
  \chemfig{C_{#1} H_{#2} O_{#3}}
}

%% pour les masse molaire et atomique (* -> en indice)
\NewDocumentCommand{\masseMol}{s m}{
  \IfBooleanTF{#1}{
    \ensuremath{M_{\chemfig{#2}}}
  }{
    \ensuremath{M(\chemfig{#2})}
  }
}
\NewDocumentCommand{\masseAtom}{s m}{
  \IfBooleanTF{#1}{
    \ensuremath{m_{\chemfig{#2}}}
  }{
    \ensuremath{m(\chemfig{#2})}
  }
}


%% Unités pour siunit
\DeclareSIUnit{\dioptre}{\text{$\delta$}}
\DeclareSIUnit{\dornic}{\text{\textdegree D}}
\DeclareSIUnit{\ppm}{\text{ppm}}
\DeclareSIUnit{\COeq}{\text{kgCO$_{2}$e}}
\DeclareSIUnit{\jour}{\text{jour}}
\DeclareSIUnit\cal{\text{cal}}
\DeclareSIUnit\kcal{\kilo\cal}
% \DeclareSIUnit{}{\text{}}


%%%% atome ou isotope 
%#1: Z, #2: A, #3: X
\makeatletter
\newcommand{\isotope}[3]{%
   \settowidth\@tempdimb{\ensuremath{\scriptstyle#1}}%
   \settowidth\@tempdimc{\ensuremath{\scriptstyle#2}}%
   \ifnum\@tempdimb>\@tempdimc%
       \setlength{\@tempdima}{\@tempdimb}%
   \else%
       \setlength{\@tempdima}{\@tempdimc}%
   \fi%
  \begingroup%
  \ensuremath{
    ^{\makebox[\@tempdima][r]{\ensuremath{\scriptstyle#1}}}
    _{\makebox[\@tempdima][r]{\ensuremath{\scriptstyle#2}}}
    \chemfig{#3}
  }%
  \endgroup%
}%
\makeatother

%% element chimique dans le tableau périodique
\makeatletter
\newcommand{\element}[2]{%
   \settowidth\@tempdimb{\ensuremath{\footnotesize #1}}%
  \begingroup%
  \ensuremath{
    _{\makebox[\@tempdimb][r]{\ensuremath{\small #1}}} 
    \chemfig[atom style={scale=1.3}]{#2}
  }%
  \endgroup%
}%
\makeatother

%% siècle
\newcommand{\siecle}[1]{
  \textsc{\romannumeral #1}\textsuperscript{e}~siècle%
}

%% texte avec une boite autour
\NewDocumentCommand{\texteEncadre}{m O{black}}{
  \textcolor{#2}{
    \frame{
      \vphantom{$\dfrac{1}{1}$} \textcolor{black}{\text{#1}}
    }
  }
}

%% case cochée
\newcommand{\caseCochee}{
  $\text{\rlap{$\checkmark$}}\square$
}


%%%%%%%%%%%%%%%%%%%%%%%%%%%%%%%%%%%%%%%%%%%%%%%%%%%%%%%%%%%%%%%%%%%%%%%%%%
%%%% Style python
\lstdefinestyle{codePython}{
  commentstyle=\color{magenta-500},
  keywordstyle=\color{green-500},
  stringstyle =\color{purple-500},
  numberstyle =\tiny\color{black!50},
  basicstyle  =\ttfamily\footnotesize,
  breakatwhitespace=false, breaklines=true, keepspaces=true,
  showspaces=false, showstringspaces=false, showtabs=false, tabsize=2,
  captionpos=b, numbers=left, numbersep=5pt,
  extendedchars=true,
  literate={é}{{\'e}}1 {è}{{\`e}}1 {à}{{\'a}}1 {°}{{\textdegree}}1 {²}{{$^2$}}1, 
}
\def\inline{\lstinline[style=codePython,language=python]}


%%%%%%%%%%%%%%%%%%%%%%%%%%%%%%%%%%%%%%%%%%%%%%%%%%%%%%%%%%%%%%%%%%%%%%%%%%
%%%% circuit tikz
\NewDocumentCommand{\fixedvlen}{O{0.5cm} m m O{}}{% [semilength]{node}{label}[extra options]
  % get the center of the standard arrow
  \coordinate (#2-Vcenter) at ($(#2-Vfrom)!0.5!(#2-Vto)$);
  % draw an arrow of a fixed size around that center and on the same line
  \draw[-Triangle, #4] ($(#2-Vcenter)!#1!(#2-Vfrom)$) -- ($(#2-Vcenter)!#1!(#2-Vto)$);
  % position the label as in the normal voltages
  \node[anchor=\ctikzgetanchor{#2}{Vlab}, #4] at (#2-Vlab) {#3};
}