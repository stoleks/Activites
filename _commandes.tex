%%%%%%%%%%%%%%%%%%%%%%%%%%%%%%%%%%%%%%%%%%%%%%%%%%%%%%%%%%%%%%%%%%%%%%%%%%
%%%% couleurs
\definecolor{darkgreen}{rgb}{0.0, 0.40, 0.10}
\definecolor{darkcyan}{rgb}{0.0, 0.55, 0.50}
\definecolor{darkred}{RGB}{148, 31, 0}
\definecolor{brightyellow}{RGB}{238, 193, 0}
\definecolor{brightred}{RGB}{224, 39, 34}

%%%% rectangle coloré
\newcommand\rectangle[3]{%
  \shorthandoff{;}
  \tikz \node (rect) [draw, fill, color=#1,
              minimum width=#2,
              minimum height=#3] {};
  \shorthandon{;}
}
\newcommand\rectangleCyan[2]{\rectangle{darkcyan}{#1}{#2}}

%%%% simple boite
\newenvironment{boite}{
  \begin{tcolorbox}
  [ breakable, enhanced jigsaw, % to break box over page
    arc = 0mm, % straight line
    colback= white, % white background
    colframe= black % dark frame
  ]
}
{ \end{tcolorbox} }


%%%%%%%%%%%%%%%%%%%%%%%%%%%%%%%%%%%%%%%%%%%%%%%%%%%%%%%%%%%%%%%%%%%%%%%%%%
%%%% pagination et sections
\newcommand{\titre}[1]{
  \begin{center}
    \textsf{\bfseries \Large #1}
  \end{center}
}
\newcommand{\sousTitre}[1]{
  \textsf{\bfseries #1}
}
\newcommand{\pasDePagination}{
  \thispagestyle{empty}
}
\newcommand{\feuilleBlanche}{
  \newpage
  \phantom{b}
  \pasDePagination
}


%%%% activité ou TP
\newcounter{acti}
\newcommand{\titreActi}[2]{
  \refstepcounter{acti}
  \titre{#1 \arabic{section}.\arabic{acti} -- #2}
}
\newcommand{\titreTP}[1]{
  \titreActi{Activité expérimentale}{#1}
}
\newcommand{\titreActivite}[1]{
  \titreActi{Activité}{#1}
}
\newcommand{\titreEvaluation}[1]{
  \titreActi{Évaluation}{#1}
}


%%%% chapitre, section et sous-section
\newcommand{\titreChapitre}[1]{
  \titre{Chapitre \arabic{section} : #1}
}
\newcommand{\titrePartie}[1]{
  \vspace*{24pt}
  \refstepcounter{subsection}
  \rectangleCyan{60pt}{1pt}
  \sousTitre{\Large \Roman{subsection} -- #1}
  \rectangleCyan{60pt}{1pt}
  \vspace*{10pt}
}
\newcounter{sousSection}
\newcommand{\titreSection}[1]{
  \vspace*{16pt}
  \refstepcounter{subsubsection}
  \setcounter{sousSection}{0}
  \rectangleCyan{30pt}{4pt}
  \sousTitre{\large \arabic{subsubsection} -- #1}
  \vspace*{10pt}
}
\newcommand{\titreSousSection}[1]{
  \vspace*{12pt}
  \refstepcounter{sousSection}
  \sousTitre{\Alph{sousSection} -- #1}
  \vspace*{8pt}
}

%%%% fixe le numéro de l'activité
\newcommand{\numeroActivite}[1]{
  \setcounter{subsection}{0}
  \setcounter{subsubsection}{0}
  \setcounter{sousSection}{0}
  \setcounter{figure}{0}
  \setcounter{document}{0}
  \setcounter{exercice}{0}
  \setcounter{QCM}{0}
  \setcounter{acti}{#1 - 1}
  \setcounter{page}{1}
}
% fixe le numéro de page (#2) et de partie (#1)
\newcommand{\numeroPartieCours}[2]{
  \newpage
  \setcounter{subsection}{#1 - 1}
  \setcounter{page}{#2}
}

%%%% lignes
\newcommand{\ligne}{
  \par\noindent\rule{\textwidth}{0.4pt}
}
\newcommand{\lignePointillee}[1]{
  \makebox[#1\textwidth]{\dotfill}
}


%%%%%%%%%%%%%%%%%%%%%%%%%%%%%%%%%%%%%%%%%%%%%%%%%%%%%%%%%%%%%%%%%%%%%%%%%%
%%%% en-tête
\newcommand{\teteGauche}[2]{
  \lhead{
    \textbf{\footnotesize #1}
    \newline
    \footnotesize #2
  }
}
\newcommand{\teteDroite}[2]{
  \rhead{
    \hfill \textbf{\footnotesize #1}
    \newline \hfill
    \footnotesize #2
  }
}
\newcommand{\enTete}[2]{
  \pagestyle{fancy}
  \setcounter{section}{#2}
  \setcounter{subsection}{0}
  \setcounter{sousSection}{0}
  \teteGauche{Lycée Condorcet}{Chapitre \arabic{section} -- #1} % left header
  \chead{} % central header
  \teteDroite{2022 -- 2023}{Seconde}
}
%
\newcommand{\enTeteFicheReussite}[1]{
  \newpage
  \setcounter{subsection}{0}
  \pasDePagination
  
  \phantom{b}
  \vspace*{-70pt}
  \titre{Fiche \og Réussir son évaluation \fg}
  \titre{#1}
  
  \vspace*{-20pt}
  \titreSection{Ce que je dois savoir}
  
  Pour savoir quoi réviser, je lis les points clés du chapitre évalués :
  \begin{itemize}
    \item Si je pense maîtriser une notion, je coche la case \ok
    \item Si je pense que je dois la retravailler, je coche la case \pasOk
  \end{itemize}
  
  Pour travailler les notions qui ne sont pas maîtrisées, je reprend les activités associés.
}
\newcommand{\basDePageFicheReussite}{
  \titreSection{Ce qu'il me reste à faire}
  
  Pour être sûr-e d'obtenir une bonne note, je m'entraîne avec les exercices corrigés du manuel indiqués dans la colonne de droite.
}
\newcommand{\questionFicheReussite}[1]{
  S'il me reste des questions, je les note ici pour les poser au début de l'évaluation :\\[4pt]
  \reponse{#1}
}
\newcommand{\coursFicheReussite}{
  Je prépare une fiche au format A4 avec toutes les notions, définitions ou grandeurs dont je pense avoir besoin pendant l'évaluation.
}


%%%%%%%%%%%%%%%%%%%%%%%%%%%%%%%%%%%%%%%%%%%%%%%%%%%%%%%%%%%%%%%%%%%%%%%%%%
%%%% exercice
% définit un booléen pour entrer ou sortir du mode correction
\newboolean{modeProf}
\setboolean{modeProf}{false}
\newcommand{\modeCorrection}{
  \setboolean{modeProf}{true}
}

% définit une commande pour afficher une question 
% #1 : question
% #2 : réponse
% #3 : nombres de lignes pour répondre
\newcommand{\question}[3]{
  \numeroQuestion #1
  
  % pointille ou correction
  \ifthenelse {\boolean{modeProf}} { % prof
    \correction{#2}
  }{ % eleve
    \vspace*{8pt}
    \lignesReponse{#3}
  }
}

%
\newcounter{exercice}
\newcommand{\numeroQuestion}{
  \refstepcounter{exercice}
  \vspace*{2pt}
  \hspace{15pt}
  \textcolor{darkcyan!75!black}{
    \textbf{\arabic{exercice}} {\small\faMinus}
  }
}

% correction
\newcommand{\correction}[1]{
  \vspace*{2pt}
  \hspace{16pt}
  \textcolor{darkcyan!60!black}{
    \faCaretRight \hspace{2pt} #1
  }
}

% trace des lignes pour répondre
\newcounter{int}
\newcommand{\lignesReponse}[1]{
  \setcounter{int}{-1}
  \loop
    \stepcounter{int}
    \ifnum \value{int} < #1
    \lignePointillee{0.99} \\[8pt]
  \repeat
  \ifnum #1 > 0
    \vspace*{-12pt}
  \fi
}

% sous questions
\newcommand{\sousQuestion}[2]{
  \hspace{15pt}
  \textcolor{darkcyan!75!black}{\textbullet} #1
  
  \vspace*{8pt}
  \reponse{#2}
}


%%%% QCM
% question QCM
\newcounter{QCM}
\newcommand{\QCM}[2]{
  \refstepcounter{QCM}
  \vspace*{2pt}
  \textcolor{darkcyan!75!black}{\textbf{\arabic{QCM}} {\small\faMinus}} #1
  
  \begin{qcm}
    #2
  \end{qcm}
}


%%%% Pour afficher les compétences
\newcommand{\competence}[1]{
  { \footnotesize \textit{(#1)} }
}


%%%% Espace pour indiquer nom, prénom et classe
\newcommand{\nomPrenomClasse}{
  \vspace*{-24pt}
  Nom : \lignePointillee{0.3}
  Prénom : \lignePointillee{0.3}
  Classe : \dotfill
}


%%%%%%%%%%%%%%%%%%%%%%%%%%%%%%%%%%%%%%%%%%%%%%%%%%%%%%%%%%%%%%%%%%%%%%%%%%
% texte à trou dans une ligne
\newcommand{\texteTrou}[2]{
  \ifthenelse {\boolean{modeProf}} {% prof
    \textcolor{darkcyan!60!black}{ \textbf{#1} }
  }{% élève
    \espaceReponse
    \lignePointillee{#2}
  }
}

% texte à trou sur plusieurs lignes
\newcommand{\texteTrouMultiLignes}[2]{
  \ifthenelse {\boolean{modeProf}} {% prof
    \textcolor{darkcyan!60!black}{ \textbf{#1} }
  }{% élève
    \reponseLigne
    \lignesReponse{#2}
  }
}

% reponse en fin de ligne
\newcommand{\reponseLigne}{
  \espaceReponse
  \dotfill \\[8pt]
}

% espace vertical pour la réponse
\newcommand{\espaceReponse}{
  \phantom{$\Frac{1}{1}$} % espace vertical
  \hspace*{-40pt} \phantom{b} % ajuste l'espace horizontal
}


%%%%%%%%%%%%%%%%%%%%%%%%%%%%%%%%%%%%%%%%%%%%%%%%%%%%%%%%%%%%%%%%%%%%%%%%%%
%%%% documents
\newcounter{document}
\newcommand{\titreDocu}[1]{
  \refstepcounter{document} % update counter
  \textbf{Document \arabic{document} -- #1} % print document title
  \addcontentsline{toc}{document}{\protect\numberline{} #1} % update table of content
}
\newenvironment{doc}[1]{
  \begin{boite}
    \titreDocu {#1} \newline
}
{ \end{boite} }

%% pour les référencer
\newcommand{\docu}[1]{
  \textbf{(Doc. #1)}
}


%%%% problematique
\newcommand{\problematique}[1]{
  \hspace{8pt}
  \flecheLongue
  \textbf{#1}
}


%%%% objectifs
\newenvironment{objBoite}{
  \begin{tcolorbox}
  [boxrule = 0pt,
  frame hidden, sharp corners,
  colback = white, enhanced,
  borderline north={2pt}{0pt}{darkcyan},
  borderline south={2pt}{0pt}{darkcyan},
  borderline west={4pt}{0pt}{darkcyan},
  borderline east={4pt}{0pt}{darkcyan}
  ]
  
}
{
  \end{tcolorbox}
}
\newenvironment{objectifs}{
  \begin{objBoite}
    \begin{center}
      \sousTitre{\large Objectifs de la séance :}
      \begin{listeObjectifs}
}
{ 
      \end{listeObjectifs}
    \end{center}
  \end{objBoite}
}

%%%% Espace pour un coup de pouce
\newenvironment{coupDePouce}{
  \vspace*{-12pt}
  \begin{boite}
    \begin{flushright}
      \textcolor{darkcyan}{\faSquareO}
    \end{flushright}
    
    \vspace*{-30pt}
    \textcolor{darkcyan}{\faThumbsUp}
    \textbf{Coup de pouce} :\\[4pt]
}
{
  \end{boite}
}


%%%% Espace pour une appréciation
\newcommand{\appreciation}[1]{
  \vspace*{8pt}
  \sousTitre{Appréciation et remarques}
  \vspace*{-8pt}
  \begin{boite}
    \vspace{#1 pt}
    \phantom{b}
  \end{boite}
}


%%%%%%%%%%%%%%%%%%%%%%%%%%%%%%%%%%%%%%%%%%%%%%%%%%%%%%%%%%%%%%%%%%%%%%%%%%
%%%% Tableau de competence
\newenvironment{tableauCompetences}{
  \centering
  \setlength{\extrarowheight}{6pt}
  \tabularx{\linewidth}{| c | X | c | c | c | c |}
    \hline
    \rowcolor{gray!20}
    \centering \textbf{Compétences} & \centering \textbf{Items} & \textbf{D} & \textbf{C} & \textbf{B} & \textbf{A}
    \\ \hline
}
{
    \\ \hline
  \endtabularx
}

%%%% Tableau de connaissances
\newenvironment{tableauConnaissances}{
  \centering
  \setlength{\extrarowheight}{6pt}
  \tabularx{\linewidth}{| X | c | c | m{0.18\linewidth} | m{0.15\linewidth} |}
    \hline
    \rowcolor{gray!20} 
    \textbf{Points clés du chapitre} & \ok & \pasOk & \textbf{En classe} & \textbf{Exercices}
    \\ \hline
}
{   
    \\ \hline
  \endtabularx
}

%%%% Tableau de correction élève
\newcommand{\correctionEleve}[1]{
  \setlength{\extrarowheight}{6pt}
  \begin{tabularx}{\linewidth}{| X | m{0.26\textwidth} | m{0.26\textwidth} | m{0.26\textwidth} |}
    \hline
    \rowcolor{gray!20} 
    \centering \textbf{Question} &
    \centering \textbf{L'erreur} &
    \centering \textbf{Analyse de l'erreur} &
    \phantom{Bla} \textbf{La correction}
    \\ \hline
    %
    \phantom{b} \vspace{#1 pt}
    & & & \\ \hline
    \phantom{b} \vspace{#1 pt}
    & & & \\ \hline
    \phantom{b} \vspace{#1 pt}
    & & & \\ \hline
    \phantom{b} \vspace{#1 pt}
    & & & \\ \hline
  \end{tabularx}
}

%%%% Tableau bilan de la correction
\newcommand{\bilanCorrection}[1]{
  \setlength{\extrarowheight}{6pt}
  \begin{tabularx}{\linewidth}{| X | X |}
    \hline
    \rowcolor{gray!20}
    \textbf{Ce que je n'avais pas compris...} &
    \textbf{Ce que maintenant j'ai compris...}
    \\ \hline
    \phantom{b} \vspace{#1 pt} % \newline \newline \newline
    & \\ \hline
  \end{tabularx}
}


%%%%%%%%%%%%%%%%%%%%%%%%%%%%%%%%%%%%%%%%%%%%%%%%%%%%%%%%%%%%%%%%%%%%%%%%%%
%%%% symboles : chevron, flèche, attention, etc.
\newcommand{\chevron}{
  \textcolor{darkcyan}{\small \faChevronRight}
}
%
\newcommand{\fleche}{
  \textcolor{darkcyan}{\faCaretRight}
}
%
\newcommand{\attention}{
  \textcolor{darkcyan}{\faExclamationTriangle}
}
%
\newcommand{\flecheLongue}{
  \textcolor{darkcyan}{\faLongArrowRight}
}
%
\newcommand{\ok}{
  \textcolor{darkcyan}{\faCheckCircle}
}
%
\newcommand{\pasOk}{
  \textcolor{darkcyan}{\faTimesCircle}
}
%
\newcommand{\pointCyan}{
  \textcolor{darkcyan}{\textbullet}
}
%
\newcommand{\mesure}{
  \hspace{15pt}
  \textcolor{darkcyan!75!black}{\faWrench\faFlask}
}


%%%%%%%%%%%%%%%%%%%%%%%%%%%%%%%%%%%%%%%%%%%%%%%%%%%%%%%%%%%%%%%%%%%%%%%%%%
%%%% Passage important
\newenvironment{encart}{
  \begin{tcolorbox}
  [boxrule = 0pt,
  frame hidden, sharp corners,
  colback = white!95!darkcyan, % background color
  breakable, enhanced jigsaw, % to break box over page
  borderline west={4pt}{0pt}{darkcyan}] % left bar
  
}
{
  \end{tcolorbox}
}

%%%% contexte
\newenvironment{contexte}{
  \begin{tcolorbox}
  [boxrule = 4pt, sharp corners,
  colframe = darkcyan,
  colback = white!95!darkcyan, % background color
  breakable, enhanced jigsaw] % to break box over page
  
}
{
  \end{tcolorbox}
}


%%%% emphase
\newcommand{\emphase}[1]{
  \textcolor{darkcyan}{\textsf{\bfseries \large #1}}
}
%
\newcommand{\important}[1]{
  \!\textcolor{darkgreen}{\textsf{\bfseries #1}}\!\!
}
%
\newcommand{\exemple}{
  \flecheLongue \textit{Exemple :}
}
\newcommand{\exemples}{
  \flecheLongue \textit{Exemples :}
}
%
\newcommand{\note}{
  \textbf{Note :}
}
%
\newcounter{appelProf}
\newcommand{\appelProf}{
  \refstepcounter{appelProf}
  \hspace{24pt} \faHandPaperO \hspace{2pt}
  \textbf{Appel n$^\circ$ \arabic{appelProf} :}
}


%%%% image
\newcommand{\image}[2]{
  \includegraphics[width=#1\linewidth]{#2}
}


%%%%%%%%%%%%%%%%%%%%%%%%%%%%%%%%%%%%%%%%%%%%%%%%%%%%%%%%%%%%%%%%%%%%%%%%%%
%%%% qcm
\newlist{qcm}{itemize}{2}
\setlist[qcm]{label=$\square$, leftmargin=2cm}


%%%% liste d'objectif
\newlist{listeObjectifs}{itemize}{2}
\setlist[listeObjectifs]{label = \chevron}


%%%% protocole
\newlist{protocole}{itemize}{2}
\setlist[protocole]{label = {\footnotesize \fleche}}


%%%% liste de points
\newlist{listePoints}{itemize}{2}
\setlist[listePoints]{label = \pointCyan}


%%%% jeu de données
\newenvironment{donnees}{
  \textbf{Données :}
  \begin{listePoints}
}{
  \end{listePoints}
}


%%%% liste avec chiffre
\newlist{enumeration}{enumerate}{2}
\setlist[enumeration]{label = \textcolor{darkcyan!75!black}{\textbf{\arabic*.}} }


%%%%%%%%%%%%%%%%%%%%%%%%%%%%%%%%%%%%%%%%%%%%%%%%%%%%%%%%%%%%%%%%%%%%%%%%%%
%%%% Séparation de la page
\newcommand{\separationTroisBlocs}[3]{
  \begin{minipage}[t]{0.3\linewidth}
    #1
  \end{minipage}
  ~
  \begin{minipage}[t]{0.3\linewidth}
    #2
  \end{minipage}
  ~
  \begin{minipage}[t]{0.3\linewidth}
    #3
  \end{minipage}
}
%%%%
\newcommand{\separationDeuxBlocs}[2]{
  \begin{minipage}[t]{0.48\linewidth}
    #1
  \end{minipage}
  \hfill
  \begin{minipage}[t]{0.48\linewidth}
    #2
  \end{minipage}
}
\newcommand{\separationBlocs}[4]{
  \begin{minipage}[t]{#1\linewidth}
    #3
  \end{minipage}
  \hfill
  \begin{minipage}[t]{#2\linewidth}
    #4
  \end{minipage}
}


%%%%%%%%%%%%%%%%%%%%%%%%%%%%%%%%%%%%%%%%%%%%%%%%%%%%%%%%%%%%%
%%%% tikz macros
%%%% rectangle
\newcommand{\tkzRect}[4]{
  \fill[color=#1] (#2,#4) -- (-#2,#4) -- (-#2,#3) -- (#2,#3);
}
\newcommand{\tkzEllipse}[4]{
  \fill[color=#1] (0,#3) ellipse (#2 and #4);
}
\newcommand{\tkzLegende}[4]{
  \draw[black, ->, very thick] (#2 + #4, #3) node[right] {#1} -- (#2, #3);
}
\newcommand{\tkzCercle}[4]{
  \filldraw [#3] (#1, #2) circle (#4pt);
}
\newcommand{\tkzCercleLigne}[5]{
  \filldraw [color = #4, fill = #3, very thick] (#1, #2) circle (#5pt);
}
%%%% tube à essais
\newcommand{\tkzTubeEssai}[3]{
  \draw[thick] (#1,#2) -- (#1,0) arc (0:-180:#1) -- (-#1,#2);
  \draw[thick] (0,#2) ellipse (#1 and #3);
}
\newcommand{\tkzBasTubeEssai}[5]{
  \fill[color=#1] (-#2,#3) -- (#2,#3) arc (0:-180:#2);
  \tkzRect{#1}{#2}{#3 - 0.01}{#4}
  \tkzEllipse{#1!85!black}{#2}{#4}{#5}
}
\newcommand{\tkzPhaseTubeEssai}[5]{
  \tkzRect{#1}{#2}{#3}{#4}
  \tkzEllipse{#1}{#2}{#3}{#5}
  \tkzEllipse{#1!85!black}{#2}{#4}{#5}
}

%%%% Point et vecteurs
\newcommand{\tkzLabel}[3]{
  \node at (#1, #2) {#3};
}
\newcommand{\tkzPointLabel}[3]{
  \filldraw (#1, #2) circle (2pt) node[above] {#3};
}
\newcommand{\tkzVecteur}[6]{
  \draw[black, ->, very thick] (#1, #2) -- (#1 + #3, #2 + #4) node[#6] {#5};
}
\newcommand{\tkzEquiv}[6]{
  \draw[black, <->, thick] (#1, #2) -- (#1 + #3, #2 + #4);
  \draw[black] (#1 + #3/2, #2 + #4/2) node[#6] {#5};
}
\newcommand{\tkzVecteurX}[4]{
  \draw[black, ->, very thick] (#1, #2) -- (#1 + #3, #2) node[above] {#4};
}
\newcommand{\tkzVecteurY}[4]{
  \draw[black, ->, very thick] (#1, #2) -- (#1, #2 + #3) node[right] {#4};
}
\newcommand{\tkzEquivY}[5]{
  \draw[#1, <->, thick] (#2, #3) -- (#2, #3 + #4);
  \draw[#1] (#2, #3 + #4/2) node[right] {#5};
}
\newcommand{\tkzEquivX}[5]{
  \draw[#1, <->, thick] (#2, #3) -- (#2 + #4, #3);
  \draw[#1] (#2 + #4/2, #3) node[below] {#5};
}

%%%% plan de classe
\newcommand{\rectangleTexte}[5]{
  \filldraw [fill=white, draw=black, ultra thick] (#1, #2) rectangle (#1 + #3, #2 + #4);
  \node at (#1 + #3/2, #2 + #4/2) [font=\sffamily] {\textbf{\large #5}};
}
% classe
\newcommand{\place}[3]{
  \rectangleTexte{#1}{#2}{3}{2}{#3}
}
\newcommand{\deuxPlaces}[4]{
  \place{#1}{#2}{#3}
  \place{#1 + 3}{#2}{#4}
}
\newcommand{\troisPlaces}[5]{
  \place{#1}{#2}{#3}
  \place{#1 + 3}{#2}{#4}
  \place{#1 + 6}{#2}{#5}
}
\newcommand{\quatrePlaces}[6]{
  \place{#1}{#2}{#3}
  \place{#1 + 3}{#2}{#4}
  \place{#1 + 6}{#2}{#5}
  \place{#1 + 9}{#2}{#6}
}
\newcommand{\rangDeuxTrois}[9]{
  \deuxPlaces {#1}{#2}     {#3}{#4}
  \troisPlaces{#1 + 7}{#2} {#5}{#6}{#7}
  \deuxPlaces {#1 + 17}{#2}{#8}{#9}
}
\newcommand{\rangDeuxQuatre}[9]{
  \deuxPlaces  {0}{#1} {#2}{#3}
  \quatrePlaces{7}{#1} {#4}{#5}{#6}{#7}
  \deuxPlaces  {20}{#1}{#8}{#9}
}
% TP
\newcommand{\paillasse}[3]{
  \rectangleTexte{#1}{#2}{4}{2}{#3}
}
\newcommand{\rangPaillasse}[6]{
  \paillasse{#1}{#2}     {#3}
  \paillasse{#1 + 4}{#2} {#4}
  \paillasse{#1 + 12}{#2} {#5}
  \paillasse{#1 + 16}{#2}{#6}
}


%%%%%%%%%%%%%%%%%%%%%%%%%%%%%%%%%%%%%%%%%%%%%%%%%%%%%%%%%%%%%%%%%%%%%%%%%%
%%%% unité, degré, vecteur, etc.
\newcommand{\unit}[1]{
  \; \mathrm{#1}
}
\newcommand{\unitb}[1]{
  \unit{\mathbf{#1}}
}
\newcommand{\degree}{
  ^\circ\!
}
\newcommand{\degreCelsius}{
  \degree\, \mathrm{C}
}
\newcommand{\Frac}[2]{
  \displaystyle \frac{#1}{#2}
}
\newcommand{\algebrique}[1]{
  \overline{\mathrm{#1}}
}
\newcommand{\reaction}{
  \!\!\schemestart \arrow(.mid east--.mid west){->}[, 0.9, ultra thick] \schemestop\!\!
}
%% grandeurs
\newcommand{\gISS}{g_\text{ISS}}
\newcommand{\MTerre}{M_\text{Terre}}
\newcommand{\RTerre}{R_\text{Terre}}
\newcommand{\inertie}{\text{inertie}}
\newcommand{\Tfus}{ T_{\text{f}} }
\newcommand{\Teb}{ T_{\text{éb}} }
\newcommand{\avogadro}{6,\!02 \times 10^{23}}
\newcommand{\saccha}{Saccharose (\chemfig{C_{12} H_{22} O_{11}})}
\newcommand{\vitamC}{Vitamine C (\chemfig{C_{6} H_{8} O_{6}})}
\newcommand{\ionsCa}{Ion calcium (\chemfig{Ca^{2+}})}
%% vecteurs
\newcommand{\FBsurA}{
  F_{B/A}
}
\newcommand{\FAsurB}{
  F_{A/B}
}
\newcommand{\vvFAsurB}{
  \vv{F}_{A/B}
}
\newcommand{\vvFBsurA}{
  \vv{F}_{B/A}
}

%% atome et isotope
\makeatletter
\newcommand{\isotope}[3]{%
   \settowidth\@tempdimb{\ensuremath{\scriptstyle#1}}%
   \settowidth\@tempdimc{\ensuremath{\scriptstyle#2}}%
   \ifnum\@tempdimb>\@tempdimc%
       \setlength{\@tempdima}{\@tempdimb}%
   \else%
       \setlength{\@tempdima}{\@tempdimc}%
   \fi%
  \begingroup%
  \ensuremath{
    ^{\makebox[\@tempdima][r]{\ensuremath{\scriptstyle#1}}}
    _{\makebox[\@tempdima][r]{\ensuremath{\scriptstyle#2}}}
    \chemfig{#3}
  }%
  \endgroup%
}%
\makeatother
%% element chimique
\makeatletter
\newcommand{\element}[2]{%
   \settowidth\@tempdimb{\ensuremath{\footnotesize #1}}%
  \begingroup%
  \ensuremath{
    _{\makebox[\@tempdimb][r]{\ensuremath{\small #1}}} 
    \chemfig[atom style={scale=1.3}]{#2}
  }%
  \endgroup%
}%
\makeatother

%% siècle
\newcommand{\siecle}[1]{
  \textsc{\romannumeral #1}\textsuperscript{e}~siècle
}
%% texte avec une boite autour
\newcommand{\texteEncadre}[1]{
  \frame{\vphantom{$\Frac{1}{10}$} \text{ #1 } }
}


%%%%%%%%%%%%%%%%%%%%%%%%%%%%%%%%%%%%%%%%%%%%%%%%%%%%%%%%%%%%%%%%%%%%%%%%%%
%%%% Couleur pour le code
\definecolor{vertCode}    {rgb}{0.2,0.6,0}
\definecolor{grisCode}    {rgb}{0.5,0.5,0.5}
\definecolor{violetCode}  {rgb}{0.58,0,0.82}
\definecolor{fondCode}    {rgb}{0.95,0.95,0.92}
%%%% Style python
\lstdefinestyle{codePython}{
    backgroundcolor=\color{fondCode},
    commentstyle=\color{magenta},
    keywordstyle=\color{vertCode},
    numberstyle=\tiny\color{grisCode},
    stringstyle=\color{violetCode},
    basicstyle=\ttfamily\footnotesize,
    breakatwhitespace=false,
    breaklines=true,
    captionpos=b,
    keepspaces=true,
    numbers=left,
    numbersep=5pt, 
    showspaces=false,
    showstringspaces=false,
    showtabs=false, 
    tabsize=2
}
\def\inline{\lstinline[style=codePython,language=python]}


%%%%%%%%%%%%%%%%%%%%%%%%%%%%%%%%%%%%%%%%%%%%%%%%%%%%%%%%%%%%%%%%%%%%%%%%%%