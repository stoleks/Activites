%%%%%%%%%%%%%%%%%%%%%%%%%%%%%%%%%%%%%%%%%%%%%%%%%%%%%%%%%%%%%
%%%% figures simples
\newcommand{\tkzRect}[4]{
  \fill[color=#1] (#2,#4) -- (-#2,#4) -- (-#2,#3) -- (#2,#3);
}
\newcommand{\tkzEllipse}[4]{
  \fill[color=#1] (0,#3) ellipse (#2 and #4);
}

% \tkzCercle {x}{y} {couleur} {rayon}
\newcommand{\tkzCercle}[4]{
  \filldraw [#3] (#1, #2) circle (#4pt);
}
% \tkzCercleLigne {x}{y} {couleurFond}{couleurTrait} {rayon}
\newcommand{\tkzCercleLigne}[5]{
  \filldraw [color = #4, fill = #3, very thick] (#1, #2) circle (#5pt);
}

%%%% tube à essais
\newcommand{\tkzTubeEssais}[3]{
  \draw[thick] (#1,#2) -- (#1,0) arc (0:-180:#1) -- (-#1,#2);
  \draw[thick] (0,#2) ellipse (#1 and #3);
}
\newcommand{\tkzBasTubeEssais}[5]{
  \fill[color=#1] (-#2,#3) -- (#2,#3) arc (0:-180:#2);
  \tkzRect{#1}{#2}{#3 - 0.01}{#4}
  \tkzEllipse{#1!85!black}{#2}{#4}{#5}
}
\newcommand{\tkzPhaseTubeEssais}[5]{
  \tkzRect{#1}{#2}{#3}{#4}
  \tkzEllipse{#1}{#2}{#3}{#5}
  \tkzEllipse{#1!85!black}{#2}{#4}{#5}
}

%%%% Point et vecteurs
\newcommand{\tkzLabel}[3]{
  \node at (#1, #2) {#3};
}
\newcommand{\tkzPointLabel}[3]{
  \filldraw (#1, #2) circle (2pt) node[above] {#3};
}
% \tkzVecteur [couleur] (x) [longueur x] (y) [longueur y] {legende} [Positionition legende] 
% ajouter une * à la fin transforme la flèche en double flèche <->
\NewDocumentCommand{\tkzVecteur}{O{black} r() O{0} r() O{0} m O{right} s}{
  \IfBooleanTF{#8}{
    \draw[#1, <->, very thick] (#2, #4) -- (#2 + #3, #4 + #5) node[#7] {#6};
  }{
    \draw[#1, ->, very thick] (#2, #4) -- (#2 + #3, #4 + #5) node[#7] {#6};
  }
}
% \tkzLegende (x) (y) [longueur fleche] {légende} 
% ajouter une * passe de la version gauche -> à la version droite <-
\NewDocumentCommand{\tkzLegende}{O{black} r() r() O{1.25} m s}{
  \IfBooleanTF{#6}{
    \draw[#1, ->, very thick] (#2 + #4, #3) node[right] {#5} -- (#2, #3);
  }{
    \draw[#1, ->, very thick] (#2, #3) node[left] {#5} -- (#2 + #4, #3);
  }
}

\newcommand{\barrePourcentage}[1]{%
  \begin{tikzpicture}
    \fill[color=couleurSec]    (0.0,    0.0) rectangle (#1*8ex, 1.5ex);
    \fill[color=couleurSec!20] (#1*8ex, 0.0) rectangle (8.0ex,  1.5ex);
  \end{tikzpicture}
}

%% Trace une flèche de progression pour les plans de travail
% \flecheProgression {<nombre de boucles>} [<largeur>] [<espacement vertical>]
\NewDocumentCommand{\flecheProgression}{m O{17} O{2.5}}{%
  \strut\vspace*{8pt}
  \begin{center}
    \begin{tikzpicture}
      \def\max{\numexpr((#1-1)*2)}
      % Premier bout pour l'alignement
      \draw[
        line width = 20pt,
        rounded corners = 10mm,
        color = couleurSec,
      ]
        (0, {(\max-0)*#3}) -- (1, {(\max-0)*#3});
      % Partie centrale répétée
      \draw[
        line width = 20pt,
        rounded corners = 10mm,
        color = couleurSec,
      ]
        \foreach \x in {0,2,...,\max}  {
          \ifnum \x < \max
            ( 1, {(\max-\x)  *#3}) -- (#2,  {(\max-\x)  *#3}) --
            (#2, {(\max-\x-1)*#3}) -- ( 0,  {(\max-\x-1)*#3}) --
            ( 0, {(\max-\x-2)*#3}) -- (1.1, {(\max-\x-2)*#3})
          \fi
        };
      % Flèche finale
      \draw[
        -{Triangle [width = 36pt, length = 16pt]}, 
        line width = 20pt,
        color = couleurSec
      ]
        (0.8, 0) -- (#2, 0);
    \end{tikzpicture}
  \end{center}
}

%%%%%%%%%%%%%%%%%%%%%%%%%%%%%%%%%%%%%%%%%%%%%%%%%%%%%%%%%%%%%
%%%% plan de classe
% Trace un texte centré dans un cadre (x, x+l) -- (y, y+h)
% #1 couleur cadre ; #2 Positionition x ;
% #3 largeur l ;     #4 Positionition y ;
% #5 hauteur h ;     #6 texte.
\NewDocumentCommand{\texteCadre}{O{black} r() O{2} r() O{2} m}{
  \filldraw [fill=white, draw=#1, ultra thick] (#2, #4) rectangle (#2 + #3, #4 + #5);
  \node at (#2 + #3/2, #4 + #5/2) [font=\sffamily] {\textbf{#6}};
}

%% place dans la classe
\NewDocumentCommand{\place}{r() m}{
  \texteCadre(#1)[3](0)[2] {#2}
}

%% Pour tracer une rangée d'élève avec 2 ou 3 colonnes
% \rang {<numero rangee>} {<eleves>} {<eleves>} [<eleves>]
\ExplSyntaxOn
% Position de la place horizontale
\int_new:N \l_rangPositionX_int
\NewDocumentCommand{\rang}{m >{\SplitList{,}} m >{\SplitList{,}} m >{\SplitList{,}} d[]}{
  \begin{tikzpicture}
    % Initialisation de la position horizontale
    \int_set:Nn \l_rangPositionX_int {0}
    % Première rangée
    \ProcessList{#2}{\rangImpl}
    \int_add:Nn \l_rangPositionX_int { 1 }
    % Deuxième rangée
    \ProcessList{#3}{\rangImpl}
    % Troisième rangée
    \IfValueT{#4}{
      \int_add:Nn \l_rangPositionX_int { 1 }
      \ProcessList{#4}{\rangImpl}
    }
  \end{tikzpicture}
  \bigskip
}
\NewDocumentCommand{\rangImpl}{m}{
  \int_add:Nn \l_rangPositionX_int { 3 }
  \place(\l_rangPositionX_int){#1} %
}
\ExplSyntaxOff


%%%% tube à essai de sang
\newcommand{\tubeEssaisSolution}[1]{
  \begin{tikzpicture}
    \tkzBasTubeEssais{#1}{0.25}{0}{0.75}{0.1} % contenu du tube
    \tkzTubeEssais{0.25}{1.5}{0.1} % tube
  \end{tikzpicture}
}

\newcommand{\tubeEssaisSangCentrifuge}[3]{
  \begin{tikzpicture}
    % phases dans le tube à essai
    \tkzBasTubeEssais{rougeSombre!75!white} {0.35}{0}{#1}{0.1}
    \tkzPhaseTubeEssais{gray!10!white}      {0.35}{#1}{#2}{0.1}
    \tkzPhaseTubeEssais{jauneClair!75!white}{0.35}{#2}{#3}{0.1}
    \tkzTubeEssais{0.35}{#3 + 1}{0.1}
    % Légende
    \tkzLegende(0.4)(#3 - 0.1) [1]{Plasma}*
    \tkzLegende(0.4)(#2 - 0.08)[1]{Globules blancs}*
    \tkzLegende(0.4)(-0.1)     [1]{Globules rouges}*
  \end{tikzpicture}
}