%%%%%%%%%%%%%%%%%%%%%%%%%%%%%%%%%%%%%%%%%%%%%%%%%%%%%%%%%%%%%
%%%% Style commun à plusieurs fichiers
\tikzset{
  rectArrondi/.style = {
    rounded corners = 8pt, inner ysep = 10pt,
    text width = 3.75cm, align = center, fill = #1,
  },
  fleche/.style = {->, line width = 6pt, color = #1}
}

%%%%%%%%%%%%%%%%%%%%%%%%%%%%%%%%%%%%%%%%%%%%%%%%%%%%%%%%%%%%%
%%%% Cercles, points et vecteurs
\pgfkeys{
  /planete/.is family, /planete,
  defaut/.style = {
    rayon = 60 pt,
    orbite = 0 pt,
    remplissage = couleurSec-100,
    contour = couleurSec-100,
    centre = T,
    satellite = {}
  },% stockage des clefs
  rayon/.store in = \planeteRayon,
  orbite/.store in = \planeteOrbite,
  centre/.store in = \planeteCentre,
  satellite/.store in = \planeteSatellite,
  contour/.store in = \planeteContour,
  remplissage/.store in = \planeteRemplissage
}
%
\NewDocumentCommand{\tikzSchemaPlanete}{O{}}{%
  \pgfkeys{/planete, defaut, #1}  
  \filldraw [color = \planeteContour, very thick, fill = white] (0,0) circle (\planeteOrbite);
  \filldraw [\planeteRemplissage] (0,0) circle (\planeteRayon);
  \filldraw (0, 0) circle (2pt) node[above] {\planeteCentre};
  \filldraw (\planeteOrbite, 0) circle (2pt) node[right] {\planeteSatellite};
}
% \tikzPointLabel (x, y) [texte] (xtexte, ytexte), trace un point avec un label; * = pas de point
\NewDocumentCommand{\tikzLabel}{s r() m d()}{
  \IfBooleanTF{#1}{
    \node at (#2) {#3};
  }{
    \IfValueTF{#4}{
      \filldraw (#2) circle (2pt);
      \node at (#4) {#3};
    }{
     \filldraw (#2) circle (2pt) node[above] {#3};
    }
  }
}
% \tikzVecteur [couleur] (x, y) (lx, ly) {legende} [position legende];  * = <->
\NewDocumentCommand{\tikzVecteur}{s O{black} r() r() m O{right}}{
  \coordinate (A) at (#3);
  \coordinate (B) at (#4);
  \coordinate (AB) at ($(A)+(B)$);
  \IfBooleanTF{#1}{
    \draw[#2, <->, very thick] (A) -- (AB) node[#6] {#5};
  }{
    \draw[#2, ->, very thick] (A) -- (AB) node[#6] {#5};
  }
}

%% Trace une petite barre de pourcentage partiellement remplie sur la ligne
\NewDocumentCommand{\barrePourcentage}{O{couleurSec} m}{%
  \begin{tikzpicture}
    \fill[color=#1]     (0.0,    0.0) rectangle (#2*8ex, 1.5ex);
    \fill[color=#1-100] (#2*8ex, 0.0) rectangle (8.0ex,  1.5ex);
  \end{tikzpicture}
}


%%%%%%%%%%%%%%%%%%%%%%%%%%%%%%%%%%%%%%%%%%%%%%%%%%%%%%%%%%%%%
%% Trace une flèche de progression pour les plans de travail
\pgfkeys{% définition de la famille de clefs
  /prog/.is family, /prog,
  defaut/.style = {
    boucles = 1,
    largeur = 17,
    coeff = 2.4
  },% stockage des clefs
  boucles/.store in = \progBoucles,
  largeur/.store in = \progLargeur,
  coeff/.store in = \progCoeff,
}
%
\newcommand{\nombreBoucle}{}
\NewDocumentCommand{\flecheProgression}{m}{%
  \pgfkeys{/prog, defaut, #1}
  \strut\vspace*{8pt}
  \renewcommand{\nombreBoucle}{\numexpr((\progBoucles - 1)*2)}
  \begin{center}
    \begin{tikzpicture}
      \tikzset{bluestyle/.style={line width = 20pt, rounded corners = 10mm, color = couleurSec}}
      % Premier bout pour l'alignement, les parenthèses sont nécessaires
      \draw[bluestyle] (0, {
        (\nombreBoucle)*\progCoeff}) -- (1, {(\nombreBoucle)*\progCoeff
      }); 
      % Partie centrale répétée
      \draw[bluestyle]
        \foreach \x in {0,2,...,\nombreBoucle}  {
          \ifnum \x < \nombreBoucle
            (1, {(\nombreBoucle - \x)  *\progCoeff}) --
            (\progLargeur,  {(\nombreBoucle - \x)  *\progCoeff})  --
            (\progLargeur, {(\nombreBoucle - \x - 1)*\progCoeff}) --
            (0, {(\nombreBoucle - \x - 1)*\progCoeff}) --
            (0, {(\nombreBoucle - \x - 2)*\progCoeff}) --
            (1.1, {(\nombreBoucle - \x - 2)*\progCoeff})
          \fi
        };
      % Flèche finale
      \draw[-{Triangle [width = 36pt, length = 16pt]}, bluestyle] (0.8, 0) -- (\progLargeur, 0);
    \end{tikzpicture}
  \end{center}
  %% retire l'espace ajouté par les boucles
  \vspace*{-35\progBoucles pt}
}


%%%%%%%%%%%%%%%%%%%%%%%%%%%%%%%%%%%%%%%%%%%%%%%%%%%%%%%%%%%%%
%%%% plan de classe

%% Pour tracer une rangée d'élève avec 2 ou 3 colonnes
% \rang <largeur> <hauteur> {<eleves>} {<eleves>} {<eleves>}
\ExplSyntaxOn
\fp_new:N \l_rangPositionX_fp% position de la place
\fp_new:N \l_largeurPlaceX_fp% largeur d'une place
\fp_new:N \l_hauteurPlaceY_fp% hauteur d'une place
\NewDocumentCommand{\rang}{%
  D<>{3} D<>{2} 
  >{\SplitList{,}}r[] >{\SplitList{,}}r[] >{\SplitList{,}}d[]
}{%
  \begin{tikzpicture}
    % Initialisation de la position horizontale et de la taille des places
    \fp_set:Nn \l_rangPositionX_fp {0}
    \fp_set:Nn \l_largeurPlaceX_fp {#1}
    \fp_set:Nn \l_hauteurPlaceY_fp {#2}
    % Première rangée
    \ProcessList{#3}{\rangImpl}
    \fp_add:Nn \l_rangPositionX_fp {1}
    % Deuxième rangée
    \ProcessList{#4}{\rangImpl}
    % Troisième rangée
    \IfValueT{#5}{
      \fp_add:Nn \l_rangPositionX_fp {1}
      \ProcessList{#5}{\rangImpl}
    }
  \end{tikzpicture}
  \bigskip
}
% \texteRectangle [couleur cadre] (x, y) (lx, ly) {texte}
\NewDocumentCommand{\texteRectangle}{O{black} r() r() m}{%
  \coordinate (A) at (#2);
  \coordinate (B) at (#3);
  \filldraw [fill=white, draw=#1, ultra thick] (A) rectangle ($(A)+(B)$);
  \coordinate (C) at (0.5*#3/2); % divide by two lx and ly
  \node at ($(A)+(C)$) [font=\sffamily] {\textbf{#4}};
}
% Commande interne qui appelle \texteRectangle pour chaque élève
\NewDocumentCommand{\rangImpl}{m}{%
  \fp_add:Nn \l_rangPositionX_fp {\l_largeurPlaceX_fp}
  \texteRectangle
    (\fp_use:N \l_rangPositionX_fp, 0) % Position
    (\fp_use:N \l_largeurPlaceX_fp, \fp_use:N \l_hauteurPlaceY_fp) % largeur et hauteur
    {#1} % élève
}
\ExplSyntaxOff


%%%%%%%%%%%%%%%%%%%%%%%%%%%%%%%%%%%%%%%%%%%%%%%%%%%%%%%%%%%%%
%%%% tube à essais
\pgfkeys{
  /tube/.is family, /tube,
  defaut/.style = {
    largeur = 0.25cm,
    hauteur = 1.5cm,
    rayon = 0.1cm,
    phase = 0.25cm,
    couleur = red-300
  },% stockage des clefs
  largeur/.store in = \tubeX,
  hauteur/.store in = \tubeY,
  rayon/.store in = \tubeRayon,
  phase/.store in = \tubePhase,
  couleur/.store in = \tubeCouleur
}
%
\NewDocumentCommand{\tikzTubeEssais}{O{}}{
  \pgfkeys{/tube, defaut, #1}
  \draw[thick] (\tubeX, \tubeY)-- (\tubeX, 0) arc (0:-180:\tubeX)-- (-\tubeX, \tubeY);
  % note : le + 0 permet d'éviter que l'unité et "and" soit collé et donc non compris par pgf
  \draw[thick] (0, \tubeY) ellipse (\tubeX+0 and \tubeRayon);
}
%
\NewDocumentCommand{\tikzPhaseBasTubeEssais}{O{}}{
  \pgfkeys{/tube, defaut, #1}
  \fill[color=\tubeCouleur] (-\tubeX, 0)-- (\tubeX, 0) arc (0:-180:\tubeX); % bas
  \fill[color=\tubeCouleur] (\tubeX, \tubeY)-- (-\tubeX, \tubeY)-- (-\tubeX, -0.01)-- (\tubeX, -0.01);% centre
  \fill[color=\tubeCouleur!85!black] (0, \tubeY) ellipse (\tubeX+0 and \tubeRayon);% ellipse du haut
}
%
\NewDocumentCommand{\tikzPhaseTubeEssais}{O{}}{
  \pgfkeys{/tube, defaut, #1}
  \fill[color=\tubeCouleur] (\tubeX, \tubePhase) -- (-\tubeX, \tubePhase) -- (-\tubeX, \tubeY) -- (\tubeX, \tubeY); % centre de la phase
  \fill[color=\tubeCouleur] (0, \tubeY) ellipse (\tubeX+0 and \tubeRayon); % bas de la phase
  \fill[color=\tubeCouleur!85!black] (0, \tubePhase) ellipse (\tubeX+0 and \tubeRayon);% haut de la phase
}

%%%% tube à essai avec une seule solution
\newcommand{\tubeEssaisSolution}[1]{
  \begin{tikzpicture}
    \tikzPhaseBasTubeEssais [couleur = #1, largeur = 0.25cm, hauteur = 0.75cm]% contenu du tube
    \tikzTubeEssais% tube
  \end{tikzpicture}
}