%%%%
\sndEnTeteCinq

%%%% titre
\vspace*{-40pt}
\numeroActivite{1}
\titreActivite{Rester frais l'été}

%%%% Objectifs
\begin{objectifs}
  \item Savoir écrire symboliquement un changement d'état.
  \item Comprendre pourquoi l'évaporation de l'eau rafraîchit.
\end{objectifs}

\begin{contexte}
  Les étés étant de plus en plus chaud, il devient vital de comprendre comment se refroidir efficacement en perdant de l'énergie.
  Pour ça, il est essentiel de comprendre l'impact des changements d'états courant dans la vie quotidienne.
  
  \problematique{
    Quels changements d'états physiques absorbent de l'énergie ?
  }
\end{contexte}


%%%% docs
\begin{doc}{Transfert thermique}
  \vspace*{-22pt}
  \begin{encart}
    Un corps chaud en contact avec un corps froid lui transfert de l'énergie, ce qui se traduit par une modification de la température des deux corps : on parle de \important{transfert thermique}.
  \end{encart}
  L'énergie transférée se note $Q$, son unité est le Joule$\unit{J}$.
  Un corps qui \textbf{reçoit un transfert thermique positif} ($Q > 0$) voit \textbf{sa température augmenter.}
  
  \begin{encart}
    Sous certaines conditions, ce transfert thermique peut mener un des deux corps à changer d'état (liquide à gaz par exemple) : on parle de \important{transformation physique}.
  \end{encart}
  \attention Le transfert thermique va \textbf{toujours} du corps chaud vers le corps froid !
\end{doc}

%%
\begin{doc}{Notation symbolique d'un changement d'état}
  \vspace*{-22pt}
  \begin{encart}
    Le changement d'état d'une espèce chimique \chemfig{X} est noté symboliquement
    \begin{equation*}
      \chemfig{X} (\text{état physique initial}) \reaction
      \chemfig{X} (\text{état physique final})
    \end{equation*}
  \end{encart}
  
  \exemple $\chemfig{H_2O}(s) \reaction \chemfig{H_2O}(l)$.
\end{doc}

%%
\begin{doc}{Transformations endothermique et exothermique}
  \vspace*{-20pt}
  \begin{encart}
    \begin{listePoints}
      \item Pendant une \important{transformation exothermique}, l'énergie du système diminue. 
      Le milieu extérieur reçoit un transfert thermique positif $Q > 0$.
      \item Pendant une \important{transformation endothermique}, l'énergie du système augmente.
      Le milieu extérieur reçoit un transfert thermique négatif $Q < 0$.
    \end{listePoints}
  \end{encart}
\end{doc}


%%
\newpage
\vspace*{-36pt}
\begin{doc}{L'éco-climatisation}
  \label{doc:climatisation}
  \vspace*{-18pt}
  \begin{wrapfigure}{l}{0.3\linewidth}
    \vspace*{-15pt}
    \centering
    \image{1}{images/transformations/eco_climatisation}
  \end{wrapfigure}
  À cause du réchauffement climatique, la consommation d'énergie liée à la climatisation ne fait qu'augmenter, avec un impact fort sur l'environnement.
  
  Des solutions plus écologiques existent : quand de l'air chaud arrive au contact de gouttelettes d'eau liquide, les gouttelettes s'évaporent.
  L'air chaud se refroidit alors rapidement grâce à l'évaporation.
\end{doc}

%%
\begin{doc}{Sueur et fraîcheur}
  \label{doc:evaporation}
  Quand l'eau s'évapore, elle passe de l'état liquide à l'état gazeux.
  Ce phénomène absorbe de l'énergie dans l'environnement proche.
  Lorsqu'on est mouillé, le transfert thermique se fait avec notre corps, qui se refroidit alors.
\end{doc}

%%
\begin{doc}{Un glaçon dans ma boisson}
  \label{doc:glacons}
  Si on veut refroidir une boisson tiède, on peut la placer dans un réfrigérateur, mais une solution bien plus rapide est de rajouter des glaçons dedans.
  
  Le principe est très simple : en fondant, les glaçons vont absorber de l'énergie, ce qui va refroidir l'eau qui les entoure.
\end{doc}


%%%% Questions
\question{
  Pour chaque documents (\ref{doc:climatisation}, \ref{doc:evaporation}, \ref{doc:glacons}), indiquer quel est le corps qui change d'état, avec l'état initial et l'état final.
}{
  bla
}{3}

\question{
  Pour chaque documents, indiquer si la transformation physique est endothermique ou exothermique, en donnant le signe du transfert thermique $Q$ reçu par le milieu extérieur.
}{
  bla
}{3}

\question{
  Pour chaque documents, écrire la notation symbolique du changement d'état.
}{
  bla
}{3}