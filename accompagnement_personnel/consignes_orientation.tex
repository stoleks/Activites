\teteSndAP

%\vspace*{-32pt}
\numeroActivite{4}
\titreActivite*{Présentation orale d'une filière}

\begin{doc}{Consigne générale}{AP_consigne_generale}
  \begin{listePoints}
    \item Vous allez travailler en groupe de \important{3 personnes.}
    \item Chaque groupe va préparer une \important{présentation orale} sur une \important{filière et ses débouchés.}
    \item La présentation devra durer \important{10 minutes.}
    \item La \important{filière} c'est :
    \begin{listePoints}
      \item Une \important{filière technologique,} avec présentation de deux spécialités.
      \item La \important{filière générale,} avec présentation d'un couple de deux spécialités.
      \item Une \important{filière professionnelle,} avec une présentation détaillée du contenu.
    \end{listePoints}
    \item Il faudra présenter dans \important{quelle lycée et quelle ville} on peut faire la filière.
    \item Les \important{débouchés} sont des métiers et les études supérieures nécessaires pour les faire.
  \end{listePoints}

  Pour la présentation orale devra être accompagnée d'un \important{diaporama} préparée à l'avance et mise sur une clé USB, ou envoyée sur pronote un week-end avant le jour de la présentation.
\end{doc}

\begin{doc}{Planning des prochaines séances}{AP_planning}
  \strut\vspace*{8pt}
  
  \flecheProgression{
    (0,  5.0) -- (16, 5.0) --
    (16, 2.5) -- (0,  2.5) --
    (0,  0)   -- (16, 0);
  }
  \vspace*{-8 cm}
  
  \begin{programmeSeance}[2]
    \seance{2 h}{Formation des  groupes et travail de recherche}
    \seance{2 h}{Finalisation du travail de recherche et préparation de la présentation}
  \end{programmeSeance}
  \vspace*{1.2 cm}
  \begin{programmeSeance}[2]
    \seance{2 h}{Passage à l'oral devant les 2GT2 et 2GT3}
    \seance{2 h}{Passage à l'oral devant les 2GT2 et 2GT3}
  \end{programmeSeance}
\end{doc}

\begin{doc}{Quelques sites ressources}{doc:AP_sites_ressources}
  \begin{listePoints}
    \item onisep : \url{https://www.onisep.fr/}  
    
    \item Horizons 21 : \url{https://www.horizons21.fr/}
    \item education.gouv : \url{https://www.education.gouv.fr/reussir-au-lycee}
  \end{listePoints}
\end{doc}