\teteSndAP
\titreActivite*{Sortie au muséum d'histoire naturelle}

% \begin{objectifs}
%   \item 
%   \item 
%   \item 
%   \item 
% \end{objectifs}

\begin{center}
  \important{ON NE TOUCHE PAS LES ÉCHANTILLONS EXPOSÉS !}
\end{center}

\begin{doc}{Pour finaliser vos dossiers}
  Par rapport aux dossiers à préparer :
  \begin{itemize}
    \item La biodiversité actuelle se trouve dans les deux galeries.
    \item La biodiversité passée se trouve dans la galerie de paléontologie niveau 1 et 2, et est un peu disséminée dans la grande galerie.
    \item L’évolution de la biodiversité au cours du temps se trouve au niveau 3 étage de la grande galerie de l'évolution.
    \item La préservation de la biodiversité en milieu urbain se trouve au niveau 2 de la grande galerie de l'évolution.
    \item La préservation de la biodiversité en milieu rural se trouve au niveau 2 étage de la grande galerie de l'évolution.
  \end{itemize}
N'oubliez pas de prendre des photos pour vos dossiers !
\end{doc}

\question{
Identifier 1 espèce actuelle pour les écosystèmes suivants :
\begin{listePoints}[2]
  \item Savane ;
  \item Forêt tropicale ;
  \item Milieu marin (océan) ;
  \item Milieu d'eau douce (rivière, lac) ;
  \item Milieu urbain ;
  \item Forêt.
\end{listePoints}
Donner le nom commun !
}{}[8]

\question{
  Donner les trois grands groupes du vivant.
}{}[2]

\question{
  Identifier dans quel grand groupe se trouve les êtres humains.
}{}[2]

\newpage
\question{
  Génétiquement, les champignons sont-ils plus proches des animaux ou des plantes ? Justifier en vous aidant des panneaux fournis dans la grande galerie de l'évolution.
}{}[6]

\question{
  Lister trois écosystèmes passés avec trois espèces animales et végétales, qui y vivaient en vous aidant des informations fournies dans la galerie de paléontologie.
}{}[6]

\question{
  Trouver trois points communs entre la structure anatomique de l'être humain et des grands singes, en vous aidant des squelettes exposés dans la galerie de paléontologie.
}{}[6]

\question{
  Trouver trois différences entre la structure anatomique de l'être humain et des grands singes, en vous aidant des squelettes exposés dans la galerie de paléontologie.
}{}[6]