\teteSndAP
\titreActivite*{Préfixe et abréviation symbolique des unités}

\begin{objectifs}
  \item Être capable d'associer le symbole d'une unité de mesure avec sa grandeur.
  \item Savoir reconnaître les préfixe du système international.
  \item Savoir convertir une unité en multiple ou sous-multiples.
\end{objectifs}

\begin{contexte}  
  La maîtrise des unités de mesure et des conversions en multiples et sous-multiples est essentielle en
  sciences et en mathématiques :
  \begin{listePoints}
    \item pour la compréhension des énoncés d’exercices ;
    \item pour la rédaction des réponses aux questions.
  \end{listePoints}
\end{contexte}

\begin{doc}{Les préfixes des unités du système international}
  Dans le système international, chaque unité multiple ou sous-multiple de 1000 est associée à un préfixe (milli, kilo, méga etc.).
  À chaque préfixe correspond une puissance de dix.
  Les symboles des préfixes se comportent donc comme des coefficients multiplicateur qui peuvent être remplacés par le produit d'une puissance de dix équivalente.

  \vspace*{-8pt}
  \begin{center}
    \begin{tblr}{
      hlines, vlines, rows = {25pt},
      colspec = {c *{8}{Q[c, m, 0.08\linewidth]}},
      column{1} = {couleurPrim!10, c}, row{1} = {couleurPrim!20, c},
    }
      Puissance &
      \num{e-12} & \num{e-9} & \num{e-6} & \num{e-3} &
      $10^0$ = 1 & \num{e3} & \num{e6} & \num{e9} \\ %
      Préfixe \\ %
      Symbole \\
    \end{tblr}

    \vspace*{-20pt}
    \flecheProgression{1}[16]
    \vspace*{-38pt}
    \important[white]{Du plus petit au plus grand}
  \end{center}

  Il existe aussi des multiples et sous-multiples pour \num{0,01}, \num{0,1}, \num{10} et \num{100}, vu que ce sont des échelles souvent manipulées au quotidien.
  
  \vspace*{-8pt}
  \begin{center}
    \begin{tblr}{
      hlines, vlines, rows = {25pt},
      colspec = {c *{4}{Q[c, m, 0.08\linewidth]}},
      column{1} = {couleurPrim!10, c}, row{1} = {couleurPrim!20, c},
    }
      Puissance & \num{e-2} & \num{e-1} & \num{e1} & \num{e2} \\ %
      Préfixe \\ %
      Symbole \\
    \end{tblr}
  \end{center}
\end{doc}


\begin{doc}{Conversion des unités}
  \important{1 - Conversion d'une unité multiple ou sous-multiple en unité de base}

  Dans ce cas, il faut remplacer le préfixe par la puissance de 10 qui lui est associée.
  
  \exemple On veut convertir l'énergie $E = \qty{3,6}{\mega\joule}$ en joule, noté \unit{\joule}.
  Pour ça on utilise que $\qty{1}{\mega\joule} = \qty{e6}{\joule}$.
  Donc $E =$ \texteTrou[0.3]{\qty{3,6e6}{\mega\joule}}

  \bigskip
  \important{2 - Conversion d'une unité de base en multiple ou sous-multiple}

  Dans ce cas, il faut diviser la grandeur par la puissance de 10 associée au préfixe que l'on veut utiliser.
  Cela revient à multiplier par la puissance de 10 d'exposant opposé.

  \exemple On veut convertir la distance $d = \qty{4,3e-9}{\m}$ (mètre) en nanomètre noté \unit{nm}.
  On utilise que $\qty{1}{\nano\m} = \qty{e-9}{\m}$ et que $1 / 10^{-9} = 10^9$. 
  Donc $d = $ \texteTrou(1){\qty{3,6e6}{\mega\joule}}
\end{doc}

\question{
  Convertir la fréquence $f = \qty{4,2}{\giga\hertz}$ en hertz noté \unit{\hertz}.
}{}[3]

\question{
  Convertir la pression $p = \qty{100 000}{\pascal}$ en hectopascal noté \unit{\hecto\pascal}.
}{}[3]

\question{
  Convertir l'intensité électrique $i = \qty{78}{\micro\ampere}$ en ampère noté \unit{\ampere}.
}{}[3]

\question{
  Convertir le temps $t = \qty{86400}{\s}$ en mégaseconde noté \unit{\mega\s}.
}{}[3]

\question{
  Convertir la masse $m = \qty{0,005}{\g}$ en milligramme noté \unit{\milli\g}.
}{}[3]

\question{
  Convertir la puissance $P = \qty{4700}{\watt}$ en kilowatt noté \unit{\kilo\watt}.
}{}[3]