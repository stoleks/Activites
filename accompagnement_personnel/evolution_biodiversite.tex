\teteSndAP
\vspace*{-36pt}
\titreActivite*{Évolution de la biodiversité}

\begin{contexte}
  La Terre a connu plusieurs épisodes de changement de biodiversité.
  Il est important de caractériser la biodiversité passée pour comprendre la biodiversité actuelle et préserver la biodiversité future.
\end{contexte}

\begin{doc}{Consignes}{doc:consignes}
Vous allez devoir produire un rendu écrit sur la notion de biodiversité. 
Pour cela, vous respecterez les différents temps suivants :
\begin{listePoints}
\item Temps 1 : Vous devez former des groupes de 4. Il y aura donc 5 groupes de 4 en tout.
\item Temps 2 : Chaque groupe choisit une des thématiques suivantes autour de la notion de la biodiversité et qu’il devra détailler dans son rendu écrit. 
\begin{itemize}
\item Biodiversité actuelle.
\item Biodiversité passée.
\item L’évolution de la biodiversité au cours du temps.
\item La préservation de la biodiversité en milieu urbain.
\item La préservation de la biodiversité en milieu rural.
\end{itemize}
\item Temps 3 : Vous commencez vos recherches pour construire progressivement votre rendu écrit. Pour cela, vous réaliserez des recherches personnelles en veillant à consulter des ressources fiables, et vous compléterez ces dernières lors de la sortie organisée le mercredi 30 avril 2025. Cette sortie consistera en la visite du Muséum d’Histoire Naturelle.
\end{listePoints}    
\end{doc}

\begin{doc}{Thématiques à choisir}{doc:thematiques}
Précisions concernant les thématiques proposées :
\begin{listePoints}
    
\item Biodiversité actuelle : méthodes de recensement, données chiffrées sur la biodiversité actuelle, menaces actuelles sur la biodiversité.
\item Biodiversité passée : méthodes de recensement, données chiffrées sur la biodiversité passé, quelques exemples d’espèces passées et éteintes de nos jours, exemples de phénomènes à l’origine de la disparition d’espèces.
\item L’évolution de la biodiversité au cours du temps : présentation des différents processus évolutifs, exemples d’évolution observables sur un temps court et sur un temps long.
\item La préservation de la biodiversité en milieu urbain : présentation des mesures/politiques mises en place pour protéger et restaurer la biodiversité en milieu urbain (ex : végétaliser, renaturaliser..)
\item La préservation de la biodiversité en milieu campagnard :  présentation des mesures/politiques mises en place pour protéger et restaurer la biodiversité en milieu rural (ex : régularisation de la chasse, réintroduction d’espèces, contrôle de l’utilisation des pesticides..).
\end{listePoints}
\end{doc}

\begin{doc}{Précisions sur le format du rendu écrit}{doc:precision_format}
Document tapé à l’ordinateur et devant obligatoirement comprendre des photographies prises par vos soins (sauf exception du milieu rural où les photographies pourront être prises sur Internet), avec toutes les sources utilisées citées explicitement.

\important{Format :} le texte doit être justifié, rédigé en Arial 12, interligne 1,5 et \important{être envoyé à votre professeur au format PDF sur sa messagerie de l’ENT 1 semaine après la sortie, soit le mercredi 7 mai.} Minimum 1 page, maximum 4 pages.
\end{doc}