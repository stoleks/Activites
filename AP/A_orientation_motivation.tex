\teteSndAP
\titreActivite*{L’orientation après la classe de seconde}

\begin{objectifs}
  \item Découverte des milieux professionnels ;
  \item Rédaction d’une lettre de motivation ;
  \item Maîtrise de l’outil numérique.
\end{objectifs}
\medskip

\begin{contexte}
  Du 16 au 27 juin, vous devrez réaliser obligatoirement un stage d’observation en milieu professionnel ou associatif. Ce stage doit rentrer en cohérence avec votre projet d’orientation.
\end{contexte}
\medskip

\begin{doc}{Ouverture aux différents milieux professionnels}
  Dans un premier temps, il convient de vous familiariser avec la diversité de métiers et de formations qui vous sont ouverts.
  Consultez les deux liens suivants :
  \begin{listePoints}
    \item \url{https://www.onisep.fr/metier/des-metiers-selon-mes-gouts} 
    \item \url{https://www.onisep.fr/metier/decouvrir-le-monde-professionnel}
  \end{listePoints}
  Ce moment vise à vous faire découvrir des formations via le site de l’ONISEP, en charge de l’insertion professionnelle en France. Il vous permet de conforter un projet existant ou de vous donner des idées.
  Soyez curieux et consultez les différentes rubriques, les fiches métiers, les quizz, les revues etc.
  L’ONISEP publie également des brochures thématiques sur un milieu ou un métier. Si un document vous intéresse, regardez s’il est disponible sur e-sidoc (accessible depuis l’ENT) et n’hésitez pas à aller au CDI pour l’emprunter. 
\end{doc}

\begin{doc}{Réaliser un curriculum vitae (CV)}
  Un curriculum vitae (CV), sert à informer les futurs employeurs ou employeuses de vos formations et expériences professionnelles passées. 
  Un curriculum vitae contient des informations 
  \begin{listePoints}
    \item sur vous : nom, prénom, âge ;
    \item sur vos contacts : téléphone et adresse mail ;
    \item sur votre formation : vos diplômes et certification en général (bac, licence, master, etc.) ;
    \item sur vos expériences professionnelles passées en rapport avec le stage ou l'emploi dans lequel vous postulez ;
    \item quelques informations personnelles, si elles sont en rapport avec le stage ou l'emploi (exemple : si on postule dans une librairie, on peut préciser qu'on lit régulièrement).
  \end{listePoints}
  Le format est libre, mais vous pouvez vous inspirer du format du CV présenté. Votre CV \important{doit être envoyé à votre professeur au format PDF sur sa messagerie de l’ENT en fin de séance.}
\end{doc}

\begin{doc}{Réaliser une lettre de motivation}
  Une lettre de motivation vise à montrer à une personne qui ne vous connaîtrait pas votre intérêt pour une formation, un emploi, un logement, une table-ronde etc. Elle doit être courte et synthétique, mais doit aussi aborder des points de votre personnalité (centres d’intérêt, projets etc.) et montrer votre connaissance du sujet pour lequel vous postulez. Vous pouvez vous inspirer de votre lettre de 3e mais démarquez-vous-en quand même.
  
  Il est attendu \important{une production personnelle} (les logiciels de détection d’intelligence artificielle sont très performants).
  \begin{listePoints}
    \item \important{Attendus :} choisir une entreprise, un cabinet, une association (réelle ou fictive) et réaliser votre lettre de motivation en cochant les différents attendus suivants : coordonnées personnelles (à gauche), coordonnées de votre interlocuteur (à droite), un objet, une date, une formule de politesse, le corps du texte (présentation, intérêt, candidature), formule de politesse finale, signature (à gauche).
    \item \important{Format :} le texte doit être justifié, rédigé en Arial 12, interligne 1,5 et \important{être envoyé à votre professeur au format PDF sur sa messagerie de l’ENT en fin de séance.} 
  \end{listePoints}
\end{doc}

% \begin{doc}{Faire un point sur la première période de l’année de 2nde}
%   Nous sommes en décembre soit à un tiers de l’année. Pour faire le point sur vos difficultés, vœux ou questionnements, je vous demande de consulter le document en pièce-jointe sur la rubrique « cahiers de texte » et de le télécharger sur votre ordinateur en créant un dossier « orientation ». Consultez-le, complétez-le et renvoyez-le à votre professeur sur la messagerie de l’ENT.
% \end{doc}