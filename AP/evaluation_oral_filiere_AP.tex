\pasDePagination
\begin{center}
  \important{Grille d'évaluation de la tâche finale AP: Orientation}
\end{center}
\vspace*{-20pt}

\begin{center}
\begin{tblr}{
  colspec = {X[c,m] X[c,m] X[c,m] X[c,m] X[c,m] X[c,m]},
  vlines, hlines,
  row{1} = {couleurPrim!20}
}
    \important{Noms \newline \vspace*{10pt}\strut } &
    0-1 & 2 & 3 & 4 & Note \hfill /20 \\
    Qualité de la forme. & 
    Production confuse, peu soignée. & 
    Production assez soignée mais peu de variété dans les supports. &
    Production soignée et variée au niveau des supports. &
    Production originale et soignée. &
    \hfill /4 \\
    %
    Message et qualité informative. & 
    Peu d'information fournies, message pas clair.	&
    Message un peu confus et des informations partielles sur la filière choisie.	& 
    Informations clés pertinentes et un message clair. & 
    Message très clair appuyé par des informations recherchées, pertinentes et précises. &
    \hfill /4 \\
    %
    Vocabulaire. & 
    Vocabulaire imprécis et peu spécifique. &
    Vocabulaire assez limité. &
    Vocabulaire suffisamment développé et adapté. &
    Vocabulaire large et parfaitement adapté. &
    \hfill /4 \\
    %
    Respect des consignes. &
    Production qui ne respecte pas les consignes. &
    Production qui respecte certaines consignes. &
    Production qui respecte la plupart des consignes. &
    Production qui respecte toutes les consignes. &
    \hfill /4 \\
    %
    Maîtrise du sujet. &
    Présentation qui n’apporte aucune information précise et reste flou sur la filière. &
    Des erreurs fréquentes, avec des lacunes dans la présentation de la filière. &
    Des imprécisions, mais qui ne nuisent pas à la compréhension.	&
    Une production précise et détaillée dans la présentation de la filière et de ses débouchés. &
    \hfill /4 \\
    %
\end{tblr}
\end{center}