%%%% début de la page
\sndEnTeteDeux

%%
\nomPrenomClasse


%%%% titre
\numeroActivite{2}
\titreActivite{Chute d'une balle}


%%%% objectifs
\begin{objectifs}
  \item Comprendre la notion de vecteur vitesse.
  \item Tracer des vecteurs vitesses.
\end{objectifs}


%%%% evaluation
\begin{tableauCompetences}
  \centering S'Approprier (APP) &
  Représenter la situation par un schéma.
  & & & &
  \\ \hline
  \centering Communication (COM) &
  Travailler en groupe, échanger entre élèves.
  & & & &
\end{tableauCompetences}


%%
\vspace*{6pt}
\problematique{Quelle est l'influence d'une translation sur la description du mouvement d'un objet ?}


%%%% documents
\begin{doc}{Chronophotographie de la chute d'une balle}
  \label{doc:chrono}
  \vspace*{-20pt}
  \begin{center}
    \image{0.9}{images/mouvements/chronophotographie_balle.png}
  \end{center}
  Une chronophotographie est une superposition de plusieurs images prises les unes après les autres avec un intervalle de temps régulier.
  Pour réaliser cette chronophotographie, \textbf{on a pris une image toutes les} $\mathbf{40} \unit{\mathbf{ms}}$.
  \bigskip
  
  Réaliser une chronophotographie permet de repérer des positions par lesquelles passent la balle, ce qui est impossible à l'oeil nu.
  Les ronds indiquent les positions de la balle, les carrés indiquent les positions du centre de masse de l'homme sur la trottinette.
\end{doc}

%%
\begin{doc}{Vecteur}
  \label{doc:vecteur}
  \vspace*{-24pt}
  \begin{encart}
    \important{Vecteur} : objet mathématique représenté par un segment fléché $\longrightarrow$ et noté avec une lettre surmontée d'une flèche $\vv{v}$.
    
    Un vecteur contient quatre information : une \important{direction}, un \important{sens}, une \important{norme,} et un \important{point d'application} (ou origine).
  
    Un vecteur est \important{constant} si sa direction, son sens et sa norme ne varie pas le long du mouvement.
  \end{encart}
\end{doc}

%%
\begin{doc}{Vecteur déplacement et vecteur vitesse d'un point}
  \label{doc:vitesse}
  Soient $P_1$ la position d'un point à l'instant $t_1$ et $P_3$ la position de ce même point à l'instant $t_3$.
  Le déplacement du point matériel entre les dates $t_1$ et $t_3$ est défini par le vecteur déplacement $\vv{P_1 P_3}$.
  Graphiquement, c'est la flèche qui relie $P_1$ à $P_3$. 
  \bigskip
  
  Le vecteur $\vv{P_1 P_3}$ est caractérisé par
  \begin{listePoints}
    \item une direction : celle de la droite $P_1 P_3$.
    \item Un sens : de $P_1$ vers $P_2$.
    \item Une norme : égale à la distance $P_1 P_3$ en mètre (m).
    \item Une origine : le point $P_1$
  \end{listePoints}
  
  \begin{encart}
    Le \important{vecteur vitesse} $\vv{v_2}$ d'un système au point $P_2$ entre les instants $t_1$ et $t_3$ a pour expression
    \begin{equation}
      \vv{v_2} = \frac{\vv{P_1 P_3}}{t_3 - t_1}
    \end{equation}
  \end{encart}
  
  Le vecteur $\vv{v_2}$ est caractérisé par :
  \begin{listePoints}
    \item une direction : parallèle au segment $P_1 P_3$ et tangent à la trajectoire.
    \item Un sens : le sens du mouvement.
    \item Une norme : $v_2 
    = \norm{\vv{v_2}}
    = \displaystyle \norm{\frac{\vv{P_1 P_3}}{t_3 - t_1}}
    = \displaystyle \frac{P_1 P_3}{t_3 - t_1}$.
    \item Une origine : $P_2$.
  \end{listePoints}
  $P_1 P_3$ est la distance entre les points $P_1$ et $P_3$ en mètre (m). $t_3 - t_1$ est la durée séparant les instants $t_1$ et $t_3$ en seconde (s). $v_2$ est la norme de la vitesse en mètre par seconde (m/s).
\end{doc}



%%%%
\newpage
\titreSection{Mouvement dans le référentiel de \lignePointillee{0.2}}

%
\exo{
  Quel est le référentiel utilisé pour décrire le mouvement de la balle et de l'homme sur la trottinette ici ?
}{1}


%%
\titreSousSection{Mouvement de l'homme sur la trottinette}

%
\exo{
  Quelle est la trajectoire de l'homme sur la trottinette ?
}{1}

%
\exo{
  Comment évolue la vitesse de l'homme sur la trottinette ? Indiquer la nature de son mouvement.
}{2}


%%%%
\titreSousSection{Mouvement de la balle}

%
\exo{
  Repérer sur la chronophotographie du document~\ref{doc:chrono}, le point de départ de la balle.
  On notera $P_1$ cette position.
  Numéroter les positions successives de la balle, que l'on notera $P_2, P_3, \ldots P_8$
}{0}

%
\exo{
  Tracer sur la photo du document~\ref{doc:chrono} le vecteur $\vv{P_2 P_3}$ et le vecteur $\vv{P_5 P_7}$.
}{0}

%
\exo{
  En utilisant l'échelle sur la photo, déterminer les normes en mètre de ces deux vecteurs.
  Indiquer si ces normes sont identiques.
}{2}

%
\exo{
  Calculer la norme en mètre par seconde de ces deux vecteurs, en vous aidant du document~\ref{doc:vitesse}.
}{2}

%
\exo{
  Schématiser le vecteur vitesse $\vv{v_2}$ entre les points $P_2$ et $P_3$ et le vecteur vitesse $\vv{v_6}$ entre les points $P_5$ et $P_7$, en vous aidant du document~\ref{doc:vitesse}.
}{0}

\feuilleBlanche

%%%%
% \newpage
% \titreSection{Mouvement dans le référentiel de la trottinette}
% 
% %
% \exo{
%   Ouvrir la vidéo de la de chute balle dans le logiciel Tracker.
% }{0}
% 
% %
% \exo{
%   Repérer dans la vidéo le moment où la balle commence à tomber. 
% }{0}
% 
% %
% \exo{
%   Sur la vidéo, réaliser le pointage de la balle.
% }{0}
% 
% %
% \exo{
%   Tracer la norme de la vitesse, la vitesse selon l'axe $x$ et la vitesse selon l'axe $y$.
%   Que remarquez vous pour la vitesse selon l'axe $x$ ?
% }{2}
% 
% %
% \exo{
%   Que pouvez-vous en déduire sur la nature du mouvement de la balle dans le référentiel de la trottinette ?
%   Représenter avec un schéma sa trajectoire.
% }{2}
% \vspace{200pt}
% 
% %
% \exo{
%   Conclure sur la position de la balle au moment où elle touche le sol par rapport à la trottinette.
% }{2}