\newpage
\begin{boiteColoree}{100}
  \centering
  \important[white]{\Large Fiche de préparation de TP}

  \important[white]{Répression des fraudes}
\end{boiteColoree}


\begin{center}
  \begin{tblr}{
      colspec = {|l X[l] |l X[l] | l X[l] |},
      width = \linewidth, hlines,
      column{1,3,5} = {couleurPrim!15},
    }
    \textbf{Date:}      & jeudi 19/09, vendredi 20/09
    & \textbf{Heures :} & 10h40 et 15h50
    & \textbf{Salle :}  & A108 (10h40), A103 (15h50) \\
    %
    \textbf{Prof :}      & Alexandre Jedrecy
    & \textbf{Matière :} & Physique-Chimie
    & \textbf{Niveau :}  & Seconde \\
  \end{tblr}
\end{center}


\begin{boiteMateriel}{Matériel élève}
  \textbf{Effectif :} 15
  \qq{}\qq{}
  \flecheLongue \textbf{5 groupes} de 3 élèves

  \begin{multicols}{2}
    \begin{protocole}
      \item 1 pipette jaugée de 10 mL.
      \item 1 poire de prélèvement.
      \item 1 éprouvette graduée de 50 mL.
      \item 2 bécher de 50 mL (1 si y en a pas assez).
    \end{protocole}
  \end{multicols}
\end{boiteMateriel}


\begin{boiteMateriel}{À préparer}
    \begin{protocole}
      \item 2 balances précises à 0,1 g (+ alim adaptés)
      \item 1 solution de 200 mL d'éthanol à 70 $\%$ (ou tout autre $\%$ si tu en as un autre déjà prêt).
      \item 2 bécher de 500 mL (ou 250 mL).
      \item 1 bouteille de sirop.
    \end{protocole}
\end{boiteMateriel}