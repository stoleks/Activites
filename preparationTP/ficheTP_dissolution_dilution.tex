\newpage
\begin{boiteColoree}{100}
  \centering
  \important[white]{\Large Fiche de préparation de TP}

  \important[white]{Dilution d’un produit désinfectant}
\end{boiteColoree}


\begin{center}
  \begin{tblr}{
      colspec = {|l X[l] |l X[l] | l X[l] |},
      width = \linewidth, hlines,
      column{1,3,5} = {couleurPrim!15},
    }
    \textbf{Date(s) :}  & jeudi 19/09 
    & \textbf{Heures :} & 13h40 (jeudi), 15h50 (vendredi)
    & \textbf{Salle(s) :} & A108 (jeudi), A103 (vendredi) \\
    %
    \textbf{Matière :}  & Physique-Chimie
    & \textbf{Niveau :} & Seconde 
    & \textbf{Prof :}   & Alexandre Jedrecy \\
  \end{tblr}
\end{center}


\begin{boiteMateriel}{Matériel élève}
  \textbf{Effectif :} 15
  \qq{}\qq{}
  \flecheLongue \textbf{5 groupes} de 3 élèves

  \begin{multicols}{2}
    \begin{protocole}
      \item 1 sabot de pesée
      \item 1 fiole jaugée de 50 mL
      \item 1 bécher de 100 mL
      \item 1 pipette pasteur
      \item 1 pissette d’eau du robinet
    \end{protocole}
  \end{multicols}
\end{boiteMateriel}


\begin{boiteMateriel}{À préparer}
  \begin{protocole}
    \item 1 bouteille de sirop
    \item 2 pipette jaugée de 10 mL + 1 verre à pied
    \item 1 bêcher 50 mL
    \item 1 bêcher 100 mL
    \item 1 éprouvette graduée 50 mL
    \item 1 paquet de sucre
    \item 1 pissette d’eau du robinet
    \item 2 balances (+ alim adapté) avec une précision de 0,1 g
  \end{protocole}
\end{boiteMateriel}