\begin{boiteMateriel}{\Large Fiche de préparation de TP}
\end{boiteMateriel}

\begin{center}
  \begin{tblr}{
      colspec = {|l X[l] |l X[l] |}, hlines, width = \linewidth,
    }
    \textbf{Date :}     & 15/11/23 et 16/11/23 
    & \textbf{Niveau :} & Terminale ST2S \\
    %
    \textbf{Matière :}  & Physique-Chimie
    & \textbf{Salle :}  & A108 (mercredi) et A104 (jeudi) \\
    %
    \textbf{Heure :}    & 12h40-14h40
    & \textbf{Prof :}   & Alexandre Jedrecy \\
    %
    \textbf{Titre :} & Fraicheur d'un lait
    & & \\
  \end{tblr}
\end{center}

\begin{boiteMateriel}{Matériel élève}
  \separationBlocs{
    \textbf{Effectif :} 15
  }{
    \flecheLongue \textbf{5 groupes} de 3 élèves
  }

  \begin{listePoints}
    \item Burette et son support
    \item Agitateur magnétique
    \item barreau aimanté
    \item Erlen-meyer 250 mL
    \item Bécher 100 mL
  \end{listePoints}
\end{boiteMateriel}

\begin{boiteMateriel}{À préparer}
  \begin{listePoints}
    \item \qty{500}{\ml} de solution d'hydroxyde de sodium $c = \qty{0,05}{\mol\per\litre}$.
    \item Bouteille de lait > \qty{0,5}{\litre}.
    \item Phénolphtaléine (ou bleu de bromothymol idéalement, mais j'ai l'impression qu'il n'y en a plus...)
  \end{listePoints}
\end{boiteMateriel}