%%%%%%%%%%%%%%%%%%%%%%%%%%%%%%%%%%%%%%%%%%%%%%%%%%%%%%%%%%%%%
%%%% figures simples
\newcommand{\tkzRect}[4]{
  \fill[color=#1] (#2,#4) -- (-#2,#4) -- (-#2,#3) -- (#2,#3);
}
\newcommand{\tkzEllipse}[4]{
  \fill[color=#1] (0,#3) ellipse (#2 and #4);
}
\newcommand{\tkzLegende}[4]{
  \draw[black, ->, very thick] (#2 + #4, #3) node[right] {#1} -- (#2, #3);
}
\newcommand{\tkzCercle}[4]{
  \filldraw [#3] (#1, #2) circle (#4pt);
}
\newcommand{\tkzCercleLigne}[5]{
  \filldraw [color = #4, fill = #3, very thick] (#1, #2) circle (#5pt);
}

%%%% tube à essais
\newcommand{\tkzTubeEssai}[3]{
  \draw[thick] (#1,#2) -- (#1,0) arc (0:-180:#1) -- (-#1,#2);
  \draw[thick] (0,#2) ellipse (#1 and #3);
}
\newcommand{\tkzBasTubeEssai}[5]{
  \fill[color=#1] (-#2,#3) -- (#2,#3) arc (0:-180:#2);
  \tkzRect{#1}{#2}{#3 - 0.01}{#4}
  \tkzEllipse{#1!85!black}{#2}{#4}{#5}
}
\newcommand{\tkzPhaseTubeEssai}[5]{
  \tkzRect{#1}{#2}{#3}{#4}
  \tkzEllipse{#1}{#2}{#3}{#5}
  \tkzEllipse{#1!85!black}{#2}{#4}{#5}
}

%%%% Point et vecteurs
\newcommand{\tkzLabel}[3]{
  \node at (#1, #2) {#3};
}
\newcommand{\tkzPointLabel}[3]{
  \filldraw (#1, #2) circle (2pt) node[above] {#3};
}
\newcommand{\tkzVecteur}[6]{
  \draw[black, ->, very thick] (#1, #2) -- (#1 + #3, #2 + #4) node[#6] {#5};
}
\newcommand{\tkzEquiv}[6]{
  \draw[black, <->, thick] (#1, #2) -- (#1 + #3, #2 + #4);
  \draw[black] (#1 + #3/2, #2 + #4/2) node[#6] {#5};
}
\newcommand{\tkzVecteurX}[4]{
  \draw[black, ->, very thick] (#1, #2) -- (#1 + #3, #2) node[above] {#4};
}
\newcommand{\tkzVecteurY}[4]{
  \draw[black, ->, very thick] (#1, #2) -- (#1, #2 + #3) node[right] {#4};
}
\newcommand{\tkzEquivY}[5]{
  \draw[#1, <->, thick] (#2, #3) -- (#2, #3 + #4);
  \draw[#1] (#2, #3 + #4/2) node[right] {#5};
}
\newcommand{\tkzEquivX}[5]{
  \draw[#1, <->, thick] (#2, #3) -- (#2 + #4, #3);
  \draw[#1] (#2 + #4/2, #3) node[below] {#5};
}


%%%%%%%%%%%%%%%%%%%%%%%%%%%%%%%%%%%%%%%%%%%%%%%%%%%%%%%%%%%%%
%%%% plan de classe
\newcommand{\rectangleTexte}[5]{
  \filldraw [fill=white, draw=black, ultra thick] (#1, #2) rectangle (#1 + #3, #2 + #4);
  \node at (#1 + #3/2, #2 + #4/2) [font=\sffamily] {\textbf{\large #5}};
}
% place dans la classe
\newcommand{\place}[3]{
  \rectangleTexte{#1}{#2}{3}{2}{#3}
}
\newcommand{\deuxPlaces}[4]{
  \place{#1}{#2}{#3}
  \place{#1 + 3}{#2}{#4}
}
\newcommand{\troisPlaces}[5]{
  \place{#1}{#2}{#3}
  \place{#1 + 3}{#2}{#4}
  \place{#1 + 6}{#2}{#5}
}
\newcommand{\quatrePlaces}[6]{
  \place{#1}{#2}{#3}
  \place{#1 + 3}{#2}{#4}
  \place{#1 + 6}{#2}{#5}
  \place{#1 + 9}{#2}{#6}
}
%%%% rangée
\newboolean{quatrePlace}
\setboolean{quatrePlace}{false}
\newcommand{\avecQuatrePlaces}{ \setboolean{quatrePlace}{true} }
\newcommand{\avecTroisPlaces} { \setboolean{quatrePlace}{false} }
\newcommand{\rangee}[9]{
  \ifthenelse {\boolean{quatrePlace}} {
    \deuxPlaces  {0}{#1} {#2}{#3}
    \quatrePlaces{7}{#1} {#4}{#5}{#6}{#7}
    \deuxPlaces  {20}{#1}{#8}{#9}
  }{
    \deuxPlaces {#1}{#2}     {#3}{#4}
    \troisPlaces{#1 + 7}{#2} {#5}{#6}{#7}
    \deuxPlaces {#1 + 17}{#2}{#8}{#9}
  }
}
\newcommand{\rang}[9]{
  \ifthenelse {\boolean{quatrePlace}} {
    \deuxPlaces  {0} {12 - 3*#1} {#2}{#3}
    \quatrePlaces{7} {12 - 3*#1} {#4}{#5}{#6}{#7}
    \deuxPlaces  {20}{12 - 3*#1} {#8}{#9}
  }{
    \deuxPlaces {0} {12 - 3*#1} {#2}{#3}
    \troisPlaces{7} {12 - 3*#1} {#4}{#5}{#6}
    \deuxPlaces {17}{12 - 3*#1} {#7}{#8}
  }
}

%%%% TP
\newcommand{\paillasse}[3]{
  \rectangleTexte{#1}{#2}{4}{2}{#3}
}
\newcommand{\rangeeTP}[6]{
  \paillasse{#1}{#2}     {#3}
  \paillasse{#1 + 4}{#2} {#4}
  \paillasse{#1 + 12}{#2}{#5}
  \paillasse{#1 + 16}{#2}{#6}
}