%%%%
\teteSndAtom

%%%% titre
\vspace*{-36pt}
\numeroActivite{5}
\titreActivite{Le cortège électronique}


%%%% Objectifs
\vspace*{-8pt}
\begin{objectifs}
  \item Comprendre la structure du cortège électronique.
  \item Comprendre la règle de remplissage des couches électroniques.
\end{objectifs}

\begin{contexte}
  Un atome est constitué d'un noyau positif entouré d'électrons négatifs, avec autant d'électrons que de protons, l'atome étant neutre.
  
  \problematique{
    Comment les électron s'organisent autour du noyau ?
  }
\end{contexte}


%%%% Documents
\begin{doc}{Rangement des électrons}
  Quand on s'appelle Hydrogène et qu'on a qu'un électron, pas besoin de ranger ses affaires.
  Mais quand on s'appelle Uranium et qu'on en a 92 autour de soi, mieux vaut mettre un peu d'ordre dans ses électrons !
  
  C'est en 1913 que Bohr a eu l'idée de répartir les électrons d'un atome en différentes couches et sous-couches, en se basant sur les travaux de Planck.
  
  \begin{wrapfigure}{r}{0.45\linewidth}
    \centering
    \vspace*{-16pt}
    \image{0.82}{images/atome/schema_couche}
    {\small Schéma des premières couches}
  \end{wrapfigure}
  
  Les couches électroniques sont numérotées $\mathbf{1, 2, 3,\ldots}$ alors que les sous couches sont repérées par des lettres : \important{s} ou \important{p}.
  Les sous-couches ne peuvent contenir qu'un nombre limité d'électrons.
  
  Ainsi, la \important{sous-couche s} ne pourra accueillir que \important{2 électrons} au maximum, alors que la \important{sous-couche p} ne pourra accueillir que \important{6 électrons} au maximum.
  
  La couche qui accueille les derniers électrons s'appelle la couche externe, les autres couches sont appelées les couches internes.
\end{doc}

\begin{doc}{Remplissage des couches électroniques}
  \vspace*{-16pt}
  \begin{wrapfigure}[6]{l}{0.2\linewidth}
    \centering
    \vspace*{-8pt}
    \image{0.6}{images/atome/klechkowski}
  \end{wrapfigure}
  Le remplissage des couches et des sous-couches se fait par ordre croissant de couches (1 puis 2 puis 3) et par ordre croissant de sous-couches (s puis p).
  
  Il suffit alors de suivre les flèches comme sur la figure de gauche.
  La première couche est la seule à ne pas posséder de couche p.
  Cette règle de remplissage s'appelle \textbf{la règle de Klechkowski}.
  
  Pour les premières couches, l'ordre de remplissage est donc \important{1s} \flecheLongue \important{2s} \flecheLongue \important{2p} \flecheLongue \important{3s} \flecheLongue \important{3p}.  On appelle \important{configuration électronique} la donnée du nombres d'électrons dans chaque couches et sous-couches. \textit{Exemple :} la configuration électronique de l'atome d'oxygène \isotope{}{8}{O} est 1s$^2$ 2s$^2$ 2p$^4$.
\end{doc}


%%%%
\newpage
\vspace*{-24pt}
\question{
  Compléter le tableau ci-dessous pour résumer l'occupation des différentes couches électroniques 
}{0}

\vspace*{-12pt}
\begin{center}
  \begin{tabular}{| l | c | c | c | c | c |}
    \hline \rowcolor{gray!20}
    \centering Couche &
    1 & \multicolumn{2}{c|}{2} &  \multicolumn{2}{c|}{3} \\ \hline
    Sous-couche &
      \hspace{30pt} & \hspace{30pt} & \hspace{90pt} &
      \hspace{30pt} & \hspace{90pt} \\ \hline
    Nombre maximal d'électrons & & & & & \\ \hline
  \end{tabular}
\end{center}

\question{
  L'atome de Silicium \chemfig{Si} possède $Z = 14$ protons.
  Schématiser ci-dessous la répartition de ses électrons.
}{0}

\begin{center}
  \image{0.4}{images/atome/schema_silicium}
\end{center}

\question{
  Donner la configuration électronique de l'atome de Silicium
}{1}

\question{ 
  Indiquer, en justifiant, la couche externe de cet atome de Silicium, ainsi que la ou les couches internes.
}{3}

\question{
  Reprendre les questions 3 et 4 pour l'atome de Carbone \chemfig{C} ($Z = 6$). Quelles différences et ressemblances avec le Silicium peut-on remarquer ?
}{4}