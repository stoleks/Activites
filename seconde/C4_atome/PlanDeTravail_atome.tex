\teteSndAtom

\vspace*{-40pt}
\titre{Plan de Travail -- \sndAtom}
\vspace*{-8pt}

%\begin{importants}
  % Le plan de travail est un cadre de travail collectif où tu as la liberté d'avancer, seul-e ou en groupe, à ton rythme.
  Ce document présente les activités et travaux pratiques à réaliser pendant les 4 semaines du chapitre.
  À chaque séance (en classe entière ou demi-groupe), tu es libre de choisir quelle activité ou TP réaliser avec ton groupe.
  Tous les documents sont imprimés sur le bureau du professeur.
  % Au début de la 2ème et 3ème semaine, une courte interrogation sera réalisé sur certaines activités.
%\end{importants}


%%%% Activités
\titre{Activités à réaliser}
\vspace*{-16pt}

\begin{multicols}{2}
  \phantom{\methode}\vspace*{-64pt}
  \begin{activite}{Ordres de grandeur}{ordre_grandeur}
    \begin{objectifs}  
      \item Revoir les puissances de 10.
      \item Apprendre à raisonner en ordres de grandeur.
    \end{objectifs}
  \end{activite}

  \phantom{\sndAtom}\vspace*{-44pt}
  \begin{TP}{Fabriquer un atome}[1 h 30]{atome}
    \begin{objectifs}
      \item Étudier la composition d'un atome.
      \item Comprendre que le nombre de protons définit un élément chimique.
      \item Savoir distinguer un ion d'un atome.
      \item Comprendre la notion d'éléments isotopes.
    \end{objectifs}
  \end{TP}
  
  \begin{TP}{Le modèle de l'atome}{modele_atome}
    \begin{objectifs}
        \item Découvrir la méthode scientifique.
        \item Utiliser la méthode scientifique pour étudier l'évolution du modèle de l'atome.
    \end{objectifs}
  \end{TP}
  
  \begin{activite}{Taille d'un atome}{taille_atome}
    \begin{prerequis}
      \item Calcul avec les puissances de 10.
      \item Utilisation des ordres de grandeur.
    \end{prerequis}
    \begin{objectifs}
      \item Comparer la taille d'un atome à des objets du quotidien pour mieux la comprendre.
      \item Utiliser les ordres de grandeurs pour mener un raisonnement.
    \end{objectifs}
  \end{activite}
\end{multicols}

\begin{multicols}{2}    
  \begin{activite}{Cortège électronique}[1 h 30]{cortege_electrons}
    \begin{prerequis}
      \item Connaître la structure d'un atome.
      \item Savoir qu'un atome a autant d'électrons qu'il a de protons.
    \end{prerequis}
    %
    \begin{objectifs}
      \item Comprendre que les électrons s'organisent en couches électroniques.
      \item Comprendre la règle de remplissage des couches électroniques.
    \end{objectifs}
  \end{activite}

  \begin{TP}{Le Tableau périodique}{tableau_periodique}
    \begin{prerequis}
      \item Connaître la structure électronique.
      \item Savoir remplir les couches et sous-couches électronique d'un atome.
    \end{prerequis}
    \begin{objectifs}
      \item Comprendre la construction du tableau périodique.
    \end{objectifs}
  \end{TP}
\end{multicols}

\vspace*{-2cm}
\begin{tikzpicture}
  [overlay, remember picture, line width=1.5mm, draw=couleurQuat]
    \draw[->, rounded corners=4mm] 
      (ordre_grandeur) 
      to (5, 15.5) to (8, 15.2) 
      to (taille_atome);
    \draw[->] (atome) -- (cortege_electrons);
    \draw[->, rounded corners=5mm] 
      (cortege_electrons) 
      to (11.5, -1) 
      to (tableau_periodique);
\end{tikzpicture}

\vspace*{1.5cm}
Note : les flèches indiquent un ordre entre certaines activités.
Idéalement, il faut avoir fait l'activité d'où part la flèche avant de faire l'activité où arrive la flèche.


%%%% Progression
\newpage
\nomPrenomClasse
\titre{Progression des activités}
\vspace*{24pt}

\flecheProgression{
  (0,  10.) -- (17, 10.) --
  (17, 7.5) -- (0,  7.5) --
  (0,  5.0) -- (17, 5.0) --
  (17, 2.5) -- (0,  2.5) --
  (0,  0)   -- (17, 0);
}
\vspace*{-12.8 cm}

\begin{programmeSeance}
  \seance{2 h}{}
  \seance{1 h}{}
  \seance{2 h}{}
\end{programmeSeance}
\vspace*{1.2 cm}

\begin{programmeSeance}
  \seance{1 h}{
    Courte évaluation sur la structure d'un atome.
  }
  \seance{2 h}{}
  \seance{1 h}{}
\end{programmeSeance}
\vspace*{1.2 cm}

\begin{programmeSeance}[2]
  \seance{2 h}{
    \strut \\ \centering
    \sousTitre{Tâche finale}
  }
  \seance{1 h}{
    \strut \\ \centering
    \sousTitre{Évaluation du chapitre}
  }
\end{programmeSeance}


%%%% Tâche finale
\begin{tacheFinale}
  Choisir un élément du tableau périodique et réaliser sa case au format $20\times\qty{20}{\cm\squared}$.
  La case devra contenir des informations microscopique (structure électronique) et des informations macroscopique (dans quels objets on trouve l'élément, etc.)
\end{tacheFinale}


%%%% Evaluation
\titre{Évaluation de l'autonomie}

\sousTitre{Les différents degrés d'autonomie}

\begin{enumerate}[label = \Alph*]
  \item Je planifie librement mon apprentissage, je coopère avec mes camarades et je sollicite de l'aide pour valider les travaux réalisés.
  \item Je travaille seul-e ou avec mes camarades à partir des documents et je sollicite régulièrement de l'aide pour avancer.
  \item J'avance uniquement quand le professeur est là pour m'aider, je n'arrive pas à planifier mon travail ou je ne fais que recopier les réponses d'un de mes camarades.
  \item J'utilise des stratégies pour éviter d'apprendre et je refuse d'essayer de faire les activités.
\end{enumerate}

\begin{tableauCompetences}
  AUTO & Travailler de manière autonome 
  & & & & \\
  %
\end{tableauCompetences}