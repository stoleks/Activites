%%%%
\teteSndAtom

%%%% titre
\vspace*{-36pt}
\numeroActivite{4}
\titreActivite{Le cortège électronique}


%%%% Objectifs
\begin{objectifs}
  \item Comprendre la structure du cortège électronique.
  \item Comprendre la règle de remplissage des couches électroniques.
\end{objectifs}

\begin{contexte}
  Un atome est constitué d'un noyau positif entouré d'électrons négatifs, avec autant d'électrons que de protons, l'atome étant neutre.
  
  \problematique{
    Comment les électron s'organisent autour du noyau ?
  }
\end{contexte}


%%%% Documents
\begin{doc}{Rangement des électrons}{doc:A4_cortege_electronique}
  Quand on s'appelle hydrogène et qu'on a qu'un électron, pas besoin de ranger ses affaires.
  Mais quand on s'appelle uranium et qu'on en a 92 autour de soi, mieux vaut mettre un peu d'ordre dans ses électrons !
  
  \begin{wrapfigure}{r}{0.45\linewidth}
    \centering
    \vspace*{-24pt}
    \image{0.82}{images/atomes/schema_couche}
    {\small Schéma des couches et sous-couches électroniques de l'oxygène \isotope{}{8}{O}}
  \end{wrapfigure}
  
  C'est en 1913 que Bohr a l'idée de répartir les électrons d'un atome en différentes couches et sous-couches, en se basant sur les travaux de Planck.
  
  Les couches électroniques sont numérotées \important{1, 2, 3.}
  Les sous couches sont repérées par des lettres : \important{s} ou \important{p}.
  Les sous-couches ne peuvent contenir qu'un nombre limité d'électrons.

  \begin{importants}  
    La \important{sous-couche s} ne peut contenir que \important{2 électrons} au maximum,
    alors que la \important{sous-couche p} ne peut contenir que \important{6 électrons} au maximum.
  \end{importants}
  
  La couche qui accueille les derniers électrons s'appelle \important{la couche externe}, les autres couches sont appelées les \important{couches internes}.
\end{doc}

\begin{doc}{Remplissage des couches électroniques}{doc:A4_remplissage_couche}
  Le remplissage des couches et des sous-couches se fait par ordre croissant de couches (1 puis 2 puis 3) et par ordre croissant de sous-couches (s puis p) dans une couche.
  
  La première couche est la seule à ne pas posséder de couche p.
  Cette règle de remplissage s'appelle \important{la règle de Klechkowski}.
  
  \begin{importants}
    Pour les premières couches, l'ordre de remplissage est
    \begin{center}
      \important{1s} \flecheLongue
      \important{2s} \flecheLongue \important{2p} \flecheLongue
      \important{3s} \flecheLongue \important{3p}
    \end{center}
  \end{importants}
  \begin{importants}  
    On appelle \important{configuration électronique} le remplissage des électrons dans chaque couches et sous-couches.
  \end{importants}
  
  \textit{Exemple :} la configuration électronique de l'atome d'oxygène \isotope{}{8}{O} est 1s$^2$ 2s$^2$ 2p$^4$.
\end{doc}


%%%%
\newpage
\numeroQuestion
Compléter le tableau ci-dessous pour résumer l'occupation des différentes couches électroniques 

\vspace*{-12pt}
\begin{center}
  \begin{tblr}{
    colspec = {| l | X[c] | X[c] | X[c] | X[c] | X[c] |}, hlines,
    row{1} = {couleurPrim!20, c}, column{1} = {couleurPrim!15}
  }
    Couche & 1 & \SetCell[c=2]{c} 2 & & \SetCell[c = 2]{c} 3 & \\
    Sous-couche & & & & & \\
    Nombre maximal d'électron & & & & & \\
  \end{tblr}
\end{center}

\mesure
L'atome de silicium \chemfig{Si} possède $Z = 14$ protons.
Schématiser ci-dessous la répartition de ses électrons.

\begin{center}
  \image{0.4}{images/atomes/schema_couche}
\end{center}

\question{
  Donner la configuration électronique de l'atome de silicium. \label{que:A4_configuration}
}{}{2}

\question{ 
  Indiquer, en justifiant, le nom de la couche externe de cet atome de silicium, ainsi que la ou les couches internes. \label{que:A4_couches}
}{}{3}

\question{
  Reprendre les questions \ref{que:A4_configuration} et \ref{que:A4_couches} pour l'atome de Carbone \chemfig{C} ($Z = 6$). Quelles différences et ressemblances avec le silicium peut-on remarquer ?
}{}{5}