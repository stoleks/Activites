%%%%
\teteSndAtom

%%%% titre
\numeroActivite{1}
\titreActivite{L'élément chimique}


%%%% Objectifs
\begin{objectifs}
  \item Apprendre la composition d'un atome.
  \item Comprendre la différence entre ion et atome.
\end{objectifs}

\begin{contexte}
  Au cours du \siecle{19}, la communauté scientifique considérait que l'atome était la plus petite \og brique \fg\, de la matière.
  Au début du \siecle{20}, deux expériences vont montrer que l'atome est composé de particules plus élémentaires :
  \begin{listePoints}
    \item en 1897, Thomson montre que l'on peut arracher des particules de charges négatives d'un atome ;
    \item en 1911, Rutherford montre que l'atome possède un noyau très petit devant la taille d'un atome, avec une charge positive.
  \end{listePoints}
  
  \problematique{
    Quelles entités composent les atomes ?
  }
\end{contexte}


%%%%
\titreSection{L'atome}

%%%%
\numeroQuestion
Légender cette représentation d'un atome en utilisant les mots proton, neutron, électron, nucléons et noyau.

\begin{center}
  \image{0.8}{images/atomes/atome.png}
\end{center}

\begin{wrapfigure}[1]{r}{0.1\linewidth}
  \vspace*{-60pt}
  \qrcode{https://phet.colorado.edu/sims/html/build-an-atom/latest/build-an-atom_fr.html}
\end{wrapfigure}

\mesure Scanner le qrcode pour accéder à l'animation.

\question{
  Dans l'application le cadre \og symbole \fg\, indique l'élément chimique fabriqué.
  Que faut-il ajouter pour changer d'élément chimique ?
}{}{2}

\begin{doc}{Notation d'un élément chimique}{doc:A1_notation_element}
  Pour distinguer les atomes on utilise la notation \isotope{A}{Z}{X}.
  \begin{encart}
    \begin{listePoints}
      \item \chemfig{X} est le symbole de l'atome considéré.
      \item $Z$ est le nombre de \texteTrou[0.3]{protons}, appelé \important{numéro atomique.}
      \item $A$ est le nombre de \texteTrou[0.3]{neutrons}, appelé \important{nombre de masse.}
    \end{listePoints}
  \end{encart}
\end{doc}

\numeroQuestion 
Compléter le document~\ref{doc:A1_notation_element}.

\question{
  \isotope{23}{11}{Na} : le sodium \chemfig{Na} possède \texteTrou{11} protons, \texteTrou{23} nucléons, \texteTrou{12} neutrons.
}{}{0}


%%%%
\titreSection{Les ions}

\question{
  Vérifier que la case \og Neutralité/Ionisation\fg\, est cochée.
  Dans quel cas un élément chimique est un atome neutre ?
  Comment appelle-t-on cet élément sinon ?
}{}{3}

\question{
  Que signifie le « + » de \chemfig{Na^+} ? Donner la composition de l'élément, c'est-à-dire son nombre de proton, neutron et électrons.
}{}{3}

\question{
  Que peut-on dire de l'ion chlorure \chemfig{Cl^{-}} et de l'ion cuivrique \chemfig{Cu^{2+}} ?
}{}{3}


%%%%
\titreSection{Les isotopes}

\question{
  Vérifier que la case \og Stabilité/Instabilité \fg\, est cochée.
  Deux atomes du même élément peuvent-ils avoir des noyaux différents ?
}{}{3}

\question{
  Que manque-t-il à l'élément \isotope{2}{2}{He} pour être stable ?
}{}{2}