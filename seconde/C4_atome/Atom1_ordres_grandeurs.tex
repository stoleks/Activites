%%%%
\teteSndAtom

%%%% titre
\numeroActivite{1}
\titreActivite{Ordre de grandeur}


%%%%
\vspace*{-24pt}
\titreSection{Notation scientifique}

%%
\begin{doc}{Les puissances de 10}{doc:A1_puissance_10}
  \begin{encart}
  \begin{listePoints}
    \item Écrire le nombre $10^n$ (avec $n = 0, 1, 2, 3, \ldots$), revient à écrire ``$1$'' suivi de $n = 0, 1, 2, 3, \ldots$ zéros. \textit{Exemple : $10^3 = 1000$}
    \item Écrire le nombre $10^{-n}$ (avec $n = 1, 2, 3, \ldots$), revient à écrire ``$0,$'' suivi de $n - 1 = 0, 1, 2, \ldots$ zéros et d'un $1$. \textit{Exemple : $10^{-2} = 0,\!01$}
    \item $10^a \times 10^b = 10^{a + b}$
    \item $\Frac{1}{10^n} 
    = \Frac{10^{-n}}{10^{-n}} \times \frac{1}{10^n} 
    = \Frac{10^{-n}}{10^{n - n}}
    = \Frac{10^{-n}}{10^0}
    = 10^{-n}$
  \end{listePoints}
  \end{encart}
\end{doc}
\bigskip

\begin{doc}{Moyen mnémotechnique}{doc:A1_decalage_virgule}
  \begin{listePoints}
    \item Si je décale la virgule de 1 rang vers la gauche, alors \texteTrouAuto{je réduis de 1 unité} la puissance de dix.
    \item Si je décale la virgule de 1 rang vers la droite, alors \dotfill \\
    de 1 unité la puissance de dix.
  \end{listePoints}
\end{doc}

\begin{doc}{La notation scientifique}{doc:A1_notation_scientifique}
  \begin{encart}
  La notation scientifique d'une quantité se présente de la façon suivante :
  \begin{equation*}
    \texteEncadre{chiffre différent de zéro}
    \;,\;
    \texteEncadre{autres chiffres} 
    \; \vphantom{\frac{1}{10}}^{\times} \;
    \texteEncadre{puissance de dix}
    \;
    \texteEncadre{\important{unité}}
  \end{equation*}
  \end{encart}
\end{doc}

\numeroQuestion Écrire les quantités suivantes en notation scientifique :
  
\separationDeuxBlocs{
  $288 \unit{h} = \dotfill$ \\[4pt]
  $1 \unit{m} = \dotfill$ \\[4pt]
  $756\, 864\, 000 \unit{s} = \dotfill$ \\[4pt]
  $638 \unit{N} = \dotfill$
}{
  $0,\!1 = \dotfill$ \\[4pt]
  $0,\!9997 \unit{g/mL} = \dotfill$ \\[4pt]
  $0,\!436 \unit{s} = \dotfill$ \\[4pt]
  $0,\!336 \unit{s} = \dotfill$
}


%%
\titreSection{Les ordres de grandeurs}

\begin{doc}{Définition d'un ordre de grandeur}{doc:A1_def_ordre_grandeur}
  \begin{wrapfigure}[3]{r}{0.1\linewidth}
    \vspace*{-32pt}
    \qrcode{https://www.youtube.com/watch?v=xTV47tuv_Fg}
  \end{wrapfigure}

  \vAligne{-36pt}
  \begin{encart}
    L'ordre de grandeur d'une quantité est la puissance de 10 la plus proche de cette quantité.
  \end{encart}
  %
  \exemple L'ordre de grandeur de \qty{60}{\s} est \qty{e2}{\s} (60 est plus proche de 100 que de 10). 
\end{doc}

\numeroQuestion Donner l'ordre de grandeur des quantités suivantes :

\separationDeuxBlocs{
  $3,\!00 \cdot 10^8 \unit{m.s^{-1}} = \dotfill$ \\[4pt]
  $1,\!67 \cdot 10^{-27} \unit{kg} = \dotfill$
}{
  $9,\!11 \cdot 10^{-31} \unit{kg} = \dotfill$ \\[4pt]
  $53 \cdot 10^{-12} \unit{m} = \dotfill$
}


%%%%
\titreSection{Le système international de mesure}

%%
\vspace*{-12pt}
\titreSousSection{Le système international}

Pour comparer des grandeurs entre elles, il faut les exprimer avec les \important{mêmes unités de mesures}. % exemple centime et euros

Pour pouvoir communiquer facilement d'un pays à un autre, le \important{système international (SI)} a été développé par la Conférence Générale des Poids et Mesures (CGPM). % histoire des sciences système métrique

Le système international est composé de \important{sept unités de base,} que l'on retrouve quotidiennement. Une part importante de nos technologies modernes dépendent de la précision avec laquelle ces unités sont définies.

\begin{center}
  \begin{tblr}{
    hlines, row{1} = {couleurPrim!20}, colspec = {|c |c |c |}
  }
    Grandeur             & Unité      & Symbole de l'unité \\
    Masse                & kilogramme & kg \\
    Temps                & seconde    & s \\
    Longueur             & mètre      & m \\
    Température          & kelvin     & K \\
    Quantité de matière  & mole       & mol \\
    Intensité électrique & ampère     & A \\
    Intensité lumineuse  & candela    & cd
  \end{tblr}
\end{center}


%%
\titreSousSection{De l’échelle microscopique à l’échelle astronomique}

\numeroQuestion
Compléter le tableau en associant à chaque objet sa longueur, puis l'ordre de grandeur de cette longueur. Pour ça, utilisez six de ces huit longueurs (attention aux unités !) :
%
\begin{center}
  \begin{tblr}{c}
  \qty{10e16}{\m} &
  \qty{6400}{\km} &
  \qty{10e20}{\m} &
  \qty{0,1}{\nm} &
  \qty{60}{\micro\m} &
  \qty{6}{\mm} &
  \qty{1000}{\km} &
  \qty{10e12}{\m}
  \end{tblr}
\end{center}

\begin{tblr}{
  colspec = {|X[-1] |X[1] |X[1] |X[1] |X[1] |X[1] |X[1] |},
  hlines, columns = {c}, row{1} = {couleurPrim!20, m}, width = \linewidth
}
  Objet &
  Épaisseur cheveux & Voie Lactée & Système solaire &
  Hexagone & Fourmi & Atome \\
  % 
  \vAligne{25pt} Image & 
  \image{1}{images/taille_objet/taille_cheveux} &
  \image{1}{images/taille_objet/taille_galaxie} &
  \image{1}{images/taille_objet/taille_systeme_solaire} &
  \image{1}{images/taille_objet/taille_france} &
  \image{1}{images/taille_objet/taille_fourmi} &
  \image{1}{images/taille_objet/taille_atome} \\
  %
  Taille & \vphantom{$\Frac{1}{1}$} & & & & & \\
  %
  Ordre de grandeur & \vphantom{$\Frac{1}{1}$} & & & & & \\
\end{tblr}