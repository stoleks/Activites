%%%%
\teteSndTran

%%%% titre
\numeroActivite{2}
\titreActivite{Transformations nucléaires et production d'énergie électrique}

%%%% Objectifs
\begin{objectifs}
  \item Connaître l'écriture symbolique d'une transformation nucléaire
  \item Comprendre la différence entre fission et fusion nucléaire.
  \item Comprendre dans les grandes lignes le fonctionnement d'une centrale électrique.
\end{objectifs}

\begin{contexte}
  Nos sociétés modernes sont gourmandes en énergies et notamment en énergie électrique pour faire fonctionner des usines, des trains, internet ou encore pour nous éclairer.
  
  \problematique{
    Comment les réactions nucléaires permettent de produire de l'énergie électrique ?
  }
\end{contexte}


%%%% docs
\begin{doc}{Rappel sur les isotopes}{doc:A1_rappel_isotope}
  \begin{importants}
    Des \important{isotopes} sont des noyaux ayant le même nombre de protons, mais un nombre différents de neutrons.
  \end{importants}
  Deux isotopes ont les mêmes propriétés chimiques, mais leurs propriétés physiques sont différentes.
  \exemple* \isotope{16}{8}{O}, \isotope{17}{8}{O} et \isotope{18}{8}{O} sont des isotopes de l'oxygène.
\end{doc}

%%
\begin{doc}{Radioactivité}
  Sous certaines conditions, un noyau peut spontanément se transformer en émettant des particules très énergétiques.
  C'est la \important{radioactivité}, le noyau est dit radioactif.
  \begin{importants}
    Il existe trois types de radioactivité, par ordre croissant de dangerosité :
    \begin{listePoints}
      \item $\alpha$, avec émission d'un noyau d'hélium \isotope{4}{2}{He};
      \item $\beta$, avec émission d'un électron \chemfig{e^{-}} ou un positron \chemfig{e^+};
      \item $\gamma$, avec émission d'un photon \chemfig{\gamma}.
    \end{listePoints}
  \end{importants}
\end{doc}

%%
\begin{doc}{Fusion et fission nucléaire}{doc:A1_fusion_fission}
  \begin{importants}
    La \important{fission nucléaire} est une transformation où un noyau massif est séparé en deux noyaux plus petit sous l'action d'un neutron \chemfig{n}.
  \end{importants}
  \exemple Fission de l'uranium $\isotope{1}{0}{n} + \isotope{235}{92}{U} \reaction \isotope{94}{38}{Sr} + \isotope{139}{54}{Xe} + 3\isotope{1}{0}{n}$.
  
  \begin{importants}
    La \important{fusion nucléaire} est une transformation où deux noyaux légers s'associent pour former un noyau plus lourd.
  \end{importants}
  \exemple Fusion du deutérium et du tritium au c\oe{}ur d'une étoile
  \begin{equation*}
    \isotope{2}{1}{H} + \isotope{3}{1}{H} \reaction \isotope{4}{2}{He} + \isotope{1}{0}{n}
  \end{equation*}
  
  \qrcodeCote{https://www.youtube.com/watch?v=1MUcizMqVAc}
  \phantom{b}\vspace*{-12pt}
  
  \begin{importants}
    La fusion et la fission sont des \important{transformations exothermiques}.
  \end{importants}

  Pour plus de détails :
\end{doc}

%%
\begin{doc}{Fonctionnement d'une centrale nucléaire à fission}{doc:A1_principe_centrale}
  \qrcodeCote{https://youtu.be/pFgTPZpjiqs?t=15}
  Une centrale nucléaire à fission est une machine thermique, qui fonctionne sur le même principe qu'une centrale à charbon ou à gaz.

  La réaction de fission génère de la chaleur, qui sert à chauffer de l'eau pour la transformer en vapeur.
  Cette vapeur va venir faire tourner un alternateur qui va générer de l'énergie électrique.
  \bigskip

  \qrcodeCote{https://www.youtube.com/watch?v=ScP-uPIEpl8}
  D'un point de vue énergétique, on transforme de l'énergie thermique en énergie mécanique, puis en énergie électrique.
  La conversion de l'énergie thermique en énergie mécanique à un rendement assez faible, de \qty{30}{\percent} à \qty{70}{\percent}.
  En revanche la conversion de l'énergie mécanique en énergie électrique a un rendement supérieure à \qty{95}{\percent}.
\end{doc}


%%
\begin{doc}{Déchet nucléaire}{doc:A1_dechets}
  Lors de la fission de l'uranium, plusieurs noyaux plus légers peuvent être formés.
  Ces noyaux sont souvent instables et donc radioactifs.
  \qty{99}{\percent} des déchets sont sans dangers, car très faiblement radioactif, mais le reste des déchets peuvent être mortels si on y est exposé trop longtemps.
  
  Il est donc important d'entreposer de manière sécurisé ces déchets, ce qui s'avère être un véritable casse-tête : aucun pays au monde n'a de solutions fiable sur le long terme pour stocker les déchets les plus dangereux.
\end{doc}


%%%% Questions