%%%%
\teteSndTran

%%%% titre
\vspace*{-40pt}
\titreActivite{Rester frais l'été}

%%%% Objectifs
\begin{objectifs}
  \item Comprendre pourquoi l'évaporation de l'eau rafraîchit.
\end{objectifs}

\begin{contexte}
  Les étés sont de plus en plus chaud. Pour se refroidir efficacement, il faut comprendre l'impact des changements d'états courants dans la vie quotidienne.
  
  \problematique{
    Quels changements d'états physiques permettent de diminuer la température ?
  }
\end{contexte}


%%%% docs
\begin{doc}{Un peu de vocabulaire}{doc:A1_vocabulaire}
  Quand on s'intéresse à l'évolution de la température et des états d'un objet, on fait de la \important{thermodynamique} (« mouvement de la chaleur » en grec).
  
  \begin{importants}
    \begin{listePoints}
      \item \important{Corps :} objet macroscopique avec des propriétés mesurable (température, pression).
      \item \important{Système :} ensemble de corps dont on étudie l'évolution.
      \item \important{Milieu extérieur :} tous les corps qui ne sont pas le système.
    \end{listePoints}
  \end{importants}
\end{doc}

\begin{doc}{Transfert thermique}{doc:A1_transfert_thermique}
  \begin{importants}
    Un corps chaud en contact avec un corps froid lui transfert de l'énergie, ce qui se traduit par une modification de la température des deux corps : on parle de \important{transfert thermique}.
  \end{importants}
  L'énergie transférée se note $Q$, son unité est le Joule \unit{\joule}.
  Un corps qui \important{reçoit un transfert thermique positif} ($Q > 0$) voit \important{sa température augmenter.}
  
  % \attention Le transfert thermique va \important{toujours} du corps chaud vers le corps froid !
  
  \begin{importants}
    Sous certaines conditions, ce transfert thermique peut mener un des deux corps à changer d'état (liquide à gaz par exemple) : on parle de \important{transformation physique}.
  \end{importants}
  On note un tel changement d'état comme une réaction chimique avec une flèche, à gauche l'état initial et à droite l'état final.
  \exemple $\eau\sol \reaction \eau\liq$.
\end{doc}

%%
\begin{doc}{Transformations endothermique et exothermique}{doc:A1_endo_exo}
  \begin{wrapfigure}{r}{0.58\linewidth}
     \image{1}{images/thermodynamique/transformation_energie}
  \end{wrapfigure}
  \phantom{b}\vspace*{-20pt}
  
  \begin{importants}
    \pointCyan Lors d'une \important{transformation exothermique}, l'énergie du système diminue. 
    Le milieu extérieur reçoit un transfert thermique positif $Q > 0$.
    \bigskip
    
    \pointCyan Lors d'une \important{transformation endothermique}, l'énergie du système augmente.
    Le milieu extérieur reçoit un transfert thermique négatif $Q < 0$.
  \end{importants}

  \attention Pour le système, le signe du transfert thermique change !
\end{doc}


%%
\begin{doc}{L'éco-climatisation}{doc:A1_climatisation}
  \begin{wrapfigure}{r}{0.3\linewidth}
    \vspace*{-34pt}
    \centering
    \image{1}{images/exterieures/livre_scolaire/eco_climatisation}
  \end{wrapfigure}
  À cause du réchauffement climatique, la consommation d'énergie liée à la climatisation ne fait qu'augmenter, avec un impact fort sur l'environnement.
  
  Des solutions plus écologiques existent : quand de l'air chaud arrive au contact de gouttelettes d'eau liquide, les gouttelettes s'évaporent.
  L'air chaud se refroidit alors rapidement grâce à l'évaporation.

  \important{Système :} les gouttelettes d'eau liquides.
\end{doc}

%%
\begin{doc}{Un glaçon dans ma boisson}{doc:A1_glacons}
  Si on veut refroidir une boisson tiède, on peut la placer dans un réfrigérateur, mais une solution bien plus rapide est de rajouter des glaçons dedans.
  
  Le principe est très simple : en fondant, les glaçons vont absorber de l'énergie, ce qui va refroidir l'eau qui les entoure.

  \important{Système :} les glaçons.
\end{doc}

%%
\begin{doc}{Sueur et fraîcheur}{doc:A1_evaporation}
  Quand l'eau s'évapore, elle passe de l'état liquide à l'état gazeux.
  Ce phénomène absorbe de l'énergie dans l'environnement proche.
  Lorsqu'on est mouillé, le transfert thermique se fait avec notre corps, qui se refroidit alors.

  \important{Système }: les gouttes de sueur.
\end{doc}


%%%% Questions
\question{
  Pour chaque documents (\ref{doc:A1_climatisation}, \ref{doc:A1_glacons}, \ref{doc:A1_evaporation}), indiquer quel est le corps qui change d'état, avec l'état initial et l'état final.
}{
  Pour le document~\ref{doc:A1_climatisation}, ce sont les gouttelettes d'eau qui passent de l'état liquide à l'état gazeux.

  Pour le document~\ref{doc:A1_glacons}, ce sont les glaçons d'eau qui passent de l'état solide à l'état liquide.

  Pour le document~\ref{doc:A1_evaporation}, c'est les gouttes de sueur, qui passent de l'état liquide à l'état gazeux.
}[5]

\question{
  Pour chaque documents, indiquer si la transformation physique est endothermique ou exothermique, en donnant le signe du transfert thermique $Q$ reçu par le milieu extérieur.
}{
  Dans tous les cas, les transformations sont endothermique, avec un transfert thermique négatif.
}[4]

\question{
  Pour chaque documents, écrire la notation symbolique du changement d'état.
}{
  Climatisation et sueur : \eau\liq \reaction \eau\gaz

  Glaçons : \eau\sol \reaction \eau\liq
}[3]