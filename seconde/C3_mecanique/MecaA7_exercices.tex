%%%% début de la page
\teteSndMouv


%%%% titre
\numeroActivite{7}
\titreActivite{Vol d'oie et saut en parachute}


%%%% objectifs
\begin{objectifs}
  \item Remobiliser les notions de référentiel, forces, vitesses
  \item Utiliser le principe d'inertie pour calculer des forces
\end{objectifs}


%%%%
\begin{doc}{Référentiel terrestre}{doc:referentiel_terrestre}
  \begin{encart}
    Sur Terre on utilise souvent le \important{référentiel terrestre} pour étudier des mouvements. Ce référentiel est lié à la surface de la Terre.
  \end{encart}
  C'est le référentiel auquel on fait spontanément référence quand on mesure une vitesse de déplacement.
\end{doc}


%%%%
\exercice{Vol d'une oie}

%%
\begin{doc}{Vol d'oie et portance}{doc:A6_vol_oie}
  \begin{center}
    \image{0.5}{images/mecanique/oie}
  \end{center}
  
  
  On considère que deux forces s'exercent sur une oie qui plane avec un mouvement rectiligne uniforme : son poids et la portance de l'air.
  L'étude se fait dans le référentiel terrestre et on néglige les forces de frottements ($\vv{f} \approx \vv{0})$.

  \textbf{Données :}
  \begin{listePoints}
    \item Masse de l'oie $m = \qty{400}{\g}$.
    \item Accélération de la pesanteur terrestre $g = \qty{9,81}{\newton \per\kg}$.
  \end{listePoints}
\end{doc}

\question{
  Les forces exercées sur l'oie se compensent-elles ? Justifier.
}{}{0}

\question{
  En déduire une relation entre les valeurs de ces deux forces.
}{}{0}

\question{
  Calculer la norme du poids P de l'oie.
}{}{0}

\question{
  En déduire la norme de la force de portance $F_\text{air}$.
}{}{0}

\question{
  Représenter la situation sur un schéma, en modélisant l'oie par un point matériel et en représentant les forces qui s'exercent sur elle, sans souci d'échelle.
}{}{0}


%%%%
\newpage
\exercice{Saut en parachute}

%%
\begin{doc}{Freinage d'un parachute à l'ouverture}{doc:A6_vitesse_parachute}
  \begin{wrapfigure}{r}{0.45\linewidth}
    \vspace*{-24pt}
    \begin{center}
      \image{1}{images/donnees/norme_vitesse_parachute}
      \small{
        Vitesse du système en fonction du temps.
      }
    \end{center}
  \end{wrapfigure}
  
  Une parachutiste saute sans vitesse initiale d'un hélicoptère en vol stationnaire.
  Après quelques secondes en chute libre, elle ouvre son parachute.
  Les frottements dus à l'air sur la toile s'expriment par une force opposée au mouvement. 
  
  Dans ce cas la norme de cette force est proportionnelle au carré de la vitesse
  \begin{equation*}
    f = k \times v^2
  \end{equation*}
  avec $f$ la force de frottements, $k$ le coefficient de frottements et $v$ la vitesse du système.

  \textbf{Données :}
  \begin{listePoints}
    \item Masse du système (parachutiste + parachute) $m = \qty{90}{\kg}$.
    \item Accélération de la pesanteur terrestre $g = \qty{9,81}{\newton \per\kg}$.
  \end{listePoints}
\end{doc}


%%
\question{
  Décrire les différentes phases du mouvement, la trajectoire étant tout le temps rectiligne.
}{}{0}

%%
\question{
  Comment varie la norme du vecteur vitesse entre 0 et \qty{15}{\s} ? Commenter.
}{}{0}

%%
\question{
  À quelle(s) force(s) est soumis le système entre 0 et \qty{12}{\s} ?
}{}{0}

%%
\question{
  Lorsque le parachute est ouvert, $k = \qty{10}{\newton \s\squared \per\m\squared}$.
  Calculer l'intensité (la norme) de la force de frottements à l'instant où la parachutiste ouvre son parachute.
}{}{0}

%%
\question{
  Expliquer le mouvement à partir de l'instant $t = \qty{16}{\s}$.
}{}{0}

%%
\question{
  Calculer la valeur du coefficient de frottements $k$ à l'instant $t = \qty{20}{\s}$.
}{}{0}



\begin{doc}{Vitesse de chute libre}
  Pour un objet tombant dans le vide sans vitesse initiale, sa vitesse au moment de toucher le sol vaut
  \begin{equation*}
    v = \sqrt{2\cdot g \cdot h}
    \qq{ou}
    h = \dfrac{v^2}{2 \cdot g}
  \end{equation*}
  où $g$ est l'accélération de pesanteur terrestre et $h$ la hauteur du point de chute.
\end{doc}

%%
\question{
  En utilisant la relation entre $h$ et la vitesse $v$, calculer de quelle hauteur tomberait la parachutiste avec sa vitesse à l'instant $t = \qty{20}{\s}$.
  Calculer la hauteur pour la vitesse à l'instant $t = \qty{12}{\s}$.
  Conclure sur l'intérêt du parachute.
}{}{0}