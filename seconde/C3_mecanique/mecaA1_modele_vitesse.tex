%%%% début de la page
\teteSndMouv

%%%% titre
%\nomPrenomClasse
\numeroActivite{1}
\titreActivite{Modéliser le mouvement}


%%%% objectifs
\vspace{-10pt}
\begin{objectifs}
  \item Comprendre les limites du modèle du point matériel.
  \item Comprendre la notion de référentiel.
  \item Comprendre la notion de vecteur.
\end{objectifs}


%%%% evaluation
% \begin{tableauCompetences}
%   COM & Travailler en groupe, échanger entre élèves.
%   & & & &
% \end{tableauCompetences}


%%%%
\titreSection{Système et référentiel}

%%%%
\vspace{-10pt}
\begin{doc}{Modèle du point matériel}{doc:A1_point_materiel}
  \begin{encart}
    \important{Système} : objet dont on étudie le mouvement.
  
    On ne va s'intéresser qu'au mouvement global du système.
    C'est pourquoi on va modéliser le système par
    \texteTrouLignes{
      Un point de même masse, localisé en son centre de masse.
    }
    \texteTrou{
      C'est le \important{modèle du point matériel.}
    }
  \end{encart}

  \fleche Le modèle du point matériel revient à ignorer toute information sur la géométrie du système étudié. 
  Les éventuelles rotations et déformations ne sont donc pas prises en compte.
\end{doc}


\begin{tblr}{
    colspec = {X[1.5,c,m] | X[1,c,m] | X[2,c] | X[2,c] },
    row{1} = {couleurPrim!20}
  }
  Système & Centre de masse & Trajectoire & Informations perdues \\ \hline
  %
  {\image{1}{images/mecanique/point_balle_tennis.png} \\ Balle de tennis} &
  Centre de la balle & & \\ \hline
  %
  {\image{1}{images/mecanique/point_roue.jpg} \\ Roue} &
  Centre de la roue & & \\ \hline
  %
  {\image{1}{images/mecanique/point_humain_course.jpg} \\ Modèle d'humain} &
  Nombril & &
\end{tblr}

%%
\begin{doc}{Référentiel}{doc:A1_referentiel}
  Pour décrire le mouvement, il faut pouvoir le repérer dans l’espace et dans le temps, pour ça on utilise un référentiel.
  
  \begin{encart}
    \important{Référentiel} : \texteTrouLignes[1]{
      objet de référence, muni d'un repère par rapport auquel on étudie le mouvement du système.
    }
  \end{encart}
  
  \begin{encart}
    La description du mouvement dépend du \important{référentiel} choisi.
    On appelle ça la \important{relativité} du mouvement.
  \end{encart}
\end{doc}


%%%%
\titreSection{Vecteur}

%%
\begin{doc}{Vecteur en physique}{doc:A1_vecteur}
  \begin{encart}
    \important{Vecteur} : objet mathématique représenté par un segment fléché $\longrightarrow$ et noté avec une lettre surmontée d'une flèche $\vv{v}$.
    
    Un vecteur contient quatre information : 
    \begin{listePoints}
      \item \texteTrou{Une direction}
      \item \texteTrou{Un sens}
      \item \texteTrou{}
      \item \dotfill 
    \end{listePoints}
  
    Un vecteur est \important{constant} si
    \texteTrouLignes[1]{sa norme, sa direction et son sens sont constants.}
  \end{encart}
  
  \fleche En physique on va se servir des vecteurs pour représenter différentes grandeurs :
  \texteTrouLignes[1]{vitesse, force, champ électromagnétique, aimantation}
  
  \attention Un vecteur n'est \textbf{jamais} égal à un nombre, qui contient moins d'information.
\end{doc}