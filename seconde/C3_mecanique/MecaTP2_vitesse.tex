%%%% début de la page
\teteSndMouv

%%
\nomPrenomClasse


%%%% titre
\numeroActivite{2}
\titreTP{Chute d'une balle}


%%%% objectifs
\begin{objectifs}
  \item Comprendre la notion de vecteur vitesse.
  \item Tracer des vecteurs vitesses.
\end{objectifs}


%%%% evaluation
\begin{tableauCompetences}
  APP &
  Représenter une situation par un schéma.
  & & & & \\
  COM &
  Travailler en groupe, échanger entre élèves.
  & & & &
\end{tableauCompetences}


%%
\vspace*{6pt}
\problematique{Quelle est l'influence d'une translation sur la description du mouvement d'un objet ?}


%%%% documents
\begin{doc}{Chronophotographie de la chute d'une balle}{doc:TP2_chrono}
  \begin{center}
    \image{0.9}{images/mecanique/chronophotographie_balle.png}
  \end{center}
  La superposition de plusieurs images prises les unes après les autres avec un intervalle de temps régulier est une \important{chronophotographie}.
  Pour réaliser cette chronophotographie, \textbf{on a pris une image toutes les} $\mathbf{\qty{40}{\ms}}$.
  Une chronophotographie permet de repérer des positions par lesquelles passent la balle.
  
  Les ronds indiquent les positions du centre de la balle, les carrés indiquent les positions du centre de masse de l'homme sur la trottinette.
\end{doc}

%%
\begin{doc}{Vecteur}{doc:TP2_vecteur}
  \begin{importants}
    \important{Vecteur} : objet mathématique représenté par un segment fléché $\longrightarrow$ et noté avec une lettre surmontée d'une flèche $\vv{v}$.
    
    Un vecteur contient quatre information : 
    \begin{multicols}{2}
      \begin{listePoints}
        \item une \important{direction}
        \item un \important{sens}
        \item une \important{norme}
        \item une \important{origine}
      \end{listePoints}
    \end{multicols}
  
    Un vecteur est \important{constant} si sa direction, son sens et sa norme ne varie pas pendant le mouvement.
  \end{importants}
\end{doc}

%%
\newpage
\vspace*{-40pt}
\begin{doc}{Vecteur déplacement}{doc:TP2_deplacement}
  Soient $P_1$ la position d'un point à l'instant $t_1$ et $P_3$ la position de ce même point à l'instant $t_3$.
  Le déplacement du point matériel entre les dates $t_1$ et $t_3$ est défini par le vecteur déplacement $\vv{P_1 P_3}$.
  Graphiquement, c'est la flèche qui relie $P_1$ à $P_3$. 
  
  Le vecteur $\vv{P_1 P_3}$ est caractérisé par
  \vspace*{-8pt}
  \begin{multicols}{2}
  \begin{listePoints}
    \item Une origine : le point $P_1$.
    \item une direction : celle de la droite $P_1 P_3$.
    \item Un sens : de $P_1$ vers $P_3$.
    \item Une norme : la distance $P_1 P_3$ en mètre \unit{m}.
  \end{listePoints}
  \end{multicols}
\end{doc}

\begin{doc}{Vecteur vitesse}{doc:TP2_vitesse}
  \begin{importants}
    Le \important{vecteur vitesse} $\vv{v_2}$ d'un système au point $P_2$ entre les instants $t_1$ et $t_3$ a pour expression
    \begin{equation*}
      \vv{v_2} = \frac{\vv{P_1 P_3}}{t_3 - t_1}
    \end{equation*}
  \end{importants}
  
  Le vecteur $\vv{v_2}$ est caractérisé par :
  \vspace*{-8pt}
  \begin{multicols}{2}
  \begin{listePoints}
    \item Une origine : $P_2$.
    \item une direction : parallèle au segment $P_1 P_3$ et tangent à la trajectoire.
    \item Un sens : le sens du mouvement.
    \item Une norme : $v_2 
    %= \norm{\vv{v_2}}
    = \displaystyle \norm{\frac{\vv{P_1 P_3}}{t_3 - t_1}}
    = \displaystyle \frac{P_1 P_3}{t_3 - t_1}$.
  \end{listePoints}
  \end{multicols}
  
  $P_1 P_3$ est la distance entre les points $P_1$ et $P_3$ en mètre \unit{\m}.
  $t_3 - t_1$ est la durée séparant les instants $t_1$ et $t_3$ en seconde \unit{\s}.
  $v_2$ est la norme de la vitesse en mètre par seconde \unit{\m/\s}.
\end{doc}



%%%%
% \titreSection{Mouvement dans le référentiel de la caméra}

%%
\vspace*{-4pt}
\titreSousSection{Mouvement de l'homme sur la trottinette}

%
\question{
  Quelle est la trajectoire de l'homme sur la trottinette ?
}{}{1}

%
\question{
  Comment évolue la vitesse de l'homme sur la trottinette ? Décrire son mouvement.
}{}{1}


%%%%
\titreSousSection{Mouvement de la balle}

%
\mesure
Repérer sur la chronophotographie du document~\ref{doc:TP2_chrono}, le point de départ de la balle.
On notera $P_1$ cette position.
Numéroter les positions successives de la balle, que l'on notera $P_1, P_2, P_3, \ldots$

%
\mesure
Tracer sur la photo du document~\ref{doc:TP2_chrono} le vecteur $\vv{P_2 P_3}$ et le vecteur $\vv{P_5 P_7}$.

%
\question{
  En utilisant l'échelle sur la photo, déterminer les normes en mètre de $\vv{P_1 P_3}$ et $\vv{P_5 P_7}$.
  Indiquer si ces normes sont identiques.
}{}{2}

%
\mesure
Schématiser le vecteur vitesse $\vv{v_2}$ entre les points $P_1$ et $P_3$
et le vecteur vitesse $\vv{v_6}$ entre les points $P_5$ et $P_7$,
en vous aidant du document~\ref{doc:TP2_vitesse}.

%
\question{
  Calculer la norme en mètre par seconde de $\vv{v_2}$ et $\vv{v_6}$, en vous aidant du document~\ref{doc:TP2_vitesse}.
}{}{3}



%%%%
% \newpage
% \titreSection{Mouvement dans le référentiel de la trottinette}
% 
% %
% \question{
%   Ouvrir la vidéo de la de chute balle dans le logiciel Tracker.
% }{0}
% 
% %
% \question{
%   Repérer dans la vidéo le moment où la balle commence à tomber. 
% }{0}
% 
% %
% \question{
%   Sur la vidéo, réaliser le pointage de la balle.
% }{0}
% 
% %
% \question{
%   Tracer la norme de la vitesse, la vitesse selon l'axe $x$ et la vitesse selon l'axe $y$.
%   Que remarquez vous pour la vitesse selon l'axe $x$ ?
% }{2}
% 
% %
% \question{
%   Que pouvez-vous en déduire sur la nature du mouvement de la balle dans le référentiel de la trottinette ?
%   Représenter avec un schéma sa trajectoire.
% }{2}
% \vspace{200pt}
% 
% %
% \question{
%   Conclure sur la position de la balle au moment où elle touche le sol par rapport à la trottinette.
% }{2}