%%%% début de la page
\teteSndMouv


%%%% titre
\numeroActivite{5}
\titreActivite{Mouvement et forces}


%%%% objectifs
\begin{objectifs}
  \item Comprendre la notion de force, connaître des exemples de forces
  \item Comprendre le lien entre mouvement et force
  \item Comprendre le principe d'inertie
\end{objectifs}
\bigskip


%%%%
\begin{doc}{Force et action mécanique}
  \label{doc:action_force_comp}
  \chevron Un corps exerce une \important{action mécanique} sur le système étudié s’il est capable d’en modifier le mouvement.
  Une action mécanique est modélisée par une \important{force.}

  \begin{encart}
    La force exercée par un corps $A$ sur un corps $B$ est représentée par un vecteur $\vvFAsurB$.
    Ce vecteur possède les caractéristiques suivantes :
    \begin{listePoints}
      \item Une \important{norme} notée $\FAsurB$, qui s'exprime en newton (N).
      \item Une \important{direction} et un \important{sens} qui dépendent de la situation.
      \item Un \important{point d'application} : le centre du système $B$.
    \end{listePoints}
  \end{encart}
\end{doc}
\bigskip

%%
\begin{doc}{Exemples de forces}
  \label{doc:exemples_forces}
  On distingue 2 types d'actions : les \important{actions de contact} (contact entre l’objet qui donne la force et l’objet qui la reçoit), et les \important{actions à distance} (pas de contact).
  \smallskip
  
  \begin{tabularx}{\linewidth}{| m{0.3\linewidth} | m{0.275\linewidth} | m{0.35\linewidth} |}
    \hline
    \rowcolor{gray!20}
    \centering Force &
    \centering Norme &
    Direction, sens 
    \\ \hline
    %
    \centering poids $\vv{P}$ &
    \centering $P = m \times g$ &
    verticale, vers le bas
    \\ \hline
    %
    \centering réaction du support $\vv{R}$ &
    \centering égale à celle du poids : \newline $R = P$ &
    perpendiculaire au support, vers le haut
    \\ \hline
    %
    \centering frottements $\vv{f}$ &
    dépend du cas étudié &
    $\vv{f}$ est opposée à la vitesse $\vv{v}$ (opposée au mouvement)
    \\ \hline
  \end{tabularx}
  \smallskip
  
  \begin{listePoints}
    \item Le poids $\vv{P}$ représente l'interaction gravitationnelle de la Terre.
    \item La réaction du support $\vv{R}$ représente l'action exercée par le support sur un objet posé dessus.
    \item Les frottements $\vv{f}$ représentent l'action d'un milieu (gaz, liquide, support solide) sur un objet qui s'y déplace.
  \end{listePoints}
\end{doc}

%%%%
\begin{tabularx}{\linewidth}
{| m{0.2\linewidth} | m{0.37\linewidth} | m{0.36\linewidth} |}
  \hline
  %
  \rowcolor{gray!20}
  \centering
  Système &
  Ballon &
  Curling
  \\ \hline
  %
  Forces appliquées &
  \phantom{\small b} \newline
  \image{1}{images/mouvements/ballon_football} &
  \phantom{\small b} \newline
  \image{1}{images/mouvements/curling}
  \\ \hline
  %
  \centering Mouvement &
  \centering Immobile &
  \phantom{b} \newline
  \\ \hline
  %
  \rowcolor{gray!20}
  \centering
  Système &
  Parachutiste &
  Skieuse
  \\ \hline
  %
  Forces appliquées &
  \phantom{\small b} \newline
  \image{1}{images/mouvements/parachutiste} &
  \phantom{\small b} \newline
  \image{1}{images/mouvements/skieur}
  \\ \hline
  %
  \centering Mouvement &
  &
  \phantom{b} \newline
  \\ \hline
\end{tabularx}

\begin{figure}[!ht]
  \caption{Représentation des forces pour quelques situations sportives}
  \label{tab:situations_sportives}
\end{figure}


%%%%
\newpage
\vspace*{-36pt}
\titreSection{Forces et mouvement}

%%
\question{
  Parmi les forces $\vv{P}$, $\vv{R}$ et $\vv{f}$, indiquer celles qui sont des forces de contact et celles qui sont des forces à distance.
}{3}

\question{
  En vous aidant des documents~\ref{doc:action_force_comp} et~\ref{doc:exemples_forces}, compléter la figure~\ref{tab:situations_sportives} :
}{0}

\vspace*{-6pt}
\sousQuestion{Sur chaque système étudié, schématiser avec des flèches la ou les forces entrant en jeu, en faisant attention à son point d'application.}{0}

\vspace*{-6pt}
\sousQuestion{Pour chaque système, indiquer son mouvement pour un ou une observatrice extérieure (trajectoire + évolution de la vitesse).}{0}


%%%%
\titreSection{Principe d'inertie}

\question{
  Répondre par vrai ou faux en justifiant à l'aide d'exemples ou de contre exemples.
}{0}

\vspace*{-6pt}
\sousQuestion{Si un objet est en mouvement, alors il est forcément accéléré.}{2}

\sousQuestion{Si un objet est en mouvement, alors il subit une force dans le sens du mouvement.}{2}

\sousQuestion{Si deux objets sont animés par les mêmes forces, alors ils suivent la même trajectoire.}{2}

\question{
  On dit que deux forces se compensent si leur sommes vectorielle est nulle.
  Pour quels systèmes de la figure~\ref{tab:situations_sportives} les forces se compensent-elles ?
}{2}

\question{
  Quel est le mouvement du système dans chaque cas où les forces se compensent ?
}{2}



\begin{doc}{Conclusion : Le principe d'inertie}
  \phantom{b}
  \vspace*{-12pt}
  
  \chevron Le \important{principe d'inertie} a été formulé pour la première fois par Newton en 1687.
  Newton s'appuyait sur les travaux de Descartes et de Galilée, et parfois on appelle ce principe la \important{première loi de Newton}.
  Sa formulation moderne est la suivante :
  
  \begin{encart}
    Si les forces qui s'exercent sur un système se compensent, alors ce système est \\[6pt]
    soit \lignePointillee{0.2}, 
    soit en mouvement \dotfill
  \end{encart}
  
  \begin{encart}
    Réciproquement, si un système est \dotfill \\[6pt]
    .\dotfill, alors les forces\\[6pt]
    . \dotfill
  \end{encart}
\end{doc}


%%%%
\titreSection{Variation du vecteur vitesse}

\question{
  Comment varie $\vv{v}$ pour un système qui a un mouvement rectiligne uniforme ?
  En déduire la variation de $\vv{v}$ pour un système soumis à des forces qui se compensent.
}{3}

\begin{doc}{Principe d'inertie et vitesse}
  \vspace*{-24pt}
  \begin{encart}
    Le principe d'inertie dit que si le vecteur vitesse \dotfill \\[6pt]
    au cours de la trajectoire, alors \dotfill \\[6pt]
    \reponse{1}
    %les forces exercées sur le système se compensent.
  \end{encart}
\end{doc}