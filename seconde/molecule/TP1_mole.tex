%%%%
\teteSndMole

%%%% titre
\vspace*{-32pt}
\titreTP{Compter un grand nombre d'entités identiques}


%%%% Objectifs
\begin{objectifs}
  \item Comprendre qu'une \important{espèce chimique} est constituée d'un très (très) grand nombre \important{d'entités chimiques}.
  \item Comprendre l'utilité de compter les entités par paquets.
  \item Comprendre le concept de mole.
\end{objectifs}

\begin{contexte}  
  Les atomes, ions et molécules sont des entités chimique qui composent toute la matière macroscopique qui nous entoure.
  
  \problematique{
    Comment compter les \important{entités chimiques} microscopiques dans une \important{espèce chimique} macroscopique ?
  }
\end{contexte}


%%%%
\titreSection{Compter des entités au quotidien}

%%
\begin{doc}{Des paquets pour mieux compter}{doc:TP1_compter_paquet}
  Au quotidien, de nombreux objets ne sont pas compté à l'unité, mais par \important{paquets.}
  Par exemple, on compte les œufs par douzaines et les feuilles de papier par ramette de 500 feuilles.
  Si on devait compter les feuilles de papier d'une ramette une par une ce serait une sacré corvée !

  On va voir l'intérêt de faire des paquets en comptant des grain de riz.
\end{doc}

\mesure On va peser $N_A = 100$ grains de riz, on note leur masse $m_\text{100 grains}$ = \texteTrou[0.1]{\qty{4}{\g / paquet}}

\question{
  Calculer la masse d'un grain de riz $m_\text{grain}$ à partir de la masse de 100 grains de riz.
}{
  On divise la masse du paquet par le nombre de grains dans le paquet 
  \begin{equation*}
    m_\text{grain} 
    = \dfrac{\qty{4}{\g / paquet}}{\qty{100}{grain / paquet}} 
    = \qty{0,04}{\g / grain}
  \end{equation*}
}[1]

\question{
  À partir de la masse d'un grain de riz, calculer le nombre $N$ de grains de riz dans un sac de riz de \qty{1}{\kg}.
}{
  Cette fois, il faut diviser la masse du sac de riz par la masse d'un grain de riz
  \begin{equation*}
    N = \dfrac{\qty{1000}{\g}}{\qty{0,04}{\g / grain}}
    = \qty{25000}{grain}
  \end{equation*}
  
}[2]

\question{
  Calculer le nombre $n$ de paquets de 100 grains de riz qu'il y a dans \qty{1}{\kg} de riz.
}{
  Il faut diviser la masse du sac de riz par la masse d'un paquet
  \begin{equation*}
    n = \dfrac{\qty{1000}{\g}}{\qty{4}{\g/ paquet}}
    = \qty{250}{paquet}
  \end{equation*}
  On peut aussi diviser le nombre de grain de riz par la taille d'un paquet
  \begin{equation*}
    n = \dfrac{\qty{25000}{grain}}{\qty{100}{grain / paquet}}
    = \qty{250}{paquet}
  \end{equation*}
}[2]


%%%%
\titreSection{Compter des entités en chimie}

%%
\begin{doc}{Masse d'une entité}{doc:A3_masse_entite}
  La masse d'une entité composée de plusieurs atomes est égale à la somme des masses des atomes de l'entité.
  
  \exemple 
  $\masseAtom{C_2 H_6 O} = 2\times \masseAtom{C} + 6\times \masseAtom{H} + \masseAtom{O}$
  
  \begin{donnees}
    \item $\masseAtom{H}  = \qty{0,17e-23}{\g}$
    \item $\masseAtom{C}  = \qty{1,99e-23}{\g}$
    \item $\masseAtom{O}  = \qty{2,66e-23}{\g}$
    % \item $\masseAtom{Ca)} = \qty{6,66e-23}{\g}$
  \end{donnees}
\end{doc}

\begin{doc}{Composition du sucre}{doc:A3_composition_sucre}
  Le sucre blanc en poudre ou en cube utilisé en pâtisserie est composée de glucose.
  La glucose est une molécule de formule brute \bruteCHO{6}{12}{6}.
\end{doc}

\question{
  Calculer la masse d'une molécule de glucose $m_\text{glucose}$ à partir de la masse des atomes qui la constitue.
}{
  \begin{align*}
    m_\text{glucose} &= 6\times\masseAtom{C} + 12\times\masseAtom{H} + 6\times\masseAtom{O} \\
    &= (6\times \num{1,99e-23}) + 12\times\num{0,17e-23} + 6\times\num{2,66e-23})\unit{g} \\
    &= \qty{29,9e-23}{\g}
  \end{align*}
}[2]

\question{
  Calculer le nombre $N$ de molécule de glucose dans un sachet de sucre de \qty{1}{\kg}.
}{
  On divise la masse du sachet par la masse d'une molécule de sucre 
  \begin{align*}
    N &= \dfrac{m_\text{sachet}}{m_\text{glucose}} \\
    &= \dfrac{\qty{1e3}{\g}}{\qty{29,9e-23}{\g}} \\
    &= \num{3,34e24}
  \end{align*}
}[2]


\begin{doc}{La mole}{doc:TP1_mole}
  Pour faciliter le comptage, en chimie on regroupe les entités en paquets qu'on appelle \important{mole.}
  \begin{importants}
    Une \important{mole} contient précisément $N_A = \qty{6,02 e23}{\per\mole}$ entités chimiques.
  \end{importants}
  \attention $N_A$ est une constante appelée \important{nombre d'Avogadro}, en hommage au scientifique Aemedeo Avogadro.
  L'unité « \unit{\per\mole} » signifie « par mole », c’est le nombre d'entités dans une mole.
\end{doc}

\question{
  Calculer le nombre $n$, en \unit{\mole}, de paquets de $N_A = \qty{6.02e23}{\per\mole}$ molécules dans un sachet de sucre de \qty{1}{\kg}.
}{
  \begin{equation*}
    n = \dfrac{N}{N_A} = \dfrac{\num{3.34e24}}{\qty{6.02e23}{\per\mole}} = \qty{5.55}{\mole} 
  \end{equation*}
}[3]

\mesure Remplir le tableau ci-dessous avec les grandeurs calculées ou mesurées.

\medskip
\begin{tblr}{
    row{1} = {couleurPrim!20, c}, hlines,
    colspec = {c | X[1] | X[1]},
    row{2-5} = {12mm, c, m},
  }
  Échantillon étudié & Sac de riz & Sachet de sucre \\
  Masse d'une entité       &
  $m_\text{riz} =$\texteTrou*[0.25]{\qty{0.04}{\g}} &
  $m_\text{glucose} =$\texteTrou*[0.25]{\qty{29,9e-23}{\g}} \\
  %
  Nombre d'entités $N$     & \num{25000} & \num{3,34e24} \\
  Taille d'un paquet $N_A$ & \num{100} & \qty{6,02e23}{\per\mole} \\
  Nombre de paquets $n$    & \num{250} & \qty{5,55}{\mole} \\
\end{tblr}


\begin{doc}{La quantité de matière}{doc:TP1_quantite_matiere}
  \begin{importants}
    En chimie le nombre de paquets s’appelle le \important{nombre de moles} ou la \important{quantité de matière.}
    On la note $n$ et son unité dans le système international s’écrit « mol ».
  \end{importants}
\end{doc}