%%%%
\teteSndMole

%%%% titre
\titreActivite{En quête de stabilité : les molécules}


%%%% Objectifs
\begin{objectifs}
  \item Comprendre la liaison covalente et les notions de doublets liants et non-liants.
  \item Comprendre que la stabilité d'une molécule est liée au remplissage de sa couche externe.
  \item Savoir analyser un schéma de Lewis pour expliquer la stabilité d'une molécule.
\end{objectifs}

\begin{contexte}
  En dehors des gaz nobles de la 18$^\text{ème}$ colonne du tableau périodique (\chemfig{He}, \chemfig{Ne}, \chemfig{Ar}, \chemfig{Kr}, etc.), les éléments ont tendance à s'associer spontanément pour former des molécules. 
  %Comme pour la formation des ions, les éléments gagnent en stabilité en complétant leur couche externe, en respectant la règle du duet ou de l'octet.
  
  \problematique{
    Quelles règles régissent la formation des molécules ?
  }
\end{contexte}


%%%% docs
\titreSection{Le modèle de Lewis de la liaison covalente}

\begin{doc}{Électrons de valences}{doc:A2_electron_valence}
  Les éléments ont tendance à s'associer en molécule, afin de gagner en stabilité en complétant leur couches électroniques externes.
  
  \begin{importants}  
    Les électrons de la couche externe sont appelés \important{électrons de valence.}
  \end{importants}  
\end{doc}
  
\begin{doc}{Doublets liants et liaison covalente}{doc:A2_doublet_liant}
  En 1916, Lewis propose un modèle simple pour schématiser la formation des liaisons entre éléments :
  \begin{importants}  
    Les éléments qui s'associent en molécule vont mettre en commun un des électrons de leur couche externe.
    Ces électrons mis en commun forment une paire appelée \important{doublet liant}.
  \end{importants}
  \begin{importants}  
    En partageant leurs électrons les éléments deviennent liés, on parle de \important{liaison covalente.}
  \end{importants}

  
  \exemple Formation de la molécule de dihydrogène \chemfig{H_2} à partir de deux éléments \isotope{}{1}{H} :
  
  \begin{center}
    {\small Schéma des deux éléments hydrogènes liés par un partage d'électron}
    \vspace{8pt}
    
    \image{0.2}{images/molecules/molecule_H2}
  \end{center}

  Pour représenter la molécule, on peut soit donner sa \important{formule brute}, soit son \important{schéma de Lewis :}
  \begin{multicols}{2}
    \begin{center}
      Schéma de Lewis de la molécule \\[4pt]

      {\Large \chemfig{H !\cCouleur{0} H}}
      %\image{0.4}{images/molecules/Lewis_H2}
    \end{center}

    \begin{center}
      Formule brute de la molécule \\[4pt]

      {\Large \chemfig{H_2}}
    \end{center}
  \end{multicols}
\end{doc}

\newpage
\question{
  Rappeler la configuration électronique de l’hydrogène \isotope{}{1}{H}, du carbone \isotope{}{6}{C}, de l’azote \isotope{}{7}{N} et de l'oxygène \isotope{}{8}{O}.
  Identifier pour chacun de ces atomes leurs électrons de valence.
}{
  \isotope{}{1}{H} : $1s^1$
  
  \isotope{}{6}{C} : $1s^2 2s^2 2p^2$
  
  \isotope{}{7}{N} : $1s^2 2s^2 2p^3$
  
  \isotope{}{8}{O} : $1s^2 2s^2 2p^4$
}[5]

\question{
  Donner le nombre d'électrons manquant à chaque élément pour que leur couche externe soit pleine et qu'ils gagnent en stabilité.
}{
  Il manque 1 électrons à l'hydrogène pour remplir la couche 1.
  Il manque 4 électrons au carbone pour remplir la couche 2, 3 à l'azote, 2 à l'oxygène.
}[3]

\question{
  Quelle molécule stable peut-on former à partir d'un carbone et de 4 hydrogènes ?
}{
  Le carbone doit former 4 liaisons pour gagner 4 électrons, et l'hydrogène 1 liaison pour gagner 1 électron.
  Donc on peut former du méthane \methane, \chemfig{H-C (-[3]H) (-[-3] H) -H}.
}[3]

\mesure Construire cette molécule à partir des modèles moléculaires.


\titreSection{Doublets non-liants et liaisons multiples}

\begin{doc}{Doublets non-liants}{doc:A2_doublet_non_liant}
  Lors de la formation d'une molécule, les électrons de valence qui ne sont pas partagés forment des paires appelées \important{doublet non-liant}.
  
  \exemple Formation de la molécule d'eau \chemfig{H_2 O} à partir de 2 atomes \isotope{}{1}{H} et d'un atome \isotope{}{8}{O} :
  \vspace{8pt}
  
  \separationBlocs{
    \begin{center}
      {\small Schéma des trois éléments se partageant des électrons}
      \vspace{8pt}
      
      \image{0.8}{images/molecules/molecule_H2O}
    \end{center}
  }{
    \begin{center}
      {\small Schéma de Lewis des doublets liants et des doublets non-liants (barres du haut)}
      \vspace{20pt}
      
      \image{0.4}{images/molecules/Lewis_H2O}
    \end{center}
  }
\end{doc}

\question{
  Indiquer combien de doublet non-liant la molécule d'eau possède.
}{
  Elle possède deux doublet non liant.
}[2]

\pasCorrection{ \newpage \vspace*{-36pt} }
\begin{doc}{Liaisons multiples}{doc:A2_liaisons_multiples}
  Pour être stables, les éléments peuvent partager plusieurs paires d’électrons et ainsi créer une liaison multiple.
  Celle-ci peut être double, comme dans le cas du dioxygène ; ou triple comme dans le cas du diazote.
  
  \centering
  \separationBlocs{
    \begin{center}
      {\small Schéma de Lewis}
      \vspace{36pt}
      
      \image{0.4}{images/molecules/Lewis_O2}
    \end{center}
  }{
    \begin{center}
      {\small Schéma des deux éléments oxygène se partageant des électrons}
      \vspace{8pt}
      
      \image{0.8}{images/molecules/molecule_O2}
    \end{center}
  }
  \vspace{12pt}
  
  \separationBlocs{
    \begin{center}
      {\small Schéma des deux éléments azote se partageant des électrons}
      \vspace{8pt}
      
      \image{0.8}{images/molecules/molecule_N2}
    \end{center}
  }{
    \begin{center}
      {\small Schéma de Lewis}
      \vspace{36pt}
      
      \image{0.4}{images/molecules/Lewis_N2}
    \end{center}
  }
\end{doc}


%%%% questions
\question{
  Quelles molécules peut-on former à partir d'un carbone, d'un d'oxygène et de plusieurs hydrogènes ?
}{
  Le carbone va former 4 liaisons et l'oxygène 2 liaisons.
  Avec 4 hydrogène on peut former du méthanol \chemfig{CH_4O} 
  \begin{center}  
    \chemfig{H-C (-[3]H) (-[-3] H) -O-H}
  \end{center}
  Avec 2 hydrogène on peut former du méthanal \chemfig{CH_2O}
  \begin{center}
    \chemfig{H-C (-[3]H) =O}
  \end{center}
}[3]

\mesure
Construire cette molécule à partir des modèles moléculaires.

\begin{doc}{Règles de stabilité}{doc:A2_regle_stabilite}
  \begin{importants}
    Pour gagner en stabilité, les éléments peuvent partager les électrons de leur couche externe en créant \important{des liaisons covalentes.}
    
    De cette manière, les éléments \texteTrouLignes[1]{complètent leur couche externe et sont donc plus stables.}
    
    Pour savoir combien de liaisons un élément peut former, il suffit de
    \texteTrouLignes[2]{compter le nombre d'électrons de valence et le nombre d'électrons manquant pour que la couche externe soit complète.}
  \end{importants}
\end{doc}

\qrcodeCote[3]{https://mirage.ticedu.fr/?p=2324}

\telechargement Télécharger l'application mirage.

\mesure Prendre une feuille de molécule, puis la scanner avec l'application pour la visualiser en 3 dimension.
Au dos de la feuille, donner la formule brute de la molécule, son schéma de Lewis et vérifier que tous les éléments ont le bon nombre d'électrons.
