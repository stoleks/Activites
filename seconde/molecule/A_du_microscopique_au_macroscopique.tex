%%%%
\teteSndMole

%%%% titre
\vspace*{-32pt}
\titreActivite{Du microscopique au macroscopique}


%%%% Objectifs
\begin{objectifs}
  \item Savoir utiliser le vocabulaire adapté entre atome, ion et molécule.
  \item Comprendre la différence entre un solide ionique et moléculaire.
  \item Comprendre grossièrement la différence entre un objet inerte et une objet biologique.
\end{objectifs}

\begin{contexte}
  On a vu qu'un atome est composé d'électrons et de nucléons.
  Les atomes peuvent ensuite former des ions ou s'associer en molécules, en respectant les règles de stabilités du duet et de l'octet.
  Les atomes, ions et molécules sont des entités chimiques microscopiques qui composent toute la matière qui nous entoure.

  
  
  \problematique{
    Quelle règles permettent de former des objets macroscopiques à partir d'entités chimiques microscopiques ?
  }
\end{contexte}


%%%% docs
\titreSection{Les espèces chimiques}

\begin{doc}{Entités chimiques}{doc:A_entites_chimiques}
  Il existe trois type d'entités chimiques :
  \begin{listePoints}
    \item les atomes, par exemple le cuivre \chemfig{Cu}.
    \item les ions, par exemple l'ion fluorure \chemfig{F^{-}}.
    \item les molécules, par exemple le méthane \chemfig{CH_{4}}.
  \end{listePoints}
  
  \begin{importants}
    Les ions positifs ($+$) \texteTrouLignes{sont appelés \important{cations}.}
    
    Les ions négatifs ($-$) \texteTrouLignes[1]{sont appelés \important{anions}. On ajoute le suffixe -ure au nom des ions.}
  \end{importants}
\end{doc}

%%
\begin{doc}{Neutralité de la matière}{doc:A3_neutralite_matiere}
  La matière macroscopique qui nous entoure est composé d'un très (très) grand nombre d'entités chimique identiques, plus de \num{e23} dans un dé à coudre d'eau !
  
  \begin{importants}
    Au niveau macroscopique, la matière est électriquement neutre.
    C'est-à-dire que sa charge électrique globale est nulle : on parle \important{d'électroneutralité.}
  \end{importants}
\end{doc}

%%
\begin{doc}{Solide ionique}{doc:A3_solide_ionique}
  \begin{importants}
    Les ions vont toujours s'associer par groupe de charges opposées pour former une espèce neutre appelée \important{solide ionique} ou \important{espèce ionique.}
  \end{importants}
  
  Mis en solution dans de l'eau, les solides ioniques se dissocient en \important{cations} (ions positifs $+$) et en \important{anions} (ions négatifs $-$).
  
  \exemple le sel est composé d'ions sodium \chemfig{Na^{+}} et d'ions chlorure \chlorure, on le note \chemfig{NaCl}.
\end{doc}

\pasCorrection{\newpage \vspace*{-30pt}}
\question{
  Parmi les ions suivants :
    \ionFerIII, \ionPotassium, \chemfig{O^{2-}}, \chlorure, \chemfig{Pb^{2+}}, \sulfate ;
  indiquer lesquels sont des anions et lesquels sont des cations
}{
  Les ions avec des charges positives sont des cations : \ionFerIII,  \ionPotassium, \chemfig{Pb^{2+}}.
  Les ions avec des charges négatives sont des anions : \chemfig{O^{2-}}, \chlorure, \sulfate.
}[2]

\question{
  Associer les cations et les anions précédents pour former des solides ioniques neutres électriquement (charge totale nulle).
}{
  Il faut que l'association des ions donne un solide neutre 
  \begin{center}
    \chemfig{Fe_2O_3}, \chemfig{FeCl_3}, \chemfig{Fe_2}$(\chemfig{SO_4})_3$, \chemfig{KCl}, \chemfig{K_2O}, \chemfig{K_2 SO_4}, \chemfig{PbO}, \chemfig{Pb Cl_2}, \chemfig{PbSO_4}
  \end{center}
}[3]


%%%%
\titreSection{Solides moléculaires et structures biologiques}

\vspace*{-4pt}
\begin{doc}{Solides moléculaires}{doc:A_moleculaire}
  \begin{wrapfigure}{r}{0.20\linewidth}
    \vspace*{-32pt}
    \centering
    \image{1.0}{images/thermodynamique/micro_macro}
  \end{wrapfigure}
  \strut\vspace*{-20pt}
  
  \begin{importants}
    Les molécules ou les atomes vont former des solides, des liquides ou des gaz en fonction des conditions de température et de pression.
    Les solides sont des empilements compact d'entités. Un solide composé de molécules est appelé un \important{solide moléculaire.}
  \end{importants}
  \exemple* l'eau est composé de molécules \chemfig{H_2O}.
  Les tubes en cuivres dans les canalisation sont composé d'atomes de cuivre \chemfig{Cu}.

  La majorité des roches qui existe naturellement sont des solides moléculaires.
\end{doc}

\begin{doc}{Molécule biologiques et êtres vivants}{doc:A_biologique}
  \begin{wrapfigure}[2]{r}{0.12\linewidth}
    \vspace*{-28pt}
    \centering  
    \qrcode{https://youtu.be/l2DBizRGIIU?t=18}
  \end{wrapfigure}
  \strut\vspace*{-20pt}
  
  \begin{importants}
    Certaines molécules à base de carbone peuvent s'associer pour former des structures complexes auto-réplicantes, c'est-à-dire qui peuvent se reproduire.
  \end{importants}
  \exemple* les cellules eucaryotes ou procaryotes sont composées d'une multitude de molécules arrangées de manière très complexe.

  \begin{importants}
    Les cellules eucaryotes peuvent s'associer pour former des structures encore plus complexes : les animaux (mammifères, insectes, oiseaux, poissons, etc.), les plantes ou les champignons.
  \end{importants}

  \vspace*{-14pt}
  \begin{center}
    \image{0.75}{images/proteines/de_la_molecule_aux_tissus}
  \end{center}
\end{doc}