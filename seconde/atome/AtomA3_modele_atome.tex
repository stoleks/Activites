%%%%
\teteSndAtom

%%%% titre
\numeroActivite{3}
\titreActivite{Le modèle de l'atome}


%%%% Objectifs
\begin{objectifs}
  \item Utiliser la méthode scientifique pour comprendre l'évolution d'un modèle.
\end{objectifs}

\begin{contexte}
  La description de la matière a considérablement évolué au cours des 3 derniers millénaires.
  À partir du \siecle{19} une séries d'observations expérimentales ont permis d'affiner le modèle de l'atome.
  
  \problematique{
    Comment la communauté scientifique a établi le modèle de l'atome moderne ?
  }
\end{contexte}


%%%% Documents
\begin{doc}{Savoirs, croyance et opinion}{doc:A3_savoir_croyance}
  En science, on fait la distinction entre un \important{savoir}, une \important{croyance} et une \important{opinion}.

  \begin{listePoints}
    \item
    \important{Un savoir} s'appuie sur des données et des faits objectifs, concrets et rationnels qui peuvent être justifiés, prouvés et qui sont validés \important{collectivement}.
    Chaque savoir peut être continuellement questionné, voire réfuté.
    Les savoirs sont donc en évolution perpétuelle et cherchent à décrire au mieux la réalité.
 
    \item 
    \important{Une croyance} est une certitude individuelle et subjective qui peut reposer sur l'autorité ou sur la confiance, mais qui n'a pas été validée par des observations objectives.
    Une croyance n'est pas justifiée rationnellement et elle ne peut donc pas être réfutée.
    Les croyances sont donc relativement figées et évoluent peu.
  
    \item
    \important{Une opinion} repose sur de multiples fondements, plus ou moins objectifs et rationnels : des savoirs, des croyances, des informations de sources diverses, des vécus individuels ou collectifs, ou encore des données culturelles et sociales.
    Une opinion est personnelle, mais elle peut être débattue, exposée, confrontée, ce qui lui permet souvent d'évoluer.
  \end{listePoints}

  Les savoirs sont des biens communs de l'humanité : ils sont très long à trouver ou à développer, mais très rapide à apprendre et à comprendre !
\end{doc}


% \begin{doc}{« Découverte » de la démarche scientifique}{doc:A3_histoire}
%   Au fil des siècles, les scientifiques, qu'ils ou elles étudient la nature ou les humain-es, ont cherché la meilleure méthode pour étudier un problème réel.

%   Pendant longtemps, sous l'influence des philosophes grecs, les scientifiques du moyen-orient et d'europe préféraient la réflexions aux observations concrètes.
%   Ce n'est qu'au cours du \siecle{17} que \important{l'observation expérimentale répétée} devient au coeur de la démarche scientifique.
%   Les expériences « de pensée » sont remplacées par les expériences réelles, ce qui permet de découvrir un nombre considérable de choses entre le \siecle{17} et le \siecle{20} : comportement de la lumière, électricité, magnétisme, mécanique quantique, chimie organique, etc.
%   \bigskip 

%   Deux éléments sont essentiels dans la \important{démarche scientifique} : 
%   \begin{listePoints}
%     \item réaliser des observations expérimentales ;
%     \item chercher à répéter l'observation de manière indépendante.
%   \end{listePoints}
%   Il vaut donc mieux 100 scientifiques « moyens » que 1 scientifique « génial ».

%   Ainsi, l'explosion du nombre de scientifiques au cours du \siecle{20} à permis d'affiner et d'augmenter les savoirs de manière considérable : il y a plus de papiers scientifiques publiés en une journée en 2023 que pendant tous le moyen-âge !
% \end{doc}



\begin{doc}{La méthode scientifique}{doc:A3_methode_scientifique}
  Pour expliquer le monde dans lequel nous vivons, en science on fait appel à des \important{modèles.} 
  Les modèles permettent de décrire un phénomène, ce sont donc des \important{image simplifiée} de la réalité.

  Pour valider ou améliorer la description d'un phénomène par un modèle, les scientifiques s'appuient sur la \important{démarche scientifique} :
  \begin{enumeration}
    \item Observation d'un phénomène.\competence{RCO}
    \item Formulation d'une problématique.\competence{APP}
    \item Proposition d'hypothèses, choix d'un modèle de description.\competence{ANA/RAI}
    \item Réalisation d'observations « expérimentales » pour tester les hypothèses et le modèle.\competence{REA}
    \item Analyse des résultats à l'aide du modèle choisi.\competence{VAL}
    \item Communication des observations et des résultats.\competence{COM}
    \item Réplication et validation collective des observations.
  \end{enumeration}

  \flecheLongue On change de modèle si une observation expérimentale le contredit.
  \bigskip

  \begin{wrapfigure}[0]{r}{0.1\linewidth}
    \vspace*{-90pt}
    \qrcode{https://fr.wikipedia.org/wiki/Biais_cognitif}
  \end{wrapfigure}
  Un des objectifs central de la démarche scientifique, c'est de diminuer certains biais propres à notre cerveaux.
  % C'est pour ça que les deux dernières étapes sont très importantes, pour que la réplication des observations puissent être réalisé par des équipes indépendantes.
\end{doc}


\newpage
\vspace*{-36pt}
\begin{doc}{Quelques observations expérimentales}{doc:A3_observations_exp_atome}
  \begin{listePoints}
    \item \textbf{1783 :} Lavoisier observe que lors d'une réaction chimique il n'y a pas de perte de matière.
    %« Rien ne se perd, rien ne se crée, tout se transforme ».
    Il décompose l'eau en deux composants qu'il nomme l'oxygène et l'hydrogène. 
    %L'hydrogène vient du grec « \important{hydro} » (eau) et « \important{gene} » (engendrer).
    \item \textbf{1897 :} Thomson observe que l’on peut arracher des particules de charges négatives d’un atome.
    Il nomme ces particules \important{électrons.}
    \item \textbf{1900 :} Planck observe que les échanges d'énergies entre lumière et matière sont \important{quantifiés.}
    C'est-à-dire que les échanges n'ont lieu que si la lumière a certaines énergies bien précises.
    \item \textbf{1911 :} Rutherford observe que l'atome possède un noyau très petit devant la taille d’un atome, avec une charge positive.
    Il nomme les particules de charges positives composant le noyau \important{protons.}
    \item \textbf{1927 :} Davisson et Germer observent que les électrons sont \important{délocalisés} dans un \important{cortège électronique.}
  \end{listePoints}
\end{doc}

\begin{doc}{Quelques modèles de l'atome}{doc:A3_modeles_atomes}
  \separationBlocs{
    \centering
    \image{0.4}{images/atomes/modele_sphere_dure} \\
    A : Sphère dure pleine et indivisible.
    \vspace*{12pt}

    \image{0.4}{images/atomes/modele_bohr} \\
    C : Comme B, mais les orbites sont \important{quantifiées} à des distances bien définies et on les appelle couches, avec du vide entre deux couches. Découvert en 1913.
  }{
    \centering
    \vspace*{-12pt}
    \image{0.4}{images/atomes/modele_orbite} \\
    B : Noyau positif avec des électrons négatifs qui orbitent autour.
    
    \image{0.4}{images/atomes/modele_plum_pudding} \\
    D : Atome neutre avec des électrons négatifs qui baignent dans un volume chargés positivement.
  }
  
  \separationBlocs{
    \centering
    \image{0.225}{images/atomes/modele_quantique} \\
    E : Noyau positif avec un \important{cortège électronique} organisé en couches appelées orbitale.
    Les électrons sont \important{délocalisés} dans ces couches : tout se passe comme si les électrons étaient à plusieurs endroits en même temps. 
  }[0.88]{
    \vspace*{-26pt}
    \qrcode{https://youtu.be/fhaZeqzTVjo}
  }[0.1]
\end{doc}


%%%%
\numeroQuestion
À l'aide des documents~\ref{doc:A3_observations_exp_atome} et~\ref{doc:A3_modeles_atomes}, associer à chaque modèle une observation qui le contredit, si cette observation existe.
Puis, réaliser une frise chronologique sur laquelle apparaît chaque modèle de l'atome, en utilisant les dates des observations expérimentales ou de découverte des modèles.