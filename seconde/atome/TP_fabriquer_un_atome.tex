%%%%
\teteSndAtom

%%%% titre
\titreTP{Fabriquer un atome}


%%%% Objectifs
\begin{objectifs}
  \item Apprendre la composition d'un atome.
  \item Comprendre la différence entre ion et atome.
\end{objectifs}

\begin{contexte}
  Au cours du \siecle{19}, la communauté scientifique considérait que l'atome était la plus petite « brique  » de la matière.
  Au début du \siecle{20}, deux expériences vont montrer que l'atome est composé de particules plus élémentaires :
  \begin{listePoints}
    \item en 1897, Thomson montre que l'on peut arracher des particules de charges négatives d'un atome ;
    \item en 1911, Rutherford montre que l'atome possède un noyau très petit devant la taille d'un atome, avec une charge positive.
  \end{listePoints}
  
  \problematique{
    Quelles entités composent les atomes ?
  }
\end{contexte}


%%%%
\titreSection{L'atome}

%%%%
\numeroQuestion
Légender cette représentation d'un atome en utilisant les mots proton, neutron, électron et nucléons.

\begin{center}
  \pasCorrection{\image{0.8}{images/atomes/atome}}
  \correction{\image{0.8}{images/atomes/atome_noyau}}
\end{center}

\qrcodeCote[2]{https://phet.colorado.edu/sims/html/build-an-atom/latest/build-an-atom_fr.html}

\mesure Scanner le qrcode pour accéder à l'animation.

\question{
  Dans l'application le cadre « symbole  » indique l'élément chimique fabriqué.
  Que faut-il ajouter pour changer d'élément chimique ?
}{
  Il faut ajouter des protons.
}[2]

\begin{doc}{Notation d'un élément chimique}{doc:A1_notation_element}
  Pour distinguer les atomes on utilise la notation \isotope{A}{Z}{X}.
  \begin{importants}
    \begin{listePoints}
      \item \chemfig{X} est le symbole de l'atome considéré.
      \item $Z$ est le nombre de \texteTrou[0.3]{protons}, appelé \important{numéro atomique.}
      \item $A$ est le nombre de \texteTrou[0.3]{neutrons}, appelé \important{nombre de masse.}
    \end{listePoints}
  \end{importants}
\end{doc}

\numeroQuestion 
Compléter le document~\ref{doc:A1_notation_element}.

\numeroQuestion
\isotope{23}{11}{Na} : le sodium \chemfig{Na} possède \texteTrou{11} protons, \texteTrou{23} nucléons, \texteTrou{12} neutrons.


%%%%
\titreSection{Les ions}

\question{
  Vérifier que la case « Neutralité/Ionisation » est cochée.
  Dans quel cas un élément chimique est un atome neutre ?
  Comment appelle-t-on cet élément sinon ?
}{
  L'élément chimique est un atome s'il a autant d'électrons que de protons.
  Sinon il possède une charge électrique et c'est un ion.
}[3]

\question{
  Que signifie le « + » de \ionSodium ? Donner la composition de l'élément, c'est-à-dire son nombre de proton, neutron et électrons.
}{
  Le « + » signifie qu'il y a une charge électrique positive autour de l'ion.
  Le \ionSodium possède le même nombre de protons et de neutrons que le sodium (11 protons et 12 neutrons), mais il n'a que 10 électrons.
}[3]

\question{
  Que peut-on dire sur le nombre d'électrons de l'ion chlorure \chlorure et de l'ion cuivrique \chemfig{Cu^{2+}} par rapport à leur atome respectif ?
}{
  L'ion chlorure a un électron supplémentaire par rapport à l'atome de chlore.
  L'ion cuivrique a deux électrons en moins par rapport à l'atome de cuivre
}[2]


%%%%
\titreSection{Les isotopes}

\question{
  Vérifier que la case « Stabilité/Instabilité » est cochée.
  Deux atomes du même élément peuvent-ils avoir des noyaux stables avec une composition différente ?
}{
  Oui, il peuvent avoir un nombre de neutrons différents, comme l'hélium 3 \isotope{3}{2}{He} et 4 \isotope{4}{2}{He}.
}[3]

\question{
  Que manque-t-il à l'élément \isotope{2}{2}{He} pour être stable ?
}{
  Il lui manque un ou deux neutrons.
}[2]
