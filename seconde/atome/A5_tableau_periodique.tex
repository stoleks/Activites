%%%%
\teteSndAtom

%%%% titre
\numeroActivite{5}
\titreActivite{Le Tableau périodique}


%%%% Objectifs
\begin{objectifs}
  \item Comprendre la construction du tableau périodique.
\end{objectifs}

\begin{contexte}
  Le tableau périodique des éléments, également appelé classification périodique des éléments ou simplement tableau périodique, représente tous les éléments chimiques découverts à ce jour.
  
 C'est le chimiste russe Dmitri Mendeleïev qui créa le tableau périodique moderne en 1869, en proposant de classer les éléments par numéro atomique croissant.

  \problematique{
    Comment construire le tableau périodique à partir des configurations électroniques des éléments ?
  }
\end{contexte}


%%%% question
\mesure
Compléter chaque carte en lui associant un élément chimique et en indiquant sa configuration électronique.

\mesure
Séparer les éléments dont la couche externe finit par une sous-couche s et les éléments dont la couche externe finit par une sous-couche p.

\mesure
En utilisant les configurations électronique, construire le tableau périodique des éléments en formant un « bloc s » et un « bloc p », en classant les éléments par numéro atomique croissant.


\question{
  Une ligne du tableau s'appelle une période.
  Quel est le point commun entre tous les éléments d'une même période ?
}{
  Tous les atomes d'une même période ont la même couche externe, avec le même nombre d'électron sur leurs couches internes.
}[4]

\question{
  Une colonne du tableau s'appelle une famille.
  Quel est le point commun entre tous les éléments d'une même famille ? (à l'exception de l'Hélium)
}{
  Tous les atomes d'une même famille ont le même nombre d'électrons sur leur couche externe.
  Les atomes d'une même famille auront tendance à former des molécules avec le même nombre de liaisons et des ions avec le même nombre de charges.
}[4]

\begin{importants}
  Quelques familles à connaître : 
  \begin{listePoints}
    \item Première colonne (sauf hydrogène) : \texteTrou{les \important{alcalins}.}
    \item Avant-dernière colonne : \texteTrou{les \important{halogènes}.}
    \item Dernière colonne : \texteTrou{les \important{gaz nobles}.}
  \end{listePoints}
\end{importants}

% \feuilleBlanche