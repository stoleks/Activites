%%%%
\teteSndSign
\numeroActivite{1}

%%%% titre
\titreActivite{Loi des noeuds et loi des mailles}


%%%% Objectifs
\begin{objectifs}
  \item Revoir quelques notions de bases des circuits électriques
  \item Revoir la loi des noeuds et la loi des mailles
\end{objectifs}


%%%% docs
\begin{doc}{Circuit électrique}
    Un circuit électrique est composé d'au moins un générateur, un récepteur (résistance, moteur, DEL, etc.) et de fils de connexion.

  \begin{encart}
    Un \important{dipôle} est un élément d'un circuit électrique possédant deux bornes.
  \end{encart}

  \begin{encart}
    Un \important{n\oe{}ud} est une connexion qui relie au moins trois dipôles entre eux.
  \end{encart}

  \begin{encart}
    Une \important{maille} est un chemin fermé, ne comportant pas forcément de générateur.
  \end{encart}
\end{doc}

\begin{doc}{Exemple de circuit}{doc:circuit_exemple_del}
  \begin{center}
  \begin{circuitikz}
    \ctikzset{bipoles/vsourceam/inner plus={\tiny $+$}}
    \ctikzset{bipoles/vsourceam/inner minus={\tiny $-$}}
    \draw (4, 2) -- (7, 2)
      to [rmeter, t=V, i=$I_3$, v=, name=uV] (7, -2) -- (4, -2)
      to [rmeter, t=A, i=$I_1$, v=, name=uA] (0, -2)
      to [ceV, i=$I_1$, v=$U_G$] (0, 2)
      to [R, l={$R$}, -*, i=$I_1$, v=, name=uR] (4, 2)
      to [empty diode, -*, i=$I_2$, v=, name=uD] (4, -2);
    \fixedvlen{uV}{$U_V$}
    \fixedvlen{uA}{$U_A$}
    \fixedvlen{uR}{$U_R$}
    \fixedvlen{uD}{$U_D$}
  \end{circuitikz}
  \end{center}
  \vspace*{-8pt}
  Ce circuit électrique permet de mesurer la caractéristique d'un dipôle, ici une diode électroluminescente (abrégée DEL).
\end{doc}

%%
\question{
  Combien de n\oe{}uds, mailles et dipôles comporte le circuit du document~\ref{doc:circuit_exemple_del} ?
}{
  5 dipôles, 2 noeuds et 2 mailles.
}{1}


%%
\begin{doc}{Association en série et en dérivation}
  Il existe deux façon d'associer des dipôles entre eux :
  \begin{listePoints}
    \item deux dipôles sont en séries s'ils sont situés dans la même maille et ne sont pas séparé par un noeud.
    \item deux dipôles sont en dérivation si leurs bornes sont connectés au même noeud.
  \end{listePoints}
\end{doc}

\newpage
\vspace*{-28pt}
\question{
  Indiquer les dipôles qui sont en série et les dipôles qui sont en dérivation.
}{
  Le générateur de tension, la résistance et l'ampèremètre sont en séries.
  Le voltmètre et la DEL sont en dérivation.
}{2}


%%
\begin{doc}{Loi des noeuds et intensité}{doc:loi_noeud_intensite}
  \chevron La quantité d'électrons qui \textbf{circulent} dans le circuit électrique se conserve.
  \textbf{Cette quantité d'électron est mesurée par l'intensité du courant notée $I$.}
  %
  \begin{encart}
    L'intensité du courant se mesure en \important{ampère} noté A, avec un ampèremètre branché en série.
  \end{encart}
  %
  \begin{encart}
    \important{Loi des noeuds} : la somme des intensités entrant dans un noeud est égale à la somme des intensité sortant du noeud.
  \end{encart}
  %
  Cette loi traduit la conservation de l'intensité du courant.
\end{doc}

\question{
  Donner la relation imposée par la loi des noeuds entre les intensités $I_1$, $I_2$ et $I_3$ dans le circuit du document~\ref{doc:circuit_exemple_del}.
}{
  $I_1 = I_2 + I_3$
}{1}


%%
\begin{doc}{Loi des mailles et tension}{doc:loi_maille_tension}
  Ce qui met en mouvement les électrons dans un circuit, c'est la différence d'état électrique entre deux points d'un circuit.
  \textbf{Cette différence d'état est mesurée par la tension électrique notée $U$.}
  %
  \begin{encart}
    La tension électrique se mesure en \important{volt} noté V, avec un voltmètre branché en dérivation.
  \end{encart}
  %
  \begin{encart}
    \important{Loi des mailles} : la somme des tensions des dipôles le long d'une maille est égale à 0 V.
  \end{encart}
  %
  \chevron Pour sommer les tensions, il faut parcourir la maille dans un sens, en \textbf{ajoutant} les tensions dont les flèches vont dans le sens du parcours et en \textbf{soustrayant} les tensions dont les flèches vont dans le sens opposé du parcours.
\end{doc}

\question{
  Donner la relation imposée par la loi des mailles entre les tensions $U_D$ et $U_V$ du document~\ref{doc:circuit_exemple_del}.
  Faire de même pour les tensions $U_R$, $U_D$, $U_A$ et $U_G$.
}{
  $U_D - U_V = 0 \unit{V}$, donc $U_D = U_V$. \\
  $-U_R - U_D - U_A + U_G = 0 \unit{V}$, donc $U_G = U_R + U_D + U_A$
}{3}