%%%%
\teteSndCorp

%%%% titre
\vspace*{-36pt}
\numeroActivite{1}
\titreActivite{Composition de l'atmosphère}


%%%% objectifs
\begin{objectifs}
  \item Comprendre comment on décrit la composition d'un mélange.
  \item Connaître la composition de l'air.
\end{objectifs}

\begin{contexte}
  L'atmosphère est un mélange de plusieurs gaz : dioxygène, diazote, dioxyde de carbone, etc.
  
  \problematique{
    Comment décrire la composition d'un mélange ?
  }
\end{contexte}


%%%% docs
\begin{doc}{Fraction volumique}{doc:A1_fraction_volumique}
  Soit une espèce chimique $E$ de volume $V_E$, dans un mélange de volume total $V$.
  La \important{proportion} ou \important{fraction volumique} de l'espèce chimique $E$ est
  \begin{equation*}
    p_{v}(E) = \frac{V_E}{V}
  \end{equation*}
  C'est une grandeur sans unité, comprise entre 0 et 1.
  On peut aussi l'exprimer en pourcentage, compris entre \qty{0}{\percent} et \qty{100}{\percent}.
  Par définition $\qty{10}{\percent} = \dfrac{10}{100} = \num{0,10}$.
\end{doc}

%%
\begin{doc}{Composition de l'atmosphère}{doc:A1_composition_atmo}
  \begin{importants}
    L’air contient \texteTrou[0.1]{\qty{78}{\percent}} de diazote \chemfig{N_2} et \texteTrou[0.1]{\qty{21}{\percent}} de dioxygène \chemfig{O_2}.
    Les autres gaz qui composent l’air sont l’argon \chemfig{Ar} (\qty{0,9}{\percent}),
    le dioxyde de carbone \chemfig{CO_2} (\qty{0,04}{\percent}),
    les gaz nobles et le méthane \chemfig{CH_4} (\qty{0,0002}{\percent}).
  \end{importants}
\end{doc}


%%%%
\question{
  Calculer le volume occupé par le diazote \chemfig{N_2} dans une salle de cours de \qty{600}{\metre\cubed}.
}{
  Le diazote occupe \qty{78}{\percent} du volume, soit $V_{\chemfig{N_2}} = 0,78 \times \qty{600}{\metre\cubed} = \qty{468}{\metre\cubed}$.
}{2}

\question{
  Même question pour le dioxygène \chemfig{O_2}.
}{
  Cette fois $V_{\chemfig{O_2}} = 0,21 \times \qty{600}{\metre\cubed} = \qty{126}{\metre\cubed}$.

}{2}

\begin{doc}{Respiration et dioxyde de carbone}{doc:A1_respiration}
  Quand on respire, on inspire du dioxygène \chemfig{O_2} qui est transformé en dioxyde de carbone \chemfig{CO_2} que l'on expire.

  Pendant une séance de cours d'une heure, le volume de dioxyde de carbone \chemfig{CO_2} double à cause de la respiration, si la salle n'est pas aérée.
\end{doc}

\question{
  Calculer la proportion volumique de dioxyde de carbone \chemfig{CO_2} après une heure de cours.
}{
  Le volume de dioxyde de carbone a doublé, on a donc une proportion deux fois plus élevée, soit \qty{0,08}{\percent}.
}{3}


%%
\begin{doc}{Fraction massique}{doc:A1_fraction_massique}
  Soit une espèce chimique $E$ de masse $m_E$, dans un mélange de masse totale $m$.
  La \important{proportion} ou \important{fraction massique} de l'espèce chimique $E$ est
  \begin{equation*}
    p_{m}(E) = \frac{m_E}{m}
  \end{equation*}
  C'est une grandeur sans unité, comprise entre 0 et 1.
  On peut aussi l'exprimer en pourcentage, compris entre \qty{0}{\percent} et \qty{100}{\percent}.
\end{doc}

\begin{doc}{Cloche en bronze}{doc:A1_cloche_bronze}
  \begin{wrapfigure}[5]{r}{0.2\linewidth}
    \vspace*{-31pt}
    \centering
    \image{0.7}{images/photos/cloche_bronze.png}
  \end{wrapfigure}
  
  Les cloches traditionnelles des temples coréens sont en bronze.
  Le bronze est un \important{alliage,} un mélange homogène entre deux métaux.
  
  Le bronze est constitué de \qty{20}{\percent} d'étain \chemfig{Sn} et de \qty{80}{\percent} de cuivre \chemfig{Cu} en masse.

  Une cloche traditionnelle pèse plusieurs centaines de kilogramme.
\end{doc}


\question{
  Exprimer les proportions massiques du cuivre et de l'étain dans une cloche en bronze sous la forme d'une division entre deux entiers les plus petits possibles.
}{
  $\qty{20}{\percent} = \dfrac{20}{100} = \dfrac{1}{5}$ pour l'étain.
  $\qty{80}{\percent} = \dfrac{80}{100} = \dfrac{4}{5}$ pour le cuivre.
}{3}

\question{
  Calculer la masse cuivre dans une cloche traditionnelle de masse $m = \qty{500}{\kg}$
}{
  La masse de cuivre vaut $0,8 \times \qty{500}{\kg} = \qty{400}{\kg}$.
}{2}

\question{
  Même question pour l'étain.
}{
  La masse d'étain vaut $0,2 \times \qty{500}{\kg} = \qty{100}{\kg}$.
}{2}

\question{
  Est-ce que l'on pourrait calculer les fractions volumiques de cuivre et d'étain à partir des fractions massiques ?
}{
  Non, car on ne sait pas quel est le volume de la cloche, ni quels sont les volumes de cuivre et d'étain dans la cloche.
}{1}
