%%%%
\teteSndCorp

%%%% titre
\vspace*{-36pt}
\titreTP{Cocktail et vinaigrette}


%%%% objectifs
\begin{objectifs}
  \item Connaître le vocabulaire associé aux corps purs et mélanges.
  \item Connaître et manipuler la verrerie de base en chimie.
  \item Comprendre la notion de masse volumique.
\end{objectifs}

\begin{contexte}
  En cuisine, mélanger deux liquides peut amener à des résultats différents selon les combinaisons.
  Préparer un cocktail ou une vinaigrette ce n'est pas la même chose !
  
  \problematique{
    Quels notions physiques et chimiques utilise-t-on pour décrire les propriétés d'un mélange ?
  }
\end{contexte}


%%%% docs
\begin{doc}{Un peu de vocabulaire}[\label{doc:vocabulaire_melange}]
  \begin{importants}
    La matière est constituée \important{d'entités chimiques} microscopiques : \texteTrou(1){atomes, molécules, ions.}
    Une \important{espèce chimique} est constituée d’un très grand nombre d’entités chimiques
identiques.
  \end{importants}
    
  \begin{importants}
    \begin{listePoints}
      \item Un \important{corps pur} est constitué de \texteTrou{une seule espèce chimique.}
      \item Un \important{mélange} est constitué de \texteTrou{plusieurs espèces chimiques.}
    \end{listePoints}
  \end{importants}
\end{doc}

%%
\begin{doc}{Type de mélange}[\label{doc:type_melange}]
  \begin{importants}
    Un mélange est \important{homogène} si on ne peut pas distinguer ses constituants.
    Un mélange homogène est constitué d'\important{une seule phase}.
  \end{importants}
  
  \begin{importants}
    Un mélange est \important{hétérogène} si on peut distinguer ses constituants.
    Un mélange hétérogène est constitué de \important{plusieurs phases}.
  \end{importants}

  \begin{importants}
    On dit que deux liquides sont \important{miscibles} s'ils forment un \texteTrou[0.1]{\important{mélange homogène.}}
  \end{importants}
  \begin{importants}
    Inversement, deux liquides sont \important{non miscibles} s'ils forment un \important{mélange hétérogène.}
  \end{importants}
  Miscible vient du latin \og misceo \fg, qui veut dire mélanger.
\end{doc}


%%
\mesure
Sur la paillasse se trouve une pissette d'eau distillée, l'huile et le sirop se trouve sur la paillasse centrale.
Dans les tubes à essais, verser :
\begin{listePoints}[3]
  \item Tube 1 : eau.
  \item Tube 2 : eau + huile.
  \item Tube 3 : eau + sirop.
\end{listePoints}

\attention Il faut faire attention à ne pas remplir les tubes à essais, quelques centimètres suffisent.

\mesure 
Utiliser les bouchons pour agiter doucement les différents mélanges.

%
\newpage
\vspace*{-24pt}
\question{
  Attendre un peu, puis schématiser le résultat obtenu dans chaque tube à essais.
}{
  Schéma 1 : tube + eau légendée ;
  Schéma 2 : tube + eau + huile au dessus ;
  Schéma 3 : tube + sirop + eau au dessus.
}
\pasCorrection{\vspace*{6cm}}

%
\question{
  Décrire le contenu des tubes en utilisant le vocabulaire des documents~\ref{doc:vocabulaire_melange} et~\ref{doc:type_melange}.
}{
  Le tube 1 contient un corps pur.
  Le tube 2 contient un mélange hétérogène, on peut distinguer l'eau et l'huile.
  Le tube 3 contient un mélange homogène.
}[3]

%
\question{
  Indiquer si l'eau et l'huile sont miscibles et si le sirop et l'huile sont miscibles.
}{
  L'eau et l'huile ne sont pas miscibles (mélange hétérogène). Donc le sirop et l'huile ne sont pas miscibles.
}[2]
  

%%
\begin{doc}{Notion de masse volumique}[\label{doc:masse_volumique}]
  \begin{importants}
    La \important{masse volumique} est une grandeur qui représente la masse par unité de volume d'un échantillon de matière.
  \end{importants}

  \separationBlocs{%
    \begin{importants}
      Si l'échantillon a une masse $m$ et un volume $V$, sa masse volumique est définie par
      \vspace*{-8pt}
      \begin{equation*}
        \rho = \dfrac{m}{V}
      \end{equation*}
    \end{importants}
  }{%
    \textbf{Données :}
    \begin{listePoints}    
      \item $\rho (\text{eau liquide}) = \qty{1,00}{\g\per\ml}$
      \item $\rho (\text{huile}) = \qty{0,92}{\g\per\ml}$
      \item $\rho (\text{sirop}) > \qty{1,00}{\g\per\ml}$
    \end{listePoints}
  }
  
  \vspace*{4pt}
  \attention La masse volumique d'un échantillon est toujours la même, quelque soit sa taille ou sa forme. 
  Par contre la masse volumique dépend des conditions de température et de pression.
\end{doc}

%
\question{
  En utilisant les informations du document~\ref{doc:masse_volumique}, formuler une hypothèse qui expliquerait pourquoi l'huile flotte au dessus de l'eau.
}{
  L'huile flotte au dessus de l'eau, car elle a une masse volumique plus petite que la masse volumique de l'eau.
}[3]

\question{
  Expliquer pourquoi l'huile devrait aussi flotter au dessus du sirop d'après cette hypothèse.
}{
  La masse volumique de l'huile est plus faible que celle du sirop, elle va donc flotter au dessus.
}[1]

%
\mesure
Vérifier l'hypothèse en versant dans un tube à essais l'huile et le sirop.

\mesure
En utilisant les connaissances accumulées sur la masse volumique, essayer de préparer un tube à essai avec trois étages de liquide distincts.