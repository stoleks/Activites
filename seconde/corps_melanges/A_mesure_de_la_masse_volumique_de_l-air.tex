%%%%
\teteSndCorp

%%%% titre
%\vspace*{-36pt}
\titreActivite{Mesure de la masse volumique de l'air}


%%%% objectifs
\begin{objectifs}
  \item Calculer la masse volumique de l'air.
\end{objectifs}

\begin{contexte}
  L'atmosphère est un mélange de plusieurs gaz : dioxygène, diazote, dioxyde de carbone, etc.
  
  \problematique{
    Comment calculer la masse volumique de l'air à partir de sa composition ou d'une expérience ?
  }
\end{contexte}


%%%% docs
\begin{doc}{Mesure de la masse volumique de l'air}{doc:A2_mesure_densite_air}
  \begin{wrapfigure}{r}{0.1\linewidth}
    \vspace*{-29pt}
    \qrcode{https://www.youtube.com/watch?v=isEo51ncsKU&t=26s}
  \end{wrapfigure}
  On peut mesurer la masse volumique de l'air en dégonflant un ballon dans une bouteille d'eau.
  La bouteille d'eau permet de mesurer le volume d'air expulsé.
  En pesant le ballon avant et après le dégonflage, on peut calculer la masse d'air expulsée.
\end{doc}

\numeroQuestion 
Schématiser les 3 étapes de l'expérience réalisée.
\pasCorrection{\vspace*{200pt}}
\correction{
  Schéma du ballon sur la balance avec $m_1$,
  schéma du ballon vidé dans une éprouvette graduée avec l'air qui prend la place de l'eau,
  schéma du ballon sur la balance avec $m_2$.
}

\numeroQuestion
Remplir le tableau ci-dessous 
\begin{tableau}{|c |c |c |c |}
  Grandeur & Masse du ballon plein $m_1 $ & Masse du ballon dégonflé $m_2$ & Volume d'air expulsé $V$ \\
  \SetCell{couleurSec-50} Valeur & \correction{\qty{483,2}{\g}} & \correction{\qty{481,4}{\g}} & \correction{\qty{1,5}{\litre}}
\end{tableau}

\question{
  Calculer la masse volumique mesurée $\rho_\text{mes}(\text{air})$.
}{
  La masse d'air expulsée est $m = m_2 - m_1 = \qty{1,8}{\g}$, soit une masse volumique
  \begin{equation*}
    \rho = \dfrac{m}{V} = \frac{1,8}{1,5} \unit{\g\per\litre} = \qty{1,2}{\g\per\litre}
  \end{equation*}
}[3]

%%
\begin{doc}{Masse volumique d'un mélange}{doc:A2_composition_atmo}
  Pour un mélange de gaz, la masse volumique du mélange est simplement la somme des masses volumique de chaque gaz pondérée par la fraction volumique de chaque gaz du mélange.

  Pour l'air, on aura donc
  \begin{equation*}
    \rho(\text{air}) = p_v(\dioxygene) \rho(\dioxygene) + p_v(\diazote) \rho(\diazote) + p_v(\chemfig{Ar}) \rho(\chemfig{Ar}) + p_v(\dioxydeDeCarbone) \rho(\dioxydeDeCarbone)
  \end{equation*}
\end{doc}

\pasCorrection{\newpage}
\question{
  Rappeler les fractions volumique des gaz composant l'air (\dioxygene, \diazote, \dioxydeDeCarbone, \chemfig{Ar}).
}{
  $p_v(\dioxygene)  = \qty{21}{\percent}$, 
  $p_v(\diazote)  = \qty{78}{\percent}$, 
  $p_v(\dioxydeDeCarbone) = \qty{0,04}{\percent}$, 
  $p_v(\chemfig{Ar})   = \qty{0,9 }{\percent}$, 
}[2]

\begin{doc}{Masse volumique des gaz composant l'air}{doc:A2_masse_volumique_gaz}
  \textbf{Données :}
  \begin{listeTirets}
    \item Masse volumique du \dioxydeDeCarbone gazeux : $\rho(\dioxydeDeCarbone) = \qty{1,87}{\g/\litre}$.
    \item Masse volumique du \dioxygene  gazeux : $\rho(\dioxygene)  = \qty{1,35}{\g/\litre}$.
    \item Masse volumique du \diazote  gazeux : $\rho(\diazote)  = \qty{1,18}{\g/\litre}$.
    \item Masse volumique de \chemfig{Ar}   gazeux : $\rho(\chemfig{Ar})   = \qty{1,78}{\g/\litre}$.
  \end{listeTirets}
\end{doc}


%%%%
\question{
  Calculer la masse volumique théorique de l'air $\rho_\text{theo} (\text{air})$.
}{
  $\rho_\text{theo} (\text{air}) 
  = (0,21\times 1,35 + 0,78\times 1,18 + 0,0004\times 1.87 + 0.009\times 1.78) \unit{\g/\litre}
  = \qty{1.22}{\g/\litre}$
}[4]

\question{
  Comparer la valeur théorique et la valeur mesurée. Est-ce qu'elles sont égales ? Est-ce qu'elles sont cohérentes ?
}{
  On trouve deux valeurs légèrement différentes, \qty{1,2}{\g/\litre} et \qty{1,22}{\g/\litre}, mais elles sont cohérentes avec la précision des mesures réalisées pendant l'expérience.
}[2]
