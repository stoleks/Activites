%%%%
\teteSndLumi
\vspace*{-24pt}

%%%% titre
\numeroActivite{1}
\titreTP{Formation des images et vision}


%%%% Objectifs
\vspace*{-4pt}
\begin{objectifs}
  \item Former une image avec une lentille convergente.
  \item Comprendre la modélisation optique de l'oeil.
\end{objectifs}

\begin{contexte}
  L'oeil humain permet de construire l'image d'un objet observé sur la rétine, qui contient des cellules capable de percevoir les couleurs (cônes) ou l'intensité lumineuse (bâtonnets).
  
  \problematique{
    Comment modéliser et comprendre la formation d'une image par un oeil ?
  }
\end{contexte}


%%%% Formation d'une image avec une lentille
\begin{doc}{Lentille convergente}{doc:TP1_lentille_convergente}
  \begin{wrapfigure}[4]{r}{0.4\linewidth}
    \centering
    \vspace*{-42pt}
    \image{0.6}{images/lumiere/schema_lentilles_conv}
  \end{wrapfigure}
  
  Cette année en optique on va travailler avec des \important{lentilles convergentes,} qui concentrent les rayons lumineux.
  Elles sont plus épaisses au centre qu'aux extrémités et sont schématisées par une double flèche fermée.

  \begin{encart}
    Une \important{lentille convergente} possède
    \begin{listePoints}
      \item un \important{centre optique} noté $O$, au centre de la lentille. 
      \item un \important{foyer image} noté $F'$ et son symétrique par rapport à $O$, le \important{foyer objet} noté $F$.
      \item une \important{distance focale} noté $f'$, qui est la distance $OF'$.
    \end{listePoints}
    
    La droite perpendiculaire à la lentille passant par $O$ est appelée \important{l'axe optique}, orientée par rapport au sens de propagation de la lumière.
  \end{encart}

  Les lentilles convergentes ont une propriétés particulières : tous les rayons lumineux qui partent d'un point et traversent la lentille vont converger en un même point, ce qui permet de reconstituer une image.
\end{doc}

\begin{doc}{Formation d'une image avec une lentille}{doc:TP1_formation_image}
  \begin{wrapfigure}[6]{r}{0.5\linewidth}
    \vspace{-20pt}
    \begin{boite}
      \important{Vocabulaire :}
      \vspace{-8pt}
      \begin{encart}
        Un \important{rayon incident} va vers la lentille.
        Un \important{rayon émergent} s'éloigne de la lentille.
      \end{encart}
    \end{boite}
  \end{wrapfigure}
  
  Trois rayons lumineux ont des propriétés particulières quand ils traversent une lentille convergente. 
  En utilisant deux rayons lumineux particuliers qui partent d'un point, on peut trouver où les rayons lumineux convergent pour former son image.

  \begin{listePoints}
    \item Tout rayon incident qui passe par le centre optique n'est pas dévié.
    \item Tout rayon incident qui passe par le foyer objet $F$ émerge parallèle à l'axe optique.
    \item Tout rayon incident parallèle à l'axe optique émerge en passant par le foyer image $F'$.
  \end{listePoints}

  \begin{center}
    \image{0.7}{images/lumiere/formation_image_lentille_conv}
  \end{center}
\end{doc}

\mesure
Placer la lentille sur le banc optique, puis repérer la position des points virtuels $F$ et de $F'$ sur le banc optique par rapport à la lentille.

\mesure
Placer la lampe avec l'objet en forme de « F » sur le banc optique, puis mesurer la taille de la lettre « F » qu'on note $AB =$ \texteTrou[0.1]{\qty{15}{\cm}}

\mesure 
Placer la lampe à une distance supérieure à $f'$, mais inférieure à $2\times f'$. 
Placer l'écran de l'autre côté du banc optique et le déplacer pour trouver la position où l'image est nette sur l'écran.
Mesurer la taille de l'image $A'B'$, la distance $OA$ et la distance              $OA'$.
Répéter cette opération en plaçant la lampe à une distance de $2f'$, puis à une distance supérieure à $2f'$.

\numeroQuestion
Remplir le tableau ci-dessous avec vos mesures.

% \vspace*{-16pt}
\begin{tableau}{
  |X[c] |X[c] |X[c] | X[c] |
}
  Position de l'objet &
  Taille de l'image $A'B'$ (\unit{\cm}) &
  Distance lentille objet $OA$ (\unit{\cm}) &
  Distance lentille image $OA'$ (\unit{\cm}) \\
  $f' < OA < 2f'$ & & & \\
  $OA = 2f'$      & & & \\
  $OA > 2f'$      & & & \\
\end{tableau}

\question{
  Pour chaque position de l'objet, calculer le \important{grandissement $\gamma = \dfrac{A'B'}{AB}$} (« gamma ») et le rapport $g = \dfrac{OA'}{OA}$.
  Est-ce que $g$ et $\gamma$ sont égaux ?
}{}{3}


%%%% Modélisation de l'oeil
\begin{doc}{Modèle simplifié de l'oeil}{doc:TP1_modele_oeil}
  \begin{wrapfigure}[8]{r}{0.45\linewidth}
    \centering
    \vspace*{-12pt}
    \image{0.9}{images/lumiere/modele_oeil_optique}
  \end{wrapfigure}
  
  L'oeil humain est un organe complexe (et fragile !) composé de plusieurs éléments.
  On peut modéliser un oeil humain en trois parties :
  
  \begin{listePoints}
    \item \important{l'iris,} avec un trou central (la pupille) de taille variable. L'iris permet de contrôler la quantité de rayons lumineux arrivant dans l'oeil.
    \item \important{le cristallin, la cornée et les humeurs,} qui dévient les rayon lumineux comme une lentille convergente.
    \item \important{la rétine,} qui reçoit les rayons lumineux et sur laquelle l'image est formée.
    Elle est composée de cônes pour percevoir les couleurs et de bâtonnets pour percevoir l'intensité lumineuse.
  \end{listePoints}

  Une fois l'image d'un objet formée sur la rétine, la lumière est transformée en signaux électriques.
  Ces signaux électriques sont transmis au cerveaux par le nerf optique, qui les utilise pour former notre vision.
\end{doc}

%%%%
\mesure
Associer chaque composant de l'oeil avec l'objet permettant de le modéliser

\vspace*{-16pt}
\begin{center}
  \begin{tableau}{|c| X[c]| X[c]| X[c]| X[c]|}
    Optique & diaphragme & lentille & écran \\
    %
    Oeil & iris & \correction{Cristallin} & \correction{rétine}\\
    %
  \end{tableau}
\end{center}


% \begin{doc}{Rayons particuliers à travers une lentille convergente}{doc:TP1_rayons_lentille}
%   \centering
%   \image{0.9}{images/lumiere/image_lentille_convergente}
% \end{doc}

% \begin{doc}{Grandissement}{doc:TP1_grandissement}
%   \begin{encart}
%     Le \important{grandissement} noté $\gamma$ (gamma) est le rapport entre la hauteur algébrique de l'image par celle de l'objet :
%     \begin{equation*}
%         \gamma = \dfrac{\algebrique{A'B'}}{\algebrique{AB}}
%     \end{equation*}
%   \end{encart}
% \end{doc}

% \begin{doc}{Synthèse}{doc:TP1_synthese}
%   \begin{encart}
%     Le signe du grandissement $\gamma$ indique si l'image obtenue est droite ($\gamma > 0$) ou inversée ($\gamma < 0$).
    
%     La valeur du grandissement indique si l'image est plus petite ($|\gamma| < 1$) ou plus grande ($|\gamma| > 1$) que l'objet.
%   \end{encart}
% \end{doc}

% \question{
%   \ref{doc:TP1_convention_optique} Indiquer le signe (positif ou négatif) de $\algebrique{AB}$ et $\algebrique{A'B'}$ sur la figure du document~\ref{doc:TP1_rayons_lentille}.
% }{
%   $\algebrique{AB} > 0$ et $\algebrique{A'B'} < 0$.
% }{3}


% \begin{doc}{Quelques conventions d'optique}{doc:TP1_convention_optique}
%   L'image d'un objet $AB$ est notée $A'B'$.
%   \begin{encart}
%     En optique les longueurs sont \important{algébriques}, c'est-à-dire qu'elles sont positives ou négatives en fonction de leur sens, on les note avec une barre $\algebrique{AB}$
%     \begin{listePoints}
%       \item $\algebrique{AB} > 0$ : B est au dessus ou à droite de A ;
%       \item $\algebrique{AB} < 0$ : B est en dessous ou à gauche de A.
%     \end{listePoints}
%   \end{encart}
% \end{doc}

% \question{
%   \ref{doc:TP1_rayons_lentille} Décrire le trajet des trois rayons particuliers construit pour une lentilles convergentes (``le rayon passant par $\ldots$ ressort de la lentille $\ldots$'').
% }{
%   \begin{listePoints}
%     \item le rayon marron passe par le centre optique $O$ et n'est pas dévié ;
%     \item le rayon rouge arrive parallèle à l'axe optique et ressort de la lentille en passant par le foyer image $F'$ ;
%     \item le rayon gris passe par le foyer objet $F$ et ressort de la lentille en étant parallèle à l'axe optique.
%   \end{listePoints}
% }{7}

% \question{
%   \label{exo:thales_grandissement}
%   \ref{doc:TP1_convention_optique}, \ref{doc:TP1_rayons_lentille} et \ref{doc:TP1_grandissement}
%   Appliquer le théorème de Thalès sur les triangles OAB et OA'B' pour établir la relation entre $\gamma$, $\algebrique{OA'}$ et $\algebrique{OA}$.
% }{
%   En appliquant le théorème de Thalès sur les deux triangles semblables $OAB$ et $OA'B'$, on obtient directement la relation suivante :
%   \begin{equation*}
%     \dfrac{\algebrique{OA'}}{\algebrique{OA}} = \dfrac{\algebrique{A'B'}}{\algebrique{AB}}
%   \end{equation*}
%   Comme par définition $\algebrique{A'B'} / \algebrique{AB} = \gamma$, on a donc que
%   \begin{equation*}
%     \gamma = \dfrac{\algebrique{OA'}}{\algebrique{OA}}
%   \end{equation*}
% }{8}

% \question{
%   Un objet a une hauteur $\algebrique{AB} = 1,\!20 \unit{m}$ et est placé à $6,\!00\unit{m}$ d'une lentille.
%   L'image formé de l'objet a une hauteur $\algebrique{A'B'} = -0,\!01 \unit{m}$.
%   En utilisant la relation calculée question~\ref{exo:thales_grandissement}, calculer la distance $OA'$ entre la lentille et l'écran.
% }{
%   Ici $\gamma = -0,\!01 \unit{m} / 1,\!20 \unit{m} = - 0,\!0083$. Et donc 
%   \begin{equation*}
%     OA' = |\gamma| \times OA = 0,\!0083 \times 6,\!00 \unit{m} = 0,\!05 \unit{m}
%   \end{equation*}
% }{5}
