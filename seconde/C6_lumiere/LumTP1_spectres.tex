%%%%
\teteSndLumi

%%%% titre
\vspace*{-36pt}
\numeroActivite{1}
\titreTP{Spectre d'émission}


%%%% Objectifs
\begin{objectifs}
  \item Comprendre la notion de spectre d'émission
\end{objectifs}

\begin{contexte}
  Il existe différentes sources lumineuse, comme le Soleil, les lampadaires, les néons, les écrans de téléphones, etc.
  
  \problematique{
    Comment caractériser la lumière émise par une source ?
  }
\end{contexte}


%%%% docs
\begin{doc}{Spectre d'émission}{doc:TP1_spectre_emission}
  La lumière est une onde électromagnétique, qui peut avoir plusieurs longueurs d'ondes.
  Nos yeux captent certaines longueurs d'ondes et y associent une couleur : c'est le domaine visible.
  
  \begin{encart}
    La donnée de toutes les longueurs d'ondes présentes dans une source lumineuse s'appelle le \important{spectre d'émission}.
    Le spectre dans le domaine visible est représenté de la manière suivante :
  \end{encart}
  \vspace{-22pt}
  \begin{center}
    \image{0.4}{images/lumiere/spectre_lumineux}
  \end{center}
\end{doc}


%%
\titreSection{Les spectre d'émissions continus}

\begin{doc}{Spectre continu}{doc:TP1_spectre_continu}
  \begin{encart}
    Un \important{spectre d'émission continu} présente une suite de raies colorées.
    Un spectre continu prend la forme d'une bande colorée unique.
  \end{encart}
\end{doc}

\begin{doc}{Lampe à incandescence}{doc:TP1_lampe_incandescence}  
  Une lampe à incandescence est composé d'un petit filament chauffé par le passage d'un courant électrique.
  En augmentant la tension d'alimentation d'une lampe à incandescence, on augmente la température du filament.
\end{doc}

\mesure
Alimenter la lampe avec le générateur.
Celui-ci propose deux tensions : $6\unit{V}$ ou $12\unit{V}$ : essayer les deux cas en les observant au spectroscope.

\question{
  Quelle différence remarquez-vous ? 
}{

}{1}

\mesure 
Schématisez vos observations ci-dessous :
\medskip

\separationBlocs{
  \centering 
  Lampe sous-alimentée ($6 \unit{V}$)
  
  \image{1}{images/lumiere/spectre_lumineux_graduee}
}{
  \centering
  Lampe correctement alimentée ($12 \unit{V}$)
  
  \image{1}{images/lumiere/spectre_lumineux_graduee}
}

\begin{doc}{Émission d'un corps chaud}{doc:TP1_corps_chaud}
  \begin{encart}
    Un corps chaud émet \texteTrouLignes[1]{un rayonnement lumineux avec un spectre continu.} 
    Les propriétés du rayonnement lumineux dépendent \texteTrouLignes[1]{de la température de l'objet.}
    Plus la température s'élève et plus le spectre s'enrichit \texteTrouLignes[2]{en petite longueur d'onde, ce qui correspond à des couleurs plus froides (bleue ou violet).}
  \end{encart}
\end{doc}

\question{
  Utilisons ce résultat pour estimer la température de surface d'une étoile.
  Bételgeuse est une étoile de couleur rouge-orange, sa température de surface vaut \qty{3800}{\degreeCelsius}.
  L’étoile Rigel est de couleur bleue. Sa température sera-t-elle plus élevée ou plus faible ? 
}{
  ...
}{2}


%%
\titreSection{Les spectres d’émission de raies}

\numeroQuestion
  En utilisant le spectroscope, observer le spectre de la lumière émise par les deux lampes disponibles : celle à vapeur de mercure et celle à vapeur de sodium.
  Schématiser vos observations ci-dessous, en faisant apparaître les espaces noirs :

\separationBlocs{
  \centering 
  Lampe à vapeur de mercure \chemfig{Hg}
  
  \image{1}{images/lumiere/spectre_lumineux_graduee}
}{
  \centering
  Lampe à vapeur de sodium \chemfig{Na}
  
  \image{1}{images/lumiere/spectre_lumineux_graduee}
}

\begin{doc}{Émission atomique ou moléculaire}{doc:TP1_emission_atomique}
  \begin{encart}
    Lorsque les entités chimiques (atomes, ions, molécules), qui composent un gaz sont excitées, elles émettent seulement \texteTrouLignes[2]{des radiations avec des longueurs d'ondes précises}
    
    Cela correspond à \texteTrouLignes[1]{des raies fines et bien définies.}
  \end{encart}
    
  Chaque entité chimique possède son propre \important{spectre d'émission} caractérisé par quelques longueurs d'ondes précises, comme chaque humain possède ses propres empreintes digitales.

  Observer un spectre d'émission permet donc \important{d'identifier} les entités présentes dans un gaz.
\end{doc}