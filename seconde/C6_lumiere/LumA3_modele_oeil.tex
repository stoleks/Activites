%%%%
\teteSndLumi
\vspace*{-32pt}

%%%% titre
\numeroActivite{3}
\titreActivite{Modélisation d'un oeil humain}


%%%% Objectifs
\vspace*{-12pt}
\begin{objectifs}
  \item Comprendre la modélisation de l'oeil
  \item Apprendre les propriétés d'une lentille convergente
\end{objectifs}

\begin{contexte}
  L'oeil humain permet de construire l'image d'un objet observé sur la rétine, qui contient des cellules capable de percevoir les couleurs (cônes) ou l'intensité lumineuse (bâtonnets).
  
  \problematique{
    Comment modéliser la formation d'une image par un oeil ?
  }
\end{contexte}


%%%% docs
\begin{doc}{Modèle simplifié de l'oeil}
  \label{doc:modele_oeil}
  L'oeil humain est un organe complexe (et fragile !) composé de plusieurs éléments.
 
  On peut modéliser un oeil humain en trois parties :
  
  \begin{wrapfigure}[8]{r}{0.45\linewidth}
    \centering
    \vspace*{-12pt}
    \image{0.95}{images/lumière/modele_oeil.png}
  \end{wrapfigure}
  \pointCyan \important{l'iris}, avec un trou central (la pupille) de taille variable. L'iris permet de contrôler la quantité de rayons lumineux arrivant dans l'oeil. \\
  \pointCyan \important{le cristallin}, qui dévie les rayon lumineux comme une lentille mince convergente. \\
  \pointCyan \important{la rétine}, qui reçoit les rayons lumineux et sur laquelle l'image est formée.
  Elle est composée de cônes pour percevoir les couleurs et de bâtonnets pour percevoir l'intensité lumineuse.
\end{doc}

\begin{doc}{Quelques conventions d'optique}
  \label{doc:convention_optique}
  \vspace{-24pt}
  \begin{encart}
    Une \important{lentille convergente} possède
    \begin{listePoints}
      \item un \important{centre optique} noté $O$, au centre de la lentille. 
      \item un \important{foyer image} noté $F'$, que l'on positionne à droite de la lentille.
      \item un \important{foyer objet} noté $F$, qui est le symétrique de $F'$ par rapport à $O$.
    \end{listePoints}
  \end{encart}
  
  La droite perpendiculaire à la lentille passant par $O$ est appelée \important{l'axe optique}.
  
  L'image d'un objet $AB$ est notée $A'B'$.
  \begin{encart}
    En optique les longueurs sont \important{algébriques}, c'est-à-dire qu'elles sont positives ou négatives en fonction de leur sens, on les note avec une barre $\algebrique{AB}$.
  \end{encart}
  \exemple $\algebrique{AB} > 0$ si B est au dessus de A et $\algebrique{AB} < 0$ si B est en dessous de A.
\end{doc}

\begin{doc}{Rayons particuliers à travers une lentille convergente}
  \label{doc:rayons_lentille}
  \centering
  \image{0.9}{images/lumière/image_lentille_convergente.png}
\end{doc}

\begin{doc}{Grandissement}
  \label{doc:grandissement}
  \vspace*{-24pt}
  \begin{encart}
    Le \important{grandissement} noté $\gamma$ (gamma) est le rapport entre la hauteur algébrique de l'image par celle de l'objet :
    \begin{equation*}
        \gamma = \Frac{\algebrique{A'B'}}{\algebrique{AB}}
    \end{equation*}
  \end{encart}
\end{doc}

%%%%
\numeroQuestion
  \docu{\ref{doc:modele_oeil}}
  Associer chaque composant de l'oeil avec l'objet permettant de le modéliser
  \begin{center}
    \begin{tabular}{|c|c|c|c|c}
      \hline
      Optique & \centering diaphragme & lentille & écran 
      \\ \hline
      Oeil & \hspace{40pt} iris \hspace{40pt} & \correction{Cristallin}\hspace{100pt} & \correction{rétine}\hspace{100pt} 
      \\ \hline
    \end{tabular}
  \end{center}

\question{
  \docu{\ref{doc:rayons_lentille}} Décrire le trajet des trois rayons particuliers construit pour une lentilles convergentes (``le rayon passant par $\ldots$ ressort de la lentille $\ldots$'').
}{
  \begin{listePoints}
    \item le rayon marron passe par le centre optique $O$ et n'est pas dévié ;
    \item le rayon rouge arrive parallèle à l'axe optique et ressort de la lentille en passant par le foyer image $F'$ ;
    \item le rayon gris passe par le foyer objet $F$ et ressort de la lentille en étant parallèle à l'axe optique.
  \end{listePoints}
}{7}

\question{
  \docu{\ref{doc:convention_optique}} Indiquer le signe (positif ou négatif) de $\algebrique{AB}$ et $\algebrique{A'B'}$ sur la figure du document~\ref{doc:rayons_lentille}.
}{
  $\algebrique{AB} > 0$ et $\algebrique{A'B'} < 0$.
}{3}

\question{
  \label{exo:thales_grandissement}
  \docu{\ref{doc:convention_optique}, \ref{doc:rayons_lentille} et \ref{doc:grandissement}}
  Appliquer le théorème de Thalès sur les triangles OAB et OA'B' pour établir la relation entre $\gamma$, $\algebrique{OA'}$ et $\algebrique{OA}$.
}{
  En appliquant le théorème de Thalès sur les deux triangles semblables $OAB$ et $OA'B'$, on obtient directement la relation suivante :
  \begin{equation*}
    \Frac{\algebrique{OA'}}{\algebrique{OA}} = \Frac{\algebrique{A'B'}}{\algebrique{AB}}
  \end{equation*}
  Comme par définition $\algebrique{A'B'} / \algebrique{AB} = \gamma$, on a donc que
  \begin{equation*}
    \gamma = \Frac{\algebrique{OA'}}{\algebrique{OA}}
  \end{equation*}
}{8}

\question{
  Un objet a une hauteur $\algebrique{AB} = 1,\!20 \unit{m}$ et est placé à $6,\!00\unit{m}$ d'une lentille.
  L'image formé de l'objet a une hauteur $\algebrique{A'B'} = -0,\!01 \unit{m}$.
  En utilisant la relation calculée question~\ref{exo:thales_grandissement}, calculer la distance $OA'$ entre la lentille et l'écran.
}{
  Ici $\gamma = -0,\!01 \unit{m} / 1,\!20 \unit{m} = - 0,\!0083$. Et donc 
  \begin{equation*}
    OA' = |\gamma| \times OA = 0,\!0083 \times 6,\!00 \unit{m} = 0,\!05 \unit{m}
  \end{equation*}
}{5}

\begin{doc}{Synthèse}
  \vspace*{-24pt}
  \begin{encart}
    Le signe du grandissement $\gamma$ indique si l'image obtenue est droite ($\gamma > 0$) ou inversée ($\gamma < 0$).
    
    La valeur du grandissement indique si l'image est plus petite ($|\gamma| < 1$) ou plus grande ($|\gamma| > 1$) que l'objet.
  \end{encart}
\end{doc}

\feuilleBlanche
