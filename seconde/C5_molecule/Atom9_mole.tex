%%%%
\teteSndMole

%%%% titre
\vspace*{-32pt}
\numeroActivite{9}
\titreActivite{Du microscopique au macroscopique}


%%%% Objectifs
\begin{objectifs}
  \item Savoir utiliser le vocabulaire adapté entre atome, ion et molécule.
  \item Comprendre qu'une \textbf{espèce chimique} est constituée d'un très (très) grand nombre d'\textbf{entités chimiques}.
  \item Comprendre l'utilité et le concept de mole.
\end{objectifs}

\begin{contexte}
  % Le nombre d'électrons définit la configuration électronique d'un atome.
  % Cette configuration électronique définit combien d'électrons un atome pourra perdre ou gagner pour former un ion plus stable. \\
  % La configuration électronique définit aussi combien d'électrons un atome pourra partager dans des liaisons covalentes pour former des molécules.
  
  Les atomes, ions et molécules sont des entités chimique.
  
  \problematique{
    Comment passe-t-on d'une \textbf{entité chimique} microscopique à une \textbf{espèce chimique} macroscopique ?
    Comment dénombrer ces entités ?
  }
\end{contexte}


%%%% docs
\titreSection{Les espèces chimiques}

\begin{doc}{Entités chimiques}
  Il existe trois type d'entités chimiques :
  \begin{listePoints}
    \item les atomes (par exemple le cuivre \chemfig{Cu}).
    \item les ions (par exemple l'ion fluorure \chemfig{F^{-}}).
    \item les molécules (par exemple le méthane \chemfig{CH_{4}}).
  \end{listePoints}
  
  \begin{encart}
    Les ions positifs ($+$) \reponseLigne % sont appelés \important{cations}.
    
    Les ions négatifs ($-$) \dotfill % sont appelés \important{anions}. On ajoute le suffixe -ure.
  \end{encart}
\end{doc}

%%
\begin{doc}{Espèces chimiques}
  La matière macroscopique qui nous entoure est composé d'un très (très) grand nombre d'entités chimique identiques.
  
  \begin{encart}
    Au niveau macroscopique, la matière est électriquement neutre.
  \end{encart}
  
  Elle a une charge électrique globale nulle : on parle d'\important{électroneutralité.}
\end{doc}

%%
\begin{doc}{Espèces ioniques}
  \vspace*{-24pt}
  \begin{encart}
    Les ions vont toujours s'associer par groupe de charges opposées pour former des espèces neutres appelées \textbf{espèces ioniques}.
  \end{encart}
  
  Mis en solution dans de l'eau, les espèces ioniques se dissocient en \textbf{cations} (ions $+$) et en \textbf{anions} (ions $-$).
  
  \exemple le sel est composé d'ions sodium \chemfig{Na^{+}} et d'ions chlorure \chemfig{Cl^{-}}.
\end{doc}


%%%%
\titreSection{Dénombrer les entités}

%%
\begin{doc}{Masse d'une entité}
  \label{doc:masse_entite}
  La masse d'une entité composée de plusieurs atomes est égale à la somme des masses des atomes de l'entité.
  
  \exemple 
  $m(\chemfig{C_2 H_6 O}) = 2\times m(\chemfig{C}) + 6\times m(\chemfig{H}) + m(\chemfig{O})$
  

  \begin{multicols}{2}
    \begin{donnees}
      \item $m(\chemfig{H}) = 1,\!67 \times 10^{-24} \unit{g}$
      \item $m(\chemfig{C}) = 1,\!99 \times 10^{-23} \unit{g}$
      \item[\phantom{.}] 
      \item $m(\chemfig{O}) = 2,\!66 \times 10^{-23} \unit{g}$
      \item $m(\chemfig{Ca^{2+}}) = 6,\!66 \times 10^{-23} \unit{g}$
    \end{donnees}
  \end{multicols}
\end{doc}

%%
\begin{doc}{Composition de la coriandre pour 100 g}
  \label{doc:coriandre}
  \centering
  \begin{tabular}{|c|c|c|c|c|}
    \hline
    \rowcolor{gray!20!white}
    Constituant &
    Eau (\chemfig{H_2 O}) & 
    Ion calcium (\chemfig{Ca^{2+}}) &
    \saccha &
    Autres
    \\ \hline
    \cellcolor{gray!20!white}
    Masse &
    $92,2 \unit{g}$ &
    $67 \times 10^{-3} \unit{g}$ &
    $0,\!82 \unit{g}$ &
    $6,\!91 \unit{g}$
    \\ \hline
  \end{tabular}
\end{doc}


%%%%
\question{
  Classer les trois constituants de la coriandre du document~\ref{doc:coriandre} selon leur type (atome, molécule ou ion).
}{3}

\question{
  Dans la formule de la Saccharose, qu'indiquent les nombres en indice ?
}{1}

\question{
  En vous aidant du document~\ref{doc:masse_entite}, calculer la masse d'\textbf{une seule entité} pour les espèces chimiques constituant la coriandre.
}{4}

\newpage
\vspace*{-12pt}
\question{
  Calculer le nombre $N$ de molécules d'eau dans 100 g de coriandre. 
  Calculer de même le nombre d'ions calcium et de molécules de Saccharose.
}{5}

\begin{center}
  \begin{tabular}{|c|c|c|c|}
    \hline
    \rowcolor{gray!20!white}
    Constituant &
    Eau (\chemfig{H_2 O}) & 
    Ion calcium (\chemfig{Ca^{2+}}) &
    \saccha
    \\ \hline
    \cellcolor{gray!20!white}
    Nombre $N$ &
    \phantom{\ionsCa}& 
    \phantom{\ionsCa}&
    \\ \hline
  \end{tabular}
\end{center}

\question{
  Quel constituant contient le plus d'entités ?
  Ces nombres vous semblent-ils simple à manipuler ?
}{2}


\begin{doc}{La mole}
  Pour compter le grand nombre d’entités que contient un échantillon de matière, il faut regrouper toutes ces entités en paquets.
  Il est plus facile de compter des paquets d’atomes que tous les atomes
  
  \exemple   
  \begin{itemize}
    \item Avec 24 \oe{}ufs, on fait 4 boîtes de 6 \oe{}ufs chacune.
    \item Avec 236 Dragibus, on fait 9,4 paquets de 25 Dragibus chacun.
  \end{itemize}
  \begin{equation*}
    N_\text{paquets} = \Frac{N_\text{tous les Dragibus}}{N_\text{Dragibus dans un paquet}}
  \end{equation*}
  
  En chimie, on ne fait pas de paquets de 6 ou de 25, mais des paquets de $\avogadro$ !
  Un tel paquet s’appelle une mole.
  Une mole contient donc $\avogadro$ atomes !
  
  \begin{encart}
    Le nombre de paquets s’appelle le \important{nombre de moles} ou la \important{quantité de matière}.
    On la note $n$ et son unité dans le système international s’écrit \og mol \fg.
    
    La taille du paquet s’appelle le \important{nombre d’Avogadro} : 
    \begin{equation*}
      N_A = \avogadro \unit{mol}^{-1}
    \end{equation*}
    
    L’unité \og mol$^{-1}$ \fg\, signifie \og par mole \fg, c’est le nombre d’atomes dans une mole.
\end{encart}
\end{doc}


\question{
  Calculer $n$ le nombre de moles que contient chaque constituant de la coriandre.
}{5}

\begin{center}
  \begin{tabular}{|c|c|c|c|}
    \hline
    \rowcolor{gray!20!white}
    Constituant &
    Eau (\chemfig{H_2 O}) & 
    Ion calcium (\chemfig{Ca^{2+}}) &
    \saccha
    \\ \hline
    \cellcolor{gray!20!white}
    Nombre $n$ &
    \phantom{\ionsCa}&
    \phantom{\ionsCa}&
    \\ \hline
  \end{tabular}
\end{center}


\question{
  Bilan : quel est l'intérêt d'utiliser les moles pour compter le nombre d'entités chimiques ?
}{3}