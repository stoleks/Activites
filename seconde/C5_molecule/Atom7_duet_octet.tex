%%%%
\teteSndMole

%%%% titre
\vspace*{-36pt}
\numeroActivite{7}
\titreActivite{En quête de stabilité : les ions}


%%%% Objectifs
\vspace*{-8pt}
\begin{objectifs}
  \item Comprendre la règle du duet et de l'octet.
\end{objectifs}

\begin{contexte}
  Dans la nature la plupart des atomes vont spontanément perdre ou gagner des électrons pour former des ions.
  
  Seuls les gaz nobles de la 18$^\text{ème}$ colonne du tableau périodique (\chemfig{He}, \chemfig{Ne}, \chemfig{Ar}, \chemfig{Kr}, etc.) se trouvent le plus souvent sous forme de gaz monoatomiques.
  C'est parce qu'ils ont une grande stabilité, on dit qu'ils ont une grande inertie chimique.
  
  \problematique{
    Comment expliquer la formation d'ions monoatomique et la charge qu'ils portent à partir de la configuration électronique des gaz nobles ?
  }
\end{contexte}


%%%% question
\titreSection{Les gaz rares}

\question{
  Compléter le tableau suivant
}{0}

\begin{center}
  \begin{tabular}{|c | c | c | c |}
    \hline \rowcolor{gray!20}
     Gaz noble &
     Numéro atomique & Nombre d'électrons &
     Configuration électronique 
     \\ \hline
     Hélium \chemfig{He} & $Z = 2$  & &
     \\ \hline
     Néon \chemfig{Ne}   & $Z = 10$ & &
     \\ \hline
     Argon \chemfig{Ar}  & $Z = 18$ & &
     \\ \hline
  \end{tabular}
\end{center}

\question{
  Comment est la couche externe pour ces trois gaz nobles ?
}{1}


%%
\titreSection{La règle du duet et de l'octet}

Pour augmenter leur stabilité, les atomes adoptent la configuration électronique du gaz noble avec le numéro atomique le plus proche.

\begin{encart}
  \pointCyan \important{Règle du duet :} les atomes de numéro atomique $Z \leq 6$ tendent à adopter la \\[6pt]
  configuration électronique de \dotfill \\[6pt]
  %l'hélium à deux électrons : \textbf{1s$^2$}.
  Ils ont \lignePointillee{0.2} 
  %2 (duet)
  électrons sur leur couche externe.
  \bigskip
  
  \pointCyan \important{Règle de l'octet :} les atomes de numéro atomique $Z > 6$ tendent à adopter \\[6pt]
  la configuration électronique externe du gaz noble le plus proche avec
  \dotfill \\[6pt]
  % huit électrons : \textbf{ns$^2$ ns$^6$}.
  Ils ont \lignePointillee{0.2}
  %8 (octet)
  électrons sur leur couche externe.
\end{encart}


%%
\titreSection{Les ions monoatomiques}

Pour adopter une configuration électronique plus stable, les atomes vont spontanément perdre ou gagner des électrons et ainsi former des ions.
\bigskip

\question{
  Le lithium \chemfig{Li} a pour numéro atomique $Z = 3$.
  Rappeler sa configuration électronique.
  Pour devenir stable, quelle règle doit-il respecter ? 
  Combien d'électrons doit-il perdre pour la respecter ?
  Quel ion formera-t-il ?
}{4}


\question{
  Mêmes questions pour le soufre \isotope{}{16}{S} ($Z = 16$).
}{4}


\question{
  Par analogie avec le soufre \isotope{}{16}{S}, pouvez-vous répondre simplement aux mêmes questions pour l'oxygène \isotope{}{8}{O} ?
}{4}


\question{
  Comment répondre à ces questions en regardant simplement le tableau périodique ?
}{4}