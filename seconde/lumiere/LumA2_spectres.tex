%%%%
\teteSndLumi

%%%% titre
\vspace*{-36pt}
\numeroActivite{2}
\titreActivite{Spectre d'émission}


%%%% Objectifs
\begin{objectifs}
  \item Comprendre la notion de spectre d'émission.
  \item Analyser le spectre d'émission d'une lampe.
\end{objectifs}

\begin{contexte}
  Il existe différentes sources lumineuse, comme le Soleil, les lampadaires, les néons, les écrans de téléphones, etc.
  
  \problematique{
    Comment caractériser la lumière émise par une source ?
  }
\end{contexte}


%%%% evaluation
\begin{tableauCompetences}
  \centering VAL &
  Comparer avec des valeurs de références.
  & & & & \\
\end{tableauCompetences}


%%%% docs
\begin{doc}{Spectre d'émission}{doc:A2_spectre_emission}
  La lumière est une onde électromagnétique, qui peut avoir plusieurs longueurs d'ondes.
  Nos yeux captent certaines longueurs d'ondes et y associent une couleur : c'est le domaine visible.
  
  \begin{importants}
    La donnée de toutes les longueurs d'ondes présentes dans une source lumineuse s'appelle le \important{spectre d'émission}.
    Le spectre dans le domaine visible est représenté de la manière suivante :
  \end{importants}
  
  \begin{center}
    \image{0.6}{images/lumiere/spectre_visible}
  \end{center}
\end{doc}


%%
\titreSection{Les spectre d'émissions continus}

\begin{doc}{Spectre continu}{doc:A2_spectre_continu}
  \begin{importants}
    Un \important{spectre d'émission continu} présente une suite de raies colorées.
    Un spectre continu prend la forme d'une bande colorée unique.
  \end{importants}
\end{doc}

\begin{doc}{Lampe à incandescence}{doc:A2_lampe_incandescence}  
  Une lampe à incandescence est composé d'un petit filament chauffé par le passage d'un courant électrique.
  En augmentant la tension d'alimentation d'une lampe à incandescence, on augmente la température du filament.
\end{doc}

\question{
  Quelles différences remarquez-vous quand la lampe est alimentée en 6 et en \qty{12}{\volt} ?
}{
  La lampe émet plus de lumière et est plus blanche quand elle est alimentée en \qty{12}{\volt}.
}{3}


\begin{doc}{Émission d'un corps chaud}{doc:A2_corps_chaud}
  \begin{importants}
    Un corps chaud émet \texteTrouLignes[1]{un rayonnement lumineux avec un spectre continu.} 
    Les propriétés du rayonnement lumineux dépendent de la température de l'objet.
    Quand \important{la température du corps augmente}, sa \important{luminosité augmente} et son spectre contient de \important{plus petites longueurs d'onde,} ce qui correspond à des couleurs plus « froides » (bleue ou violet).
  \end{importants}
\end{doc}

\question{
  Utilisons ce résultat pour estimer la température de surface d'une étoile.
  Bételgeuse est une étoile de couleur rouge-orange, sa température de surface vaut \qty{3800}{\degreeCelsius}.
  L’étoile Rigel est de couleur bleue. Sa température sera-t-elle plus élevée ou plus faible ? 
}{
  Comme sa couleur est bleue, la longueur d'onde associée est plus petite pour l'étoile Rigel que pour l'étoile Bételgeuse. 
  Donc sa température est plus élevée d'après la loi des corps chaud.
}{2}


%%
\titreSection{Les spectres d’émission de raies}

\begin{doc}{Émission atomique ou moléculaire}{doc:A2_emission_atomique}
  \begin{importants}
    Lorsque les entités chimiques (atomes, ions, molécules), qui composent un gaz sont excitées, elles émettent des radiations avec des longueurs d'ondes précises.
    
    Cela correspond à des \important{raies fines et bien définies} dans le spectre d'émission.
  \end{importants}
  
  \begin{wrapfigure}{r}{0.55\linewidth}
    \centering
    \vspace*{-22pt}
    \image{1}{images/donnees/spectre_gaz}
  \end{wrapfigure}
    
  Chaque entité chimique possède son propre \important{spectre d'émission} caractérisé par des longueurs d'onde précises, comme chaque humain possède ses propres empreintes digitales.
  \medskip

  Observer un spectre d'émission permet donc \important{d'identifier} les entités présentes dans un gaz.
  \medskip

  En regardant le spectre d'une source lumineuse, on peut donc déterminer les éléments chimiques qui composent la source.
\end{doc}


\question{
  En utilisant le spectroscope et en comparant avec les spectres données dans le document~\ref{doc:A2_emission_atomique}, indiquer si les lampes éclairant la classe contiennent de l'hydrogène, du néon ou du mercure.
}{
  En utilisant les spectroscope, on peut observer différentes raies d'émission.
}{5}