%%%%
\teteSndLumi

%%%% titre
\vspace*{-30pt}
\numeroActivite{1}
\titreActivite{Ondes lumineuses}


%%%% Objectifs
\begin{objectifs}
  \item Connaître la vitesse de la lumière.
  \item Comprendre la notion de longueur d'onde.
  \item Comprendre la notion de rayonnement monochromatique.
\end{objectifs}

\begin{contexte}
  La lumière est en fait une onde électromagnétique, constitué d'un champs électrique et d'un champs magnétique.
  
  \problematique{
    Quelles sont les propriétés de cette onde électromagnétique ?
  }
\end{contexte}


%%%% docs
\begin{doc}{Onde électromagnétique}{doc:A1_onde_EM}
  \begin{importants}
    Une onde est une \important{perturbation} qui se \important{propage,} sans transport de matière.
  \end{importants}
  
  Une onde électromagnétique est une perturbation du champs électrique et magnétique qui se propage.
  Une onde peut être décrite par un certain nombre de propriétés qui la définisse.
  Cette année on va se concentrer sur sa \important{vitesse de propagation} et sur sa \important{longueur d'onde,} notée $\lambda$ (« lambda »).
  
  \begin{importants}
    Une onde est dite \important{monochromatique} (une couleur) si elle a une longueur d'onde bien définie.
    
    Une onde est dite \important{polychromatique} (plusieurs couleurs) si elle est la superposition de plusieurs ondes monochromatique.
  \end{importants}
\end{doc}

\question{
  Chercher et donner des exemples de phénomènes qui s'apparentent à des ondes.
}{
  Les vagues sur la mer, le son, les séismes, la vibration d'une corde de guitare, la vibration d'une plaque métallique, les vaguelette crée sur une surface d'eau quand on y jette un objet, etc.
}{8}


%%%%
\begin{qcm}{
  Le soleil est une source de lumière qui émet une onde électromagnétique
}
  \item monochromatique, avec une longueur d'onde.
  \item \reponseQCM polychromatique, avec plusieurs longueurs d'onde.
\end{qcm}

\begin{qcm}{
  Un laser est une source de lumière qui émet une onde électromagnétique
}
  \item \reponseQCM monochromatique, avec une longueur d'onde.
  \item polychromatique, avec plusieurs longueurs d'onde.
\end{qcm}


%%
\begin{doc}{Spectre électromagnétique}{doc:A1_spectre_EM}
  Le spectre électromagnétique est le classement des ondes électromagnétique par longueur d'onde. 
  \begin{center}
    \image{0.8}{images/lumiere/spectre_EM}
  \end{center}
  Le domaine visible se trouve entre \important{400 nm (violet)} et \important{700 nm (rouge)} de longueur d'onde et représente une petite partie du spectre électromagnétique.
\end{doc}



%%
\begin{doc}{Vitesse de propagation}{doc:A1_vitesse_propagation}
  \begin{importants}
    Dans le vide, une onde électromagnétique se propage à la vitesse de la lumière notée $c$
    \begin{equation*}
      c = \qty{3,00e8}{\m\per\s}
    \end{equation*}
  \end{importants}
\end{doc}

Pour mieux visualiser la vitesse de la lumière, on va la comparer avec la vitesse d'un TGV.
Un TGV a une vitesse de pointe de $\qty{300}{\km\per\hour} = \qty{83,3}{\m\per\s}$.
  
\question{
  Calculer le temps que met le TGV pour parcourir \qty{e6}{\m} (distance Paris-Marseille).
}{
  \begin{align*}
     t_{\text{TGV}} &= \frac{d_\text{Paris-Marseille}}{v_\text{TGV}} \\
       &= \frac{\qty{e6}{\m}}{\qty{83,3}{\m\per\s}} \\
       &= \qty{1.20e4}{\s}
  \end{align*}
  \vspace*{-24pt}
  \phantom{b}
}{2}

\question{
  Calculer le temps que met la lumière pour parcourir \qty{e6}{\m}.
  Comparer les deux temps de parcours.
}{
  \begin{align*}
    t_{\text{lumière}} &= \frac{d_\text{Paris-Marseille}}{c} \\
      &= \frac{\qty{e6}{\m}}{\qty{3,0e8}{\m\per\s}} \\
      &= \qty{3,3e-3}{\s}
  \end{align*}
  \phantom{b}\\[-24pt]
  La lumière est beaucoup plus rapide qu'un TGV : le temps que le TGV arrive à Marseille, la lumière aura fait 2 millions de fois l'aller-retour !
}{3}


%%
\begin{doc}{Longueur d'onde et énergie}{doc:A1_longueur_onde}
  L'énergie d'une onde électromagnétique est liée à sa longueur d'onde.
  Plus la longueur d'onde est petite et plus l'énergie d'une onde électromagnétique est élevée. 
  Il peut être dangereux d'être exposé à une onde électromagnétique avec une énergie élevée, qui pourrait endommager les tissus vivants.
  
  Une onde électromagnétique très énergétique, dans le domaine des rayons X, peut briser les liaisons covalentes d'une molécules ou arracher des électrons d'un atome, ce qui peut tuer des cellules vivantes.
\end{doc}

\question{
  Expliquer pourquoi un laser rouge est moins dangereux qu'un laser bleu.
}{
  Un laser rouge émet une onde électromagnétique avec une longueur d'onde plus élevée qu'un laser bleu. L'énergie de cette onde électromagnétique est donc plus faible et le laser rouge est moins dangereux.
}{3}