%%%%
\teteSndLumi

%%%% titre
\vspace*{-30pt}
\numeroActivite{3}
\titreActivite{Grandissement d'une image}


%%%% Objectifs
\begin{objectifs}
  \item Comprendre l'approche géométrique pour construire l'image d'un objet avec une lentille convergente à partir de rayons lumineux particuliers.
\end{objectifs}


%%%%
\begin{doc}{Rappel sur la détermination graphique d'une image}{doc:A3_formation_image}
  \begin{importants}
    Une lentille convergente possède un \important{centre optique $O$,} un \important{foyer image $F'$}et un \important{foyer objet $F$.}
    La droite perpendiculaire à la lentille passant par le centre optique $O$ est appelée \important{l'axe optique.}
  \end{importants}
  L'image d'un objet $AB$ est notée $A'B'$.
  
  \begin{center}
    \image{0.75}{images/lumiere/image_lentille_convergente}
  \end{center}
  
  \begin{importants}
    Trois rayons ont des propriétés particulières pour une lentille convergente :
  \begin{listePoints}
    \item Tout rayon incident qui passe par le centre optique n'est pas dévié.
    \item Tout rayon incident qui passe par le foyer objet $F$ émerge parallèle à l'axe optique.
    \item Tout rayon incident parallèle à l'axe optique émerge en passant par le foyer image $F'$.
  \end{listePoints}
  \end{importants}
  Pour trouver où se forme l'image d'un point, on trace deux rayons particuliers qui partent de ce point. 
  L'image du point sera nette là où ces rayons lumineux s'intersectent (se croisent).
\end{doc}

%%
\begin{doc}{Grandissement d'une image}{doc:A3_grandissement}
  
  \begin{importants}
    En optique les longueurs sont \important{algébriques,} c'est-à-dire qu'elles sont positives ou négatives en fonction de leur sens, on les note avec une barre $\algebrique{AB}$.
  \end{importants}
  \begin{listePoints}
    \item $\algebrique{AB} > 0$, si B est au dessus de A (ou si B est à droite de A) ;
    \item $\algebrique{AB} < 0$, si B est en dessous de A (ou si B est à gauche de A).
  \end{listePoints}
  
  \begin{importants}
    Le \important{grandissement} noté $\gamma$ (gamma) est le rapport entre la hauteur algébrique de l'image et celle de l'objet
    \begin{equation*}
      \gamma = \dfrac{\algebrique{A'B'}}{\algebrique{AB}}
    \end{equation*}
  \end{importants}
  Si $\gamma < 0$ l'image est renversée.
  Si $|\gamma| > 1$ l'image est plus grande que l'objet. 
  Si $|\gamma| < 1$ l'image est plus petite que l'objet.
\end{doc}


%%%% 
\nomPrenomClasse

\numeroQuestion
Tracer l'image $A'B'$ pour chacun des 3 cas suivants, $P$ est un point tel que $\algebrique{OP} = 2 \times \algebrique{OF}$.

\begin{center}  
  \image{0.7}{images/lumiere/formation_image_lentille_conv0002}
  \vspace*{24pt}
  
  \image{0.7}{images/lumiere/formation_image_lentille_conv0003}
  \vspace*{24pt}
  
  \image{0.7}{images/lumiere/formation_image_lentille_conv0004}
\end{center}


\question{
  Est-ce que l'image $A'B'$ obtenue graphiquement est cohérente avec celle observée dans ces 3 situations pendant le TP \arabic{section}.1 ?
}{
  Oui, on retrouve bien les trois configurations étudiées pendant le TP : un cas où l'image est agrandie avec un objet proche de la lentille ;
  un cas ou l'objet et l'image ont la même taille et sont à la même distance de la lentille ;
  un cas où l'image est rétrecie avec un objet loin de la lentille.
}[3]


\question{
  En utilisant le théorème de Thalès sur les triangles ABO et A'B'O dans le document~\ref{doc:A3_formation_image}, montrer que 
  $\gamma = \algebrique{OA'} / \algebrique{OA} = g$, comme mesuré dans le TP \arabic{section}.1.
}{
  ...
}[3]