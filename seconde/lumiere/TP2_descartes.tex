%%%%
\teteSndLumi
%\nomPrenomClasse

%%%% titre
\numeroActivite{2}
\titreTP{La réfraction de la lumière}

% \begin{tableauCompetences}
%   REA & Réaliser une série de mesures avec précision.
% \end{tableauCompetences}


%%%% Objectifs
\begin{objectifs}
  \item Comprendre comment décrire le phénomène de réfraction.
  \item Découvrir la loi de Snell-Descartes.
\end{objectifs}

\begin{contexte}
  La lumière se propage en ligne droite dans un même milieu transparent.
  Lorsque la lumière passe d'un milieu à un autre sa direction de propagation change : c'est le phénomène de \important{réfraction.}

  En arrivant avec certains angles, la lumière peut aussi être \important{réfléchie}, c'est le phénomène de \important{réflexion.}
  
  \problematique{
    Comment décrire mathématiquement le phénomène de réfraction et de réflexion ?
  }
\end{contexte}


%%%% docs
\begin{doc}{Indice de réfraction}{doc:TP2_refraction}
  Quand la lumière se propage dans un milieu, sa vitesse est réduite.
  
  \begin{importants}
    La capacité d'un milieu à réduire la vitesse de la lumière est mesurée par un nombre que l'on appelle \important{l'indice de réfraction} et que l'on note $n_\text{milieu}$.
    
    Dans le milieu, la vitesse de la lumière est
    \begin{equation*}
      c_\text{milieu} = \dfrac{c}{n_\text{milieu}}
    \end{equation*}
  \end{importants}
  
  \exemple
  \begin{listePoints}
    \item L'air a un indice de réfraction $n_\text{air} = 1,\!00$ et donc $c_\text{air} = c = 3,\!00 \times 10^8 \unit{m.s}^{-1}$.
    \item L'eau a un indice de réfraction $n_\text{eau} = 1,\!33$ et donc $c_\text{eau} = 2,\!26 \times 10^8 \unit{m.s}^{-1}$.
  \end{listePoints}
\end{doc}

\begin{doc}{Mesure de l'indice de réfraction}{doc:TP2_exp_disque_optique}
  \begin{wrapfigure}{r}{0.5\linewidth}
    \vspace*{-35pt}
    \centering
    \image{0.9}{images/lumiere/disque_optique_refraction}
  \end{wrapfigure}
  \important{Matériel utilisé :}
  \begin{itemize}
    \item 1 source de lumière alimentée en 12 V continu ;
    \item 1 demi-cylindre de plexiglas sur son disque-support gradué en degrés.
  \end{itemize}
  \bigskip

  Votre professeur préféré a réalisé les mesures suivantes avec ce dispositif expérimental :
  \begin{center}
    \begin{tblr}{
      columns = {c},
      hlines, vlines,
      column{1} = {l, couleurPrim!20},
    }
      Angle d'incidence $i_1$   & 0 & 5 & 10 & 15 & 20 & 30 & 40 & 50 & 60 & 70 & 80 & 90 \\
      Angle de réfraction $i_2$ & 0 & 3.3 & 6.7 & 9.9 & 13.2 & 19.5 & 25.4 & 30.7 & 35.3 & 38.8 & 41.0 & 41.8 \\
    \end{tblr}
  \end{center}
\end{doc}

\mesure
Ouvrir le programme python \texttt{refraction\_1.py} et le lire en entier.

\mesure
Dans le programme python \texttt{refraction\_1.py}, repérer les lignes correspondant aux angles $i_1$ et $i_2$ mesurés.
Les remplir avec les valeurs du document~\ref{doc:TP2_exp_disque_optique} et lancer le programme.

\begin{doc}{La proportionnalité}{doc:TP2_proportionnalite}
  Deux grandeurs $a$ et $b$ sont \important{proportionnelles} si le graphique représentant la grandeur $a$ en fonction de la grandeur $b$ est une droite passant par l'origine du repère.
  Ces deux grandeurs $a$ et $b$ sont alors reliées par l'égalité 
  \begin{equation*}
    a = k\times b
  \end{equation*}
  Dans cette égalité $k$ est une constante. $k$ est le \important{coefficient directeur} de la droite.
\end{doc}


%%%%
\question{
  Est-ce que l'on a une relation de proportionnalité entre $i_1$ et $i_2$ ? Justifier à partir du graphique obtenu.
}{
  Non, car les points ne sont pas alignés sur une droite.
}[2]

\mesure
Ouvrir le programme python \texttt{refraction\_2.py} et repérer les lignes correspondant aux angles $i_1$ et $i_2$.
Les remplir en les copiant depuis \texttt{refraction\_1.py} et lancer le programme.

\question{
  Est-ce que l'on a une relation de proportionnalité entre $\sin(i_1)$ et $\sin(i_2)$ ?
  Justifier à partir du graphique obtenu.
}{
  Oui, car les point sont alignés sur une droite.
}[2]


%%%%
\begin{doc}{Loi de Snell-Descartes}{doc:TP2_loi_snell_descartes}
  \begin{importants}
    Lorsque la lumière passe d'un milieu d'indice $n_1$ à un milieu d'indice $n_2$, alors
    \begin{listePoints}
      \item le rayon incident, le rayon réfracté et la normale sont \texteTrouLignes[1]{dans le même plan.}
      \item \texteTrou[1]{$n_1 \sin(i_1) = n_2 \sin(i_2)$ pour la réfraction.}
      \item \texteTrou[1]{$i_3 = i_1$ pour la réflexion.}
    \end{listePoints}
    
    La relation entre l'angle d'incidence $i_1$ et l'angle de réfraction $i_2$ s'appelle la \important{loi de Snell-Descartes}.
  \end{importants}
  
  On retrouve bien la relation de proportionnalité mesurée :
  \begin{equation*}
    \sin(i_2) = \dfrac{n_1}{n_2} \times \sin(i_1)
    n_2 \sin(i_2) = n_1 \sin(i_1)
  \end{equation*}
\end{doc}

\question{
  En utilisant la valeur du coefficient directeur 
  $k = n_\text{air} / n_\text{plexiglas}$
  calculée par le second programme python, calculer la valeur de l'indice de réfraction $n_\text{plexiglas}$.
}{
  On trouve que $k = \num{0,67}$ et comme $k = \dfrac{n_1}{n_2}$, 
  \begin{align*}
    0,67 =& \dfrac{n_1}{n_2} \\
    n_2 \times 0,67 =& n_1 \\
    n_2 = \dfrac{n_1}{0,67} = \dfrac{1,00}{0,67} = 1,49
  \end{align*}
}[3]