%%%%
\teteSndMeth

%%%% titre
\vspace*{-36pt}
\titreActivite{Notation scientifique et unités}

% \begin{objectifs}
%   \item Revoir les puissances de 10 et la notation scientifique
% \end{objectifs}

%%%%
\vspace*{-20pt}
\titreSection{Rappels sur les puissance de 10}

%%
\begin{doc}{Les puissances de 10}{doc:A1_puissance_10}
  Les puissances indiquent qu'on va répéter une multiplication ($2^3 = 2 \times 2 \times 2 = 8$).
  
  Pour lire les puissances de 10, il suffit de suivre deux règles simple
  \begin{importants}
    \pointCyan Écrire le nombre $10^a$ (avec $a = 0, 1, 2, 3, \ldots$), revient à écrire ``$1$'' suivi de $a = 0, 1, 2, 3, \ldots$ zéros. \\
    \exemple \num{e3} = \texteTrou[0.2]{\num{1000}}

    \pointCyan Écrire le nombre $10^{-a}$ (avec $a = 1, 2, 3, \ldots$), revient à écrire ``$0,$'' suivi de $a - 1 = 0, 1, 2, \ldots$ zéros et d'un $1$. \\
    \exemple \num{e-2} = \texteTrou[0.2]{\num{0,01}}
  \end{importants}
\end{doc}


\numeroQuestion Écrire les nombres correspondant aux puissances de 10 suivantes : \\
$\num{e2}  =$ \texteTrou[0.1]{\num{100}} \qq{}
$\num{e5}  =$ \texteTrou[0.1]{\num{100000}} \qq{}
$\num{e-3} =$ \texteTrou[0.1]{\num{0,001}} \qq{}
$\num{e-1} =$ \texteTrou[0.1]{\num{0,1}}

\numeroQuestion Écrire les nombres suivants comme le produit d'un nombre compris entre 0 et 9 et d'une puissance de 10 \exemple \num{600} = \num{6,00e2} :
\pasCorrection{\vspace*{-4pt}}
\begin{listePoints}[2]
  \setlength\itemsep{-4pt}
  \item \num{100000}  = \texteTrouLignes{\num{1e5}}
  \item \num{1}       = \texteTrouLignes{\num{1e0}}
  \item \num{9000000} = \texteTrouLignes{\num{9e6}}
  \item \num{0,1}     = \texteTrouLignes{\num{1e-1}}
  \item \num{0,0006}  = \texteTrouLignes{\num{1e-4}}
  \item \num{0,00705} = \texteTrouLignes{\num{7,05e-3}}
\end{listePoints}


%%
\pasCorrection{\vspace*{-26pt}}
\begin{doc}{Règles de calculs}{doc:A1_calcul_puissance_10}
  Il y a deux règles de calculs à connaître pour les puissances de 10
  \begin{importants}
    \pointCyan $10^a \times 10^b = 10^{a + b}$ \\   
    \pointCyan $10^{-a} = \dfrac{1}{10^a}$
  \end{importants}
\end{doc}


\numeroQuestion Réaliser les calculs suivants :
\pasCorrection{\vspace*{-4pt}}
\begin{listePoints}[2]
  \setlength\itemsep{-4pt}
  \item $10^2 \times 10^1 =$ \texteTrouLignes{$10^{2 + 1} = 10^3 = 1000$}
  \item $10^4 \times 10^{-3} =$ \texteTrouLignes{$10^{4 - 3} = 10^1 = 10$}
  \item $10^{-2} \times 10^{-3} =$ \texteTrouLignes{$10^{-2 - 3} = 10^{-5} = 0,00001$}
  \item $10^{-1} \times 10^{-5} \times 10^4 =$ \texteTrouLignes{$10^{-1 - 5 + 4} = 10^{-2} = 0,01$}
\end{listePoints}


%%
\pasCorrection{\vspace*{-26pt}}
\begin{doc}{Moyen mnémotechnique}{doc:A1_decalage_virgule}
  \begin{listePoints}
    \setlength\itemsep{-4pt}
    \item Si je décale la virgule de 1 rang vers la gauche, alors
    \texteTrou[0.25]{je réduis} de 1 unité la puissance de dix. \texteTrou{J'ai divisé par 10.}
    \item Si je décale la virgule de 1 rang vers la droite, alors
    \texteTrou[0.35]{j'augmente} de 1 unité la puissance de dix. \texteTrou{J'ai multiplié par 10.}
  \end{listePoints}
\end{doc}


%%%%
\newpage
\vspace*{-36pt}
\titreSection{Notation scientifique}

\begin{doc}{La notation scientifique}{doc:A1_notation_scientifique}
  \begin{importants}
  La \important{notation scientifique} d'une quantité se présente de la façon suivante :
  % Textes
  \begin{center}
    \texteEncadre{chiffre différent de zéro}
    \hspace{5pt}
    \texteEncadre{autres chiffres} 
    \qq{}
    \texteEncadre{puissance de dix}
    \texteEncadre{\important{unité}}
  \end{center}
  % Virgule et multiplication
  \vspace*{-32pt}
  \begin{tikzpicture}  
    \draw [white] (0,0) circle;
    \draw [couleurSec, thick] (5.6,-0.4) circle [radius=0.3] node {$,$};
    \draw [couleurSec, thick] (9.5,0.0) circle [radius=0.3] node {$\times$};
  \end{tikzpicture}
  \end{importants}
\end{doc}

\numeroQuestion Écrire les quantités suivantes en notation scientifique :
\vspace*{-8pt}
\begin{listePoints}[2]
  \item \qty{288}{\hour}      = \texteTrouLignes{\qty{2,88e2}{\hour}\\}
  \item \qty{1}{\m}           = \texteTrouLignes{\qty{1e0}{\m}\\}
  \item \qty{756 864 000}{\s} = \texteTrouLignes{\qty{7,56 864 000}{8\s}\\}
  \item \qty{638}{\newton}    = \texteTrouLignes{\qty{6,38e2}{\newton}}
  \item \qty{0,01}{\percent}  = \texteTrouLignes{\qty{1,0e-2}{\percent}\\}
  \item \qty{8960}{\g/\l}     = \texteTrouLignes{\qty{8,96e3}{\g/\l}\\}
  \item \qty{0,436}{\s}       = \texteTrouLignes{\qty{4,36e-1}{\s}\\}
  \item \qty{0,336}{\s}       = \texteTrouLignes{\qty{3,36e-1}{\s}}
\end{listePoints}
\vspace*{-16pt}

\attention Il faut \important{toujours} préciser \important{l'unité} d'une grandeur quand on réalise un calcul !
Les grandeurs sans unités sont rares en physique-chimie.


%%%%
\titreSection{Le système international de mesure}

%%
\begin{doc}{Le système international}{doc:A1_SI}
  Pour comparer des grandeurs entre elles, il faut les exprimer avec les \important{mêmes unités de mesures.} % exemple centime et euros
  
  Pour pouvoir communiquer facilement d'un pays à un autre, le \important{système international (SI)} a été développé par la Conférence Générale des Poids et Mesures (CGPM).
  Le système international est composé de \important{sept unités de bases.}

  En physique on est amené à décrire des \important{échelles} très variées, par exemple quand on mesure la taille d'un cheveu ($\sim \qty{e-6}{\metre}$) ou la taille d'une planète ($\sim \qty{e7}{\metre})$.
  
  \begin{importants}
    Pour simplifier la manipulation des grandeurs éloignées de l'unité, chaque \important{puissance de \num{1000}} est associée à un \important{préfixe} dans le système international.
  \end{importants}

  \begin{center}
    \begin{tblr}{
      hlines, vlines, row{1} = {couleurPrim!20}, colspec = {c c c l}
    }
      Puissance  & Préfixe & Symbole & Nombre décimal \\
      \num{e12}  & tera    & T       & \num{1 000 000 000 000} \\
      \num{e9}   & giga    & G       & \num{1 000 000 000} \\
      \num{e6}   & mega    & M       & \num{1 000 000} \\
      \num{e3}   & kilo    & k       & \num{1 000} \\
      $10^0$     &         &         & \num{1} \\
      \num{e-3}  & milli   & m       & \num{0,001} \\
      \num{e-6}  & micro   & $\mu$   & \num{0,000 001} \\
      \num{e-9}  & nano    & n       & \num{0,000 000 001} \\
      \num{e-12} & femto   & f       & \num{0,000 000 000 001}
    \end{tblr}
  \end{center}
\end{doc}
