%%%% début de la page
\teteSndMouv

\nomPrenomClasse
\titreTP{Décrire le mouvement}

\begin{objectifs}
  \item Décrire un mouvement.
  \item Comprendre la notion de référentiel.
  \item Comprendre que le mouvement dépend du référentiel.
\end{objectifs}

\begin{contexte}
  En fonction du point de vue avec lequel on observe un objet qui bouge, son mouvement peut changer d'apparence.

  \problematique{
    Comment décrire le mouvement d'un objet en fonction du référentiel choisi ?
  }
\end{contexte}


%%%% evaluation
\pasCorrection{
\begin{tableauCompetences}
  APP & Représenter une situation par un schéma avec une légende. \\
  COM & Travailler en groupe, communiquer à l'oral. \\
\end{tableauCompetences}
}

%%%%
\begin{doc}{Un peu de vocabulaire}{doc:TP1_vocabulaire}
  \begin{importants}
    \important{Système} : objet dont on étudie le mouvement.
  \end{importants}
  
  \begin{importants}
    \important{Trajectoire} : ensemble des positions successives occupées par le système.
  \end{importants}
  
  Le \important{mouvement} d'un système est donné par la description de sa trajectoire et de l'évolution de sa vitesse.
\end{doc} 


\begin{doc}{Type de trajectoires}{doc:TP1_trajectoires}
  Trajectoire \important{rectiligne} : \texteTrou{trajectoire représentée par une droite.}
  
  \texteTrou[0.5]{Trajectoire circulaire} : trajectoire représentée par un cercle.
  
  Trajectoire \important{curviligne} : \texteTrou{trajectoire représentée par une courbe.}
\end{doc}


\begin{doc}{Vitesse et accéleration}{doc:TP1_vitesse}
  Vitesse \important{uniforme} (constante) : le système n’accélère pas.
  
  La vitesse augmente : \texteTrouLignes{le système accélère.}
  
  La vitesse diminue : \texteTrouLignes{le système décélère.}
  
  Si la vitesse \important{est constante et nulle,} on dit que le système est \important{immobile}.
\end{doc}


%%%%
\numeroQuestion
Compléter les documents~\ref{doc:TP1_trajectoires} et~\ref{doc:TP1_vitesse}.

\fleche Pour la suite de cette activité, vous allez choisir entre l'étude du mouvement des oies ou de la Lune.
Vous présenterez ensuite les résultats de votre étude au reste de la classe à l'oral.

\fleche Vous rendrez ensuite une compte-rendu détaillée en suivant les questions sur le \important{mouvement que vous n'avez pas choisi.}
Il faudra donc être attentif-ve à ce que disent vos camarades !


%%%%
\pasCorrection{ \newpage \vspace*{-40pt} }
\titreSousSection{Étude du mouvement des oies}

Le compteur du bateau affiche une vitesse $v_\text{bateau} = \qty{36}{\km/\hour}$.

\question{
  Pour la personne qui filme les oies, quelle est la vitesse des oies ?
}{
  Elle est nulle, l'oie apparaît immobile.
}

\mesure Schématiser la trajectoire des oies si on les observe depuis la berge.
\vspace*{120pt}

\question{
  Décrire le mouvement des oies depuis le bateau et depuis la berge.
}{
  Depuis le bateau l'oie est immobile.
  Depuis la berge l'oie a une trajectoire rectiligne uniforme.
}[2]


%%%%
\titreSousSection{Étude du mouvement de la Lune}

La Lune tourne autour de la Terre a une vitesse $v_\text{Lune} = \qty{3700}{\km/\hour}$.


\question{ 
  Décrire le mouvement de la Lune depuis le point de vue centré sur la Terre.
}{
  La Lune a une trajectoire circulaire uniforme.
}[1]

\mesure Schématiser la trajectoire de la Lune depuis le point de vue centré sur la Terre et  depuis le point de vue centré sur le Soleil.
\vspace*{120pt}


%%%%
\titreSousSection{Notion de référentiel}

\question{
  Convertir la vitesse $v_\text{Lune}$ en \unit{\m/\s}.
  \textit{Rappel :} \qty{1}{\km} = \qty{e3}{\metre}, \qty{1}{\hour} = \qty{3600}{\s}.
}{
  \begin{equation*}
    v_\text{Lune}
    = \num{3,7e3} \cdot \dfrac{\unit{\km}}{\unit{\hour}}
    = \num{3,7e3} \times \dfrac{\num{e3}}{\num{3.6e3}} \cdot \dfrac{\unit{m}}{\unit{s}}
    = \num{1,03e3} \cdot \dfrac{\unit{\m}}{\unit{s}}
  \end{equation*}
}[2]

\question{
  Quelle distance la Lune parcours pendant 1 seconde ?
  Comparer avec la longueur de sa trajectoire, qui est de \qty{2,4e6}{\km}.
}{
  \begin{equation*}
    d 
    = v_\text{Lune} \times \qty{1}{\s}
    = \qty{1,03e3}{\m\per\s} \times \qty{1}{\s}
    = \qty{1,03e3}{\m}
  \end{equation*}
  Elle a donc parcouru moins d'un millième de sa trajectoire, c'est comme si sur une échelle de 1 mètre on parcourait 1 millimètre.
}[1]

\question{
  Peut-on décrire la trajectoire de la Lune en l'observant à l’œil nu pendant 1 seconde ?
}{
  Non, car elle n'a presque pas bougé sur sa trajectoire et on ne perçoit donc pas son mouvement.
}[2]
\nopagebreak

\enlargethispage{8pt}
\nopagebreak
\begin{importants}
  On voit que le mouvement dépend du point de vue d'observation et du temps passé à observer un objet. 
  Il faut donc bien définir le \important{référentiel} utilisé pour étudier le mouvement.
\end{importants}