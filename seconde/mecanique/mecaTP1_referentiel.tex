%%%% début de la page
\teteSndMouv

%%%%
\nomPrenomClasse


%%%% titre
\numeroActivite{1}
\titreTP{Décrire le mouvement}


%%%% objectifs
\begin{objectifs}
  \item Décrire un mouvement.
  \item Comprendre la notion de référentiel.
  \item Comprendre que le mouvement dépend du référentiel.
\end{objectifs}


%%%% evaluation
\begin{tableauCompetences}
  APP &
  Représenter une situation par un schéma avec une légende.
  & & & & \\
  %
  COM &
  Travailler en groupe, communiquer à l'oral.
  & & & & \\
\end{tableauCompetences}

%%%%
\vspace*{6pt}
\begin{doc}{Un peu de vocabulaire}{doc:TP1_vocabulaire}
  \begin{importants}
    \important{Système} : objet dont on étudie le mouvement.
  \end{importants}
  
  \begin{importants}
    \important{Trajectoire} : ensemble des positions successives occupées par le système.
  \end{importants}
  
  Le \important{mouvement} d'un système est donné par la description de sa trajectoire + l'évolution de sa vitesse.
\end{doc} 


\begin{doc}{Type de trajectoires}{doc:TP1_trajectoires}
  Trajectoire \important{rectiligne} : \texteTrou{trajectoire représentée par une droite.}
  
  \texteTrou{Trajectoire circulaire} : trajectoire représentée par un cercle.
  
  Trajectoire \important{curviligne} : \texteTrou{trajectoire représentée par une courbe.}
\end{doc}


\begin{doc}{Vitesse et accéleration}{doc:TP1_vitesse}
  Vitesse \important{uniforme} (constante) : le système n’accélère pas.
  
  La vitesse augmente : \texteTrouLignes{le système accélère.}
  
  La vitesse diminue : \texteTrouLignes{le système décélère.}
  
  Si \texteTrou[0.5]{la vitesse est constante et nulle}, on dit que le système est \important{immobile}.
\end{doc}


%%%%

\numeroQuestion
Compléter les documents~\ref{doc:TP1_trajectoires} et~\ref{doc:TP1_vitesse}.

\fleche Pour la suite de cette activité, vous allez choisir entre l'étude du mouvement des oies ou de la Lune.
Vous présenterez ensuite les résultats de votre étude au reste de la classe à l'oral.

\fleche Vous rendrez ensuite une compte-rendu détaillée en suivant les questions sur le \textbf{mouvement que vous n'avez pas choisi.}
Il faudra donc être attentif à ce que disent vos camarades !


%%%%
\newpage
\titreSousSection{\'Etude du mouvement des oies}

Le compteur du bateau affiche une vitesse $v_\text{bateau} = \qty{3,6e1}{\km/\hour}$.

\vspace*{6pt}
\numeroQuestion Pour la personne qui filme les oies, quelle est la vitesse des oies ?

\numeroQuestion Pour une personne se trouvant sur la berge, quelle est la vitesse des oies ?

\numeroQuestion Schématiser la trajectoire des oies si on les observe depuis la berge.

\numeroQuestion Indiquer le mouvement des oies depuis le bateau et la berge.

%%
\titreSousSection{\'Etude du mouvement de la Lune}

La Lune tourne autour de la Terre à une vitesse $v_\text{Lune} = \qty{3,7e3}{\km/\hour}$
et la Terre tourne autour du Soleil à une vitesse $v_\text{Terre} = \qty{1,1e5}{\km/\hour}$.

\begin{figure}[!ht]
  \begin{subfigure}{0.48\linewidth}
    \centering
    \image{0.8}{images/mecanique/terre_lune.png}
    \caption{Point de vue centré sur la Terre}
    \label{fig:terre_lune}
  \end{subfigure}
  \begin{subfigure}{0.48\linewidth}
    \centering
    \image{0.8}{images/mecanique/terre_lune_soleil.png}
    \caption{Point de vue centré sur le Soleil}
    \label{fig:terre_lune_soleil}
  \end{subfigure}
\end{figure}

\vspace*{-6pt}
\numeroQuestion Depuis le point de vue centré sur la Terre, quelle est la vitesse de la Lune ?

\numeroQuestion Schématiser la trajectoire de la Lune depuis ce point de vue et indiquer son mouvement.

\numeroQuestion Peut-on décrire la vitesse de la Lune depuis le point de vue centré sur le Soleil ?

\numeroQuestion Schématiser la trajectoire de la Lune depuis ce point de vue.


%%%%
\titreSousSection{Notion de référentiel}

\question{
  Convertir la vitesse $v_\text{Lune}$ en \unit{\m/\s}.
  \textit{Rappel :} \qty{1}{\km} = \qty{e3}{\metre}, \qty{1}{\hour} = \qty{3,6e3}{\s}.
}{}{2}

\question{
  Quelle distance la Lune parcours pendant 1 seconde ?
  Comparer avec la longueur de sa trajectoire, qui est de \qty{2,4e6}{\km}.
}{}{1}

\question{
  Peut-on décrire la trajectoire de la Lune en l'observant pendant 1 seconde ?
}{}{2}

\question{
  Conclusion : pourquoi est-il important de définir le référentiel, qui est l’endroit où on se place et le temps passé à observer, avant d'étudier un mouvement ?
}{}{1}