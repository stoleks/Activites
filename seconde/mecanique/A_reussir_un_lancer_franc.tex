%%%% début de la page
\teteSndMouv

\nomPrenomClasse
\titreActivite{Réussir un lancer franc}

\begin{objectifs}
  \item Utiliser des outils numériques pour analyser un mouvement.
\end{objectifs}

%%%%
\begin{doc}{Lancer franc au basket}[\label{doc:lancer_franc}]
  \begin{center}
    \image{0.75}{images/mecanique/lancer_franc}
  \end{center}
\end{doc}


%%%%
\begin{doc}{Programme python pour tracer la trajectoire}[\label{doc:code_python}]
  \vspace*{-8pt}
  \lstset{style=codePython, language=python}
\begin{lstlisting}
# Pour avoir des modules de calculs et d'affichage
import numpy as np              # calcul
import matplotlib.pyplot as plt # affichage

# calcul de la trajectoire selon x et y
def positionX (v0, alpha, t) : # v0 t cos(alpha)
    return v0 * t * np.cos (alpha * np.pi / 180) 
def positionY (v, alpha, t) :  # -g t² /2 + v0 t sin(alpha)
    return -9.81/2*(t**2) + (v0 * t)*np.sin (alpha * np.pi / 180) 

# Vitesse et angle initial, nombre de points a calculer
v0 = float (input ("Valeur de la vitesse initiale v0 (m/s) = "))
alpha = float (input ("Valeur de l'angle de départ (°) = "))
points = 50

# calcul des positions x et y
x, y = [],[]              # initialisation des positions
deltaT = 2 / (points - 1) # temps entre 2 points
for i in range(points) :  # pour chaque position i
  x.append (positionX (v0, alpha, i * deltaT)) # calcul de l'abscisse en mètre
  y.append (positionY (v0, alpha, i * deltaT)) # calcul de l'ordonnée en mètre
  
# réglage du graphique
plt.xlabel (r"$x$ (en m)")            # légende de l'abscisse
plt.ylabel (r"$y$ (en m)")            # légende de l'ordonnee
plt.title ("Trajectoire de la balle") # titre du graphique
# tracé des points repérés et des vecteur vitesses
plt.plot (x, y, "o", markersize = 5) # affiche des points rond ('o'), taille 4
plt.xlim (0, 5.0)                    # limite du graphique selon x
plt.ylim (0, 1.6)                    # limite du graphique selon y
plt.xticks (np.arange(0, 5, 0.5))    # finesse de la grille selon x
plt.yticks (np.arange(0, 1.6, 0.2))  # finesse de la grille selon y
plt.grid ()                          # affichage de la grille
plt.show ()                          # affichage du graphique
\end{lstlisting}
  \vspace*{-8pt}
\end{doc}

\newpage
\vspace*{-40pt}
\begin{doc}{Exemple de trajectoire tracée par le programme}[\label{doc:exemple_trajectoire}]
  \begin{center}  
    \image{0.95}{images/mecanique/lancer_franc_trajectoire}
  \end{center}
\end{doc}

%%
\question{
  Dans le programme python du document~\ref{doc:code_python}, repérer les lignes de code permettant de calculer la position de la balle.
}{
  Ce sont les lignes 20 à 21, qui appellents les fonctions des lignes 6 à 9.
}[1]


\question{
  Dans le programme python du document~\ref{doc:code_python}, repérer les lignes de code permettant de choisir la valeur de la vitesse $v_0$ ainsi que la valeur de l’angle $\alpha$.
}{
  Ce sont les lignes 12 et 13.
}[1]

\question{
  D’après les documents~\ref{doc:lancer_franc} et~\ref{doc:code_python}, quelles sont les valeurs des coordonnées $(x ; y)$ que doit atteindre le ballon pour que le panier soit marqué ? Schématiser ce point sur le graphique du document~\ref{doc:exemple_trajectoire}.
}{
  $(\num{4,70} ; \num{0,62})$
}[1]

\question{
  Exécuter le fichier python pour tracer la trajectoire du ballon de Basket. Chercher des valeurs pour $v_0$ et $\alpha$ permettant de faire rentrer le ballon dans le panier.
}{
  $v_0 = 7,3$ et $\alpha = 45$ ou $v_0 = 7,6$ et $\alpha = 60$
}[1]


\begin{doc}{Vecteur vitesse}
  \begin{importants}
    Le \important{vecteur vitesse $\vv{v_2}$} d'un système au point $P_2$ entre les instants $t_1$ et $t_3$ a pour expression
    \begin{equation*}
      \vv{v_2} = \frac{\vv{P_1 P_3}}{t_3 - t_1} \qq{en} \unit{\m / \s}
    \end{equation*}
  \end{importants}
  
  Le vecteur vitesse $\vv{v_2}$ est caractérisé par :
  \begin{listePoints}[2]
    \item Une origine : $P_2$.
    \item une direction : parallèle au segment $P_1 P_3$ et tangent à la trajectoire.
    \item Un sens : le sens du mouvement.
    \item Une norme : $v_2 
    %= \norm{\vv{v_2}}
    = \displaystyle \norm{\frac{\vv{P_1 P_3}}{t_3 - t_1}}
    = \displaystyle \frac{P_1 P_3}{t_3 - t_1}$.
  \end{listePoints}
  
  Le \important{vecteur déplacement $\vv{P_1 P_3}$} est caractérisé par
  \begin{listePoints}[2]
    \item Une origine : le point $P_1$.
    \item une direction : celle de la droite $P_1 P_3$.
    \item Un sens : de $P_1$ vers $P_3$.
    \item Une norme : la distance $P_1 P_3$ en mètre \unit{m}.
  \end{listePoints}
\end{doc}

\mesure Sans se préoccuper de leurs longueurs, tracer les vecteurs vitesses $\vv{v_2}$, $\vv{v_6}$ et $\vv{v_{10}}$ sur le doc.~\ref{doc:exemple_trajectoire}.
