%%%% début de la page
\teteSndMouv


%%%% titre
\vspace*{-32pt}
\numeroActivite{2}
\titreActivite{Modéliser une action par une force}


%%%% objectifs
\begin{objectifs}
  \item Comprendre la notion de force
  \item Connaître des exemples de forces
\end{objectifs}

%%%%
\begin{doc}{Force et action mécanique}{doc:A2_action_force}
  \begin{importants}  
    Un corps exerce une \important{action mécanique} sur un système étudié \texteTrouLignes[1]{s’il est capable d’en modifier le mouvement.}
  \end{importants}
  
  Une action mécanique est modélisée par une \important{force.}

  \begin{importants}
    La force exercée par un corps $A$ sur un corps $B$ est représentée par un vecteur $\vvFAsurB$.
    Ce vecteur possède les caractéristiques suivantes :
    \begin{listePoints}
      \item Une \important{valeur} notée $\FAsurB$, qui s'exprime en newton noté \unit{\newton}.
      \item Une \important{direction} et un \important{sens} qui dépendent de la situation.
      \item Une \important{origine}, appelée \important{point d'application} : le centre du système $B$.
    \end{listePoints}
  \end{importants}
\end{doc}

\mesure
Une personne pousse un carton. 
Représenter la force $\vv{F}_\text{personne/carton}$ qu'exerce la personne sur le carton.

\vspace*{-8pt}
\begin{center}
  \image{0.23}{images/mecanique/personne_carton}
\end{center}


%%
\begin{doc}{Exemples de forces}{doc:A2_exemples_forces}
  On distingue 2 types d'actions :
  \begin{listePoints}
    \item les \important{actions de contact} (contact entre l’objet qui donne la force et l’objet qui la reçoit),
    \item les \important{actions à distance} (pas de contact).
  \end{listePoints}
  
  \begin{tblr}{
    colspec = {|c |c |X[c] |}, hlines,
    row{1} = { couleurPrim!20 },
  }
    Force & Valeur & Direction, sens \\
    %
    poids $\vv{P}$ &
    $P = m \times g$ &
    verticale, vers le bas \\
    %
    réaction du support $\vv{R}$ &
    égale au poids $R = P$ &
    perpendiculaire au support, vers le haut \\
    %
    frottements $\vv{f}$ &
    dépend du cas étudié &
    opposés à la vitesse $\vv{v}$ \\
  \end{tblr}
  \smallskip
  
  \begin{listePoints}
    \item $\vv{P}$ représente l'interaction gravitationnelle de la Terre.
    \item $\vv{R}$ représente l'action exercée par le support sur un objet posé dessus.
    \item $\vv{f}$ représentent l'action d'un milieu (gaz, liquide, support solide).
  \end{listePoints}
  \attention Si un objet est \important{immobile par rapport au milieu,} il n'y a pas de frottements.
\end{doc}

\vspace*{-16pt}
\question{
  Parmi les forces $\vv{P}$, $\vv{R}$ et $\vv{f}$, indiquer celles qui modélisent une action de contact et celles qui modélisent une action à distance.
}{}{3}


%%%%
\begin{center}
  \begin{tblr}{
    colspec = {|X[c,m] |X[c,m] |}, hlines,
    row{1,3} = {couleurPrim!20},
  }
    Ballon & Curling \\
    \image{0.8}{images/mecanique/ballon_football} &
    \image{0.8}{images/mecanique/curling} \\
    %
    Parachutiste & Skieuse \\
    \image{0.8}{images/mecanique/parachutiste} &
    \image{0.8}{images/mecanique/skieur} \\  
  \end{tblr}
\end{center}

%%
\mesure 
En vous aidant des documents~\ref{doc:A2_action_force} et~\ref{doc:A2_exemples_forces}, compléter le tableau :
\begin{listePoints}
  \item Schématiser la ou les forces entrant en jeu, en faisant attention à leurs points d'application.
  \item Tracer la somme de toutes les forces entrant en jeu.
\end{listePoints}