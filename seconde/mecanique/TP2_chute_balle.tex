%%%% début de la page
\teteSndMouv

%%
\nomPrenomClasse

%%%% titre
\numeroActivite{2}
\titreTP{Chute d'une balle}
\vspace*{-12pt}

%%%% objectifs
\begin{objectifs}
  \item Comprendre la notion de vecteur vitesse.
  \item Tracer des vecteurs vitesses.
\end{objectifs}

\begin{contexte}
  Quand on change de référentiel, la description du mouvement s'en trouve modifié.
  Par exemple, la trajectoire d'une balle qui tombe dans un référentiel mobile et totalement différente en fonction du référentiel choisi.

  \problematique{Quel est l'impact du référentiel choisi sur le mouvement ?}
\end{contexte}


%%%% evaluation
\pasCorrection{
\begin{tableauCompetences}
  REA & réaliser un protocole en suivant toutes les étapes.
\end{tableauCompetences}
}


%%%% documents
\begin{doc}{Chronophotographie de la chute d'une balle}{doc:TP2_chrono}  
  \qrcodeCote{https://www.fizziq.org/fizziqclassique}
  La superposition de plusieurs images prises les unes après les autres avec un intervalle de temps régulier est une \important{chronophotographie}.

  Pour étudier une chronophotographie, on va utiliser \important{FizziQ,} une application qui permet de mesurer des grandeurs avec un smartphone.
  Une fois dans l'application Fizziq, il faut cliquer sur \important{Analyse cinématique} dans le menu \important{Mesures}. Puis choisir \important{Chronophotographie} et ensuite \important{Chute libre}. 

  Une fois dans la chronophotographie, il faut 
  \begin{protocole}
    \item régler l'échelle des distances en utilisant la règle de référence de \qty{1}{\m} ;
    \item appuyer sur \important{Pointage} ;
    \item repérer la position de la balle en faisant glisser le curseur rouge dessus ;
    \item toucher l'écran pour enregistrer la position de la balle ;
    \item répéter l'opération pour chaque position de la balle ;
    \item appuyer sur \important{Résultats} ;
    \item choisir $T$(s), $y$(m) et $V_y$(m/s), pour avoir le temps, la position et la vitesse verticale.
  \end{protocole}
  On arrive alors dans le cahier de mesures, qui contient les mesures réalisées et permet de les tracer.
\end{doc}

\telechargement Télécharger l'application FizziQ en scannant le QR code du document~\ref{doc:TP2_chrono}.

\mesure Réaliser le protocole du document~\ref{doc:TP2_chrono}, puis tracer la position $y$ en fonction du temps $T$.

\question{
  Une fois dans le graphique, appuyer sur \important{modélisation}, est-ce que le modèle linéaire $y = a\, t + b$ passe près des points et décrit bien la position en fonction du temps ?
}{}[2]

\question{
  Appuyer de nouveau sur \important{modélisation,} est-ce que le modèle quadratique $y = a\, t^2 + b\, t + c$ décrit bien la position en fonction du temps ? 
  Noter la valeur du coefficient $a$.
}{}[2]


\pasCorrection{ \newpage \vspace*{-30pt} }
\question{
  Tracer maintenant la vitesse $V_y$ en fonction du temps $T$.
  Quel modèle décrit mieux la vitesse en fonction du temps ? 
  Noter la valeur du coefficient $a$.
}{}[2]

%%
\begin{doc}{Vecteur}{doc:TP2_vecteur}
  \begin{importants}
    \important{Vecteur} : objet mathématique représenté par un segment fléché \reaction et noté avec une lettre surmontée d'une flèche $\vv{v}$.    
    Un vecteur contient quatre information : 
    \begin{listePoints}[4]
      \item une \important{direction}
      \item un \important{sens}
      \item une \important{norme}
      \item une \important{origine}
    \end{listePoints}
    \vspace*{-4pt}
  
    Un vecteur est \important{constant} si sa direction, son sens et sa norme ne varie pas pendant le mouvement.
  \end{importants}
  On peut aussi représenter un vecteur avec des \important{coordonnées} dans un référentiel.
  C'est ce qui est fait dans Fizziq : comme les vecteurs sont en 2 dimensions, ils sont représentés par 2 nombres.
  \begin{listePoints}[2]
    \item La position $\vv{P} = (x, y)$.
    \item La vitesse $\vv{V} = (V_x, V_y)$.
  \end{listePoints}
  où $x$ représente l'axe horizontal et $y$ l'axe vertical.
\end{doc}


\begin{doc}{Mouvement de la chute d'une balle dans un référentiel en mouvement}{doc:TP2_chute_mouvement}
  Pour étudier la chute d'une balle en mouvement, on va réaliser une vidéo où quelqu'un tient une balle, court, puis lâche la balle en courant.
  
  Une fois la vidéo réalisée, on va l'analyser avec FizziQ, en cliquant sur \important{Analyse cinématique} dans le menu \important{Mesures}, puis sur \important{Cinématique par vidéo} et ensuite sur \important{Mes vidéos}.
  
  Finalement, il ne reste plus qu'à \important{répéter le protocole du document~\ref{doc:TP2_chrono},} pour repérer la position de la balle au cours du temps.
\end{doc}

\mesure Par groupe de 3, réaliser le protocole du document~\ref{doc:TP2_chute_mouvement}, puis sauvegarder le temps $T$, la position verticale $y$ et la vitesse verticale $v_y$.

\question{
  Tracer le graphique de $y(T)$. La trajectoire est-elle identique à celle de la chute libre ?
}{}[1]

\question{
  En utilisant l’outil de modélisation, quel est le meilleur modèle pour décrire la position $y$ ? Linéaire ou quadratique ? Noter la valeur du coefficient $a$.
}{}[2]

\question{
  Même question pour la vitesse $v_y$.
}{}[2]

\question{
  Comparer la chute libre et la chute en mouvement : est-ce que la position $y$ et la vitesse $v_y$ suivent des modèles différents ?
}{}[3]