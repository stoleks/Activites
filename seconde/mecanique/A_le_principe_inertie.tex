%%%% début de la page
\teteSndMouv

\vspace*{-36pt}
\titreActivite{Le principe d'inertie}

\vspace*{-16pt}
\begin{objectifs}
  \item Comprendre la notion d'inertie
  \item Comprendre le principe d'inertie.
\end{objectifs}


\begin{doc}{Inertie d'un corps}
  \begin{importants}
    \important{L'inertie} est la tendance qu'ont les corps à rester dans le même état (repos ou mouvement), en l'absence de forces appliquées.
  \end{importants}
    
  C'est la masse qui mesure l'inertie : plus un objet a une masse élevée et plus il a de l'inertie.
  Donc plus un objet est lourd, plus il faut exercer une force importante pour changer son mouvement.
  
  \exemple Faire rouler un caddie vide est facile, mais c'est plus difficile quand il est rempli !
\end{doc}


\begin{doc}{Forces qui se compensent}[\label{doc:compensation_forces}]
  \begin{importants} 
    On dit que des forces se compensent si leur somme est égale au vecteur nul $\vv{0}$.
  \end{importants}
  Pour que la somme de deux vecteurs soit nulles, il faut qu'ils aient même \important{direction,} même \important{valeur,} mais un \important{sens opposé.}
  Pour la somme de trois vecteurs, on commence par sommer deux vecteurs, puis on somme le vecteur obtenu avec le troisième restant.
  
  \centering
  \image{1}{images/mecanique/forces_systemes}
  \legende{Ballon de foot immobile, parachutiste qui tombe à une vitesse constante, cycliste qui freine, skieuse qui avance à une vitesse constante.}
\end{doc}

\newpage
\question{
  Pour quels systèmes du document~\ref{doc:compensation_forces} les forces se compensent-elles ?
}{
  Dans les quatres situations on a des forces qui se compensent, avec une somme des vecteurs nulle. 
}[2]

\question{
  Quel est le mouvement du système dans chaque cas où les forces se compensent ?
}{
  Soit le système est immobile, soit sa vitesse est rectiligne uniforme.

  Dit autrement, pour faire bouger un objet ou pour modifier sa trajectoire, il faut qu'il soit soumis à des forces.
}[2]



\begin{doc}{Le principe d'inertie et sa contraposée}
  \chevron Le \important{principe d'inertie} a été formulé pour la première fois par Newton en 1687.
  Newton s'appuyait sur les travaux de Descartes et de Galilée, et parfois on appelle ce principe la \important{première loi de Newton}.
  Sa formulation moderne est la suivante :
  
  \begin{importants}
    Si les forces qui s'exercent sur un système se compensent, alors ce système est \texteTrou(2){soit immobile, soit en mouvement rectiligne uniforme.}
  \end{importants}
  
  \begin{importants}
    Réciproquement, si un système est
    \texteTrou(3){immobile ou en mouvement rectiligne uniforme, alors les forces qui s'exercent sur lui se compensent.}
  \end{importants}
\end{doc}


%%%%
\question{
  Comment varie $\vv{v}$ pour un système qui a un mouvement rectiligne uniforme ?
  En déduire la variation de $\vv{v}$ pour un système soumis à des forces qui se compensent.
}{
  Le vecteur vitesse est constant pour un mouvement rectiligne uniforme.
  Donc le vecteur vitesse est constant si le système est soumis à des forces qui se compensent.
}[3]

\begin{doc}{Principe d'inertie et vitesse}
  \begin{importants}
    Le principe d'inertie dit que si le vecteur vitesse
    \texteTrou(2){
      est constant sur toute la trajectoire, alors les forces exercées sur le système se compensent.
    }
  \end{importants}
\end{doc}