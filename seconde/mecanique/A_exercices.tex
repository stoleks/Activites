%%%% début de la page
\teteSndMouv

\titreActivite{Vol d'oie et saut en parachute}

\begin{objectifs}
  \item Remobiliser les notions de référentiel, forces, vitesses
  \item Utiliser le principe d'inertie pour calculer des forces
\end{objectifs}


%%%%
\begin{doc}{Référentiel terrestre}{doc:referentiel_terrestre}
  \begin{importants}
    Sur Terre on utilise souvent le \important{référentiel terrestre} pour étudier des mouvements. Ce référentiel est lié à la surface de la Terre.
  \end{importants}
  C'est le référentiel auquel on fait spontanément référence quand on mesure une vitesse de déplacement.
\end{doc}


%%%%
\exercice{Vol d'une oie}

%%
\begin{doc}{Vol d'oie et portance}{doc:A6_vol_oie}
  \begin{center}
    \image{0.6}{images/mecanique/oie}
  \end{center}
  
  
  On considère que deux forces s'exercent sur une oie qui plane avec un mouvement rectiligne uniforme : son poids et la portance de l'air.
  L'étude se fait dans le référentiel terrestre et on néglige les forces de frottements ($\vv{f} \approx \vv{0})$.

  \important{Données :}
  \begin{listePoints}
    \item Masse de l'oie $m = \qty{400}{\g}$.
    \item Accélération de la pesanteur terrestre $g = \qty{9,81}{\newton \per\kg}$.
  \end{listePoints}
\end{doc}

\question{
  Les forces exercées sur l'oie se compensent-elles ? Justifier en utilisant son mouvement et le principe d'inertie.
}{
  Comme l'oie a un mouvement rectiligne uniforme, d'après le principe d'inertie, les forces qui se compensent sur elle se compensent.
}

\question{
  En déduire une relation entre les valeurs de ces deux forces.
}{
  
  Comme ces deux forces se compensent, elles doivent avoir la même valeur, donc $P = F_\text{air}$.
}

\question{
  Calculer la norme du poids P de l'oie.
}{
  \begin{equation*}
    P = m \times g = \qty{0,400}{\kg} \times \qty{9.81}{\newton\per\kg} = \qty{3,924}{\newton}
  \end{equation*}
}

\question{
  En déduire la norme de la force de portance $F_\text{air}$.
}{
  On a $F_\text{air} = P = \qty{3,924}{\newton}.$
}

\question{
  Représenter la situation sur un schéma, en modélisant l'oie par un point matériel et en représentant les forces qui s'exercent sur elle, sans souci d'échelle.
}{}


%%%%
\pasCorrection{\newpage}
\exercice{Saut en parachute}

%%
\begin{doc}{Freinage d'un parachute à l'ouverture}{doc:A6_vitesse_parachute}
  \begin{wrapfigure}{r}{0.45\linewidth}
    \vspace*{-24pt}
    \begin{center}
      \image{1}{images/donnees/norme_vitesse_parachute}
      \small{
        Vitesse du système en fonction du temps.
      }
    \end{center}
  \end{wrapfigure}
  
  Une parachutiste saute sans vitesse initiale d'un hélicoptère en vol stationnaire.
  Après quelques secondes en chute libre, elle ouvre son parachute.
  Les frottements dus à l'air sur la toile s'expriment par une force opposée au mouvement. 
  
  Dans ce cas la norme de cette force est proportionnelle au carré de la vitesse
  \begin{equation*}
    f = k \times v^2
  \end{equation*}
  avec $f$ la force de frottements, $k$ le coefficient de frottements et $v$ la vitesse du système.

  \important{Données :}
  \begin{listePoints}
    \item Masse du système (parachutiste + parachute) $m = \qty{90}{\kg}$.
  \end{listePoints}
  \vAligne{-34pt}
  
  \begin{listePoints}
    \item Accélération de la pesanteur terrestre $g = \qty{9,81}{\newton \per\kg}$.
  \end{listePoints}
\end{doc}

%%
\question{
  Décrire les trois phases du mouvement, la trajectoire étant tout le temps rectiligne.
}{
  On a un mouvement rectiligne accéléré entre 0 et 12 secondes, puis rectiligne décéléré entre 12 et 16 secondes, puis rectiligne uniforme de 16 à 25 secondes.
}

%%
\question{
  Que se passe-t-il à \qty{12}{\s} pour que la vitesse diminue aussi rapidement ?
}{
  Le parachute s'ouvre, ce qui augmente brusquement les frottements de l'air.
}

%%
\question{
  Lorsque le parachute est ouvert, $k = \qty{10}{\newton \s\squared \per\m\squared}$.
  Calculer l'intensité (la valeur) de la force de frottements à l'instant où la parachutiste ouvre son parachute.
}{
  \begin{equation*}
    f = k \times v^2
    = \qty{10}{\newton \s\squared \per \m\squared} \times (\qty{52}{\m\per\s})^2
    = \qty{27040}{\newton}
  \end{equation*}
}

%%
\question{
  En utilisant le principe d'inertie, expliquer le mouvement à partir de l'instant $t = \qty{16}{\s}$.
}{
  À partir de 16 secondes, les frottements de l'air compensent le poids et le mouvement devient rectiligne uniforme.
}

%%
% \question{
%   Calculer la valeur du coefficient de frottements $k$ à l'instant $t = \qty{20}{\s}$.
% }{
%   Comme les forces se compensent
%   \begin{align*} 
%     f &= P \\
%     k \times v^2 &= m \times g \\
%     k &= \dfrac{m \times g}{v^1} \\
%     k &= \dfrac{90 \times 9,81}{7^2} \unit{\newton \s\squared \per \m\squared} \\
%     k &= e
%   \end{align*}
% }



\begin{doc}{Vitesse de chute libre}{doc:A7_vitesse_chute_libre}
  Pour un objet tombant dans le vide sans vitesse initiale, sa vitesse au moment de toucher le sol vaut
  \begin{equation*}
    v = \sqrt{2\cdot g \cdot h}
    \qq{ou}
    h = \dfrac{v^2}{2 \cdot g}
  \end{equation*}
  où $g$ est l'accélération de pesanteur terrestre et $h$ la hauteur du point de chute.
\end{doc}

%%
\question{
  En utilisant la relation entre la hauteur $h$ et la vitesse $v$, calculer la hauteur de laquelle il faudrait tomber pour atteindre la vitesse du parachutiste à l'instant $t = \qty{20}{\s}$.
}{
  Pour $t = \qty{20}{\s}$, $v = \qty{7}{\m\per\s}$, donc $h = \dfrac{7^2}{2 \times 9,81} \unit{\m} = \qty{2,5}{\m}$.
}

\question{
  En utilisant la même relation entre la hauteur $h$ et la vitesse $v$, calculer la hauteur de laquelle il faudrait tomber pour atteindre la vitesse du parachutiste à l'instant $t = \qty{12}{\s}$.
  Conclure sur l'intérêt du parachute.
}{
  Pour $t = \qty{12}{\s}$, $v = \qty{52}{\m\per\s}$, donc $h = \dfrac{52^2}{2 \times 9,81} \unit{\m} = \qty{137,8}{\m}$.

  Donc avec le parachute ouvert c'est « comme si » on tombait d'un escabeau, alors que sans le parachute c'est comme si on tombait d'un gratte-ciel.
}