%%%%
\teteSndSign
\numeroActivite{2}

%%%% titre
\titreTP{Caractéristique d'un dipôle et loi d'Ohm}


%%%% Objectifs
\begin{objectifs}
  \item Revoir quelques notions de bases sur les circuits électriques
  \item Trouver la loi d'Ohm
\end{objectifs}


%%%% docs
\begin{doc}{Circuit électrique}
    Un circuit électrique est composé d'au moins un générateur, un récepteur (résistance, moteur, DEL, etc.) et de fils de connexion.

  \begin{importants}
    Un \important{dipôle} est un élément d'un circuit électrique possédant deux bornes.
  \end{importants}

  \begin{importants}
    Un \important{n\oe{}ud} est une connexion qui relie au moins trois dipôles entre eux.
  \end{importants}

  \begin{importants}
    Une \important{maille} est un chemin fermé, ne comportant pas forcément de générateur.
  \end{importants}    
\end{doc}

\begin{doc}{Tracé de la caractéristique d'un dipôle}{doc:circuit_loi_ohm}
  \begin{center}
  \begin{circuitikz}
    \ctikzset{bipoles/vsourceam/inner plus={\tiny $+$}}
    \ctikzset{bipoles/vsourceam/inner minus={\tiny $-$}}
    \draw (4, -2)
      to [short, i=$I_R$] (0, -2)
      to [rmeterwa, t=G, i=$I_R$] (0, 2)
      to [rmeter, t=A, i=$I_R$] (4, 2)
      to [R, l={$R$}, -*, i=$I_R$] (4, -2) -- (6, -2)
      to [rmeter, t=V] (6, 2)
      to [short, -*] (4, 2);
  \end{circuitikz}
  \end{center}
  Ce circuit électrique permet de mesurer la caractéristique d'un dipôle, ici une résistance.
\end{doc}

%%
\mesure 
Réaliser le montage électrique du document~\ref{doc:circuit_loi_ohm}, avec une résistance $R = \ldots\ldots$
Faire vérifier le circuit.

\question{
  Combien de n\oe{}uds, mailles et dipôles comporte le circuit du document~\ref{doc:circuit_loi_ohm} ?
}{
  4 dipôles, 2 noeuds et 2 mailles.
}{1}

\mesure
Mesurer la caractéristique de la résistance :
\begin{listePoints}
  \item faire varier la tension $U$ aux bornes du générateur entre 0 et 10 V ;
  \item mesurer la valeur de l'intensité $I_R$ qui traverse la résistance pour chaque tension ;
  \item noter chaque couple de valeur $(I_R, U)$ dans le tableau suivant :
\end{listePoints}

\begin{tblr}{| X[0.75, c] | X[0.75, c] | X[0.75, c] | X[0.75, c] | X[0.75, c] | X[0.75, c] | X[0.75, c] | X[0.75, c] |}
  \hline
  $U$ (V)    & & & & & & \\ \hline
  %
  $I_R$ (mA) & & & & & & \\ \hline
\end{tblr}


%%
\begin{doc}{Point maths}
 Pour tracer la représentation graphique de $U = f(I)$, il faut mettre $U$ en ordonnée et $I$ en abscisse.

 $U$ et $I$ sont proportionnels si la représentation graphique de $U = f(I)$ est une droite.

  Le coefficient directeur d'une droite $(AB)$ non parallèle à l'axe des ordonnées est égal à $\dfrac{y_B - y_A}{x_B - x_A}$.
\end{doc}

\question{
  Tracer $U = f(I_R)$ à partir de vos mesures.
  Les grandeurs $U$ et $I_R$ sont-elles proportionnelles ?
}{

}{2}

\question{
  Mesurer le coefficient de proportionnalité $k$ reliant $U$ et $I_R$, tel que $U = k \times I_R$.
  En comparant $k$ et la valeur de la résistance $R$, que remarquez-vous ?
}{

}{3}


%%
\begin{doc}{Loi d'Ohm}{doc:loi_Ohm}
  \begin{importants}
    La loi d'Ohm relie la tension $U_R$ aux bornes d'un résistor de résistance $R$ et l'intensité du courant $I_R$ qui le traverse.

    Son expression est :
    \begin{equation*}
      \ldots\ldots\ldots
    \end{equation*}
  \end{importants}
  %
  \begin{center}
    \begin{circuitikz}
      \draw (0, 0) to [R, l={$R$}, i=$I_R$, v=$U_R$] (3, 0);
    \end{circuitikz}
    
    {\small Schéma d'une résistance avec la tension à ses bornes et l'intensité qui la traverse}
  \end{center}
\end{doc}