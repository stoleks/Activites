\teteSndMeth
\vspace*{-30pt}
\titreActivite{Ordre de grandeur}

\begin{objectifs}
  \item Revoir la notation scientifique et le système international des unités.
  \item Découvrir les raisonnements rapides avec des ordres de grandeur.
\end{objectifs}

\begin{contexte}
  En physique on utilise des grandeurs avec des tailles très variables, c'est pour ça qu'on utilise la \important{notation scientifique.}
  En physique et dans la vie de tous les jours, on peut être amené à essayer de raisonner rapidement sur des grandeurs mal connues. Pour ça on peut utiliser les \important{ordres de grandeur.}
  Enfin, pour comparer et communiquer sur des grandeurs, il faut pouvoir les mesurer avec des unités communes, c'est l'intérêt du \important{système international de mesure.}
\end{contexte}


%%%%
\vspace*{-12pt}
\titreSection{Notation scientifique}

%%
\begin{doc}{Les puissances de 10}
  \begin{importants}
  \begin{listePoints}
    \item Écrire le nombre $10^n$ (avec $n = 0, 1, 2, 3, \ldots$), revient à écrire ``$1$'' suivi de $n = 0, 1, 2, 3, \ldots$ zéros. \textit{Exemple : $10^3 = 1000$}
    \item Écrire le nombre $10^{-n}$ (avec $n = 1, 2, 3, \ldots$), revient à écrire ``$0,$'' suivi de $n - 1 = 0, 1, 2, \ldots$ zéros et d'un $1$. \textit{Exemple : $10^{-2} = 0,\!01$}
    \item $10^a \times 10^b = 10^{a + b}$
    \item $\dfrac{1}{10^n} 
    = \dfrac{10^{-n}}{10^{-n}} \times \dfrac{1}{10^n} 
    = \dfrac{10^{-n}}{10^{n - n}}
    = \dfrac{10^{-n}}{10^0}
    = 10^{-n}$
  \end{listePoints}
  \end{importants}
\end{doc}

\begin{doc}{La notation scientifique}
  \begin{importants}
  La \important{notation scientifique} d'une quantité se présente de la façon suivante :
  % Textes
  \begin{center}
    \texteEncadre{chiffre différent de zéro}
    \qq{}
    \texteEncadre{autres chiffres} 
    \qq{}
    \texteEncadre{puissance de dix}
    \texteEncadre{\important{unité}}
  \end{center}
  % Virgule et multiplication
  \vspace*{-30pt} \hspace*{4pt}
  \begin{tikzpicture}  
    \draw [white] (0,0) circle;
    \draw [couleurSec, thick] (5.5,-0.25) circle [radius=0.3] node {$,$};
    \draw [couleurSec, thick] (9.65,0) circle [radius=0.3] node {$\times$};
  \end{tikzpicture}
  \end{importants}
\end{doc}

\numeroQuestion Écrire les quantités suivantes en notation scientifique :
  
\separationBlocs{
  \qty{288}{\hour}    = \texteTrou(0){\qty{2,88e2}{\hour}}
  \correction{\newline}
  \qty{756864000}{\s} = \texteTrou(0){\qty{7,56864}{\s}}
}{
  \qty{638}{\newton}   = \texteTrou(0){\qty{6,38}{\newton}}
  \correction{\newline}
  \qty{0,9997}{\g/\ml} = \texteTrou(0){\qty{9,997e-1}{\g/\ml}}
}


%%
\titreSection{Les ordres de grandeurs}

\begin{doc}{Définition d'un ordre de grandeur}
  \qrcodeCote[1]{https://edurl.fr/XbmNG2ww}
  %https://www.youtube.com/watch?v=xTV47tuv_Fg}

  \vAligne{-36pt}
  \begin{importants}
    L'ordre de grandeur d'une quantité est la puissance de 10 la plus proche de cette quantité.
  \end{importants}
  %
  \exemple L'ordre de grandeur de \qty{60}{\s} est \qty{e2}{\s} (60 est plus proche de 100 que de 10). 
\end{doc}


\newpage
\vspace*{-28pt}
\numeroQuestion Donner l'ordre de grandeur des quantités suivantes :

\separationBlocs{
  \qty{3,00e8}{\m\per\s} = \texteTrou(0){\qty{e8}{\m\per\s}}
  \correction{\newline}
  \qty{1,67e-27}{\kg}    = \texteTrou(0){\qty{e-27}{\kg}}
}{
  \qty{9,11e-31}{\kg} = \texteTrou(0){$\num{e1} \times \qty{e-31}{\kg} = \qty{e-30}{\kg}$}
  \correction{\newline}
  \qty{53e-12}{\m}    = \texteTrou(0){$\num{e2} \times \qty{e-12}{\m} = \qty{e-10}{\m}$}
}


%%%%
\titreSection{Le système international de mesure}

%%
\vspace*{-12pt}
\titreSousSection{Le système international}

Pour comparer des grandeurs entre elles, il faut les exprimer avec les \important{mêmes unités de mesures}. % exemple centime et euros

Pour pouvoir communiquer facilement d'un pays à un autre, le \important{système international (SI)} a été développé par la Conférence Générale des Poids et Mesures (CGPM). % histoire des sciences système métrique

Le système international est composé de \important{sept unités de base,} que l'on retrouve quotidiennement. Une part importante de nos technologies modernes dépendent de la précision avec laquelle ces unités sont définies.

\begin{center}
  \begin{tblr}{
    hlines, row{1} = {couleurSec-100}, colspec = {|c |c |c |}
  }
    Grandeur             & Unité      & Symbole de l'unité \\
    Masse                & kilogramme & \unit{\kg} \\
    Temps                & seconde    & \unit{\s} \\
    Longueur             & mètre      & \unit{\m} \\
    Température          & kelvin     & \unit{\kelvin} \\
    Quantité de matière  & mole       & \unit{\mole} \\
    Intensité électrique & ampère     & \unit{\ampere} \\
    Intensité lumineuse  & candela    & \unit{\candela}
  \end{tblr}
\end{center}


%%
\titreSousSection{De l’échelle microscopique à l’échelle astronomique}

\numeroQuestion
Compléter le tableau en associant à chaque objet sa longueur, puis l'ordre de grandeur de cette longueur. Pour ça, utilisez ces six longueurs (attention aux unités !) :
%
\begin{center}
  \begin{tblr}{c}
    \qty{e20}{\m} &
    \qty{0,1}{\nm} &
    \qty{60}{\micro\m} &
    \qty{6}{\mm} &
    \qty{1000}{\km} &
    \qty{e10}{\m}
  \end{tblr}
\end{center}

\begin{tblr}{
  colspec = {X[-1] X[1] X[1] X[1] X[1] X[1] X[1]},
  vlines, hlines, row{3, 4} = {48pt},
  columns = {c}, row{1} = {couleurSec-100, m}, 
  width = \linewidth,
}
  Objet &
  Épaisseur cheveux &
  Voie Lactée &
  Système solaire &
  Hexagone &
  Fourmi &
  Atome \\
  % 
  Image & 
  \image{1}{images/exterieures/taille_cheveux} &
  \image{1}{images/exterieures/taille_galaxie} &
  \image{1}{images/exterieures/taille_systeme_solaire} &
  \image{1}{images/exterieures/taille_france} &
  \image{1}{images/exterieures/taille_fourmi} &
  \image{1}{images/exterieures/taille_atome} \\
  % 
  Taille & 
  \correction{\qty{60}{\micro\m}} &
  \correction{\qty{e20}{\m}} &
  \correction{\qty{e10}{\m}} &
  \correction{\qty{1000}{\km}} &
  \correction{\qty{6}{\mm}} &
  \correction{\qty{0,1}{\nm}} \\
  %
  Ordre de grandeur en mètre &
  \correction{\qty{e-5}{\m}} &
  \correction{\qty{e20}{\m}} &
  \correction{\qty{e10}{\m}} &
  \correction{\qty{e6}{\m}} &
  \correction{\qty{e-2}{\m}} &
  \correction{\qty{e-10}{\m}} \\
\end{tblr}
