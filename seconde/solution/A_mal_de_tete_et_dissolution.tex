%%%% début de la page
\teteSndSolu

%%%% titre
\vspace*{-36pt}
\titreActivite{Mal de tête et dissolution}

%%%% objectifs
\begin{objectifs}
  \item Calculer une concentration massique.
\end{objectifs}


%%%% contexte
\begin{contexte}
  Inès, 8 ans, a mal à la tête et son père décide de lui donner du paracétamol pour la soulager, sauf qu'il ne possède que des comprimés pour adulte !

  \problematique{Comment le père va-t-il calculer la bonne dose à administrer à sa fille ?}
\end{contexte}


%%%% documents
\begin{doc}{Solution, solvant et soluté}{doc:A1_solution}
  \begin{importants}
    Une \important{solution} est un mélange homogène.
    Le \important{solvant} est le composant majoritaire du mélange.
    Les \important{solutés} sont les espèces qui sont dispersées par le solvant.
    \begin{center}
        \important{solvant + solutés = solution}
    \end{center}
  \end{importants}
\end{doc}

\begin{doc}{Le paracétamol}{doc:A1_paracétamol}
  \begin{wrapfigure}[5]{r}{0.3\linewidth}
    \vspace*{-24pt}
    \centering
    \chemname{\chemfig{!\paracetamol}}{paracétamol}
  \end{wrapfigure}
  
  Le paracétamol est un antidouleur qui peut être dangereux pour le foie s'il est consommé en trop grande quantité.
  Un comprimé pour adulte a une masse $m_1 = \qty{500}{\milli\g}$, alors qu'un comprimé pour enfant a une masse $m_2 = \qty{300}{\milli\g}$.
  
  Pour calmer le mal de tête d'Inès, le père décide qu'il va \important{dissoudre} un comprimé de paracétamol pour adulte dans un verre d'eau de volume $V_1 = \qty{25}{\centi\litre}$.
\end{doc}


\question{
  Donner le solvant et les solutés de la solution préparée par le père.
}{
  Le solvant de la solution est l'eau, le soluté est le paracétamol.
}[2]


\begin{doc}{Concentration massique}{doc:A1_concentration_massique}
  \begin{importants}
    La \important{concentration massique $\mathbf{c}$} mesure la quantité de soluté présent dans une solution.
    C'est le rapport de la masse de \important{soluté} dissous sur le volume total de la \important{solution}
    \begin{equation*}
      c = \frac{m_\text{soluté}}{V_\text{solution}}
    \end{equation*}
  \end{importants}
\end{doc}

\question{
  Convertir le volume $V_1$ de la solution en millilitre, noté \unit{\ml}.
}{
  \begin{equation*}  
    V_1 = \qty{25}{\centi\litre} = \qty{250}{\ml}
  \end{equation*}
}[1]

\question{
  Calculer la concentration $c$ en \unit{\mg/\ml} de paracétamol dans le verre d'eau.
}{
  \begin{equation*}
    c = \dfrac{m_1}{V_1}
      = \dfrac{\qty{500}{\mg}} {\qty{250}{\ml}}
      = \qty{2,0}{\mg/\ml}
  \end{equation*}
}[4]

\newpage
\question{
  Quel volume $V_2$ de la solution (du verre d'eau) Inès doit-elle boire pour avaler $m_2 = \qty{300}{\milli\g}$ de paracétamol ?
}{
  \begin{equation*}
    V_2 = \dfrac{m_2}{c}
        = \dfrac{\qty{300}{\mg}} {\qty{2,0}{\mg/\ml}}
        = \qty{150}{\ml}
        = \qty{15,0}{\centi\litre}
  \end{equation*}
}[6]
