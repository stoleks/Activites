%%%% début de la page
\teteSndSolu

%%%% titre
\vspace*{-36pt}
\titreTP{Dosage du sucre par étalonnage}

%%%% objectifs
\begin{objectifs}
  \item Apprendre le vocabulaire sur les solutions.
  \item Comprendre la notion de concentration massique
  \item Comprendre le principe de la dilution et de la dissolution
\end{objectifs}


%%%% contexte
\begin{contexte}
  Le sucre couramment présent dans notre alimentation est le saccharose.
  Cette espèce chimique peut entraîner des risques pour la santé si on en consomme trop.
  Il est donc important de pouvoir déterminer la quantité de sucre consommée par jour.

  \problematique{Comment déterminer la masse de saccharose présent dans un sirop ?}
\end{contexte}


%%%% documents
\begin{doc}{Solution, solvant et soluté}{doc:TP1_solution}
  \begin{importants}
    \chevron Une \important{solution} est un mélange homogène. \\
    Le \important{solvant} est le composant majoritaire du mélange.
    Les \important{solutés} sont les espèces qui sont dispersées dans le solvant.
  \end{importants}
  
  \begin{center}
    \important{Solvant + Soluté(s) = Solution}
  \end{center}
  
  \begin{importants}
    On parle de \important{solution aqueuse} si le solvant est l'eau \chemfig{H_2 O}.
  \end{importants}
\end{doc}

\begin{doc}{Composition d'un sirop}{doc:TP1_sirop}
  Le constructeur annonce que le sirop est composé d'\textit{eau}, de \textit{sucre} de \textit{jus de citron} et d'\textit{acide citrique} principalement.
\end{doc}


\question{
  Donner le solvant et les solutés présents dans le sirop.
}{
  Le solvant du sirop est l'eau, les solutés sont le sucre, le jus de citron et l'acide citriques.
}[2]


%%%%
\begin{doc}{Concentration en soluté}{doc:TP1_concentration}
  \begin{importants}
    La \important{concentration massique $\mathbf{c}$} mesure la quantité de soluté présent dans une solution.
    C'est le rapport de la masse $m$ de \important{soluté} dissous dans le volume $V$ de la \important{solution}
    \begin{equation*}
      c = \frac{m_\text{soluté}}{V_\text{solution}}
    \end{equation*} 
  \end{importants}
  % \attention Il faut bien distinguer \important{concentration massique} et \important{masse volumique}.
  % La concentration mesure la masse de soluté contenue dans une solution.
  % La masse volumique mesure la masse d'un échantillon contenue dans un volume donné.
\end{doc}


\begin{doc}{Dissolution du sucre dans l'eau}{doc:TP1_protocole_dissolution}
  \begin{protocole}
      \item Peser une masse donnée de sucre avec une balance de précision.
      \item Mettre le sucre dans une fiole jaugée de 50 mL.
      \item Compléter la fiole jaugée jusqu'à mi-hauteur avec de l'eau distillée, agiter.
      \item Compléter jusqu'au trait de jauge avec de l'eau distillée.
      \item Verser le mélange dans un bêcher de 100 mL.
  \end{protocole}
\end{doc}


%%%% questions
\newpage
\vspace*{-28pt}

\mesure
En utilisant le Document~\ref{doc:TP1_protocole_dissolution}, préparer un mélange de \qty{50}{\ml} d'eau et de de sucre.

\mesure
Mesurer et noter la masse volumique du mélange préparé $\rho =$ \texteTrou[0.1]{\qty{0,15}{\g/\ml}}

\question{
  Calculer la concentration massique de sucre dans la solution aqueuse préparé.
}{
  Avec une masse de sucre de $\qty{10}{\g}$, on a une concentration massique 
  \begin{equation*}
    c = \dfrac{\qty{10}{\g}} {\qty{50}{\ml}}
    = \qty{0,2}{\g/\ml}
  \end{equation*}
}[1]


%%%%
\begin{doc}{Mesure de concentration}{doc:TP1_dosage}
  \begin{importants}
    On parle de \important{dosage} quand on mesure la concentration d'une espèce chimique présente dans une solution.
  \end{importants}
  \begin{importants}
    Un \important{dosage par étalonnage} consiste à déterminer la concentration d’une espèce chimique en comparant une grandeur physique caractéristique de la solution, à la même grandeur physique mesurée pour des solutions étalon.
  \end{importants}
\end{doc}
 
\numeroQuestion 
En utilisant le papier millimétré, tracer la masse volumique en fonction de la concentration massique de sucre dans l'eau.

\numeroQuestion
En déduire la concentration massique de sucre dans la sirop $c_\text{sirop} =$ \texteTrou[0.1]{\qty{0,6}{\g/\ml}}


%%%%
\begin{doc}{Principe d'une dilution}{doc:TP1_principe_dilution}
  \begin{wrapfigure}[5]{r}{0.5\linewidth}
    \vspace*{-16pt}
    \centering
    \begin{multicols}{4}
    \image{1.1}{images/chimie/protocoles/dissoDilu0007} \\[0pt]
    \footnotesize{$S_0$}
    
    \image{1.1}{images/chimie/protocoles/dissoDilu0008}
    
    \image{1.1}{images/chimie/protocoles/dissoDilu0010}
    
    \image{1.1}{images/chimie/protocoles/dissoDilu0011} \\[0pt]
    \footnotesize{$S_1$}
    \end{multicols}
  \end{wrapfigure}
  \vAligne{-40pt}
  
  \begin{importants}
    Le principe de la \important{dilution} est de \important{diminuer la concentration} en soluté dans une solution en rajoutant du \important{solvant.}
  \end{importants}
  La solution de départ est appelée \important{solution mère}, notée $S_0$.
  La solution obtenue après dilution est appelée \important{solution fille}, notée $S_1$.

  Pour diluer une solution, il faut
  \begin{protocole}
    \item Prélever un volume $V_0$ de la solution à l'aide de la pipette graduée.
    Le bas du ménisque doit atteindre la graduation supérieure.
    \item Introduire la solution prélevée dans la fiole jaugée de volume $V_1$.
    \item Ajouter de l'eau distillée dans la fiole jaugée jusqu'aux $2/3$ et agiter doucement. Compléter jusqu'à ce que le bas du ménisque atteigne le trait de jauge.
    \item Fermer la fiole et l'agiter en la retournant plusieurs fois.
    \item Verser la solution fille obtenue dans un bécher.
  \end{protocole}
\end{doc}


%%%%
\begin{doc}{Facteur de dilution}{doc:TP1_dilution}  
  Le \important{facteur de dilution} est le rapport du volume de la solution fille sur le volume de la solution mère
  \begin{equation*}
    F = \frac{V_\text{1}}{V_\text{0}}
  \end{equation*}
  On dit qu'on a dilué $F$ fois une solution.
\end{doc}

\mesure 
Diluer \important{2 fois} le sirop et mesurer sa masse volumique. 

\question{
  En déduire la concentration massique en sucre.
  Que constatez-vous ?
}{
  Pour diluer 2 fois, il faut que $F = 2 = \dfrac{V_\text{1}}{V_\text{0}}$, on aura donc un volume final $V_\text{1} = 2\times V_\text{0}$ deux fois plus grand que le volume initial, avec donc une concentration massique 2 fois plus faible.

  On constate que la concentration massique a été divisée par le facteur de dilution.
}
