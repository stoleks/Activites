%%%% début de la page
\teteSndSolu


%%%% titre
\vspace*{-36pt}
\numeroActivite{2}
\titreTP{Dosage d'un antiseptique}


%%%% objectifs
\begin{objectifs}
  \item Comprendre la notion de concentration massique.
  \item Doser la quantité de permanganate de potassium présente dans du Dakin.
\end{objectifs}


%%%% contexte
\begin{contexte}
  Le Dakin est une solution antiseptique qui sert à nettoyer des plaies. Le principe actif du Dakin est stabilisé par l'ajout de permanganate de potassium \chemfig{KMnO_4}.
  Le permanganate de potassium donne une teinte violette au Dakin.
  
  \problematique{Comment mesurer la concentration en \chemfig{KMnO_4} dans le Dakin ?}
\end{contexte}


%%%%
\begin{doc}{Concentration en soluté}{doc:TP2_concentration}
  \begin{importants}
    La \important{concentration massique $\mathbf{c}$} mesure la quantité de soluté présent dans une solution.
    C'est le rapport de la masse $m$ de \important{soluté} dissous dans le volume $V$ de la \important{solution}
    \begin{equation*}
      c = \frac{m_\text{soluté}}{V_\text{solution}}
    \end{equation*}
  \end{importants}
\end{doc}


%%%%
\begin{doc}{Dakin}{doc:TP2_dakin}
  Le Dakin est une solution aqueuse d'hypochlorite de sodium \chemfig{Na ClO}.
  Du permanganate de potassium \chemfig{K MnO_4} est ajouté à la solution, pour qu'elle ne soit pas dégradée par l'exposition au rayonnement UV du Soleil.
  
  \fleche Sur une bouteille de Dakin il est indiqué que la concentration de \chemfig{KMnO_4} vaut $\approx \qty{0,01}{\g/\litre}$.
\end{doc}

%
\question{
  Donner le solvant et les solutés de la solution de Dakin.
}{
  Le solvant est l'eau, les solutés sont le permanganate de potassium et l'hypochlorite de sodium.
}{2}


%%%%
\begin{doc}{Mesure de concentration d'une solution colorée}{doc:TP2_dosage}  
  \begin{importants}
    Une \important{échelle de teinte} permet de mesurer la concentration d'un soluté coloré.
  \end{importants}

  La teinte d'une solution est proportionnelle à la concentration en soluté.
  On prépare une série de solutions \important{étalons} dont on connaît la concentration et on compare leur teinte avec la solution dont on veut mesurer la concentration.
  
  \attention Il faut comparer les teintes avec des verreries identiques, la teinte s'assombrit avec l'épaisseur.
\end{doc}


%%%%
\begin{doc}{Protocole d'une dilution}{doc:TP2_protocole_dilution}
  \begin{wrapfigure}[5]{r}{0.5\linewidth}
    \vspace*{-20pt}
    \centering
    \begin{multicols}{4}
    \image{1.1}{images/chimie/protocoles/dissoDilu0007} \\[0pt]
    \footnotesize{$S_0$}
    
    \image{1.1}{images/chimie/protocoles/dissoDilu0008}
    
    \image{1.1}{images/chimie/protocoles/dissoDilu0010}
    
    \image{1.1}{images/chimie/protocoles/dissoDilu0011} \\[0pt]
    \footnotesize{$S_1$}
    \end{multicols}
  \end{wrapfigure}
  \vAligne{-40pt}
  
  \begin{importants}
    La \important{dilution} est la \important{diminution de la concentration} en soluté d'une solution en rajoutant du \important{solvant.}
  \end{importants}
  La solution de départ est appelée \important{solution mère}, notée $S_0$.
  La solution obtenue après dilution est appelée \important{solution fille}, notée $S_1$.

  Pour diluer une solution, il faut
  \begin{protocole}
    \item Prélever un volume $V_0$ de la solution à l'aide d'une pipette graduée.
    \important{Le bas du ménisque} doit atteindre la graduation supérieure.
    \item Introduire la solution prélevée dans la fiole jaugée de volume $V_1$.
    \item Ajouter de l'eau distillée dans la fiole jaugée jusqu'aux $2/3$ et agiter doucement.
    Compléter jusqu'à ce que \important{le bas du ménisque} atteigne le trait de jauge.
    \item Fermer la fiole et l'agiter en la retournant plusieurs fois.
    \item Verser la solution fille obtenue dans un bécher.
  \end{protocole}
\end{doc}

%%%%
\begin{doc}{Facteur de dilution}{doc:TP1_dilution}  
  Le \important{facteur de dilution} est le rapport du volume de la solution fille sur le volume de la solution mère et il est égal au rapport des concentrations des solutions mère et fille.
  \begin{equation*}
    F = \dfrac{V_1}{V_0} = \dfrac{c_0}{c_1}
  \end{equation*}
\end{doc}


%
\question{
  On souhaite réaliser une échelle de teinte composée de 4 solutions étalon pour mesurer la concentration de permanganate de potassium dans le Dakin.

  \begin{center}
    \begin{tblr}{c | X[1,c] | X[1,c] | X[1,c] | X[1,c]}
      Solution étalon & 1 & 2 & 3 & 4 \\ \hline
      Concentration (\unit{\g/\litre}) & \correction{\num{0,05}} & \correction{\num{0,025}} & \correction{\num{0,0125}} & \correction{\num{0,0063}} 
    \end{tblr}
  \end{center}
  
  Calculer le facteur de dilution entre les différentes solutions.
}{
  On divise par deux la concentration pour passer de la solution 1 à la solution 2, de la 2 à la 3 et de la solution 3 à la solution 4.
  Donc le facteur de dilution est $F = 2$.
}{1}

%
\question{
  Justifier l’intervalle des concentrations proposées pour l’échelle de teinte, à partir de la valeur attendue de la concentration en permanganate de potassium.
}{
  La valeur attendue de la concentration ($c = \qty{0,01}{\g/\litre}$) se trouve bien dans l'intervalle proposé.
}{1}

%
\question{
  Sachant que le volume de la fiole jaugée est $V_1 = \qty{50}{\ml}$, donner le volume de la solution mère $V_0$ à prélever pour avoir un facteur de dilution $F = 2$.
}{
  On doit avoir un volume deux fois plus faible, soit $V_0 = \qty{25}{\ml}$.
}{2}

%
\mesure
Réaliser l'échelle de teinte en effectuant trois dilutions successives.
Verser quelques millilitres de chaque solutions dans des tubes à essais.

%
\mesure
Utiliser l'échelle de teinte pour encadrer la valeur de la concentration en permanganate de potassium dans le Dakin.
Est-elle cohérente avec celle du constructeur ?
\pasCorrection{\lignesDeReponse{2}}
\correction{Oui, on trouve une concentration $\qty{0,0125}{\g/\litre} < c < \qty{0,0063}{\g/\litre}$.}

%
\question{
  Proposer une autre échelle de teinte pour améliorer la précision de la mesure (donner une liste de concentration).
}{
  On pourrait utiliser une échelle de teinte avec les concentrations suivantes : 0.015, 0.012, 0.0094, 0.0075, 0.006 \unit{\g/\litre} ($F = 1.25$).
}{1}
