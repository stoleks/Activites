%%%% début de la page
\teteSndSolu

%%%% titre
\vspace*{-36pt}
\titreActivite{Hémoglobine et anémie}

%%%% objectifs
\begin{objectifs}
  \item Mesurer une concentration massique à l'aide d'une échelle de teinte.
\end{objectifs}


%%%% contexte
\begin{contexte}
  Pour assurer son bon fonctionnement, l'organisme d'un être humain a besoin de fer \chemfig{Fe}.
  On dit qu'une personne souffre d'anémie si la concentration massique en fer dans le sang est trop faible.
  Le fer est transporté par une molécule dans le sang : l'hémoglobine.

  \problematique{Comment vérifier qu'une personne ne souffre pas d'anémie ?}
\end{contexte}


%%%% documents
\begin{doc}{Concentration en hémoglobine}{doc:A2_anemie}
  Mesurer la concentration massique en hémoglobine dans le sang permet de détecter les cas d'anémies.
  On parle d'anémie si cette concentration massiques est inférieure a
  \qty{1,2}{\g\per\litre} pour une femme et \qty{1,3}{\g\per\litre} pour un homme.
  Pour mesurer cette concentration, on peut réaliser une échelle de teinte, car c'est l'hémoglobine qui donne sa teinte rouge au sang.

  \separationBlocs{
    \begin{center}
      \begin{tblr}{c| c| c| c| c| c}
        Solution &
        \tubeEssaisSolution{red-600} &
        \tubeEssaisSolution{red-400} &
        \tubeEssaisSolution{red-200} &
        \tubeEssaisSolution{red-100} &
        \tubeEssaisSolution{red-50}  \\
        %
        & 1 & 2 & 3 & 4 & 5 \\ \hline
        %
        Concentration \unit{\g/\litre} & \num{1,4} & \num{1,3} & \num{1,2} & \num{1,1} & \num{1,0} \\ \hline
      \end{tblr} \\[4pt]

      \legende{
        Schéma de l'échelle de teinte réalisée, avec les solutions étalons et leurs concentrations.
      }
    \end{center}
  }[0.6]{
    \begin{center}
      \tubeEssaisSolution{red-150}
  
      \legende{Échantillon de sang à doser.}
    \end{center}
  }[0.4]
\end{doc}

%
\question{
  Rappeler avec vos mots le principe général d'un dosage par étalonnage (que veut-on mesurer et comment fait-on).
}{
  On cherche à mesurer une concentration en comparant les teintes de différentes solutions.
  C'est possible, car la teinte est proportionnelle à la concentration.
}[3]

%
\question{
  Pour préparer des solutions, on peut effectuer une dilution ou une dissolution. Indiquer en justifiant laquelle des deux méthode on utilise pour passer de la solution 2 à la solution 3.
}{
  On réalise une dilution, car on diminue la concentration.
}[2]

%
\question{
  En utilisant la figure du document~\ref{doc:A2_anemie}, indiquer en justifiant la concentration en hémoglobine de l'échantillon de sang.
}{
  La teinte de l'échantillon se trouve entre celle de la solution 2 et 3,
  donc sa concentration se trouve entre 1,3 et \qty{1,2}{\g\per\litre} d'hémoglobine.
}[3]

%
\question{
  L'échantillon vient d'une femme. Indiquer en justifiant si elle souffre d'anémie ou non.
}{
  Elle ne souffre pas d'anémie, car sa concentration en hémoglobine est supérieure à \qty{1,2}{\g\per\litre}.
}[2]
