%%%%
\teteSndChim
\vspace*{-6pt}
\nomPrenomClasse

%%%% titre
\numeroActivite{3}
\titreActivite{Détartrage chimique}

%%%% Objectifs
% \begin{objectifs}
%   \item Savoir écrire une réaction chimique équilibrée.
% \end{objectifs}

\begin{contexte}
  Le tartre est un dépôt solide de calcaire, le carbonate de calcium \chemfig{CaCO_3}.
  Lorsqu'une bouilloire est entartrée, ses performances sont réduites.
  Ainsi, il est important de détartrer régulièrement sa bouilloire avec du vinaigre blanc par exemple.
  \problematique{
    Quelle quantité de vinaigre blanc doit-on utiliser pour détartrer complètement le fond d'une bouilloire ?
  }
\end{contexte}


%%%% docs
\begin{doc}{Réaction chimique de détartrage}
  \label{doc:reaction_detartrage}
  Le tartre est un dépôt solide de calcaire, le carbonate de calcium \chemfig{CaCO_3}.
  Pour détartrer, il faut transformer cette espèce solide en espèces solubles dans l'eau ou gazeuses.
  Pour ça, on peut réaliser une réaction acido-basique entre un acide et le carbonate de calcium.
  Le vinaigre blanc ménager contient de l'acide éthanoïque \chemfig{C_2H_4O_2}.

  Lors de la réaction entre le carbonate de calcium et l'acide éthanoïque, on fait les observations suivantes :
  \begin{itemize}
    \item il y a un dégagement gazeux qui trouble l'eau de chaux ;
    \item la quantité d'eau dans le système augmente ;
    \item il se produit des ions calcium \chemfig{Ca^{2+}} et des ions éthanoate \chemfig{C_2H_3O_2^{-}}.
  \end{itemize}
    
  Ainsi, le carbonate de calcium solide s'est transformé en produits solubles dans l'eau ou gazeux.
\end{doc}

%%
\begin{doc}{Masse d'une mole des réactifs}
  La masse d'une mole est appelée la \important{masse molaire}.

  \begin{donnees}
    \item Une mole de calcaire \chemfig{CaCO_3} a une masse de $100 \unit{g}$.
    \item Une mole d'acide éthanoïque \chemfig{C_2H_4O_2} a une masse de $60 \unit{g}$.
  \end{donnees}
\end{doc}

%%
\begin{doc}{Les astuces de mamie}
  \label{doc:astuces_detartrage}
  Internet regorge de trucs et astuces pour le détartrage d'une bouilloire mais peu de sites s'accordent sur les quantités à utiliser.

  Certains recommandent de mettre environ $0,\!2 \unit{L}$ de vinaigre blanc à $12\degree$.
  D'autres conseillent de mettre la moitié de la bouteille de $1,\!0 \unit{L}$.
  D'autres encore proposent de mettre toute la bouteille de $1,\!0 \unit{L}$.
  
  \textbf{Note :} du vinaigre blanc à $12\degree$ contient $120\unit{g}$ d'acide éthanoïque pour $1,\!00\unit{L}$.
  Les degrés correspondent à une concentration massique, ici $c = 120 \unit{g / L}$.
\end{doc}

%%
\begin{doc}{Observations expérimentales}
  \label{doc:observations_detartrage}
  Les trois quantités citées ont été testées dans une bouilloire avec un dépôt pesant $90 \unit{g}$ de carbonate de calcium \chemfig{CaCO_3}.
  Voici les résultats obtenus :
  \begin{listePoints}
    \item avec $0,\!2 \unit{L}$ de vinaigre blanc, il reste un important dépôt solide ;
    \item avec la moitié d'une bouteille de $1,\!0 \unit{L}$, il reste un dépôt solide ;
    \item avec $1,\!0 \unit{L}$, il n'y a plus aucun solide présent au fond de la bouilloire.
  \end{listePoints}
\end{doc}


%%%% Questions
%\vspace*{-24pt}
% \begin{boite}
%   \textbf{Questions version \og découverte \fg}
% \end{boite}
%
\numeroQuestion
En vous aidant des documents, rédiger un rapport complet sur le détartrage chimique qui contiendra :
\begin{itemize}
  \item la réaction chimique qui permet d'éliminer le tartre ;
  \item une conclusion argumentée sur le volume de vinaigre blanc à $12\degree$ à utiliser pour détartrer le fond d'une bouilloire.
\end{itemize}
Les arguments doivent s'appuyer sur des calculs et être confirmés par des observations expérimentales.


%%%% Coups de pouce
\newpage
\pasDePagination
\setcounter{countCoupDePouce}{0}
\vspace*{-52pt}

%
\begin{coupDePouce}
  Lister les réactifs et les produits de la réaction en vous aidant du document~\ref{doc:reaction_detartrage}.
  Il y a 2 réactifs et 4 produits.
  L'eau de chaux se trouble en présence de dioxyde de carbone \chemfig{CO_2}.
\end{coupDePouce}

%
\begin{coupDePouce}
  Pour ajuster la réaction chimique, il faut commencer par ajuster la charge électrique totale avec un coefficient stoechiométrique.
  
  Une fois la charge électrique totale ajustée, il faut ajuster chaque éléments chimiques, en se rappelant que les coefficients stoechiométriques s'appliquent à la molécule entière. 
  Par exemple $2\chemfig{H_2O}$ veut dire qu'il y a 4 hydrogènes et 2 oxygènes.
\end{coupDePouce}

%
\begin{coupDePouce}
  Les coefficients stoechiométriques indiquent dans quelle proportion les réactifs sont transformés en produits.
  
  Ici il faut transformer 2 mole d'acide éthanoïque (coefficient stoechiométrique $ = 2$) pour transformer 1 mole de calcaire (coefficient stoechiométrique $ = 1$).
  
  En utilisant la masse d'une mole de calcaire et celle d'une mole d'acide éthanoïque, on peut déterminer la masse d'acide éthanoïque nécessaire pour éliminer le calcaire.
\end{coupDePouce}

%
\begin{coupDePouce}
  Pour obtenir la quantité de matière en mole de calcaire, il faut diviser la masse de calcaire par la masse d'une mole.
  %on a $n = \frac{80 \unit{g}}{100 \unit{g / mol}} = 0,\!8 \unit{mol}$ de calcaire.
  
  La quantité de matière $n$ d'acide éthanoïque est deux fois celle du calcaire.
  La masse d'acide éthanoïque est simplement sa quantité de matière $n$, multiplié par la masse d'une mole $M = 60 \unit{g / mol}$, soit $m = n \times M$.
  % soit $1,\!6 \unit{mol}$. Ces $1,\!6 \unit{mol}$ ont une masse $m = 1,\!6 \unit{mol} \times 60 \unit{g / mol} = 96 \unit{g}$.
\end{coupDePouce}

%
\begin{coupDePouce}
  Une fois que l'on connaît la masse d'acide éthanoïque nécessaire, comme on connaît le degré du vinaigre blanc, on peut en déduire le volume de vinaigre blanc qu'il faut utiliser.
  
  Le degré relie la masse d'acide éthanoïque et le volume de vinaigre blanc.
  Il faut diviser la masse calculée par le degré pour obtenir un volume en litre.
\end{coupDePouce}

%
\begin{coupDePouce}
  En calculant on trouve un volume théorique de vinaigre blanc de $0,\!9 \unit{L}$.
  Pour ce volume, les $90 \unit{g}$ de calcaire auront disparu, car transformés en ions solubles ou en gaz.
  
  En comparant avec ce qui est effectivement observé expérimentalement, on peut conclure sur la validité de la modélisation de la réaction chimique.
\end{coupDePouce}