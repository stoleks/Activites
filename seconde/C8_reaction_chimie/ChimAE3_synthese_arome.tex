%%%%
\teteSndChim

%%%% titre
\vspace*{-38pt}
\numeroActivite{3}
\titreTP{Synthèse de l'éthanoate d'isoamyle}

%%%% Objectifs
\begin{objectifs}
  \item Synthétiser une molécule naturelle et comprendre le montage à reflux.
  \item Réaliser un protocole en respectant les consignes de sécurités.
\end{objectifs}

\begin{contexte}
  Les arômes des aliments sont liées à des molécules captées par notre nez, auquel notre cerveau associe une odeur.
  
  \problematique{
    Comment synthétiser une molécule naturelle responsable de l'arôme de banane ?
  }
\end{contexte}


%%%% docs
\begin{doc}{Protocole de synthèse}{doc:protocole_synthese_banane}
  Dans le ballon, introduire :
  \begin{listePoints}
    \item 10,0 mL d'alcool isoamylique ;
    \item 15 mL d'acide éthanoïque ;
    \item 2 ou 3 pierre ponces ;
    \item 1 mL d'acide sulfurique. \attention \textbf{opération réalisée par l'enseignant} \attention
  \end{listePoints}
  
  Fixer le ballon au montage à reflux et lancer la circulation d'eau.
  Porter le mélange réactionnel à ébullition et chauffer à reflux pendant 30 minutes.
  Descendre le chauffe-ballon et laisser refroidir le ballon à l'air.
\end{doc}

\mesure Réaliser le protocole de synthèse du document~\ref{doc:protocole_synthese_banane}.


%%
\begin{doc}{Synthèse de l'éthanoate d'isoamyle}{doc:synthese_banane}
  La réaction de synthèse est
  \vspace*{-4pt}
  \begin{center}
    \chemfig{C_2H_4O_2}(l) + \chemfig{C_5H_{12}O}(l) \reaction \chemfig{C_7H_{14}O_2}(l) + \chemfig{H_2O}(l)
  \end{center}
  \vspace*{-4pt}
  
  \begin{tblr}{
    colspec = {X[1.5, c, m] X[1, c] X[1, c] X[1, c] X[1, c]},
    hlines, vlines
  }
    %
    Nom & Acide éthanoïque & Alcool isoamylique & Éthanoate d'isoamyle & Acide sulfurique \\
    %
    Formule & \chemfig{C_2H_4O_2} &
    \chemfig{C_5H_{12}O} &
    \chemfig{C_7H_{14}O_2} &
    (\chemfig{2H^+ ;\; SO_4^{2-}}) \\
    %
    Masse volumique &
    \qty{1,05}{\g\per\ml} &
    \qty{0,81}{\g\per\ml} &
    \qty{0,87}{\g\per\ml} &
    \qty{1,83}{\g\per\ml} \\
    %
    Solubilité dans l'eau & Grande & Moyenne & Faible & Grande \\
    %
    Solubilité dans l'eau salée & Grande & Très faible & Très faible & Grande \\
    Sécurité &
    \picto{0.38}{flambe}~\picto{0.38}{ronge} &
    \picto{0.38}{flambe}~\picto{0.38}{altere} &
    \picto{0.38}{flambe} &
    \picto{0.38}{ronge}
    \\
  \end{tblr}
\end{doc}

%%
\newpage
\vspace*{-24pt}
\question{
  Vérifier que la réaction de synthèse du document~\ref{doc:synthese_banane} est ajustée (équilibrée) en comptant chaque élément chimique du côté des réactifs et du côté des produits.
}{
  Oui elle est ajustée, on a autant d'élément carbone (7), hydrogène (16) et oxygène (3) des deux côtés.
}{2}


%%
\begin{doc}{Montage à reflux}{doc:montage_reflux}
  \vspace*{-24pt}
  \begin{encart}
    \begin{center}
      \image{0.5}{images/chimie/montage_reflux.png}
      
      \vspace*{-8pt}
      \textbf{\small{Schéma du montage à reflux}}
    \end{center}
  \end{encart}
  Pour accélérer la réaction de synthèse, on va chauffer le milieu réactionnel.
  
  Un montage à reflux permet de \texteTrouMultiLignes{chauffer le milieu réactionnel sans perte de quantité de matière à cause de l'évaporation.}{1}
  
  Le fonctionnement est le suivant : les vapeurs du milieu réactionnel passe au centre du réfrigérant.
  Cette vapeur est refroidie par l'eau qui circule sur les côtés du réfrigérant.
  En refroidissant, cette vapeur va se liquéfier et former des gouttes liquides, qui vont retomber dans le milieu réactionnel.
  Ainsi, on évite les pertes de réactifs due à l'évaporation du milieu réactionnel quand il est chauffé.
\end{doc}

\numeroQuestion Légender le schéma du montage à reflux du document~\ref{doc:montage_reflux}.



%%
\begin{doc}{Récupération du produit d'intérêt}{doc:decantation_synthese_banane}
  Après réalisation de la synthèse, procéder à un relargage : introduire dans le ballon 25 mL d'eau salée saturée.
  \vspace*{-8pt}
  \begin{center}
    \image{1}{images/chimie/protocole_decantation}
  \end{center}
  \vspace*{-16pt}
  Verser le mélange réactionnel dans l'ampoule à décanter.
  Agiter, puis laisser décanter.
  Éliminer la phase aqueuse dans un bécher, recueillir alors la phase organique dans un tube fermé.
\end{doc}


%%%% Questions

\question{
  En vous aidant du tableau du document~\ref{doc:synthese_banane}, expliquer pourquoi on introduit de l'eau salée pour récupérer le produit d'intérêt, l'éthanoate d'isoamyle.
}{
  L'éthanoate d'isoamyle est peu soluble dans l'eau salée, contrairement à toutes les autres espèces.
  Cela va permettre de l'isoler par décantation en créant un mélange hétérogène.
}{2}

\mesure Récupérer l'éthanoate d'isoamyle, en réalisant le protocole du document~\ref{doc:decantation_synthese_banane}


%%
\bigskip
\begin{doc}{Les arômes d'un gâteau à la banane}{doc:A_arome_banane}
  L'arôme d'un fruit ne dépend pas d'un seul type de molécule.
  Pour recomposer un arôme de pomme, il faut au moins 50 molécules différentes, dans les bonnes proportions.
  Dans la banane, l'arôme est essentiellement dû à une seule molécule : l'éthanoate d'isoamyle \chemfig{C_7H_{12}O_2}.
  
  Pour faire un gâteau à la banane une méthode consiste à extraire l'arôme de la banane, par exemple en écrasant des bananes dans la préparation.
  Autre méthode : synthétiser la molécule d'éthanoate d'isoamyle, à partir d'alcool isoamylique.
  Cette fois, l'arôme n'est plus appelé \og naturel \fg, mais \og identique au naturel \fg.
  
  \textbf{Il n'y a aucune différence entre la molécule extraite de la banane et celle synthétisée}, dont la formule est aussi \chemfig{C_7H_{12}O_2}.
\end{doc}


%%
\newpage
\begin{doc}{Préparer un gâteau à la banane}{doc:A_gateau_banane}
  Dans une recette, pour préparer un gâteau au goût de banane, il faut 250 g de banane ou 1 mL d'arôme de banane.
  
  \begin{donnees}
    \item 50 mL d'arôme de banane synthétisé coûte 3 \euro.
    \item 1 kg de banane coûte 3 \euro.
  \end{donnees}
\end{doc}

\question{
  Pour quelle(s) raison(s) pourrait-on privilégier un arôme de banane synthétisé pour réaliser des gâteaux ?
}{
  Utiliser un arôme synthétisé coûte moins cher. 
}{3}