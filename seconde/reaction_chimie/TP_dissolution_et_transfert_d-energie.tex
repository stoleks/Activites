%%%%
\teteSndChim

%%%% titre
\vspace*{-28pt}
\titreTP{Dissolution et transfert d'énergie}

%%%% Objectifs
\begin{objectifs}
  \item Comprendre la notion de réaction endothermique et exothermique.
  \item Réaliser des dissolutions en respectant les consignes de sécurité.
\end{objectifs}

\begin{contexte}
  Quand on ajoute de l'acide chlorhydrique dans de la soude, une réaction chimique a lieu et la température de la solution augmente.
  On dit que la réaction est \important{exothermique} : de l'énergie a été libérée.
  
  \problematique{
    Peut-on contrôler la température à la fin de la réaction en changeant les conditions initiales ?
  }
\end{contexte}


%%%% docs
\begin{doc}{Réaction endothermique et exothermique}{doc:TP_endo_exothermique}
  Une transformation endothermique nécessite d'absorber de l'énergie pour avoir lieu.
  Cette perte d'énergie sous forme de transfert thermique implique un abaissement de la température du milieu extérieur.
  
  \begin{importants}
    Pour une réaction chimique en solution, la solution va donc voir sa \important{température diminuer} si la réaction est \important{endothermique.}
  \end{importants}
  
  Il est ainsi possible de faire baisser la température chimiquement, par exemple si on dissout dans de l'eau une espèce chimique dont la dissolution est endothermique.
  
  \attention Toutes les transformations de dissolution ne sont pas endothermique !
  
  \begin{importants}
    Inversement, la solution va voir sa \important{température augmenter} si la réaction chimique est \important{exothermique.}
  \end{importants}
\end{doc}

%%
\begin{doc}{Le chlorure d'ammonium}{doc:chlorure_ammonium}
  Le chlorure d'ammonium \chemfig{NH_4Cl}, est un solide blanc à température ambiante.
  Il est irritant pour les yeux et nocif en cas d'ingestion.
  \important{On portera donc des lunettes de protection pendant toute les manipulations}.
  
  Le chlorure d'ammonium est soluble dans l'eau jusqu'à une certaine limite : on ne pourra dissoudre que \qty{37,2}{\g} dans \qty{100}{\ml} d'eau à \qty{20}{\degreeCelsius}.

  Lors de la dissolution du chlorure d'ammonium dans l'eau, il se dissocie en ses ions constitutifs : les ions ammonium \ammonium, et les ions chlorure \chlorure.

  \smallskip
  \attention Danger du \chemfig{NH_4Cl} : H302 (toxicité aiguë) ; H319 (irritation des yeux).
\end{doc}

%%
\begin{doc}{Dissolution à réaliser}{doc:dissolution_protocole}
  Pour réaliser la réaction de dissolution décrite dans le document~\ref{doc:chlorure_ammonium}, prendre 2 béchers et verser dans chacun \qty{50}{\ml} d’eau distillée.
  
  Mesurer la masse d'eau distillée versée $m_\text{eau} =$ \correction{\qty{50,0}{\g}}
  
  Ajouter les masses suivantes de chlorure d'ammonium \chemfig{NH_4 Cl} :
  \begin{listePoints}
    \item bécher 1 : $m_1 = \qty{4,0}{\g}$
    \item bécher 2 : $m_2 = \qty{10,0}{\g}$
  \end{listePoints}
\end{doc}


%%%% Questions
\question{
  Écrire la réaction de dissolution du chlorure d'ammonium dans l'eau.
}{
  \begin{equation*}  
    \chemfig{NH_4 Cl}(s) \reaction \ammonium + \chlorure
  \end{equation*}
}[1]

%%
\mesure
Réaliser les dissolutions demandées dans le document~\ref{doc:dissolution_protocole}. 
Mesurer la température initiale $T_i$ avant l’ajout du solide, puis la température finale $T_f$ lorsque celle-ci ne varie plus.
Noter les résultats dans le tableau suivant :
\begin{center}
  \begin{tblr}{
    colspec = {c c c X[c]}, hlines, vlines,
    row{1} = {couleurSec-100},
  }
    &
    Température initiale $T_i$ &
    Température finale $T_f$ &
    Variation de température $\Delta T = T_f - T_i$ \\
    %
    Bécher 1 &
    \correction{\qty{20,0}{\degreeCelsius}} &
    \correction{\qty{15,8}{\degreeCelsius}} &
    \correction{\qty{-4,2}{\degreeCelsius}} \\
    Bécher 2 &
    \correction{\qty{20,0}{\degreeCelsius}} &
    \correction{\qty{7,4 }{\degreeCelsius}} &
    \correction{\qty{-12,6}{\degreeCelsius}} \\
    % Bécher 3 & & &
    % \\ \hline
  \end{tblr}
\end{center}

%%
\question{
  La réaction de dissolution est-elle endothermique ou exothermique ? Justifier.
}{
  La réaction est endothermique, car la température de la solution baisse pendant la dissolution.
}[2]

%%
\question{
  Quel est l'impact de la masse de \chemfig{NH_4Cl} sur la variation de la température ?
}{
  La variation de température augmente avec la masse : plus la masse est élevée et plus la température diminue.
}[1]

%%
\question{
  Calculer l’énergie absorbée par la réaction de dissolution $E = m_\text{eau} \times c_\text{eau} \times \Delta T$.
  \important{Donnée :}
  La capacité thermique de l’eau vaut $c_\text{eau} = \qty{4,180}{\joule\per\g\per\degreeCelsius}$
}{
  On calcule l'énergie à l'aide de la relation littérale donnée
  \begin{align*}
    E_1 &= m_\text{eau} \times c_\text{eau} \times \Delta T_1 \\
        &= \qty{50,0}{\g} \times \qty{4,180}{\joule\per\g\per\degreeCelsius} \times (\num{-4.2} - \num{15,8})\unit{\degreeCelsius} \\
        &= \qty{878}{\joule}.
  \end{align*}

  On calcule de la même façon $E_2$, donc $E_1 = \qty{878}{\joule}$ et $E_2 = \qty{2633}{\joule}$.
}[3]

%%
\question{
  Calculer l'énergie de dissolution massique $E_m = - E / m$, avec $m$ la masse de chlorure d'ammonium dissoute.
  Comparer avec la valeur de référence $E_m = \qty{276,3}{\joule\per\g}$.
}{
  Dans le premier cas, on a 
  \begin{equation*}
    E_m = \dfrac{E_1}{m_1}
    = \dfrac{\qty{878}{\joule}}{\qty{5,0}{\g}}
    = \qty{219,5}{\joule\per\g}
  \end{equation*}
  Dans le second cas on trouve $E_m = E_2 / m_2 = \qty{263.3}{\joule\per\g}$.

  Dans les deux cas on trouve une valeur plus faible que celle attendue, ce qui peut s'expliquer par le contact entre le bécher et l'air extérieur, qui entraine des transferts de températures et augmente la température de l'eau.
}[2]
