%%%%
\teteSndChim

%%%% titre
\titreTP{Extincteur chimique}

%%%% Objectifs
\begin{objectifs}
  \item Comprendre qu'une réaction chimique microscopique peut modéliser plusieurs transformations macroscopiques.
  \item Comprendre le principe de réactif limitant.
\end{objectifs}

\begin{contexte}
  Le bicarbonate de sodium est un produit utilisé couramment pour le nettoyage ou la cuisine, sa formule brute est \chemfig{NaHCO_3}.

  Associé avec du vinaigre blanc dans un extincteur, il peut aussi servir à former du dioxyde de carbone pour éteindre les incendies.
  
  \problematique{
    Quelles quantités de vinaigre ou de bicarbonate faut-il mettre pour avoir un extincteur efficace ?
  }
\end{contexte}


%%%% Exp
\begin{doc}{Protocole pour réaliser un mini extincteur}{doc:TP1_exp_extincteur}
  \begin{protocole}
    \item Remplir à moitié le bécher de vinaigre d'alcool.
    \item À l'aide d'une éprouvette graduée, verser \qty{20}{\ml} de vinaigre d'alcool dans la fiole jaugée.
    \item Peser une masse $m$ de bicarbonate de soude, choisie dans le tableau ci-dessous.
    \begin{center}
      \begin{tblr}{
        cells = {c}, hlines, vlines,
        column{1} = {couleurSec-100}
      }
        Masse $m$ de bicarbonate &
        \qty{0,5}{\g} &
        \qty{1,0}{\g} &
        \qty{1,5}{\g} &
        \qty{2,6}{\g} &
        \qty{4,0}{\g} \\
      \end{tblr}
    \end{center}
    \item Verser le bicarbonate pesé dans un ballon en baudruche.
    \item Entourer le col de la fiole jaugée avec le ballon de baudruche.
    \item Redresser et agiter doucement le ballon de baudruche pour faire tomber le bicarbonate de sodium.
    \item Ne plus toucher au ballon.
  \end{protocole}
\end{doc}

\mesure Après l'avoir lu en entier, réaliser le protocole du document~\ref{doc:TP1_exp_extincteur}.
Noter vos observations dans le tableau ci-dessous :
\begin{center}
  \begin{tblr}{
    colspec = {c c c}, hlines, vlines,
    column{1} = {couleurSec-50},
    row{1} = {X[c], couleurSec-100}
  }
    Masse de \bicarbonateDeSodium & Présence de \bicarbonateDeSodium solide & Gonflement du ballon (+, ++, +++, ++++) \\
    \qty{0,5}{\g} \\
    \qty{1,0}{\g} \\
    \qty{1,5}{\g} \\
    \qty{2,6}{\g} \\
    \qty{4,0}{\g} \\
  \end{tblr}
\end{center}


%%
\begin{doc}{Réactif limitant}{doc:TP1_reactif_limitant}
  Une réaction chimique s'arrête quand un des réactifs est complètement transformé.
  \begin{importants}
    Dans une réaction chimique, le \important{réactif limitant} est le réactif qui est totalement transformé, qui disparaît complètement.
    Il est dit « \important{limitant} », car il est responsable de l'arrêt de la transformation.
  \end{importants}
\end{doc}

\question{
  En vous aidant de vos observations pour justifier, indiquer quel est le réactif limitant pour les 5 cas étudiés.
}{
  ...
}[3]



%%%% Theo
\begin{doc}{Réaction chimique dans l'extincteur}{doc:TP1_reaction_extincteur}
  Le bicarbonate de sodium \chemfig{NaHCO_3} se présente sous la forme d'une poudre solide.
  Pour produire du dioxyde de carbone gazeux, on réalise une réaction acio-basique avec un acide et le bicarbonate de sodium.
  
  Le vinaigre blanc ménager contient de l'acide éthanoïque \chemfig{C_2H_4O_2}.
  Lors de la réaction entre le bicarbonate de sodium et l'acide éthanoïque, on fait les observations suivantes :
  \begin{listePoints}
    \item il y a un dégagement gazeux de dioxyde de carbone \chemfig{CO_2} ;
    \item la quantité d'eau liquide dans le système augmente ;
    \item des ions sodium \chemfig{Na^{+}} sont produits ;
    \item des ions éthanoate \chemfig{C_2H_3O_2^{-}} sont produits.
  \end{listePoints}
\end{doc}

\question{
  Lister les réactifs de la réaction chimique, en précisant leur états physique.
}{}[2]

\question{
  Lister les produits de la réaction chimique, en précisant leur états physique.
}{}[2]

\question{
   Écrire la réaction chimique dans l'extincteur, avec à gauche de la flèche les réactifs et à droite les produits.
}{}[4]

%%
% \begin{doc}{Masse d'une mole des réactifs}{doc:A_}
%   La masse d'une mole est appelée la \important{masse molaire}.
% 
%   \begin{donnees}
%     \item Une mole de calcaire \chemfig{CaCO_3} a une masse de $100 \unit{g}$.
%     \item Une mole d'acide éthanoïque \chemfig{C_2H_4O_2} a une masse de $60 \unit{g}$.
%   \end{donnees}
% \end{doc}
