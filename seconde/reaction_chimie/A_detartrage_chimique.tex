%%%%
\teteSndChim
\vspace*{-6pt}
\nomPrenomClasse
\titreActivite{Détartrage chimique}

%%%% Objectifs
% \begin{objectifs}
%   \item Savoir écrire une réaction chimique équilibrée.
% \end{objectifs}

\begin{contexte}
  Le tartre est un dépôt solide de calcaire, le carbonate de calcium \chemfig{CaCO_3}.
  Lorsqu'une bouilloire est entartrée, ses performances sont réduites.
  Ainsi, il est important de détartrer régulièrement sa bouilloire avec du vinaigre blanc par exemple.
  
  \problematique{
    Quelle quantité de vinaigre blanc doit-on utiliser pour détartrer complètement le fond d'une bouilloire ?
  }
\end{contexte}


%%%% docs
\begin{doc}{Réaction chimique de détartrage}{doc:reaction_detartrage}
  Le tartre est un dépôt solide de calcaire, le carbonate de calcium \chemfig{CaCO_3}.
  Pour détartrer, il faut transformer cette espèce solide en espèces solubles dans l'eau ou gazeuses.
  Pour ça, on peut réaliser une réaction acido-basique entre un acide et le carbonate de calcium.
  Le vinaigre blanc ménager contient de l'acide éthanoïque \chemfig{C_2H_4O_2}.

  Lors de la réaction entre le carbonate de calcium et l'acide éthanoïque, on fait les observations suivantes :
  \begin{itemize}
    \item il y a un dégagement gazeux qui trouble l'eau de chaux ;
    \item la quantité d'eau dans le système augmente ;
    \item il se produit des ions calcium \ionCalcium et des ions éthanoate \chemfig{C_2H_3O_2^{-}}.
  \end{itemize}
    
  Ainsi, le carbonate de calcium solide s'est transformé en produits solubles dans l'eau ou gazeux.
\end{doc}

%%
\begin{doc}{Masse d'une mole des réactifs}{doc:A_masse_mole_reactif}
  La masse d'une mole est appelée la \important{masse molaire}.

  \begin{donnees}
    \item Une mole de calcaire \chemfig{CaCO_3} a une masse de $100 \unit{g}$.
    \item Une mole d'acide éthanoïque \chemfig{C_2H_4O_2} a une masse de $60 \unit{g}$.
  \end{donnees}
\end{doc}

%%
\begin{doc}{Les astuces de mamie}{doc:astuces_detartrage}
  Internet regorge de trucs et astuces pour le détartrage d'une bouilloire mais peu de sites s'accordent sur les quantités à utiliser.

  Certains recommandent de mettre environ \qty{0,2}{\litre} de vinaigre blanc à \qty{12}{\degree}.
  D'autres conseillent de mettre la moitié de la bouteille de \qty{1,0}{\litre}.
  D'autres encore proposent de mettre toute la bouteille de \qty{1,0}{\litre}.
  
  \textbf{Note :} du vinaigre blanc à \qty{12}{\degree} contient \qty{120}{\g} d'acide éthanoïque pour \qty{1,00}{\litre}.
  Les degrés correspondent à une concentration massique, ici $c = \qty{120}{\g\per\litre}$.
\end{doc}

%%
\begin{doc}{Observations expérimentales}{doc:observations_detartrage}
  Les trois quantités citées ont été testées dans une bouilloire avec un dépôt pesant $90 \unit{g}$ de carbonate de calcium \chemfig{CaCO_3}.
  Voici les résultats obtenus :
  \begin{listePoints}
    \item avec \qty{0,2}{\litre} de vinaigre blanc, il reste un important dépôt solide ;
    \item avec la moitié d'une bouteille de \qty{1,0}{\litre}, il reste un dépôt solide ;
    \item avec \qty{1,0}{\litre}, il n'y a plus aucun solide présent au fond de la bouilloire.
  \end{listePoints}
\end{doc}


%%%% Questions
%\vspace*{-24pt}
% \begin{boite}
%   \textbf{Questions version \og découverte \fg}
% \end{boite}
%
\numeroQuestion
En vous aidant des documents, rédiger un rapport complet sur le détartrage chimique qui contiendra :
\begin{itemize}
  \item la réaction chimique qui permet d'éliminer le tartre ;
  \item une conclusion argumentée sur le volume de vinaigre blanc à \qty{12}{\degree} à utiliser pour détartrer le fond d'une bouilloire.
\end{itemize}
Les arguments doivent s'appuyer sur des calculs et être confirmés par des observations expérimentales.
