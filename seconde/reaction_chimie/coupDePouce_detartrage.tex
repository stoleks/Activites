\pasCorrection{\newpage\pasDePagination\vspace*{-52pt}}
\setcounter{coupDePouceNum}{0}

%
\begin{coupDePouce}
  Lister les réactifs et les produits de la réaction en vous aidant du document~\ref{doc:reaction_detartrage}.
  Il y a 2 réactifs et 4 produits.
  L'eau de chaux se trouble en présence de dioxyde de carbone \dioxydeDeCarbone.
\end{coupDePouce}

%
\begin{coupDePouce}
  Pour ajuster la réaction chimique, il faut commencer par ajuster la charge électrique totale avec un coefficient stœchiométrique.
  
  Une fois la charge électrique totale ajustée, il faut ajuster chaque éléments chimiques, en se rappelant que les coefficients stœchiométriques s'appliquent à la molécule entière. 
  Par exemple $2\eau$ veut dire qu'il y a 4 hydrogènes et 2 oxygènes.
\end{coupDePouce}

%
\begin{coupDePouce}
  Les coefficients stœchiométriques indiquent dans quelle proportion les réactifs sont transformés en produits.
  
  Ici il faut transformer 2 mole d'acide éthanoïque (coefficient stœchiométrique $ = 2$) pour transformer 1 mole de calcaire (coefficient stœchiométrique $ = 1$).
  
  En utilisant la masse d'une mole de calcaire et celle d'une mole d'acide éthanoïque, on peut déterminer la masse d'acide éthanoïque nécessaire pour éliminer le calcaire.
\end{coupDePouce}

%
\begin{coupDePouce}
  Pour obtenir la quantité de matière en mole de calcaire, il faut diviser la masse de calcaire par la masse d'une mole.
  On a une quantité $n = \dfrac{\qty{80}{\g}}{\qty{100}{\g / \mole}} = \qty{0,8}{\mole}$ de calcaire.
  
  La quantité de matière $n$ d'acide éthanoïque est deux fois celle du calcaire.
  La masse d'acide éthanoïque est simplement sa quantité de matière $n$, multiplié par la masse d'une mole $M = \qty{60}{\g / \mole}$, soit $m = n \times M$.
  
  Ces \qty{1,6}{\mole} ont une masse $m = \qty{1,6}{\mole} \times \qty{60}{\g / \mole} = \qty{96}{\g}$.
\end{coupDePouce}

%
\begin{coupDePouce}
  Une fois que l'on connaît la masse d'acide éthanoïque nécessaire, comme on connaît le degré du vinaigre blanc, on peut en déduire le volume de vinaigre blanc qu'il faut utiliser.
  
  Le degré relie la masse d'acide éthanoïque et le volume de vinaigre blanc.
  Il faut diviser la masse calculée par le degré pour obtenir un volume en litre.
\end{coupDePouce}

%
\begin{coupDePouce}
  En calculant on trouve un volume théorique de vinaigre blanc de $\qty{0,9}{\litre}$.
  Pour ce volume, les $\qty{90}{\g}$ de calcaire auront disparu, car transformés en ions solubles ou en gaz.
  
  En comparant avec ce qui est effectivement observé expérimentalement, on peut conclure sur la validité de la modélisation de la réaction chimique.
\end{coupDePouce}