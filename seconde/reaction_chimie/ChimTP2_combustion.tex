%%%%
\teteSndChim

%%%% titre
\numeroActivite{2}
\titreTP{Se chauffer au gaz}
%\smallskip

%%%% Objectifs
\begin{objectifs}
  \item Ajuster une réaction chimique à l'aide de coefficients stoechiométriques.
  \item Comprendre la notion de réaction endothermique et exothermique. 
  \item Calculer le volume de gaz nécessaire pour faire bouillir \qty{1}{\litre} d'eau.
  \item Réaliser des dissolutions en respectant les consignes de sécurités.
\end{objectifs}

\begin{contexte}
  Dans les chaudière à gaz (chauffe-eau) ou dans les cuisinières à gaz, on utilise la combustion du méthane pour chauffer de l'eau ou des aliments.
  
  \problematique{
    Quelle est la réaction chimique de la combustion du méthane ?
  }
\end{contexte}


%%%% docs
\begin{doc}{La combustion du méthane}{doc:TP2_combustion}
  Le méthane \chemfig{CH_4} réagit avec le dioxygène \chemfig{O_2} lors de sa combustion pour former deux produits.
  La combustion produit deux gaz :
  \begin{listePoints}
    \item de la vapeur d'eau \chemfig{H_2O}, identifiée avec du sulfate de cuivre anhydre ;
    \item du dioxyde de carbone \chemfig{CO_2}, identifié avec de l'eau de chaux.
  \end{listePoints}
\end{doc}


%%%% Questions
\question{
  Lister les réactifs et les produits de la réaction de combustion du méthane.
}{
  Les réactifs sont le méthane \chemfig{CH_4} et le dioxygène \chemfig{O_2}.
  Les produits sont le dioxyde de carbone \chemfig{CO_2} et l'eau \eau.
}{2}

\question{
  Écrire la réaction chimique de combustion du méthane, en précisant les états physiques de chaque espèce chimique.
}{
  \begin{center}
      \chemfig{CH_4}(g) + \chemfig{O_2}(g)
      \reaction
      \chemfig{CO_2}(g) + \chemfig{H_2O}(g)
  \end{center}
}{2}

%%
\begin{doc}{Ajustage d'une réaction}{doc:TP2_equlibrage_reaction}
  Au cours d'une réaction chimique, les éléments chimiques présents dans les réactifs se réarrangent pour former des produits et les liaisons chimiques changent.
  \begin{importants}
    Il y a \important{conservation} 
    \begin{listePoints}
      \item \important{des éléments chimiques} ;
      \item \important{de la charge électrique} totale.
    \end{listePoints}
  \end{importants}
  \begin{importants}
    Pour assurer cette \important{conservation}, il faut \important{ajuster} la réaction chimique avec des coefficients devant les éléments chimiques.
    Ces coefficients sont appelés \important{coefficient stoechiométrique.}
  \end{importants}
  
  Exemple de la réaction d'un acide avec du magnésium :
  \begin{equation*}
    \underset{\text{1 atome de magnésium}}{\chemfig{Mg}(s)}
    + \underset{\text{\important{2} ions hydrogènes}}{\important{2} \;\; \chemfig{H^+ }(aq)}
    \reaction
    \underset{\text{1 ion magnésium II}}{\chemfig{Mg^{2+}}(aq)}
    + \underset{\text{1 molécule de dihydrogène}}{\chemfig{H_2}(g)}
  \end{equation*}
  On vérifie bien qu'il y a le même nombre de charges positives, de magnésium \chemfig{Mg} et d'hydrogène \chemfig{H}, dans l'état initial et dans l'état final.
\end{doc}


\newpage
\vspace*{-24pt}
\question{
  Ajuster la réaction de combustion du méthane à l'aide de coefficients stoechiométriques.
  Commencer par ajuster le nombre d'atomes d'hydrogène.
}{
  \begin{equation*}
    \chemfig{CH_4}(g) + 2\chemfig{O_2}(g)
    \reaction
    \chemfig{CO_2}(g) + 2\chemfig{H_2O}(g)
  \end{equation*}
}{2}


%%%% bonus
\vspace*{-4pt}
\begin{doc}{Le propane}{doc:TP2_propane}
  Parfois le gaz utilisé pour se chauffer est du propane et non du méthane.
  La formule chimique de la molécule de propane est \chemfig{C_3 H_{8}}.
  Le propane réagit avec le dioxygène et sa combustion forme les mêmes produits que la combustion du méthane.
\end{doc}

\question{
  Écrire la réaction de combustion du propane ajustée avec des coefficients stoechiométriques.
  Préciser l'état physique des réactifs et des produits.
}{
  ...
}{2}

\vspace*{-4pt}
\begin{doc}{L'eau de chaux}{doc:TP2_eaux_de_chaux}
  L'eau de chaux est une solution aqueuse saturée en ion calcium \chemfig{Ca^{2+}} et en ion hydroxyde \chemfig{HO^{-}}.
  En réagissant avec le dioxyde de carbone \chemfig{CO_2}, l'eau de chaux forme du calcaire \chemfig{Ca CO_3} et de l'eau \chemfig{H_2O}
\end{doc}

\question{
  Écrire la réaction de formation du calcaire dans l'eau de chaux en présence de dioxyde de carbone et l'ajuster avec des coefficients stoechiométrique.
}{
  ...
}{2}

\numeroQuestion
Ajuster les réactions chimiques suivantes en écrivant, si nécessaire, les coefficients stoechiométriques devant chaque élément chimique :
\newcommand{\localEcart}{18}
\begin{align*}
  \ldots\chemfig{C}(s) + \ldots\chemfig{O_2}(g)
  &\reaction \ldots\chemfig{CO_2}(g)
  \\[\localEcart pt]
  %
  \ldots\chemfig{Fe}(s) + \ldots\chemfig{H^{+}}(aq)
  &\reaction \ldots\chemfig{Fe^{2+}}(aq) + \ldots\chemfig{H_2}(g)
  \\[\localEcart pt]
  %
  \ldots\chemfig{Fe}(s) + \ldots\chemfig{O_2}(g)
  &\reaction \ldots\chemfig{Fe_2O_3}(s)
  \\[\localEcart pt]
  %
  \ldots\chemfig{C_2H_6O}(l) + \ldots\chemfig{O_2}(g)
  &\reaction \ldots\chemfig{CO_2}(g) + \ldots\chemfig{H_2O}(l)
  \\[\localEcart pt]
  %
  \ldots\chemfig{Cu^{2+}}(aq) + \ldots\chemfig{HO^{-}}(aq)
  &\reaction \ldots\chemfig{Cu{(HO)}_2}(s)
  \\[\localEcart pt]
  %
  \ldots\chemfig{Fe}(s) + \ldots\chemfig{H_2O}(l) + \ldots\chemfig{O_2}(g)
  &\reaction \ldots\chemfig{Fe{(HO)}_2}(s)
  % \\[\localEcart pt]
  % %
  % \ldots\chemfig{Fe{(OH)}_2}(s) + \ldots\chemfig{H_2O}(l) + \ldots\chemfig{O_2}(g)
  % &\reaction \ldots\chemfig{Fe{(OH)}_3}(s)
  % \\[\localEcart pt]
  % %
  % \ldots\chemfig{Fe{(OH)}_3}(s)
  % &\reaction \ldots\chemfig{Fe_2O_3}(s) + \ldots\chemfig{H_2O}(l)
  %
\end{align*}
\vspace*{-16pt}

%%
\begin{minipage}[t]{0.48\linewidth}\vspace{0pt}
  \numeroQuestion
  Pour travailler la notion d'ajustement :

  \begin{center}  
    \qrcode{https://www.pccl.fr/physique_chimie_college_lycee/quatrieme/chimie/reactions_chimiques_flash.htm}
  \end{center}
\end{minipage}
\begin{minipage}[t]{0.48\linewidth}\vspace{0pt}
  \numeroQuestion
  Pour aller plus loin :

  \begin{center}  
    \qrcode{https://phet.colorado.edu/sims/html/balancing-chemical-equations/latest/balancing-chemical-equations_fr.html}
  \end{center}
\end{minipage}


\begin{doc}{Réaction endothermique et exothermique}{doc:TP2_endo_exo}
  Une transformation endothermique nécessite d'absorber de l'énergie pour avoir lieu.
  Cette perte d'énergie sous forme de transfert thermique implique un abaissement de la température du milieu extérieur.
  
  \begin{importants}
    Pour une réaction chimique en solution, la solution va donc voir sa \important{température diminuer} si la réaction est \important{endothermique.}
  \end{importants}
  
  Il est ainsi possible de faire baisser la température chimiquement, par exemple si on dissout dans de l'eau une espèce chimique dont la dissolution est endothermique.
  
  \begin{importants}
    Inversement, la solution va voir sa \important{température augmenter} si la réaction chimique est \important{exothermique.}
  \end{importants}
\end{doc}

%%
\begin{multicols}{2}
  \begin{doc}{Le chlorure de sodium}{doc:TP2_chlorure_sodium}
    Le chlorure de sodium \chemfig{NaCl} est un solide blanc à température ambiante : c'est le sel de table.
    \vspace*{4pt}
    
    Le chlorure de sodium est soluble dans l'eau jusqu'à une certaine limite : on ne pourra dissoudre que \qty{3,52}{\g} dans \qty{10}{\ml} d'eau à \qty{20}{\degreeCelsius}.
    \vspace*{5pt}
  
    Lors de la dissolution du chlorure de sodium dans l'eau, il se dissocie en ses ions constitutifs : les ions sodium \chemfig{Na^+}, et les ions chlorure \chemfig{Cl^{-}}.
  \end{doc}
  
  \begin{doc}{L'hydroxyde de sodium}{doc:TP2_hydroxyde_sodium}
    L'hydroxyde de sodium \chemfig{NaOH} compose la soude, qui est utilisée pour déboucher les canalisations.
    \vspace*{-4pt}

    \begin{importants}  
      \attention L'hydroxyde de sodium est fortement corrosif, on portera donc des gants, une blouse et des lunettes pendant toutes les manipulations.
    \end{importants}
    \vspace*{-4pt}
  
    Dans l'eau, \chemfig{NaOH} se dissocie en ses ions constitutifs : les ions sodium \chemfig{Na^+}, et les ions hydroxyde \chemfig{HO^{-}}.
  \end{doc}
\end{multicols}

%%
\begin{doc}{Dissolution à réaliser}{doc:TP2_dissolution_protocole}
  \begin{protocole}
      
    \item Prendre 2 béchers et verser dans chacun \qty{10}{\ml} d’eau distillée.
    \item Mesurer la masse d'eau distillée versée $m_\text{eau} =$ \texteTrou[0.1]{\qty{10,0}{\g}}
    \item Mesurer la température initiale de l'eau des deux béchers.
    \item Ajouter \qty{3,0}{\g} de chlorure de sodium dans un bécher.
    \item Ajouter une pastille de soude dans l'autre bécher.
    \item Mesurer la température finale de l'eau des deux béchers après une dizaine de secondes.
  \end{protocole}
\end{doc}


%%%% Questions
% \question{
%   Écrire la réaction de dissolution du chlorure d'ammonium dans l'eau.
% }{
%   \chemfig{NH_4 Cl}(s) \reaction \chemfig{NH_4^+} + \chemfig{Cl^{-}}
% }{1}

\mesure
Réaliser les dissolutions demandées dans le document~\ref{doc:TP2_dissolution_protocole}. 
Noter les mesures de températures dans le tableau suivant :
\begin{center}
  \begin{tblr}{
    colspec = {c c c X[c]},  hlines, vlines,
    row{1} = {couleurPrim!20},
  }
    &
    Température initiale $T_i$ &
    Température finale $T_f$ &
    Variation de température $\Delta T = T_f - T_i$ \\
    %
    Bécher 1 &
    \correction{$20,0 \unit{\degreeCelsius}$} &
    \correction{$15,8 \unit{\degreeCelsius}$} &
    \correction{$-4,2 (\degreeCelsius)$} \\
    Bécher 2 &
    \correction{$20,0 \unit{\degreeCelsius}$} &
    \correction{$7,4 \unit{\degreeCelsius}$} &
    \correction{$-12,6 \unit{\degreeCelsius}$} \\
  \end{tblr}
\end{center}

%%
\question{
  Parmi les deux dissolution, indiquer laquelle est exothermique et laquelle est endothermique. Justifier.
}{
  La réaction est endothermique, car la température de la solution baisse pendant la dissolution.
}{2}

%%
\question{
  Calculer l’énergie libérée par les deux réactions de dissolution $E = m_\text{eau} \times c_\text{eau} \times \Delta T$.
  \important{Donnée :}
  La capacité thermique de l’eau vaut $c_\text{eau} = \qty{4,180}{\joule\per\g\per\degreeCelsius}$
}{
  $E_1 = \qty{875}{\joule}$,
  $E_2 = \qty{1750}{\joule}$,
  $E_2 = \qty{2625}{\joule}$
}{3}

%%
% \question{
%   Calculer l'énergie de dissolution massique $E_m = - E / m$, avec $m$ la masse de chlorure d'ammonium dissoute.
%   Comparer avec la valeur de référence $E_m = \qty{276,3}{\joule\per\g}$.
% }{
%   ...
% }{2}

\begin{doc}{Énergie de combustion du méthane}{doc:TP2_energie_combustion}
  Comme la réaction de dissolution de la soude, la combustion du méthane libère de l'énergie.
  L'énergie libérée dépend du volume de gaz qui est brûlé.
  L'énergie de combustion volumique du méthane est $E_V = \qty{3,641e4}{\joule\per\m\cubed}$, donc l'énergie libérée quand on brûle un volume $V$ de gaz est simplement 
  \begin{equation*}
    E = E_V \times V
  \end{equation*}

  Comme on connaît la capacité thermique massique de l'eau, on peut calculer l'énergie nécessaire pour faire bouillir 1 litre d'eau :
  \begin{align*}
    E &= m_\text{eau} \times c_\text{eau} \times \Delta T \\
    E_V \times V &= \rho_\text{eau} \times V_\text{eau} \times c_\text{eau} \times \Delta T
  \end{align*}

  On peut diviser par $E_V$ des deux côté de l'équation pour calculer le volume de gaz nécessaire pour faire bouillir un volume donné d'eau :
  \begin{equation*}
    V = \dfrac{\rho_\text{eau} c_\text{eau}}{E_V} V_\text{eau} \Delta T
  \end{equation*}

  \begin{donnees}
    \item $\rho_\text{eau} = \qty{1000}{\g\per\litre}$
    \item $c_\text{eau} = \qty{4,180}{\joule\per\g\per\degreeCelsius}$
  \end{donnees}
\end{doc}

\question{
  Rappeler la température d'ébullition de l'eau.
}{}{1}

\question{
  Calculer $\Delta T$ si on veut faire bouillir de l'eau initialement à \qty{20}{\degreeCelsius}.
}{}{2}

\question{
  Calculer le volume de gaz nécessaire pour faire bouillir 1 litre d'eau initialement à \qty{20}{\degreeCelsius}.
}{}{3}