%%%%
\teteSndChim
\numeroActivite{2}
\nomPrenomClasse

%%%% titre
\titreTP{Corrosion d'un métal avec de l'acide}

\begin{tableauCompetences}
  \centering REA &
  Réaliser un protocole. Suivre les règles de sécurités.
  & & & &
  \\
  %
  \centering VAL &
  Confronter un modèle à des résultats expérimentaux.
  & & & &
  %
\end{tableauCompetences}



%%%% Objectifs
\begin{objectifs}
  \item Comprendre ce qu'est un réactif limitant et savoir l'identifier.
\end{objectifs}

\begin{contexte}
  Quand on met du magnésium solide en contact avec de l'acide chlorhydrique, le magnésium et l'acide réagissent chimiquement, on parle de corrosion.
  
  \problematique{
    Pour quelles conditions initiales le magnésium solide va-t-il être complètement transformé chimiquement ?
  }
\end{contexte}
\bigskip


%%%% docs
\begin{doc}{Protocole expérimental}{doc:protocole_corrosion_magnesium}
  \begin{listePoints}
    \item Mettre des gants et des lunettes de protection.
    \item Découper une bande de magnésium de masse $m = \ldots\ldots \unit{g}$.
    \item Casser la bande en petit morceaux et les placer dans un tube à essais.
    \item Verser environ $10\unit{mL}$ d'acide chlorhydrique dans un bécher de $50 \unit{mL}$.
    \item Prélever $6,\!0 \unit{mL}$ d'acide chlorhydrique et les verser dans le tube à essais contenant le magnésium.
  \end{listePoints}
\end{doc}

%%
\mesure 
Réaliser le protocole du document~\ref{doc:protocole_corrosion_magnesium} et attendre la fin de la réaction chimique (plus d'effervescence).
Verser $\sim 1\unit{mL}$ de la solution obtenue dans 3 tubes à essais.


%%
\bigskip
\begin{doc}{Corrosion du magnésium avec un acide}{doc:corrosion_fer}
  \begin{equation*}
    \underset{\text{1 atome de magnésium}}{\chemfig{Mg}(s)}
    + \underset{\text{2 ions hydrogènes}}{2\chemfig{H^+}(aq)}
    \reaction
    \underset{\text{1 ion magnésium II}}{\chemfig{Mg^{2+}}(aq)}
    + \underset{\text{1 molécule de dihydrogène}}{\chemfig{H_2}(g)}
  \end{equation*}
  On vérifie bien qu'il y a le même nombre de charges positives, de magnésium \chemfig{Mg} et d'hydrogène \chemfig{H}, dans l'état initial et dans l'état final.
\end{doc}

%%
\question{
  Lister les réactifs et les produits pour la corrosion du magnésium par un acide, en indiquant leurs états physiques.
}{
  Réactifs : magnésium solide et ion hydrogène en solution. \\
  Produits : ion magnésium II en solution et dihydrogène gazeux.
}{2}


\newpage
\question{
  Indiquer s'il reste du magnésium solide à la fin de la réaction.
}{
  ...
}{1}

%%
\begin{doc}{Tests d'identifications}{doc:tests_identifications_corrosion}
  \begin{center}
    \begin{tblr}{
      colspec = {c c c}, hlines, vlines,
      row{1} = {couleurPrim!20}
    }
      Ion à tester & Solution utilisée & Résultat du test positif \\
      %
      Chlorure \chemfig{Cl^{-}} &
      Nitrate d'argent \chemfig{AgNO_3} &
      Précipité blanc \\
      %
      Hydrogène \chemfig{H^+} &
      Bleu de thymol &
      Couleur rouge-violette \\
      %
      Magnésium \chemfig{Mg^{2+}} &
      Solution d'hydroxyde de sodium &
      Précipité blanc \\
    \end{tblr}
  \end{center}
\end{doc}

\mesure
Réaliser chaque test d'identification du document~\ref{doc:tests_identifications_corrosion} : verser quelques gouttes de la solution utilisée dans \textbf{un} des tubes à essais préparés.

Noter les résultats des tests dans le tableau suivant:
\begin{center}
  \begin{tblr}{
    colspec = {c c c}, hlines, vlines,
    row{1} = {couleurPrim!20}
  }
    Ion testé & Résultat démonstration & Résultat élève \\
    %
    \hspace{200pt} & & \phantom{Résultat démonstration} \\
    %
    & & \\
    %
    & & \\
  \end{tblr}
\end{center}
\bigskip

\question{
  En vous aidant des tests d'identification, lister toutes les espèces chimiques présentes au début de la réaction et lister celles présentes à la fin de la réaction, dans votre cas comme dans celui de la démonstration.
}{
  ...
}{4}

\question{
  Quelle espèces chimiques ont été transformées au cours de la transformation chimique ?
}{
  ...
}{4}


%%
\question{
  Pouvez-vous valider avec ces tests d'identifications la réaction proposée dans le document~\ref{doc:corrosion_fer} ?
}{
  ...
}{3}


%%
\begin{doc}{Réactif limitant}{doc:A_}
  \begin{importants}
    Dans une réaction chimique, le \important{réactif limitant} est le réactif qui est totalement transformé, qui disparaît complètement.
    Il est dit \og \important{limitant} \fg, car il est responsable de l'arrêt de la réaction.
  \end{importants}
\end{doc}

\question{
  En vous aidant de vos observations, indiquer quel est le réactif limitant de la réaction de corrosion du magnésium avec un acide dans votre cas et dans celui de la démonstration. Justifier.
}{
  ...
}{4}

% Masse molaire HCl : 36,458 g/mol
% 1 mol/L -> 36,458 g/L = 0,036 g/mL
% Masse HCl dans 4 mL : 0,144 g  | dans 10 mL : 0,36 g
% Mole HCl dans 10 mL : 0,01 mol
% Masse molaire Mg : 24,30 g/mol
% Masse magnésium Mg : 0,04 g | 0,03 g | 0,05 g | 0,06 g
% Mole dans 0,05 g : 0,002 mol
%           0,08 g : 0,003 mol
%           0,10 g : 0,004 mol