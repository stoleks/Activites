%%%%
\sndEnTeteQuatre

%%%% titre
\vspace*{-36pt}
\numeroActivite{3}
\titreTP{Formation d'une image}


%%%% Objectifs
\begin{objectifs}
  \item Utiliser une lentille convergente pour former une image.
  \item Vérifier la modélisation d'une lentille convergente.
\end{objectifs}

\begin{contexte}
  Une lentille mince convergente est un objet en verre, plus épais au centre qu'au niveau de sa bordure.
  Ce système à la propriété de faire converger les rayons lumineux qui le traverse.
  
  \problematique{
    Comment utiliser une lentille convergente pour former l'image d'un objet ?
  }
\end{contexte}
\bigskip


%%%%
\begin{doc}{Rappel sur la détermination graphique d'une image}
  \label{doc:formation image}
  \vspace*{-24pt}
  
  \begin{encart}
    Une lentille convergente possède un \important{centre optique} O, un \important{foyer image} $F'$ et un \important{foyer objet} $F$.
  \end{encart}
  La droite perpendiculaire à la lentille passant par le centre optique $O$ est appelée \important{l'axe optique}.
  L'image d'un objet $AB$ est notée $A'B'$.
  
  \begin{center}
    \image{0.8}{images/lumière/image_lentille_convergente.png}
  \end{center}
  
  \begin{encart}
    Trois rayons ont des propriétés particulières pour une lentille convergente :
    \begin{listePoints}
      \item un rayon passant par le centre optique $O$ n'est pas dévié ;
      \item un rayon qui arrive parallèle à l'axe optique sort de la lentille en passant par le foyer image $F'$ ;
      \item un rayon qui arrive en passant par le foyer objet $F$ sort de la lentille parallèlement à l'axe optique.
    \end{listePoints}
  \end{encart}
\end{doc}

%%
\begin{doc}{Rappel sur le grandissement}
  \label{doc:grandissement}
  \vspace*{-24pt}
  \begin{encart}
    Le \important{grandissement} noté $\gamma$ (gamma) est le rapport entre la hauteur algébrique de l'image et celle de l'objet
    $\gamma = \Frac{\algebrique{A'B'}}{\algebrique{AB}}$
  \end{encart}
  Si $\gamma < 0$ l'image est renversée.
  Si $\gamma > 1$ l'image est plus grande que l'objet. 
  Si $\gamma < 1$ l'image est plus petite que l'objet.
  
  \begin{encart}
    En optique les longueurs sont \important{algébriques}, c'est-à-dire qu'elles sont positives ou négatives en fonction de leur sens, on les note avec une barre $\algebrique{AB}$.
  \end{encart}
  \exemple $\algebrique{AB} > 0$ si B est au dessus de A (ou si B est à droite de A) et $\algebrique{AB} < 0$ si B est en dessous de A (ou si B est à gauche de A).
\end{doc}
\bigskip


%%%%
\mesure
Fixer la bague représentant le \og d \fg\, à la lampe, placer l’écran le plus loin possible et placer la lentille convergente entre la lampe et l’écran.

Sans toucher à la lampe ou l'écran, trouver les deux positions de la lentille pour former l’image du \og d \fg\, la plus nette possible sur l’écran.
  
Choisir la position nette pour laquelle la lentille est proche de l’écran et \textbf{ne plus bouger la lentille}.

\question{
  \label{exo:grandissement_def}
  Mesurer avec le mètre la taille de l’image $\algebrique{A'B'}$ sur l’écran et la taille $\algebrique{AB}$ de l’objet (lettre d) sur la lampe.
  Calculer $\gamma$ à l'aide de ces mesures.
}{0}
\vspace*{-8pt}
\begin{equation*}
  \algebrique{A'B'} = \ldots \ldots \ldots
  \qq{}
  \algebrique{AB} = \ldots \ldots \ldots
  \qq{}
  \gamma = \ldots \ldots \ldots
\end{equation*}

\question{
  En utilisant le théorème de Thalès dans le document~\ref{doc:formation image}, montrer que 
  $\gamma = \Frac{\algebrique{OA'}}{\algebrique{OA}}$
}{3}

\question{
  \label{exo:grandissement_thales}
  Mesurer les distance $\algebrique{OA'}$ et $\algebrique{OA}$ et calculer de nouveau $\gamma$.
}{0}
\vspace*{-8pt}
\begin{equation*}
  \algebrique{OA'} = \ldots \ldots \ldots
  \qq{}
  \algebrique{OA} = \ldots \ldots \ldots
  \qq{}
  \gamma = \ldots \ldots \ldots
\end{equation*}

\question{
  En comparant les valeurs de $\gamma$ calculées question~\ref{exo:grandissement_def} et~\ref{exo:grandissement_thales}, est-ce que la modélisation proposée dans le document~\ref{doc:formation image} vous semble valide ?
}{2}
