%%%%
\sndEnTeteQuatre

%%%% titre
\vspace*{-36pt}
\numeroActivite{1}
\titreTP{Spectre d'émission}


%%%% Objectifs
\begin{objectifs}
  \item Comprendre la notion de spectre d'émission
\end{objectifs}

\begin{contexte}
  Il existe différentes sources lumineuse, comme le Soleil, les lampadaires, les néons, les écrans de téléphones, etc.
  
  \problematique{
    Comment caractériser la lumière émise par une source ?
  }
\end{contexte}

\vspace*{-12pt}
\titrePartie{Spectre d'émission}
\vspace*{-12pt}


%%%% docs
\begin{doc}{Spectre d'émission}
  La lumière est une onde électromagnétique, qui peut avoir plusieurs longueurs d'ondes.
  Nos yeux captent certaines longueurs d'ondes et y associent une couleur : c'est le domaine visible.
  
  \begin{encart}
    La donnée de toutes les longueurs d'ondes présentes dans une source lumineuse s'appelle le \important{spectre d'émission}.
    Le spectre dans le domaine visible est représenté de la manière suivante :
  \end{encart}
  \vspace{-22pt}
  \begin{center}
    \image{0.5}{images/lumière/spectre_lumineux.jpg}
  \end{center}
\end{doc}

%%
\vspace*{-12pt}
\titreSection{Les spectre d'émissions continus}
\vspace*{-12pt}

\begin{doc}{Spectre continu}
  \vspace*{-24pt}
  \begin{encart}
    Un spectre d'émission continu présente une suite de raies colorées.
    Un spectre continu prend la forme d'une bande colorée unique.
  \end{encart}
\end{doc}
  
\textbf{Influence de la température :} en augmentant la tension d'alimentation d'une lampe à incandescence, on augmente sa température.

\question{
  Alimenter la lampe avec le générateur.
  Celui-ci propose deux tensions : $6\unit{V}$ ou $12\unit{V}$ : essayer les deux cas en les observant au spectroscope.
  Quelle différence remarquez-vous ? 
  Schématisez vos observations ci-dessous :
}{0}

\separationDeuxBlocs{
  \centering 
  Lampe sous-alimentée ($6 \unit{V}$)
  
  \image{1}{images/lumière/spectre_lumineux_graduee.jpg}
}{
  \centering
  Lampe correctement alimentée ($12 \unit{V}$)
  
  \image{1}{images/lumière/spectre_lumineux_graduee.jpg}
}

\begin{doc}{Émission d'un corps chaud}
  \vspace*{-24pt}
  \begin{encart}
    Un corps chaud émet un rayonnement lumineux de spectre \reponseLigne
    \lignePointillee{0.3}
    Les propriétés du rayonnement lumineux dépendent \\[8pt] \lignePointillee{0.5} \\[8pt]
    Plus la température s'élève et plus le spectre s'enrichit en couleur \reponseLigne
    \lignePointillee{0.3}, donc avec des \lignePointillee{0.2} longueurs d'onde.
  \end{encart}
\end{doc}

\question{
  Utilisons ce résultat pour estimer la température de surface d'une étoile.
  Bételgeuse est une étoile de couleur rouge-orange, sa température de surface vaut $3800 \degreCelsius$. 
  L’étoile Rigel est de couleur bleue. Sa température sera-t-elle plus élevée ou plus faible ? 
}{2}


%%
\titreSection{Les spectres d’émission de raies}

\question{
  En utilisant le spectroscope, observer le spectre de la lumière émise par les deux lampes disponibles : celle à vapeur de mercure et celle à vapeur de sodium.
  Schématiser vos observations ci-dessous, en faisant apparaître les espaces noirs :
}{0}

\separationDeuxBlocs{
  \centering 
  Lampe à vapeur de mercure \chemfig{Hg}
  
  \image{1}{images/lumière/spectre_lumineux_graduee.jpg}
}{
  \centering
  Lampe à vapeur de sodium \chemfig{Na}
  
  \image{1}{images/lumière/spectre_lumineux_graduee.jpg}
}

\begin{doc}{Émission atomique ou moléculaire}
  \vspace{-24pt}
  \begin{encart}
    Lorsque les entités chimiques (atomes, ions, molécules), qui composent un gaz \\[8pt]
    sont excitées, elles émettent seulement \reponseLigne
    \reponse{1} % des radiations avec des longueurs d'ondes précises
    
    Cela correspond à \dotfill %des raies bien définies
  \end{encart}
    
  Chaque élément chimique possède son propre \important{spectre d'émission} caractérisé par quelques longueurs d'ondes précises.
\end{doc}