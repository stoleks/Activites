%%%%
\sndEnTeteTrois

%%%% titre
\vspace*{-20pt}
\numeroActivite{8}
\titreActivite{En quête de stabilité : formation des molécules}


%%%% Objectifs
\begin{objectifs}
  \item Comprendre la liaison covalente et les notions de doublet liant et non-liant.
  \item Comprendre que la stabilité d'une molécule est liée à la règle du duet et de l'octet (couche externe complète).
  \item Savoir analyser un schéma de Lewis pour expliquer la stabilité d'une molécule.
\end{objectifs}

\begin{contexte}
  En dehors des gaz nobles de la 18$^\text{ème}$ colonne du tableau périodique (\chemfig{He}, \chemfig{Ne}, \chemfig{Ar}, \chemfig{Kr}, etc.), les atomes ont tendance à s'associer spontanément pour former des molécules. 
  %Comme pour la formation des ions, les atomes gagnent en stabilité en complétant leur couche externe, en respectant la règle du duet ou de l'octet.
  
  \problematique{
    Quelles règles régissent la formation des molécules ?
  }
\end{contexte}


%%%% docs
\titreSection{Le modèle de Lewis}

\begin{doc}{Doublet liant}
  Les atomes ont tendance à s'associer en molécule, afin de gagner en stabilité en complétant leur couches électronique externe.
  
  En 1916, Lewis propose un modèle simple pour schématiser la formation des liaisons entre atomes.
  Dans ce modèle, les atomes qui s'associent en molécule vont mettre en commun un des électrons de leur couche externe.
  Ces électrons mis en commun forment une paire appelée \og \important{doublet liant} \fg.
  
  En partageant leurs électrons les atomes deviennent liés, on parle de \important{liaison covalente.}
  
  \exemple Formation de la molécule de dihydrogène \chemfig{H_2} à partir de deux atomes \isotope{}{1}{H} :
  \vspace{8pt}
  
  \separationDeuxBlocs{
    \begin{center}
      {\small Schéma des deux atomes d'hydrogènes}
      \vspace{8pt}
      
      \image{0.5}{images/molecules/molecule_H2}
    \end{center}
  }{
    \begin{center}
      {\small Schéma de Lewis de la liaison}
      \vspace{8pt}
      
      \image{0.4}{images/molecules/Lewis_H2}
    \end{center}
  }
\end{doc}

\newpage
\begin{doc}{Doublet non-liant}
  Les électrons de la couche externe sont appelés \important{électrons de valence.}
  Lors de la formation d'une molécule, les électrons de valence qui ne sont pas partagés forment des paires appelées \og \important{doublet non-liant} \fg.
  
  \exemple Formation de la molécule d'eau \chemfig{H_2 O} à partir de 2 atomes \isotope{}{1}{H} et d'un atome \isotope{}{8}{O} :
  \vspace{8pt}
  
  \separationDeuxBlocs{
    \begin{center}
      {\small Schéma des trois atomes associés}
      \vspace{8pt}
      
      \image{1}{images/molecules/molecule_H2O}
    \end{center}
  }{
    \begin{center}
      {\small Schéma de Lewis des doublets liants et des doublets non-liants (en noir)}
      \vspace{20pt}
      
      \image{0.4}{images/molecules/Lewis_H2O}
    \end{center}
  }
\end{doc}

\begin{doc}{Liaisons multiples}
  Pour être stables, les atomes peuvent partager plusieurs paires d’électrons et ainsi créer une liaison multiple.
  Celle-ci peut être double, comme dans le cas du dioxygène ; ou triple comme dans le cas du diazote.
  
  \centering
  \separationDeuxBlocs{
    \begin{center}
      {\small Schéma de Lewis}
      \vspace{36pt}
      
      \image{0.4}{images/molecules/Lewis_O2}
    \end{center}
  }{
    \begin{center}
      {\small Schéma des deux atomes d'oxygène}
      \vspace{8pt}
      
      \image{1}{images/molecules/molecule_O2}
    \end{center}
  }
  \vspace{12pt}
  
  \separationDeuxBlocs{
    \begin{center}
      {\small Schéma des deux atomes d'azote}
      \vspace{8pt}
      
      \image{1}{images/molecules/molecule_N2}
    \end{center}
  }{
    \begin{center}
      {\small Schéma de Lewis}
      \vspace{36pt}
      
      \image{0.4}{images/molecules/Lewis_N2}
    \end{center}
  }
\end{doc}


%%%% questions
\newpage
\question{
  Rappeler la configuration électronique de l’hydrogène \isotope{}{1}{H}, du carbone \isotope{}{6}{C}, de l’azote \isotope{}{7}{N} et de l'oxygène \isotope{}{8}{O}.
  Identifier pour chacun de ces atomes leurs électrons de valence.
}{5}

\question{
  Donner le nombre d'électrons manquant à chaque atome pour que leur couche externe soit pleine.
  Comment peuvent-ils gagner en stabilité ?
}{3}

\question{
  Quelle molécule stable peut-on former à partir d'un atome de carbone et plusieurs atomes d'hydrogènes ?
}{2}

\question{
  Même question pour un atome d'azote et plusieurs atomes d'hydrogènes.
}{2}

% \question{
%   Construire ces molécules à partir des modèles moléculaires.
% }{0}

\begin{encart}
  Pour gagner en stabilité, les atomes peuvent partager les électrons de leur couche \\[8pt]
  externe en créant \dotfill \\[8pt] % des liaisons covalentes
  De cette manière, les atomes \dotfill \\[8pt]
  \reponse{1} %complètent leur couche externe et sont donc stables.
  
  Pour savoir combien de liaisons un atome peut former, il suffit de \dotfill \\[8pt] % compter le nombre d'électrons de valence et le nombre d'électrons manquant pour qu'elle soit complète
  \reponse{1}
\end{encart}

\feuilleBlanche