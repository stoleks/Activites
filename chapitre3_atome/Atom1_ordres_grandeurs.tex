%%%%
\sndEnTeteTrois

%%%% titre
\numeroActivite{1}
\titreActivite{Ordre de grandeur}


%%%%
\vspace*{-24pt}
\titreSection{Notation scientifique}

%%
\begin{doc}{Les puissances de 10}
  \vspace*{-24pt}
  \begin{encart}
  \begin{listePoints}
    \item Écrire le nombre $10^n$ (avec $n = 0, 1, 2, 3, \ldots$), revient à écrire ``$1$'' suivi de $n = 0, 1, 2, 3, \ldots$ zéros. \textit{Exemple : $10^3 = 1000$}
    \item Écrire le nombre $10^{-n}$ (avec $n = 1, 2, 3, \ldots$), revient à écrire ``$0,$'' suivi de $n - 1 = 0, 1, 2, \ldots$ zéros et d'un $1$. \textit{Exemple : $10^{-2} = 0,\!01$}
    \item $10^a \times 10^b = 10^{a + b}$
    \item $\Frac{1}{10^n} 
    = \Frac{10^{-n}}{10^{-n}} \times \frac{1}{10^n} 
    = \Frac{10^{-n}}{10^{n - n}}
    = \Frac{10^{-n}}{10^0}
    = 10^{-n}$
  \end{listePoints}
  \end{encart}
\end{doc}
\bigskip

% \begin{doc}{Moyen mnémotechnique}
%   \vspace*{-12pt}
%   \begin{listePoints}
%     \item Si je décale la virgule de 1 rang vers la gauche, alors \dotfill \\
%     de 1 unité la puissance de dix.
%     \item Si je décale la virgule de 1 rang vers la droite, alors \dotfill \\
%     de 1 unité la puissance de dix.
%   \end{listePoints}
% \end{doc}

\begin{doc}{La notation scientifique}
  \vspace*{-24pt}
  \begin{encart}
  La notation scientifique d'une quantité se présente de la façon suivante :
  \begin{equation*}
    \texteEncadre{chiffre différent de zéro}
    \;,\;
    \texteEncadre{autres chiffres} 
    \; \vphantom{\frac{1}{10}}^{\times} \;
    \texteEncadre{puissance de dix}
    \;
    \texteEncadre{\important{unité}}
  \end{equation*}
  \end{encart}
\end{doc}

\exo{
  Écrire les quantités suivantes en notation scientifique :
}{0}
\separationDeuxBlocs{
  $288 \unit{h} = \dotfill$ \\[4pt]
  $1 \unit{m} = \dotfill$ \\[4pt]
  $756\, 864\, 000 \unit{s} = \dotfill$ \\[4pt]
  $638 \unit{N} = \dotfill$
}{
  $0,\!1 = \dotfill$ \\[4pt]
  $0,\!9997 \unit{g/mL} = \dotfill$ \\[4pt]
  $0,\!436 \unit{s} = \dotfill$ \\[4pt]
  $0,\!336 \unit{s} = \dotfill$
}


%%
\titreSection{\href{https://www.youtube.com/watch?v=xTV47tuv_Fg}{Les ordres de grandeurs}}

\begin{doc}{Définitions}
  \vspace*{-24pt}
  \begin{encart}
  L'ordre de grandeur d'un nombre est la puissance de 10 la plus proche de ce nombre.
  \end{encart}
\end{doc}

\exo{
  Donner l'ordre de grandeur des quantités suivantes :
}{0}
\separationDeuxBlocs{
  $3,\!00 \cdot 10^8 \unit{m.s^{-1}} = \dotfill$ \\[4pt]
  $1,\!67 \cdot 10^{-27} \unit{kg} = \dotfill$
}{
  $9,\!11 \cdot 10^{-31} \unit{kg} = \dotfill$ \\[4pt]
  $53 \cdot 10^{-12} \unit{m} = \dotfill$
}


%%%%
\newpage
\vspace*{-36pt}
\titreSection{Le système international de mesure}

%%
\vspace*{-12pt}
\titreSousSection{Le système international}

Pour comparer des grandeurs entre elles, il faut les exprimer avec les \important{mêmes unités de mesures}. % exemple centime et euros

Pour pouvoir communiquer facilement d'un pays à un autre, le \important{système international (SI)} a été développé par la Conférence Générale des Poids et Mesures (CGPM). % histoire des sciences système métrique

Le système international est composé de \important{sept unités de base,} que l'on retrouve quotidiennement. Une part importante de nos technologies modernes dépendent de la précision avec laquelle ces unités sont définies.

\begin{center}
  \begin{tabular}{| c | c | c |}
    \hline
    \rowcolor{gray!20}
    Grandeur             & Unité      & Symbole de l'unité
    \\ \hline
    Masse                & kilogramme & kg
    \\ \hline
    Temps                & seconde    & s
    \\ \hline
    Longueur             & mètre      & m
    \\ \hline
    Température          & kelvin     & K
    \\ \hline
    Quantité de matière  & mole       & mol
    \\ \hline
    Intensité électrique & ampère     & A
    \\ \hline
    Intensité lumineuse  & candela    & cd 
    \\ \hline
  \end{tabular}
\end{center}


%%
\titreSousSection{De l’échelle microscopique à l’échelle astronomique}

\exo{
  Compléter le tableau en associant à chaque objet sa longueur, puis l'ordre de grandeur de cette longueur. Pour ça, utilisez six de ces huit longueurs (attention aux unités !) :
  \begin{equation*}
    10^{16} \unit{m} \quad
    6400 \unit{km} \quad
    10^{20}\unit{m} \quad
    0,\!1\unit{nm} \quad
    60\unit{\mu m} \quad
    6 \unit{mm} \quad
    1000 \unit{km} \quad
    10^{12} \unit{m}
  \end{equation*}
}{0}

\vspace*{-24pt}
\begin{center}
  \begin{tabularx}{\linewidth}
  {| c | m{0.1025\linewidth} | m{0.105\linewidth}| m{0.1025\linewidth}
   | m{0.105\linewidth} | m{0.105\linewidth} | m{0.105\linewidth} |}
    \hline 
    Objet & 
    Épaisseur cheveux &
    Rayon Voie Lactée &
    Rayon système solaire &
    France métropolitaine &
    Fourmi &
    Atome
    \\ \hline
    Image & 
    \vphantom{b}\vspace*{-16pt} \image{1}{images/taille_objet/taille_cheveux} &
    \image{1}{images/taille_objet/taille_galaxie} &
    \image{1}{images/taille_objet/taille_systeme_solaire} &
    \image{1}{images/taille_objet/taille_france} &
    \image{1}{images/taille_objet/taille_fourmi} &
    \image{1}{images/taille_objet/taille_atome}
    \\ \hline
    Taille & \vphantom{$\Frac{1}{1}$} & & & & &
    \\ \hline
    Ordre de grandeur & \vphantom{$\Frac{1}{1}$} & & & & &
    \\ \hline
  \end{tabularx}
\end{center}