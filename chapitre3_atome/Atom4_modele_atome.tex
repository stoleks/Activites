%%%%
\sndEnTeteTrois

%%%% titre
\vspace*{-36pt}
\numeroActivite{4}
\titreActivite{Le modèle de l'atome}


%%%% Objectifs
\begin{objectifs}
  \item Utiliser la méthode scientifique pour comprendre l'évolution d'un modèle.
\end{objectifs}

\begin{contexte}
  La description de la matière a considérablement évolué au cours des 3 derniers millénaires.
  \`A partir du \siecle{19} une séries d'observations expérimentales ont permis d'affiner le modèle de l'atome.
  
  \problematique{
    Comment la communauté scientifique a établi le modèle de l'atome moderne ?
  }
\end{contexte}


%%%% Documents
\begin{doc}{La méthode scientifique}
  Pour expliquer le monde dans lequel nous vivons, en science on fait appel à des \important{modèles.} 
  Les modèles permettent de décrire un phénomène, ce sont donc des \textbf{image} de la réalité.

  Pour valider ou améliorer la description d'un phénomène par un modèle, les scientifiques s'appuient sur la \important{méthode scientifique} :
  \begin{enumeration}
    \item Observation d'un phénomène et formulation d'une problématique.\competence{RCO, APP}
    \item Proposition d'hypothèses, choix d'un modèle de description.\competence{ANA/RAI}
    \item Tests expérimentaux des hypothèses et du modèle.\competence{REA}
    \item Analyse des résultats.\competence{VAL}
    \item Communication des résultats.\competence{COM}
  \end{enumeration}

  \flecheLongue Il faut changer de modèle si une observation expérimentale le contredit.
\end{doc}


\begin{doc}{Quelques observations expérimentales}
  \label{doc:observations_exp_atome}
  \vspace*{-20pt}
  \begin{listePoints}
    \item \textbf{1783 :} Lavoisier observe que lors d'une réaction chimique il n'y a pas de perte de matière \og Rien ne se perd, rien ne se crée, tout se transforme \fg.
    Il décompose l'eau en deux composants qu'il nomme l'oxygène et l'hydrogène. 
    %L'hydrogène vient du grec \og \textit{hydro} \fg\; (eau) et \og \textit{gene} \fg\; (engendrer).
    \item \textbf{1897 :} Thomson observe que l’on peut arracher des particules de charges négatives d’un atome.
    Il nomme ces particules \textit{électrons}.
    \item \textbf{1900 :} Planck observe que les échanges d'énergies entre lumière et matière sont \textit{quantifiés}.
    C'est-à-dire que les échanges n'ont lieu que si la lumière a certaines énergies bien précises.
    \item \textbf{1911 :} Rutherford observe que l'atome possède un noyau très petit devant la taille d’un atome, avec une charge positive.
    Il nomme les particules de charges positives composant le noyau \textit{protons}.
    \item \textbf{1927 :} Davisson et Germer observent que les électrons sont délocalisés dans un \textit{cortège électronique}.
  \end{listePoints}
\end{doc}

\begin{doc}{Quelques modèles de l'atome}
  \label{doc:modeles_atomes}
  \vspace*{18pt}
  \separationDeuxBlocs{
    \centering
    \image{0.48}{images/atome/modele_sphere_dure} \\
    A : Sphère dure pleine et indivisible.
    \vspace*{20pt}

    \image{0.48}{images/atome/modele_bohr} \\
    C : Noyau positif avec des électrons négatifs qui orbitent autour.
    Les orbites sont à des distances bien définies et on les appelle couches, avec du vide entre deux couches.
  }{
    \centering
    \image{0.48}{images/atome/modele_orbite} \\
    B : Noyau positif avec des électrons négatifs qui orbitent autour.
    
    \image{0.48}{images/atome/modele_plum_pudding} \\
    D : Atome neutre avec des électrons négatifs qui baignent dans un volume chargés positivement.
  }
  
  \centering
  \image{0.24}{images/atome/modele_quantique} \\
  E : Noyau positif avec un cortège électronique organisé en couches appelées orbitale.
  Les électrons sont \textit{délocalisés} dans ces couches : tout se passe comme si les électrons étaient à plusieurs endroits en même temps.
\end{doc}


%%%%
\exo{
  À l'aide des documents~\ref{doc:observations_exp_atome} et~\ref{doc:modeles_atomes}, associer à chaque modèle une observation qui le contredit, si cette observation existe.
}{0}

\exo{
  Réaliser une frise chronologique sur laquelle apparaît chaque modèle de l'atome, en utilisant les dates des observations expérimentales.
}{0}