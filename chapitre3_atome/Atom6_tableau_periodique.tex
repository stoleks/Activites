%%%%
\sndEnTeteTrois

%%%% titre
\vspace*{-36pt}
\numeroActivite{6}
\titreActivite{Le Tableau périodique}


%%%% Objectifs
\vspace*{-8pt}
\begin{objectifs}
  \item Comprendre la construction du tableau périodique.
\end{objectifs}

\begin{contexte}
  Le tableau périodique des éléments, également appelé classification périodique des éléments ou simplement tableau périodique, représente tous les éléments chimiques découverts à ce jour.
  
 C'est le chimiste russe Dmitri Mendeleïev qui créa le tableau périodique moderne en 1869, en proposant de classer les éléments par numéro atomique croissant.

  \problematique{
    Comment construire le tableau périodique à partir des configurations électroniques des éléments ?
  }
\end{contexte}


%%%% question
\question{
  Compléter chaque carte en lui associant son élément et en indiquant sa configuration électronique.
}{0}

\question{
  Séparer les éléments dont la couche externe est de type s et les éléments dont la couche externe est de type p.
}{0}

\question{
  En utilisant les configurations électronique, construire le tableau périodique des éléments en formant un \og bloc s\fg\, et un \og bloc p \fg, en classant les éléments par numéro atomique croissant.
}{0} 

\question{
  Une ligne du tableau s'appelle une période.
  Quel est le point commun entre tous les éléments d'une même période ?
%}{2}
}{0}
\correction{
  Tous les atomes d'une même période ont la même couche externe, avec le même nombre d'électron sur leurs couches internes.
}

\question{
  Une colonne du tableau s'appelle une famille.
  Quel est le point commun entre tous les éléments d'une même famille ? (à l'exception de l'Hélium)
%}{3}
}{0}
\correction{
  Tous les atomes d'une même famille ont le même nombre d'électrons sur leur couche externe.
  Les atomes d'une même famille auront tendance à former des molécules avec le même nombre de liaisons et des ions avec le même nombre de charges.
}

\begin{encart}
  Quelques familles à connaître : 
  \begin{listePoints}
    \item Première colonne (sauf hydrogène) : 
    %\dotfill \vspace{4pt}
    les \important{alcalins}.
    \item Avant-dernière colonne : 
    les \important{halogènes}.
    %\dotfill \vspace{4pt}
    \item Dernière colonne : 
    les \important{gaz nobles}.
    %\dotfill \vspace{4pt}
  \end{listePoints}
\end{encart}

%\feuilleBlanche