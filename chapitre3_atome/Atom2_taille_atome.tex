%%%%
\sndEnTeteTrois

%%%% titre
\numeroActivite{2}
\titreActivite{Taille d'un atome}


%%%% evaluation
\begin{tableauCompetences}
  \centering APP &
  Extraire une information.
  & & & &
  \\ \hline
  \centering REA &
  Utiliser les puissances de 10 et les ordres de grandeurs.
  & & & &
  \\ \hline
  \centering COM &
  Travailler en groupe en se répartissant le travail.
  & & & &
\end{tableauCompetences}
\smallskip


%%%% Objectifs
\begin{contexte}
  La matière est constituée d'objets très petits, comme les atomes.
  Visualiser la taille réelle d'un atome et la répartition de sa masse dans le volume qu'il occupe est une tâche difficile.

  \problematique{
    On va utiliser les ordres de grandeurs pour mieux appréhender les caractéristiques d'un atome, en utilisant des objets du quotidien.
  }
\end{contexte}
\smallskip


%%%% Documents
\begin{doc}  {Extrait de \textit{La vie à fil tendu} de Georges \textsc{Charpak} (1924-2010, prix Nobel de physique 1992)}
  Lorsque j'entrai au laboratoire dirigé par Joliot au Collège de France, la connaissance que j'avais de la structure de la matière ne devait guère dépasser celle  acquise par un lycéen de 1993 abonné à de bonnes revues de vulgarisation.
  Je les résume rapidement : la matière est composée de molécules, elles-mêmes constituées d'atomes, eux-mêmes constitués de noyaux entourés d'un cortège d'électrons.
  Les noyaux portent une charge électrique positive qui est de même valeur et de signe opposé à la charge des électrons qui gravitent autour du noyau.
  \bigskip   

  Le noyau de l'hydrogène ne contient qu’un seul proton et un seul neutron.
  Le proton porte une charge électrique positive, c'est la charge électrique élémentaire notée \og e \fg ; le neutron, quant à lui, est neutre électriquement et a sensiblement la même masse.
  Tous deux s'associent de façon très compacte pour constituer les noyaux qui sont au coeur des atomes peuplant notre univers.
  Ils s'entourent d'un cortège d'électrons dont la charge compense exactement celle des protons.
  En effet, la matière est neutre, sinon elle exploserait en raison de la répulsion qu'exercent l'une sur l'autre des charges de même signe, positif ou négatif.
  \bigskip   
             
  Il faut avoir en tête l'échelle des dimensions.
  Le diamètre d'un atome est voisin d'un centième de millionième de centimètre.
  Celui d'un noyau est cent mille fois plus petit.
  On voit donc que presque toute la masse d'un atome est concentrée en un noyau central et que, loin sur la périphérie, se trouve un cortège qui est fait de particules de charge électrique négative, les électrons.
  C'est ce cortège seul qui gouverne le contact des atomes entre eux et donc tous les phénomènes perceptibles de notre vie quotidienne.
\end{doc}    

%%
\begin{doc}{Propriétés des constituants d'un atome}
 \label{doc:propriete_atome}
  \vspace*{-24pt}
  \begin{encart}
    \begin{center}
      \begin{tabular}{| c | c | c | c |}
        \hline
        %
        \rowcolor{gray!20}
        & Proton & Neutron & Électron
        \\ \hline
        % 
        Nombre dans un atome &
        Z & A - Z & Z
        \\ \hline
        %
        Charge &
        Positive $(+ e)$ & &
        \\ \hline
        %
        Masse &
        $1,\!67 \times 10^{-27} \unit{kg}$ &
        $1,\!67 \times 10^{-27} \unit{kg}$ &
        $9,\!11 \times 10^{-31} \unit{kg}$
        \\ \hline
      \end{tabular}
    \end{center}
  \end{encart}
\end{doc}


%%%%
\exo{
  Compléter la ligne \og charge \fg\; du tableau du document~\ref{doc:propriete_atome}.
}{0}

\exo{
  De quoi est constitué un atome ?
}{2}

\exo{
  Un éléphant d'Asie a en moyenne une masse de $4000 \unit{kg}$. Quelle est l'ordre de grandeur de sa masse ?
}{1}

\exo{
  Si un atome d'hydrogène avait la masse d'un éléphant, quelle serait la masse d'un électron en ordre de grandeur ? Quel animal pourrait avoir cette masse ?
}{3}

\exo{
  Quelle est le diamètre d'un atome et de son noyau ? Exprimer ces distances en mètre à l'aide des puissances de 10.
}{3}

\exo{
  Si le diamètre d'un noyau était égal à la taille d'une fourmi de $1 \unit{mm}$, quelle serait la taille en mètre du diamètre d'un atome ?
}{3}