%%%%
\sndEnTeteSix

%%%% titre
\vspace*{-40pt}
\numeroActivite{2}
\titreTP{Corrosion d'un métal par de l'acide}

%%%% Objectifs
\begin{objectifs}
  \item Comprendre la notion d'avancement d'une réaction.
  \item Comprendre ce qu'est un réactif limitant.
  \item Réaliser un protocole en respectant les consignes de sécurités.
\end{objectifs}

\begin{contexte}
  Quand on met un métal comme le magnésium en contact avec de l'acide chlorhydrique, le métal et l'acide réagissent chimiquement pour former du dihydrogène et des ions magnésium II \chemfig{Mg^{2+}}.
  
  \problematique{
    Pourquoi tout le magnésium solide n'est-il pas transformé en ions magnésium II ?
  }
\end{contexte}


%%%% docs
\begin{doc}{Réactif limitant}
  \vspace*{-22pt}
  \begin{encart}
    Dans une réaction chimique, le \important{réactif limitant} est le réactif qui est totalement transformé, qui disparaît complètement.
    Il est dit \og \important{limitant} \fg, car il est responsable de l'arrêt de la réaction.
  \end{encart}
\end{doc}


%%%% Questions
\question{
  ...
}{
  ...
}{1}


%%
\begin{doc}{Corrosion du magnésium par un acide}
  \label{doc:corrosion_fer}
  \vspace*{-14pt}
  \begin{equation*}
    \underset{\text{1 atome de magnésium}}{\chemfig{Mg}(s)}
    + \underset{\text{2 ions hydrogènes}}{2\chemfig{H^+}\phantom{(}}
    \reaction
    \underset{\text{1 ion magnésium II}}{\chemfig{Mg^{2+}}\phantom{(}}
    + \underset{\text{1 molécule de dihydrogène}}{\chemfig{H_2}(g)}
  \end{equation*}
  On vérifie bien qu'il y a le même nombre de charges positives, de magnésium \chemfig{Mg} et d'hydrogène \chemfig{H}, dans l'état initial et dans l'état final.
\end{doc}

%%
\question{
  Lister les réactifs et les produits pour la corrosion du magnésium par un acide, en indiquant leurs état physique.
}{
  Réactifs : fer solide et ion hydrogène.
  
  Produits : ion fer et dihydrogène.
}{2}