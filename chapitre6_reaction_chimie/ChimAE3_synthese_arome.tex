%%%%
\sndEnTeteSix

%%%% titre
\vspace*{-40pt}
\numeroActivite{3}
\titreTP{Synthèse de l'éthanoate d'isoamyle}

%%%% Objectifs
\begin{objectifs}
  \item Synthétiser une molécule naturelle.
  \item Réaliser un protocole en respectant les consignes de sécurités.
\end{objectifs}

\begin{contexte}
  Les arômes des aliments sont liées à des molécules captées par notre nez, auquel notre cerveau associe une odeur.
  
  \problematique{
    Comment synthétiser une molécule naturelle responsable de l'arôme de banane ?
  }
\end{contexte}


%%%% docs
\begin{doc}{Protocole de synthèse}
  \label{doc:protocole_synthese_banane}
  Dans le ballon, introduire :
  \begin{listePoints}
    \item 10 mL d'alcool isoamylique ;
    \item 15 mL d'acide éthanoïque ;
    \item 1 mL d'acide sulfurique. \attention \textbf{opération réalisée par l'enseignant} \attention
  \end{listePoints}
  
  Fixer le ballon au montage à reflux et lancer la circulation d'eau.
  Porter le mélange réactionnel à ébullition et chauffer à reflux pendant 30 minutes.
  Descendre le chauffe-ballon et laisser refroidir le ballon à l'air.
\end{doc}

\mesure Réaliser le protocole de synthèse du document~\ref{doc:protocole_synthese_banane}.

\question{
  En utilisant le nom des molécules, écrire la réaction chimique modélisant la synthèse de l'éthanoate d'isoamyle.
}{
  alcool isoamylique + acide éthanoïque \reaction éthanoate d'isoamyle + eau
}{1}


%%
\begin{doc}{Synthèse de l'éthanoate d'isoamyle}
  \begin{tabularx}{\textwidth}{| X | C{0.15} | C{0.15} | C{0.15} | C{0.15} |}
    \hline
    %
    Nom & Acide éthanoïque & Alcool isoamylique & Éthanoate d'isoamyle & Acide sulfurique
    \\ \hline
    %
    Formule & \chemfig{C_2H_4O_2} & \chemfig{C_5H_{12}O} & \chemfig{C_7H_{14}O_2} & (\chemfig{2H^+ ;\; SO_4^{2-}})
    \\ \hline
    %
    Masse volumique & $1,\!05$ g/mL & $0,\!81$ g/mL & $0,\!87$ g/mL & $1,\!83$ g/mL
    \\ \hline
    %
    Solubilité dans l'eau & Grande & Moyenne & Faible & Grande
    \\ \hline
    %
    Solubilité dans l'eau salée & Grande & Très faible & Très faible & Grande
    \\ \hline
    %
    Danger &
    \image{0.4}{images/pictogrammes/picto_flambe}~\image{0.4}{images/pictogrammes/picto_ronge} & 
    \image{0.41}{images/pictogrammes/picto_flambe}~\image{0.41}{images/pictogrammes/picto_altere} &
    \image{0.48}{images/pictogrammes/picto_flambe} &
    \image{0.48}{images/pictogrammes/picto_ronge}
    \\ \hline
  \end{tabularx}
\end{doc}


%%
\begin{doc}{Récupération du produit d'intérêt}
  Après réalisation de la synthèse, procéder à un relargage : introduire dans le ballon 25 mL d'eau salée saturée.
  Verser le mélange réactionnel dans l'ampoule à décanter.
  Agiter, puis laisser décanter.
  Éliminer la phase aqueuse dans un bécher, recueillir alors la phase organique dans un tube fermé.
\end{doc}


%%%% Questions
\question{
  ...
}{
  ...
}{1}