\tetePremStssAlim
\vspace*{-24pt}
\titreActivite{Produire de la nourriture à grande échelle}

\begin{contexte}
  Les sols sur lesquels poussent des végétaux sont constitués de matières biologiques décomposées (le humus) et de matières minérales (argile, calcaire, sable, etc.).
  L'association du humus et de l'argile forme un \important{complexe argilo-humique} (noté CAH), qui va retenir les cations (\ionPotassium, \ionSodium, \ammonium, etc.), ce qui va permettre au plantes de prospérer.

  \problematique{
    Comment exploiter les sols agraires tout en préservant le complexe argilo-humique ?
  }
\end{contexte}

\begin{doc}{Le métabolisme des plantes}
  Pour pouvoir pousser, les plantes ont besoin de produire un grand nombre de cellules, elles mêmes constituées d'un grand nombre de biomolécules.

  \begin{importants}
    Les parois des cellule végétales sont composées de \important{celluloses,} un polymère du glucose.
    Leurs membranes sont composées de \important{phospholipides.} 
    Leurs cytoplasmes contiennent des \important{protéines,} constitués d'acides aminés, et des organites constitués de protéines et \important{d'acides ribonucléiques (ARN),} comme les ribosomes.
    Enfin, leurs noyaux sont composées \important{d'acide désoxyribonucléique (ADN).}
  \end{importants}

  Pour synthétiser ces biomolécules, les plantes ont besoin de récupérer certains éléments chimiques dans le sol et dans l'air.
  Principalement du carbone \carbone, de l'oxygène \oxygene, de l'hydrogène \hydrogene, du phosphore \phosphore et de l'azote \azote.

  \begin{listePoints}
    \item Pour récupérer du carbone, de l'oxygène et de l'azote, les plantes synthétisent du glucose grâce à la \important{photosynthèse} qui a lieue dans les \important{chloroplastes.}
    Les éléments chimiques sont directement récupérés dans l'atmosphère (\dioxydeDeCarbone, \eau) ou dans le sol (\eau).
  
    \item Pour récupérer du phosphore les plantes utilisent les ions phosphates \ionPhosphate qui sont présents dans le sol.
  
    \item Pour récupérer de l'azote, les plantes utilisent les ions nitrates \nitrate, nitrites \nitrite ou ammonium \ammonium qui sont présents dans le sol.
  \end{listePoints}
  Les plantes ont aussi besoin d'ions potassium \ionPotassium en grande quantité pour réguler un nombre important de leur fonctions internes et pour être en bonne santé.
\end{doc}

\begin{doc}{Le rôle du complexe argilo-humique pour les plantes}
  Quand de la matière organique (comme des déchets végétaux, des selles ou des cadavres d'animaux), se décomposent dans le sol, un grand nombre de cations (\ionPotassium, \ionSodium, \ammonium, etc.) et d'anions (\ionPhosphate, \nitrate, \nitrite, \sulfate, etc.) sont libérés.

  Le complexe argilo-humique est composé de granulats chargés négativement, formés en partie par les vers de terre, ce qui va attirer les cations qui passent à proximité et les stocker dans le sol. 
  Ces cations vont à leur tour attirer les anions environnants, mais plus faiblement, et les stocker.

  \begin{importants}
    Le complexe argilo-humique fait donc office de « garde manger » pour les plantes : quand les plantes absorbent les ions dont elles ont besoin pour pousser, les ions stockés dans le complexe argilo-humiques vont retourner dans le sol et devenir disponible pour les plantes.
  \end{importants}

  Si le complexe argilo-humique est saturé, les ions traversent le sol quand il est mouillé par de l'eau, jusqu'à atteindre une nappe phréatique ou une rivière.
  On parle de \important{lessivage} des ions.
  Les ions ainsi lessivés polluent les eaux et ont des effets néfastes sur la vie aquatique s'ils sont présents en trop grande quantité.
\end{doc}

\begin{doc}{Les engrais NPK dans l'agriculture intensive}
  Spontanément, dans la nature un équilibre se crée entre la pousse des plantes, qui prélèvent des ions du sol, et la décomposition de la matière organique, qui enrichit le sol en ions.

  Cet équilibre est brisé si on pratique de l'agriculture intensive, pour produire constamment de la nourriture en grande quantité sur une même parcelle de terrain. 
  \begin{importants}
    Pour compenser les ions prélevés dans le sol, les agriculteur-ices utilisent des engrais qui contiennent des ions phosphates \ionPhosphate, nitrates \nitrate et potassium \ionPotassium, qui sont les plus importants pour les plantes.
    On les appelle les engrais \important{NPK,} pour « engrais azote, phosphore et potassium ».
  \end{importants}
  
  L'utilisation de ses engrais n'est pas sans risque, car le lessivage peut être important s'ils sont épandu avant une averse, ce qui peut mener à des pollutions intenses des cours d'eaux.

  % \vspace*{-12pt}
  % \begin{wrapfigure}[2]{r}{0.2\linewidth}
  %   \begin{tikzpicture}
  %     % Carré légende
  %     \fill[color = couleurSec]     (0ex, 9.5ex) rectangle (2.5ex, 7ex);
  %     \fill[color = couleurSec-300] (0ex, 6ex)   rectangle (2.5ex, 3.5ex);
  %     \fill[color = couleurSec-100] (0ex, 2.5ex) rectangle (2.5ex, 0ex);
  %     % Légende
  %     \node[right] at (3ex, 8.25ex) {\important[black]{Agriculture}};
  %     \node[right] at (3ex, 5.ex)   {\important[black]{Eaux usées}};
  %     \node[right] at (3ex, 1.25ex) {\important[black]{Industries}};
  %   \end{tikzpicture}
  % \end{wrapfigure}
  % \strut
  
  % \begin{importants}
  %   L'agriculture intensive \important{est la première source de pollution des eaux en France et en Europe.}
  % \end{importants}
  
  % \begin{center}
  %   \begin{tikzpicture}
  %     % Barres
  %     \fill[color = couleurSec]     (0.0 , 0.0)      rectangle (0.66*90ex, 3ex);
  %     \fill[color = couleurSec-300] (0.66*90ex, 0.0) rectangle (0.88*90ex, 3ex);
  %     \fill[color = couleurSec-100] (0.88*90ex, 0.0) rectangle (90ex, 3ex);
  %     % Valeurs
  %     \node at (0.33*90ex, 1.5ex) {\important[white]{66 \%}};
  %     \node at (0.77*90ex, 1.5ex) {\important[black]{22 \%}};
  %     \node at (0.94*90ex, 1.5ex) {\important[black]{12 \%}};
  %   \end{tikzpicture}
  % \end{center}

  % \vspace*{-12pt}
  % Ainsi, \qty{66}{\percent} de la pollution aux nitrates vient de certaines pratiques agricoles, \qty{12}{\percent} de rejets industriel et \qty{22}{\percent} des eaux usées en ville.
\end{doc}

\begin{doc}{L'usage intensif des pesticides pour protéger les cultures intensives}
  Pour « protéger » les cultures des « nuisibles », de nombreux pesticides sont utilisés pour l'agriculture intensive conventionnelle.
  Cet usage intensif des pesticides \important{entraine un effondrement des populations d'insectes et d'oiseaux en Europe.}

  Or, sans insectes les matières organiques se décomposent beaucoup plus difficilement.
  Sans décomposition le sol ne s'enrichit plus en nutriments et les plantes ne peuvent plus pousser (sauf en utilisant des engrais).
  Les oiseaux jouent aussi un rôle essentiel pour la reproduction des plantes et l'enrichissement de certains sols. 

  La disparition des insectes et des oiseaux, en plus d'être une catastrophe écologique, est donc aussi une catastrophe agricole, qui ne peut que mener à une perte de productivité.% et donc à moins de nourriture, ou à une nourriture de moins bonne qualité, pauvre en nutriments.
\end{doc}

\begin{doc}{Des alternatives pour limiter l'impact environnemental de l'agriculture intensive}
  \begin{listePoints}
    \item \important{Limiter l'utilisation d'engrais NPK,} en organisant une rotation des cultures ou en laissant en jachère les champs une année pour que le sol regagne des nutriments.
    \item \important{Privilégier l'agriculture biologique,} qui interdit l'usage de la plupart des intrants (pesticide et engrais) et limite donc les risques de pollution.
    \item \important{Généraliser la permaculture,} qui cherche à construire un écosystème artificiel centré sur les plantes que nous mangeons, pour que les cultures s'auto-entretiennent en utilisant la biodiversité locale.
    \item \important{Limiter très fortement ou interdire l'usage de certains pesticides,} pour protéger les populations d'insectes et d'oiseaux qui s'effondrent en Europe.
    \item \important{Économiser l'eau,} en privilégiant des plantes peu consommatrices, en paillant les sols et en utilisant des systèmes d'arrosage goutte à goutte quand c'est possible.
  \end{listePoints}
\end{doc}