%%%%
\tetePremStssOrga
\titreActivite{Nomenclature en chimie organique}

%%%% objectifs
\begin{objectifs}
  \item Savoir nommer des molécules organiques simples.
  \item Savoir reconnaître la fonction principale d'une molécule organique à partir de son nom.
\end{objectifs}

\begin{contexte}
  Il existe des millions de molécules organiques, certaines avec des propriétés similaires.

  \problematique{
    Comment nommer ces molécules selon leur propriétés et leur structures ?
  }
\end{contexte}


\begin{doc}{Principe de la nomenclature}{doc:A4_principe_nomenclature}
  \begin{importants}  
    La \important{nomenclature} est l'ensemble des règles établies pour nommer les molécules organiques.
  \end{importants}
   
  La nomenclature moderne repose sur deux principes :
  \begin{listePoints}
    \item décrire la \important{géométrie} de la molécule nommée ;
    \item indiquer les \important{fonction organiques} présentes dans la molécule.
  \end{listePoints}
\end{doc}

%%
\begin{doc}{Nommer une chaîne carbonée}{doc:A4_chaine_carbonee}
  Toute molécule organique possède au moins une chaîne carbonée.
  Pour nommer une chaîne carbonée, on va associer un \important{préfixe} avec un \important{suffixe}.
  Le suffixe dépend de la fonction organique, mais le préfixe est déterminé par le nombre de carbones qui composent la chaîne.
  \begin{importants}
  \begin{center}
    \begin{tblr}{
      columns = {c}, vlines, hlines,
      row{1} = {couleurPrim!25},
      row{2} = {white},
      column{1} = {couleurPrim!15},
    }
      Nombre de carbone \chemfig{C} 
      & 1 & 2 & 3 & 4 & 5 & 6 \\
      Préfixe
      & meth- & éth- & prop- & but- & pent- & hex- \\
    \end{tblr}
  \end{center}  
\end{importants}
\end{doc}

%%
\titreSousSection{Règles pour les alcanes, alcènes ou alcynes}

\begin{doc}{Les alcanes}{doc:A4_alcanes}
  \begin{wrapfigure}{r}{0.45\linewidth}   
    \vspace*{-30pt}
    \begin{boite}
      \begin{importants}
        Un \important{hydrocarbure} est une molécule qui ne contient que des éléments carbones et hydrogènes.
      \end{importants}
      \begin{importants}
        Un hydrocarbure est \important{saturé} (en hydrogène) s'il ne comporte que des \important{liaisons simples}. \\
        Si l'hydrocarbure comporte des \important{liaisons doubles} ou \important{triples}, on dit qu'il est \important{insaturé.}
      \end{importants}
      % \vAligne{-38pt}
    \end{boite}
  \end{wrapfigure}
  \phantom{b}\vspace*{-14pt}
  
  \begin{importants}
    Une molécule d'alcane est un \important{hydrocarbure} saturé, composé de \important{liaisons simples}.
  \end{importants}
  Pour nommer un alcane, il faut déterminer la chaîne carbonée la plus longue qui compose la molécule. \\
  On écrit alors le préfixe lié à la longueur de la chaîne et on ajoute le suffixe « \important{-ane} ». \\
  Un alcane a toujours une formule brute de la forme \chemfig{C_{n} H_{2(n + 1)}} (\chemfig{C_4H_{10}} par exemple).
  
  \vspace*{4pt}
  \exemple \chemfig{H_3C - CH_2 - CH_3} trois carbones dans la chaîne, donc prop- + -ane : propane.
\end{doc}

%%%% Question
\question{
  Nommer les molécules suivantes :
  \begin{equation*}  
    \chemfig{H_3C -CH_2 -CH_2 -CH_3} \qq{}
    \chemfig{-[1] !\cb !\ch !\cb !\ch} \qq{}
    \chemfig{H -C !\HH -C !\HHH}
  \end{equation*}
}{
  Butane, hexane et éthane.
}[1]


%%
\begin{doc}{Les alcènes}{doc:OA_2alcenes}
  \begin{importants}
    Les alcènes sont des hydrocarbures avec au moins une liaison double.
    Le suffixe « -ane », devient « \important{-ène} ».
    On indique le (ou les) numéro de la liaison double avant le suffixe, de sorte que \important{le numéro soit le plus petit possible}.
  \end{importants}
  \exemple \chemfig{H_3C- CH_2 - CH = CH -CH_3} cinq carbones dans la chaîne (pent-) et la liaison double se trouve en position 3 ou 2 (si on compte depuis la droite).
  Donc pent $+$ 2 $+$ ène : pent-2-ène.
\end{doc}

%%
\begin{doc}{les alcynes}{doc:A4_alcynes}
  \begin{importants}
    Les alcynes sont des hydrocarbures avec au moins une liaison triple.
    Le suffixe « -ane », devient « \important{-yne} ».
    On indique le (ou les) numéro de la liaison triple avant le suffixe, de sorte que \important{le numéro soit le plus petit possible}, comme pour les alcènes.
  \end{importants}
  \exemple \chemfig{-[1] ~[-1]} : trois carbones dans la chaîne (prop-) et la liaison triple se trouve en position 1.
  Donc prop-1-yne ou propyne (le 1 est implicite).
\end{doc}



%%%%
\titreSousSection{Règles pour les ramifications}

\begin{doc}{Ramification à la chaîne principale}{doc:A4_ramification}
  \begin{importants}  
    Une \important{ramification} est un substituant qui remplace un hydrogène sur la chaîne principale.
  \end{importants}
  Si le substituant est un \important{alkyle} (un hydrocarbure), son nom prend le suffixe « \important{-yl} ».

  \exemple* \chemfig{CH_3 -[6]} : méthyl, \chemfig{CH_2 (-[6]) -CH_3} éthyl.
\end{doc}

\begin{doc}{Nommer une ramification}{doc:A4_nom_ramification}
  \begin{importants}
  Pour nommer une molécule contenant des ramifications, il faut :
  \begin{listePoints}
    \item trouver la \important{plus longue chaîne carbonée} pour déterminer son nom.
    \item \important{Numéroter} la chaîne carbonée afin que la ramification ait le numéro le plus \important{petit possible}, comme pour les alcènes ou les alcynes.
    \item Placer le \important{numéro} et le \important{nom} de l'alkyle avant le nom de la chaîne.
  \end{listePoints}
  \end{importants}
  S'il y a plusieurs ramifications, leurs noms sont placés par ordre alphabétique.
\end{doc}

\vspace*{-30pt}
\begin{minipage}[T]{0.48\linewidth}
  \numeroQuestion
  Nommer la molécule suivante :\\[4pt]
  \centering
  \chemfig{H_3C- CH (-[3]CH_3) - CH (-[-3]CH_3) -CH_3 - CH_3}
\end{minipage}
\begin{minipage}[T]{0.48\linewidth}
  \pasCorrection{\phantom{b} \lignesDeReponse{3}}

  \correction{
    La chaîne principale a 5 atomes, donc -pentane.
    Deux ramifications sont en position 2 (avec un méthyl) et 3 (avec un méthyl).
    Donc le nom de cette molécule est 3,2-méthyl-pentane.
  }
\end{minipage}

%%
\titreSousSection{Règles pour les groupes caractéristiques}

\begin{doc}{Groupes caractéristiques}{doc:A4_nom_groupe_carac}
  %% Molécule d'exemple
\vspace*{-4pt}
\begin{wrapfigure}[5]{r}{0.58\linewidth}
  \vspace*{-30pt}
  \centering
  \begin{tikzpicture}[help lines/.style={thin,draw=black!50}]
    % chaine principale et carbone fonctionnel
    \large
    \node[draw] at (3,3) { \chemfig{
      H_3C-CH-CH_2 -\textcolor{couleurSec}{\textsf{\textbf{C}}} H-CH_3
      }
    };
    \draw (5, 2.25) node[right] {\textbf{chaîne principale}};
    \draw[couleurSec] (3.7, 3.7) node[right] {\textbf{carbone fonctionnel}};
    % Ramification
    \draw[very thick, couleurPrim] (1.51, 2.79) -- (1.51, 2.29);
    \draw[couleurPrim] (2.5, 1.3)  node[left] {\textbf{ramification}};
    \node[draw, couleurPrim] at (1.8, 2) { \chemfig{CH_3} };
    % Alcool
    \draw[very thick, violet] (4.11, 2.79) -- (4.11, 2.29);
    \draw[violet] (3.6, 1.3)  node[right] {\textbf{groupe caractéristique}};
    \node[draw, violet] at (4.28, 2){ \chemfig{OH_{}} };
  \end{tikzpicture}
\end{wrapfigure}
%

Pour nommer les molécules contenant des groupes caractéristiques, on utilise les règles décrites dans le tableau ci-dessous, en respectant la priorité des fonctions organiques.

\begin{importants}
  Le \important{carbone fonctionnel} désigne le carbone contenant la fonction de la molécule.
\end{importants}

Pour les cétones, alcools et amines, le numéro est celui du \important{carbone fonctionnel,} comme pour les ramifications il \important{doit être le plus petit possible.}

($R_1$) et ($R_2$) représentent les noms des chaînes carbonées auxquels les groupes caractéristiques sont attachées. 

%%%% Tableau avec les groupes caractéristiques
\vspace*{2pt}
\begin{center}
\begin{tblr}{
  width = \linewidth, hlines, vlines,
  colspec = {c c c Q[wd=0.4\linewidth]},
  column{2,4} = {couleurPrim!10},
  row{1} = {couleurPrim!20},
  rows = {m}, columns = {c}
}
  Priorité & Famille fonctionnelle & Formule & Nom si famille prioritaire \\
  %
  1 & Acide carboxylique
  & \chemfig{\textcolor{couleurQuat}{C} !\alkyleG !\cetoneCouleur \textcolor{couleurQuat}{OH}}
  & acide ($R_1$)-oïque \\
  %
  2 & Ester
  & \chemfig{\textcolor{couleurQuat}{C} !\alkyleG !\cetoneCouleur \textcolor{couleurQuat}{O} -[1] R_2}
  & ($R_1$)-oate de ($R_2$)-yle \\
  %
  3 & Amide
  & \chemfig{\textcolor{couleurQuat}{C} !\alkyleG !\cetoneCouleur \textcolor{couleurQuat}{N} H_2}
  & ($R_1$)-amide \\
  %
  4 & Aldéhyde
  & \chemfig{\textcolor{couleurQuat}{C} !\alkyleG !\cetoneCouleur \textcolor{couleurQuat}{H}}
  & ($R_1$)-al \\
  %
  5 & Cétone
  & \chemfig{\textcolor{couleurQuat}{C} !\alkyleG !\cetoneCouleur R_2}
  & ($R_1$)-(numéro)-one \\
  %
  6 & Alcool
  & \chemfig{R_1 - \textcolor{couleurQuat}{OH}}
  & ($R_1$)-(numéro)-ol \\
  %
  7 & Amine & \chemfig{R_1 - \textcolor{couleurQuat}{NH_2}}
  & ($R_1$)-(numéro)-amine \\
  %
  8 & Éther
  & \chemfig{R_1 -[1,,,,couleurQuat] \textcolor{couleurQuat}{O} -[-1,,,,couleurQuat] R_2}
  & ($R_1$)-oxy-($R_2$) \\
\end{tblr}
\end{center}

\vspace*{2pt}
\begin{importants}
  \attention Pour ces 8 familles organiques, vous devez savoir :
  \begin{listePoints}
    \item les noms de chacune des familles et leur groupes fonctionnels ;
    \item les reconnaître dans une molécule si on vous en donne une représentation.
  \end{listePoints}
\end{importants}
\end{doc}

\question{
  Nommer la molécule du document~\ref{doc:A4_nom_groupe_carac}.
}{
  4-méthyl-pent-2-ol
}[2]

\question{
  Un produit couramment consommé en France contient de l'éthanol.
  Donner, en justifiant, la famille fonctionnelle principale de cette molécule.
}{
  C'est un alcool, car son nom finit en -ol.
}[2]

\question{
  L'acétate d'isoamyle, ou éthanoate de 3-méthylbutyle en nomenclature, est une molécule odorante qui donne son goût caractéristique à la banane.
  Donner, en justifiant, la famille fonctionnelle principale de cette molécule.
}{
  C'est un ester, car son nom finit en -yle et contient un -oate.
}[2]

\question{
  Le linalol, ou 3,7-diméthylocta-1,6-dién-3-ol en nomenclature, est une molécule odorante qu'on trouve dans la lavande.
  Donner, en justifiant, la famille fonctionnelle principale de cette molécule.
}{
  C'est un alcool, car son nom finit en -ol.
}[2]

\question{
  La frambinone, ou 4-(4-hydroxyphényl)butan-2-one en nomenclature, est une molécule odorante responsable du goût de la framboise.
  Donner, en justifiant, la famille fonctionnelle principale de cette molécule.
}{
  C'est une cétone, car son nom finit en -one.
}[2]
