\tetePremStssOrga
\vspace*{-30pt}
\titreActivite{Les molécules organiques}

%%%% objectifs
\begin{objectifs}
  \item Rappeler les règles de formation des molécules.
  \item Introduire la notion de valence d'un élément chimique.
\end{objectifs}

\begin{contexte}
  Les atomes de carbones peuvent se lier entre eux pour former des \important{chaînes carbonées,} de formes et de tailles variées.
  Ces chaînes carbonées, une fois liée à des atomes d'hydrogène, d'oxygène ou d'azote, forment des \important{molécules organiques.}
  Il existe ainsi des millions de molécules organiques différentes, ce sont elles qui sont les briques de bases de tout être vivant sur Terre.

  \problematique{
    Comment comprendre la structure des molécules organiques ?
  }
\end{contexte}


%%
\begin{doc}{Éléments composant un corps humain}
  Le corps humain est composé majoritairement de 4 éléments chimiques :
  \vspace*{-4pt}
  \begin{listePoints}[2]
    \item l'oxygène   \chemfig{O} (\qty{65}{\percent} en masse),
    \item le carbone  \chemfig{C} (\qty{18}{\percent}),
    \item l'hydrogène \chemfig{H} (\qty{10}{\percent})
    \item et l'azote  \chemfig{N} (\qty{3}{\percent}).
  \end{listePoints}
  
  \begin{importants}
    \important{Numéro atomique :} il correspond au nombre de protons d'un atome et est noté $Z$ : \isotope{X}{Z}{}
    Par neutralité de l'atome, c'est aussi son nombre d'électrons.
  \end{importants}
   \exemple le carbone possède 6 protons et il est noté \isotope{C}{6}{}
\end{doc}


%%
\begin{doc}{Configuration électronique}
  À partir du numéro atomique d'un atome, on peut déterminer sa structure électronique en couche.
  
  \begin{importants}
    La couche 1 contient au maximum \important{2 électrons} et les couches 2 et 3 contiennent jusqu'à \important{8 électrons}.
  \end{importants}
  Les électrons d'un atome vont se placer dans les couches par ordre croissant en les remplissant : d'abords 2 électrons dans la couche 1, puis 8 dans la couche 2, puis 8 dans la couche 3.
  On utilise \important{la configuration électronique} pour préciser le remplissage des couches
  \exemple \isotope{N}{7}{} : $1^2\; 2^5$. 
  \begin{importants}
    La dernière couche remplie est la \important{couche externe.}
  \end{importants}
  
\end{doc}

\question{
  Donner la configuration électronique de l'oxygène \isotope{O}{8}{}, du carbone \isotope{C}{6}{} et de l'hydrogène \isotope{H}{1}{}.
  Indiquer le numéro de leurs couches externes.
}{}[3]


%%%%
\begin{doc}{Liaison moléculaire}
  \begin{importants}
    Les atomes cherchent à remplir leur couche externe pour gagner en stabilité : c'est la règle du \important{duet} (couche 1) ou de \important{l'octet} (couche 2 ou 3).
  \end{importants}
  
  %
  \begin{importants}
    Pour former des molécules, les atomes partagent les électrons de leur couche externe pour former des \important{liaison covalentes}.
    Chaque liaison covalente apporte 1 électron à l'atome, ce qui lui permet de remplir sa couche externe.
  \end{importants}
  \begin{importants}
    La \important{valence} est le nombre de liaisons formées par l'atome.
    Un élément peut être mono (1 liaison), bi (2 liaison), tri (3 liaison) ou tétravalent (4 liaisons).
  \end{importants}
  %
  Pour connaître la valence d'un atome, il suffit donc de compter combien d'électrons il lui manque pour remplir sa couche externe.

  \exemple \isotope{C}{6}{} : $1^2\; 2^4$,
  il lui manque \important{4} électrons pour compléter sa couche externe et respecter la règle de \important{l'octet.}
  Il fera donc \important{4} liaisons, il est \important{tétravalent.}
\end{doc}

\question{
  Indiquer combien d'électrons il manque à l'oxygène \isotope{O}{8}{} pour respecter la règle de l'octet, le nombre de liaisons ainsi formées et sa valence.
}{
  \isotope{O}{8}{} : $1^2\; 2^6$,
  il lui manque 2 électrons pour respecter la règle de l'octet, il formera donc 2 liaisons. Il est bivalent.
}[2]

\question{
  Même question pour l'azote \isotope{N}{7}{} et l'hydrogène \isotope{H}{1}{}.
}{
  Il manque 3 électrons à l'azote pour respecter la règle de l'octet, l'azote formera donc 3 liaisons. Il est trivalent. \\
  Il manque 1 électron à l'hydrogène pour respecter la règle du duet, l'hydrogène formera donc 1 liaison. Il est monovalent.
}[4]


%%
\begin{doc}{Liaisons multiples}
  \begin{importants}
    Pour compléter leur couche externe et respecter la règle de l'octet, deux atomes peuvent se lier en formant 2 ou 3 liaisons covalentes.
    
    On dit qu'il y a une \important{liaison double} ou une \important{liaison triple}
  \end{importants}
\end{doc}

\question{
  Indiquer si les liaisons sont simples, triples ou doubles sur les molécules suivantes :
  \begin{equation*}
    \chemfig{N ~N} \qq{}
    \chemfig{O =C =O} \qq{}
    \chemfig{H -C ~N}
  \end{equation*}
}{
  Liaison triple, liaison double et double, liaison simple et triple
}


\begin{doc}{Valence des éléments \carbone, \hydrogene, \oxygene, \azote}
  \begin{center}
    \begin{tblr}{c c}
      \SetCell{cyan-100} \chemfig{C ([3]-) ([-3]-) ([6]-) -} \chemfig{C ([4]-) ([-4]-) =} &
      \SetCell{red-100}  \chemfig{- N ([3]-) -} \\
      %
      \SetCell{cyan-100} Le carbone est tétravalent &
      \SetCell{red-100}  L'azote est trivalent \\
      %
      \SetCell{purple-100} \chemfig{O =} \chemfig{O ([6]-) (!\vide{3}) -} &
      \SetCell{blue-100}   \chemfig{H (!\vide{-3}) -} \\
      %
      \SetCell{purple-100} L'oxygène est bivalent &
      \SetCell{blue-100}   L'hydrogène est monovalent \\
    \end{tblr}
  \end{center}
\end{doc}
