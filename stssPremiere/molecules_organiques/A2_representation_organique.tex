%%%%
\tetePremStssOrga

%%%% titre
\numeroActivite{2}
\vspace*{-30pt}
\titreActivite{Représenter les molécules organiques}


%%%% objectifs
\begin{objectifs}
  \item Connaître les quatre représentations des molécules organiques.
\end{objectifs}

\begin{contexte}
  Les \important{molécules organiques} sont composées de \important{chaînes carbonées,} auxquelles sont ajoutés des atomes d'hydrogène, d'oxygène ou d'azote le plus souvent.
  
  \problematique{
    Comment représenter les molécules organiques ?
  }
\end{contexte}

%%
\vspace*{-12pt}
\titreSection{La formule brute}

\vspace*{-8pt}
\begin{doc}{Formule brute}{doc:formule_brute_A1}
  \begin{importants}
    Elle précise séparément le nombre d'éléments présents dans la molécule.
  \end{importants}
  \exemple* Le butane \chemfig{C_4 H_{10}}, l'éthanol \bruteCHO{2}{6}{} ou l'acide carbonique \bruteCHO{}{2}{3}
  
  Elle permet de calculer facilement les \important{masses molaires} et de vérifier si deux molécules sont \important{isomères}.
  Par contre elle \important{ne permet pas} de déterminer la géométrie d'une molécule.

  \begin{importants}
    Deux molécules sont \important{isomères} si elles ont la même formule brute, mais un agencement des atomes différents.
  \end{importants}
  \exemple* Le glucose et le fructose sont isomères de formules brutes \bruteCHO{6}{12}{6}, mais ce ne sont pas les mêmes molécules, car leurs géométries sont différentes.
\end{doc}

L'oxybenzone est une molécule utilisée pour protéger des UVA et B issu du soleil.
Sa formule brute est \bruteCHO{14}{12}{3}.

\question{%
  Indiquer le nombre d'élément d'hydrogène, d'oxygène et de carbone dans la molécule d'oxybenzone.
}{%
  Il y a 12 hydrogènes, 3 oxygènes et 14 carbones.
}{3}

\vspace*{8pt}
\begin{wrapfigure}[5]{r}{0.3\linewidth}
  \vspace*{-22pt}
  \image{1}{images/organique/alanine.png}
\end{wrapfigure}

L'alanine est un acide aminé utilisé dans le corps humain pour former des protéines.
Sa représentation avec un modèle moléculaire est présentée ci-contre avec le code couleur suivant :
\begin{listePoints}[2]
  \item Blanc : hydrogène.
  \item Rouge : oxygène.
  \item Noir : carbone.
  \item Bleu : azote.
\end{listePoints}

\medskip
\question{%
  Donner la formule brute de l'alanine
}{%
  \chemfig{C_2 H_7 O_3 N}
}{2}

\question{
  Compter les liaisons de chaque carbone et vérifier qu'ils ont bien la bonne valence.
}{

}{1}


%%
\newpage
\vspace*{-30pt}
\begin{doc}{Formule développée}{doc:formule_developpee_A1}
  \begin{importants}  
    Elle représente tous les éléments chimiques et toutes les liaisons dans le même plan, ce qui permet de \important{préciser la géométrie d'une molécule.}
  \end{importants}

  \exemple*
  \vspace*{-18pt}
  \begin{center}
    \chemname{\chemfig{H -C (-[3]O (-[5]H)) (-[-3]H) -C !\saturationH}}{éthanol}
    %
    \qq{}
    \chemname{\chemfig{Cl -C !\paireH -Si !\saturationH}}{chlorométhylsilane}
  \end{center}
\end{doc}


%%
\begin{doc}{Formule semi-développée}{doc:formule_semi_developpee_A1}
  \begin{importants}
    Comme la formule développée, elle représente tous les éléments chimiques, mais elle ne détaille pas les liaisons des éléments \important{hydrogènes}.
  \end{importants}

  \exemple*
  \vspace*{-8pt}
  \begin{center}
    \chemname{\chemfig{HO -CH_2 -CH_3}}{éthanol}
    %
    \qq{}
    \chemname{\chemfig{Cl -CH_2 -SiH_3}}{chlorométhylsilane}
  \end{center}
\end{doc}

%%
\begin{doc}{Formule topologique}{doc:formule_topologique_A1}
  \begin{importants}  
    Elle représente les liaisons \important{carbone-carbone \chemfig{C - C}} par des segments formant des angles.
    Chacune des extrémités d'un segment représente un carbone, sauf si un autre élément chimique y est attaché.
    Les éléments \important{carbones} et les \important{hydrogènes} qui sont attachés aux carbones \important{ne sont pas représentés}.
    Tous les autres éléments chimiques sont représentés normalement.
  \end{importants}

  \exemple*
  \vspace*{-20pt}
  \begin{center}
    \chemname{
      \chemfig{HO-[1]-[-1]}
    }{éthanol}
    %
    \qq{}
    \chemname{
      \chemfig{Cl -[1] -[-1]SiH_3}
    }{chlorométhylsilane}
    %
    \qq{}
    \chemname{
      \chemfig{!\paracetamol}
    }{paracétamol}
  \end{center}
\end{doc}

\question{
  Donner la formule brute, semi-développée et développée du paracétamol.
}{}{6}
