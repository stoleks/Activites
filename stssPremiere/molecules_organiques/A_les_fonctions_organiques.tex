\tetePremStssOrga
\vspace*{-34pt}
\titreActivite{Les fonctions organiques}

%%%% objectifs
\begin{objectifs}
  \item Connaître les 7 groupes caractéristiques et les 8 familles fonctionnelles associées.
\end{objectifs}

\begin{doc}{Fonctions organiques}{doc:A2_fonction_organique}
  Certaines séquences d'éléments donnent des \important{propriétés} spécifiques aux molécules organiques que l’on classe en différentes familles ou fonctions organiques ou encore famille fonctionnelle.

  % \begin{importants}  
    En ST2S on étudie à 8 familles : \important{alcool, aldéhyde, cétone, acide carboxylique, ester, éther, amine et amide.}
  % \end{importants}
  \medskip

  \begin{tblr}{
    colspec = {c c c Q[t, wd=0.26\linewidth]}, hlines, vlines,
    column{2, 3} = {couleurSec-50}, row{1} = {couleurSec100},
    cell{3}{1} = {r=2}{c},
    rows = {m}, columns = {c}
  }
    Groupe caractéristique & Famille organique & Formule & Exemple \\
    %
    Hydroxyle & Alcool
    & \chemfig{R_1 - \textcolor{couleurQuat}{OH}} 
    & {\chemfig{-[1] -[-1] OH} \\[1pt] éthanol} \\
    %
    Carbonyle & \vAligne{-40pt} Cétone
    & \vAligne{-60pt} \chemfig{\textcolor{couleurQuat}{C} !\alkyleG !\cetoneCouleur R_2}
    & {\chemfig{-[1] !\carbonyle -[1]} \\[1pt] butan-2-one} \\
    %
    & Aldéhyde
    & \chemfig{\textcolor{couleurQuat}{C} !\alkyleG !\cetoneCouleur \textcolor{couleurQuat}{H}}
    & {\chemfig{O=[-3] -H} \\[1pt] méthanal } \\
    %
    Carboxyle & Acide carboxylique
    & \chemfig{\textcolor{couleurQuat}{C} !\alkyleG !\cetoneCouleur \textcolor{couleurQuat}{OH}}
    & {\chemfig{-[-1] -[1] !\carboxyle} \\[1pt] acide propanoïque} \\
    %
    \vAligne{-34pt} Ester & \vAligne{-34pt} Ester
    & \chemfig{R_1 -[1] \textcolor{couleurQuat}{C} !\cetoneCouleur \textcolor{couleurQuat}{O} -[1] R_2}
    & {\chemfig{-[1] -[-1] -[1] !\ester -[1] -[-1]} \\[1pt] butanoate d'éthyle} \\
    %
    Éther-oxyde & Éther
    & \chemfig{R_1 -[1,,,,couleurQuat] \textcolor{couleurQuat}{O} -[-1,,,,couleurQuat] R_2}
    & {\chemfig{-[-1] -[1] O -[-1] -[1]} \\[1pt] éthoxyéthane} \\
    %
    Amine & Amine
    & \chemfig{R_1 - \textcolor{couleurQuat}{NH_2}}
    & {\chemfig{-[1] -[-1] -[1] NH_2} \\[1pt] propan-1-amine} \\
    %
    Amide & Amide
    & \vAligne{-48pt} \chemfig{\textcolor{couleurQuat}{C} !\alkyleG !\cetoneCouleur \textcolor{couleurQuat}{N} (-[-3] R_3) - R_2}
    & {\chemfig{-[-1] -[1] !\amide H_2} \\[1pt] propanamide}
  \end{tblr}
  \smallskip

  $R_1,$ $R_2$ et $R_3$ sont des chaînes carbonées appelées \important{« radicaux alkyles ».}

  \begin{importants}
    Pour trouver les groupes caractéristiques d'une molécule, il faut repérer tous les éléments qui ne sont ni des carbones, ni des hydrogènes.
  \end{importants}
\end{doc}

\begin{doc}{Radicaux alkyle}
  \begin{importants}
    Les \important{« radicaux alkyles »,} notés R, sont des morceaux de chaînes carbonées composées de liaisons simples avec des hydrogènes.
  \end{importants}

  \centering
  \begin{tblr}{
    columns = {5cm, c}, vlines, hlines,
    row{1} = {couleurSec-100},
    row{2} = {4cm}
  }
    Méthyle & Éthyle & Propyle \\
    \\
  \end{tblr}
\end{doc}

\question{
  Identifier les fonctions organiques qui sont présentes dans les molécules suivantes
  
  \begin{center}
    \begin{tblr}{c c c c c}
      \chemfig{-[1] O !\cb !\ch} &
      \chemfig{H!\teteAcideDev C !\HHH} &
      \chemfig{-[1] !\ester !\ch !\cb} &
      \chemfig{H_3C -C !\carbonyleDev H} &
      \chemfig{-[-1] !\ch !\carbonyle !\ch} \\
      % 
      1 & 2 & 3 & 4 & 5
    \end{tblr}
  \end{center}
  \phantom{bla}
  \vspace*{-12pt}
}{}[7]


\begin{doc}{Identification des familles organiques}
  Pour identifier une famille organique dans une molécule, il faut chercher si elle comporte des \important{oxygènes \chemfig{O}} ou des \important{azotes \chemfig{N}}. 
  
  Si elle comporte un oxygène \chemfig{O} doublement lié à un carbone (\chemfig{O=}), alors il faut regarder le voisinage du carbone
  \begin{listePoints}
      \item s'il y a un groupe hydroxyle \chemfig{OH}, on a un \important{acide carboxylique.}
      \item s'il y a un oxygène \chemfig{O}, on a un \important{ester.}
      \item s'il y a un azote \chemfig{N}, on a un \important{amide.}
      \item s'il y a un hydrogène \chemfig{H}, on a un \important{aldéhyde.}
      \item sinon on a une \important{cétone.}
  \end{listePoints}
  Sinon, si elle a un groupe \chemfig{OH}, c'est un \important{alcool ;} si elle a un azote \chemfig{N}, c'est un \important{amine ;} et si elle a un oxygène \chemfig{O}, c'est un \important{éther.}
\end{doc}