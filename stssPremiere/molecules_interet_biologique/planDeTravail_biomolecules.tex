\tetePremStssBiol

% \vspace*{-40pt}
\titre{Plan de Travail -- \premStssBiol}
\vspace*{-8pt}

\begin{importants}
  Le plan de travail est un cadre de travail collectif où tu as la liberté d'avancer, seul-e ou en groupe, à ton rythme.  
  Ce document, \important{qui sera ramassé et évalué,} présente les activités et travaux pratiques à réaliser pendant les 3,5 semaines du chapitre.
  À chaque séance (en classe entière ou demi-groupe), tu es libre de choisir quelle activité ou TP réaliser avec ton groupe.
  Tous les documents sont imprimés sur le bureau du professeur.
\end{importants}


%%%% Activités
\titre{Activités à réaliser}
\vspace*{-16pt}

\begin{multicols}{2}
  \begin{TP}{Les glucides}[2 h]{glucides}
    \begin{prerequis}
      \item Connaître la formule topologique.
      \item Savoir identifier les fonctions alcool, aldéhyde et cétone.
    \end{prerequis}
    \begin{objectifs}  
      \item Étudier la structure des glucides.
      \item Savoir que le fructose et le glucose peuvent exister sous forme linéaire ou cyclique.
      \item Connaître la différence entre un sucre lent et un sucre rapide.
    \end{objectifs}
  \end{TP}
  \smallskip

  \begin{activite}{Les lipides}[1,5 h]{lipides}
    \begin{prerequis}
      \item Connaître la formule topologique et semi-développée.
      \item Savoir identifier les fonctions acide carboxylique et ester.
    \end{prerequis}
    \begin{objectifs}
      \item Étudier la structure des lipides.
      \item Distinguer un acide gras saturé/insaturé.
      \item Voir la structure d'un triglycéride.
    \end{objectifs}
  \end{activite}

  \begin{activite}{Les protéines}{proteines}
    \begin{prerequis}
      \item Savoir passer de la formule topologique à la formule semi-développée.
      \item Savoir identifier les fonctions amine, acide carboxylique et amide.
    \end{prerequis}
    \begin{objectifs}
      \item Voir la structure d'un acide $\alpha$-aminé.
      \item Savoir identifier des acides aminés dans une chaîne peptidique.
      \item Étudier la structure des protéines.
    \end{objectifs}
  \end{activite}
  \smallskip

  \begin{TP}{Les vitamines}[1,5 h]{vitamines}
    \begin{prerequis}
      \item Connaître la formule topologique.
      \item Savoir identifier la fonction alcool.
    \end{prerequis}
    \begin{objectifs}
      \item Définir une vitamine.
      \item Étudier la structure de la vitamine C.
      \item Réaliser des tests pour identifier les propriétés de la vitamine C.
    \end{objectifs}
  \end{TP}
\end{multicols}


Note :
Les TP doivent être réalisé en salle expérimentale (en demi-groupe), il faut en tenir compte pendant la planification.


%%%% Progression
\newpage
\nomPrenomClasse
\titre{Progression des activités}
\vspace*{12pt}

\flecheProgression{3}
\vspace*{-353 pt}

\begin{programmeSeance}
  \seance{1 h}{}
  \seance{2 h}{}
  \seance{1 h}{}
\end{programmeSeance}

\begin{programmeSeance}[2]
  \seance{2 h}{}
  \seance{1 h}{}
\end{programmeSeance}

\begin{programmeSeance}[2](0)
  \seance{2 h}{ \important{Tâche finale} }
  \seance{1 h}{ \important{Évaluation du chapitre} }
\end{programmeSeance}


%%%% Tâche finale
\begin{tacheFinale}
  Préparer une affiche \important{par groupe de 4} sur un type de biomolécules (lipide, glucide, protéine, vitamine), qui présente de manière synthétique sa structure générale  (s'il y en a une), quelques propriétés chimiques et quelques propriétés biologiques.
  L'affiche doit être compréhensible pour un-e élève de première \textsc{ST2S} qui n'aurait pas encore vu-e les biomolécules.
\end{tacheFinale}


%%%% Evaluation
\titre{Évaluation de l'autonomie}

\important{Les différents degrés d'autonomie}

\begin{enumerate}[label = \Alph*]
  \item Je planifie librement mon apprentissage, je coopère avec mes camarades et je sollicite de l'aide pour valider les travaux réalisés.
  \item Je travaille seul-e ou avec mes camarades à partir des documents et je sollicite régulièrement de l'aide pour avancer.
  \item J'avance uniquement quand le professeur est là pour m'aider, je n'arrive pas à planifier mon travail ou je ne fais que recopier les réponses d'un-e de mes camarades.
  \item J'utilise des stratégies pour éviter d'apprendre et je refuse d'essayer de faire les activités.
\end{enumerate}

\begin{tableauCompetences}
  AUTO & Travailler de manière autonome \\
\end{tableauCompetences}