%%%%
\tetePremStssBiol
\titreTP{Les vitamines}

\begin{objectifs}
  \item Comprendre ce que c'est qu'une vitamine.
  \item Étudier un exemple de vitamine : la vitamine C.
\end{objectifs}

\begin{contexte}
  Contrairement aux glucides, lipides et protéines, les vitamines n'ont pas de structures particulière distinguable.
  Les vitamines sont simplement des molécules essentielles au bon fonctionnement du corps humain.
  On va étudier un exemple de vitamine : la vitamine C.

  \problematique{
    Quelle sont les propriétés de la vitamine C ?
  }
\end{contexte}


%%%%%
\begin{doc}{La vitamine C : l'acide ascorbique}
  \begin{wrapfigure}{l}{0.3\linewidth}
    \centering
    {\small \chemfig[atom sep = 1.5em]{!\acideAscorbique} }
    
    \legende{Acide ascorbique, de formule brute \bruteCHO{6}{8}{6}}
  \end{wrapfigure}
  La vitamine C est une molécule, l'acide ascorbique. 
  Elle remplit plusieurs fonctions dans l'organisme :
  \begin{listePoints}
    \item défense contre les infections virales ou bactériennes ;
    \item protection contre le vieillissement des cellules grâce à son action anti-oxydante ;
    \item protection de la paroi des vaisseaux sanguins ;
    \item meilleur assimilation du fer ;
    \item formation du collagène ; cicatrisation des plaies ; etc.
  \end{listePoints}
\end{doc}

\question{
  Recopier la molécule composant la vitamine C, puis entourer et nommer les fonctions organiques présentes.
}{}[7]

\begin{doc}{Mise en évidence des propriétés de la vitamine C}[\label{doc:test_vitamine}]
  Le permanganate de potassium est une \important{solution oxydante.}
  En milieu acide et en présence de fonction alcool, le permanganate de potassium se décolore (la solution passe de violette à transparente).

  Le protocole suivant est réalisé :
  \begin{protocole}
    \item un comprimé de vitamine C est dissout dans de l'eau ;
    \item le pH de cette solution est mesuré, on trouve un pH de 4 ;
    \item \qty{2}{\ml} de permanganate de potassium est versé dans un tube à essai ;
    \item quelques gouttes de solution contenant de la vitamine C sont versées dans le tube à essai ;
    \item on observe alors une décoloration de la solution de permanganate.
  \end{protocole}
\end{doc}

\question{
  La solution contenant la vitamine C est-elle acide ? Justifier.
}{}[2]

\question{
  Quelle famille organique de la vitamine C est identifiée par le test réalisé dans le doc.~\ref{doc:test_vitamine} ?
}{}[2]


\begin{doc}{Pourquoi il faut consommer de la vitamine C}
  \extrait[D'après \url{https://www.anses.fr/fr/content/vitamine-c}\hspace{1.2cm}\strut]{
  La vitamine C permet de consolider les fibres de collagène, constitutives du tissu conjonctif qui soutient les cellules et structure ainsi les autres tissus.
  Elle intervient dans la synthèse de molécules impliquées dans la transmission nerveuse (ex. noradrénaline).
  Elle assure un rôle protecteur des tissus en captant les substances oxydantes.
  Enfin, elle facilite l’absorption du fer non héminique (présent dans les aliments d’origine végétale comme les légumineuses ou les noix). 
  \medskip
  
  Les besoins en vitamine C peuvent être couverts en consommant des fruits [\important{frais}] tels que les cassis et les agrumes, et des légumes, en particulier le persil et les poivrons.
  \medskip

  La pathologie spécifique liée à la carence en vitamine C est le scorbut.
  Elle se manifeste par un saignement des gencives, un déchaussement des dents ou encore des douleurs des articulations.
  }
  \vspace*{-36pt}
  \begin{flushright}
    \qrcode[height = 1cm]{https://edurl.fr/BU9GIbYR}
  \end{flushright}
\end{doc}

\question{
  Quelles sont les aliments qui contiennent de vitamine C ? Quels conseils nutritionnels pourrait-on donner à une personne qui a une carence en vitamine C ?
}{}[3]

\question{
  Le scorbut avait disparu en France au cours du \siecle{19}, mais depuis 2020 ans des cas réapparaissent chez des jeunes enfants dans les populations pauvres.
  Quelle politique de santé publique pourrait-on mettre en place pour lutter contre le scorbut chez les enfants ?
}{}[6]
