%%%%
\tetePremStssBiol
\vspace*{-36pt}
\titreActivite{Les protéines}

\begin{objectifs}
  \item Étudier la structure des protéines et des acides aminés.
\end{objectifs}

\begin{contexte}
  Les protéines sont des molécules complexes qui permettent à nos organismes de fonctionner en remplissant en ensemble varié de rôles en son sein.

  \problematique{
    Quelle est la structure des protéines ?
  }
\end{contexte}



%%%%%
\begin{doc}{Acides alpha-aminés}{doc:A7_acides_amines}
  \vspace*{-24pt}
  \begin{wrapfigure}{r}{0.3\linewidth}
    \centering
    \vspace*{-13pt}
    \begin{tikzpicture}
      %%%% Alanine
      \fill[red-150,rounded corners = 8pt] % Carboxyle
        (0.5,0)-- (0.5,0.8)-- (1.1,1.45)-- (1.8,1.45)--
        (2.1, 0)-- (2.1,-0.7)-- (1.2,-0.7)-- cycle;
      \fill[cyan-100,rounded corners = 8pt] (0.5,0) rectangle (-0.7,-0.9); % Amine
      %%%% Glycine
      \fill[red-150,rounded corners = 8pt] % Carboxyle
        (0,-2.8)-- (0,-2)-- (0.6,-1.35)-- (1.3,-1.35)--
        (1.6, -2.5)-- (1.6,-3.5)-- (0.6,-3.5)-- cycle;
      \fill[cyan-100,rounded corners = 8pt] (0.15,-2.8) rectangle (-0.95,-3.65); % Amine
      %%%% Molécules
      \node at (0,0)    {\chemname {\chemfig{!\alanineSemiDev}} {Molécule d'alanine}};
      \node at (0,-2.8) {\chemname {\chemfig{!\glycineSemiDev}} {Molécule de glycine}};
    \end{tikzpicture}
  \end{wrapfigure}
  \phantom{b}
  
  \begin{importants}
    Les \important{acides $\alpha$-aminés (alpha-aminés)} sont des molécules composées d'une fonction \important{acide carboxylique} et d'une fonction \important{amine,} d'où leur nom.
    Ces deux fonctions sont liées au même carbone fonctionnel.
  \end{importants}
  
  Sur Terre, les organismes vivants synthétisent et utilisent 20 acides aminés.
  Parmi ces 20 acides aminés, 9 ne sont pas synthétisés par le corps humain, on dit que ce sont des \important{acides aminés essentiels,} qui doivent être apportés par une alimentation équilibrée.
\end{doc}

\begin{doc}{Quelques acides $\alpha$-aminés essentiels}{doc:A7_aa_essentiels}
  \centering
  \chemname {\chemfig{!\isoleucine}} {Isoleucine}
  \chemname {\chemfig{!\leucine}}    {Leucine}
  \chemname {\chemfig{!\methionine}} {Méthionine}
  \chemname {\chemfig{!\valine}}     {Valine}
\end{doc}

\question{
  En identifiant ses groupes fonctionnels, Justifier que l'isoleucine est un acide $\alpha$-aminé.
}{}[2]

\numeroQuestion Entourer les groupes fonctionnels de la leucine et de la méthionine.


\begin{doc}{Séquence de 3 acides $\alpha$-aminés dans l'insuline humaine}{doc:A7_insuline_aa}
  \begin{center}
    \chemfig{H_2N !\glycineSemiDevH !\HN !\isoleucineSemiDevB !\NH !\valineSemiDevH OH}
    
    % \chemfig{[:-30] H_2N !\glycineH \chembelow{N}{H} !\isoleucineB \chembelow{N}{H} !\valineH OH}
  \end{center}
\end{doc}

%%%%
\begin{doc}{Les protéines}{doc:A7_proteines}
  Deux acides aminés peuvent se lier quand un groupe carboxyle réagit avec un groupe amine, c'est la \important{liaison peptidique.}

  \vspace*{-14pt}
  \begin{center}
  \begin{tikzpicture}
    \fill[rectArrondi = orange-150] (2.15,0.5) rectangle (3.1,-1.1);
    \fill[rectArrondi = orange-150] (2.2,0.5) rectangle (3.95,-0.4);
    \fill[rectArrondi = orange-150] (3,-0.4) rectangle (3.95,1.3);
    \node[orange-700, align = center] at (3,-1.4) {\textbf{Liaison peptidique}};
    \node at (0,0) {
      \chemfig{R_1-!\acideAmineSD} + \chemfig{R_2-!\acideAmineSD}
      \reaction
      \chemfig{H_2N !\acideAmineSDH{-[::90] R_1} N(-[::90] H) !\acideAmineSDB{-[::-90] R_2} OH} + \eau
    };
  \end{tikzpicture}
  \end{center}
  \vspace*{-18pt}

  Comme tous les acides aminés possèdent un \important{groupe amine} et un \important{groupe carboxyle}, cette réaction peut se répéter pour former une chaîne d'acides aminés appelée \important{polypeptide}.

  \begin{importants}
    Une \important{protéine} est un polypeptide qui s'est replié sur lui même.
    Ce repliement lui donne une structure tridimensionnelle unique, qui lui confère une \important{fonction biologique} particulière.
  \end{importants}
  \begin{center}
    \image{1}{images/proteines/structure_proteines}
  \end{center}
\end{doc}


\begin{doc}{Rôle des protéines dans l'organisme}{doc:A7_proteine_organisme}
  Les protéines sont omniprésentes dans tous les organismes vivants : elles sont les petites ouvrières qui en assure le bon fonctionnement.
  
  Elles remplissent un ensemble varié de fonctions, de l'échelle d'une cellule (réplication ou transcription de l'ADN, fabrication de protéines, structure de la cellule, etc.), à l'échelle du corps entier (transport d'oxygène, transmission d'information, structure des muscles, etc.).

  Par exemple, l'hémoglobine permet de transporter le dioxygène des poumons jusqu'aux cellules.
  L'insuline permet de signaler aux cellules de capter le glucose qui circule dans le sang.
  Des enzymes digestives permettent de digérer les glucides complexes pour les transformer en glucose.

  Contrairement aux glucides et aux lipides, les protéines sont dénaturées en \important{urée dans le foie} une fois utilisées.
  L'urée est ensuite évacuée par les urines.
\end{doc}

\question{
  Nommer le groupe caractéristique formé au cours de la liaison peptidique.
}{}[1]

\question{
  Découper en 3 acides $\alpha$-aminés la séquence dans l'insuline présentée dans le document~\ref{doc:A7_insuline_aa} et les nommer en vous aidant des documents~\ref{doc:A7_acides_amines} et~\ref{doc:A7_aa_essentiels} pour reconnaître leur structure.
}{}[2]
