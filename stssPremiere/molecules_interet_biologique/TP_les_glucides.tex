%%%%
\tetePremStssBiol
\titreTP{Les glucides}

\begin{objectifs}
  \item Étudier la structure des glucides.
  \item Comprendre la différence entre un sucre rapide et un sucre lent.
\end{objectifs}

\begin{contexte}
  Les glucides sont une part essentielle de notre alimentation.
  Tous les glucides que nous ingérons sont transformés en glucose au cours de la digestion,
  qui peut ensuite être utilisé par nos cellules.

  \problematique{
    Quelle est la structure des glucides ?
  }
\end{contexte}



%%%%%
\begin{doc}{Sucres rapides et sucres lents}{doc:TP1_sucre_rapide_lent}
  \begin{importants}
    On classe les glucides en deux catégories :
    \begin{listePoints}
      \item les \important{sucres rapides,} qui sont des molécules simples composées d'un groupe carbonyle et d'au moins 2 groupes hydroxyle. Les sucres rapides sont facilement digérés ;
      \item les \important{sucres lents,} qui sont composés de plusieurs sucres rapides liés entre eux.  
    \end{listePoints}
    L'assimilation des sucres lents par l'organisme est... lente et permet un apport régulier en sucre rapide pendant toute la digestion.
  \end{importants}

  On trouve des sucres rapides dans les fruits, le miel, la farine blanche, le riz blanc et la plupart des sodas et sucreries.
  Les sucres lents se trouvent dans les féculents (pomme de terre, maïs, blé, etc.), les légumineuses (haricot rouge, pois chiche, etc.), la farine complète ou le riz complet.
\end{doc}

\begin{doc}{Le glucose et le fructose sous forme linéaire et cyclique}{doc:TP1_glucose_fructose}
  \begin{multicols}{2}
  \begin{center}
    \separationBlocs{
      \centering
      \chemfig{[:-90]!\glucoseSemiDev}
    }{
      \centering
      \chemfigHaworth{!\glucoseHaw} \\[4pt]
    }
    Glucose

    \separationBlocs{
      \centering
      \chemfig{[:-90] !\fructoseSemiDev} \;
    }{
      \centering
      \chemfigHaworth{!\fructoseHaw} \\[4pt]
    }
    Fructose
  \end{center}
  \end{multicols}
\end{doc}

\question{
  À partir de la forme linéaire du glucose et du fructose, entourer et nommer les fonctions organiques présentes dans ces deux molécules.
}{}[3]

\newpage
\vspace*{-32pt}
\question{
  Donner la formule brute du glucose et du fructose.
  Ces molécules sont-elles isomères ?
}{}[3]


\begin{doc}{Une partie de l'amidon}{doc:TP1_amidon}
  \begin{center}
    \chemfigHaworth{!\amylopectineHaw} \\[8pt]
  
    \important{Amylopectine,} molécule composant l'amidon
  \end{center}
\end{doc}

\question{
  Indiquer, en justifiant, dans quelle catégorie de glucide est l'amidon.
}{}[2]

\begin{doc}{Le test de Fehling}{doc:TP1_test_fehling}
  Le test à la liqueur de Fehling permet de déterminer si une solution contient des fonctions aldéhydes.

  \begin{protocole}
    \item Mélanger \qty{0,5}{\ml} de liqueur de Fehling avec \qty{1}{\ml} de solution à tester dans un tube à essais.
    \item Chauffer le tube à essais contenant le mélange quelques minutes dans un bécher contenant $\sim$\qty{20}{\ml} d'eau pour réaliser un bain-marie.
  \end{protocole}

  \begin{center}
    \begin{tblr}{
      colspec = {X[l] X[l] X[l]}, hlines, vlines,
      column{1} = {couleurSec-50}, row{1} = {couleurSec-100}
    }
      & Présence d'aldéhyde & Présence de cétone \\
      %
      Observation dans le tube à essais &
      Apparition d'un précipité rouge brique &
      La solution reste bleue. \\
    \end{tblr}
  \end{center}
\end{doc}

\mesure
Récupérer le bac contenant le support avec les deux tubes à essais et la pince en bois.
Brancher le réchaud électrique \important{et prévenir votre prof pour un rappel de sécurité} (pas de gants au dessus d'une surface chaude, manipulation du tube chauffé avec une pince en bois).

\mesure
Mettre en œuvre le protocole expérimental du document~\ref{doc:TP1_test_fehling} permettant de distinguer des solutions de glucose et de fructose.

\question{
  Indiquer si un des deux tubes contient du glucose et préciser l'observation qui permet de l'affirmer.
}{}[3]
