\vAligne{-60pt}
\titre{Lire une consigne}

\numeroQuestion Lisez ce test en entier.

\begin{doc}{Test de lecture rapide}{doc:lecture_rapide}
  \begin{enumerate}
    \item Lisez toutes les consignes de ce test.
    \item Écrivez vos nom et prénom dans la colonne de droite.
    \item Effectuez l'opération : 27 + 12 =
    \item Tracez un triangle en haut à droite.
    \item Écrivez deux fois le mot sphinx en bas de la feuille.
    \item Soulignez le mot "Lecture" dans le titre.
    \item Comptez mentalement jusqu'à 14.
    \item Complétez le mot "architect..."
    \item Trouvez un mot qui rime avec "Onyx"
    \item Complétez la suite 1 3 5 7
    \item Soulignez tous les mots "Complétez" du texte.
    \item Quelle est la date de la prise de la Bastille ?
    \item Quel est le contraire de "Belliqueux"?
    \item Ne faites que la consigne n°2.
  \end{enumerate}
\end{doc}

\titreSection{Qu'est-ce qu'une \og consigne \fg\; ?}

- Proposez votre définition :
\lignesDeReponse{2}

- Donnez des exemples :
\lignesDeReponse{2}

- Copiez une définition trouvée dans un dictionnaire.
\lignesDeReponse{2}

- Mise en commun.
\lignesDeReponse{4}

\begin{itemize}
  \item Est-ce qu'une consigne est importante ? Pourquoi ?
  \item Lisez-vous toujours soigneusement les consignes ?
  \item Comprenez-vous toujours les consignes ?
  \item Que se passe-t-il si vous ne comprenez pas la consigne ?
\end{itemize}

\begin{doc}{Les formes de consignes}{doc:formes_consignes}
  \begin{listePoints}
    \item Une consigne c'est un énoncé \important{injonctif} qui commande à l'élève une tâche. 
    La consigne est généralement courte et précise.
    Elle peut être orale, écrite, gestuelle ou implicite (le ou la prof lance une vidéo, cela veut dire que l'on écoute pour essayer de comprendre).
    \item \important{Consigne simple} et \important{consigne composée}
    \begin{listeTirets}
      \item  La \important{consigne simple} implique une activité unique et limitée: \exemple Tracez la droite D passant par le point A.
      Activité préparatoire à un processus de raisonnement en lycée.
      Ce n'est que le début et non l'enjeu visé.
      C'est une consigne fermée :
      \exemple Relevez le champ lexical de la peur (Consigne de Français de niveau début collège).
      \item La \important{consigne composée} implique une tâche complexe identifiée grâce à ses compétences de lecteur.
      D'où un \important{problème} pour certains.
      Il faut repérer l'ordre des tâches; hiérarchiser les étapes qui ne sont pas toujours rappelées contrairement au collège.
      Elles vous demandent, à vous élèves, d'analyser les difficultés de réalisation de la tâche, de construire un processus, donc de mettre en place une stratégie.
      La consigne ouverte implique que vous ne soyez plus seulement "exécutantes" mais vous adoptiez un comportement de recherche et d'analyse.
    \end{listeTirets}
  \end{listePoints}
  
  Le risque pour vous avec ces consignes "ouvertes" ou comportant de l'implicite, c'est de ne pas comprendre ce que le professeur attend de vous.
  Le contrat ne vous semble pas clair; la consigne ne rencontre pas d'écho en vous.
  
  \exemple \og Montrez que les points A, B, C ne sont pas alignés \fg. \og Que pouvez vous dire de ces points A, B, C ? \fg

  Au lycée les consignes sont moins guidées et laisse place à l'implicite.
  Il existe encore des consignes d'aides, auxiliaires, mais au service du produit final attendu.
  
  \exemple en Français : Donnez deux axes d'étude pour ce récit (Consigne de procédure).
  Écrivez un paragraphe argumentatif qui expliquera l'utilisation de l'ironie dans ce texte ( Consigne de critère formel).

  \begin{encart}  
    Le premier \important{objectif de l'élève}
    c'est de \important{comprendre les consignes}
    et donc d'être une \important{lectrice compétente}.
    Et ce quelle que soit la discipline !
  \end{encart}
\end{doc}

\titreSection{Progresser dans la lecture des consignes.}

\numeroQuestion Lister les difficultés pour trouver les leviers.
\begin{listePoints}
  \item \textbf{Mauvais comportement de lecteur :} Vous survolez la consigne; vous la devinez plus que vous n'en repérez rigoureusement les indices. Vous avez alors une représentation erronée de la tâche à accomplir. En particulier pour les consignes orales : manque d'écoute, de réflexion pendant l'écoute. A l'écrit il faut une lecture "profonde", les Anglais disent "deep attention".
  \item \textbf{Mauvaise lecture :} vous ratez une ponctuation, une nuance de syntaxe. \og Repérez les mouvements de population indiqués sur les 2 cartes : expliquez-les par la situation géopolitique de la région \fg. Difficulté de compréhension du vocabulaire : un mot technique n'est pas connu, parfois à cause d'un manque de travail en amont du cours : chaque discipline a son jargon propre. \og Étudiez le registre du texte. Relevez les modaux dans les lignes \fg...
  \item \textbf{Polysémie de certains mots :} "encadrer" faire un cadre; encadrer un nombre...
  \item \textbf{Verbes flous :} Comparez, Recherchez, Commentez ... Ces injonctions n'ont pas le même écho en fonction de la discipline. C'est l'écoute et la participation au cours, en amont, qui vous permettent de retrouver ce que l'on entend par "Commentez" en Histoire, en BPH, en physique-chimie etc.
  \item \textbf{Difficulté à relier la consigne et les savoirs étudiés en amont.} Vous ne reconnaissez pas le réinvestissement attendu par votre professeur. Le lien entre l'exercice et la partie du cours en jeu ne vous apparait pas spontanément. Souvent pour toutes les raisons données précédemment.
\end{listePoints}

\numeroQuestion Exercices et réflexes à mettre en place.

\hspace{16pt}\pointCyan \textbf{Énonciation spécifique de la consigne :} Repérez dans les énoncez suivants ceux qui sont des
consignes.

\begin{encart}
  Le test Elisa consiste à rechercher dans le sérum d'un individu des anticoprs anti VIH (ils sont détectables 3 à 4 semaines après la contamination).
  Un autre test, le western blot, permet de rechercher des antigènes du VIH avant l'apparition des anticorps. 
  Indiquez la principale différence entre les tests western blot et Elisa.
\end{encart}
\begin{encart}
  Le triple du carré d'un nombre entier est égal au double de ce nombre. Quel est ce nombre ?
\end{encart}
\begin{encart}
  Ce texte est le début de l'autobiographie de Jean-Jacques Rousseau;
\end{encart}

\hspace{16pt}\pointCyan \textbf{Syntaxe de la consigne :} Repérez les verbes et termes injonctifs centres de la consigne et l'objet de l'action ou le thème à traiter.

\begin{encart}  
  Timbre émis par la République du Sénégal en 1961 pour le premier anniversaire de l'indépendance du pays.
  Montrez que l'image traduit l'indépendance du Sénégal mais aussi l'influence de la France sur ce pays.
\end{encart}


\hspace{16pt}\pointCyan \textbf{Repérez les indicateurs numériques, spatiaux, logiques, etc.
Important pour le nombre de tâches à accomplir, l'ordre des tâches (à numéroter) :} repérer en particulier les \og \important{et / ou / puis / après} \fg.

\begin{encart}
  Démontrez, d'après leur position et leur structure, que ces organes articulés sont des membres postérieurs réduits chez la baleine actuelle.
\end{encart}

\begin{encart}
  Trouvez un argument dans le texte afin de valider l'hypothèse selon laquelle la vision des couleurs est un caractère héréditaire.
\end{encart}

\begin{encart}  
  Relevez sur la carte les principales aires urbaines françaises. L'évolution de l'aire urbaine nantaise est-elle un cas unique ?
\end{encart}

\begin{encart}  
  ¿En qué medida puede ser una experiencia enriquecedora el salir de sus propias fronteras? (15 líneas)
\end{encart}


\hspace{16pt}\pointCyan \textbf{Vocabulaire spécifique :}
S'approprier en fonction des disciplines les termes indiquant les activités mentales attendues.
Réaliser le glossaire par discipline " Observer, analyser, comparer, caractériser, démontrer, justifier,
classer, nommer, situer, interpréter, calculer, rédiger, discuter, justifier, expliquer, argumenter, mettre en évidence, dans quelle mesure..."

\begin{encart}  
  Expliquez le fonctionnement de l'appareil de Golgi.
\end{encart}

\begin{encart}
  Expliquez pourquoi la molécule d'eau est polaire.
\end{encart}

\begin{encart}
  Comparez les conceptions du voyage qui s'expriment dans ces textes   
\end{encart}


\hspace{16pt}\pointCyan \textbf{Modéliser la tâche à accomplir :}
anticiper le résultat, les étapes de réalisation, les ressources et savoirs à utiliser.

\begin{encart}
  L'homme représenté sur la peinture adresse à la foule un discours prophétique. Imaginez ce discours et rapportez le indirectement.
\end{encart}

\begin{encart}
  Imaginez la lettre qu'aurait pu adresser Ionesco à un metteur en scène de sa pièce à propos du dénouement.
  Dans cette lettre, il explique comment, selon lui, l'actrice doit jouer le rôle de Marguerite et précise les éléments de mise en scène qui accompagnent la mort du roi.
  Rédigez cette lettre en vous fondant sur vos expériences personnelles de spectateur et vos lecture
\end{encart}


\hspace{16pt}\pointCyan \textbf{Formuler la réponse de manière adéquate} en fonction des "habitudes" de la discipline.


\titreSection{S'exercer à rédiger des consignes pour ses camarades}

\begin{tblr}{
  colspec = {|X[l] |X[l] |}, hlines
}
  Formuler la question dont la réponse serait : &
  En premier lieu, le personnage suscite le rire par l'excès de sa réaction face au vol;
  il semble aimer autant son argent qu'il pourrait aimer un ami, 
  puisqu'il le personnifie et emploie des termes et des adjectifs affectueux...
  Dans un second temps, nous observons que la perte de son argent lui est si douloureuse que Harpagon est victime d'hallucinations ridicules.
  La didascalies nous informe qu'il pense arrêter le voleur en se prenant "lui-même le bras".
  À ces comiques de caractère, de mots et de gestes s'ajoute enfin le comique de répétition. \\
  Dicter un programme de construction de la figure. & 
  \centering
  \image{1}{AP/figure_consigne}
\end{tblr}