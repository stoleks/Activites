\begin{center}
  \large \sousTitre{Sororité : La solidarité politique entre les femmes}
\end{center}

Les femmes constituent le principal groupe victime de l'oppression sexiste. À l'instar d'autres formes d'oppression collective, le sexisme est perpétué par des structures institutionnelles et sociales ; par les individus qui dominent, exploitent ou oppriment ; et par les victimes elles-mêmes, amenées par la socialisation à adopter des comportements qui les rendent complices du statu quo.
[...]
%L'idéologie de la suprématie masculine incite les femmes à penser qu'elles ne valent rien tant qu'elles ne sont pas liées ou unies à des hommes.
On nous enseigne que les relations que nous entretenons les unes avec les autres amoindrissent notre expérience au lieu de l'enrichir.
On nous enseigne que les femmes sont \og naturellement \fg\; ennemies des femmes, que la solidarité n'existera jamais entre nous parce que nous ne pouvons et ne devons pas nous unir les unes aux autres.
Nous avons bien appris ces leçons.
Nous devons les désapprendre pour construire un mouvement féministe durable.
Nous devons apprendre à vivre et à travailler dans la solidarité.
Nous devons apprendre le véritable sens et la vraie valeur de la sororité.
[...] \bigskip

% Alors que le mouvement féministe contemporain aurait dû former les femmes à la solidarité politique, la sororité n'a pas été envisagée comme un accomplissement révolutionnaire que les femmes s'efforceraient d'atteindre par la lutte. Telle que la concevaient les mouvements de libération des femmes, la sororité se fondait sur l'idée d'une oppression commune. Il va sans dire que ce furent surtout les femmes de la bourgeoisie blanche, de tendance libérale ou radicale, qui cultivèrent la notion d'oppression commune. L'\og oppression commune \fg\; était un mot d'ordre mensonger et malhonnête qui masquait la véritable nature de la réalité sociale vécue par les femmes, sa complexité et sa variété. Les attitudes sexistes, le racisme, les privilèges de classe et toute une kyrielle d'autres préjugés divisent les femmes. Elles ne peuvent s'unir durablement qu'à la condition de reconnaître ces divisions et de prendre les mesures nécessaires à leur élimination. Certes, il est important de mettre en lumière les expériences vécues par l'ensemble des femmes, mais il existe aussi des clivages, et ce n'est pas avec des v\oe{}ux pieux et de belles idées romantiques qu'on les fera disparaître.

Depuis quelques années, la \og sororité \fg\; telle qu'elle s'exprime dans les slogans, les devises ou les cris de ralliement féministes ne suggère plus que l'union fait la force. Certaines militantes semblent désormais penser que nous ne pouvons nous unir, étant donné nos différences. 
[...]
%Mais en abandonnant la notion de sororité pour exprimer la solidarité politique, on affaiblit le mouvement féministe. 
%La solidarité renforce la lutte de résistance.
Il ne peut y avoir de mouvement féministe de masse contre l'oppression sexiste sans un front uni : les femmes doivent prendre l'initiative et démontrer la force de la solidarité. 
[...]
%Si nous ne parvenons pas à montrer que les barrières séparant les femmes peuvent être éliminées, que la solidarité peut exister, nous ne pouvons espérer transformer la société dans son ensemble.
La sororité est passée à l'arrière-plan parce que beaucoup de femmes, irritées par les grands discours sur \og l'oppression commune \fg, l'identité partagée et la ressemblance, ont critiqué, voire rejeté, le mouvement féministe dans son ensemble.
L'appel à la sororité a en effet souvent été perçu comme une man\oe{}uvre manipulatrice et opportuniste des bourgeoises blanches, un vernis rhétorique servant à masquer l'exploitation et l'oppression perpétuées par des femmes sur d'autres femmes.
[...] \bigskip

S'il est vrai que nous avons beaucoup à gagner à nous unir, nous ne pouvons pourtant pas développer de liens durables ni de véritable solidarité politique à partir du modèle de sororité créé par la tendance bourgeoise du féminisme.
Pour ce courant, l'union des femmes se fonde sur une expérience collective de la victimisation, d'où l'importance de la notion d'oppression commune.
Cette conception du lien entre les femmes reflète directement la pensée de la suprématie masculine blanche.
L'idéologie sexiste enseigne aux femmes que la féminité implique d'être une victime.
Au lieu de rejeter cette équation (qui ne rend pas compte de l'expérience féminine, car dans leur vie quotidienne la plupart des femmes ne sont pas constamment des \og victimes \fg\; passives et vulnérables), les féministes y ont souscrit, faisant de la condition de victime le dénominateur commun qui permet aux femmes de s'unir : les femmes devaient se concevoir comme des \og victimes \fg\; pour se sentir concernées par le mouvement féministe.
[...]
%L'union des femmes-victimes semblait impliquer que les femmes sûres d'elles-mêmes et indépendantes n'avaient pas leur place dans le mouvement féministe.
C'est cette logique qui a amené plus d'une militante blanche (aux côtés des hommes noirs) à suggérer que les femmes noires étaient si \og fortes \fg\; qu'elles n'avaient pas besoin de s'impliquer dans le mouvement féministe.
Et c'est pour cela que beaucoup de femmes blanches ont quitté le mouvement quand elles ont cessé de se représenter comme des victimes.
L'ironie est que les femmes qui ont le plus revendiqué le statut de \og victimes \fg\; étaient souvent plus privilégiées et avaient plus de pouvoir que la grande majorité des femmes de notre société.
Les travaux sur les violences faites aux femmes permettent d'éclairer ce paradoxe.
Les femmes qui subissent quotidiennement l'exploitation et l'oppression ne peuvent se permettre de renoncer au sentiment qu'elles exercent un tant soit peu de contrôle sur leur vie.
Elles ne peuvent se permettre de se penser simplement comme des \og victimes \fg, car leur survie dépend de leur capacité à exercer sans relâche le peu de pouvoir personnel dont elles disposent.
Ces femmes compromettraient leur équilibre psychologique si elles s'associaient à d'autres femmes sur la base d'une condition victimaire commune.
C'est sur la base de forces et de ressources communes qu'elles s'associent à d'autres femmes : tel est le type de lien qui constitue l'essence de la sororité.
\bigskip

À partir du moment où les féministes se définissaient comme une association de \og victimes \fg, elles n'étaient pas tenues de se confronter à la complexité de leur propre expérience.
Elles ne se sentaient pas obligées de se remettre en question, de s'interroger sur l'influence du sexisme, du racisme et des privilèges de classe dans leur perception des femmes qui ne faisaient pas partie de leur groupe racial et social.
Le fait de s'identifier comme \og victimes \fg\; leur permettait d'abdiquer toute responsabilité dans la construction et la perpétuation du sexisme, du racisme et de l'exclusion sociale, ce qu'elles firent en insistant pour que seuls les hommes soient considérés comme des ennemis.
Elles évitaient ainsi de reconnaître l'ennemi intérieur et de s'y confronter.
Elles n'étaient pas prêtes à renoncer à leurs privilèges et à effectuer le \og sale boulot \fg\; indispensable au développement d'une conscience politique radicale (c'est-à-dire la lutte et la confrontation que réclame la politisation, et toutes les tâches fastidieuses qui font partie du quotidien militant) : ce travail doit commencer par une critique et une évaluation personnelle honnêtes de son statut social, de ses valeurs, de ses croyances politiques, etc.
La sororité a donc fini par devenir un nouveau moyen de fuir la réalité.
Cette conception de la solidarité entre femmes était déterminée par une certaine représentation de la féminité blanche, fondée sur des préjugés de classe et de race : il fallait protéger la lady blanche, la bourgeoise, de tout ce qui aurait pu la déranger ou la déstabiliser en la mettant à l'abri des réalités négatives susceptibles de conduire à la confrontation.
En ce sens, la sororité prescrivait aux s\oe{}urs un amour mutuel \og inconditionnel \fg\; ; elles devaient éviter le conflit et minimiser les dissensions ; elles ne devaient pas se critiquer les unes les autres, surtout en public.
Pendant un temps ces règles créèrent une illusion d'unité qui neutralisa les rivalités, l'hostilité, les désaccords perpétuels et la critique outrancière (l'invective), qui étaient souvent la norme dans les groupes féministes.
Aujourd'hui, cette interprétation de la sororité se retrouve dans de nombreux sous-groupes constitués sur des identités communes ([...] universitaires blanches, anarcha-féministes, etc.) ; si leurs membres se soutiennent et se protègent mutuellement, elles considèrent en revanche avec une hostilité souvent incroyable les femmes qui ne font pas partie de leur groupe.
La manière dont ces femmes unies dans un cercle choisi renforcent leur solidarité en excluant et en dévalorisant les étrangères relève d'un type de relations féminines propre au système patriarcal : la seule différence est que cette exclusion se pratique au nom du féminisme.
\bigskip

Les militantes féministes ne développeront pas la solidarité politique entre femmes en reprenant à leur compte les conceptions validées par l'idéologie culturelle dominante.
Nous devons poser nos conditions.
Au lieu de nous unir sur la base d'une condition victimaire universelle ou par rapport à un ennemi commun fictif, nous pouvons nous rassembler autour de l'engagement politique dans un mouvement féministe expressément destiné à éradiquer l'oppression sexiste.
Alors, nos énergies ne seraient pas monopolisées par la lutte pour l'égalité avec les hommes ou par la seule résistance à la domination masculine.
Nous ne nous contenterions plus des explications simplistes de la structure de l'oppression sexiste -- les braves filles contre les vilains garçons.
Pour pouvoir résister à la domination masculine, nous devons rompre avec le sexisme, travailler pour transformer la conscience des femmes.
En réfléchissant ensemble sur la socialisation sexiste pour nous en libérer, nous nous renforcerions mutuellement et nous construirions une base solide à partir de laquelle développer la solidarité politique.
\bigskip

Entre hommes et femmes, le sexisme prend en général la forme de la domination masculine et de ses corollaires -- discrimination, exploitation, oppression.
Mais les valeurs suprémacistes masculines se traduisent également dans la méfiance, le peur et la concurrence qui opposent les femmes les unes aux autres.
C'est le sexisme qui conduit les femmes à se percevoir comme des menaces les unes pour les autres sans raison apparente.
Le sexisme leur enseigne à être des objets sexuels pour les hommes ; mais quand des femmes qui ont rejeté ce rôle considèrent avec hauteur et mépris celles qui \og n'en sont pas là \fg, elles restent sous l'emprise du sexisme.
Le sexisme conduit les femmes à dénigrer les tâches parentales en survalorisant leur emploi et leur carrière.
De même, c'est parce qu'elles adhèrent à l'idéologie sexiste que certaines femmes enseignent à leurs enfants qu'il n'existe que deux types de schémas comportementaux : la domination ou la soumission.
Le sexisme apprend aux femmes à détester les femmes, et, consciemment ou non, nous ne cessons de mettre cette leçon de haine en pratique dans nos échanges quotidiens. [...]
\bigskip

Partout aux états-Unis, des femmes consacrent chaque jour une bonne partie de leur temps à s'en prendre verbalement à d'autres femmes en se livrant à des commérages malveillants (à ne pas confondre avec les aspects positifs du bavardage).
À la télévision, les séries et comédies dramatiques ne cessent de nous montrer des relations féminines dominées par l'agressivité, le mépris et la rivalité.
Dans les cercles féministes, le sexisme se manifeste à travers le dédain, l'indifférence et les commentaires malveillants dirigés contre les femmes qui n'ont pas intégré le mouvement.[...]
\bigskip
%Cette tendance apparaît tout particulièrement à l'université, où l'on considère souvent les études féministes comme une discipline ou un programme sans lien avec le mouvement féministe. Dans son allocution inaugurale à Barnard College en mai 1979, l'écrivaine noire Toni Morrisson déclarait :

    %\og J'ai envie de vous dire (pas de vous demander, mais de vous dire) de ne pas participer à l'oppression de vos s\oe{}urs. Les mères qui maltraitent leurs enfants sont des femmes, et ce n'est pas une institution, mais une autre femme, qui doit se dévouer pour les en empêcher. Les mères qui mettent le feu à des bus scolaires sont des femmes, et ce n'est pas une institution, mais une autre femme, qui doit leur dire de ne pas aller au bout de leur geste. Les femmes qui bloquent la promotion des carrières d'autres femmes sont des femmes, et c'est une autre femme qui doit venir au secours de la victime. Les employés des services sociaux qui humilient leurs clientes sont parfois des femmes, et il appartient à leurs collègues féminines de désamorcer leur colère. Je trouve inquiétante la violence avec laquelle les femmes se comportent entre elles : violence au travail, violence de la compétition, violence affective. Je trouve inquiétant de voir des femmes disposées à réduire d'autres femmes en esclavage. Je trouve inquiétante l'indécence croissante avec laquelle les femmes de pouvoir n'hésitent pas à se comporter en tueuses. \fg


Pour construire un mouvement féministe politisé et représentatif, les femmes doivent redoubler d'efforts afin de surmonter leur aliénation mutuelle, qui persistera tant que la sociabilisation sexiste n'aura pas été désapprise et qui se traduit par l'homophobie, la tendance à juger d'après les apparences, les conflits entre femmes ayant des pratiques sexuelles différentes.
Jusqu'à présent, le mouvement féministe n'a pas réussi à transformer les relations de femme à femme, surtout lorsqu'elles sont étrangères l'une à l'autre ou viennent de milieux sociaux différents -- alors qu'il a permis de tisser des liens individuels et collectifs.
Il faut aider les femmes à désapprendre le sexisme : c'est la condition pour construire des relations personnelles fortes et, au-delà, l'unité politique.
\bigskip

Le racisme est une autre limite à la solidarité entre femmes. L'idéologie de la sororité telle que l'ont formulée les militantes féministes contemporaines n'a jamais admis que la discrimination raciste, l'exploitation et l'oppression des femmes issues des minorités ethniques par les femmes blanches empêchent ces deux groupes de se rassembler autour d'intérêts politiques communs.
\bigskip

%En outre, les différences culturelles peuvent rendre la communication problématique, et c'est particulièrement vrai des relations entre femmes noires et femmes blanches. Au cours de l'histoire, les premières furent nombreuses à faire l'expérience des inégalités raciales à travers l'autorité directe que les secondes exerçaient sur elles, de manière souvent plus brutale et plus déshumanisante que les hommes blancs racistes. Aujourd'hui, bien que la domination soit essentiellement exercée par des hommes adhérant aux idées patriarcales et suprémacistes, ce sont souvent des femmes blanches qui représentent le supérieur immédiat ou la figure d'autorité à laquelle sont confrontées les femmes noires sur le plan professionnel. Conscientes des privilèges que la domination raciale offre aux blancs des deux sexes, les femmes noires n'ont pas tardé à réagir aux appel à la sororité en soulignant qu'il était contradictoire de leur demander de participer à la libération de celles qui les exploitaient. Nous sommes nombreuses à avoir interprété l'appel à la sororité comme une invitation à soutenir un mouvement qui ne s'adressait pas à nous. [...] Nombreuses à avoir eu l'impression que le mouvement de libération des femmes servait les intérêts des bourgeoises blanches aux dépens des femmes pauvres issues des classes populaires, souvent noires. La sororité reposait donc sur des bases bien fragiles et pour les femmes noires, ç'aurait été faire preuve de naïveté politique que de rejoindre ce mouvement. Cependant, hier comme aujourd'hui, les luttes à travers lesquelles s'est construite la participation politique des femmes noires suggèrent qu'il aurait mieux valu insister sur le développement de la solidarité politique et clarifier le sens de cette notion.

%Tout en pratiquant la discrimination et l'exploitation des femmes noires, les blanches les considèrent avec envie et se posent comme leurs rivales. Aucun de ces processus d'interaction ne crée les conditions propices au développement de relations de confiance et de réciprocité. Après avoir oublié le racisme dans la théorie et la praxis féministes, les blanches ont laissé à d'autres la responsabilité d'attirer l'attention sur la race. Ne prenant aucune initiative dans les débats sur le racisme ou les privilèges raciaux, elles pouvaient réagir à ce que disaient les femmes non blanches intervenant sur ces sujets, sans pour autant changer quoi que ce soit à la structure du mouvement féministe et sans perdre leur hégémonie. Elles pouvaient insister sur la nécessité d'augmenter le nombre de femmes de couleur dans les organisations féministes, les encourager à participer davantage, mais elles ne s'attaquaient jamais de front au racisme. [...]

%Si le racisme est un enjeu primordial pour les féministes, ce n'est pas seulement parce qu'il existe parmi les militantes blanches. Ces dernières ne représentent qu'une fraction des femmes de la société. Quand bien même elles auraient toutes été antiracistes dès le départ, l'élimination du racisme n'en resterait pas moins un enjeu essentiel du féminisme. Le racisme est un problème fondamental pour les féministes car il est indissociable de l'oppression sexiste. En Occident, les fondements philosophiques du racisme et du sexisme sont identiques. Influencées par des valeurs ethnocentriques, les théoriciennes féministes ont été amenées à faire passer le sexisme avant le racisme, élaborant ainsi une conception de l'évolution culturelle qui ne correspond en rien à notre expérience de vie. Aux états-Unis, le maintien de la suprématie blanche a toujours été une priorité au moins aussi importante que le maintien d'une stricte division des rôles sexuels. Il n'y a rien d'étonnant à ce que l'intérêt pour les droits des femmes s'exacerbe chaque fois que surgit un mouvement de masse antiraciste. Nul n'est naïf au point d'ignorer que si un état dominé par la suprématie blanche est sommé de satisfaire les besoins des noirEs oppriméEs et/ou les besoins des femmes blanches (et notamment celles de la bourgeoisie), il sera dans son intérêt de satisfaire les blancs. Un mouvement radical visant à mettre fin au racisme (une cause pour laquelle tant de gens sont morts) est bien plus menaçant qu'un mouvement conçu pour permettre aux femmes blanches de s'élever dans la société.

Reconnaître l'importance de la lutte antiraciste ne diminue nullement la valeur ou la nécessité du mouvement féministe. La théorie féministe serait d'une grande utilité si elle montrait aux femmes comment le racisme et le sexisme s'articulent au lieu d'opposer le combat contre ces deux formes d'oppression ou d'évacuer purement et simplement la question du racisme.
L'un des principaux enjeux du combat féministe portait sur le droit des femmes à contrôler leur corps.
Or le concept de la suprématie blanche repose sur l'idéologie de la perpétuation de la race blanche.
Le maintien de la domination du monde par la race blanche passe par le contrôle patriarcal du corps de toutes les femmes.
Une militante qui s'efforce quotidiennement d'aider les femmes à obtenir le contrôle de leurs corps ne peut être raciste sans nier et saper son propre combat.
Quand les femmes blanches s'attaquent à la suprématie blanche, elles participent simultanément à la lutte contre l'oppression sexiste.
Ce n'est qu'un exemple de la manière dont l'oppression raciste et l'oppression sexiste se recoupent et se complètent.
Il y en a bien d'autres, et les théoriciennes féministes devraient les examiner.[...]
\bigskip 

\begin{flushright}
  Bell Hooks
\end{flushright}

\vfill

{\small
  Ce texte est paru en 1986 dans le n$^\circ$23 de « Feminist Review », sous le titre original : ``Sisterhood : Political Solidarity between Women''.
  
  Il s’agit d’une version remaniée du chapitre 4 de « Feminist Theory : from Margin to Center », South End Press, Boston, 1984.
  
  Cette traduction en français, par Anne Robatel, a été publiée dans « Black feminism, Anthologie du féminisme africain-américain, 1975-2000 », L’Harmattan, Bibliothèque du féminisme, Paris, 2008.
}