\tetePremStssVisi
\titreActivite{Vision humaine et modèle optique}

%%%% objectifs
\begin{objectifs}
  \item Comprendre le mécanisme de la vision chez les humains.
  \item Comprendre la formation des images sur la rétine.
\end{objectifs}

\begin{contexte}
  La vue est un des sens les plus important chez nous, elle permet d'observer notre environnement grâce à nos yeux.
  Les yeux forment des images réelles, interprétées par notre cerveau pour former des images mentales.
  
  \problematique{
    Comment est formée une image sur la rétine et comment le cerveau interprète cette information pour donner notre vision ?
  }
\end{contexte}

%%%% docs
\begin{doc}{Modélisation de l’œil}
  \begin{wrapfigure}{r}{0.5\linewidth}
    \vspace*{-30pt}
    \centering
    \image{1}{images/lumiere/modele_oeil_optique}

    \vspace*{-12pt}
    \small{Schéma simplifié de l’œil et de sa modélisation optique.}
  \end{wrapfigure}

  Dans l’œil, la lumière se propage à travers plusieurs milieux \important{transparents} et \important{homogènes} pour former une image sur la \important{rétine :}
  \begin{listePoints}
    \item la \important{cornée} est une membrane transparente qui concentre la lumière sur le cristallin ;
    \item le \important{cristallin} est une lentille élastique ;
    \item \important{l'humeur} aqueuse est liquide ;
    \item \important{l'humeur} vitrée est gélatineuse.
  \end{listePoints}

  En optique, on modélise l’œil par un système simple :
  \begin{listePoints}
    \item \important{L'iris} est modélisé par un \important{diaphragme} ;
    \item \important{Le cristallin} et les milieux transparents sont modélisés par une lentille convergente ;
    \item \important{La rétine} est modélisée par un écran.
  \end{listePoints}
\end{doc}

\question{
  Compter le nombre de milieux traversés et le nombre de réfractions subies par la lumière avant d'arriver sur la rétine.
}{}[2]

\question{
  Rappeler comment la lumière se propage dans un milieu transparent homogène.
}{}[2]


\begin{doc}{L'iris et le cristallin}
  \important{L'iris} est une membrane circulaire qui permet à l’œil de régler le diamètre de la \important{pupille.} 
  
  Quand le diamètre de la pupille augmente, la lumière entre davantage dans l’œil.
  Cette dilatation permet d'avoir une image plus lumineuse, mais diminue la netteté de l'image.

  L'iris se contracte principalement quand la luminosité ambiante augmente, grâce à deux muscles opposés : un système radial dilatateur (fibres orientées vers l'extérieur comme les rayons d'une roue) et un un système concentrique constricteur (fibres disposées de façon concentrique, en sphincter).

  Le \important{cristallin} est une lentille déformable dans l'œil, placée juste derrière l'iris.
  Le cristallin concentre les rayons lumineux pour former une image des objets observés sur la rétine.
  Pour pouvoir ajuster la netteté des objets observés, le cristallin peut se déformer grâce à des muscles, c'est ce qui permet notamment de voir net de près.
\end{doc}

\question{
  Expliquer pourquoi on visualise mieux les détails d'un objet quand il est bien éclairé.
}{}[3]


\begin{doc}{La rétine}
  La lumière qui traverse l’œil forme une image nette de l'objet observé sur la \important{rétine.}
  
  La rétine est constitué de deux types de terminaisons nerveuses photosensibles :
  \begin{listePoints}
    \item les \important{bâtonnets,} qui sont sensibles pour 1 longueur d'onde et permettent de voir quand il y a une faible luminosité ;
    \item les \important{cônes,} qui sont sensible dans 3 longueurs d'onde différentes et permettent d'avoir une vision en couleur.
  \end{listePoints}
  S'ils reçoivent de la lumière dans leur gamme de sensibilité, les cônes et les bâtonnets émettent un signal dans le \important{nerf optique,} permettant à notre cerveaux de reconstituer une image en couleur.
\end{doc}

\question{
  Expliquer pourquoi on distingue moins bien les couleurs dans un endroit sombre.
}{}[3]


\begin{doc}{Défaut de l'œil et de la vision}
  Les troubles de la visions sont dû à des défauts de l'œil.
  Si c'est la forme de l'œil ou le cristallin qui a des défauts, la personne aura des problèmes de netteté (images floues).
  Si ce sont la rétine qui a des défauts, la personne aura des problèmes avec la vision des couleurs ou sera partiellement aveugle.

  Quelques exemples de troubles commun :
  \begin{listePoints}
    \item \important{la myopie,} qui empêche de voir net de loi. À cause d'un défaut dans la forme de l'œil ou le cristallin, la formation des images se fait avant la rétine.
    \item \important{l'hypermétropie,} qui empêche de voir net de près, pour des raisons inverses de la myopie, l'image est formée après la rétine.
    \item \important{la presbytie,} qui est lié à un durcissement du cristallin en vieillissant, ce qui empêche d'effectuer une mise au point pour voir net de près.
    \item \important{le daltonisme,} qui empêche de distinguer certaines couleurs. Les différentes formes de daltonisme sont liées à des défauts dans les cônes.
  \end{listePoints}
\end{doc}