%%%%
\tetePremStssVisi
\titreTP{Formation d'une image}

%%%% objectifs
\begin{objectifs}
  \item Comprendre la formation des images par une lentille.
\end{objectifs}

\begin{contexte}
  Les lentilles permettent de former des images réelles (sur un écran) ou virtuelle (en regardant au travers de la lentille).
  
  \problematique{
    Comment se forment des images à travers une lentille ?
  }
\end{contexte}

%%%%
\begin{doc}{Les lentilles convergentes et divergentes}{doc:TP1_lentilles}
  \begin{wrapfigure}[10]{r}{0.4\linewidth}
    \centering
    \vspace*{-42pt}
    \image{0.6}{images/lumiere/schema_lentilles_conv}
    \image{0.6}{images/lumiere/schema_lentilles_div}
  \end{wrapfigure}
  
  Il existe deux types de lentilles en optique : 
  \begin{listePoints}
      \item les lentilles \important{convergentes,} qui concentrent les rayons lumineux.
      Elles sont plus épaisses au centre qu'aux extrémités et sont schématisées par une double flèche fermée.
      \item Les lentilles \important{divergentes,} qui étalent les rayons lumineux.
      Elles sont plus fines au centre qu'aux extrémités et sont schématisée par une double flèche ouverte.
  \end{listePoints}

  On peut déterminer le trajet de la lumière qui passe dans ces lentilles en utilisant les lois de Snell-Descartes, mais on va voir que la lumière suit trois règles simples en les traversant.
\end{doc}

\begin{doc}{Modélisation de la propagation de la lumière par des lentilles}{doc:TP1_formation_image}
  % \begin{wrapfigure}[8]{r}{0.5\linewidth}
  %   \vspace*{-18pt}

  % \end{wrapfigure}
  
  En optique une lentille est modélisée avec
  \begin{listePoints}
    \item son \important{axe optique}, perpendiculaire à la lentille et passant en son centre O, appelé \important{centre optique}. L'axe optique est orienté dans le sens de propagation de la lumière ;
    \item son \important{foyer objet $F$} et \important{son foyer image $F'$}, qui sont équidistants au centre O et placés sur l'axe optique.
  \end{listePoints}

  \begin{boite}
    \vspace{-4pt}
    \important{Vocabulaire :}
    \vspace{-8pt}
    \begin{importants}
      \begin{listePoints}
        \item Un \important{rayon incident} va vers la lentille.
        \item Un \important{rayon émergent} s'éloigne de la lentille.
        \item La \important{distance focale $f'$} est la distance entre $O$ et $F'$, $OF'$.
      \end{listePoints}
    \end{importants}
    \vspace{-4pt}
    
    $f'$ est négative pour une lentille divergente et positive pour une lentille convergente.
  \end{boite}    

  Trois rayons ont des propriétés particulières, communes au lentilles convergentes et divergentes :
  
  \begin{listePoints}
    \item Tout rayon incident qui passe par le centre optique n'est pas dévié.
    \begin{center}
      \image{0.4}{images/lumiere/conv_centre}
      \image{0.4}{images/lumiere/div_centre}
    \end{center}
  
    \item Tout rayon (ou son prolongement) qui passe par le foyer objet $F$ émerge parallèle à l'axe optique.
    \begin{center}
      \image{0.4}{images/lumiere/conv_foyer_objet}
      \image{0.4}{images/lumiere/div_foyer_objet}
    \end{center}
  
    \item Pour tout rayon incident qui arrive parallèle à l'axe optique, le rayon émergent (ou son prolongement) passe par le foyer image $F'$.
    \begin{center}
      \image{0.4}{images/lumiere/conv_foyer_image}
      \image{0.4}{images/lumiere/div_foyer_image}
    \end{center}
  \end{listePoints}
\end{doc}

%%
\mesure
Tracer le faisceau émergent dans les deux cas suivants.
\begin{center}
  \image{0.4}{images/lumiere/conv_faisceau_para}
  \image{0.4}{images/lumiere/div_faisceau_para}
\end{center}

\begin{doc}{Construire l'image d'un objet}{doc:TP1_construction_image}
  \begin{wrapfigure}[7]{r}{0.7\linewidth}
    \vspace*{-34pt}
    \centering
    \image{1}{images/lumiere/formation_image_lentille_conv0002}
  \end{wrapfigure}
  Pour construire l'image $A'B'$ d'un objet $AB$ à travers une lentille convergente, on cherche l'image du point $B$ en traçant deux des trois rayon incidents particuliers, avec leur rayons émergents.

  L'intersection des rayons émergents donne l'image réelle $B'$.
  Si on doit utilise les prolongements des rayons émergents, l'image est virtuelle.

  Le rapport de la taille de l'image sur la taille de l'objet est appelée le \important{grandissement}, noté \important{$\gamma$} (« gamma ») :
  \begin{equation*}
    \gamma = \dfrac{\text{Taille de l'image}}{\text{Taille de l'objet}}
  \end{equation*}
  Si l'image est renversée, on met un signe $-$ devant la valeur du grandissement.
\end{doc}

% \numeroQuestion
% Tracer l'image de $AB$ dans le document~\ref{doc:TP1_construction_image}.

\mesure En utilisant la distance focale $f'$ de la lentille, repérer le foyer objet $F$ sur le banc.
Repérer également le point $P$ situé à la distance $2f'$ à gauche de $O$.

\mesure Pour différentes positions de l'objet entre $F$ et $O$ ; puis entre $F$ et $P$ ; puis entre $P$ et l'infini ; chercher s'il est possible de former une image réelle nette sur l'écran.
Si ce n'est pas possible d'avoir une image réelle, chercher s'il existe une image virtuelle nette en regardant à travers la lentille.

\newpage
\numeroQuestion
Consigner vos résultats dans le tableau ci-dessous.
\begin{center}
  \begin{tableau}{
    |X[c] | X[c] |X[c] |X[c] | X[c] |
  }
    Position de l'objet & Image réelle ou virtuelle & 
    Image droite ou renversée & Taille de l'image $A'B'$ en \unit{cm} &
    Grandissement (si image réelle) \\
    Entre $O$ et $F$ & & & & \\
    Entre $F$ et $P$ & & & & \\
    En $P$ & & & & \\
    Entre P et l'infini & & & & \\
  \end{tableau}
\end{center}

\numeroQuestion
Tracer l'image $A'B'$ pour chacun des 4 cas suivants.

\begin{center}
  \image{0.75}{images/lumiere/formation_image_lentille_conv0001}
  \vspace*{24pt}
  
  \image{0.7}{images/lumiere/formation_image_lentille_conv0002}
  \vspace*{24pt}
  
  \image{0.7}{images/lumiere/formation_image_lentille_conv0003}
  \vspace*{24pt}
\end{center}

\begin{center}
  \image{0.7}{images/lumiere/formation_image_lentille_conv0004}
\end{center}


\question{
  Est-ce que l'image $A'B'$ obtenue graphiquement est cohérente avec celle observée dans les 4 situations ?
}{

}{4}