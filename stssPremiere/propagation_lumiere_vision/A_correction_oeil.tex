%%%%
\tetePremStssVisi
\titreActivite{Les principaux défauts de l'œil}

%%%% objectifs
\begin{objectifs}
  \item Comprendre les trois grands défauts de l'œil et le principe de leur correction.
\end{objectifs}

\begin{contexte}
  La myopie, l'hypermétropie et la presbytie sont les trois défauts de l'œil les plus courants.
  Ils sont liés à un problème de \important{stigmatisme} de l'œil, c'est-à-dire que l'image d'un point ressemble à une tâche, ce qui donne une vision floue.
  
  \problematique{
    Comment corriger les problèmes de vue à l'aide de lentilles minces ?
  }
\end{contexte}

%%%%
\begin{doc}{Vergence d'une lentille}{doc:TP2_vergence}
  \begin{importants}
    La \important{vergence $V$} d'une lentille est l'inverse de sa distance focale, elle s'exprime en \important{dioptrie,} noté \unit{\dioptre}
    \begin{equation*}  
      V = \dfrac{1}{f'} = \dfrac{1}{OF'}
    \end{equation*}
  \end{importants}

  La vergence est positive + pour les lentilles convergentes et négative - pour les lentilles divergentes.
\end{doc}

\begin{doc}{Principe de l'accommodation}{doc:TP2_accommodation}
  Pour qu'un objet soit vu net, il faut qu'il y ait \important{stigmatisme}, c'est-à-dire que l'image d'un point observé soit un point sur la rétine.
  Pour assurer ce stigmatisme à toute distance, les muscles ciliaires qui entourent le cristallin peuvent en modifier la forme pour le rendre plus ou moins convergent.
  Ce phénomène s'appelle \important{l'accommodation.}

  \begin{wrapfigure}[19]{r}{0.4\linewidth}
    \vspace*{-20pt}
    \centering
    \image{0.9}{images/lumiere/oeil_accommodation} \\
    \legende{Schéma d'un œil normal en vision de près, sans et avec accommodation.}
    
    \image{0.9}{images/lumiere/oeil_repos} \\
    \legende{Schéma d'un œil sain au repos.}
  \end{wrapfigure}

  \pointCyan \important{Pour un objet proche.}
  
  Sans accommodation l'image d'un point sur la rétine serait une tâche et on verrait flou.
  Pour compenser, le cerveau va contracter les muscles ciliaires pour « gonfler » le cristallin et augmenter sa vergence, ce qui permet de former une image nette.

  Plus l'objet observé est proche, plus la distance focale $f'$ de l'œil est courte : le foyer image se rapproche du cristallin pour former une image sur la rétine.
  
  \pointCyan \important{Pour un objet lointain.}
  
  Au delà de \qty{6}{\m} on considère que l'objet observé est à « l'infini ».
  Les rayons lumineux arrivent alors parallèles sur l'œil, qui est conçu pour former une image exactement sur la rétine.
  Les muscles ciliaires sont au repos et le cristallin à sa forme normale.

  \pointCyan \important{Limites de l'accommodation.}
  
  L'accommodation est bornée par deux distances, le \important{punctum remotum PR} et le \important{punctum proximum PP.}
  \begin{listePoints}
    \item Le \important{punctum proximum} est le point le plus proche pouvant être vu net.
    Pour un œil normal accommodé au maximum, le punctum proximum se trouve à \qty{20}{\cm}.
    Le punctum proximum diminue avec l'âge : c'est la presbytie.
  \end{listePoints}
  
  \begin{listePoints}
    \item Le \important{punctum remotum} est le point le plus loin pouvant être vu net.
    Pour un œil normal au repos, le punctum remotum se trouve à l'infini.
  \end{listePoints}
\end{doc}

\begin{doc}{Les défauts de l'œil et leur correction}{doc:TP2_defauts}
  \important{L'hypermétropie}, la \important{presbytie}, la \important{myopie} et \important{l'astigmatisme} sont des défauts de l'œil où l'image n'est pas formée correctement sur la rétine, ce qui entraine une vision floue.

  \begin{wrapfigure}[10]{r}{0.4\linewidth}
    \centering
    \vspace*{-2pt}
    \image{0.9}{images/lumiere/oeil_myope}
    \legende{Schéma d'un œil myope regardant au loin avec ou sans correction.}
    % \\[4pt]
    
    % \image{0.9}{images/lumiere/oeil_hypermetrope}
    % \legende{Schéma d'un œil hypermétrope regardant au loin avec ou sans correction.}
  \end{wrapfigure}
  
  Pour corriger ses défauts, on ajoute une lentille mince pour corriger \important{la vergence} de l'œil, afin que l'image se forme sur la rétine avec un stigmatisme parfait.

  \begin{importants}
    On peut considérer que deux lentilles minces accolées sont équivalentes à une seule lentille mince.
    La vergence de cette lentille est simplement la somme de la vergence des deux lentilles séparées.
  \end{importants}

  %
  \pointCyan \important{La myopie.}

  Un œil myope a une vergence $V_\text{œil}$ trop élevée, ce qui entraine une formation de l'image devant la rétine.
  Pour corriger ce défaut, on ajoute donc une lentille divergente de vergence $V_\text{lentille} < 0$ devant l'œil afin de diminuer la vergence totale de l'œil $V = V_\text{œil} + V_\text{lentille}$.

  %
  \pointCyan \important{L'hypermétropie.}

  Un œil hypermétrope a une vergence $V_\text{œil}$ trop faible, ce qui entraine une formation de l'image derrière la rétine.
  Pour corriger ce défaut, on ajoute donc une lentille convergente de vergence $V_\text{lentille} > 0$ devant l'œil afin d'augmenter a vergence totale de l'œil $V = V_\text{œil} + V_\text{lentille}$.

  %
  \pointCyan \important{La presbytie.}

  Un œil presbyte est comme un œil hypermétrope, sauf que c'est lié à l'âge.
  En vieillissant le cristallin perd en élasticité et les muscles ciliaires fatiguent, ce qui empêche l'œil d'accommoder de près.
  On corrige ce défaut comme l'hypermétropie.

  %
  \pointCyan \important{L'astigmatisme.}

  L'astigmatisme est lié à un défaut de courbure du cristallin ou de la cornée, qui ne ressemble plus à une sphère aplatie, mais à un ballon de rugby aplati.
  Pour corriger ce défaut, on crée des verres avec une épaisseur variable pour compenser la surface irrégulière de la cornée ou du cristallin.
\end{doc}

\begin{wrapfigure}[10]{r}{0.5\linewidth}
  \centering
  \vspace*{-2pt}
  \image{0.75}{images/lumiere/oeil_a_completer} \\
  \legende{œil hypermétrope} \\[8pt]

  \image{1}{images/lumiere/oeil_zone_accommodation} \\
  \legende{Zones de vision nette comparées}
\end{wrapfigure}

\mesure
L'œil ci-contre est hypermétrope.
Placer son foyer et prolonger les rayons lumineux qui arrivent dans l'œil.

\begin{doc}{Accommodation d'un œil myope et hypermétrope}{doc:TP2_accommodation}
  Pour un œil myope, le punctum proximum est plus proche de l'œil que pour un œil normal, mais le punctum remotum n'est plus à l'infini.

  Pour un œil hypermétrope, le punctum proximum est plus loin de l'œil que pour un œil normal, mais le punctum remotum est toujours à l'infini.
\end{doc}

\mesure 
Placer sur les axes ci-contre le punctum proximum PP, le punctum remotum PR et hachurer la zone de visibilité nette de chaque œil.