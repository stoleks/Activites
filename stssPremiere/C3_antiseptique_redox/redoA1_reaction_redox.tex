%%%%
\tetePremStssElec

%%%% titre
\vspace*{-36pt}
\numeroActivite{1}
\titreActivite{Prévenir les risques d'électrisation}


%%%% objectifs
\begin{objectifs}
  \item Revoir l'intensité du courant et la tension électrique
  \item Connaître les dangers liés à une électrisation
\end{objectifs}

\begin{contexte}
  Anna projette de réaliser quelques travaux sur les prises électriques dans sa maison.
  
  \problematique{
    Comment éviter tout risque d'électrisation ?
  }
\end{contexte}


%%%% docs
\begin{doc}{La prise de terre}{doc:A1_prise_terre}
  \begin{wrapfigure}[3]{r}{0.1\linewidth}
    \centering
    \vspace*{-32pt}
    \qrcode{https://www.youtube.com/watch?v=BazTgHMuA8k}
  \end{wrapfigure}
  La prise de terre a une importance vitale, car la prise de terre assure que le courant électrique s'évacue dans le sol lorsqu'un appareil est mal isolé.
  Concrètement la prise de terre est un câble métallique qui finit sur un piquet enfoui dans le sol.
  
  \begin{wrapfigure}{l}{0.2\linewidth}
    \centering
    \vspace*{-14pt}
    \image{1}{images/photos/prise_terre.jpg}
  \end{wrapfigure}
  Ce câble permet de dévier le courant électrique qui s'échapperait d'un appareil dans la terre, d'où le nom de prise de terre.

  Des pertes de courant peuvent survenir lorsqu'un câble d'alimentation abîmé est dénudé et que les fils électriques entrent en contact avec l'armature métallique d'un l'appareil.
  Sans prise de terre, le courant traverserait le corps de la première personne qui toucherait l'appareil !
\end{doc}