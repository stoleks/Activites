%%%%
\tetePremStssRedo

%%%% titre
\vspace*{-36pt}
\numeroActivite{1}
\titreActivite{Les réaction d'oxydoréduction}


%%%% objectifs
\begin{objectifs}
  \item Savoir qu'un \textbf{oxydant} est une espèce qui \textbf{obtient} des électrons.
  \item Savoir qu'un \textbf{réducteur} est une espèce qui \textbf{relâche} des électrons.
  \item Apprendre la méthode pour écrire une réaction d'oxydoréduction.
\end{objectifs}

\begin{contexte}
  Un acide et une base forment un couple si l'on peut passer de l'un à l'autre par la perte ou le gain de proton(s) \chemfig{H^+}.

  Pour les réaction d'oxydoréduction, il s'agit de couple oxydant/réducteur, reliés par la perte ou le gain d'électron(s).
  
  \problematique{
    Comment décrire une réaction d'oxydoréduction ?
  }
\end{contexte}


%%%% docs
\begin{doc}{Couple oxydant réducteur}{doc:A1_couple_redox}
  \begin{encart}  
    Un \important{oxydant} est une espèce chimique capable d'\important{obtenir} un ou plusieurs \important{électrons.}

    Un \important{réducteur} est une espèce chimique capable de \important{relâcher} un ou plusieurs \important{électrons.}
  \end{encart}

  Un oxydant et un réducteur forment un couple oxydant/réducteur, si l'on peut passer de l'un à l'autre par le gain ou la perte d'électrons.
  Le couple est noté Ox/Red. \exemple{\chemfig{Zn^{2+}}/\chemfig{Zn}}.

  À chaque couple oxydant/réducteur, on associe une demi-équation
  \begin{center}
    oxydant + $n$ \chemfig{e^{-}} = réducteur

    $n$ est le nombre d'électrons échangés
  \end{center}
  \vspace*{-4pt}
  L'égalité symbolise que la réaction chimique est possible dans les deux sens.
  
  \begin{encart}
    \begin{listePoints}
      \item Ox + $n$ \chemfig{e^{-}} \!\!\reaction Red
      : il s'agit d'une \important{réduction.}
      L'oxydant est \important{réduit} (se transforme en réducteur).
      \item  Red \!\!\reaction Ox + $n$ \chemfig{e^{-}}
      : il s'agit d'une \important{oxydation.}
      Le réducteur est \important{oxydé} (se transforme en oxydant).
    \end{listePoints}
  \end{encart}
\end{doc}

%%
\begin{doc}{La réaction d'oxydoréduction}{doc:A1_reaction_redox}
  \begin{encart}
    Une réaction d'\important{oxydoréduction} a lieu quand on met en contact un oxydant et un réducteur de deux couples différents.
  \end{encart}
  
  Elle met donc en jeu deux couples oxydant/réducteur.
  Par exemple avec un couple du fer : \chemfig{Fe^{3+}}/\chemfig{Fe} ; et un couple de l'oxygène : \chemfig{O_2}/\chemfig{O^{2-}}.

  Le gaz \chemfig{O_2} va réagir avec le solide \chemfig{Fe}, pour se transformer en ion \chemfig{Fe^{3+}} et en ion \chemfig{O^{2-}} (phénomène de rouille).

  \begin{encart}
    Les électrons ne sont jamais libres.
    Il y a transfert d'électrons du réducteur vers l'oxydant.
  \end{encart}
\end{doc}

\question{
  Indiquer quel espèce chimique est l'oxydant et quel espèce chimique est le réducteur pour le couple associé au fer et pour le couple associé à l'oxygène.
}{

}{1}

%%
\begin{doc}{Méthode d'écriture d'une équation d'oxydoréduction}{doc:A1_methode_redox}
  Pour écrire la réaction d'oxydoréduction entre les ions argent \chemfig{Ag^+} et le cuivre \chemfig{Cu}, il faut suivre la méthode suivante :
  \begin{enumeration}
    \item \textbf{Repérer} dans chaque couple quel oxydant réagit avec quel réducteur.
    \item \textbf{Écrire} les demi-équations de réaction pour chaque couple dans le « bon » sens, avec les réactifs à droite et les produits à gauche.
    \item \textbf{Ajuster} les deux demi-équations pour qu'il y ait le même nombre d'électrons échangés.
    \item \textbf{Additionner} les deux demi-équations afin d'obtenir l'équation d'oxydoréduction.
  \end{enumeration}
  \attention Il ne doit pas y avoir d'électron dans l'équation finale !
\end{doc}

%%
\begin{doc}{L'arbre de Diane}{doc:A1_arbre_diane}
  \begin{wrapfigure}[4]{r}{0.1\linewidth}
    \vspace*{-24pt}
    \qrcode{https://www.youtube.com/watch?v=Wl8ZuN_7gaI}
  \end{wrapfigure}
  
  On introduit dans un erlenmeyer une solution incolore concentrée en ions argent $\chemfig{Ag^+}(aq)$.
  On plonge ensuite un morceau de cuivre solide $\chemfig{Cu}(s)$.

  Après quelques minutes, le morceau de cuivre s'est recouvert de paillettes d'éclat métallique et la solution est devenue bleue.

  Les demi-équations intervenant dans  cette réaction sont
  \begin{align*}
    \chemfig{Cu^{2+}}(aq) + 2\chemfig{e^{-}} \reaction& \chemfig{Cu}(s) \\
    \chemfig{Ag^+}(aq) + \chemfig{e^{-}} \reaction& \chemfig{Ag}(s)
  \end{align*}
\end{doc}

\question{
  Quelle observation macroscopique montre que du métal d'argent s'est formé ?
}{}{3}

\question{
  Identifier les réactifs et les produits de la réaction de l'arbre de Diane.
}{}{3}

\question{
  À l'aide des demi-équations fournies, identifier les couples Ox/Red qui interviennent dans la réaction de l'arbre de Diane.
}{}{2}

\question{
  Écrire la réaction d'oxydoréduction qui modélise la transformation de l'arbre de Diane.
}{}{5}