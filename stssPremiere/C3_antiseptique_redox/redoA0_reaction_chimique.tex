%%%%
\tetePremStssRedo

%%%% titre
\vspace*{-40pt}
\numeroActivite{0}
\titreActivite{Réaction chimique}

%%%% Objectifs
\begin{objectifs}
  \item Revoir comment une réaction chimique modélise une transformation macroscopique.
\end{objectifs}


%%%% docs
\begin{doc}{Observations macroscopiques}{doc:A0_macro}
  Pendant une transformation chimiques, des espèces chimiques interagissent, réarrangent leurs atomes, et forment d'autres espèces chimiques.
  Les espèces présentes initialement sont les \important{réactifs.} Celles présentes au final après la transformation sont les \important{produits.}
  
  Pour modéliser la transformation, il faut \important{identifier} les espèces chimiques qui réagissent et celles qui se forment.
  Pour ça, on observe ce qu'il se passe d'un point de vue macroscopique : formation d'un gaz ou d'un solide, disparition d'un solide, changement de couleur, etc.
  %Il est aussi possible d'utiliser des tests d'identification des espèces chimiques.
  
  \begin{encart}
    Les observations expérimentales macroscopiques permettent d'écrire l'équation de la \important{réaction} modélisant la transformation chimique microscopique, en identifiant les \important{réactifs} et les \important{produits.}
  \end{encart}
\end{doc}

\begin{doc}{Modélisation de la réaction}{doc:A0_modelisation}
  L'écriture de la réaction chimique permet de transcrire la transformation des réactifs en produit.
  
  \begin{encart}
    La réaction est symbolisée par une flèche. À gauche de la flèche se trouvent les \important{réactifs} qui se transforment et à droite de la flèche se trouvent les \important{produits} formés :
    \begin{center}
      réactif 1 + réactif 2 + \texteTrou{1}  \reaction produit 1 + produit 2 + \texteTrou{1} 
    \end{center}
  \end{encart}
  
  Au cours d'une réaction chimique, rien ne se perd, rien ne se crée. \important{Il doit donc y avoir le même nombre d'atomes et de charges de chaque côté de la réaction}.
  Seuls les liaisons des molécules peuvent être modifiées pendant une réaction chimique.
\end{doc}

\begin{doc}{Notation des états physiques}{doc:A0_notation_etats_physiques}
  Les réactifs et les produits peuvent se trouver dans différents états physiques.
  Pour indiquer dans quel état se trouve une espèce chimiques, on écrit son état entre parenthèse à côté de sa formule chimique : $(g)$ pour un gaz, $(l)$ pour un liquide, $(s)$ pour un solide et $(aq)$ pour des solutés en solution aqueuse.
\end{doc}


\begin{doc}{Combustion du charbon}{doc:A0_combustion_charbon}
  On modélise la combustion du charbon avec du dioxygène par la réaction chimique suivante :
  \begin{equation*}
    \chemfig{C}(s) + \chemfig{O_2}(g) \reaction \chemfig{CO_2}(g)
  \end{equation*}
  On vérifie bien qu'il y a le même nombre d'atome de carbone et d'oxygène des deux côté de la réaction chimique.
\end{doc}


%%%% Questions
\question{
  Lister les réactifs et les produits pour la combustion du charbon en présence d'oxygène, en indiquant leurs état physique.
}{
  Réactifs : carbone solide et dioxygène gazeux.
  
  Produits : dioxyde de carbone gazeux.
}{1}


%%
\begin{doc}{Ajustage d'une réaction}{doc:A0_equlibrage_reaction}
  Au cours d'une réaction chimique, les éléments chimiques présents dans les réactifs se réarrangent pour former des produits et les liaisons chimiques changent.
  \begin{encart}
    Il y a \important{conservation} 
    \begin{listePoints}
      \item \important{des éléments chimiques} ;
      \item \important{de la charge électrique} totale.
    \end{listePoints}
  \end{encart}
  \begin{encart}
    Pour assurer cette \important{conservation}, il faut \important{ajuster} la réaction chimique avec des coefficients devant les éléments chimiques.
    Ces coefficients sont appelés \important{coefficient stoechiométrique.}
  \end{encart}
\end{doc}

\question{
  Ajuster la réaction de combustion du méthane
  \begin{center}
      \chemfig{CH_4}(g) + \chemfig{O_2}(g)
      \reaction
      \chemfig{CO_2}(g) + \chemfig{H_2O}(g)
  \end{center}
  à l'aide de coefficients stoechiométriques.
  Commencer par ajuster le nombre d'atomes d'hydrogène.
}{
  \begin{equation*}
    \chemfig{CH_4}(g) + 2\chemfig{O_2}(g)
    \reaction
    \chemfig{CO_2}(g) + 2\chemfig{H_2O}(g)
  \end{equation*}
}{3}

\numeroQuestion
Ajuster les réactions chimiques suivantes en écrivant, si nécessaire, les coefficients stoechiométriques devant chaque élément chimique :
\newcommand{\localEcart}{20}
\begin{center}
  %
  \texteTrou{1} \chemfig{Fe}(s) + \texteTrou{2} \chemfig{H^{+}}(aq)
  \reaction \texteTrou{1} \chemfig{Fe^{2+}}(aq) + \texteTrou{1} \chemfig{H_2}(g)
  \\[\localEcart pt]
  %
  \texteTrou{4} \chemfig{Fe}(s) + \texteTrou{3} \chemfig{O_2}(g)
  \reaction \texteTrou{2} \chemfig{Fe_2O_3}(s)
  \\[\localEcart pt]
  %
  \texteTrou{1} \chemfig{C_2H_6O}(l) + \texteTrou{3} \chemfig{O_2}(g)
  \reaction \texteTrou{2} \chemfig{CO_2}(g) + \texteTrou{3} \chemfig{H_2O}(l)
  \\[\localEcart pt]
  %
  \texteTrou{1} \chemfig{Cu^{2+}}(aq) + \texteTrou{2} \chemfig{HO^{-}}(aq)
  \reaction \texteTrou{1} \chemfig{Cu{(HO)}_2}(s)
  \\[\localEcart pt]
  %
  \texteTrou{2} \chemfig{Fe}(s) + \texteTrou{2} \chemfig{H_2O}(l) + \texteTrou{1} \chemfig{O_2}(g)
  \reaction \texteTrou{2} \chemfig{Fe{(HO)}_2}(s)
  \\[\localEcart pt]
  %
  \texteTrou{1} \chemfig{Fe{(OH)}_2}(s) + \texteTrou{2} \chemfig{H_2O}(l) + \texteTrou{1} \chemfig{O_2}(g)
  \reaction \texteTrou{2} \chemfig{Fe{(OH)}_3}(s)
  \\[\localEcart pt]
  %
  \texteTrou{2} \chemfig{Fe{(OH)}_3}(s)
  \reaction \texteTrou{1} \chemfig{Fe_2O_3}(s) + \texteTrou{3} \chemfig{H_2O}(l)
  %
\end{center}

%%
\begin{wrapfigure}{r}{0.1\linewidth}
  \centering
  \qrcode{https://phet.colorado.edu/sims/html/balancing-chemical-equations/latest/balancing-chemical-equations_fr.html}
\end{wrapfigure}
\numeroQuestion Pour s'entraîner :