\tetePremStssRedo
\vspace*{-36pt}
\titreActivite{Risques associés au produits oxydants}

%%%% objectifs
\begin{objectifs}
  \item Comprendre les mesures de précaution à employer avec des produits oxydants.
\end{objectifs}

\begin{contexte}
  Les produits oxydants nécessitent de respecter strictement des règles de sécurités pour éviter des accidents et pour une efficacité optimale.
  
  \problematique{
    Comment utiliser un produit oxydant en toute sécurité ?
  }
\end{contexte}


%%%% docs
\begin{doc}{Précautions d'emploi et toxicité}
  Il faut respecter plusieurs règles pour utiliser des antiseptiques et des désinfectants.
  \begin{listePoints}
    \item Ils sont dangereux à fortes concentration et doivent donc être dilués.
    \item Il ne faut pas utiliser deux produits en même temps, leur action pourrait être inhibée.
    \item Il ne faut pas mélanger les antiseptiques ou les désinfectants avec autre chose que de l'eau.
  \end{listePoints}

  \begin{tblr}{
    colspec = {c Q[l, wd=0.26\linewidth] Q[l, wd=0.25\linewidth] Q[l, wd=0.26\linewidth]},
    hlines, vlines, row{1} = {couleurSec-100}
  }
    {Produit \\ oxydant} &
    Peroxyde d'hydrogène (eau oxygénée) &
    Eau de Javel & Solution de diiode \\
    %
    {Précautions \\ et dangers} &
    \pointCyan Nocif par ingestion ou inhalation. \newline
    \pointCyan Peut provoquer des brûlures de la peau, des lésions oculaires graves, des irritations des voies respiratoires. \newline
    \pointCyan Peut provoquer un incendie ou une explosion. \newline
    \pointCyan Corrosif si concentré. &
    \pointCyan Ne jamais ingérer. \newline
    \pointCyan Peut provoquer des brûlures de la peau et des lésions oculaires graves. \newline
    \pointCyan Ne pas mélanger avec des acides (dégage un gaz toxique). \newline
    \pointCyan Très toxique pour les organismes aquatiques. &
    \pointCyan Ne pas ingérer ou avaler. \newline
    \pointCyan Irritation de la peau. \newline
    \pointCyan Peut impacter le fonctionnement de la thyroïde si utilisation répétée. \\
    %
    Stockage &
    Locaux ventilés, à l'abri de la lumière, des hautes températures, de tout combustible. &
    Locaux ventilés, à l'abri de tout rayonnement solaire et des hautes températures, à l'écart des acides et des matière organiques. &
    Locaux ventilés, à l'abri des hautes températures, à l'écart de produits susceptible de réagir avec du diiode. \\
    Conservation &
    15 jours après ouverture. &
    3 mois si concentrée, 6 à 12 mois diluée. &
    1 mois après ouverture.    
  \end{tblr}
\end{doc}

\question{
  Quels sont les précautions communes à ces trois produits oxydants ?
}{
  Ce sont des produits corrosifs et nocifs, donc il ne faut pas les ingérer et les mettre (à forte concentration) sur la peau.
}[2]

\question{
  Indiquer les propriétés d'un local qui permettrait de stocker ces trois produits oxydants.
}{
  Il faut un local ventilé, à l'abri des hautes températures et des rayonnements solaires.
}[2]

\newpage\vspace*{-42pt}
\begin{doc}{Quelques espèces oxydantes}
  \vspace*{-4pt}
  Les principes actifs des antiseptiques et désinfectants agissent par \important{oxydation.}
  \vspace*{4pt}

  \centering
  \begin{tblr}{
    colspec = {X[l] |c |r c l}, hlines, width=\linewidth,
    %column{4} = {8pt},
    row{1} = {couleurSec-100},
    row{5} = {m},
  }
    Principe actif &
    Couples Ox/red &
    \SetCell[c=3]{c} Demi-équation d'oxydoréduction \\
    %
    \SetCell[r = 2]{l} Eau oxygénée &
    \eauOxygenee/\eau &
    $\eauOxygenee + 2\ionHydrogene + 2\electron$ &
    \reaction &
    $2\eau$ \\
    &
    \dioxygene/\eauOxygenee & 
    $\dioxygene + 2\ionHydrogene + 2\electron$ &
    \reaction &
    $\eauOxygenee$ \\
    %
    Eau de Javel &
    \hypochlorite/\chlorure &
    $\hypochlorite + 2\ionHydrogene + 2\electron$ &
    \reaction &
    $\chlorure + \eau$ \\
    %
    Dichlore &
    $\dichlore/\chlorure$ &
    $\dichlore + 2\electron$ &
    \reaction &
    $2\chlorure$ \\
    %
    Diiode &
    \diiode/\iodure &
    $\diiode + 2\electron$ &
    \reaction &
    $2\iodure$ \\
    %
    Permanganate de potassium &
    \permanganate/\ionManganeseII &
    $\permanganate + 8\ionHydrogene + 5\electron$ &
    \reaction &
    $\ionManganeseII + 4\eau$ \\
    %
  \end{tblr}
\end{doc}


\begin{doc}{Eau de Javel et produit acide : un mélange dangereux !}
  L'eau de Javel est une solution aqueuse basique d’hypochlorite de sodium (\ionSodium, \hypochlorite) et de chlorure de sodium (\ionSodium, \chlorure).
  Un produit acide contient des ions \ionHydrogene.

  Le dichlore $\dichlore$ est un gaz toxique, car le dichlore se combine avec l'eau présente dans les muqueuses pour former des acides qui attaquent les tissus.
\end{doc}

\question{
  Établir l'équation de la réaction d'oxydoréduction entre les ions hypochlorites $\hypochlorite$ et les ions chlorures $\chlorure$ (couple \dichlore/\chlorure).
}{
  Ici l'hypochlorite est un oxydant et le chlorure un réducteur.
  \begin{enumeration}
    \item On identifie les couples associées aux espèces chimiques qui réagissent : \hypochlorite/\chlorure et \dichlore/\chlorure.
    \item On écrit les demi-équations d'oxydoréduction avec les réactifs à gauche
    \begin{align*}
      \hypochlorite + 2\ionHydrogene + 2\electron &\reaction \chlorure + 2\eau \\
      2\chlorure &\reaction \dichlore + 2\electron
    \end{align*}
    \item Le nombre d'électrons est le même pour les deux demi-équations, donc il suffit de les additionner côté par côté en enlevant les électrons
    \begin{align*}
      \hypochlorite + 2\ionHydrogene + 2\chlorure &\reaction \chlorure + 2\eau + \dichlore \\
      \hypochlorite + 2\ionHydrogene + \chlorure &\reaction 2\eau + \dichlore
    \end{align*}
  \end{enumeration}
}[5]

\question{
  Pourquoi cette réaction ne peut avoir lieu que dans un milieu acide ?
}{
  Les ions hydrogènes \ionHydrogene se trouvent en milieu acide. En milieu neutre ou basique, leur concentration est trop faible.
}[2]

\question{
  Quel est le gaz toxique dégagé par la réaction ?
}{
  C'est le dichlore \dichlore, qui attaque les muqueuses.
}[1]


%%%%
\begin{doc}{Nettoyer une plaie}
  Judith s'est écorchée le genou et mélange de l'eau oxygénée avec du permanganate de potassium pour soigner sa plaie.
  Les couples Oxydant/Réducteur sont $\dioxygene/\eauOxygenee$ et
  $\permanganate/\ionManganeseII$.
  
  Au moment de l'application, le mélange devient incolore et forme une mousse.
\end{doc}

%%
\question{
  Établir l'équation de la réaction d'oxydoréduction entre l'eau oxygénée et le permanganate de potassium.
  Expliquer la formation de mousse.
}{
  Ici le permanganate de potassium est un oxydant et l'eau oxygénée est un réducteur.
  \begin{enumeration}
    \item On identifie les couples associées aux espèces chimiques qui réagissent : \permanganate/\ionManganeseII et \dioxygene/\eauOxygenee.
    \item On écrit les demi-équations d'oxydoréduction avec les réactifs à gauche
    \begin{align*}
      \permanganate + 8\ionHydrogene + 5\electron &\reaction \ionManganeseII + 4\eau \\
      \eauOxygenee &\reaction \dioxygene + 2\ionHydrogene + 2\electron
    \end{align*}
    \item On multiplie les demi-équations par 2 et par 5 respectivement pour obtenir le même nombre d'électrons 
    \begin{align*}
      2\permanganate + 16\ionHydrogene + 10\electron &\reaction 2\ionManganeseII + 8\eau \\
      5\eauOxygenee &\reaction 5\dioxygene + 10\ionHydrogene + 10\electron
    \end{align*}
    \item Finalement, on additionne les demi-équations en enlevant les électrons
    \begin{equation*}
      2\permanganate + 16\ionHydrogene + 5\eauOxygenee
      \reaction
      2\ionManganeseII + 8\eau + 5\dioxygene + 10\ionHydrogene
    \end{equation*}
    et on simplifie les ions hydrogènes
    \begin{equation*}
      2\permanganate + 6\ionHydrogene + 5\eauOxygenee
      \reaction
      2\ionManganeseII + 8\eau + 5\dioxygene
    \end{equation*}
  \end{enumeration}
}[4]
