%%%%
\tetePremStssElec

\titre{Synthèse}

% \titreSection{Rappels}

\pointCyan Un courant électrique est caractérisé par 
\begin{listeTirets}
  \item une \important{tension électrique}, notée $U$ et mesurée en volt noté \unit{\volt}.
  \item une \important{intensité du courant}, notée $I$ et mesurée en ampère noté \unit{\ampere}.
\end{listeTirets}

\begin{importants}
  \pointCyan La puissance du courant, notée $P$ et mesurée en watt noté \unit{\watt},
  est le produit de la tension et de l'intensité
  \begin{equation*}
    P = U\times I
  \end{equation*}
\end{importants}

\pointCyan Soit un dipôle resistif de résistance $R$, traversée par une intensité $I_R$ et soumis à une tension $U_R$.
\begin{importants}
  La loi d'Ohm relie la tension, l'intensité et la résistance :
  
  \separationBlocs{
    \begin{equation*}
      U_R = R \times I_R \qq{}\qq{ou}\qq{}
      I_R = \dfrac{U_R}{R}
    \end{equation*} 
  }{
    \begin{center}
      \begin{circuitikz}
        \draw (0, 0) to [R, l={$R$}, i=$I_R$, v=$U_R$] (3, 0);
      \end{circuitikz}
    \end{center}
  }
\end{importants}

\titreSection{Les caractéristiques de la tension du secteur}

\pointCyan La \important{tension du secteur} est alternative, avec les propriétés suivantes :

\begin{tableau}{c c c c c}%{X[c] X[c] X[c] X[c] X[c]}
  Fréquence $f$ & Période $T$ & Tension $U$ & $U_\text{max}$ & $U_\text{min}$ \\
  %
  \qty{50}{\hertz} & \qty{20}{\ms} &
  \qty{230}{\volt} & \qty{325}{\volt} & \qty{-325}{\volt} \\
\end{tableau}

\pointCyan On peut lire les valeurs de la période $T$ 
et de la tension maximale $U_\text{max}$ sur un oscillogramme

\begin{center}
  % Oscillogramme
  \def\ver{3.6} % longueur verticale
  \def\hor{5.0} % longueur horizontale
  \begin{tikzpicture}[
    trace/.style={couleurPrim!75!black, ultra thick, samples = 100},
    screen/.style={couleurPrim!10, thick},
    axes/.style={couleurPrim, thick}
  ]
    % Fond de l'écran
    \fill[screen] (-\hor, -\ver) rectangle (\hor, \ver);
    % Grille et axes
    \draw[thin, couleurPrim!30] (-\hor, -\ver) grid (\hor, \ver);
    \draw[axes] (-\hor, 0) -- (\hor, 0); % temps
    \draw[axes] (0, -\ver) -- (0, \ver); % 
    % Graduation
    \pgfmathparse{\hor-1}
    % axe x
    \foreach[parse = true] 
      \i in {-\hor+0.25, -\hor+0.5,..., \hor-0.25} \draw[axes] (\i, -0.1) -- (\i, 0.1);
    \foreach[parse = true] \i in {-\hor,...,\hor} \draw[axes] (\i,-0.2) -- (\i,0.2);
    % axe y
    \foreach[parse = true]
      \i in {-\ver+0.10, -\ver+0.35,..., \ver-0.10} \draw[axes] (-0.1, \i) -- (0.1, \i);
    \foreach[parse = true] \i in {-3,...,3} \draw[axes] (-0.2,\i) -- (0.2,\i);
    % Tension sinusoïdale
    \draw[trace] plot(\x, {3.25*sin((1.57*(\x + 1)) r)}); % r = radian
    % Échelle
    \tkzVecteur(-5) (-3)[1]  {\textbf{100 V}} [right]
    \tkzVecteur(-5)[1] (-3)  {\textbf{5 ms}} [below right]
    \tkzVecteur(-4)[4] (3.3) {\textbf{Période} $\mathbf{T}$ \hspace{12pt}\phantom{b}} [below left]*
    \draw[ultra thick] (3.8, 3.26)  -- (4.2, 3.26)  node[right] {$\mathbf{U_\text{max}}$};
    \draw[ultra thick] (1.8, -3.26) -- (2.2, -3.26) node[right] {$\mathbf{U_\text{min}}$};
  \end{tikzpicture}
\end{center}

\separationBlocs{
  \pointCyan
  La tension $U$ se calcule avec la relation
  \begin{importants}
    \vspace*{-10pt}
    \begin{equation*}
      U = \dfrac{U_\text{max}}{\sqrt{2}}
      \hspace{-12pt}
      \begin{split}
        & U_\text{max} : \text{tension maximale en \unit{\volt}} \\
        & U : \text{\important{tension efficace} en \unit{\volt}}
      \end{split}
    \end{equation*}
  \end{importants}
}{
  \pointCyan
  La fréquence $f$ se calcule avec la relation
  \begin{importants}
    \vspace*{-10pt}
    \begin{equation*}
      f = \dfrac{1}{T} 
      \hspace{-12pt}
      \begin{split}
        & f : \text{fréquence en hertz \unit{\hertz}} \\
        & T : \text{période en seconde \unit{\s}}
      \end{split}
    \end{equation*}
  \end{importants}
}


\titreSection{Le rôle d'un disjoncteur}

\pointCyan L'intensité du courant mesure la quantité d'électron qui traversent un matériau conducteur en une seconde.
Si l'intensité est trop élevée, on parle de surintensité.

\textbf{Au quotidien, deux situations provoquent une surintensité :}

\separationBlocs{
  \pointCyan Le court-circuit (câble abîmé ou faux contact dans un appareil) \vspace{4pt}

  \begin{circuitikz}
    \draw (3, 1) to [short, i=$I_1$] (0, 1)
      to [american voltage source] (0, -1) 
      % to [rmeterwa, t=G] (0, -1)
      to [lamp, i=$I_1$] (3, -1)
      to [R, l={$R$}] (3, 1) -- (0, 1);
  \end{circuitikz}
  \begin{circuitikz}
    \draw (3, 1) to [short, i=$I_2$] (0, 1)
      to [american voltage source] (0, -1) 
      % to [rmeterwa, t=G] (0, -1)
      to [lamp, i=$I_2$] (3, -1)
      to [R, l={$R$}] (3, 1) -- (0, 1);
    \draw[ultra thick, red] (0.75, -1) -- (0.75, 0) -- (2.2, 0) -- (2.2, -1);
  \end{circuitikz}
  
  \begin{equation*}
    I_2 > I_1
  \end{equation*}
}{
  \pointCyan Trop d'appareil en dérivation (branchés sur une même multiprise) \vspace{4pt}
  
    \begin{circuitikz}
    \draw (3, 1) to [short, i=$I_1$] (0, 1)
      to [american voltage source] (0, -1) 
      % to [rmeterwa, t=G] (0, -1)
      to [lamp, i=$I_1$] (3, -1)
      to [R, l={$R$}] (3, 1) -- (0, 1);
  \end{circuitikz}
  \begin{circuitikz}
    \draw (3, 0.5) to [short, i=$I_2$] (0, 0.5)
      to [american voltage source] (0, -1) 
      % to [rmeterwa, t=G] (0, -1)
      to [lamp, i=$I_2$] (3, -1)
      to [R, l={$R$}] (3, 0.5) -- (0, 0.5);
    \draw[very thick, red] (0, -2) to [lamp] (3, -2);
    \draw[very thick, red] (0, -1) -- (0, -3) to [lamp] (3, -3) -- (3, -1);
  \end{circuitikz}
}

\textbf{En cas de surintensité :}
\begin{listePoints}
  \item risque d'incendie
  \item risque d'abîmer les appareils électriques
\end{listePoints}

\begin{importants}
  \pointCyan Le \important{disjoncteur} protège des risques de surintensité. 
  Il ouvre le circuit électrique quand l'intensité dépasse une certaine valeur, écrite sur son boîtier.
\end{importants}


\titreSection{Mettre à la terre pour protéger contre l'électrisation}

\pointCyan Le corps humain conduit le courant électrique.

\pointCyan \important{L'électrisation} est le passage d'un courant électrique dans le corps.

\pointCyan \important{L'électrocution} est la mort par électrisation, quand l'intensité du courant $I$ est trop élevée.

\pointCyan La résistance du corps humain diminue quand il est mouillé.
Comme $I = U / R$, c'est moins dangereux d'être électrisé quand notre peau est sèche que quand on est mouillé.

\medskip
\begin{wrapfigure}[3]{r}{0.31\linewidth}
  \centering
  \image{1}{images/electricite/prise_murale}    
\end{wrapfigure}

\textbf{Pour limiter les risques d'électrisation :}
\begin{listePoints}
  \item Ne pas mettre d'appareils électriques à côté d'une source d'eau.
  \item Ne pas toucher d'appareils électriques avec les mains mouillées.
  \item Couper le courant avant de faire des travaux électriques.
\end{listePoints}

\medskip
\textbf{Comment la prise de terre protège de l'électrisation :}

\begin{importants}
  Le fil de terre permet d'évacuer le courant électrique dans le sol en cas de faux contact.
  
  Au lieu de passer par le corps de la personne qui touche l'appareil défectueux, le courant électrique passe par la prise de terre.
\end{importants}

\separationBlocs{
  \centering
  \image{0.9}{images/electricite/electrisation}
  \vspace{-2pt}

  Sans fil de terre
}{
  \centering
  \image{0.9}{images/electricite/electrisation_terre}
  \vspace{-2pt}

  Avec fil de terre
}