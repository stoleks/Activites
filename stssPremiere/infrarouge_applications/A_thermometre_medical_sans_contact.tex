\tetePremStssLumi
\vspace*{-32pt}
\titreActivite{Thermomètre médical sans contact}

%%%% objectifs
\begin{objectifs}
  \item Comprendre le fonctionnement d'un thermomètre médical sans contact.
\end{objectifs}

\begin{contexte}
  Pour mesurer la température corporelle d'une personne rapidement et sans contact, on utilise un thermomètre sans contact.
  
  \problematique{
    Quels principes physique sont utilisés par un thermomètre sans contact pour mesurer une température ?
  }
\end{contexte}


%%%% docs
\begin{doc}{Émission d'un corps chaud}
  \begin{importants}
    La surface d'un corps émet un \important{rayonnement électromagnétique}, dont l'intensité dépend de la température du corps $T$.
  \end{importants}

  Comme on l'a vu, pour des température « faible » ($< \qty{1000}{\degreeCelsius}$) le rayonnement est dans le domaine des infrarouge IR, avec une longueur d'onde supérieure à \qty{700}{\nm}.
\end{doc}

\begin{doc}{Mesure de température avec un thermomètre infrarouge}
  Le rayonnement émis par le corps observé est focalisé par une lentille sur un capteur qui génère une tension électrique.
  Cette tension électrique \important{dépend de l'intensité du rayonnement émis.}
  
  Le signal est amplifié et transformé en une grandeur proportionnelle à la température du corps, grâce à un traitement numérique.
  La température mesurée est ensuite affichée sur un écran.

  La mesure de température sans contact présente plusieurs avantages :
  \begin{listePoints}
    \item temps de mesure très court ;
    \item mesure non invasive ;
    \item possible de mesurer des objets en mouvement.
  \end{listePoints}

  \begin{center}
    \image{0.8}{images/thermodynamique/capteur_IR}
  \end{center}

  On ne peut mesurer que la température de la surface d'un corps avec un thermomètre IR.
\end{doc}

\begin{doc}{Thermomètres médicaux sans contact}
  Les thermomètres médicaux sont conçus pour mesurer les températures du corps humain.

  L'intensité du rayonnement infrarouge est convertie en tension électrique, puis l'appareil calcule et affiche la température.

  \important{Caractéristiques techniques d'un thermomètre médical IR :}
  \begin{listePoints}[2]
    \item plage de mesure : de \qty{32,0}{\degreeCelsius} à \qty{42,0}{\degreeCelsius} ;
    \item précision : $\pm \qty{0,2}{\degreeCelsius}$ ;
    \item affichage : 3 digits ;
    \item sensibilité du capteur IR : de \qty{8}{\micro\m} à \qty{14}{\micro\m}.
  \end{listePoints}
\end{doc}


%%%%
\question{
  Un thermomètre affiche une température de \qty{36,8}{\degreeCelsius}. 
  Calculer la plage de température possible du corps à l'aide de la précision de la mesure, c'est-à-dire la plus petite et la plus grande température possible.
}{}[2]

\question{
  À l'aide de la loi de Wien
  \begin{equation*}
    \lambda = \dfrac{\qty{2.9e-3}{\kelvin\m}}{T (\unit{\kelvin})}
  \end{equation*}
  calculer la longueur d'onde d'intensité maximale émise par un corps à une température de \qty{32}{\degreeCelsius}.
}{}[3]

\question{
  Le capteur IR est-il adapté pour mesurer de telle température ?
}{}[3]

\question{
  On mesure la tension électrique fournie par le capteur pour différentes température :

  \begin{center}
    \begin{tblr}{
      cells = {c}, hlines, vlines, column{1} = {l, couleurSec-100}
    }
      Température $T$ en \unit{\degreeCelsius} & 32,0 & 34,5 & 37,0 & 39,5 & 42,0 \\
      Tension $U$ en \unit{\milli\volt}        & 512  & 1120 & 1635 & 2055 & 2430 \\
    \end{tblr}
  \end{center}
  
  En utilisant une méthode graphique ou numérique, déterminer la température d'un corps correspondant à une tension de \qty{1728}{\milli\volt}.
}{}[4]

\question{
  Quel type de lentille doit-on utiliser dans le thermomètre pour concentrer la lumière ?
  Quelle est le nom de la distance entre la lentille et le capteur ?
}{}[3]