%%%%
\tetePremStssLumi

%%%% titre
\numeroActivite{3}
\titreActivite{Danger des infrarouges}


%%%% objectifs
\begin{objectifs}
  \item Comprendre les risques spécifiques associés au infrarouges.
\end{objectifs}

\begin{contexte}
  Tous les objets à des températures usuelles ($T < \qty{1000}{\degreeCelsius}$) émettent des rayonnements électromagnétique dans le domaine des infrarouges.
  
  \problematique{
    Ce rayonnement est-il dangereux pour la santé ?
  }
\end{contexte}


%%%% docs
\begin{doc}{Les différents domaines d'infrarouge.}{doc:A3_emission_corps}
  Les émissions infrarouges sont classées en 3 domaines : infrarouge proche (A), infrarouge moyen (B) et infrarouge lointain (C).
  Ces domaines ont été établis à partir des propriétés d'absorption des tissus du corps humain.

  \begin{tableau}{|c |c |c |c |}
    Domaine & IR-A & IR-B & IR-C \\
    %
    Longueur d'onde &
    \qty{800}{\nm} $< \lambda <$ \qty{1400}{\nm} & 
    \qty{1400}{\nm} $< \lambda <$ \qty{3000}{\nm} & 
    \qty{3}{\micro\m} $< \lambda <$ \qty{e3}{\micro\m} \\
    %
  \end{tableau}
\end{doc}

\begin{doc}{Infrarouges et sécurité}{doc:A3_capteur_infrarouge}
  La lumière infrarouge est souvent qualifiée de \important{rayonnement thermique}, car quand on reçoit des infrarouges notre corps le perçoit comme de la chaleur.
  En général, les rayonnements infrarouges sont sans danger, \important{contrairement aux rayonnements ultraviolets.}
  
  Il existe cependant des risques de brûlure pour les yeux et la peaux, si on est exposé à des rayonnements infrarouge intense pendant une longue durée.
  Comme, par exemple, à proximité d'objet chauffé à haute température ($T > \qty{500}{\degreeCelsius}$).

  Les \important{infrarouge-A (IR-A)} endommagent surtout la rétine, les yeux étant transparent à ce type d'onde.
  Le cristallin peut être endommagé par des rayonnements \important{IR-A et IR-B} intenses.
  La cornée peut être endommagée par des rayonnements \important{IR-B et IR-C} intenses et prolongés.
  
  Pour la rétine le risque est ici principalement lié à l'utilisation de laser infrarouge, qui sont invisibles et intenses.  
\end{doc}


\begin{doc}{Les métiers à risques IR}{doc:A3_metiers_risques}
  Certains métiers sont exposés à des rayonnements infrarouges intenses pendant de longues durée : personnes travaillant dans les fonderies, souffleur ou souffleuse de verre, sapeur-pompier, soudeurs et soudeuse, etc.
  Mais aussi les chercheuses et chercheurs travaillant avec des lasers infrarouge très puissant.

  Pour se protéger des rayonnements, il faut porter des lunettes spéciales munies de filtres IR.
\end{doc}


%%%%
\question{
  À l'aide de la loi de Wien, calculer la longueur d'onde d'intensité maximale émise par un corps à une température de \qty{37}{\degreeCelsius}.
  Cette longueur d'onde correspond-elle au domaine proche IR-A, moyen IR-B ou lointain IR-C ?
}{}{4}

\newpage
\question{
  Les rayonnements infrarouges émis par votre corps représentent-ils un danger pour les personnes qui vous entourent ?
}{}{4}

\question{
  Donner une solution pour protéger des risques liés aux IR les ouvrier-es qui travaillent dans les fonderies.
}{}{4}