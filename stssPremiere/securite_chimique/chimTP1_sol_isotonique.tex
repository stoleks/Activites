%%%%
\tetePremStssChim

%%%% titre
\numeroActivite{1}
\titreTP{Préparation d'une solution isotonique par dissolution}


%%%% objectifs
\begin{objectifs}
  \item Revoir la préparation d'une solution par dissolution.
  \item Revoir la concentration massique.
\end{objectifs}

\begin{contexte}
  Le glucose (sucre) contenu dans nos muscles permet à notre corps de fournir un effort intensif.
  Cependant, les réserves en glucose sont limitées, il faut donc les renouveler pour continuer à fournir un effort important.
  Un moyen efficace de renouveler ces ressources est de boire avant et pendant l'effort des boissons isotoniques.
  Une boisson isotonique contient une quantité bien précise de glucose.
  
  \problematique{
    Comment préparer une boisson isotonique ?
  }
\end{contexte}
\bigskip


%%%%
\begin{doc}{Solution}{doc:TP1_solution}
  \begin{importants}
     Une \important{solution} est un mélange homogène.
     Le \important{solvant} est le composant majoritaire du mélange. Le \important{soluté} est l'espèce qui est dispersée dans le solvant.
  \end{importants}
\end{doc}

\begin{doc}{Notion de concentration massique}{doc:TP1_concentration_massique}
  \begin{importants}
    La \important{concentration} massique d’une espèce en solution dans un solvant, est notée $C_m$.
    La concentration massique représente la masse $m_\solute$ de soluté (c'est à dire d'espèce dissoute) dans un volume $V_\solution$ de solution.
    On a alors la relation :
    \begin{equation*}
      c_m = \dfrac{ m_\solute }{ V_\solution }
    \end{equation*}
  \end{importants}

  \exemples les solutions ci-dessous contiennent un nombre de plus en plus petit de particules de masse $m = \qty{1}{\g}$.
  Comme le volume des solutions diminue aussi, la concentration massique reste identique.
  %
  \vspace*{-30pt}
  \begin{center}
    \begin{tblr}{
      colspec = {c c c}, width = 0.5\linewidth
    }
      \image{0.3}{images/chimie/concentration0001} &
      \image{0.3}{images/chimie/concentration0002} &
      \image{0.3}{images/chimie/concentration0003} \\
      \qty{8}{\g} dans \qty{1,00}{\litre} & \qty{4}{\g} dans \qty{0,50}{\litre} & \qty{2}{\g} dans \qty{0,25}{\litre} \\
      $c_m = \qty{8}{\g/\litre}$  & $c_m = \qty{8}{\g/\litre}$  & $c_m = \qty{8}{\g/\litre}$
    \end{tblr}
  \end{center}
\end{doc}

%%
\question{
  Donner l'unité de la concentration massique $c_m$. Citer une autre grandeur qui s'exprime avec la même unité, s'agit-il de la même chose ?
}{
  Unité : \unit{\g\per\litre}. C'est l'unité de la masse volumique, qui représente la densité d'un corps.
}{2}

\begin{doc}{Boisson isotonique d'une joggeuse}{doc:TP1_boisson_joggeuse}
  Avant de partir courir, une joggeuse se prépare une boisson isotonique.
  Elle introduit \qty{10}{\g} de sel \chemfig{NaCl} et 6 morceaux de glucose \bruteCHO{6}{12}{6} (du sucre) de \qty{5}{\g} chacun dans une bouteille de \qty{1}{\litre}, qu'elle remplit d'eau.
\end{doc}

\question{
  Calculer la concentration massique en chlorure de sodium \chemfig{NaCl}, puis en glucose.
}{
  \begin{align*}
    c_{m,\text{sel}} &= \dfrac{\qty{10}{\g}}{\qty{1}{\litre}} = \qty{10}{\g\per\litre} \\
    c_{m,\text{sucre}} &= \dfrac{\qty{6\times5}{\g}}{\qty{1}{\litre}} = \qty{30}{\g\per\litre}
  \end{align*}
}{3}

\question{
  Calculer la masse de sel et la masse de sucre qu'il faut mettre dans une fiole jaugée de $\qty{100}{\mL}$ pour réaliser la même boisson isotonique.
}{
  \begin{align*}
    m_\text{sel} &= \qty{10}{\g\per\litre} \times \qty{0,100}{\litre} = \qty{1,0}{\g} \\
    m_\text{sucre} &= \qty{30}{\g\per\litre} \times \qty{0,100}{\litre} = \qty{3,0}{\g}
  \end{align*}
}{3}

\numeroQuestion
Remettre dans l'ordre le protocole de dissolution.

\begin{boite}
  \vAligne{12cm}
\end{boite}

\numeroQuestion Une fois validé, réaliser le protocole de dissolution pour préparer la boisson isotonique.
