%%%%
\tetePremStssChim

%%%% titre
\numeroActivite{5}
\vspace*{-30pt}
\titreActivite{Autoprotolyse de l'eau}

\begin{objectifs}
  \item Définir un acide et une base selon le modèle de Br\o{}nsted
  \item Écrire l'équation d'une réaction acido-basique à partir des couples acide/base.
\end{objectifs}

\begin{contexte}
  Les solutions sont acides si elles contiennent majoritairement des ions oxonium \oxonium
  et basiques si elles contiennent majoritairement des ions hydroxydes \hydroxyde.
  Si les concentrations molaires de ces ions deviennent importantes, les solutions acides ou basiques présentent un danger.

  \problematique{
    Pourquoi n'existe-t-il pas des solutions contenant en même temps des ions oxonium et des ions hydroxyde en quantité importante ?
  }
\end{contexte}

\begin{doc}{L'autoprotolyse de l'eau}{doc:A3_modele_bronsted}
  L'eau pure ne contient pas uniquement des molécules d'eau \chemfig{H_2O}.
  Elle contient toujours des \important{ions oxonium} \oxonium et des \important{ions hydroxyde} \hydroxyde : ce sont les ions de l'eau.

  Ces ions existent car les molécules d'eau \chemfig{H_2O} s'échangent des protons \chemfig{H^+}.
  En effet, l'eau est une espèce chimique spéciale, car elle peut jouer le rôle d'un acide comme d'une base.
  \begin{importants}  
    On dit que l'eau est une molécule \important{ampholyte}, ou que c'est une espèce \important{amphotère}.
  \end{importants}

  On obtient donc cette réaction acido-basique, qui s’appelle l’autoprotolyse de l’eau : 
  \begin{center}
    \chemfig{H_2O} + \chemfig{H_2 O} = \oxonium + \hydroxyde
  \end{center}
  \attention L'égalité indique que la réaction peut se faire dans les deux sens.
\end{doc}


\question{
  Donner les deux couples acides/bases de l'eau.
}{
  \eau/\hydroxyde et \oxonium/\eau sont les deux couples acides bases de l'eau.
}{2}


\begin{doc}{Le produit ionique de l'eau}{doc:produit_ionique}
  \begin{importants}  
    Pour toute solution aqueuse diluée à une température fixe, le produit des concentrations molaires en ions oxonium \oxonium et hydroxyde \hydroxyde reste constant.
  \end{importants}
  
  Ce produit s’appelle le produit ionique de l’eau, noté Ke, sans unité.  
  \begin{center}
    Ke = $\left[ \oxonium \right] \times \left[ \hydroxyde \right]$
  \end{center}
  Lorsque la température est de \qty{25}{\degreeCelsius}, Ke = \num{e-14}.
  Les concentrations molaires sont données en \unit{\mole/\litre}.
\end{doc}

\question{
  Calculer les concentrations molaires [\oxonium] et [\hydroxyde] pour de l'eau pure à \qty{25}{\degreeCelsius}.
}{

}{3}

\newpage
\question{
  Pourquoi dit-on que l'eau est neutre d'un point de vue acido-basique ?
}{
  
}{2}

\question{
  Calculer les concentrations molaires [\oxonium] et [\hydroxyde] pour des solutions avec un pH = 2, 6, 8, 12.
}{
  
}{10}

\question{
  Comment varie [\hydroxyde] lorsque [\oxonium] augmente ?
}{
  La concentration en ions hydroxyde diminue dans celle en ion oxonium augmente.
}{2}

\question{
  Répondre à la problématique de la séance.
}{
  Les ions oxonium et hydroxyde sont en équilibres dans une solution aqueuse, à cause de la réaction d'autoprotolyse de l'eau.
}{3}