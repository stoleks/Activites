%%%%
\tetePremStssChim
\titreActivite{Un précurseur de la dopamine}[Révision]

\exercice{La DOPA}

\medskip
La DOPA, appelée en nomenclature systématique 3,4-dihydroxy-L-phenylalanine,
est une molécule chirale très étudiée en neurobiochimie et dans l’industrie pharmaceutique.

\medskip
Elle est un précurseur de la dopamine et constitue actuellement le médicament le plus utilisé dans le traitement de la maladie de Parkinson malgré des effets secondaires sérieux.

\medskip
\begin{donnees}
  \item $M(H) = \qty{1} {\g\per\mole}$
  \item $M(C) = \qty{12} {\g\per\mole}$
  \item $M(N) = \qty{14} {\g\per\mole}$
  \item $M(O) = \qty{16} {\g\per\mole}$
  \item \qty{1000}{\micro\g} = \qty{1}{\mg}
  \item \qty{1000}{\mg} = \qty{1}{\g}
\end{donnees}

\bigskip
\question{
  Contre quelle maladie permet de lutter la DOPA ?    
}{
  La DOPA permet de lutter contre la maladie de Parkinson.
}[2]

\question{
  La formule brute de la DOPA est \chemfig{C_9 H_{11} N O_4}.
  Calculer la masse molaire moléculaire de la DOPA. 
}{
  \begin{equation*}
    \masseMol{C_9 H_{11} N O_4} = 11M(H) + 9M(C) + 4M(O) + M(N) 
    = \qty{197}{\g\per\mole}
  \end{equation*}
}[2]

On prescrit de la DOPA a une patiente. 
Elle doit ingérer un médicament contenant \qty{1000}{\mg} de DOPA.

\question{
  Calculer la quantité de matière de DOPA contenue dans le médicament.
}{
  Pour trouver la quantité de DOPA dans le médicament, on divise la masse de DOPA par la masse molaire
  \begin{equation*}
    n(\text{DOPA}) = \dfrac{ m(\text{DOPA}) }{ \masseMol{DOPA} }
    =  \dfrac{ \num{1000}\dfrac{\unit{\g}}{1000} }{ \qty{197}{\g\per\mole} }
    = \qty{5.1e-3}{\mole}
  \end{equation*}
}[2]

\question{
  La patiente a \qty{5.5}{\litre} de sang dans le corps.
  Calculer la concentration molaire en DOPA dans le sang de la patiente après ingestion de son médicament.
}{
  Il faut diviser la quantité de DOPA par le volume de sang
  \begin{equation*}
    c(\text{DOPA}) = \dfrac{ n(\text{DOPA}) }{V_\text{sang}}
    = \dfrac{ \qty{5.1e-3}{\mole} }{ \qty{5.5}{\litre} }
    = \qty{9.8e-4}{\mole\per\litre}
  \end{equation*}
}[2]

\question{
  Le sang est une solution aqueuse, composée d'eau, de globules rouge, de plaquettes et de globules blanc.
  Indiquer quels sont les solutés et le solvant dans le sang.
}{
  Le solvant est l'eau, les solutés sont les globules blanc et rouge et les plaquettes.
}[2]

\newpage
La formule chimique de la DOPA est 
\begin{center}
  \chemfig{!\DOPA}
\end{center}
On va noter cette forme \chemfig{D} pour simplifier. 
Une fois dans le sang, on trouve la DOPA sous la forme d'ion DOPAnium : 
\begin{center}
  \chemfig{!\DOPAH}
\end{center}
Que l'on va noter \chemfig{DH^+}.

\question{
  Indiquer laquelle de ces deux formes est l'acide et laquelle est la base, puis donner le couple acide base associé à la DOPA.
}{
  L'ion DOPAnium est l'acide, car il contient plus d'hydrogène que la DOPA. Le couple acide/base est donc \chemfig{DH^+}/\chemfig{D}.
}[2]

\question{
  Dans le sang, la DOPA réagit avec l'ion oxonium \oxonium. 
  Donner la réaction acide/base entre l'ion oxonium et la DOPA.
  \begin{donnees}
    \item Couple de l'eau : \oxonium/\eau
    \item Couple de la DOPA : \chemfig{DH^+}/\chemfig{D}
  \end{donnees}\phantom{b}
}{
  \begin{equation*}
    \chemfig{D} + \oxonium \reaction \chemfig{DH^+} + \eau
  \end{equation*}
}[4]