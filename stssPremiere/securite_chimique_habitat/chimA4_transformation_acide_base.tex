%%%%
\tetePremStssChim

%%%% titre
\numeroActivite{4}
\titreActivite{Transformations acido-basique}

\begin{objectifs}
  \item Définir un acide et une base selon le modèle de Br\o{}nsted.
  \item Savoir écrire la demi-réaction d'un couple acido-basique.
  \item Écrire une réaction acido-basique à partir des couples acide/base.
\end{objectifs}

\begin{contexte}
  Les transformations acido-basiques sont très courante dans la vie de tous les jours, comme par exemple quand on utilise du vinaigre blanc pour enlever le calcaire accumulé dans une bouilloire.

  \problematique{
    Comment modéliser une transformation chimique avec une réaction chimique ?
  }
\end{contexte}

\begin{doc}{Acide et base selon le modèle de Br\o{}nsted}{doc:A3_acide_base_bronsted}
  D'un point de vue microscopique, on peut modéliser les transformations acido-basique à l'aide de simple échange d'ion hydrogène \chemfig{H^+}.

  \begin{multicols}{2}
  \begin{importants}
    Un \important{acide} est une molécule capable de \important{céder} un ion \chemfig{H^+}.
  \end{importants}
  \exemple l'acide carbonique peut céder un ion \chemfig{H^+} pour former l'ion hydrogénocarbonate
  \begin{center}
    \chemfig{H_2CO_3} = \chemfig{HCO^{-}_3} + \chemfig{H^+}
  \end{center}

  \begin{importants}
    Une \important{base} est une molécule capable de \important{capter} un ion \chemfig{H^+}.
  \end{importants}
  \exemple l'ammoniac peut capter un ion \chemfig{H^+} pour former l'ion ammonium
  \begin{center}
    \chemfig{NH_3} + \chemfig{H^+} = \chemfig{NH_4^+}
  \end{center}
  \end{multicols}
\end{doc}

\begin{doc}{Couple acido-basique}{doc:A3_couple_acido_basique}
  \begin{importants}
    Un \important{acide \chemfig{AH}} et une \important{base \chemfig{A^{-}}} sont conjugués s'ils sont reliés par des échanges d'ions hydrogène \chemfig{H^+}.
    \begin{center}
      \chemfig{AH} = \chemfig{A^{-}} + \chemfig{H^+}
    \end{center}
    On dit alors que l'acide et la base forment un \important{couple acido-basique,} qu'on note \chemfig{AH/A^{-}} (acide/base).
  \end{importants}

  \attention Pour passer de l'acide à la base, il suffit donc d'enlever un ou deux hydrogène dans la molécule.
\end{doc}

\question{
  Identifier les couples acido-basiques parmi les deux demi-réactions du document~\ref{doc:A3_acide_base_bronsted}.  
}{}{1}

\begin{doc}{Transformation acido-basique}{doc:A3_transfo_acido_basique}
  \begin{importants}
    Une \important{réaction acido-basique} a lieue quand on met en présence \important{l'espèce basique} d'un couple avec \important{l'espèce acide} d'un autre couple.

    Les produits formés sont alors les \important{espèces conjugués} des deux réactifs.
  \end{importants}

  \exemple L'acide carbonique \chemfig{H_2CO_3} peut réagir avec l'ammoniac \chemfig{NH_3}
  \begin{center}
    \chemfig{H_2CO_3} + \chemfig{NH_3} \reaction \chemfig{HCO_3^{-}} + \chemfig{NH_4^+}
  \end{center}
\end{doc}

\begin{doc}{Écriture d'une réaction acido-basique à l'aide des demi-réactions}{doc:A3_methode}
  \begin{importants}
    Pour écrire une réaction acido-basique, on peut suivre la méthode suivante :
    \begin{enumeration}
      \item \important{Repérer} dans chaque couple quel acide réagit avec quel base.
      \item \important{Écrire} les demi-réactions pour chaque couple dans le « bon » sens.
      \item \important{Ajuster} les demi-réactions pour qu'il y ait le même nombre d'ions hydrogène échangés.
      \item \important{Additionner} les deux demi-réactions afin d'obtenir la réaction acido-basique
    \end{enumeration}
  \end{importants}

  \exemple On a deux couples : \oxonium/\eau et \chemfig{HCl}/\chemfig{Cl^{-}}.
  
  On fait réagir l'acide chlorhydrique \chemfig{HCl} avec l'eau \eau. 
  On a donc les demi-réactions suivantes :
  \begin{align*}
    \chemfig{HCl} &= \chemfig{Cl^{-}} + \chemfig{H^+} \\
    \eau + \chemfig{H^+} &= \oxonium
  \end{align*}
  On peut donc additionner les deux demi-réactions (côté par côté) pour obtenir la réaction entre l'eau et l'acide chlorhydrique
  \begin{equation*}
    \chemfig{HCl} + \eau = \chemfig{Cl^{-}} + \oxonium 
  \end{equation*}
  
  \attention Il ne doit pas y avoir d'ions hydrogène dans la réaction finale !
\end{doc}

\begin{doc}{Détartrage d'une bouilloire}{doc:A3_detartrage}
  Pour enlever le calcaire accumulé dans une bouilloire, on peut y verser du vinaigre blanc.
  
  Le calcaire est composé d'ions calcium \chemfig{Ca^{2+}} et \important{d'ions carbonate \chemfig{CO_3^{2-}}.} Le vinaigre est composé \important{d'acide éthanoïque \chemfig{CH_3COOH}.}
  
  Quand on verse du vinaigre sur du calcaire, une réaction acido-basique transforme le calcaire en dioxyde de carbone dissout dans l'eau, noté \chemfig{H_2O,CO_2}.
  \smallskip

  \important{Couples acido-basique :}
  \begin{listePoints}
    \item \chemfig{H_2O,CO_2}/\chemfig{CO_3^{2-}}
    \item \chemfig{CH_3COOH}/\chemfig{CH_3COO^{-}}
  \end{listePoints}
\end{doc}

\question{
  Identifier l'acide et la base qui réagissent ensemble pendant le détartrage.
}{}{2}

\question{
  Écrire les demi-réactions associées dans le bon sens.
}{}{2}

\question{
  Ajuster et additionner les demi-réactions pour obtenir la réaction acido-basique.
}{}{3}