%%%%
\tetePremStssChim
\titreActivite{Acide, base et entretien des cheveux}

%%%%
\begin{objectifs}
  \item Définir un acide et une base selon le modèle de Br\o{}nsted.
  \item Écrire l’équation d’une réaction acido-basique à partir des couples acide/base.
\end{objectifs}

\begin{contexte}
  Sur certains sites de beauté, il est conseillé d’utiliser du vinaigre de cidre et du bicarbonate de soude pour entretenir ses cheveux.
  Inès, âgée de 8 ans, se verse les deux produits sur les cheveux sans les diluer.
  Catastrophe ! Une émulsion gazeuse se forme aussitôt sur sa tête !

  \problematique{
    Que s'est-il passé quand le bicarbonate de soude et le vinaigre de cidre se sont mélangés ?
  }
\end{contexte}

%%%%
\begin{doc}{Le modèle de Br\o{}nsted}
  \qrcodeCote[3]{https://www.youtube.com/watch?v=EztCKi3oJEI}

  Joannes Nicolaus Br\o{}nsted est un chimiste danois du début du 20ème siècle.
  Il est connu pour sa définition des substances acides et basiques : 
  \begin{importants}    
    \begin{listePoints}
      \item Un acide est une molécule capable de libérer un ion \ionHydrogene (un proton)
      \item Une base est une molécule qui reçoit un ion \ionHydrogene
    \end{listePoints}
  \end{importants}
  
  L'acide, noté \chemfig{AH}, se transforme en sa base conjuguée, notée \chemfig{A^{-}}, en perdant un proton \ionHydrogene.
  La base conjuguée \chemfig{A^{-}} se transforme en l'acide \chemfig{AH} quand elle capte un proton \ionHydrogene.

  \begin{importants}
    On parle de couple acide/base noté ici \chemfig{AH}/\chemfig{A^{-}}.    
  \end{importants}
  \attention l'acide est toujours à gauche et la base est toujours à droite.

  \exemple
  \texteTrou(0){\chemfig{HCl}/\chlorure, l'acide est \chemfig{HCl} et la base \chlorure est dans ce couple.}
\end{doc}

\question{
  Indiquer l'acide et la base du couple acide/base \eau/\hydroxyde.
}{}[1]

\begin{doc}{Réaction acido-basique}
  \begin{importants}  
    Lors d'une réaction chimique acido-basique, l'acide d'un couple réagit avec la base d'un autre couple.
  \end{importants}
  
  \exemple On a deux couples : \oxonium/\eau et \ammonium/\ammoniac.

  Si on mélange les ions oxonium \chemfig{H_3 O^{+}}, un acide, avec l'ammoniac \ammoniac, une base, on va avoir une réaction chimique
  \vspace*{-12pt}
  \begin{center}
    \oxonium + \ammoniac \reaction \eau + \ammonium
  \end{center}
\end{doc}

\question{
  Établir la réaction acido-basique entre le couple \oxonium/\eau et le couple \eau/\hydroxyde.  
}{
  \begin{equation*}
    \oxonium + \hydroxyde = \eau + \eau = 2\eau
  \end{equation*}
}[2]

%%
\newpage
\vspace*{-40pt}

\begin{doc}{Le bicarbonate de soude}[\label{doc:bicarbonate}]
  \begin{wrapfigure}{r}{0.2\linewidth}
    \vspace*{-38pt}
    \centering
    \image{0.63}{images/exterieures/bicarbonate}
  \end{wrapfigure}
  Le bicarbonate de sodium ou bicarbonate de soude (abus de langage) est nommé hydrogénocarbonate de sodium en nomenclature moderne.
  C’est un composé chimique dont la formule brute est \bicarbonateDeSodium.
  Il se présente sous la forme de fins cristaux blancs, solubles dans l’eau, qui forme les ions sodium \ionSodium et hydrogénocarbonate \bicarbonate en solution.
\end{doc}

\begin{doc}{Le vinaigre de cidre}[\label{doc:vinaigre}]
  \begin{wrapfigure}{l}{0.2\linewidth}
    \vspace*{-22pt}
    \centering
    \image{0.65}{images/exterieures/vinaigre}
  \end{wrapfigure}
  Le vinaigre est une solution aqueuse à faible concentration en acide éthanoïque de
  formule \chemfig{CH_3COOH}, qui rentre principalement dans l'alimentation humaine comme condiment et conservateur alimentaire. 
  Le vinaigre résulte d'une transformation d'une solution aqueuse d'éthanol (le vin ou ici le cidre) exposée à l'air, et fermenté à l’aide de micro-organisme.
  Cela explique son étymologie de « vin aigre » devenu « vinaigre ».
  Le vinaigre de cidre à 8° comporte \qty{8}{\g} d’acide éthanoïque par litre de solution
\end{doc}

\begin{doc}{Couples acide/base à connaitre}[\label{doc:couple_acide_base}]
  \centering
  \begin{tblr}{
    hlines, vlines, colspec = {
      Q[c,m, wd=0.2\linewidth] c Q[c,m, wd=0.19\linewidth] c Q[c,m, wd=0.19\linewidth]
    },
    row{1} = {couleurSec-100},
    row{2} = {couleurSec-50},
    width = \linewidth,
  }
    & \SetCell[c=2]{c} Forme acide & & \SetCell[c=2]{c} Forme basique \\
    Couple acide/base & Formule brute & Nom & Formule brute & Nom \\
    \oxonium/\eau                 & \oxonium & ion oxonium                 & \eau & eau \\
    \eau/\hydroxyde               & \eau & eau                             & \hydroxyde & ion hydroxyde \\
    \chemfig{HCl}/\chlorure       & \chemfig{HCl} & acide chlorhydrique    & \chlorure & ion  \\
    {\chemfig{CH_3COOH}/ \\
    \chemfig{CH_3COO^{-}}}        & \chemfig{CH_3COOH} & acide éthanoïque  & \chemfig{CH_3COO^{-}} & ion éthanoate \\
    \acideCarbonique/\bicarbonate & \acideCarbonique & acide carbonique    & \bicarbonate & ion hydrogéno-carbonate \\
    \bicarbonate/\carbonate       & \bicarbonate & ion hydrogéno-carbonate & \carbonate & ion carbonate \\
    \ammonium/\ammoniac           & \ammonium & ion ammonium               & \ammoniac & ammoniac \\
  \end{tblr}
\end{doc}

\question{
  En utilisant les documents~\ref{doc:bicarbonate},~\ref{doc:vinaigre} et~\ref{doc:couple_acide_base}, donner les couples acide/base présents dans le bicarbonate de soude et dans le vinaigre de cidre.
}{
  Bicarbonate de soude : \acideCarbonique/\bicarbonate et \bicarbonate/\carbonate
  
  Acide éthanoïque : \chemfig{CH_3COOH}/\chemfig{CH_3COO^{-}}
}[2]

\question{
  Établir la réaction acido-basique qui a lieue quand on mélange du bicarbonate de soude et du vinaigre de cidre.
}{
  \begin{equation*}
    \chemfig{CH_3COOH} + \bicarbonate = \chemfig{CH_3COO^{-}} + \acideCarbonique
  \end{equation*}
}[2]

\question{
  En pratique, l'acide carbonique se décompose spontanément en eau et en dioxyde de carbone.
  Modifier la réaction acido-basique en conséquence.
}{
  \begin{equation*}
    \acideCarbonique = \eau + \dioxydeDeCarbone
  \end{equation*}
}[1]
