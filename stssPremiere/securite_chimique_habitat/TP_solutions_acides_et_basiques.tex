\tetePremStssChim
\vspace*{-36pt}
\titreTP{Solutions acides et basiques}

\begin{objectifs}
  \item Connaître la relation $[\oxonium] = 10^{-pH}$.
  \item Savoir mesurer le pH d'une solution.
\end{objectifs}


%%%% Activité
\begin{contexte}
  Le vinaigre blanc, l'eau de javel ou simplement l'eau du robinet peuvent servir à l'entretien de la maison.
  Ces solutions aqueuses n'ont pas les mêmes propriétés, notamment parce qu'elles n'ont pas le même pH.

  \problematique{
    Comment mesurer le pH de ces solutions ? Quelles informations apporte-t-il ?
  }
\end{contexte}


%%%% Documents 
\begin{doc}{Le pH des solutions aqueuses}
  \qrcodeCote[3]{https://www.youtube.com/watch?v=hJ4y2B8ueUo}
  
  Sur les étiquettes d'eau minérales, on peut lire le pH suivi d'une valeur voisine de 7.
  Le pH est une grandeur qui se mesure et qui \important{n'a pas d'unité.}
  
  \begin{importants}
    Pour les solutions aqueuse la valeur du pH est comprise entre
    \begin{center}
      \texteTrou{0} $\leq$ pH $\leq$ \texteTrou{14}
    \end{center}
  \end{importants}

  \begin{importants}
    \begin{listePoints}
      \item Les solutions \important{acides} ont un pH \texteTrou{< 7}
      \item Les solutions \important{neutres} ont un pH \texteTrou(0){= 7. C'est le pH de l'eau.}
      \item Les solutions \important{basiques} ont un pH \texteTrou{> 7}
    \end{listePoints}
  \end{importants}
\end{doc}

\begin{doc}{La mesure du pH d'une solution}
  \pointCyan \important{L’indicateur coloré}
  C’est une espèce chimique qui change de \important{couleur} en fonction de la valeur du pH de la solution.
  Introduire quelques gouttes d’un indicateur coloré permet donc de déterminer le caractère acide, basique ou neutre d’une solution. \\
  \flecheLongue \texteTrou(0){Méthode \important{peu} précise.}

  \pointCyan \important{Le papier pH}
  Le papier pH est une bande de papier imbibée d’un indicateur universel.
  En déposant une goutte de solution sur un morceau de papier pH, on détermine une valeur \important{approximative} du pH en \important{comparant} la couleur obtenue avec celle du \important{nuancier.} \\  
  \flecheLongue \texteTrou(0){Méthode \important{moyennement} précise.}

  \pointCyan \important{Le pH-mètre}
  Le pH-mètre est un appareil de mesure constitué d’une électrode reliée à un boîtier électronique indiquant la valeur du pH.
  Le pH-mètre doit être étalonné avec deux solutions tampons (dont le pH est constant et connu).
  Une fois étalonné, on rince l’électrode et on la plonge dans la solution aqueuse dont on cherche à déterminer le pH. \\
  \flecheLongue \texteTrou(0){Méthode \important{très} précise.}
\end{doc}

\begin{doc}{Protocoles de mesure}
  Avec un indicateur coloré :
  \begin{protocole}
    \item Verser quelques millilitre de la solution dont on étudie le pH dans 3 tubes à essais
    \item Ajouter quelques gouttes de Bleu de Bromothymol (BBT)
    \item En milieu acide le BBT est jaune, bleu en milieu basique et vert en milieu neutre.
  \end{protocole}
  
  Avec le papier pH :
  \begin{protocole}
    \item Placer un petit morceau de papier pH dans une coupelle
    \item Y déposer quelques gouttes de solution
    \item Comparer la couleur du papier obtenue au nuancier pour déterminer la valeur du pH
  \end{protocole}
  
  Avec le pH-mètre :
  \begin{protocole}
    \item Verser un peu de solution dans un bécher
    \item Plonger la sonde du pH-mètre dans la solution dont on cherche le pH
    \item Lire la valeur du pH sur l’écran
  \end{protocole}
\end{doc}

\mesure Mesurer et noter la valeur du pH du vinaigre blanc, de l'eau de javel et de l’eau du robinet, puis compléter le tableau ci-dessous

\begin{center}   
  \begin{tblr}{
      hlines, vlines, width = \linewidth,
      colspec = {c *{3}{Q[c, wd=0.225\linewidth]}},
      row{1} = {couleurSec-100},
    }
    Mesure de pH      & Eau & vinaigre blanc & javel \\
    indicateur coloré & & & \\
    papier pH         & & & \\
    pH-mètre          & & &
  \end{tblr}
\end{center}
%\smallskip

\question{
  Indiquer le caractère acide, basique ou neutre de chaque solution, en justifiant.
}{
  Eau : neutre. Javel : basique. Vinaigre blanc : acide.
}[3]

\question{
  Quelle est la méthode la plus précise pour mesurer le pH ?
}{
  C'est le pH-mètre qui permet d'avoir la meilleure précision.
}[1]


%%
\begin{doc}{Concentration en ions oxonium \oxonium d’une solution et pH}
  La valeur du pH d’une solution permet de connaître la concentration molaire
  (en \unit{\mole\per\litre}) en \important{ions oxonium \oxonium} dans la solution.
  \begin{importants}
    Cette concentration se note [\oxonium] et elle est définie par :
    \begin{equation*}
      [\oxonium] = 10^{-\text{pH}}
    \end{equation*}
  \end{importants}
\end{doc}

\question{
  Calculer la concentration en ions oxonium \oxonium de chaque solutions.
}{
  
}[3]