%%%%
\tetePremStssElec

%%%% titre
\vspace*{-36pt}
\numeroActivite{1}
\titreActivite{Prévenir les risques d'électrisation}


%%%% objectifs
\begin{objectifs}
  \item Revoir l'intensité du courant et la tension électrique
  \item Connaître les dangers liés à une électrisation
\end{objectifs}

\begin{contexte}
  Anna projette de réaliser quelques travaux sur les prises électriques dans sa maison.
  
  \problematique{
    Comment éviter tout risque d'électrisation ?
  }
\end{contexte}


%%%% docs
\begin{doc}{La prise de terre}{doc:A1_prise_terre}
  \begin{wrapfigure}[3]{r}{0.1\linewidth}
    \centering
    \vspace*{-32pt}
    \qrcode{https://www.youtube.com/watch?v=BazTgHMuA8k}
  \end{wrapfigure}
  La prise de terre a une importance vitale, car la prise de terre assure que le courant électrique s'évacue dans le sol lorsqu'un appareil est mal isolé.
  Concrètement la prise de terre est un câble métallique qui finit sur un piquet enfoui dans le sol.
  
  \begin{wrapfigure}{l}{0.2\linewidth}
    \centering
    \vspace*{-14pt}
    \image{1}{images/photos/prise_terre.jpg}
  \end{wrapfigure}
  Ce câble permet de dévier le courant électrique qui s'échapperait d'un appareil dans la terre, d'où le nom de prise de terre.

  Des pertes de courant peuvent survenir lorsqu'un câble d'alimentation abîmé est dénudé et que les fils électriques entrent en contact avec l'armature métallique d'un l'appareil.
  Sans prise de terre, le courant traverserait le corps de la première personne qui toucherait l'appareil !
\end{doc}

\begin{doc}{Tension et intensité}{doc:A1_tension_intensite}
  \begin{importants}
    Un courant électrique est caractérisé par deux grandeurs : 
    \begin{listePoints}
      \item la tension électrique, exprimée en volt \unit{\volt}.
      \item l'intensité du courant, exprimée en ampère \unit{\ampere}.
    \end{listePoints}
  \end{importants}
  
  La tension détermine l'énergie libérée par le courant.
  En pratique, plus la tension est élevée, plus le risque de brûlure est grand.
  
  L'intensité du courant traversant un corps est responsable de contractions musculaires et de ruptures de fibres nerveuses, appelée « sidération ».
  Lorsque l'intensité augmente, des réactions de plus en plus intense apparaissent, allant jusqu'à la mort par arrêt cardiaque.
\end{doc}

\begin{doc}{Accident électrique}{doc:A1_accident_elec}
  Dans les accidents électrique, on distingue l’électrisation de l’électrocution :
  \begin{importants}  
    \begin{listePoints}      
      \item l’électrisation : c’est la réaction du corps due à un contact accidentel avec l’électricité ;
      \item l’électrocution : c’est l’électrisation qui entraine la mort.
    \end{listePoints}
  \end{importants}
\end{doc}

\begin{doc}{Résistance du corps humain}{doc:A1_resistance_humain}
  Lorsqu’il est soumis à une tension électrique, le corps humain conduit le courant électrique.
  \begin{tableau}{|c |c |c |c |c |}
    État de la peau & Peau sèche & Peau humide & Peau mouillée & Peau immergée \\
    Résistance $R$ & \qty{5000}{\ohm} & \qty{2500}{\ohm} & \qty{1000}{\ohm} & \qty{500}{\ohm}
  \end{tableau}
\end{doc}


\begin{doc}{Effet physiologique observée}{doc:A1_effet_physiologique}
  \begin{center}
    \image{0.5}{images/sante/danger_electrisation_ampere}
    \image{0.46}{images/donnees/effet_physiologique_electrisation}
    
    \begin{tblr}{
      hlines, vlines, colspec = {l X[l]},
      row{1} = {couleurSec-100, c},
      row{2} = {blue-150},   row{3} = {green-100},
      row{4} = {yellow-150}, row{5} = {red-100},
    }
      Zone & Principaux effets physiologiques constatés \\
      AC-1 & Aucune réaction \\
      AC-2 & Sensations désagréables mais pas d’effets physiologiques dangereux \\
      AC-3 & Tétanisation musculaire avec risque de paralysie respiratoire mais sans fibrillation ventriculaire \\
      AC-4 & Fibrillation ventriculaire, possibilités d’arrêt respiratoire, d’arrêt cardiaque, de brûlures graves, etc.
    \end{tblr}
    \vspace*{2pt}
    
    Effets physiologiques du courant alternatif en fonction de l’intensité \\
    du courant électrique et de la durée d’exposition
  \end{center}
\end{doc}


%%%%
\question{
  Anna commence par descendre à la cave pour vérifier l'état de l'installation de la prise de terre.
  À l’aide du document~\ref{doc:A1_prise_terre}, expliquer quels dangers sont écartés grâce à la prise de terre.
}{

}{2}
      
\question{
  Trouver la signification des mots : tétanisation, paralysie et fibrillation ventriculaire.
}{

}{3}

\question{
  Anna touche accidentellement, avec ses mains mouillées, un fil électrique dénudé pendant \qty{100}{\milli\s}.
  Elle est exposée à un courant électrique d’intensité égale à \qty{50}{\milli\ampere}.
  À l’aide du document~\ref{doc:A1_effet_physiologique} prévoir quels vont être les effets physiologiques.
}{

}{3}

\numeroQuestion
En vous aidant des documents et de vos connaissances, prévoir sur une feuille quelle serait la gravité des effets si :
\begin{itemize}
  \item ses mains étaient sèches,
  \item le contact était beaucoup plus long et durait \qty{1}{\s},
  \item la tension d'alimentation était de \qty{25}{\volt} au lieu de \qty{230}{\volt}.
\end{itemize}
Donner alors quelques règles afin d’éviter une électrisation.