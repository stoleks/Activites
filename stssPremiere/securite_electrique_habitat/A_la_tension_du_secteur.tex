\tetePremStssElec
\vspace*{-36pt}
\titreActivite{La tension du secteur}

\begin{objectifs}
  \item Comprendre les caractéristiques de la tension du secteur.
\end{objectifs}

\begin{contexte}
  Camille a acheté un chargeur de téléphone sur un site de commerce en ligne chinois.

  \problematique{
    Ce chargeur peut-il fonctionner avec la tension du secteur fournie en France ?
  }
\end{contexte}


\begin{doc}{Tension du secteur en Europe}
  \begin{wrapfigure}[15]{r}{0.6\linewidth}
    \vspace*{-16pt}
    \centering
    % Oscillogramme
    \def\ver{3.6} % longueur verticale
    \def\hor{5.0} % longueur horizontale
    \begin{tikzpicture}[
      trace/.style={couleurPrim!75!black, ultra thick, samples = 80},
      screen/.style={couleurSec-50, thick},
      axes/.style={couleurPrim, thick}
    ]
      % Fond de l'écran
      \fill[screen] (-\hor, -\ver) rectangle (\hor, \ver);
      % Grille et axes
      \draw[thin, couleurPrim!30] (-\hor, -\ver) grid (\hor, \ver);
      \draw[axes] (-\hor, 0) -- (\hor, 0); % temps
      \draw[axes] (0, -\ver) -- (0, \ver); % 
      % Graduation
      \pgfmathparse{\hor-1}
      % axe x
      \foreach[parse = true] 
        \i in {-\hor+0.2, -\hor+0.4,..., \hor-0.2} \draw[axes] (\i,-0.1) -- (\i,0.1);
      \foreach[parse = true] \i in {-\hor,...,\hor} \draw[axes] (\i,-0.2) -- (\i,0.2);
      % axe y
      \foreach[parse = true]
        \i in {-\ver+0.2, -\ver+0.4,..., \ver-0.2} \draw[axes] (-0.1,\i) -- (0.1,\i);
      \foreach[parse = true] \i in {-3,...,3} \draw[axes] (-0.2,\i) -- (0.2,\i);
      % Tension sinusoïdale
      \draw[trace] plot(\x, {3.25*sin((1.57*\x + 1.57) r)}); % r = radian
      % Échelle
      \tikzVecteur(-5,-3) (0,1)  {\textbf{100 V}} [right]
      \tikzVecteur(-5,-3) (1,0)  {\textbf{5 ms}} [below right]
      \tikzVecteur*(-4,3.3) (4,0) {\textbf{Période} $\mathbf{T}$ \hspace{12pt}\strut} [below left]
    \end{tikzpicture}

    Oscillogramme de la tension du secteur.
  \end{wrapfigure}
  
  En Europe la tension du secteur est la tension électrique délivrée dans les habitations.

  La tension du secteur est \important{sinusoïdale,} car le courant est \important{alternatif}.
  Elle est définie par deux grandeurs 
  \begin{listePoints}
    \item une \important{fréquence} $f$;
    \item une \important{tension efficace} $U$.
  \end{listePoints}

  La tension efficace est proportionnelle à la valeur maximale que peut atteindre la tension du secteur 
  \begin{equation*}
    U = \dfrac{U_\text{max}}{\sqrt{2}}
  \end{equation*}

  On peut mesurer la valeur maximale et la fréquence sur un \important{oscillogramme.}
\end{doc}

\question{
  Utiliser l'oscillogramme pour calculer la valeur de $U_\text{max}$ et de la période $T$.
}{

}[2]

\question{
  Calculer la tension efficace $U$ à l'aide de la tension maximale $U_\text{max}$.
}{}[2]

\question{
  Calculer la fréquence et $f$ à l'aide de la période $T$.
}{}[2]


\begin{doc}{Caractéristiques du chargeur de téléphone}
  \begin{listePoints}[2]
    \item Entrée : 100-\qty{240}{\volt} AC 50-\qty{60}{\hertz}.
    \item Sortie : \qty{5,00}{\volt} DC.
    \item Ampérage : \qty{1}{\ampere}.
    \item Puissance : \qty{5}{\watt}.
  \end{listePoints}
\end{doc}

\question{
  Camille peut-elle utiliser le chargeur en le branchant sur une prise en France ?
}{}[2]