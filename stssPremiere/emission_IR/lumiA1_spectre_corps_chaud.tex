%%%%
\tetePremStssLumi

%%%% titre
\numeroActivite{1}
\titreActivite{Spectre électromagnétique et rayonnement thermique d'un corps chaud}


%%%% objectifs
\begin{objectifs}
  \item Savoir que la lumière est une onde électromagnétique.
  \item Comprendre le lien entre la température d'un objet et la lumière que l'objet émet.
  \item Connaître la loi de Wien qui relie température et maximum d'émission.
\end{objectifs}

\begin{contexte}
  Les corps très chaud, comme les étoiles, les lampes incandescentes ou la lave en fusion, émettent de la lumière.
  
  \problematique{
    Comment la température d'un corps influence ses émissions lumineuses ?
  }
\end{contexte}


%%%% docs
\begin{doc}{La lumière : une onde électromagnétique}{doc:A1_onde_electromagnetique}
  \begin{importants}
    La lumière est une \important{onde électromagnétique.}
    C'est-à-dire un champ électrique et un champ magnétique qui se propagent.
  \end{importants}

  Quand on disperse de la lumière blanche avec un prisme ou un réseau, on observe qu'elle est composée des couleurs de l'arc-en-ciel.
  \begin{center}
    \image{0.6}{images/lumiere/spectre_visible}
  \end{center}
  \begin{center}
    \image{0.8}{images/lumiere/spectre_EM}
  \end{center}

  À chaque couleur est associée une radiation de longueur d'onde $\lambda$, on appelle ça le \important{spectre électromagnétique.}

  \begin{importants}    
    On va s'intéresser à trois domaines du spectre électromagnétique.
    \begin{listePoints}
      \item \important{Domaine visible :} radiation avec une longueur d'onde entre \qty{400}{\nm} et \qty{800}{\nm}.
      \item \important{Domaine InfraRouge (IR) :} radiation avec une longueur d'onde supérieure à \qty{800}{\nm} et inférieure à \qty{1}{\mm}.
      \item \important{Domaine UltraViolet (UV) :} radiation avec une longueur d'onde inférieure à \qty{400}{\nm} et supérieure à \qty{1}{\nm}.
    \end{listePoints}
  \end{importants}
\end{doc}

\question{
  Est-ce que les longueurs d'ondes augmentent quand on passe du rouge au violet ?
}{}{2}


%%
\newpage
\vspace*{-36pt}
\begin{doc}{Spectre du rayonnement thermique des corps chauds}{doc:A1_spectre_corps_chaud}
  \begin{wrapfigure}[5]{r}{0.45\linewidth}
    \vspace*{-20pt}
    \centering
    \image{0.95}{images/thermodynamique/loi_emission_corps_chaud}
  \end{wrapfigure}
  \strut\vspace*{-22pt}
  
  \begin{importants}  
    Un corps chauffé à une température $T$ émet des ondes électromagnétiques, c'est-à-dire de la lumière.
    Le \important{spectre} de la lumière émise est une fonction continue, qu'on appelle le \important{rayonnement thermique des corps chauds.}
  \end{importants}

  Le graphique à droite représente le spectre du rayonnement thermique d'un corps chaud à une température $T = \qty{3000}{\kelvin}$.
  Ce spectre à une intensité maximale $I_{max}$ pour une longueur d'onde $\lambda_{max}$.
  \medskip

  \begin{importants}
    Conversion d'un degré Celsius en kelvin :
    $T(\unit{\kelvin}) = T(\unit{\degreeCelsius}) + 273$. 
    \exemple $\qty{20}{\degreeCelsius} = (20 + 273) \unit{\kelvin} = \qty{293}{\kelvin}$.
  \end{importants}
\end{doc}

\question{
  Convertir $T = \qty{37}{\degreeCelsius}$ en kelvin \unit{\kelvin}.
}{}{1}


\begin{doc}{Loi de Wien}{doc:A1_loi_Wien}
  \begin{wrapfigure}[10]{r}{0.45\linewidth}
    \vspace*{-28pt}
    \centering
    \image{0.95}{images/thermodynamique/emission_corps_chaud}
  \end{wrapfigure}
  
  Quand la température d'un corps augmente, on observe deux choses :
  \begin{listeTirets}
    \item l'intensité de la lumière émise est plus élevée, le corps est plus lumineux ;
    \item la longueur d'onde d'intensité maximale $\lambda_{max}$ diminue et la composition de la lumière émise change.
  \end{listeTirets}
  Le deuxième point a une implication très concrète pour nous : un objet change de couleur si sa température est suffisamment élevée.
  \smallskip

  \begin{importants}
    La relation entre la longueur d'onde $\lambda_{max}$ et la température $T$ du corps est donnée par la \important{loi de Wien}
    \begin{equation*}
      \lambda_{max} = \dfrac{C}{T}
    \end{equation*}
    \begin{listePoints}
      \item $\lambda_{max}$ est la longueur d'onde du maximum d'intensité en mètre.
      \item $T$ est la température du corps exprimée en Kelvin noté \unit{\kelvin}.
      \item $C$ est la constante de Wien $C = \qty{2,9e-3}{\kelvin\m}$
    \end{listePoints}
  \end{importants}
\end{doc}


%%%%
\mesure
À l'aide d'un tableur, tracer $\lambda_{max}$ en fonction de la température $T$ avec la loi de Wien, pour $T$ comprise entre \qty{300}{\kelvin} et \qty{10000}{\kelvin}.

\mesure
Utiliser le graphique obtenu pour déterminer la plus petite et la plus haute valeur de la température où $\lambda_{max}$ est dans le domaine du visible.

\question{
  Déterminer à l'aide du tableur la longueur d'onde de la radiation émise par un corps humain à une température de \qty{37}{\degreeCelsius}.
  A quel domaine du spectre électromagnétique appartient-elle ?
}{}{1}