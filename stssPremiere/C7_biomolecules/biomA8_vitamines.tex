%%%%
\tetePremStssStru

%%%% titre
\numeroActivite{8}
\titreActivite{Les vitamines}

\begin{objectifs}
  \item Comprendre ce que c'est qu'une vitamine.
  \item Étudier un exemple de vitamine : la vitamine C.
\end{objectifs}

\begin{contexte}
  Contrairement aux glucides, lipides et protéines, les vitamines n'ont pas de structures particulière distinguable.
  Les vitamines sont simplement des molécules essentielles au bon fonctionnement du corps humain.
  On va étudier un exemple de vitamine : la vitamine C.

  \problematique{
    Quelle sont les propriétés de la vitamine C ?
  }
\end{contexte}



%%%%%
\begin{doc}{La vitamine C : l'acide ascorbique}{doc:A8_vitamineC}
  \begin{center}
    \chemfig{
      HO -[-1] -[1] (-[3] OH) -[-1]
     *5(-(-OH) =(-OH) -(=O)-O-) % cycle
    }\\[4pt]
    
    \legende{Acide ascorbique, de formule brute \bruteCHO{6}{8}{6}}
  \end{center}
  La vitamine C est une molécule, l'acide ascorbique. 
  Elle remplit plusieurs fonctions dans l'organisme :
  \begin{listePoints}
    \item défense contre les infections virales ou bactériennes ;
    \item protection contre le vieillissement des cellules grâce à son action anti-oxydante ;
    \item protection de la paroi des vaisseaux sanguins ;
    \item cicatrisation ;
    \item etc.
  \end{listePoints}
\end{doc}

\question{
  Entourer et nommer les fonctions organiques dans la vitamine C
}{}{3}

\numeroQuestion

\mesure
