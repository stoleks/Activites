%%%%
\tetePremStssStru

%%%% titre
\numeroActivite{6}
\titreActivite{Les lipides}

\begin{objectifs}
  \item Étudier la structure des lipides.
  \item Comprendre la composition d'un triglycéride.
\end{objectifs}

\begin{contexte}
  Les lipides sont des corps gras et ne sont pas solubles dans l'eau.
  Ils entrent dans la constitution de la membranes de nos cellules et forment aussi une importante réserves d'énergie dans notre organisme.

  \problematique{
    Quelle est la structure des lipides ?
  }
\end{contexte}



%%%%%
\begin{doc}{Structure d'un lipide}{doc:A6_structure_lipide}
  Un lipide est une molécule constitué de de deux parties :
  \begin{listePoints}
    \item une tête hydrophile, qui peut se mélanger avec de l'eau ;
    \item une queue hydrophobe, qui ne peut pas se mélanger avec de l'eau.
  \end{listePoints}
  On va voir étudier deux types de lipides : les acides gras et les triglycérides.
\end{doc}

\begin{doc}{Les acides gras}{doc:A6_acide_gras}
  \begin{encart}
    Les \important{acides gras} sont des \important{acides carboxyliques} qui possèdent une longue chaîne carbonée sans ramification.
    Les acides gras peuvent être 
    \begin{listePoints}
      \item \important{saturés} (en hydrogène) si la chaîne carbonée ne comporte que des liaisons carbone-carbone simples ;
      \item \important{insaturés} si la chaîne carbonée comporte au moins une liaison carbone-carbone double.
    \end{listePoints}
  \end{encart}
  
  \begin{center}
    \chemname{
      \chemfig[atom sep = 1.75em]{OH-[1]!\palmitique}
    }{
      Acide palmitique, un acide gras saturé.
    }

    \chemname{
      \chemfig[atom sep = 1.75em]{H!\oleique}
    }{
      Acide oléique, un acide gras insaturé.
    }
  \end{center}

  Dans le cadre d'une alimentation saine, il faut limiter les acides gras saturés et privilégier les lipides riches en acides gras insaturés.
\end{doc}

\begin{doc}{Les triglycérides}{doc:A6_triglycerides}
  \vspace*{-18pt}
  \begin{wrapfigure}[2]{r}{0.3\linewidth}
    \centering
    \chemname{
      \chemfig{CH_2 (-[3]OH) -CH (-[3]OH) -CH_2(-[3]OH)}
    }{
      Glycérol
    }
  \end{wrapfigure}
  \vphantom{b}
  \begin{encart}
    Les \important{triglycérides} sont des \important{triester} composés d'un \important{glycérol} et de trois \important{acides gras} (pas forcément trois fois le même).
  \end{encart}

  \begin{center}
    \chemfig[atom sep=1.25em]{[:-60]!\tripalmitine} 
    \qq{} ou \qq{}
    \chemfig{
      H C (!\teteAcideDev C_{15} H_{31}) 
      (-[3,1.7,2,2] H_2C (!\teteAcideDev C_{15} H_{31}))
      -[-3,1.7,2,2] H_2 C (!\teteAcideDev C_{15} H_{31})
    } \\[8pt]
    
    Tripalmitine, triglycéride composé d'un glycérol et de trois acide palmitique
  \end{center}
\end{doc}

\begin{doc}{Micelles et membrane cellulaire}{doc:A6_micelle_membrane}
  La structure particulière des molécules de lipides mène à la formation de structure particulière dans de l'eau liquide.
  Les queue hydrophobe étant repoussée par les molécules d'eau, elles vont s'agglomérer et former des structures ou les queues sont isolés de l'eau environnante : \important{les micelles.}
  
  Des exemples de micelles sont \important{les couches bi-lipidique,} composée de deux couches de lipides avec les têtes hydrophile orientée vers l'extérieur, ce qui permet à leur queue hydrophobes de ne pas rentrer en contact avec de l'eau. 
  Les interactions électrostatiques entre les différentes parties de la membrane la pousse à former une sphère (comme une bulle de savon), avec un extérieur et un intérieur : c'est la base \important{d'une membrane cellulaire.}

  Les membranes cellulaire sont plus complexe qu'une simple couche bi-lipidique : elles sont aussi composées de \important{protéines}, qui permettent de renforcer la structure de la membrane cellulaire et de contrôler ce qui sort et ce qui entre de la cellule.

  TODO : FIGURE MICELLE ET MEMBRANE
\end{doc}

\question{
}{}{2}

\numeroQuestion

\mesure
