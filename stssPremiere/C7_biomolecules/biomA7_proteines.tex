%%%%
\tetePremStssStru

%%%% titre
\numeroActivite{6}
\titreActivite{Les protéines}

\begin{objectifs}
  \item Étudier la structure des protéines.
  \item Comprendre que les protéines sont composés d'une ou de plusieurs chaînes d'acides aminés.
\end{objectifs}

\begin{contexte}
  Les protéines sont des molécules complexes qui permettent à nos organismes de fonctionner en remplissant en ensemble varié de rôles en son sein.

  \problematique{
    Quelle est la structure des protéines ?
  }
\end{contexte}



%%%%%
\begin{doc}{Acides aminés}{doc:A7_acides_amines}
  \begin{encart}
    Les \important{acides aminés} sont des molécules composées d'une fonction \important{acide carboxylique} et d'une fonction \important{amine} (d'où leur nom).
  \end{encart}
  Sur Terre, les organismes vivants synthétisent et utilisent 20 acides aminés.
  Parmis ces 20 acides aminés, 8 ne sont pas synthétisés par le corps humain, on dit que ce sont des \important{acides aminés essentiels,} qui doivent être apportés par une alimentation équilibrée.
\end{doc}

\begin{doc}{Les protéines}{doc:A7_proteines}
  Les acides aminés peuvent se lier au cours d'une réaction chimique en formant une \important{liaison peptidique}.
  La chaîne d'acides aminés ainsi formée est appelée \important{polypeptide}.

  \begin{encart}
    Une \important{protéine} est un polypeptide qui s'est replié sur lui même.
    Ce repliement lui donne une structure tridimensionnelle, qui lui confère une \important{fonction biologique} particulière.
  \end{encart}
  SCHEMA AMINE -> POLYPEPTIDE -> PROTEINE
  
  EXEMPLE DE PROTEINES : hémoglobine, 
\end{doc}


\question{
}{}{2}

\numeroQuestion

\mesure
