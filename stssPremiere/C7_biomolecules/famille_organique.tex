%%%%
\tetePremStssStru

%%%% titre
\numeroActivite{3}
\vspace*{-34pt}
\titreActivite{Les fonctions organiques}


%%%% objectifs
\begin{objectifs}
  \item Connaître les 7 groupes caractéristiques et les 8 familles fonctionnelles associées.
\end{objectifs}

%%
\begin{doc}{Fonctions organiques}{doc:OA2_fonction_organique}

  Certaines séquences d'éléments donnent des \important{propriétés} spécifiques aux molécules organiques que l’on classe en différentes \important{familles.}
  $R_1,$ $R_2$ et $R_3$ sont des chaînes carbonées appelées \og radicaux alkyles \fg.\\[4pt]
  
  \begin{tblr}{
    width = \linewidth,
    colspec = {|c |c |c |X |}, hlines,
    column{2} = {couleurPrim!20},
    row{1} = {couleurPrim!10},
    cell{3}{1} = {r=2}{c},
    rows = {m}, columns = {c}
  }
    Groupe caractéristique & Famille fonctionnelle & Formule & Exemple \\
    %
    Hydroxyle & Alcool
    & \chemfig{R_1 - OH} 
    & \chemname[-2pt]{\chemfig{-[1] -[-1] OH}}{éthanol} \\
    %
    Carbonyle & Cétone
    & \chemfig{R_1-[1] C (=[3] O) -[-1]R_2}
    & \chemname[-2pt]{\chemfig{-[1] !\carbonyle -[1]}}{butan-2-one} \\    
    %
    & Aldéhyde
    & \chemfig{R_1-[1] C (=[3] O) -[-1]H}
    & \chemname[-2pt]{\chemfig{(- H) =[3] O }}{méthanal ou formaldéhyde} \\
    %
    Carboxyle & Acide carboxylique
    & \chemfig{R_1-[1] C (=[3] O) -[-1]OH}
    & \chemname[1pt]{\chemfig{-[-1] -[1] !\carboxyle}}{acide propanoïque} \\
    %
    Ester & Ester
    & \chemfig{R_1 -[1] C(=[3] O) -[-1]O -[1]R_2}
    & \chemname[1pt]{\chemfig{-[-1] -[1] !\ester -[1] -[-1]}}{propanoate d'éthyle} \\
    %
    Éther-oxyde & Éther
    & \chemfig{R_1 -[1] O -[-1] R_2}
    & \chemname[1pt]{\chemfig{-[-1] -[1] O -[-1] -[1]}}{éthoxyéthane} \\
    %
    Amine & Amine
    & \chemfig{R_1 - NH_2}
    & \chemname[-2pt]{\chemfig{-[1] -[-1] NH_2}}{methan-1-amine} \\
    %
    Amide & Amide
    & \chemfig{R_1-[1] C (=[3] O) -[-1]N (-[-3] R_3) - R_2}
    & \chemname{\chemfig{-[-1] -[1] !\amide H_2}}{propanamide}
  \end{tblr}
\end{doc}