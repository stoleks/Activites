%%%%
\tetePremStssRout

%%%% titre
\vspace*{-36pt}
\numeroActivite{1}
\titreActivite{Distance d'arrêt et accidents de la route}


%%%% objectifs
\begin{objectifs}
  \item Comprendre la différence entre distance de réaction, de freinage et d'arrêt.
  \item Connaître quelques facteurs augmentant les risques d'accidents.
\end{objectifs}

\begin{contexte}
  Un automobiliste dans sa voiture arrive avec une vitesse de \qty{40}{\km\per\hour} sur une route, quand soudain une enfant coure sur la route pour récupérer son ballon, \qty{25}{\m} devant la voiture.
  
  \problematique{
    La voiture va-t-elle s'arrêter à temps ?
  }
\end{contexte}


%%%% docs
\begin{doc}{Distance d'arrêt, de freinage et de réaction}{doc:A1_distances_arret}
  \begin{center}
    \image{1}{images/mecanique/distance_arret}
  \end{center}
  \begin{importants}
    \begin{listePoints}
      \item La \important{distance de réaction} est la distance que parcourt un véhicule entre le moment où la conductrice voit l'obstacle et quand elle commence à freiner.
      \item La \important{distance de freinage} est la distance que parcourt un véhicule entre le moment où la conductrice commence à freiner et quand le véhicule est à l'arrêt.
      \item La \important{distance d'arrêt} est simplement la somme de ces deux distances.
    \end{listePoints}
    
    Le \important{temps de réaction} $t_R$ d'un-e humain-e est de \qty{1,0}{\s} en moyenne.
    On peut calculer la distance de réaction $d_R$ à partir du temps de réaction $t_R$ et de la vitesse du véhicule $v$ :
    \begin{equation*}
      d_R = t_R \times v
    \end{equation*}
  \end{importants}
  
  \centering 
  \begin{tblr}{
    colspec = {|l |c |c |c |c |c |}, hlines, column{1} = {couleurPrim!20}
  }
    Vitesse $v$ (\unit{\km\per\hour})
    & 40 & 80 & 90 & 110 & 130 \\
    %
    Distance de réaction $d_R$ (\unit{\m})
    & 11,1 & & 25,0 & 30,6 & \\
    %
    Distance de freinage $d_F$ sur sol sec (\unit{\m})
    & 10,3 & & 52,0 & 78,1 & 108,5 \\
    %
    Distance d'arrêt $d_A$ sur sol sec (\unit{\m})
    & 21,4 & 63,4 & & 108,7 & 144,6 \\
    %
    Distance de freinage $d_F$ sur sol mouillé (\unit{\m})
    & 15,0 & 59,9 & 75,9 & & 158,4 \\
    %
    Distance d'arrêt $d_A$ sur sol mouillé (\unit{\m})
    & & & & 144,6 & \\
    %
  \end{tblr}
\end{doc}

\numeroQuestion
Compléter le tableau du document~\ref{doc:A1_distances_arret}

\question{
  Dans le tableau l'état du sol n'est pas indiqué pour la distance de réaction $d_R$, pourquoi ?
}{}{2}

\question{
  Indiquer si la voiture va s'arrêter avant de percuter l'enfant.
}{}{2}

\newpage
\vspace*{-36pt}
\question{
  Est-ce que la distance de freinage est doublée quand la vitesse du véhicule est doublée ?
}{}{2}

\mesure
À la main ou à l'aide d'un logiciel, tracer la distance de freinage en fonction de la vitesse au carré $v^2$, après avoir converti la vitesse en \unit{\m\per\s}.

\question{
  Est-ce que $v^2$ et $d_F$ sont deux grandeurs proportionnelles ?
}{}{2}


%%%%
\begin{contexte}
  La mère de l'enfant a été très choquée par la scène.
  Comme elle est journaliste, elle veut rédiger un article sur la sécurité routière et se pose la question suivante :
  
  \problematique{
    quels sont les facteurs de risques dans les accidents de la route ?
  }
\end{contexte}

%%
\begin{doc}{Les facteurs intervenant dans les accidents de la route}{doc:A1_facteurs_accident}
  \begin{multicols}{2}
    \begin{listePoints}  
      \item La vitesse est la première cause d'accident mortel.
      Réduire la vitesse moyenne de \qty{10}{\percent} réduit les morts de presque \qty{40}{\percent} !
      \item Un taux d'alcool entre \qty{0,5}{\g\per\litre} et \qty{0,8}{\g\per\litre} multiplie par 6 les risques d'accident mortel.
      \item La prise de stupéfiants multiplie par 3 les risques d'accident mortel.
      \item La somnolence au volant multiplie par 7 les risques d'accident.
      \item \important{Utiliser son téléphone} au volant multiplie par 5 les risques d'accidents.
    \end{listePoints}
  \end{multicols}
\end{doc}


\begin{doc}{L'évolution du nombre de morts à cause des accidents de la route}{doc:A1_mort_route}
  \centering
  \image{0.78}{images/donnees/evolution-nombre-de-morts-sur-les-routes}
\end{doc}

%%%%
\numeroQuestion
Rédiger un article synthétique de quelques lignes qui dresse la liste des facteurs de risques, en indiquant sur quelle distance ils ont un impact (freinage ou réaction) et si une législation les limite.
