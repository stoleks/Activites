%%%%
\tetePremStssRedo
\vspace*{-36pt}
\titreActivite{Les réaction d'oxydoréduction}


%%%% objectifs
\begin{objectifs}
  \item Savoir qu'un \important{oxydant} est une espèce qui \important{obtient} des électrons.
  \item Savoir qu'un \important{réducteur} est une espèce qui \important{relâche} des électrons.
  \item Apprendre la méthode pour écrire une réaction d'oxydoréduction.
\end{objectifs}

\begin{contexte}
  Un acide et une base forment un couple si l'on peut passer de l'un à l'autre par la perte ou le gain de proton \ionHydrogene.

  Pour les réaction d'oxydoréduction, il s'agit de couple oxydant/réducteur, reliés par la perte ou le gain d'électron.
  
  \problematique{
    Comment décrire une réaction d'oxydoréduction ?
  }
\end{contexte}


%%%% docs
\begin{doc}{Couple oxydant réducteur}
  \begin{importants}  
    Un \important{oxydant} est une espèce chimique capable d'\important{obtenir} un ou plusieurs \important{électrons.}

    Un \important{réducteur} est une espèce chimique capable de \important{relâcher} un ou plusieurs \important{électrons.}
  \end{importants}

  Un oxydant et un réducteur forment un couple oxydant/réducteur, si l'on peut passer de l'un à l'autre par le gain ou la perte d'électrons.
  Le couple est noté Ox/Réd. \exemple{\chemfig{Zn^{2+}}/\chemfig{Zn}}.

  À chaque couple oxydant/réducteur, on associe une demi-équation
  \begin{center}
    oxydant + $n$ \electron \reaction réducteur

    $n$ est le nombre d'électrons échangés, \electron représente un électron.
  \end{center}
  
  \begin{importants}
    \begin{listePoints}
      \item Oxydant + $n$ \electron \reaction Réducteur
      : il s'agit d'une \important{réduction.}
      L'oxydant est \important{réduit} (se transforme en réducteur).
      \item  Réducteur \reaction Oxydant + $n$ \electron
      : il s'agit d'une \important{oxydation.}
      Le réducteur est \important{oxydé} (se transforme en oxydant).
    \end{listePoints}
  \end{importants}
\end{doc}

%%
\begin{doc}{La réaction d'oxydoréduction}
  \begin{importants}
    Une réaction \important{d'oxydoréduction} a lieu quand on met en contact un oxydant et un réducteur de deux couples différents.
  \end{importants}
  
  Elle met donc en jeu deux couples oxydant/réducteur.
  Par exemple avec un couple du fer : \ionFerIII/\chemfig{Fe} ; et un couple de l'oxygène : \dioxygene/\ionOxygene

  Le gaz \dioxygene va réagir avec le solide \chemfig{Fe}, pour se transformer en ion \ionFerIII et en ion \ionOxygene, qui vont se combiner pour former de la rouille \chemfig{Fe_2O_3}.

  \begin{importants}
    Les électrons ne sont jamais libres.
    Il y a transfert d'électrons du réducteur vers l'oxydant.
  \end{importants}
\end{doc}

\question{
  Indiquer quel espèce chimique est l'oxydant et quel espèce chimique est le réducteur pour le couple associé au fer et pour le couple associé à l'oxygène.
}{
  L'oxydant est l'ion fer III \ionFerIII et le dioxygène \dioxygene, car ils sont à gauche dans le couple.
  Le réducteur est le fer \chemfig{Fe} et l'ion oxygène \ionOxygene.
}[1]

%%
\begin{doc}{L'arbre de Diane}
  \qrcodeCote[4]{https://edurl.fr/TZC3Cd5M}

  \moleculesGras
  On introduit dans un erlenmeyer une solution incolore de nitrate d'argent, qui est concentrée en \important{ions argent $\ionArgent\aq$.}
  On plonge ensuite un \important{fil de cuivre solide $\chemfig{Cu}\sol$.}

  Après quelques minutes, le morceau de cuivre s'est recouvert de \important{paillettes argentées $\chemfig{Ag}\sol$} et la solution \important{est devenue bleue.}
  Cette couleur bleue est liée à \important{la présence d'ion $\chemfig{Cu^{2+}}\aq$}
  \moleculesNormales

  Les demi-équations intervenant dans  cette réaction sont
  \begin{align*}
    \chemfig{Cu^{2+}}\aq + 2\electron &\reaction \chemfig{Cu}\sol \\
    \ionArgent\aq + \electron &\reaction \chemfig{Ag}\sol
  \end{align*}
\end{doc}

\question{
  À l'aide des demi-équations fournies, identifier les couples Oxydant/Réducteur qui interviennent dans la réaction de l'arbre de Diane.
}{
  On a un couple avec le cuivre \chemfig{Cu^{2+}}/\chemfig{Cu} et un avec l'argent \chemfig{Ag^+}/\chemfig{Ag}.
}[2]

\question{
  Identifier les réactifs et les produits de la réaction de l'arbre de Diane.
}{
  Dans l'arbre de Diane, le cuivre solide réagit avec les ions argent.
}[2]

%%
\begin{doc}{Méthode d'écriture d'une équation d'oxydoréduction}[\label{doc:methode_redox}]
  Pour écrire la réaction d'oxydoréduction entre les ions argent \ionArgent et le cuivre \chemfig{Cu}, il faut suivre la méthode suivante :
  \begin{enumeration}
    \item \important{Repérer} dans les deux couples quel oxydant réagit avec quel oxydant.
    \item \important{Écrire} les demi-équations de réaction pour chaque couple dans le « bon » sens, avec les réactifs à gauche et les produits à droite.
    \item \important{Ajuster} les deux demi-équations pour qu'il y ait le même nombre d'électrons échangés en rajoutant un coefficient multiplicateur devant les demi-équations si nécessaire. 
    \item \important{Additionner} les deux demi-équations afin d'obtenir l'équation d'oxydoréduction.
  \end{enumeration}
  \attention Il ne doit pas y avoir d'électrons dans l'équation finale !

  \exemple L'eau oxygénée peut réagir sur elle-même, car elle intervient dans deux couples 
  ($\chemfig{H_2O_2} + 2\ionHydrogene + 2\electron \reaction \eau$) et
  ($\dioxygene + 2\ionHydrogene + 2\electron \reaction \chemfig{H_2O_2}$) : 
  \begin{alignat*}{3}
     & \chemfig{H_2O_2} + \chemfig{2H^{+}} &&+ \chemfig{2e^{-}} &&\reaction \chemfig{2H_2O} \\
    +& &&\chemfig{H_2O_2}  &&\reaction \dioxygene + \chemfig{2H^{+}} + \chemfig{2e^{-}} \\
     & &&\chemfig{2H_2O_2} &&\reaction \dioxygene + \chemfig{2H_2O}
  \end{alignat*}
\end{doc}

\question{
  En suivant la procédure du document~\ref{doc:methode_redox}, écrire la réaction d'oxydoréduction qui modélise la transformation de l'arbre de Diane.
}{
  On commence par mettre les demi-équations dans le bon sens, avec les réactifs à gauche :
  \begin{align*}
    \chemfig{Cu^{2+}}\aq + 2\electron &\reaction \chemfig{Cu}\sol \\
    \chemfig{Ag}\sol &\reaction \ionArgent\aq + \electron
  \end{align*}
  Puis on ajuste le nombre d'électrons :
  \begin{align*}
    \chemfig{Cu^{2+}}\aq + 2\electron &\reaction \chemfig{Cu}\sol \\
    2\chemfig{Ag}\sol &\reaction 2\ionArgent\aq + 2\electron
  \end{align*}
  Et finalement on additionne les deux demi-équations :
  \begin{align*}
    \chemfig{Cu^{2+}}\aq + 2\chemfig{Ag}\sol + 2\electron &\reaction \chemfig{Cu}\sol 2\ionArgent\aq + 2\electron \\
    \chemfig{Cu^{2+}}\aq + 2\chemfig{Ag}\sol &\reaction \chemfig{Cu}\sol 2\ionArgent\aq
  \end{align*}
}[4]

