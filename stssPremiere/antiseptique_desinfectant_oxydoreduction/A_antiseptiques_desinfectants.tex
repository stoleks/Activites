%%%%
\tetePremStssRedo

%%%% titre
\vspace*{-36pt}
\numeroActivite{2}
\titreActivite{Antiseptiques et désinfectants}


%%%% objectifs
\begin{objectifs}
  \item Comprendre le principe de fonctionnement d'un antiseptique et d'un désinfectant
\end{objectifs}

\begin{contexte}
  Depuis des siècles les humain-es essayent de lutter contre les infections.
  D'abords grâce à des plantes médicinales, puis de nos jours grâce à des solution désinfectantes ou antiseptiques.
   
  \problematique{
    Quelle est la différence entre un antiseptique et un désinfectant ?
    Comment agissent-ils ?
  }
\end{contexte}


%%%% docs
\begin{doc}{Définition d'un antiseptique et d'un désinfectant}{doc:A2_definitions}
  \begin{importants}
    Un \important{antiseptique} est capable d'empêcher la prolifération ou de tuer des micro-organismes sur des \important{tissus vivant.}
  \end{importants}
  L'antiseptique doit être toléré par la peau ou les muqueuses et ne réduit que temporairement la quantité de micro-organismes.

  \begin{importants}
    Un \important{désinfectant} est capable de tuer et d'empêcher la prolifération des micro-organismes sur des \important{objets inertes.}
  \end{importants}
  
  Les désinfectants et les antiseptiques reposent sur des principes actifs qui agissent par \important{oxydation.}
\end{doc}

%%
\begin{doc}{Action oxydante sur les micro-organismes}{doc:A2_action_oxydante}
  \begin{importants}  
    Les antiseptiques et les désinfectant vont \important{oxyder les molécules responsable de la survie ou de la duplication des micro-organismes}, pour inhiber leur action ou les détruire.
  \end{importants}

  Précisément ils peuvent :
  \begin{listePoints}
    \item \important{détruire} ou \important{dénaturer} des \important{protéines membranaires} ;
    \item \important{modifier} des \important{enzymes} et empêcher leur action ;
    \item \important{dénaturer} des \important{acides nucléiques} composant son ADN ou ARN.
  \end{listePoints}
  Ces actions mènent à la mort ou à l'incapacité de se répliquer de la cellule.
  \begin{center}
    \image{0.5}{images/sante/bacterie}

    Schéma d'une bactérie
  \end{center}
\end{doc}

\question{
  Quelle est la différence entre un antiseptique et un désinfectant ?
}{

}{1}

\question{
  Quelle est la propriété chimique des principes actifs présent dans les antiseptiques ou les désinfectants ?
}{}{1}

\question{
  Détailler comment l'action oxydante des antiseptiques ou des désinfectants agit sur les différentes parties de la cellule d'un micro-organisme.
}{}{3}


%%%%
\begin{doc}{Un peu de vocabulaire}{doc:A2_vocabulaires}
  \begin{importants}
    Les antiseptiques ou les désinfectants sont \important{virucide, bactéricide, fongicide ou sporicide} s'ils peuvent être létal sur les virus, les bactéries, les champignons ou les spores.
  \end{importants}

  \begin{importants}
    Les antiseptiques ou les désinfectants sont \important{bactériostatiques} s'ils stoppent la prolifération des bactéries.
  \end{importants}
\end{doc}


%%%%
\begin{doc}{Un antiseptique et un désinfectant usuel}{doc:A2_exemples_antiseptiques}
  \centering
  \begin{tblr}{
    colspec = {c Q[c, wd = 0.35\linewidth] Q[c, wd = 0.35\linewidth] },
    hlines, vlines,
    row{1} = {couleurSec-100}
  }
    Principe actif & Diiode \diiode & Ion hypochlorite \hypochlorite \\
    %
    Produit commercial & Bétadine, Teinture d'iode & Eau de Javel, Dakin \\
    %
    Catégorie & Antiseptique & Désinfectant \\
    %
    Actions &
    Bactéricide, virucide, sporicide, fongicide &
    Bactéricide, virucide, sporicide, fongicide \\
    %
    Usages &
    Brûlures et plaies superficielles. Antiseptie du champ opératoire. &
    Désinfection des sols, surfaces, bassins urinaires, canalisation. Action blanchissante. \\
    %
    Couple Ox/Red &
    $\diiode\aq/\chemfig{I^{-}}\aq$ &
    $\hypochlorite\aq/\chlorure\aq$ \\
    %
    {Demi-équation \\ d'oxydoréduction} &
    $\diiode\aq + 2\electron \reaction 2\chemfig{I^{-}}\aq$ &
    $\hypochlorite\aq + 2\ionHydrogene\aq + 2\electron \reaction \chlorure\aq + \eau\liq$
  \end{tblr}
\end{doc}

%%
\question{
  La bétadine est l'antiseptique le plus utilisé en milieu hospitalier, expliquer pourquoi.
}{}{2}

\question{
  Expliquer pourquoi le diiode et l'ion hypochlorite sont des oxydants.
}{}{2}