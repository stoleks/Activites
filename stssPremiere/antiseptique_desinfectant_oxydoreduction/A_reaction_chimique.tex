\tetePremStssRedo
\vspace*{-40pt}
\titreActivite{Réaction chimique}

%%%% Objectifs
\begin{objectifs}
  \item Revoir comment une réaction chimique modélise une transformation macroscopique.
\end{objectifs}


%%%% docs
\begin{doc}{Observations macroscopiques}
  Pendant une transformation chimiques, des espèces chimiques interagissent, réarrangent leurs atomes, et forment d'autres espèces chimiques.
  Les espèces présentes initialement sont les \important{réactifs.} Celles présentes au final après la transformation sont les \important{produits.}
  
  Pour modéliser la transformation, il faut \important{identifier} les espèces chimiques qui réagissent et celles qui se forment.
  Pour ça, on observe ce qu'il se passe d'un point de vue macroscopique : formation d'un gaz ou d'un solide, disparition d'un solide, changement de couleur, etc.
  %Il est aussi possible d'utiliser des tests d'identification des espèces chimiques.
  
  \begin{importants}
    Les observations expérimentales macroscopiques permettent d'écrire l'équation de la \important{réaction} modélisant la transformation chimique microscopique, en identifiant les \important{réactifs} et les \important{produits.}
  \end{importants}
\end{doc}

\begin{doc}{Modélisation de la réaction}
  L'écriture de la réaction chimique permet de transcrire la transformation des réactifs en produit.
  
  \begin{importants}
    La réaction est symbolisée par une flèche. À gauche de la flèche se trouvent les \important{réactifs} qui se transforment et à droite de la flèche se trouvent les \important{produits} formés :
    \begin{center}
      réactif 1 + réactif 2 + \ldots  \reaction produit 1 + produit 2 + \ldots
    \end{center}
  \end{importants}
  
  Au cours d'une réaction chimique, rien ne se perd, rien ne se crée. \important{Il doit donc y avoir le même nombre d'atomes et de charges de chaque côté de la réaction}.
  Seuls les liaisons des molécules peuvent être modifiées pendant une réaction chimique.
\end{doc}

\begin{doc}{Notation des états physiques}
  Les réactifs et les produits peuvent se trouver dans différents états physiques.
  Pour indiquer dans quel état se trouve une espèce chimiques, on écrit en indice son état entre parenthèse à côté de sa formule chimique : (g) pour un gaz, (l) pour un liquide, (s) pour un solide et (aq) pour des solutés en solution aqueuse.
\end{doc}


\begin{doc}{Combustion du charbon}
  On modélise la combustion du charbon avec du dioxygène par la réaction chimique suivante :
  \begin{equation*}
    \chemfig{C}\sol + \dioxygene\gaz \reaction \dioxydeDeCarbone\gaz
  \end{equation*}
  On vérifie bien qu'il y a le même nombre d'atome de carbone et d'oxygène des deux côté de la réaction chimique.
\end{doc}


%%%% Questions
\question{
  Lister les réactifs et les produits pour la combustion du charbon en présence d'oxygène, en indiquant leurs état physique.
}{
  Réactifs : carbone solide et dioxygène gazeux.
  
  Produits : dioxyde de carbone gazeux.
}[1]


%%
\begin{doc}{Ajustage d'une réaction}
  Au cours d'une réaction chimique, les éléments chimiques présents dans les réactifs se réarrangent pour former des produits et les liaisons chimiques changent.
  \begin{importants}
    Il y a \important{conservation} 
    \begin{listePoints}
      \item \important{des éléments chimiques} ;
      \item \important{de la charge électrique} totale.
    \end{listePoints}
  \end{importants}
  \begin{importants}
    Pour assurer cette \important{conservation}, il faut \important{ajuster} la réaction chimique avec des coefficients devant les éléments chimiques.
    Ces coefficients sont appelés \important{coefficients stœchiométriques.}
  \end{importants}
\end{doc}

\question{
  Ajuster la réaction de combustion du méthane
  \begin{equation*}
    \methane\gaz + \dioxygene\gaz
    \reaction
    \dioxydeDeCarbone\gaz + \eau\gaz
  \end{equation*}
  à l'aide de coefficients stœchiométriques.
  Commencer par ajuster le nombre d'atomes d'hydrogène.
}{
  \begin{equation*}
    \methane\gaz + 2\dioxygene\gaz
    \reaction
    \dioxydeDeCarbone\gaz + 2\eau\gaz
  \end{equation*}
}[3]

\numeroQuestion
Ajuster les réactions chimiques suivantes en écrivant, si nécessaire, les coefficients stœchiométriques devant chaque élément chimique :
\newcommand{\localEcart}{16}
\begin{center}
  %
  \texteTrou{1} \chemfig{Fe}\sol + \texteTrou{2} \chemfig{H^{+}}\aq
  \reaction \texteTrou{1} \ionFerII\aq + \texteTrou{1} \chemfig{H_2}\gaz
  \\[\localEcart pt]
  %
  \texteTrou{4} \chemfig{Fe}\sol + \texteTrou{3} \dioxygene\gaz
  \reaction \texteTrou{2} \chemfig{Fe_2O_3}\sol
  \\[\localEcart pt]
  %
  \texteTrou{1} \chemfig{C_2H_6O}\liq + \texteTrou{3} \dioxygene\gaz
  \reaction \texteTrou{2} \dioxydeDeCarbone\gaz + \texteTrou{3} \eau\liq
  \\[\localEcart pt]
  %
  \texteTrou{1} \chemfig{Cu^{2+}}\aq + \texteTrou{2} \chemfig{HO^{-}}\aq
  \reaction \texteTrou{1} \chemfig{Cu{(HO)}_2}\sol
  \\[\localEcart pt]
  %
  \texteTrou{2} \chemfig{Fe}\sol + \texteTrou{2} \eau\liq + \texteTrou{1} \dioxygene\gaz
  \reaction \texteTrou{2} \chemfig{Fe{(HO)}_2}\sol
  \\[\localEcart pt]
  %
  \texteTrou{2} \chemfig{Fe{(OH)}_3}\sol
  \reaction \texteTrou{1} \chemfig{Fe_2O_3}\sol + \texteTrou{3} \eau\liq
  \\[\localEcart pt]
  %
  \texteTrou{1} \chemfig{Fe{(OH)}_2}\sol + \texteTrou{2} \eau\liq + \texteTrou{1} \dioxygene\gaz
  \reaction \texteTrou{2} \chemfig{Fe{(OH)}_3}\sol
  \\[\localEcart pt]
  %
\end{center}

%%
\begin{wrapfigure}[1]{r}{0.1\linewidth}
  \centering
  \qrcode{https://phet.colorado.edu/sims/html/balancing-chemical-equations/latest/balancing-chemical-equations_fr.html}
\end{wrapfigure}
\numeroQuestion Pour s'entraîner :