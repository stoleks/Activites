%%%%
\tetePremStssRout

%%%% titre
\numeroActivite{2}
\titreActivite{Voiture et vélo : danger sur la ville ?}


%%%% objectifs
\begin{objectifs}
  \item Savoir calculer une énergie cinétique à partir de sa relation $E_c = \dfrac{1}{2} m v^2$.
  \item Comprendre comment se transfère l'énergie pendant une collision.
\end{objectifs}

\begin{contexte}
  Une cycliste et une automobiliste roule toutes les deux à \qty{30}{\km\per\hour} en ville, en grillant de manière répétée des feux rouges.
  
  \problematique{
    Les deux véhicules représentent-ils le même danger pour les piéton-nes et les autres véhicules ?
  }
\end{contexte}


%%%% docs
\begin{doc}{L'énergie cinétique}{doc:A2_energie_cinetique}
  \begin{encart}  
    Quand un objet de masse $m$ se déplace avec une vitesse $v$ en \unit{\m\per\s}, cet objet possède une \important{énergie cinétique notée $E_c$}
    \begin{equation*}
      E_c = \dfrac{1}{2} m v^2 \qq{en joule noté \unit{\joule}} 
    \end{equation*}
  \end{encart}
  \important{L'énergie se conserve toujours}, mais elle peut changer de forme.
\end{doc}

\begin{doc}{Énergie, freinage et collision}{doc:A2_energie_collision}
  \begin{encart}
    Quand un véhicule freine, \important{l'énergie cinétique} est convertie en \important{énergie thermique} par les frottements : la température des pneus et du sol augmente.
  
    Pendant une collision, l'énergie cinétique est convertie en \important{énergie de déformation.}
  \end{encart}
  %
  \begin{center}
    \image{0.9}{images/mecanique/energie_collision}
  \end{center}
  
  \important{L'énergie de déformation} est responsable de la déformation
  \begin{listePoints}
    \item du véhicule et de l'obstacle éventuel ;
    \item des personnes dans et en dehors du véhicule.
  \end{listePoints}
  Pour les personnes les « déformations » sont dramatiques : on parle de blessures, fractures, hémorragies, etc.

  \attention Ici on fait une modélisation simplifiée : en réalité l'énergie de déformation est liée à des phénomènes complexes à l'échelle microscopique (destruction des cellules, ruptures des liaisons moléculaires, etc.).
\end{doc}

\begin{doc}{Quelques données}{doc:A2_voiture_velo}
  En France 
  \begin{itemize}
      \item la voiture moyenne a une masse de \qty{1 250}{\kg} en 2023 ;
      \item la masse moyenne d'une femme est de \qty{67,3}{\kg} en 2020 ;
      \item la masse moyenne d'un homme est de \qty{81,2}{\kg} en 2020 ;
      \item les vélos vendu ont en moyenne une masse de \qty{12,0}{\kg}.
  \end{itemize}
\end{doc}

\question{
  Expliquer pourquoi le freinage permet de réduire l'énergie de déformation pendant une collision.
}{}{3}

\question{
  Convertir \qty{30}{\km\per\hour} en \unit{\m\per\s}.
}{}{2}

\question{
  Calculer l'énergie cinétique d'une cycliste roulant à une vitesse de \qty{30}{\km\per\hour}.
}{}{2}

\question{
  Calculer l'énergie cinétique d'un automobiliste roulant à une vitesse de \qty{30}{\km\per\hour}.
}{}{2}

\question{
  Entre la cycliste et l'automobiliste, qui représente le plus grand danger a priori ?
}{}{2}
*

\begin{doc}{Égalité des énergies cinétiques}{doc:A2_egalite_cinetique}
  En négligeant la forme d'une voiture et d'un vélo, pour que la cycliste soit aussi dangereuse que l'automobiliste, il faudrait que les deux aient la même énergie cinétiques.
  
  Pour ça la cycliste devrait avoir une vitesse :
  \begin{equation*}
    v_\text{vélo} = \sqrt{\dfrac{m_\text{auto}}{m_\text{vélo}}} v_\text{auto}
  \end{equation*}
\end{doc}

\question{
  Calculer la vitesse que devrait avoir une cycliste pour avoir la même énergie cinétique que l'automobiliste qui roule à \qty{30}{\km\per\hour}.
  Commenter.
}{}{3}