%%%%
\tetePremStssStru

%%%% titre
\numeroActivite{5}
\titreActivite{Les glucides}

\begin{objectifs}
  \item Étudier la structure des glucides.
  \item Comprendre la différence entre un sucre rapide et un sucre lent.
\end{objectifs}

\begin{contexte}
  Les glucides sont une part essentielle de notre alimentation.
  Tous les glucides que nous ingérons sont transformés en glucose au cours de la digestion,
  qui peut ensuite être utilisé par nos cellules.

  \problematique{
    Quelle est la structure des glucides ?
  }
\end{contexte}



%%%%%
\begin{doc}{Sucres rapides et sucres lents}{doc:A5_sucre_rapide_lent}
  \begin{importants}
    On classe les glucides en deux catégories :
    \begin{listePoints}
      \item les \important{sucres rapides,} qui sont des molécules simples et facilement digérés ;
      \item les \important{sucres lents,} qui sont composés de plusieurs sucres rapides liés entre eux.  
    \end{listePoints}
    L'assimilation des sucres lents par l'organisme est lente et permet un apport régulier en sucre rapide pendant toute la digestion.
  \end{importants}

  On trouve des sucres rapides dans les fruits, le miel, la farine blanche, le riz blanc et la plupart des sodas et sucreries.
  Les sucres lents se trouvent dans les féculents (pomme de terre, maïs, blé, etc.), les légumineuses (haricot rouge, pois chiche, etc.), la farine complète ou le riz complet.
\end{doc}

\begin{doc}{Le glucose et le fructose sous forme linéaire et cyclique}{doc:A5_glucose_fructose}
  \begin{multicols}{2}
  \begin{center}
    %% glucose linéaire
    \chemfig{
      HO -C (-H)
        (-[3] C (-H) (-[6] HO)
          (-[3] C (-[5]H) =[1] O)
        ) % aldehyde et alcool
      -[-3] C (-H) (-[6] HO)
      -[-3] C (-H) (-[6] HO)
      -[-3] C (-H) (-[6] HO)
      -[-3] H
    }
    \phantom{b}
    % glucose hamworth
    \chemfig[cram width=2pt, atom sep=2.5em]{!\glucoseHamw} \\[4pt]
    Glucose

    %% fructose linéaire
    \chemfig{
      HO -C (-H)
        (-[3] C (=O)
          (-[3] C (-[3]H) (-[6]H) -OH)
        ) % cétone et alcool
      -[-3] C (-H) (-[6] HO)
      -[-3] C (-H) (-[6] HO)
      -[-3] C (-H) (-[6] HO)
      -[-3] H
    }
    \phantom{b}
    % fructose hamworth
    \chemfig[cram width=2pt, atom sep=2.5em]{!\fructoseHamw} \\[4pt]
    Fructose
  \end{center}
  \end{multicols}
\end{doc}

\question{
  À partir de la forme linéaire du glucose et du fructose, entourer et nommer les fonctions organiques présentes dans ces deux molécules.
}{}{4}

\newpage
\question{
  Donner la formule brute du glucose et la formule brute du fructose.
  Ces molécules sont-elles isomères ?
}{}{2}


\begin{doc}{Une partie de l'amidon}{doc:A5_amidon}
  \begin{center}
    \chemfig[cram width=2pt, atom sep=2.5em]{
      ...\phantom{B}-[-1] !\amylopectineHamw
      !\amylopectineHamw O -[-3]
      !\amylopectineCentraleHamw
      !\amylopectineHamw !\amylopectineHamw O -[1]...
    } \\[8pt]
  
    \important{Amylopectine,} molécule composant l'amidon
  \end{center}
\end{doc}

\begin{doc}{Le test de Fehling}{doc:A5_test_fehling}
  Le test à la liqueur de Fehling permet de déterminer si une solution contient des fonctions aldéhydes.

  \begin{protocole}
    \item Mélanger \qty{0,5}{\ml} de liqueur de Fehling avec \qty{1}{\ml} de solution à tester dans un tube à essais.
    \item Chauffer le tube à essais aec le mélange quelques minutes dans un bain-marie.
  \end{protocole}

  \begin{center}

    \begin{tblr}{
      colspec = {X[l] X[l] X[l]}, hlines, vlines,
      column{1} = {couleurPrim!10},
      row{1} = {couleurPrim!20}
    }
      &
      Présence d'une fonction aldéhyde &
      Présence d'une fonction cétone \\
      %
      Observation dans le tube à essais &
      Apparition d'un précipité rouge brique &
      La solution reste bleue. \\
    \end{tblr}
  \end{center}
\end{doc}

\question{
  Indiquer, en justifiant, la catégorie de glucides dans laquelle se trouvent le glucose, le fructose et l'amidon.
}{}{3}

\mesure
Mettre en oeuvre un protocole expérimental permettant de distinguer des solutions de glucose et de fructose.