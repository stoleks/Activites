%%%%
\tetePremStssElec

%%%% titre
\vspace*{-36pt}
\numeroActivite{1}
\titreActivite{Risques électriques}


%%%% objectifs
\begin{objectifs}
  \item Revoir l'intensité du courant et la tension électrique
  \item Connaître les dangers liés à une électrisation
\end{objectifs}

\begin{contexte}
  Anna projette de réaliser quelques travaux sur les prises électriques dans sa maison.
  Elle réalise préalablement quelques tests. 
  Elle se munit d'un multimètre pour mesurer la tension électrique aux bornes de la prise du secteur, elle utilise son tournevis testeur pour identifier la phase.
  Puis elle descend à la cave pour vérifier l'état de l'installation de la prise de terre.
  
  \problematique{
    Comment éviter tout risque d'électrisation ?
  }
\end{contexte}


%%%% docs
\begin{doc}{La prise de terre}{doc:A1_prise_terre}
  \begin{wrapfigure}{r}{0.1\linewidth}
    \centering
    \vspace*{-30pt}
    \qrcode{https://www.youtube.com/watch?v=BazTgHMuA8k}
  \end{wrapfigure}
  Pour la plupart des personnes, la prise de terre n'est rien de plus qu'un élément mentionné sur le schéma électrique de la maison.
  Et pourtant, cet élément à première vue banal a une importance vitale, car la prise de terre assure que le courant électrique s'évacue dans le sol lorsqu'un appareil est mal isolé.

  \begin{wrapfigure}{l}{0.2\linewidth}
    \images{1}{images/photos/prise_terre}
  \end{wrapfigure}

  Le principe de la prise de terre est simple : il s'agit de dévier le courant électrique qui s'échapperait d'un appareil sur un conducteur métallique (câble) qui finit sur un piquet (tige) enfoui dans le sol.

  Ces pertes de courant peuvent survenir par exemple lorsqu'un câble d'alimentation abîmé est dénudé et que les fils électriques entrent en contact avec l'armature de l'appareil.
  Sans fil de terre, le courant traverserait le corps de la première personne qui toucherait l'appareil !
\end{doc}

%%
\begin{doc}{Tension et intensité}{doc:A1_}
  \begin{encart}
    Un courant électrique est caractérisé par deux grandeurs : 
    \begin{listePoints}
      \item la tension électrique, exprimée en Volts \unit{\volt}.
      \item l'intensité du courant, exprimée en Ampères \unit{\ampere}.
    \end{listePoints}
  \end{encart}
  
  La tension détermine la quantité de chaleur libérée par le courant.
  En pratique, plus la tension est élevée, plus le risque de brûlure est grand.
  Ce sont « les volts qui brûlent ». [...]
  
  L'intensité d'un courant traversant un corps est responsable de la contraction musculaire et des ruptures de fibres nerveuses, appelée « sidération ».
  Lorsque l'intensité augmente, on définit des seuils successifs à partir desquels apparaissent les différentes réactions au courant électrique, allant jusqu'à la mort par arrêt cardiaque.
  Par opposition aux « volts qui brûlent », c'est « l'intensité qui tue ». [...]
\end{doc}

\begin{doc}{Risques}{doc:A1_}
  Dans les accidents d’origine électrique touchant les personnes, il faut distinguer l’électrisation de l’électrocution :
  \begin{listePoints}      
    \item l’électrisation : c’est la réaction du corps due à un contact accidentel avec l’électricité ;
    \item l’électrocution : c’est l’électrisation qui débouche sur une issue fatale.
  \end{listePoints}
  
  Les autres risques lors de ces accidents peuvent être :
  \begin{listePoints}
    \item les brûlures de contact et internes ;
    \item les brûlures thermiques (arcs électriques, projections...) ;
    \item l’électricité peut être aussi à l’origine d’incendie ou d’explosion.
  \end{listePoints}
\end{doc}

\begin{doc}{Résistance du corps humain}{doc:A1_}
  Lorsqu’il est soumis à une tension électrique, le corps humain conduit le courant électrique.
  \begin{tableau}{|c |c |c |c |c |}
    État de la peau & Peau sèche & Peau humide & Peau mouillée & Peau immergée \\
    Résistance $R$ & \qty{5000}{\ohm} & \qty{2500}{\ohm} & \qty{1000}{\ohm} & \qty{500}{\ohm}
  \end{tableau}
\end{doc}

\begin{doc}{Effet physiologique observée}{doc:A1_}
  \begin{tblr}{
    hlines,
    colspec = {|l |X[l] |},
    row{1} = {couleurPrim!20},
    row{2} = {green!50},
    row{3} = {red!50},
    row{4} = {purple!50},
    row{5} = {black!40},
  }
    Zone & Principaux effets physiologiques constatés \\
    AC-1 & aucune réaction \\
    AC-2 & sensations désagréables mais pas d’effets physiologiques dangereux \\
    AC-3 & tétanisation musculaire avec risque de paralysie respiratoire mais sans fibrillation ventriculaire \\
    AC-4 & fibrillation ventriculaire, possibilités d’arrêt respiratoire, d’arrêt cardiaque, de brûlures graves, etc.
  \end{tblr}
\end{doc}

%%%%
\question{
}{
}{1}

\numeroQuestion
