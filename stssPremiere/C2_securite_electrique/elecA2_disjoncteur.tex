%%%%
\tetePremStssElec

%%%% titre
\vspace*{-36pt}
\numeroActivite{2}
\titreActivite{Les risques liés au matériel}

\begin{objectifs}
  \item Relier l'intensité du courant électrique à la détérioration d'appareils électriques.
  \item Décrire le principe d'un disjoncteur.
\end{objectifs}


%%%%
\titreSection{Le rôle du disjoncteur}

\begin{contexte}  
  Myriam veut rajouter des prises électriques chez elle et donc ajouter sur son tableau électrique un nouveau disjoncteur divisionnaire.
  Avant de démarrer les travaux, Myriam vérifie son disjoncteur différentiel.

  \problematique{Comment peut-elle s'assurer de réaliser des travaux en toute sécurité ?}
\end{contexte}

\begin{doc}{Installation électrique}{doc:A2_installation_electrique}
  \begin{center}
    \image{1}{images/electricite/schema_electrique_maison}
  \end{center}
\end{doc}

\begin{doc}{Le rôle d'un disjoncteur}{doc:A2_disjoncteur}
  \begin{wrapfigure}[8]{r}{0.2\linewidth}
    \vspace*{-24pt}
    \centering
    \image{1}{images/electricite/disjoncteur_differentiel}
  \end{wrapfigure}
  Le disjoncteur sert à protéger une installation électrique et s'installe dans le tableau électrique.
  Un disjoncteur protège contre deux types de défauts :
  \begin{listePoints}
    \item protection contre les surcharges : il s'agit d'une surconsommation du récepteur branché sous le disjoncteur.
    \item protection contre les courts circuits : c'est un défaut correspondant à un contact direct entre phase et neutre.
  \end{listePoints}
  Ces défaut peuvent entraîner une destruction localisée (un appareil électrique) ou à des dégâts importants (incendie d'origine électrique).
  
  Le disjoncteur sert à isoler électriquement la partie ou se situe le défaut.
  Pour ça, le disjoncteur sectionne le passage du courant, comme un interrupteur.
\end{doc}

\begin{doc}{Court-circuit}{doc:A2_court_circuit}
  \begin{wrapfigure}{r}{0.1\linewidth}
    \vspace*{-32pt}
    \qrcode{https://www.youtube.com/watch?v=UCTwDyTFyns}
  \end{wrapfigure}
  Pour s'informer sur les risques d'un court-circuit, Myriam regarde une vidéo dans laquelle un court-circuit est réalisé avec un brin de paille de fer.
  Dans la vidéo, la résistance électrique d'un brin de paille de fer vaut \qty{0,03}{\ohm}.
\end{doc}

%
\question{
  Expliquer pourquoi Myriam va vérifier son disjoncteur avant de démarrer ses travaux.
}{}{2}

\question{
  De quels défauts électrique protège un disjoncteur ? Comment le disjoncteur agit-il ?
}{}{3}

\question{
  En appliquant la loi d'Ohm, calculer la valeur de l'intensité du courant électrique $I$ dans un brin de paille de fer, en sachant que la tension électrique vaut $U = \qty{4,5}{\volt}$. Conclure.
}{}{3}

\question{
  Du point de vue électrique, trouver le point commun entre cette vidéo, un court-circuit électrique et une installation en surcharge.
  Quels sont les risques dans ces deux cas ?
}{}{3}

\question{
  Si Myriam constatait un départ d'incendie d'origine électrique, quelle(s) action(s) lui conseilleriez-vous de faire, et dans quel ordre ?
}{}{4}


%%%%
\titreSection{Les caractéristique électriques des appareils électriques}

\begin{contexte}
  Durant les travaux, Myriam a dû déplacer des appareils électriques afin de les connecter sur la même prise murale, grâce à une multiprise.
  
  \problematique{Les appareils peuvent-ils fonctionner normalement et sans danger ?}
\end{contexte}

\begin{doc}{Caractéristique de la multiprise}{doc:A2_multiprise}
  \begin{wrapfigure}{r}{0.2\linewidth}
    \vspace*{-24pt}
    \centering
    \image{1}{images/electricite/multiprise}
  \end{wrapfigure}

  \phantom{b}\vspace*{-16pt}
  \begin{listeTirets}
    \item Nombre de prise : 3 prises.
    \item Puissance maximale : \qty{3500}{\watt}.
    \item Intensité du courant maximale : \qty{16}{\ampere}.
    \item Tension électrique : \qty{230}{\volt}
  \end{listeTirets}
\end{doc}

\begin{doc}{Appareils connectés à la multiprise}{doc:A2_appareils_multiprise}
  Les appareils électriques qu'elle a reliés sur la même prise sont présentés dans ce tableau :
  
  \begin{tableau}{|c |c |c |c |}    
    Appareil & Intensité nominale & Puissance nominale \\
    Sèche-linge & \qty{13}{\ampere} & \qty{3000}{\watt} \\
    Lave-linge  & \qty{13}{\ampere} & \qty{3000}{\watt} \\
    Four à micro-ondes &  \qty{7}{\ampere} & \qty{1650}{\watt}
  \end{tableau}
\end{doc}

\begin{doc}{Lampe thérapeutique}{doc:A2_lampe_therapeutique}
  \begin{center}
    \centering
    \image{0.5}{images/electricite/lampe_therapeutique_IR}
  \end{center}
  Suite à ses travaux, Myriam commande une ampoule infrarouge thérapeutique afin de soigner ses douleurs récurrentes au dos.
  
  Quelques jours plus tard, elle reçoit son colis, mais ne lit pas les caractéristiques de la lampe.
\end{doc}

\question{
  À l'aide des documents~\ref{doc:A2_multiprise} et~\ref{doc:A2_appareils_multiprise}, expliquer si Myriam peut connecter sur la même multiprise les trois appareils et les faire fonctionner simultanément sans risques ?
}{}{3}

\question{
  En vous aidant des informations données dans le document~\ref{doc:A2_lampe_therapeutique} et en admettant que l'intensité du courant circulant dans la lampe ne dépend pas de la tension d'alimentation, expliquer la conséquence d’utiliser la lampe infrarouge en France.
}{}{3}