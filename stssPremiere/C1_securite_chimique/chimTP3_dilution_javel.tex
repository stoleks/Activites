%%%%
\tetePremStssChim

%%%% titre
\vspace*{-36pt}
\numeroActivite{3}
\titreTP{Dilution d’un produit désinfectant}

% \begin{tableauCompetences}
%   APP & Rechercher et utiliser des informations dans un document. & & & & \\
%   REA & Réaliser des calculs. Réaliser un protocole en respectant les consignes de sécurités. & & & & \\
% \end{tableauCompetences}
\begin{objectifs}
  \item Connaître les pictogrammes de sécurité
  \item Savoir réaliser une dilution et calculer un facteur de dilution
\end{objectifs}


%%%% Documents 
\begin{doc}{Les pictogrammes de sécurités}{doc:TP3_picto_secu}
  Les pictogrammes de sécurités sont à connaître par c\oe{}ur ! \\[4pt]

  %% Tableau avec les pictogrammes
  \NewDocumentCommand{\pictoTableau}{m O{65}}{
    \hspace{16pt} \image{0.45}{images/securite/picto_#1} \vAligne{-#2pt}
  }
  \begin{tblr}{
    colspec = {|X[-1, h, m] |X[4, m] |}, hlines,
    row{1} = {couleurPrim!20, c}
  }
    \textbf{Pictogramme} & \textbf{Signification} \\
    %
    \pictoTableau{ronge} &
    {\texteTrou{Corrosif.} \\
    Je peux attaquer ou détruire les métaux.
    Je ronge la peau et/ou les yeux en cas de contact.} \\
    %
    \pictoTableau{toxique} &
    {\texteTrou{Toxique, irritant, narcotique.} \\
    J'empoisonne à forte dose.
    J'irrite la peau, les yeux et/ou les voies respiratoires.
    Je peux provoquer des allergies, de la somnolence ou des vertiges.} \\
    %
    \pictoTableau{tue}[55] &
    {\texteTrou{Toxique.} \\
    J’empoisonne rapidement, même à faible dose.} \\
    %
    \pictoTableau{explose} &
    {\texteTrou{Explosif.} \\
    Je peux exploser au contact d’une flamme, d’une étincelle, d’électricité statique, sous l’effet de la chaleur, de frottements ou d’un choc.} \\
    %
    \pictoTableau{flambe} &
    {\texteTrou{Inflammable.} \\
    Je peux m’enflammer au contact d’une flamme, d’une étincelle, d’électricité statique, sous l’effet de la chaleur, de frottements ou au contact de l’air ou de l’eau.} \\
    %
    \pictoTableau{flamber} &
    {\texteTrou{Comburant.} \\
    Je peux provoquer ou aggraver un incendie ou même provoquer une explosion en présence de produits inflammables.} \\
    %
    \pictoTableau{pression} &
    {\texteTrou{Gaz sous pression.} \\
    Je peux exploser sous l’effet de la chaleur.
    Je peux causer des brûlures ou blessures liées au froid.} \\
    %
    \pictoTableau{pollue} &
    {\texteTrou{Dangereux pour l'environnement.} \\
    Je provoque des effets néfastes sur les organismes du milieu aquatique, sur les êtres vivants.} \\
    %
    \pictoTableau{sante} &
    {\texteTrou{Mutagène, cancerogène, reprotoxique.} \\
    Je peux provoquer le cancer, modifier l’ADN, nuire à la fertilité ou au f\oe{}tus, altérer le fonctionnement des organes.
    Je peux être mortel en cas d’ingestion dans les voies respiratoires.}
    %
  \end{tblr}
\end{doc}


%%%% Activité
\begin{contexte}
  L'eau de Javel est un produit ménager très couramment utilisé pour désinfecter les salles
de bain ou cuisines.
  On trouve dans le commerce des solutions prêtes à l’emploi mais d’autres
doivent être diluées pour une bonne utilisation.

  \problematique{Comment réaliser ces solutions ?}
\end{contexte}

\begin{doc}{Étiquette d'une eau de Javel}{doc:TP3_eau_javel}
  \begin{wrapfigure}{r}{0.2\linewidth}
    \vspace*{-16pt}
    \centering
    \image{0.44}{images/securite/picto_toxique}~\image{0.44}{images/securite/picto_pollue}
    \image{0.44}{images/securite/picto_ronge}
  \end{wrapfigure}
  
  Dans la maison, pour désinfecter :
  \begin{listePoints}
    \item Les surfaces lavables : diluer $3 + \frac{1}{2}$ verre d’eau de Javel dans \qty{5}{\litre} d’eau, laver.
    Laisser agir 15 minutes puis rincer.
    \item Les canalisations : diluer 1 verre dans \qty{1}{\litre} d’eau, verser.
    \item La poubelle : diluer $1 + \frac{1}{2}$ verre d’eau de Javel dans \qty{1}{\litre} d’eau, frotter.
    Laisser agir 15 minutes puis rincer.
  \end{listePoints}
  \begin{donnees}
    \item \num{1} verre = \qty{14}{\centi\litre}
  \end{donnees}
\end{doc}

\begin{doc}{Matériel disponible}{doc:TP3_verrerie_disponible}
  \begin{listePoints}
    \item Éprouvette graduée.
    \item Pipettes jaugées de \qty{5,0}{\ml}, \qty{10,0}{\ml}.
    \item Pipette graduée.
    \item Fioles jaugées de \qty{25,0}{\ml}, \qty{100,0}{\ml}, \qty{200,0}{\ml}.
    \item Bécher.
  \end{listePoints}
\end{doc}

\titreSection{Questions préliminaires}

\question{
  En utilisant le document~\ref{doc:TP3_eau_javel}, calculer le volume d'eau de Javel qu'il faut utiliser pour préparer la solution pour surfaces lavables.
}{}{3}

\question{
  Comparer le volume trouvé aux \qty{5}{\litre} d'eau nécessaires pour réaliser la dilution.
}{}{1}


\titreSection{Travail à réaliser}

\mesure
Utilisez les documents proposés et les questions préliminaires pour préparer une solution pour surfaces lavables.
Vous détaillerez le raisonnement suivi et rédigerez votre compte rendu, en respectant les étapes de la démarche scientifique.
  
\chevron[couleurSec] Proposition d’organisation :
\begin{protocole}
  \item Par groupe de 4, reformuler la problématique et proposer un protocole pour préparer la solution.
  \item Par 2, mettre en oeuvre le protocole en respectant les consignes de sécurité.
  \item Seule, rédiger un compte-rendu en incluant des schémas légendés et une conclusion.
\end{protocole}