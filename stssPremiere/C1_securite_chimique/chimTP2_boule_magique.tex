%%%%
\tetePremStssChim

%%%% titre
\numeroActivite{2}
\titreTP{La boule magique}

\begin{tableauCompetences}
  APP & Rechercher et organiser l'information & & & & \\
  REA & Réaliser des calculs. Réaliser un protocole en respectant les consignes de sécurités. & & & & \\
\end{tableauCompetences}

%%%% objectifs
\begin{objectifs}
  \item Connaître et utiliser la relation $n = m / M$
  \item Mettre en oeuvre un protocole de dissolution en respectant les consignes de sécurités
  \item Comprendre la notion de concentration molaire
\end{objectifs}

\begin{contexte}
  Willy le marabout prétend être capable de voir l’avenir !
  Pour cela, il place ses client-es devant une mystérieuse boule bleue.
  Les client-es répètent leurs questions dans leur tête.
  Si la boule reste bleue alors la réponse à la question est \og non \fg, si la couleur se disperse alors la réponse est \og oui \fg.
  
  \problematique{
    Comment préparer la solution chimique présente dans la boule bleue ?
  }
\end{contexte}
\bigskip


%%%%
\begin{doc}{Recette de la boule magique}{doc:TP2_protocole_sol_magique}
  Pour préparer la solution \og magique \fg\; dans un erlenmeyer, il faut 
  \begin{listePoints}
    \item mettre \qty{6,0e-2}{\mole} d'hydroxyde de sodium de formule brute \chemfig{NaOH}
    \item mettre \qty{1,7e-2}{\mole} de glucose de formule brute \bruteCHO{6}{12}{6}
    \item ajouter \qty{125}{\ml} d'eau de formule brute \chemfig{H_2 O}
    \item agiter légèrement
    \item enfin ajouter une goutte de bleu de méthylène de formule brute \chemfig{C_{16}H_{18}Cl N_3 S}
    \item agiter à nouveau en fermant le bouchon
    \item mesurer le temps nécessaire à la décoloration de la solution.
  \end{listePoints}

  \attention Port de la blouse, des gants et des lunettes de protection obligatoire !
\end{doc}

\question{
  Donner la relation qui permet de calculer la masse à partir de la masse molaire et de la quantité de matière ($m = \ldots$).
  Rappeler les unités des grandeurs dans la relation.
}{
  \begin{equation*}
    m = n \times M
  \end{equation*}
}{2}

\begin{doc}{Masse molaire de quelques éléments chimiques}{doc:TP2_donnees_molaire}
  \begin{donnees}
    \item $\masseMol{H}  = \qty{1,0}{\g\per\mole}$
    \item $\masseMol{C}  = \qty{12,0}{\g\per\mole}$
    \item $\masseMol{N}  = \qty{14,0}{\g\per\mole}$
    \item $\masseMol{O}  = \qty{16,0}{\g\per\mole}$
    \item $\masseMol{Na} = \qty{23,0}{\g\per\mole}$
    \item $\masseMol{Cl} = \qty{35,0}{\g\per\mole}$
    \item $\masseMol{S}  = \qty{32,0}{\g\per\mole}$
  \end{donnees}
\end{doc}


\newpage
\vspace*{-30pt}
\question{
  Calculer les masses molaires des molécules d'hydroxyde de sodium \chemfig{NaOH} et de glucose \bruteCHO{6}{12}{6}.
}{
 
}{4}

\question{
  Calculer la masse de glucose \bruteCHO{6}{12}{6} et d'hydroxyde de sodium \chemfig{NaOH} à prélever.
  Appeler le professeur pour vérifier le calcul.
}{

}{3}

\mesure Après avoir bien mis ses gants et les lunettes de protection, effectuer la pesée du glucose et de l'hydroxyde de sodium.

\mesure réaliser le reste du protocole décrit dans le document~\ref{doc:TP2_protocole_sol_magique}.
\bigskip

\question{
  Calculer la masse molaire du bleu de méthylène \chemfig{C_{16}H_{18}Cl N_3 S}.
}{

}{2}

\question{
  Calculer la masse de bleu de méthylène qu'il faudrait prélever pour respecter le protocole du document~\ref{doc:TP2_protocole_sol_magique}.
  Peut-on mesurer cette masse avec le matériel disponible ?
}{

}{2}

\begin{doc}{Notion de concentration molaire}{doc:TP2_}
  Ici on a réalisé une solution avec des \important{concentrations molaires} bien précises.
  
  \begin{encart}
    Comme la concentration massique, la \important{concentration molaire $c$} désigne la quantité de matière $n$ de soluté dissous dans un volume de solution donné
    \begin{equation*}
      c = \dfrac{n_\solute}{V_\solution}
    \end{equation*}
  \end{encart}
  \begin{listePoints}
    \item $c$ : concentration molaire en \unit{\mole/\litre}
    \item $n_\solute$ : quantité de matière du soluté en \unit{\mole}
    \item $V_\solution$ : volume de la solution en \unit{\litre}
  \end{listePoints}
\end{doc}

\question{
  Calculer les concentration molaire en glucose \bruteCHO{6}{12}{6} et en hydroxyde de sodium \chemfig{NaOH} dans la solution préparée.
}{

}{4}