%%%%
\tetePremStssChim

%%%% titre
\numeroActivite{3}
\titreActivite{Entretien des cheveux}

\begin{objectifs}
  \item Définir un acide et une base selon le modèle de Br\o{}nsted
  \item Écrire l’équation d’une réaction acido-basique à partir des couples acide/base.
\end{objectifs}

\begin{contexte}
  Sur certains sites de beauté, il est conseillé d’utiliser du vinaigre de cidre et du bicarbonate de soude pour entretenir ses cheveux.
  Inès, âgée de 8 ans, se verse les deux produits sur les cheveux sans les diluer.
  Catastrophe ! Une émulsion gazeuse se forme aussitôt sur sa tête !

  \problematique{
    Que s'est-il passé quand le bicarbonate de soude et le vinaigre de cidre se sont mélangés ?
  }
\end{contexte}

\begin{doc}{Le modèle de Br\o{}nsted}{doc:A3_modele_bronsted}
  \begin{wrapfigure}[3]{r}{0.1\linewidth}
    \qrcode{https://www.youtube.com/watch?v=EztCKi3oJEI}
  \end{wrapfigure}

  Joannes Nicolaus Br\o{}nsted est un chimiste danois du début du 20ème siècle.
  Il est connu pour sa définition des substances acides et basiques : 
  \begin{importants}    
    \begin{listePoints}
      \item Un acide est une molécule capable de donner un ion \chemfig{H^+} (un proton)
      \item Une base est une molécule qui reçoit un ion \chemfig{H^+}
    \end{listePoints}
  \end{importants}
  
  L'acide noté \chemfig{AH}, se transforme en sa base conjuguée notée \chemfig{A^{-}} en perdant un proton \chemfig{H^+}.
  La base conjuguée \chemfig{A^{-}} se transforme en l'acide \chemfig{AH} quand elle capte un proton \chemfig{H^+}.

  \begin{importants}
    On parle de couple acide/base noté ici \chemfig{AH}/\chemfig{A^{-}}.    
  \end{importants}
  \attention l'acide est toujours à gauche et la base est toujours à droite.

  \exemple
  \texteTrouLignes{\chemfig{HCl}/\chemfig{Cl^{-}}, l'acide est \chemfig{HCl} et la base \chemfig{Cl^{-}} est dans ce couple.}
\end{doc}

\question{
  Indiquer l'acide et la base du couple acide/base \chemfig{H_2O}/\chemfig{HO^{-}}.
}{}{1}

\begin{doc}{Réaction acido-basique}{doc:A3_acido_basique}
  \begin{importants}  
    Lors d'une réaction chimique acido-basique, l'acide d'un couple réagit avec la base d'un autre couple.
  \end{importants}
  
  \exemple On a deux couples : \chemfig{H_3O^{+}}/\chemfig{H_2O} et \chemfig{NH_4^{+}}/\chemfig{NH_3}.

  Si on mélange les ions oxonium \chemfig{H_3 O^{+}}, un acide, avec l'ammoniac \chemfig{NH_3}, une base, on va avoir une réaction chimique
  \vspace*{-12pt}
  \begin{center}
    \chemfig{H_3O^{+}} + \chemfig{NH_3} \reaction \chemfig{H_2O} + \chemfig{NH_4^{+}}
  \end{center}
\end{doc}

\question{
  Établir la réaction acido-basique entre le couple \chemfig{H_3O^{+}}/\chemfig{H_2O} et le couple \chemfig{H_2O}/\chemfig{HO^{-}}.  
}{}{2}

%%
\newpage
\vspace*{-40pt}

\begin{doc}{Le bicarbonate de soude}{doc:A3_bicarbonate}
  \begin{wrapfigure}{r}{0.2\linewidth}
    \vspace*{-38pt}
    \centering
    \image{0.63}{images/photos/bicarbonate}
  \end{wrapfigure}
  Le bicarbonate de sodium ou bicarbonate de soude (abus de langage) est nommé hydrogénocarbonate de sodium en nomenclature moderne.
  C’est un composé chimique dont la formule brute est \chemfig{NaHCO_3}.
  Il se présente sous la forme de fins cristaux blancs, solubles dans l’eau, qui forme les ions sodium \chemfig{Na^+} et hydrogénocarbonate \chemfig{HCO_3^{-}} en solution.
\end{doc}

\begin{doc}{Le vinaigre de cidre}{doc:A3_vinaigre}
  \begin{wrapfigure}{l}{0.2\linewidth}
    \vspace*{-22pt}
    \centering
    \image{0.65}{images/photos/vinaigre}
  \end{wrapfigure}
  Le vinaigre est une solution aqueuse à faible concentration en acide éthanoïque de
  formule \chemfig{CH_3COOH}, qui rentre principalement dans l'alimentation humaine comme condiment et conservateur alimentaire. 
  Le vinaigre résulte d'une transformation d'une solution aqueuse d'éthanol (le vin ou ici le cidre) exposée à l'air, et fermenté à l’aide de micro-organisme.
  Cela explique son étymologie de \og vin aigre \fg\; devenu \og vinaigre \fg.
  Le vinaigre de cidre à 8° comporte \qty{8}{\g} d’acide éthanoïque par litre de solution
\end{doc}

\begin{doc}{Couples acide/base à connaitre}{doc:A3_couple_acide_base}
  \begin{tblr}{
    hlines, colspec = {|X[c,m] |c |X[c,m] |c |X[c,m] |},
    row{1} = {couleurPrim!20},
    row{2} = {couleurPrim!10},
    width = \linewidth,
  }
    & \SetCell[c=2]{c} Forme acide & & \SetCell[c=2]{c} Forme basique \\
    Couple acide/base & Formule brute & Nom & Formule brute & Nom \\
    \chemfig{H_3O^+}/\chemfig{H_2O}         & \chemfig{H_3O^+} & ion oxonium          & \chemfig{H_2O} & eau \\
    \chemfig{H_2O}/\chemfig{HO^{-}}         & \chemfig{H_2O} & eau                    & \chemfig{HO^{-}} & ion hydroxyde \\
    \chemfig{HCl}/\chemfig{Cl^{-}}          & \chemfig{HCl} & acide chlorhydrique     & \chemfig{Cl^{-}} & ion chlorure \\
    {\chemfig{CH_3COOH}/\\    
    \chemfig{CH_3COO^{-}}}                  & \chemfig{CH_3COOH} & acide éthanoïque   & \chemfig{CH_3COO^{-}} & ion éthanoate \\
    \chemfig{H_2CO_3}/\chemfig{HCO_3^{-}}   & \chemfig{H_2CO_3} & acide carbonique    & \chemfig{HCO_3^{-}} & ion hydrogéno-carbonate \\
    \chemfig{HCO_3^{-}}/\chemfig{CO_3^{2-}} & \chemfig{HCO_3^{-}} & ion hydrogéno-carbonate & \chemfig{CO_3^{2-}} & ion carbonate \\
    \chemfig{NH_4^+}/\chemfig{NH_3}         & \chemfig{NH_4^+} & ion ammonium         & \chemfig{NH_3} & ammoniac \\
  \end{tblr}
\end{doc}

\question{
  En utilisant les documents~\ref{doc:A3_bicarbonate},~\ref{doc:A3_vinaigre} et~\ref{doc:A3_couple_acide_base}, donner les couples acide/base composant le bicarbonate de soude et présent dans le vinaigre de cidre.
}{}{2}

\question{
  Établir la réaction acido-basique qui a lieue quand on mélange du bicarbonate de soude et du vinaigre de cidre.
}{}{2}

\question{
  En pratique, l'acide carbonique se décompose spontanément en eau et en dioxyde de carbone.
  Modifier la réaction acido-basique en conséquence.
}{}{1}