%%%%
\tetePremStssBiom
\titreActivite{Les aliments comme source d'énergie}

\begin{doc}{Les aliments sont sources d'énergie}{doc:A_aliments_sources}
  Le \important{métabolisme} est l'ensemble des réactions chimiques qui permettent aux cellules d'un organisme de fonctionner.
  Cet ensemble de réactions chimiques a besoin \important{d'énergie} pour avoir lieu et cette énergie est issue de la \important{dégradation des aliments avalés.}
\end{doc}

\begin{doc}{Les différents types de nutriments}{doc:A_nutriments}
  Pendant la digestion, les molécules complexes dans les aliments vont être fragmentées en nutriments simples par \important{hydrolyse} (« \textit{délier avec l'eau} » en grec).
  Les nutriments sont ensuite transportés jusqu'aux organes et aux muscles.
  On peut distinguer deux types de nutriments :
  \begin{listePoints}
    \item les \important{macronutriments énergétiques} qui doivent être apportés en grande quantités à l'organisme et servent de sources d'énergie. Ce sont les \important{glucides,} les \important{lipides} et les \important{protéines.}
    \item les \important{micronutriments} qui doivent être apportés en plus petites quantités à l'organisme et ne servent pas à produire de l'énergie.
    Ce sont les \important{vitamines,} les \important{minéraux} et les \important{oligoéléments} (oligo veut dire « \textit{peu abondant} » en grec).
  \end{listePoints}
\end{doc}

\begin{doc}{Avoir une alimentation équilibrée}{doc:A_alimentation_equilibre}
  Aucun aliment seul ne contient tous les nutriments nécessaires au bon fonctionnement du corps humain.
  Il faut donc avoir une \important{alimentation variée,} avec au minimum \important{5 fruits et légumes par jour,} c'est-à-dire au moins \important{5 portions de fruits et légumes par jour.}

  \begin{importants}
    « Une \important{[portion] de légumes} (frais, surgelés ou en conserve) représente 80 à 100 grammes. Cela peut correspondre à, par exemple :
    \begin{listePoints}
      \item 1 petite assiette de crudités ou de légumes cuits ;
      \item 1 tomate moyenne ou 1 carotte ;
      \item 1 bol de soupe (250 ml) ou 1 part de gratin de légumes.
    \end{listePoints}
    Une \important{portion de fruits} correspond par exemple à :
    \begin{listePoints}
      \item 1 pomme, 1 poire ou 1 banane ;
      \item 2 clémentines, 2 kiwis ou 2 gros abricots ;
      \item 1 petit bol de salade de fruits ou de fruits au sirop (à consommer sans le sirop) ;
      \item 1 petit pot de compote sans sucre ajouté.
    \end{listePoints}
    \important{Jus de fruits et yaourts aux fruits : zéro portion de fruit.} »
    \sourceExtrait{https://www.ameli.fr}
  \end{importants}

  Les fruits et légumes sont essentiels, car ils contiennent des \important{fibres alimentaires,} qui permettent d'avoir un \important{microbiote intestinal} sain et fonctionnel.

  \begin{importants}  
    Il faut par ailleurs respecter l'apport énergétique (AE) recommandé par l'agence nationale de la santés :
    \begin{listePoints}
      \item \num{35} à \qty{40}{\percent} des apports énergétique doivent venir des lipides ;
      \item \num{10} à \qty{20}{\percent} des apports énergétique doivent venir des protéines ;
      \item \num{40} à \qty{55}{\percent} des apports énergétique doivent venir des glucides.
    \end{listePoints}
  \end{importants}
\end{doc}

\newpage
\vspace*{-36pt}
\begin{doc}{Valeur énergétique des aliments}{doc:A_valeur_energetique}
  \begin{importants}  
    \important{La valeur énergétique} d'un aliment est exprimée en kilojoules noté \unit{\kilo\joule}.
  \end{importants}
  Pour des raisons historiques, on peut aussi l'exprimer en \important{calorie} noté \unit{\cal} ou en kilocalories noté \unit{\kcal}.
  Une \important{calorie} est l'énergie requise pour augmenter de \qty{1}{\degreeCelsius} un gramme d'eau.

  La valeur énergétique d'un aliment correspond à la chaleur produite pendant leur combustion avec du dioxygène dans l'organisme.
  Pour calculer la valeur énergétique d'un aliment, il faut faire la somme de l'énergie apporté par les glucides, les lipides et les protéines.
  
  \begin{importants}
    L'énergie fournie par les macronutriments est de
    \begin{listePoints}[2]
      \item \qty{38}{\kilo\joule\per\g} pour les lipides ;
      \item \qty{17}{\kilo\joule\per\g} pour les protéines ;
      \item \qty{17}{\kilo\joule\per\g} pour les glucides.
    \end{listePoints}
  \end{importants}

  \begin{donnees}[2]
    \item $\qty{1}{\cal} = \qty{4,18}{\joule}$
    \item $\qty{1}{\kilo\joule} = \qty{1000}{\joule}$
    \item A 18 ans, la ration alimentaire recommandée est de \qty{2300}{\kcal} par jour.
  \end{donnees}
  \vspace*{-12pt}
  Pour calculer la valeur énergétique d'un aliment, on additionne l'énergie fournie par les glucides, les protéines et les lipides
  \begin{equation*}
    E_\text{aliment} 
    = (m_\text{glucide} + m_\text{protéine}) \times \qty{17}{\kilo\joule\per\g}
    +  m_\text{lipide} \times \qty{38}{\kilo\joule\per\g}
  \end{equation*}
\end{doc}

\begin{doc}{Exemple d'un repas du déjeuner}{doc:A_exemple_repas}
  À la cantine le repas suivant est servi :
  \vspace*{-8pt}
  \begin{multicols}{2}
    \begin{itemize}
      % \item \qty{50}{\g} de laitue ;
      \item \qty{100}{\g} de quinoa cuit ;
      \item \qty{150}{\g} de steak de soja ;
      \item \qty{40}{\g} de clémentine ;
      \item \qty{25}{\g} de pain complet.
      % \item[\vspace{\fill}]
    \end{itemize}
  \end{multicols}
  
  \begin{tblr}{
    colspec = {l X[c] X[c] X[c]}, hlines, vlines,
    column{1} = {couleurSec-50}, row{1} = {couleurSec-100}, 
  }
    Composition en \unit{\g} pour \qty{100}{\g} d'aliment & \important{Glucide} & \important{Lipides} & \important{Protéines} \\
    %
    % Laitue        & \num{1.4} & \num{0.2} & \num{1.3} \\
    Quinoa cuit   & \num{21}  & \num{1.9} & \num{4.4} \\
    Steak de soja & \num{0}   & \num{9.0} & \num{20}  \\
    Clémentine    & \num{12}  & \num{0.19}  & \num{0.8} \\
    Pain complet  & \num{44}  & \num{1.5} & \num{9.0} \\
  \end{tblr}
\end{doc}


\question{
  Calculer la valeur énergétique $E$ de chaque aliment pour \qty{100}{\g}
}{}[4]

\question{
  Calculer la valeur énergétique totale du repas fourni à la cantine.
}{}[2]

\question{
  Vérifier si ce repas fourni $\qty{35}{\percent}$ de la ration alimentaire journalière pour un-e ado. 
}{}[1]
