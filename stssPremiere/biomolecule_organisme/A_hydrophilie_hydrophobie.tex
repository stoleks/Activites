%%%%
\tetePremStssBiom

%%%% titre
\titreActivite{Hydrophilie et hydrophobie}


%%%% objectifs
\begin{objectifs}
  \item Comprendre la notion d'hydrophilie et d'hydrophobie.
\end{objectifs}

\begin{contexte}
  
  \problematique{
  }
\end{contexte}


%%%% docs
\begin{doc}{}{doc:1}
\end{doc}

%%
\begin{doc}{}{doc:2}
\end{doc}


%%%%
\question{
}{
}{}

\numeroQuestion



%%%% DANS UNE ACTIVITE SUR LA POLARITE DE L'EAU
% \begin{doc}{Micelles et membrane cellulaire}{doc:A6_micelle_membrane}
%   La structure des molécules de lipides mène à la formation de structure particulière dans de l'eau liquide.
%   Les queue hydrophobe étant repoussée par les molécules d'eau, elles vont s'agglomérer et former des structures ou les queues sont isolées de l'eau environnante : \important{les micelles.}
  
%   Des exemples de micelles sont \important{les couches bi-lipidique,} composée de deux couches de lipides avec les têtes hydrophile orientée vers l'extérieur, ce qui permet à leur queue hydrophobes de ne pas rentrer en contact avec de l'eau. 
%   Les interactions électrostatiques entre les différentes parties de la membrane la pousse à former une sphère (comme une bulle de savon), avec un extérieur et un intérieur : c'est la base \important{d'une membrane cellulaire.}

%   Les membranes cellulaire sont plus complexe qu'une simple couche bi-lipidique : elles sont aussi composées de \important{protéines}, qui permettent de renforcer la structure de la membrane cellulaire et de contrôler ce qui sort et ce qui entre de la cellule.

%   TODO : FIGURE MICELLE ET MEMBRANE
% \end{doc}