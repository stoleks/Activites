%%%%
\tetePremStssBiom
\vspace*{-32pt}
\titreActivite{Production d'énergie dans la cellule}

\begin{doc}{La respiration cellulaire}{doc:A_respiration_cellulaire}
  \begin{importants}
    La \important{respiration cellulaire} est l'ensemble des mécanismes chimiques, permettant de produire et de stocker de l'énergie dans les molécules \important{d'adénosine triphosphate} (notée ATP) au sein des cellules, à partir de la combustion du glucose.
  \end{importants}
  
  Les cellules possèdent deux moyens de fabriquer de l'ATP à partir de la combustion du glucose :
  \begin{listePoints}
    \item La filière aérobie (« \textit{vie avec air} » en grec), qui a lieue en présence de dioxygène \dioxygene, dans les \important{mitochondries} et qui produit jusqu'à \num{36} ATP.
    \item La filière anaérobie (« \textit{vie sans air} » en grec), qui a lieue en absence de dioxygène \dioxygene, dans le \important{cytosol.} Elle produit \num{2} ATP et de l'acide lactique.
  \end{listePoints}

  Pour la filière aérobie, le glucose de formule brute \chemfig{C_6 H_{12} O_6} va réagir avec le dioxygène \dioxygene issue du sang au cours d'une réaction de combustion
  \begin{equation*}
    \chemfig{C_6 H_{12} O_6} + 6 \dioxygene 
    \reaction 6 \eau + 6 \dioxydeDeCarbone
  \end{equation*}

  Cette réaction d'oxydoréduction génère un flux d'électrons échangés qui va permettre d'enclencher un ensemble de réactions appelées \important{cycle de Krebs} au sein des mitochondries.
  
  Chaque cycle de Krebs produit de l'ATP à partir \important{d'adénosine diphosphate} (notée ADP), grâce à une enzyme : \important{l'ATP-synthase,} qui se trouve dans les crêtes mitochondriales.
  La production d'ATP peut se résumer par la réaction suivante
  \begin{equation*}
    \chemfig{ADP} + \chemfig{HPO_4^{2-}} \reaction \chemfig{ATP} + \eau
  \end{equation*}

  \begin{importants}  
    L'ATP permet de stocker l'énergie dans la cellule sous une forme facilement récupérable.
  \end{importants}
\end{doc}

\begin{doc}{L'adénosine triphosphate : une réserve d'énergie pour les cellules}{doc:A_reserve_energie}
  Dans les cellules, l'énergie est stockée dans une molécule : l'adénosine triphosphate (notée ATP).
  Au entrant en contact avec l'eau contenue dans la cellule, l'adénosine triphosphate va se transformer en adénosine diphosphate (notée ADP), en libérant de l'énergie directement utilisable par la cellule
  \begin{equation*}
    \chemfig{ATP} + 2\eau 
    \reaction 
    \chemfig{ADP} + \chemfig{HPO_4^{2-}} + \oxonium + \text{énergie}
  \end{equation*}
  Cette réaction s'appelle \important{l'hydrolyse de l'adénosine triphosphate.}
  Elle permet à la cellule de fonctionner en récupérant l'énergie stockée dans les molécules d'ATP.

  \begin{center}  
    \chemfigHaworth{!\ATPHaw}
    
    \legende{
      Formule topologique de l'adénosine triphosphate. 
      L'adénosine triphosphate (ATP), est constitué de trois groupes phosphate, d'un ribose et d'une adénine
    }
  \end{center}

  \begin{center}  
    \chemfigHaworth{!\ADPHaw}

    \legende{
      Formule topologique de l'adénosine diphosphate. 
      L'adénosine diphosphate (ADP), est constitué de deux groupes phosphate, d'un ribose et d'une adénine
    }
  \end{center}
\end{doc}

\question{
  Au cours d'un exercice physique prolongé, l'apport en dioxygène diminue et de l'acide lactique se forme dans les cellules, ce qui mène à des crampes et des douleurs musculaires.
  Expliquer cette formation d'acide lactique à l'aide du document~\ref{doc:A_respiration_cellulaire}.
}{}[4]

\question{
  Recopier la molécule d'adénosine triphosphate (ATP).
  Entourer et nommer les groupes caractéristiques que vous reconnaissez.
}{}[6]

\question{
  Réaliser une carte mentale qui synthétise les différentes étapes de productions, stockages et récupérations de l'énergie dans une cellule.
}{}