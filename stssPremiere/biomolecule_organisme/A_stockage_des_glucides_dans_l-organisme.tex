\tetePremStssBiom
\titreActivite{Stockage des glucides dans l'organisme}

\begin{doc}{Classification des glucides}
  \begin{importants}
    Les \important{glucides} sont des molécules organiques possédants \important{1 groupe carbonyle \chemfig{C=O}} et \important{au moins 2 groupes hydroxyles \chemfig{HO} ou \chemfig{OH}}.
  \end{importants}
  Dans nos organismes, les briques de bases des glucides sont trois molécules isomères de formules brutes \chemfig{C_6 H_{12} O_6} : le \important{glucose,} le \important{fructose} et le \important{galactose.}
  Tous les autres glucides sont des \important{polymères} de ces sucres simples.
  \begin{importants}
    Un \important{polymère} et une macromolécule assemblée à partir d'une même unité de base répétée plusieurs fois.
    Les molécules composant le polymère sont appelées \important{monomères} et leur nombre de répétition et appelé \important{indice de polymérisation.}
  \end{importants}

  \begin{center}
  \begin{tikzpicture}
    %%%% Base
    \node[rectArrondi = purple-200] (glucides) {Les glucides \\ \reaction 1 carbonyle et plus de 2 hydroxyles};
    \node[rectArrondi= blue-200, below left= 0.5 and 0.4 of glucides] (mono) {Glucides simples, monosaccharides, \important[black]{non hydrolysables}};
    \node[rectArrondi= red-200!75, below right= 0.5 and 0.2 of glucides] (poly) {Glucides complexes, polysaccharides, \important[black]{hydrolysables}};
    %%%% Oses
    \node[rectArrondi= cyan-100, below left=  1.5 and -1.9 of mono] (aldose) {Aldose (aldéhyde)};
    \node[rectArrondi= cyan-100, below right= 1.5 and -1.9 of mono] (cetose) {Cétose (cétone)};
    %%
    \node[below left=  2 and -3   of aldose] (glu)  {\chemname{\chemfig{!\glucoseHaw}}{glucose}};
    \node[below right= 2 and -1   of aldose] (gala) {\chemname{\chemfig{!\galactoseHaw}}{galactose}};
    \node[below right= 2 and -2.2 of cetose] (fru)  {\chemname{\chemfig{!\fructofuranoseHaw}}{fructose}};
    %%%% Osides
    \node[rectArrondi= orange-100!75, below left=  1.5 and -1.9 of poly] (disac)   {Disaccharides (2 oses)};
    \node[rectArrondi= orange-100!75, below right= 1.5 and -1.9 of poly] (polysac) {Polysaccharides (n oses)};
    %%
    \node[below left= 5.5 and -0.5 of disac] (lactose) {Lactose = Glucose + Galactose};
    \node[below left= 6.5 and -0.5 of disac] (maltose) {Maltose = Glucose + Glucose};
    \node[below left= 7.5 and -0.5 of disac] (saccha)  {Saccharose = Glucose + Fructose};
    %%
    \node[below right= 2 and -2 of polysac, text width = 2cm] (amidon) {Amidon (plante)};
    \node[below right= 4 and -2 of polysac, text width = 2cm] (glycogene) {Glycogène (animaux)};
    
    %%%% Flèches
    \draw[fleche=purple-300] (glucides) -| (mono);
    \draw[fleche=purple-300] (glucides) -| (poly);
    %
    \draw[fleche=blue-300] (mono) -- ($(mono) + (0,-1.5)$) -| (aldose);
    \draw[fleche=blue-300] (mono) -- ($(mono) + (0,-1.5)$) -| (cetose);
    %
    \draw[fleche=red-300] (poly)  -- ($(poly) + (0,-1.5)$) -| (disac);
    \draw[fleche=red-300] (poly)  -- ($(poly) + (0,-1.5)$) -| (polysac);
    %
    \draw[fleche=cyan-200] (aldose) -- ($(aldose) + (0,-1.5)$) -| (glu);
    \draw[fleche=cyan-200] (aldose) -- ($(aldose) + (0,-1.5)$) -| (gala);
    \draw[fleche=cyan-200] (cetose) -- ($(cetose) + (0,-1.5)$) -| (fru);
    %
    \draw[fleche=orange-200!75] ($(disac) + (0.5, -0.8)$) |- (lactose);
    \draw[fleche=orange-200!75] ($(disac) + (1, -0.8)$)   |- (maltose);
    \draw[fleche=orange-200!75] ($(disac) + (1.5, -0.8)$) |- (saccha);
    %
    \draw[fleche=orange-200!75] ($(polysac) + (-1,-0.8)$)   |- (amidon);
    \draw[fleche=orange-200!75] ($(polysac) + (-1.5,-0.75)$) |- (glycogene);
  \end{tikzpicture}
  \end{center}

  Le maltose est un polymère du glucose avec un indice de polymérisation de \num{2}.
  L'amidon et le glycogène sont des polymères du glucose avec un indice de polymérisation allant de \num{600} à \num{100000}.
\end{doc}

\begin{doc}{Hydrolyse des glucides}
  Les glucides complexes servent de \important{réserve d'énergie} pour l'organisme. 
  Cette énergie stockée peut être récupérée au cours d'une réaction \important{d'hydrolyse,} ou le glucide complexe va être coupé en un polymère plus petit et un monomère, ou deux monomères pour les disaccharides.

  \begin{importants}
    Pour que la réaction d'hydrolyse ait lieu, il faut que le milieu dans lequel se trouvent les glucides complexes \important{soit acide} ou qu'il contienne \important{une enzyme} adaptée pour couper le glucide.
  \end{importants}

  \exemple Réaction d'hydrolyse du saccharose en milieu acide ou en présence de saccharase :
  \begin{equation*}
    \chemfig{!\saccharoseHaw} \; + \; \eau 
    \; \reaction \; 
    \chemfig{!\glucoseHaw} + \chemfig{!\fructofuranoseHaw}
  \end{equation*}
\end{doc}

\question{
  Identifier les deux espèces chimiques formées pendant l'hydrolyse du saccharose.
}{
  À gauche on a le glucose, à droite on a le fructose.
}[1]

\question{
  Donner la formule brute du saccharose.
}{
  \chemfig{C_{12} H_{22} O_{11}}.
}[1]

\question{
  Réécrire la réaction d'hydrolyse du saccharose en utilisant uniquement les formules brutes des molécules.
}{
  Le saccharose est coupé en deux par l'eau pour former un glucose et un fructose :
  \begin{equation*}
    \chemfig{C_{12} H_{22} O_{11}} + \eau \reaction
    \chemfig{C_6 H_{12} O_6} + \chemfig{C_6 H_{12} O_6}
  \end{equation*}
}[2]

\begin{doc}{Stockage des glucides}
  Quand le glucose présent dans le sang n'est pas utilisé comme carburant par un organisme, il est stocké sous forme \important{de glycogènes} (chez les animaux) ou \important{d'amidon} (chez les plantes).

  Le \important{glycogène} (« \textit{origine du glucose} ») est un polymère de glucose, formé dans l'organisme au cours d'une réaction de \important{polycondensation}, c'est-à-dire une réaction de condensation en chaîne du glucose, selon l'équation suivante :
  \begin{equation*}
    n \left(\chemfig{C_6 H_{12} O_6}\right) \reaction
    \chemfig{HO-[,0.8] \parentheseG C_6 H_{10} O_5 -[,1.2]\parentheseD_{\phantom{B} n}}\chemfig{H}
    + (n-1)\eau
  \end{equation*}

  Ce glycogène constitue une réserve d'énergie facilement synthétisable et facilement accessible pour l'organisme par hydrolyse.

  Une autre façon de stocker le glucose est de former des \important{acides gras,} puis des \important{triglycérides,} qui seront stockés dans les \important{tissus graisseux} chez les animaux.
  La formation de triglycérides à partir de glucoses est réalisée dans le foie et implique plusieurs protéines, qui jouent un rôle de régulation et d'enzymes nécessaires pour synthétiser des triglycérides.

  On retiendra que les organisme stockent l'énergie contenue dans le glucose sous deux formes :
  \begin{listePoints}
    \item le glycogène pour former des réserves rapidement. C'est la source d'énergie qui est utilisée en premier par les cellules.
    \item les triglycérides pour former des réserves sur la durée, qui seront utilisées en cas d'absence de glycogène. À masse égale, les triglycérides libèrent 6 fois plus d'énergie que les glycogènes.
  \end{listePoints}
\end{doc}