%%%%
\tetePremStssMeth

%%%% titre
\numeroActivite{1}
\vspace*{-36pt}
\titreActivite{L'analyse dimensionnelle}

\begin{objectifs}
  \item Comprendre la notion d'équation homogène
  \item Réaliser de l'analyse dimensionnelle
\end{objectifs}

\begin{contexte}
  En physique, une relation est correcte si elle est \important{homogène :} les membres de droites et de gauche de l'égalité doivent être exprimé avec la même \important{unité.}

  \problematique{Comment vérifier que les deux côté d'une égalité sont bien exprimés dans la même unité ?}
\end{contexte}

%%%%
\vspace*{-8pt}
\titreSection{Les puissances négatives}
\vspace*{-8pt}

%%
\begin{doc}{Puissance négative}{doc:A2_puissance_negative}
  Une puissance indique combien de fois on répète une multiplication.
  ($3^3 = 3\times 3 \times 3 = 27$)

  Une puissance \important{négative} correspond à une division par une puissance.
  $\left(5^{-2} = \dfrac{1}{5^2}\right)$

  \begin{encart}
    On a les mêmes règles de calculs avec les unités.
    $\left(\dfrac{1}{\unit{\s}} = \unit{\per\s}, \quad
    \unit{\per\m\cubed} = \dfrac{1}{\unit{\m\cubed}}\right)$
  \end{encart}
\end{doc}

\begin{doc}{Multiplication d'unité}{doc:A2_multiplication_unite}
  \begin{encart}
    Quand on multiplie deux unités entre elles, la multiplication est indiquée par un point médian $\cdot$
    
    \exemple $\unit{\kilo\watt\hour} = \unit{\kilo\watt}\times\unit{\hour}$
  \end{encart}
\end{doc}


\begin{multicols}{2}
  \numeroQuestion Relier les valeurs égales entre elles.
  \begin{center}
    \begin{tblr}{ colspec = {c c X[1,c] c c}, width = 0.5\linewidth }
      $4^{-2}$  & \pointCyan & & \pointCyan & $\dfrac{1}{10}$ \\
      $25^{-1}$ & \pointCyan & & \pointCyan & \num{0,04} \\
      $10^{-1}$   & \pointCyan & & \pointCyan & $\dfrac{1}{4^2}$ \\
                &            & & \pointCyan & \num{0,10}
    \end{tblr}
  \end{center}
  
  \numeroQuestion Relier les unités égales entre elles.
  \begin{center}
    \begin{tblr}{ colspec = {c c X[1,c] c c}, width = 0.5\linewidth }
      \unit{\m\per\second}                 & \pointCyan & & \pointCyan & $\dfrac{\unit{\kg}}{\unit{\cubic\m}}$ \\
      \unit{\kg\per\cubic\m}               & \pointCyan & & \pointCyan & \unit{\cubic\m\per\s} \\
      $\dfrac{\unit{\cubic\m}}{\unit{\s}}$ & \pointCyan & & \pointCyan & $\dfrac{\unit{m}}{\unit{s}}$ \\
      \unit{\m/\s}                         & \pointCyan & & &
    \end{tblr}
  \end{center}
\end{multicols}


%%%%
\titreSection{Opérations et unités}

\sisetup{unit-color = couleurQuat}

\begin{doc}{Calcul d'une unité}{doc:A2_produits_quotient}
  \begin{encart}  
    Si une grandeur est le produits de plusieurs grandeurs, son unité est le produit des unités de ces grandeurs.

    De même si une grandeur est le quotient de plusieurs grandeurs.
  \end{encart}

  \exemple Une vitesse $v = \dfrac{d (\unit{\m})}{\Delta t (\unit{s})}$ s'exprime en $\dfrac{\unit{\m}}{\unit{s}}$, c'est-à-dire en \unit{\m/\s} ou \unit{\m\per\s}.

  \begin{encart}
    Pour additionner ou soustraire deux grandeurs, elles doivent être de même unités.

    Le résultat du calcul s'exprime dans les même unités que les grandeurs additionnées ou soustraites.
  \end{encart}

  \exemple La masse d'une molécule d'eau $\eau$ est la somme de la masse des atomes qui la compose 
  $m_{\eau} = 2\times m_H + m_O 
  = 2\times\qty{1,7e-27}{\kg} + \qty{26,7e-27}{\kg}
  = \qty{30,1e-27}{\kg}$
\end{doc}

\numeroQuestion Sans calcul, déterminer l'unité du membre de gauche de l'égalité. \\

\begin{tblr}{
    colspec = {X[2,l] | X[2,c] }, width = \linewidth,
    row{1} = {couleurPrim!20}, hlines
  }
  Grandeur & Unité \\
  Longueur $L = L_1 (\unit{\m}) + L_2 (\unit{m}) + L_3 (\unit{\m})$ \vphantom{$\dfrac{1}{2}$} & \\
  Fréquence $f = \dfrac{1}{T (\unit{\s})}$ & \\
  Concentration massique $c = \dfrac{m (\unit{\kg})}{V (\unit{\m\cubed})}$ & \\
  Intensité du courant $I = \dfrac{R_1 (\unit{\ohm})}{R_1 (\unit{\ohm}) + R_2 (\unit{\ohm})} \times I_1 (\unit{\ampere})$
\end{tblr}


%%%%
\titreSection{Homogénéité}

\begin{doc}{Relation homogène}{doc:A2_homogene}
  \begin{encart}  
    Une relation entre grandeurs ne peut être correcte que si elle est \important{homogène.}
    C'est-à-dire si les membres à droite et à gauche de l'égalité s'exprime avec les \important{même unités.}
  \end{encart}
  
  Toute égalité entre deux grandeurs qui ne peuvent pas s'exprimer avec les mêmes unités est donc forcément \important{fausse.}
  On dit \important{qu'elle n'est pas homogène.}
  Vérifier l'homogénéité d'une équation c'est faire de \important{l'analyse dimensionnelle.}
\end{doc}

\numeroQuestion
Calculer les unités des grandeurs des deux côtés de l'égalité des relations suivantes.
Barrer les relations qui \textbf{ne sont pas homogènes.}
\begin{alignat*}{2}
  v &= \dfrac{f}{d} 
  &\hspace{5cm}
  F &= G\times\dfrac{m_1 \times m_2}{d^2} \\
  %
  m &= m_1 \times m_2
  &\hspace{5cm}
  v &= f \times d \\
  %
  m &= c_m \times V
  &\hspace{5cm}
  V_0 &= \dfrac{c_{m,1}}{c_{m,0}} V_1
\end{alignat*}

\textbf{Données :} unités des différentes grandeurs 

\begin{center}
  \begin{tblr}{ row{1} = {couleurPrim!20}, colspec = {c|c}, hlines }
    Grandeur & Unité \\
    $f$ & \unit{\per\s} (ou \unit{\hertz}) \\
    $d$ & \unit{\m} \\
    $m$ & \unit{\kg} \\
  \end{tblr}
  ~
  \begin{tblr}{ row{1} = {couleurPrim!20}, colspec = {c|c}, hlines }
    Grandeur & Unité \\
    $F$ & \unit{\kg\m\per\s\squared} (ou \unit{\newton}) \\
    $G$ & \unit{\m\cubed \per\kg \per\s\squared} \\
    $V$ & \unit{\litre} \\
  \end{tblr}
  ~
  \begin{tblr}{ row{1} = {couleurPrim!20}, colspec = {c|c}, hlines }
    Grandeur & Unité \\
    $c_m$ & \unit{\kg\per\litre} \\
    $t$ & \unit{\s} \\
    $v$ & \unit{\m\per\s}
  \end{tblr}
\end{center}