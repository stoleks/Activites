%%%%
\tetePremStssRedo

%%%% titre
\vspace*{-36pt}
\numeroActivite{3}
\titreActivite{Conservation, précaution d'emploi et risques associés au produits oxydants}


%%%% objectifs
\begin{objectifs}
  \item Comprendre les mesures de précaution à employer avec des produits oxydants.
\end{objectifs}

\begin{contexte}
  Les produits oxydants nécessitent de respecter strictement des règles de sécurités pour éviter des accidents et pour une efficacité optimale.
  
  \problematique{
    Comment utiliser un produit oxydant en toute sécurité ?
  }
\end{contexte}


%%%% docs
\begin{doc}{Précautions d'emploi et toxicité}{doc:A3_precautions}
  Il faut respecter plusieurs règles pour utiliser des antiseptiques et des désinfectants.
  \begin{listePoints}
    \item Ils sont dangereux à fortes concentration et doivent donc être dilués.
    \item Il ne faut pas utiliser deux produits en même temps, leur action pourrait être inhibée.
    \item Il ne faut pas mélanger les antiseptiques ou les désinfectants avec autre chose que de l'eau.
  \end{listePoints}

  \begin{tblr}{
    colspec = {c Q[l, wd=0.26\linewidth] Q[l, wd=0.25\linewidth] Q[l, wd=0.26\linewidth]},
    hlines, vlines, row{1} = {couleurPrim!20}
  }
    {Produit \\ oxydant} &
    Peroxyde d'hydrogène (eau oxygénée) &
    Eau de Javel & Solution de diiode \\
    %
    {Précautions \\ et dangers} &
    \pointCyan Nocif par ingestion ou inhalation. \newline
    \pointCyan Peut provoquer des brûlures de la peau, des lésions oculaires graves, des irritations des voies respiratoires. \newline
    \pointCyan Peut provoquer un incendie ou une explosion. \newline
    \pointCyan Corrosif si concentré. &
    \pointCyan Ne jamais ingérer. \newline
    \pointCyan Peut provoquer des brûlures de la peau et des lésions oculaires graves. \newline
    \pointCyan Ne pas mélanger avec des acides (dégage un gaz toxique). \newline
    \pointCyan Très toxique pour les organismes aquatiques. &
    \pointCyan Ne pas ingérer ou avaler. \newline
    \pointCyan Irritation de la peau. \newline
    \pointCyan Peut impacter le fonctionnement de la thyroïde si utilisation répétée. \\
    %
    Stockage &
    Locaux ventilés, à l'abri de la lumière, des hautes températures, de tout combustible. &
    Locaux ventilés, à l'abri de tout rayonnement solaire et des hautes températures, à l'écart des acides et des matière organiques. &
    Locaux ventilés, à l'abri des hautes températures, à l'écart de produits susceptible de réagir avec du diiode. \\
    Conservation &
    15 jours après ouverture. &
    3 mois si concentrée, 6 à 12 mois diluée. &
    1 mois après ouverture.    
  \end{tblr}
\end{doc}

\question{
  Quels sont les précautions communes à ces trois produits oxydants ?
}{}{2}

\question{
  Indiquer les propriétés d'un local qui permettrait de stocker ces trois produits oxydants.
}{}{2}


\newpage
\vspace*{-36pt}
\begin{doc}{Principes actifs courants}{doc:A3_principes_actifs}
  Les principes actifs des antiseptiques et désinfectants agissent par \important{oxydation.}
  
  \begin{tableau}{l |c |l }
    Principe actif & Couples Ox/red & Demi-équation d'oxydoréduction \\
    \SetCell[r = 2]{l} Eau oxygénée &
    $\chemfig{H_2O_2}\aq/\chemfig{H_2O}\liq$ &
    $\chemfig{H_2O_2}\aq + 2\chemfig{H^+}\aq + 2\electron \reaction 2\chemfig{H_2O}\liq$ \\
    &
    $\chemfig{O_2}\gaz/\chemfig{H_2O_2}\aq$ & 
    $\chemfig{O_2}\gaz + 2\chemfig{H^+}\aq + 2\electron \reaction \chemfig{H_2O_2}\aq$ \\
    %
    \SetCell[r = 2]{l} Eau de Javel &
    $\chemfig{ClO^{-}}\aq/\chemfig{Cl^{-}}\aq$ &
    $\chemfig{ClO^{-}}\aq + 2\chemfig{H^+}\aq + 2\electron
    \reaction \chemfig{Cl^{-}}\aq + \chemfig{H_2O}\liq$ \\
    %
    &
    $\chemfig{ClO^{-}}\aq/\chemfig{Cl_2}\gaz$ &
    $2\chemfig{ClO^{-}}\aq + 4\chemfig{H^+}\aq + 2\electron
    \reaction \chemfig{Cl_2}\gaz + \chemfig{H_2O}\liq$ \\
    %
    Diiode &
    $\chemfig{I_2}\aq/\chemfig{I^{-}}\aq$ &
    $\chemfig{I_2}\aq + 2\electron \reaction 2\chemfig{I^{-}}\aq$ \\
    %
    {Permanganate \\ de potassium} &
    $\chemfig{MnO_4^{-}}\aq/\chemfig{Mn^{2+}}\aq$ &
    $\chemfig{MnO_4^{-}}\aq + 8\chemfig{H^+}\aq + 5\electron
    \reaction \chemfig{Mn^{2+}}\aq + 4\chemfig{H_2O}\liq$ \\
    %
  \end{tableau}
\end{doc}


\begin{doc}{Eau de Javel et produit acide : un mélange dangereux !}{doc:A3_javel_acide}
  L'eau de Javel est une solution aqueuse basique d’hypochlorite de sodium (\chemfig{Na^{+}}, \chemfig{ClO^{-}}) et de chlorure de sodium (\chemfig{Na^{+}}, \chemfig{Cl^{-}}).
  Un produit acide contient des ions \chemfig{H^+}.

  L'ion chlorure est un réducteur dans le couple $\chemfig{Cl_2}\gaz/ \chemfig{Cl^{-}}\aq$.
  La demi-équation d'oxydoréduction associée est
  $\chemfig{Cl_2}\gaz + 2\electron \reaction 2\chemfig{Cl^{-}}\aq$.

  Le dichlore $\chemfig{Cl_2}\gaz$ est un gaz toxique, car le dichlore se combine avec l'eau présente dans les muqueuses pour former des acides qui attaquent les tissus.
\end{doc}


\question{
  Établir l'équation de la réaction d'oxydoréduction entre les ions hypochlorites $\chemfig{ClO^{-}}\aq$ et les ions chlorures $\chemfig{Cl^{-}}\aq$.
}{

}{3}

\question{
  Pourquoi cette réaction ne peut avoir lieu que dans un milieu acide ?
}{}{2}

\question{
  Quel est le gaz toxique dégagé par la réaction ?
}{}{1}


%%%%
\begin{doc}{}{doc:A3_}
  Judith s'est écorchée le genou et mélange de l'eau oxygénée avec du permanganate de potassium pour soigner sa plaie.
  Les couples Ox/Red sont $\chemfig{O_2}\gaz/\chemfig{H_2O_2}\aq$ et
  $\chemfig{MnO_4^{-}}\aq/\chemfig{Mn^{2+}}\aq$.
  
  Au moment de l'application, le mélange devient incolore et forme une mousse.
\end{doc}



%%
\question{
  Établir l'équation de la réaction d'oxydoréduction entre l'eau oxygénée et le permanganate de potassium.
  Expliquer la formation de mousse.
}{}{3}