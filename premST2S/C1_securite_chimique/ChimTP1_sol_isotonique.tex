%%%%
\tetePremStssChim

%%%% titre
\vspace*{-36pt}
\numeroActivite{1}
\titreTP{Préparation d'une solution isotonique par dissolution}


%%%% objectifs
\begin{objectifs}
  \item Revoir la préparation d'une solution par dissolution.
  \item Revoir la concentration massique.
\end{objectifs}

\begin{contexte}
  Le sucre contenu dans nos muscles permet à notre corps de fournir un effort intensif.
  Cependant, les réserves en sucre sont limitées, il faut donc les renouveler pour continuer à fournir un effort important.
  Un moyen efficace de renouveler ces ressources est de boire avant, et pendant l'effort des boissons isotoniques.
  Une boisson isotonique contient une quantité bien précise de glucose (sucre).

  Il existe plusieurs boissons de ce type dans le commerce, mais leur prix est élevé malgré une préparation simple.
  \problematique{
    Comment préparer une boisson isotonique ?
  }
\end{contexte}


%%%%
\begin{doc}{Notion de concentration massique}{doc:TP1_concentration_massique}
  La concentration massique d’une espèce en solution dans un solvant, est notée $C_m$.
  Elle désigne la masse $m_\solute$ de soluté (c'est à dire d'espèce dissoute) dans un volume Vsolution de solution.
  On a alors la relation :
  \begin{equation*}
    c_m = \dfrac{ m_\solute }{ V_\solution }
  \end{equation*}

  \exemples les solutions ci-dessous contiennent un nombre de plus en plus petit de particule.
  Comme leur volumes diminuent aussi, la concentration massique reste identique.
  \begin{tblr}{
    colspec = {|c |c |c |}, hlines, 
  }
    & & \\
    8 dans \qty{1,00}{\litre} & 4 dans \qty{0,50}{\litre} & 2 dans \qty{0,25}{\litre} \\
    $c_m = \qty{8}{\per\litre}$  & $c_m = \qty{8}{\per\litre}$  & $c_m = \qty{8}{\per\litre}$
  \end{tblr}
\end{doc}

%%
\question{
  Donner l'unité de la concentration massique $c_m$. Citer une autre grandeur qui s'exprime avec la même unité, s'agit-il de la même chose ?
}{
  Unité : \unit{\g\per\litre}. C'est l'unité de la masse volumique, qui représente la densité d'un corps.
}{1}

\begin{doc}{Boisson isotonique d'une joggeuse}{doc:TP1_boisson_joggeuse}
  Avant de partir courir, une joggeurse se prépare une boisson isotonique.
  Elle introduit \qty{10}{\g} de sel \chemfig{NaCl} et 6 morceaux de glucose \bruteCHO{6}{12}{6} de \qty{5}{\g} chacun dans une bouteille de \qty{1}{\litre}, qu'elle remplit d'eau.
\end{doc}

\question{
  Calculer la concentration massique en chlorure de sodium \chemfig{NaCl}, puis en glucose \bruteCHO{6}{12}{6}.
}{
  \begin{align*}
    c_{m,\text{sel}} &= \dfrac{\qty{10}{\g}}{\qty{1}{\litre}} = \qty{10}{\g\per\litre} \\
    c_{m,\text{sucre}} &= \dfrac{\qty{6\times5}{\g}}{\qty{1}{\litre}} = \qty{30}{\g\per\litre}
  \end{align*}
}{2}

\question{
  Calculer la masse de sel et la masse de sucre qu'il faut mettre dans une fiole jaugée de $\qty{100}{\mL} = \qty{0,100}{\litre}$ pour réaliser la même boisson isotonique.
}{
  \begin{align*}
    m_\text{sel} &= \qty{10}{\g\per\litre} \times \qty{0,100}{\litre} = \qty{1,0}{\g} \\
    m_\text{sucre} &= \qty{30}{\g\per\litre} \times \qty{0,100}{\litre} = \qty{3,0}{\g}
  \end{align*}
}{2}

\numeroQuestion
Mettre les images dans l'ordre pour reconstituer, le protocole de dissolution.
Pour chaque image, indiquer le chiffre en dessous de l'image qui correspond à l'action à réaliser parmi les phrases suivantes :
\begin{enumerate}
  \item Peser le solide grâce à la coupelle de pesée
  \item Agiter la fiole jaugée jusqu'à dissolution du solide.
  \item Remplir la fiole jaugée aux deux tiers avec de l'eau distillée.
  \item Agiter pour homogénéiser la solution.
  \item Compléter la fiole jaugée avec de l'eau distillée jusqu’au trait de jauge.
  \item Tarer la balance.
  \item Introduire le solide dans la fiole jaugée grâce à un entonnoir.
  \item Rincer la coupelle et l'entonnoir avec de l'eau distillée.
\end{enumerate}

\numeroQuestion Une fois validé, réaliser le protocole de dissolution pour préparer la boisson isotonique.
