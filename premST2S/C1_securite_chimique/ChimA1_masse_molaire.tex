%%%%
\tetePremStssChim

%%%% titre
\vspace*{-36pt}
\numeroActivite{1}
\titreActivite{Compter les entités comme une chimiste}


%%%% objectifs
\begin{objectifs}
  \item Revoir la notion de mole.
  \item Découvrir la notion de masse molaire.
  \item Calculer des quantités de matière.
\end{objectifs}

\begin{contexte}
  Les objets macroscopiques qui nous entourent sont constitués d'un grand nombre d'entités chimiques microscopiques.
  
  \problematique{
    Comment compter et mesurer les entités chimiques présentent dans des objets du quotidien ?
  }
\end{contexte}


%%%% docs
\begin{doc}{Espèce chimique et corps pur}{doc:A1_espece_corps_pur}
  \begin{encart}
    La matière est constituée \important{d'entités chimiques} microscopiques : atomes, molécules, ions.
    Une \important{espèce chimique} est constituée d'un ensemble d'entités chimiques
identiques.
  \end{encart}
  \begin{encart}
    Un \important{corps pur} est un échantillon (solide, liquide ou gazeux) composé d'une \important{espèce chimique.}
    Un \important{mélange} est un échantillon composé de plusieurs \important{espèce chimique.}
  \end{encart}
\end{doc}

%%
\begin{doc}{Composition de la coriandre pour 100 \unit{\g}}{doc:A1_coriandre}
  \begin{wrapfigure}{r}{0.13\linewidth}
    \vspace*{-49pt}
    \image{1}{images/photos/coriandre.jpg}
  \end{wrapfigure}
  
  \begin{tblr}{
    colspec = {|c |c |c |c |c |}, hlines, row{1} = {couleurPrim!20}
  }
    Constituant & Eau \chemfig{H_2O} &
    Ion calcium \chemfig{Ca^{2+}} & Saccharose \bruteCHO{12}{22}{11} & autres \\
    %
    Masse & \qty{92,2}{\g} & \qty{67e-3}{\g} & \qty{0,82}{\g} & \qty{6,91}{\g}
  \end{tblr}
\end{doc}


%%%%
\question{
  La coriandre est-elle un corps pur ou un mélange ? Justifier.
}{
  C'est un mélange, elle est constitué de plusieurs espèces chimiques.
}{2}


%%
\begin{doc}{La mole}{doc:A1_mole}
  Un échantillon de sucre en poudre est un corps pur, il ne contient que des molécules de glucose de formule brute \bruteCHO{6}{12}{6}.

  Le nombre d'entité de glucose contenu dans un échantillon est gigantesque, de l'ordre de \num{e23} !
  \begin{equation*}
    \num{e23} = \num{100 000 000 000 000 000 000 000}
  \end{equation*}

  \begin{wrapfigure}{r}{0.25\linewidth}
    \centering \vspace*{-50pt}
    \qrcode{https://youtu.be/TEl4jeETVmg?t=16} \\[4pt]
    \image{0.7}{images/photos/Avogadro_Amedeo}
  \end{wrapfigure}
  %
  Pour faciliter le comptage, en chimie on regroupe les entités en des paquets qu'on appelle \important{mole.}
  \begin{encart}
    Une \important{mole} contient précisement $N_A = \qty{6,02 e23}{\per\mole}$ entités chimiques.
  \end{encart}
  \attention $N_A$ est une constante appelée \textbf{nombre d'Avogadro}, en hommage au scientifique Aemedeo Avogadro.
  L'unité \og \unit{\per\mole} \fg\! signifie \og par mole \fg, c’est le nombre d'atomes dans une mole.
\end{doc}

%%
\newpage
\vspace*{-36pt}
\begin{doc}{Masse molaire}{doc:A1_masse_molaire}
  \begin{encart}
    Chaque \important{atome} possède une \important{masse molaire} atomique, qui correspond à \textbf{la masse d'une mole d'atome}.
    La masse molaire se note $M$ et s'exprime en \unit{\g/\mole} ou \unit{\g\per\mole}.
  \end{encart}
  Les masses molaires sont indiquées dans le tableau périodique des éléments.

  %% Tableau périodique
  \vspace*{-4pt}
  \begin{center}
    \hspace*{40pt}
    \tableauPeriodique{
      \node[name=H, Element]               {\elementH};
      \node[name=C, right of=H, Nonmetal]  {\elementC};
      \node[name=O, right of=C, Nonmetal]  {\elementO};
      \node[name=Ca, right of=O, Nonmetal] {\elementCa};
      %% Légende
      \tkzVecteur{12.1}{-1}{-2}{0}{Masse molaire}{below right}
    }
  \end{center}
  \vspace*{-20pt}

  %
  \begin{encart}
    La masse molaire d'une \important{molécule} est \textbf{la somme de la masse molaire de ses constituants}.
  \end{encart}
  Elle peut être donnée, ou calculée à partir de la formule brute de la molécule.

  \exemple pour la molécule de dioxyde de carbone \chemfig{CO_2}, sa masse molaire vaut
  \begin{equation*}
    \masseMol{CO_2} = \masseMol{C} + 2 \times \masseMol{O}
    = \qty{12,0}{\g\per\mole} + 2\times\qty{16,0}{\g\per\mole}
    = \qty{44,0}{\g\per\mole}
  \end{equation*}

  %
  \begin{encart}
    La masse molaire des ions est identique à la masse molaire de l'atome ou de la molécule liée.
  \end{encart}

  \exemples $\masseMol{Mn} = \masseMol{Mn^{2+}}$,
  $\masseMol{H_3O^{+}} = \masseMol{H_3O}$.
\end{doc}

%%%%
\question{
  Calculer la masse molaire des trois constituants de la coriandre dont la formule brute est précisée dans le document~\ref{doc:A1_coriandre}.
}{
  
}{3}


%%
\begin{doc}{Quantité de matière}{doc:A1_quantite_matiere}
  \begin{encart}
    La \important{quantité de matière}, notée $n$, est la grandeur qui détermine le nombre d'entité chimique dans un échantillon.
    Son \important{unité est la mole}, notée \unit{\mole}.
  \end{encart}

  Pour mesurer la quantité de matière d'une espèce chimique dans un échantillon,
  il faut le peser et utiliser la relation suivante
  \begin{equation*}
    n_\espece = \dfrac{m_\espece}{M_\espece}
  \end{equation*}
  Cette relation lie la quantié de matière $n_\espece$, la masse $m_\espece$ et la masse molaire $M_\espece$ de l'espèce.
\end{doc}

%%%%
\question{
  Calculer la quantité de matière des trois constituants de la coriandre, en utilisant la masse molaire déjà calculée et leurs masses données dans le document~\ref{doc:A1_coriandre}.
}{
  
}{3}
