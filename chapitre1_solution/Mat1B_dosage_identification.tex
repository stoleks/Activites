%%%% début de la page
\newpage
\enTete{Corps purs et solutions}{1}

%%%%
\nomPrenomClasse

%%%% titre
\numeroActivite{5}
\titreTP{Dosage d'un antiseptique -- Identification d'espèces chimiques}


%%%% objectifs
\begin{objectifs}
  \item Réaliser une échelle de teinte \textbf{en autonomie.}
  \item Mesurer la concentration en \chemfig{KMnO_4} présente dans du Dakin.
  \item Réaliser des tests chimiques.
\end{objectifs}


%%%%
\titrePartie{Dosage d'un antiseptique}


%%%% evaluation
\begin{tableauCompetences}
  \centering APP --
  Rechercher l'information.
  & & & &
  \\ \hline
  %
  \centering ANA/RAI --
  Élaborer un protocole.
  & & & &
  \\ \hline
  %
  %
  \centering REA --
  Réaliser un protocole, un calcul.
  & & & &
  \\ \hline
  %
  \centering VAL --
  Comparer avec des valeurs de références.
  & & & &
\end{tableauCompetences}


\bigskip
\problematique{Mesurer la concentration de permanganate de potassium présent dans du Dakin à l'aide d'une échelle de teinte.}


%%%%
\begin{doc}{Dakin}
  \label{doc:dakin}
  Le Dakin est une \textbf{solution aqueuse} d'hypochlorite de sodium (\chemfig{Na ClO}).
  Du permanganate de potassium (\chemfig{K MnO_4}) est ajouté à la solution, pour qu'elle ne soit pas dégradée par l'exposition au rayonnement UV du Soleil.
  
  \fleche Le constructeur indique que la concentration de \chemfig{KMnO_4} est de l'ordre de $0,\!01 \unit{g/L}$ dans le Dakin.
  
  \fleche On dispose d'une solution de permanganate de potassium (\chemfig{K MnO_4}) avec une concentration massique $c = 0,\!08 \unit{g/L}$.
\end{doc}


%%%%
\exo{
  Donner une série de concentration pour des solutions étalons, de sortes que l'échelle de teinte réalisée permette d'encadrer la concentration annoncée par le constructeur.\competence{APP, ANA/RAI}
}{0}

\begin{center}
  \setlength{\extrarowheight}{4pt}
  \begin{tabular}{c | c | c | c | c}
    Solution étalon & \phantom{00}1\phantom{00} & \phantom{00}2\phantom{00}& \phantom{00}3\phantom{00} & \phantom{00}4\phantom{00} \\
    \hline
    Concentration (g/L) & 0,08 &  &  & 
  \end{tabular}
\end{center}


%%%%
\exo{
  Indiquer les volumes des solutions mère $V_\text{mère}$ à introduire dans une fiole jaugée de $50 \unit{mL}$ pour préparer chaque solution.\competence{REA}
}{0}
  
\begin{center}
  \setlength{\extrarowheight}{4pt}
  \begin{tabular}{c | c | c | c}
    Solution étalon & \phantom{00}2\phantom{00}& \phantom{00}3\phantom{00} & \phantom{00}4\phantom{00} \\
    \hline
    $V_\text{mère}$ (mL) &  &  &
  \end{tabular}
\end{center}

\appelProf Appeler le professeur si vous êtes bloqué-es.


%%%%
\exo{
  Préparer les solutions étalons par dilution successives.\competence{REA}
}{0}


%%%%
\exo{
  En utilisant l'échelle de teinte réalisée, donner un encadrement pour la concentration de permanganate de potassium présent dans le Dakin.
  La réponse doit être rédigée.\competence{REA, VAL}
}{3}

\appelProf Appeler le professeur si vous êtes bloqué-es.


%%%%
\exo{
  Indiquer si votre mesure est cohérente avec la concentration annoncée par le constructeur.\competence{APP, VAL}
}{2}


%%%%
\newpage
\vspace*{-36pt}
\titrePartie{Identification d'espèces chimiques}

%%%% contexte
\begin{encart}
  \emphase{Contexte :}
  
  Les eaux minérales sont des solutions aqueuses, qui contiennent plusieurs ions de nature et de masses différentes.
  Impropre à une consommation régulière, elles peuvent servir dans des régimes spécifiques.
\end{encart}

\problematique{Comment déterminer les ions présents dans des eaux minérales ?}


%%%% documents
\begin{doc}{Composition de trois eaux minérales}
  \label{doc:composition_eau}
  \vspace*{-24pt}
  \begin{center}
    \textbf{Vichy St Yorre}
    
    \setlength{\extrarowheight}{6pt}
    \begin{tabular}{l r|l r}
      \rowcolor{gray!20} \multicolumn{4}{c}{Minéralisation en mg/L}
      \\ \hline
      Bicarbonates (\chemfig{CO_3^{2-}}) & 4368 &
      Chlorures    (\chemfig{Cl^{-}})    & 322  \\ \hline
      Sodium       (\chemfig{Na^+})      & 1708 &
      Sulfates     (\chemfig{SO_4^{2-}}) & 174  \\ \hline
      Potassium    (\chemfig{K^+})       & 110  &
      Calcium      (\chemfig{Ca^{2+}})   & 90   \\ \hline
      Fluorures    (\chemfig{F^{-}})     & 1    &
      Magnésium    (\chemfig{Mg^{2+}})   & 11   \\ \hline
    \end{tabular}

    %
    \vspace*{12pt}
    \textbf{Volvic}
    
    \setlength{\extrarowheight}{6pt}
    \begin{tabular}{l r|l r}
      \rowcolor{gray!20} \multicolumn{4}{c}{Minéralisation en mg/L}
      \\ \hline
      Bicarbonates (\chemfig{CO_3^{2-}}) & \phantom{1}65,3 &
      Chlorures    (\chemfig{Cl^{-}})    & 8,4  \\ \hline
      Sodium       (\chemfig{Na^+})      & \phantom{1}9,4  &
      Sulfates     (\chemfig{SO_4^{2-}}) & 6,9  \\ \hline
      Potassium    (\chemfig{K^+})       & 5,7  &
      Calcium      (\chemfig{Ca^{2+}})   & 9,9  \\ \hline
      Nitrates     (\chemfig{NO_3^{-}})  & 6,3  &
      Magnésium    (\chemfig{Mg^{2+}})   & 6,1  \\ \hline
    \end{tabular}

    %
    \vspace*{12pt}
    \textbf{Hépar}
    
    \setlength{\extrarowheight}{6pt}
    \begin{tabular}{l r|l r}
      \rowcolor{gray!20}
      \multicolumn{4}{c}{Minéralisation en mg/L}
      \\ \hline
      Bicarbonates (\chemfig{CO_3^{2-}}) & 383,7 &
      Chlorures    (\chemfig{Cl^{-}})    & 11    \\ \hline
      Sodium       (\chemfig{Na^+})      & 14,2  &
      Sulfates     (\chemfig{SO_4^{2-}}) & 1479  \\ \hline
      Potassium    (\chemfig{K^+})       & 4     &
      Calcium      (\chemfig{Ca^{2+}})   & 549   \\ \hline
      Nitrates     (\chemfig{NO_3^{-}})  & 4,3   &
      Magnésium    (\chemfig{Mg^{2+}})   & 119   \\ \hline
    \end{tabular}
  \end{center}
\end{doc}


%%%%
\newpage
\begin{doc}{Tests caractéristiques de certains ions}
  \label{doc:tests_ions}
  \vspace*{-12pt}
  \begin{center}
    \setlength{\extrarowheight}{6pt}
    \begin{tabular}{| c | c | c |}
      \hline
      \rowcolor{gray!20}
      Ion à tester & Réactif utilisé & Résultat du test positif 
      \\ \hline
      %
      Chlorures (\chemfig{Cl^{-}}) &
      Solution de nitrate d'argent &
      Précipité blanc, noircit* \\ \hline
      %
      Sulfates  (\chemfig{SO_4^{2-}}) &
      Solution de chlorure de baryum &
      Précipité blanc \\ \hline
      %
      Calcium (\chemfig{Ca^{2+}}) &
      Solution d'oxalate d'ammonium &
      Précipité blanc \\ \hline
      %
      Magnésium (\chemfig{Mg^{2+}}) &
      Solution d'hydroxyde de sodium &
      Précipité blanc \\ \hline
    \end{tabular}
    
    \bigskip
    * Le précipité blanc noircit à la lumière.
  \end{center}
\end{doc}


%%%%
On a trois béchers (A, B, C) contenant des eaux minérales, que vous voulez identifier.

\exo{
  Réaliser le protocole suivant :
  \begin{enumerate}
    \item Laver et sécher 4 tubes à essais.
    \item Verser dans chaque tube à essais quelques mL de l'eau d'un bécher.
    \item Réaliser un test différent dans chaque tube à essais à l'aide des 4 réactifs.
    \item Noter si un précipité se forme et son abondance dans le tableau suivant ($-$, $+$, $++$, $+++$)
  \end{enumerate}
}{0}

\begin{center}
  \setlength{\extrarowheight}{10pt}
  \begin{tabular}{l | c | c | c}
    \rowcolor{gray!20} Test réalisé & Bécher A & Bécher B & Bécher C \\ \hline
    Nitrate d'argent    & & & \\ \hline
    Chlorure de baryum  & & & \\ \hline
    Oxalate d'ammonium  & & & \\ \hline
    Hydroxyde de sodium & & &
  \end{tabular}
\end{center}

\exo{
  En vous aidant des documents~\ref{doc:composition_eau} et~\ref{doc:tests_ions}, en déduire l'eau minérale contenu par chaque bécher.
}{3}